%========================================
%  Classical Math Textbook Template
%========================================
\documentclass[12pt,oneside]{book}

%----------------------------------------
% Encoding, language, fonts
%----------------------------------------
\usepackage[utf8]{inputenc}
\usepackage[T1]{fontenc}
\usepackage[english]{babel}
\usepackage{lmodern}
\usepackage{microtype}



%----------------------------------------
% Page layout
%----------------------------------------
\usepackage{geometry}
\geometry{
  a4paper,
  margin=1in
}

\usepackage{setspace}
\onehalfspacing

%----------------------------------------
% Math packages
%----------------------------------------
\usepackage{amsmath,amssymb,amsthm}
\usepackage{mathtools}

\usepackage[shortlabels]{enumitem}
\setlist[enumerate,1]{label=(\alph*)}
\setlist[enumerate,2]{label=(\roman*)}


\numberwithin{equation}{chapter}

% Common shortcuts
\newcommand{\R}{\mathbb{R}}
\newcommand{\C}{\mathbb{C}}
\newcommand{\N}{\mathbb{N}}
\newcommand{\Z}{\mathbb{Z}}
\newcommand{\dd}{\,\mathrm{d}}
\newcommand{\e}{\mathrm{e}}
\newcommand{\ii}{\mathrm{i}}

%----------------------------------------
% Theorem-like environments
%----------------------------------------
\theoremstyle{plain}
\newtheorem{theorem}{Theorem}[chapter]
\newtheorem{lemma}[theorem]{Lemma}
\newtheorem{proposition}[theorem]{Proposition}
\newtheorem{corollary}[theorem]{Corollary}

\theoremstyle{definition}
\newtheorem{definition}[theorem]{Definition}
\newtheorem{example}[theorem]{Example}

\theoremstyle{remark}
\newtheorem{remark}[theorem]{Remark}

% Problems embedded in the text (worked examples)
\theoremstyle{definition}
\newtheorem{problem}{Problem}[chapter]

% Exercises at the end of sections, numbered by section
\newtheorem{exercise}{Exercise}[section]

% Classical "Solution." environment
\newenvironment{solution}{%
  \begin{proof}[Solution]%
}{%
  \end{proof}%
}

%----------------------------------------
% Graphics, lists, hyperlinks
%----------------------------------------
\usepackage{graphicx}
\graphicspath{{figures/}}

\usepackage{enumitem}
\setlist{nosep}

\usepackage[hidelinks]{hyperref}

%----------------------------------------
% Title info
%----------------------------------------
\title{%
  \Huge Methods in Applied Mathematics\\[0.5em]
  \Large A Personal Textbook and Problem Notebook
}
\author{Your Name}
\date{\today}

%========================================
% Document
%========================================
\begin{document}

\frontmatter
\maketitle
\tableofcontents

\chapter*{Preface}

This book is intended as both a classical textbook and a personal
notebook for studying mathematical methods at the graduate level.
It is organized in two complementary ways:

\begin{itemize}
  \item \emph{By topic}, following the core course outline
        (complex analysis, Fourier analysis, differential equations).
  \item \emph{By exam and problem}, which you can develop later
        in separate chapters or appendices.
\end{itemize}

\mainmatter

%========================================
% Part I: Applied Analysis
%========================================
\part{Applied Analysis}

\chapter{Complex Analysis}

\section{Complex Variables and Complex-valued Functions}
% TODO: Add narrative / plan for this section.

% TODO: Use prompts_for_sections.py to design examples and add them here.

\section{Analytic Functions and Integration along Contours}
% --- Narrative plan (auto-generated) ---
% This section develops the idea that analytic functions in the complex plane are not only infinitely differentiable but are also tightly controlled by their integrals along curves, or contours. We will introduce contour integration, learn how to parametrize curves and compute integrals directly, and then uncover the remarkable consequences of analyticity such as Cauchy’s integral formula and path independence of integrals in simply connected regions. These ideas turn complex integration from a formal analogue of real-variable line integrals into a powerful computational and conceptual tool.
%
% From the point of view of applied mathematics, contour integrals lie at the heart of many techniques used to solve ordinary and partial differential equations, to evaluate integrals that arise in Fourier and Laplace analysis, and to understand stability in dynamical systems. For example, the inversion of the Laplace transform, the representation of solutions to the heat and wave equations, and many classical asymptotic evaluations of oscillatory integrals all rely on deforming contours in the complex plane and exploiting analyticity. The connection between contour integrals and harmonic functions also ties this material to potential theory, fluid flow, and electrostatics.
%
% Conceptually, this section forms a bridge between the local theory of complex differentiation and the global theory of complex integration and singularities. The fundamental theorems about analytic functions and contour integrals will later justify the residue calculus used to evaluate real integrals and to locate poles of transfer functions in control theory, as well as supporting the complex-analytic foundations of Fourier series and transforms. Our goal is to build intuition and technique through guided problems, gradually moving from direct parametrization of curves to more abstract arguments based on analyticity and contour deformation.

% ===== Example 1: Direct Computation of a Contour Integral on a Circle (inquiry-based) =====
\begin{problem}[Direct Computation of a Contour Integral on a Circle]
One of the simplest and most important closed curves in the complex plane is the unit circle.  Many contour integrals that arise later in complex analysis reduce, after suitable changes of variables, to integrals taken around circles.  In this problem you will practice turning a complex line integral over the unit circle into an ordinary real integral, and then evaluating it directly.  The goal is to become comfortable with the mechanics of parametrizing a contour and computing with $z(\theta)$ and $dz$, before we rely on more powerful theorems.

Let $C$ denote the unit circle traversed once counterclockwise, that is,
\[
C = \{z \in \mathbb{C} : |z| = 1\}.
\]

\begin{enumerate}[(a)]
  \item First, parametrize the unit circle.  Find a function $z(\theta)$, defined for $0 \leq \theta \leq 2\pi$, whose image is $C$ traced once counterclockwise.  Then compute $\dfrac{dz}{d\theta}$ and write $dz$ in terms of $d\theta$.
  
  Hint: Think about the polar representation of a point on the unit circle, and recall $e^{i\theta} = \cos\theta + i\sin\theta$.

  \item Consider the contour integral
  \[
    \oint_{C} z^{2}\,dz.
  \]
  Use your parametrization $z(\theta)$ from part (a) to rewrite this contour integral as an ordinary integral with respect to $\theta$ over the interval $[0,2\pi]$.  Simplify the integrand as far as you can.

  Hint: If you choose $z(\theta)=e^{i\theta}$, what is $z(\theta)^2$?  What is $dz$ in terms of $d\theta$?

  \item In your expression from part (b), an integral of the form
  \[
    \int_{0}^{2\pi} e^{ik\theta}\,d\theta
  \]
  appears for some integer $k$.  Analyze this integral in general.

  \begin{enumerate}[(i)]
    \item Compute $\displaystyle \int_{0}^{2\pi} e^{ik\theta}\,d\theta$ for an integer $k\neq 0$.
    
    Hint: Think of $e^{ik\theta}$ as the derivative of something with respect to $\theta$, and use the Fundamental Theorem of Calculus.
    
    \item What happens when $k=0$?  How does $\displaystyle \int_{0}^{2\pi} e^{0\cdot \theta}\,d\theta$ compare?
  \end{enumerate}

  \item Use your work in parts (b) and (c) to evaluate
  \[
    \oint_{C} z^{2}\,dz.
  \]
  Explain clearly which value of $k$ from part (c) you are using, and why the integral has the value that it does.

  As a further step, consider $\oint_C z^n\,dz$ for an integer $n\ge 0$.  Using the same parametrization $z(\theta)=e^{i\theta}$, what kind of integral in $\theta$ do you obtain?  Based on part (c), what do you expect the value of this integral to be when $n\ge 0$?

  \item (Extensions and ``what if'' questions.)

  \begin{enumerate}[(i)]
    \item Suppose now that $C_R$ is the circle of radius $R>0$ centered at the origin, parametrized by $z(\theta) = R e^{i\theta}$ for $0\le \theta \le 2\pi$.  Repeat the setup for
    \[
      \oint_{C_R} z^{2}\,dz
    \]
    and (without necessarily computing every step) predict how the value of this integral compares to the case $R=1$.  Is it different, or the same?  Why?

    \item Consider the function $f(z)=e^{iz}$.  Set up, but do not attempt to evaluate explicitly, the integral
    \[
      \oint_{C} e^{iz}\,dz
    \]
    using the parametrization $z(\theta) = e^{i\theta}$.  Write the resulting integrand as a function of $\theta$.

    Hint: Substitute $z(\theta)$ into $e^{iz}$ and remember to multiply by $dz$.

    After you have written the integral in terms of $\theta$, reflect on how complicated it looks compared to the $z^{2}$ case.  How might this motivate the search for more general theorems about integrals of analytic functions around closed contours?
  \end{enumerate}
\end{enumerate}
\end{problem}

% ===== Example 1: Direct Computation of a Contour Integral on a Circle (full solution) =====
\begin{problem}[Direct Computation of a Contour Integral on a Circle]
Let $C$ be the unit circle $\{z\in\mathbb{C}:|z|=1\}$ oriented counterclockwise.

\begin{enumerate}[(a)]
  \item Parametrize $C$ and use this parametrization to compute explicitly
  \[
    \oint_{C} z^{2}\,dz.
  \]
  \item Using the same parametrization, write the contour integral
  \[
    \oint_{C} e^{iz}\,dz
  \]
  as an ordinary integral with respect to a real variable, and comment briefly on the complexity of evaluating it directly.
\end{enumerate}
\end{problem}

\begin{solution}
We first represent the unit circle in a convenient way and then convert the contour integrals into ordinary real integrals.

\medskip\noindent
\textbf{Parametrization of the unit circle.}
A standard parametrization of the unit circle $C$ traversed once counterclockwise is
\[
  z(\theta) = e^{i\theta} = \cos\theta + i\sin\theta, \qquad 0 \le \theta \le 2\pi.
\]
Differentiating with respect to $\theta$ gives
\[
  \frac{dz}{d\theta} = i e^{i\theta},
\]
so
\[
  dz = i e^{i\theta}\,d\theta.
\]
This expresses both $z$ and $dz$ in terms of the real parameter $\theta$ on the interval $[0,2\pi]$.

\medskip\noindent
\textbf{(a) Computation of $\displaystyle \oint_C z^2\,dz$.}
We now write the contour integral as a real integral:
\[
  \oint_C z^2\,dz
  = \int_{0}^{2\pi} \bigl(z(\theta)\bigr)^2\, z'(\theta)\,d\theta
  = \int_{0}^{2\pi} \bigl(e^{i\theta}\bigr)^2 \cdot i e^{i\theta}\,d\theta.
\]
Since $(e^{i\theta})^2 = e^{2i\theta}$, the integrand simplifies to
\[
  (e^{i\theta})^2 \cdot i e^{i\theta} = i e^{3i\theta}.
\]
Thus
\[
  \oint_C z^2\,dz
  = i \int_{0}^{2\pi} e^{3i\theta}\,d\theta.
\]
This is now an elementary integral.  An antiderivative of $e^{3i\theta}$ with respect to $\theta$ is $\dfrac{1}{3i}e^{3i\theta}$, so by the Fundamental Theorem of Calculus,
\[
  \int_{0}^{2\pi} e^{3i\theta}\,d\theta
  = \left[\frac{1}{3i}e^{3i\theta}\right]_{\theta=0}^{\theta=2\pi}
  = \frac{1}{3i}\bigl(e^{3i(2\pi)} - e^{0}\bigr)
  = \frac{1}{3i}(e^{6\pi i}-1).
\]
Since $e^{6\pi i} = 1$, this becomes
\[
  \int_{0}^{2\pi} e^{3i\theta}\,d\theta
  = \frac{1}{3i}(1-1) = 0.
\]
Therefore,
\[
  \oint_C z^2\,dz
  = i \cdot 0 = 0.
\]

It is often convenient to see this as an instance of the more general fact that, for any nonzero integer $k$,
\[
  \int_0^{2\pi} e^{ik\theta}\,d\theta = 0.
\]
Here, we have $k=3$, so the integral vanishes.

This computation illustrates the basic procedure for evaluating a contour integral directly: parametrize the contour, substitute $z(\theta)$ and $dz = z'(\theta)\,d\theta$, and reduce the problem to an ordinary integral on a real interval.

\medskip\noindent
\textbf{(b) Writing $\displaystyle \oint_C e^{iz}\,dz$ as a real integral.}
We now perform the same change of variables for the integral of $e^{iz}$,
\[
  \oint_C e^{iz}\,dz.
\]
Using $z(\theta)=e^{i\theta}$ and $dz=i e^{i\theta}\,d\theta$, we obtain
\[
  \oint_C e^{iz}\,dz
  = \int_{0}^{2\pi} e^{i z(\theta)}\, z'(\theta)\,d\theta
  = \int_{0}^{2\pi} e^{i e^{i\theta}} \cdot i e^{i\theta}\,d\theta.
\]
Thus the contour integral can be written as
\[
  \oint_C e^{iz}\,dz
  = \int_0^{2\pi} i e^{i\theta} e^{i e^{i\theta}}\,d\theta.
\]
This is now an ordinary real integral, but its integrand is much more complicated than in the polynomial case: it is a composition of exponentials $e^{i e^{i\theta}}$ multiplied by $i e^{i\theta}$.  There is no simple elementary antiderivative in terms of familiar functions.

One could, in principle, expand $e^{i e^{i\theta}}$ as a power series and integrate term-by-term, but this quickly becomes messy.  This difficulty is precisely what motivates the development of deeper results in complex analysis.  For example, once we know that $e^{iz}$ is an entire (everywhere analytic) function, Cauchy's integral theorem tells us immediately that
\[
  \oint_C e^{iz}\,dz = 0
\]
for any closed contour $C$, without any explicit computation.  Our direct computation for $z^{2}$ shows how to perform such integrals by hand in simple cases; the comparison with $e^{iz}$ illustrates why more powerful theorems about analytic functions and contour integrals are so useful in practice.

\medskip
In summary, this example demonstrates the central ideas of this section: parametrize a contour in the complex plane, express $z$ and $dz$ in terms of a real parameter, convert the contour integral into an ordinary integral on a real interval, and then evaluate (or at least analyze) the resulting integral.  It also shows that while this direct technique works well for simple integrands like $z^2$, more sophisticated analytic methods become indispensable for more complicated analytic functions.
\end{solution}

% ===== Example 2: Path Independence and Analytic Primitives (inquiry-based) =====
\begin{problem}[Path Independence and Analytic Primitives]
In many physical models, such as the work done by a force field along a path, a key question is whether the result depends only on the starting and ending points or also on the particular path taken. In complex analysis, contour integrals of analytic functions often behave like conservative vector fields from multivariable calculus. In this problem you will explore, through explicit computations, when integrals of complex functions are independent of path and how singularities obstruct this behavior. This will naturally lead you to the concepts of primitives (antiderivatives) and simply connected domains.

Consider the functions $f(z) = 2z$ and $g(z) = \dfrac{1}{z}$, and various paths in the complex plane connecting the same pair of points.

\smallskip

(a) Let $f(z)=2z$. Consider two paths from $1$ to $-1$:
\begin{itemize}
  \item $\gamma_1$: the straight line segment from $1$ to $-1$,
  \item $\gamma_2$: the upper semicircle of radius $1$ centered at the origin, that is, the arc of the unit circle $|z|=1$ from $1$ to $-1$ going counterclockwise.
\end{itemize}
Parametrize each path and compute the contour integrals
\[
\int_{\gamma_1} 2z\,dz
\quad\text{and}\quad
\int_{\gamma_2} 2z\,dz.
\]
Do you obtain the same value for both integrals?  

Hint: For $\gamma_1$, you can parametrize by $z(t) = 1-2t$ for $t\in[0,1]$. For $\gamma_2$, you can parametrize by $z(t) = e^{it}$ for $t\in[0,\pi]$.

\smallskip

(b) Based on your computations in part (a), try to explain why the integral of $2z$ appears to be independent of the path between $1$ and $-1$. 
\begin{itemize}
  \item Find a function $F(z)$ such that $F'(z) = 2z$ for all $z\in\mathbb{C}$. 
  \item Use this $F$ to give a formula for $\displaystyle\int_\gamma 2z\,dz$ for any smooth path $\gamma$ from $1$ to $-1$ that stays in $\mathbb{C}$. 
\end{itemize}
How does this relate to the familiar Fundamental Theorem of Calculus on the real line?

% Hint: Think about $F(z) = z^2$ and try to differentiate it.

\smallskip

(c) Now turn to $g(z) = \dfrac{1}{z}$, which is not defined at $z=0$. Consider two different paths from $1$ to $-1$ that avoid the origin:
\begin{itemize}
  \item $\alpha$: the upper semicircle of radius $1$ centered at the origin, given by $|z|=1$ from $1$ to $-1$ (counterclockwise),
  \item $\beta$: the lower semicircle of radius $1$ centered at the origin, given by $|z|=1$ from $1$ to $-1$ (clockwise).
\end{itemize}
Compute the contour integrals
\[
\int_{\alpha} \frac{1}{z}\,dz
\quad\text{and}\quad
\int_{\beta} \frac{1}{z}\,dz.
\]
Do you get the same value, or do they differ? By how much?

Hint: Parametrize $\alpha$ and $\beta$ by exponentials:
\[
\alpha(t) = e^{it},\quad t\in[0,\pi],
\qquad
\beta(t) = e^{-it},\quad t\in[0,\pi].
\]

\smallskip

(d) Compare your findings for $f(z)=2z$ and $g(z)=1/z$.
\begin{itemize}
  \item For $f(z)=2z$, the integrals along different paths from $1$ to $-1$ agreed. Can you explain this directly in terms of the antiderivative $F(z)$ that you found in part (b)? Why must any contour integral of $2z$ from $z=a$ to $z=b$ have the same value, no matter which path you choose?
  \item For $g(z)=1/z$, the integrals along $\alpha$ and $\beta$ between the same endpoints do not agree. What does this suggest about the existence of an antiderivative $G$ of $1/z$ on the punctured plane $\mathbb{C}\setminus\{0\}$?
\end{itemize}

Hint: Suppose there were a function $G$ with $G'(z)=1/z$ on $\mathbb{C}\setminus\{0\}$. What would $\displaystyle\int_{\alpha} \frac{1}{z}\,dz$ and $\displaystyle\int_{\beta} \frac{1}{z}\,dz$ have to be in terms of $G$?

\smallskip

(e) (Extensions and ``what if'' questions.)
\begin{itemize}
  \item[(i)] Consider the closed unit circle $C$ given by $z(t)=e^{it}$ for $t\in[0,2\pi]$. Use your computations in part (c) to find $\displaystyle\oint_{C} \frac{1}{z}\,dz$. How is this related to the difference between $\displaystyle\int_{\alpha} \frac{1}{z}\,dz$ and $\displaystyle\int_{\beta} \frac{1}{z}\,dz$?
  \item[(ii)] Imagine removing from $\mathbb{C}\setminus\{0\}$ a ray from the origin to infinity, for example the negative real axis. On this ``cut'' plane, it is possible to define a single-valued branch of the complex logarithm $\log z$. If $G(z) = \log z$ is such a branch, what can you say about $\dfrac{d}{dz}\log z$ on this cut plane? On which kinds of regions do you then expect $\displaystyle\int \frac{1}{z}\,dz$ to be path independent, and what geometric property do these regions share?
\end{itemize}

Hint: Think about whether every closed loop in the region can be continuously shrunk to a point without crossing a singularity.
\end{problem}

% ===== Example 2: Path Independence and Analytic Primitives (full solution) =====
\begin{problem}[Path Independence and Analytic Primitives]
Let $f(z)=2z$ and $g(z)=1/z$.

(a) Compute $\displaystyle\int_{\gamma_1}2z\,dz$ where $\gamma_1$ is the straight line segment from $1$ to $-1$, and $\displaystyle\int_{\gamma_2}2z\,dz$ where $\gamma_2$ is the upper semicircle $|z|=1$ from $1$ to $-1$ (counterclockwise). Show that these two integrals are equal and identify a primitive $F$ of $f$ explaining this path independence.

(b) Compute $\displaystyle\int_{\alpha}\frac{1}{z}\,dz$ and $\displaystyle\int_{\beta}\frac{1}{z}\,dz$, where $\alpha$ is the upper semicircle $|z|=1$ from $1$ to $-1$ (counterclockwise) and $\beta$ is the lower semicircle $|z|=1$ from $1$ to $-1$ (clockwise). Show that these two integrals are different and deduce that $g(z)=1/z$ does not admit a primitive on $\mathbb{C}\setminus\{0\}$. Briefly explain how these computations illustrate the roles of analyticity, primitives, and singularities in path independence of contour integrals.
\end{problem}

\begin{solution}
We analyze the two functions $f(z)=2z$ and $g(z)=1/z$ by computing contour integrals along different paths with the same endpoints.

\medskip
\noindent\textbf{(a) The function $f(z)=2z$.}

\emph{Integral along the straight line segment.}  
Let $\gamma_1$ be the straight segment from $1$ to $-1$. A convenient parametrization is
\[
\gamma_1(t) = 1 - 2t,\qquad t\in[0,1].
\]
Then $\gamma_1(0)=1$, $\gamma_1(1)=-1$, and
\[
\gamma_1'(t) = -2.
\]
The contour integral is
\[
\int_{\gamma_1} 2z\,dz
= \int_0^1 2\,\gamma_1(t)\,\gamma_1'(t)\,dt
= \int_0^1 2(1-2t)(-2)\,dt
= \int_0^1 (-4 + 8t)\,dt.
\]
Evaluating this integral,
\[
\int_0^1 (-4 + 8t)\,dt
= \bigl(-4t + 4t^2\bigr)\Big|_{0}^{1}
= (-4 + 4) - 0 = 0.
\]
Hence
\[
\int_{\gamma_1} 2z\,dz = 0.
\]

\emph{Integral along the upper semicircle.}  
Let $\gamma_2$ be the upper semicircle of radius $1$ centered at the origin from $1$ to $-1$, oriented counterclockwise. A standard parametrization is
\[
\gamma_2(t) = e^{it},\qquad t\in[0,\pi].
\]
Then $\gamma_2(0)=1$, $\gamma_2(\pi)=-1$, and
\[
\gamma_2'(t) = i e^{it}.
\]
We compute
\[
\int_{\gamma_2} 2z\,dz
= \int_0^\pi 2\,\gamma_2(t)\,\gamma_2'(t)\,dt
= \int_0^\pi 2 e^{it} \cdot i e^{it}\,dt
= \int_0^\pi 2i\,e^{2it}\,dt.
\]
This is an elementary exponential integral:
\[
\int_0^\pi 2i\,e^{2it}\,dt
= 2i \cdot \frac{e^{2it}}{2i}\Big|_{0}^{\pi}
= e^{2i\pi} - e^{0}
= 1 - 1
= 0.
\]
Therefore,
\[
\int_{\gamma_2} 2z\,dz = 0.
\]

Thus the integrals of $2z$ along $\gamma_1$ and $\gamma_2$ from $1$ to $-1$ are equal:
\[
\int_{\gamma_1} 2z\,dz = \int_{\gamma_2} 2z\,dz = 0.
\]

\emph{Primitive and path independence.}  
We now identify a primitive of $f(z)=2z$. Consider
\[
F(z) = z^2.
\]
Then for all $z\in\mathbb{C}$,
\[
F'(z) = 2z = f(z),
\]
so $F$ is an antiderivative (or primitive) of $f$ on the entire complex plane. Since $F$ is defined and differentiable everywhere, for any smooth path $\gamma$ from $a$ to $b$ in $\mathbb{C}$ we have, by the complex version of the Fundamental Theorem of Calculus,
\[
\int_{\gamma} 2z\,dz = F(b) - F(a) = b^2 - a^2.
\]
In particular, for $a=1$ and $b=-1$, this gives
\[
\int_{\gamma} 2z\,dz = (-1)^2 - 1^2 = 1 - 1 = 0,
\]
independent of the path $\gamma$ connecting $1$ to $-1$. Our explicit computations for $\gamma_1$ and $\gamma_2$ agree with this general principle: the existence of a primitive on the domain implies path independence of the contour integral between fixed endpoints.

\medskip
\noindent\textbf{(b) The function $g(z)=1/z$.}

Now $g(z) = 1/z$ is not defined at $z=0$. We work on the punctured plane $\mathbb{C}\setminus\{0\}$ and consider two paths from $1$ to $-1$ that avoid the origin.

\emph{Upper semicircle.}  
Let $\alpha$ be the upper semicircle $|z|=1$ from $1$ to $-1$, oriented counterclockwise. As before, parametrize by
\[
\alpha(t) = e^{it},\qquad t\in[0,\pi],
\]
so that $\alpha'(t) = i e^{it}$. Then
\[
\int_{\alpha} \frac{1}{z}\,dz
= \int_0^\pi \frac{1}{\alpha(t)}\,\alpha'(t)\,dt
= \int_0^\pi \frac{1}{e^{it}} \cdot i e^{it}\,dt
= \int_0^\pi i\,dt
= i t\Big|_0^\pi
= i\pi.
\]

\emph{Lower semicircle.}  
Let $\beta$ be the lower semicircle $|z|=1$ from $1$ to $-1$, oriented clockwise. A convenient parametrization is
\[
\beta(t) = e^{-it},\qquad t\in[0,\pi],
\]
so that $\beta(0)=1$, $\beta(\pi)=-1$, and
\[
\beta'(t) = -i e^{-it}.
\]
We compute
\[
\int_{\beta} \frac{1}{z}\,dz
= \int_0^\pi \frac{1}{\beta(t)}\,\beta'(t)\,dt
= \int_0^\pi \frac{1}{e^{-it}} \cdot (-i e^{-it})\,dt
= \int_0^\pi -i\,dt
= -i t\Big|_0^\pi
= -i\pi.
\]

Thus the two integrals between the same endpoints differ:
\[
\int_{\alpha} \frac{1}{z}\,dz = i\pi,\qquad
\int_{\beta} \frac{1}{z}\,dz = -i\pi.
\]
Their difference is
\[
\int_{\alpha} \frac{1}{z}\,dz - \int_{\beta} \frac{1}{z}\,dz
= i\pi - (-i\pi) = 2\pi i.
\]

\emph{Failure of a global primitive for $1/z$ on $\mathbb{C}\setminus\{0\}$.}  
Suppose, for the sake of contradiction, that there existed a complex function $G$ defined on $\mathbb{C}\setminus\{0\}$ such that
\[
G'(z) = \frac{1}{z}
\quad\text{for all } z\in \mathbb{C}\setminus\{0\}.
\]
Then $G$ would be a primitive of $1/z$ on the punctured plane. For any smooth path $\gamma$ from a point $a$ to a point $b$ in $\mathbb{C}\setminus\{0\}$, the complex Fundamental Theorem of Calculus would give
\[
\int_{\gamma} \frac{1}{z}\,dz = G(b) - G(a),
\]
depending only on the endpoints, not on the path.

In particular, for our two paths $\alpha$ and $\beta$ from $1$ to $-1$, we would have
\[
\int_{\alpha} \frac{1}{z}\,dz = G(-1) - G(1)
= \int_{\beta} \frac{1}{z}\,dz.
\]
But we have computed explicitly that
\[
\int_{\alpha} \frac{1}{z}\,dz = i\pi,\qquad
\int_{\beta} \frac{1}{z}\,dz = -i\pi,
\]
which are not equal. This contradiction shows that no such global primitive $G$ can exist on $\mathbb{C}\setminus\{0\}$. The path dependence of the integral reveals the obstruction created by the singularity at the origin.

Equivalently, consider the closed unit circle $C$ given by $z(t)=e^{it}$, $t\in[0,2\pi]$. Traversing the full circle counterclockwise decomposes into first following $\alpha$ and then (the reverse of) $\beta$; hence
\[
\oint_{C} \frac{1}{z}\,dz
= \int_{\alpha} \frac{1}{z}\,dz
   - \int_{\beta} \frac{1}{z}\,dz
= i\pi - (-i\pi)
= 2\pi i \neq 0.
\]
If a primitive $G$ existed on $\mathbb{C}\setminus\{0\}$, then every integral of $1/z$ over a closed contour in that domain would have to vanish, since
\[
\oint_{\gamma} \frac{1}{z}\,dz = G(z_0) - G(z_0) = 0
\]
for any closed loop $\gamma$ starting and ending at $z_0$. Our explicit nonzero value for the integral around $C$ again shows that $1/z$ has no primitive on the punctured plane.

\medskip
\noindent\textbf{Conceptual summary and connection to the chapter.}

The function $f(z)=2z$ is entire (analytic on all of $\mathbb{C}$) and possesses a global primitive $F(z)=z^2$. As a consequence, for any two points $a,b\in\mathbb{C}$ and any two piecewise smooth contours $\gamma_1,\gamma_2$ from $a$ to $b$, one has
\[
\int_{\gamma_1} 2z\,dz = \int_{\gamma_2} 2z\,dz = F(b)-F(a),
\]
so the integral is path independent. This behavior parallels conservative vector fields and the real Fundamental Theorem of Calculus.

By contrast, $g(z)=1/z$ is analytic on $\mathbb{C}\setminus\{0\}$ but has an isolated singularity at the origin. The punctured plane is not simply connected: any loop that winds once around $0$ cannot be continuously shrunk to a point without crossing the singularity. Our computations show that
\[
\oint_{|z|=1} \frac{1}{z}\,dz = 2\pi i\neq 0,
\]
and that integrals between the same endpoints along different paths encircling the origin can differ by $2\pi i$. This demonstrates that, on a non–simply connected domain containing a singularity in its “hole,” an analytic function may fail to admit a global primitive and contour integrals can become path dependent.

Thus this example illustrates central ideas of the section on analytic functions and contour integration: the connection between analyticity and primitives, the criterion of path independence, the special role of simply connected domains, and the impact of singularities on the behavior of contour integrals.
\end{solution}

% ===== Example 3: Using Cauchy’s Integral Formula to Evaluate Integrals (inquiry-based) =====
\begin{problem}[Using Cauchy’s Integral Formula to Evaluate Integrals]
In many real-variable courses, computing an integral often means finding a clever substitution or performing several lines of algebraic manipulation. In complex analysis, something surprising happens: for analytic functions, the value of certain contour integrals is completely determined by the value of the function (and its derivatives) at a single point inside the contour. Cauchy’s integral formula is the central tool that makes this possible. In this problem you will discover how to use it to evaluate integrals of the form
\[
\oint_C \frac{f(z)}{z-z_0}\,dz
\quad\text{and}\quad
\oint_C \frac{f(z)}{(z-z_0)^2}\,dz
\]
without ever parametrizing the contour explicitly, and you will see why the detailed shape of the contour does not matter.

Throughout, let $f$ be analytic on an open set $\Omega\subset\mathbb{C}$ that contains a point $z_0$ and a simple closed curve $C$ which is positively oriented and lies entirely in $\Omega$, with $z_0$ in the interior of $C$.

\medskip

(a) As a warm-up, take $f(z) = z^3$, $z_0 = 1$, and let $C$ be the circle $|z-1| = 2$ oriented counterclockwise.  

\quad(i) Parametrize $C$ as $z(\theta) = 1 + 2e^{i\theta}$, $0 \le \theta \le 2\pi$, and write the integral
\[
\oint_C \frac{z^3}{z-1}\,dz
\]
as an integral with respect to $\theta$. You do not need to carry out every algebraic step, but you should simplify enough to see what kind of expression you would have to integrate.  

\quad(ii) Based on your expression, would you expect this to be a pleasant integral to compute directly? Why might you want a different method?  

% Hint: Try to see what happens if you cancel a factor $(z-1)$ or expand $(1+2e^{i\theta})^3$.

\medskip

(b) Recall Cauchy’s integral formula: if $f$ is analytic in a domain containing $C$ and its interior, and if $z_0$ lies inside $C$, then
\[
f(z_0) = \frac{1}{2\pi i}\oint_C \frac{f(z)}{z-z_0}\,dz.
\]

\quad(i) Rewrite this formula to give a direct expression for the integral $\displaystyle \oint_C \frac{f(z)}{z-z_0}\,dz$ in terms of $f(z_0)$.  

\quad(ii) Apply your formula to the function $f(z) = z^3$ and point $z_0 = 1$, with $C$ as in part (a). What is the value of $\displaystyle \oint_C \frac{z^3}{z-1}\,dz$?  

\quad(iii) How does this answer compare with what you would have obtained by direct parametrization?  

% Hint: In (i), just multiply both sides of Cauchy's formula by $2\pi i$.

\medskip

(c) Now consider integrals with a second power in the denominator. Let $f(z) = e^z$, $z_0 = 1$, and again let $C$ be the circle $|z-1| = 2$. We want to evaluate
\[
\oint_C \frac{e^z}{(z-1)^2}\,dz
\]
without parametrizing $C$.

\quad(i) Starting from Cauchy’s integral formula
\[
f(w) = \frac{1}{2\pi i}\oint_C \frac{f(z)}{z-w}\,dz,
\]
differentiate both sides with respect to $w$ (you may assume differentiation under the integral sign is valid for analytic $f$) to obtain a formula for $f'(w)$ in terms of an integral of the form $\displaystyle \oint_C \frac{f(z)}{(z-w)^2}\,dz$.  

\quad(ii) Evaluate the integral
\[
\oint_C \frac{e^z}{(z-1)^2}\,dz
\]
using the formula you found in (i).  

% Hint: For (i), use the chain rule on $(z-w)^{-1}$: $\frac{d}{dw}\frac{1}{z-w} = \frac{1}{(z-w)^2}$ (up to a sign). Be careful with the sign.

\medskip

(d) In the examples above we used a specific circle $C$ centered at $z_0 = 1$. Now think more geometrically.

\quad(i) Suppose $C_1$ and $C_2$ are two positively oriented simple closed curves in $\Omega$ that both enclose $z_0$, and suppose $f$ is analytic on a region that contains both curves and the region between them. Explain why
\[
\oint_{C_1} \frac{f(z)}{z-z_0}\,dz
\quad\text{and}\quad
\oint_{C_2} \frac{f(z)}{z-z_0}\,dz
\]
must be equal.  

\quad(ii) Using your answer to (i), explain why the value of
\[
\oint_C \frac{f(z)}{z-z_0}\,dz
\quad\text{or}\quad
\oint_C \frac{f(z)}{(z-z_0)^2}\,dz
\]
depends only on the value of $f$ (or $f'$) at $z_0$ and not on the particular shape or size of the contour $C$, as long as $C$ stays in the domain of analyticity and winds once around $z_0$.

% Hint: Consider the closed contour formed by going around $C_1$ and then back along $C_2$ in the opposite direction, and apply Cauchy's theorem (the integral of an analytic function around a closed curve in a domain where it is analytic is zero).

\medskip

(e) Explorations and extensions.

\quad(i) What happens if the point $z_0$ lies \emph{outside} the contour $C$? Use Cauchy’s theorem (or an appropriate variant of Cauchy’s formula) to predict the value of
\[
\oint_C \frac{f(z)}{z-z_0}\,dz.
\]
How does this compare with your intuition if you tried to parametrize and integrate directly?  

\quad(ii) More generally, one can show that for each integer $n \ge 1$,
\[
f^{(n)}(z_0) = \frac{n!}{2\pi i}\oint_C \frac{f(z)}{(z-z_0)^{n+1}}\,dz.
\]
Assuming this higher-order version of Cauchy’s formula, evaluate
\[
\oint_{|z-1|=3} \frac{z^2+1}{(z-1)^3}\,dz.
\]
Then explain why the same answer would hold for any simple closed contour enclosing $1$ on which $z^2+1$ is analytic.

% Hint: Identify $f(z)$ and the integer $n$ so that your integral matches the general formula. Then compute the appropriate derivative $f^{(n)}(1)$.
\end{problem}

% ===== Example 3: Using Cauchy’s Integral Formula to Evaluate Integrals (full solution) =====
\begin{problem}[Using Cauchy’s Integral Formula to Evaluate Integrals]
Let $f$ be analytic on an open set containing a simple closed positively oriented contour $C$ and its interior, and let $z_0$ be a point in the interior of $C$.

(a) Use Cauchy’s integral formula to evaluate
\[
\oint_{|z-1|=2} \frac{z^3}{z-1}\,dz.
\]

(b) By differentiating Cauchy’s integral formula, obtain the identity
\[
f'(z_0) = \frac{1}{2\pi i}\oint_C \frac{f(z)}{(z-z_0)^2}\,dz,
\]
and use it to evaluate
\[
\oint_{|z-1|=2} \frac{e^z}{(z-1)^2}\,dz.
\]

(c) Briefly explain why the values of the integrals in parts (a) and (b) do not change if we replace the circle $|z-1|=2$ by any other simple closed contour in the domain of analyticity that winds once around $1$.
\end{problem}

\begin{solution}
We illustrate how Cauchy’s integral formula turns what would otherwise be nontrivial contour integrals into immediate evaluations in terms of the values and derivatives of analytic functions at a single point.

\medskip

\noindent\textbf{(a) Evaluating $\displaystyle \oint_{|z-1|=2} \frac{z^3}{z-1}\,dz$.}

Cauchy’s integral formula states that if $f$ is analytic on and inside a simple closed positively oriented contour $C$, and $z_0$ lies in the interior of $C$, then
\[
f(z_0) = \frac{1}{2\pi i}\oint_C \frac{f(z)}{z-z_0}\,dz.
\]
Rewriting, we have
\[
\oint_C \frac{f(z)}{z-z_0}\,dz = 2\pi i\, f(z_0).
\]

In our integral, the contour $C$ is the circle $|z-1|=2$, so $C$ is a simple closed positively oriented contour, and the point $z_0=1$ lies in its interior. The integrand is
\[
\frac{z^3}{z-1},
\]
so we may take $f(z) = z^3$, which is entire (analytic everywhere), hence analytic on and inside $C$. Therefore Cauchy’s integral formula applies and gives
\[
\oint_{|z-1|=2} \frac{z^3}{z-1}\,dz
= 2\pi i\, f(1)
= 2\pi i \cdot 1^3
= 2\pi i.
\]

Thus, instead of parametrizing the circle and integrating a somewhat complicated expression, we obtain the answer in one line by recognizing the integrand as being of the form $f(z)/(z-z_0)$.

\medskip

\noindent\textbf{(b) Derivative version and evaluation of $\displaystyle \oint_{|z-1|=2} \frac{e^z}{(z-1)^2}\,dz$.}

We start again from Cauchy’s integral formula in the form
\[
f(w) = \frac{1}{2\pi i}\oint_C \frac{f(z)}{z-w}\,dz,
\]
valid for each point $w$ in the interior of $C$, provided $f$ is analytic on and inside $C$.

We differentiate both sides with respect to $w$. On the left-hand side we obtain $f'(w)$. On the right-hand side, the contour $C$ and the variable of integration $z$ are independent of $w$, so we may differentiate under the integral sign (this is justified because $f$ is analytic and the integrand depends analytically on both $z$ and $w$). We get
\[
\frac{d}{dw}\left(\frac{1}{2\pi i}\oint_C \frac{f(z)}{z-w}\,dz\right)
= \frac{1}{2\pi i}\oint_C \frac{\partial}{\partial w}\!\left(\frac{f(z)}{z-w}\right)dz.
\]
Since $f(z)$ does not depend on $w$, we have
\[
\frac{\partial}{\partial w}\left(\frac{f(z)}{z-w}\right)
= f(z)\,\frac{\partial}{\partial w}\left(\frac{1}{z-w}\right).
\]
Using $\frac{d}{dw}\,(z-w)^{-1} = (z-w)^{-2}$ (the derivative is $+1/(z-w)^2$ because $\frac{d}{dw}(z-w) = -1$ and thus
$\frac{d}{dw}(z-w)^{-1} = -1\cdot (z-w)^{-2}\cdot(-1) = (z-w)^{-2}$), we obtain
\[
\frac{\partial}{\partial w}\left(\frac{1}{z-w}\right) = \frac{1}{(z-w)^2},
\]
and hence
\[
\frac{\partial}{\partial w}\left(\frac{f(z)}{z-w}\right) = \frac{f(z)}{(z-w)^2}.
\]

Putting this into the integral, we obtain
\[
f'(w) = \frac{1}{2\pi i}\oint_C \frac{f(z)}{(z-w)^2}\,dz.
\]
Now setting $w = z_0$ in the interior of $C$ yields the \emph{first derivative version} of Cauchy’s formula:
\[
f'(z_0) = \frac{1}{2\pi i}\oint_C \frac{f(z)}{(z-z_0)^2}\,dz.
\]

We can rewrite this identity to express the integral directly:
\[
\oint_C \frac{f(z)}{(z-z_0)^2}\,dz = 2\pi i\, f'(z_0).
\]

We now apply this to the specific integral
\[
\oint_{|z-1|=2} \frac{e^z}{(z-1)^2}\,dz.
\]
Here we take $f(z) = e^z$, which is entire, and $z_0=1$, which lies in the interior of the circle $|z-1|=2$. Hence we may apply the formula and obtain
\[
\oint_{|z-1|=2} \frac{e^z}{(z-1)^2}\,dz = 2\pi i\, f'(1).
\]
Since $f'(z) = e^z$ as well, we have $f'(1) = e^1 = e$. Therefore
\[
\oint_{|z-1|=2} \frac{e^z}{(z-1)^2}\,dz = 2\pi i\, e.
\]

Again, the value of a seemingly complicated integral is obtained immediately from the derivative of $f$ at the center of the circle.

\medskip

\noindent\textbf{(c) Independence of the particular contour.}

We now explain why the answers in parts (a) and (b) do not depend on the specific circle $|z-1|=2$, but only on the fact that the contour winds once around $z_0 = 1$ in a region where the function is analytic.

First consider the case of the integrand $f(z)/(z-z_0)$. Let $C_1$ and $C_2$ be two simple closed positively oriented contours in a domain on which $f$ is analytic, and assume that $z_0$ lies in the interior of both $C_1$ and $C_2$. Suppose further that the region between $C_1$ and $C_2$ also lies in the domain of analyticity of $f$. We claim that
\[
\oint_{C_1} \frac{f(z)}{z-z_0}\,dz = \oint_{C_2} \frac{f(z)}{z-z_0}\,dz.
\]

To see this, consider the closed contour formed by first traversing $C_1$ in the positive direction, then traversing $C_2$ in the negative direction. Denote this combined contour by $C = C_1 - C_2$. The integrand
\[
g(z) := \frac{f(z)}{z-z_0}
\]
is analytic on the region between $C_1$ and $C_2$, because $f$ is analytic there and $z_0$ lies strictly inside both curves, so $z-z_0 \neq 0$ in the region between them. Hence $g$ is analytic on and inside the contour $C$. By Cauchy’s theorem, the integral of an analytic function around a closed contour in a domain where it is analytic is zero, so
\[
\oint_C g(z)\,dz = 0.
\]
But this integral is precisely
\[
\oint_{C_1} \frac{f(z)}{z-z_0}\,dz - \oint_{C_2} \frac{f(z)}{z-z_0}\,dz = 0,
\]
which implies
\[
\oint_{C_1} \frac{f(z)}{z-z_0}\,dz = \oint_{C_2} \frac{f(z)}{z-z_0}\,dz.
\]

Thus the value of $\displaystyle \oint_C f(z)/(z-z_0)\,dz$ is the same for any such contour $C$ enclosing $z_0$ once. In particular, for the integrand $z^3/(z-1)$, the value of the integral is always $2\pi i f(1) = 2\pi i$, no matter which simple closed contour in the domain encloses $1$.

An entirely analogous argument applies to the integrand $f(z)/(z-z_0)^2$. If we define
\[
h(z) := \frac{f(z)}{(z-z_0)^2},
\]
then $h$ is analytic in any region that excludes $z_0$. Provided the region between two contours $C_1$ and $C_2$ does not contain $z_0$ and lies in the domain of analyticity of $f$, we again have by Cauchy’s theorem that
\[
\oint_{C_1} h(z)\,dz = \oint_{C_2} h(z)\,dz.
\]
Thus, when $C$ winds once around $z_0$ in a region where $f$ is analytic, the integral $\displaystyle \oint_C f(z)/(z-z_0)^2\,dz$ depends only on $f'(z_0)$ via the derivative version of Cauchy’s formula:
\[
\oint_C \frac{f(z)}{(z-z_0)^2}\,dz = 2\pi i\, f'(z_0),
\]
and not on the particular geometric shape or radius of the contour.

In summary, these examples illustrate two central ideas of this section on analytic functions and integration along contours:

\begin{itemize}
  \item Cauchy’s integral formula and its derivative versions allow us to evaluate contour integrals involving analytic functions and simple poles at $z_0$ directly in terms of the values of $f$ and its derivatives at $z_0$.
  \item For analytic integrands of the form $f(z)/(z-z_0)^{n+1}$, the value of the contour integral depends only on the analytic data at the point $z_0$ and the winding number of the contour around $z_0$, not on the detailed geometry of the contour itself.
\end{itemize}

This is a hallmark of complex analysis: analytic functions are rigid, and many contour integrals are controlled entirely by local singular behavior and the topology of the path, rather than by explicit parametrizations.
\end{solution}

% ===== Example 4: Estimating Contour Integrals with the ML Inequality (inquiry-based) =====
\begin{problem}[Estimating Contour Integrals with the ML Inequality]
In many applications of complex analysis we do not actually need the exact value of a contour integral, but only a good bound on its size. Such bounds allow us, for example, to justify sending the endpoints of a contour to infinity, or to move a contour slightly without changing a real integral. In this problem you will practice using the ML inequality to control the size of contour integrals, with the Gaussian-type integrand $e^{-z^{2}}$ as your main example. Along the way you will see how these estimates combine with Cauchy's theorem to compare integrals along different horizontal lines.

Let $f(z)=e^{-z^{2}}$.

\smallskip

(a) \textbf{Deriving the ML inequality.}  
Suppose $C$ is a smooth contour given by a parametrization $\gamma:[a,b]\to\mathbb{C}$, and $f$ is continuous on $C$. Assume that
\[
|f(z)|\le M\quad\text{for all }z\in C
\]
and that the length of $C$ is $L=\mathrm{length}(C)$. Show that
\[
\left|\int_{C} f(z)\,dz\right|\le M L.
\]
(Hint: Write $\displaystyle\int_C f(z)\,dz$ as $\displaystyle\int_a^b f(\gamma(t))\,\gamma'(t)\,dt$ and apply the triangle inequality together with the definition of the length of a curve.)

\smallskip

(b) \textbf{A first, somewhat crude, estimate on a circle.}  
For $R>0$, let $C_R$ be the circle $|z|=R$ oriented counterclockwise. Use the inequality $|e^{w}| = e^{\Re w}$ to show that on $C_R$ one has
\[
|e^{-z^{2}}|\le e^{R^{2}}.
\]
Then use your result from part (a) to give an upper bound on
\[
\left|\int_{C_R} e^{-z^{2}}\,dz\right|.
\]
Is this bound useful if you want to send $R\to\infty$ and hope the integral tends to zero? Briefly explain.

% Hint: On $|z|=R$ one has $|z^2|=R^2$, and $-\Re(z^2)\le |z^2|$.

\smallskip

(c) \textbf{A better contour: a horizontal strip and vertical sides.}  
Fix a real number $a>0$. For each $R>0$, consider the rectangle $C_R$ with vertices $-R$, $R$, $R+ia$, and $-R+ia$, oriented counterclockwise.

Let $\Gamma_R^{(1)}$ be the right vertical side from $R$ to $R+ia$ and $\Gamma_R^{(2)}$ the left vertical side from $-R+ia$ back to $-R$.

\begin{itemize}
  \item[(i)] Show that for $z=x+iy$ one has $|e^{-z^{2}}| = e^{-(x^{2}-y^{2})}$.
  
  \item[(ii)] Use this formula to show that on $\Gamma_R^{(1)}$ one has
  \[
  |e^{-z^{2}}|\le e^{-(R^{2}-a^{2})}.
  \]
  (A completely analogous estimate holds on $\Gamma_R^{(2)}$.)
  
  \item[(iii)] Use the ML inequality to bound
  \[
  \left|\int_{\Gamma_R^{(1)}} e^{-z^{2}}\,dz\right|\quad\text{and}\quad
  \left|\int_{\Gamma_R^{(2)}} e^{-z^{2}}\,dz\right|.
  \]
  Show that each of these bounds tends to $0$ as $R\to\infty$.
\end{itemize}

Hint: On the vertical side $\Gamma_R^{(1)}$ we have $x=R$ and $0\le y\le a$, so $x^2-y^2\ge R^2-a^2$. The length of $\Gamma_R^{(1)}$ is $a$.

\smallskip

(d) \textbf{Comparing two real integrals via contour deformation.}  
Still with $a>0$ fixed and $C_R$ the same rectangle, write the contour integral
\[
\int_{C_R} e^{-z^{2}}\,dz
\]
as a sum of four integrals over the sides: the bottom side (along the real axis from $-R$ to $R$), the top side (along the line $\Im z = a$), and the two vertical sides $\Gamma_R^{(1)}$ and $\Gamma_R^{(2)}$.

\begin{itemize}
  \item[(i)] Parameterize each side and write the four contributions explicitly as real integrals. Be careful with the direction of traversal of the top side.
  
  \item[(ii)] Use the fact that $e^{-z^{2}}$ is entire (hence analytic everywhere) and Cauchy's theorem to conclude that
  \[
  \int_{C_R} e^{-z^{2}}\,dz = 0.
  \]
  Rearrange this identity to express
  \[
  \int_{-R}^{R} e^{-(x+ia)^{2}}\,dx
  \]
  in terms of $\int_{-R}^{R} e^{-x^{2}}\,dx$ and the integrals along the vertical sides.
  
  \item[(iii)] Now let $R\to\infty$ and use part (c) to show that
  \[
  \int_{-\infty}^{\infty} e^{-(x+ia)^{2}}\,dx
  =
  \int_{-\infty}^{\infty} e^{-x^{2}}\,dx.
  \]
\end{itemize}

Hint: The integral along the top side will naturally appear with a minus sign because that side is traversed from $R+ia$ to $-R+ia$.

\smallskip

(e) \textbf{Extensions and variations.}
\begin{itemize}
  \item[(i)] Suppose instead of $f(z)=e^{-z^{2}}$ we consider $f(z)=e^{-z^{2}}p(z)$, where $p$ is a polynomial. How would you adapt the estimates in part (c) to show that the integrals over the vertical sides of $C_R$ still tend to zero as $R\to\infty$? What condition on the degree of $p$ is actually needed?
  
  \item[(ii)] In part (c) we kept the height $a$ of the rectangle fixed as $R\to\infty$. Imagine now that $a=a(R)$ also grows with $R$. For example, consider $a(R)=\sqrt{R}$ or $a(R)=R$. For which growth rates of $a(R)$ can you still use the ML inequality to show that the integrals over the vertical sides go to zero? What does this tell you about how far you can move the contour while keeping such error terms under control?
\end{itemize}

\end{problem}

% ===== Example 4: Estimating Contour Integrals with the ML Inequality (full solution) =====
\begin{problem}[Estimating Contour Integrals with the ML Inequality]
Let $a>0$ and $f(z)=e^{-z^{2}}$. For each $R>0$, let $C_R$ be the rectangle with vertices $-R$, $R$, $R+ia$, and $-R+ia$, oriented counterclockwise.

\begin{enumerate}
  \item[(i)] State the ML inequality and use it to show that the integrals
  \[
  \int_{\Gamma_R^{(1)}} e^{-z^{2}}\,dz
  \quad\text{and}\quad
  \int_{\Gamma_R^{(2)}} e^{-z^{2}}\,dz,
  \]
  over the right and left vertical sides of $C_R$, tend to $0$ as $R\to\infty$.

  \item[(ii)] Write $\displaystyle\int_{C_R} e^{-z^{2}}\,dz$ as the sum of integrals over the four sides of $C_R$, and use Cauchy's theorem together with part (i) to prove that
  \[
  \int_{-\infty}^{\infty} e^{-(x+ia)^{2}}\,dx
  =
  \int_{-\infty}^{\infty} e^{-x^{2}}\,dx.
  \]
\end{enumerate}

Explain briefly how this example illustrates the use of the ML inequality to control parts of a contour when deforming it.
\end{problem}

\begin{solution}
We begin by recalling the ML inequality and then apply it to the vertical sides of the rectangle. After that we use Cauchy's theorem to relate integrals along two horizontal lines.

\medskip

\emph{Step 1: The ML inequality.}  
Let $C$ be a smooth contour parametrized by $\gamma:[a,b]\to\mathbb{C}$ and let $f$ be continuous on $C$. Suppose that $|f(z)|\le M$ for all $z\in C$, and denote by $L$ the length of $C$. Then
\[
\int_{C} f(z)\,dz = \int_a^b f(\gamma(t))\,\gamma'(t)\,dt.
\]
Taking absolute values and using the triangle inequality gives
\[
\left|\int_{C} f(z)\,dz\right|
  = \left|\int_a^b f(\gamma(t))\,\gamma'(t)\,dt\right|
  \le \int_a^b \left|f(\gamma(t))\right|\,\left|\gamma'(t)\right|\,dt.
\]
Since $\left|f(\gamma(t))\right|\le M$ on $[a,b]$, we obtain
\[
\left|\int_{C} f(z)\,dz\right|
  \le M \int_a^b \left|\gamma'(t)\right|\,dt
  = M\,L.
\]
This is the ML inequality.

\medskip

\emph{Step 2: Bounding $e^{-z^{2}}$ on the vertical sides.}  
Write $z=x+iy$, with $x,y\in\mathbb{R}$. Then
\[
z^{2}=(x+iy)^{2}=x^{2}-y^{2}+2ixy,
\]
so
\[
-z^{2}=-(x^{2}-y^{2})-2ixy.
\]
Hence
\[
\left|e^{-z^{2}}\right|
  = \left|e^{-(x^{2}-y^{2})-2ixy}\right|
  = e^{-(x^{2}-y^{2})},
\]
because the modulus of $e^{i\theta}$ is $1$ for any real $\theta$.

Now consider the right vertical side $\Gamma_R^{(1)}$ of $C_R$, which runs from $R$ to $R+ia$. Points on this segment have the form $z=R+iy$ with $0\le y\le a$, so $x=R$ and $y\in[0,a]$. Then
\[
x^{2}-y^{2} = R^{2}-y^{2} \ge R^{2}-a^{2}.
\]
Therefore,
\[
\left|e^{-z^{2}}\right|
 = e^{-(x^{2}-y^{2})}
 \le e^{-(R^{2}-a^{2})}
\quad\text{for all }z\in\Gamma_R^{(1)}.
\]
An identical calculation applies to the left vertical side $\Gamma_R^{(2)}$, where $z=-R+iy$ with $0\le y\le a$. There we again have $x^{2}=R^{2}$ and $y^{2}\le a^{2}$, so
\[
\left|e^{-z^{2}}\right|\le e^{-(R^{2}-a^{2})}
\quad\text{for all }z\in\Gamma_R^{(2)}.
\]

\medskip

\emph{Step 3: Applying the ML inequality to the vertical sides (part (i)).}  
The length of each vertical side is $a$, since the imaginary part runs from $0$ to $a$.

On $\Gamma_R^{(1)}$ we have $|f(z)|\le e^{-(R^{2}-a^{2})}$ and $\mathrm{length}(\Gamma_R^{(1)})=a$. By the ML inequality,
\[
\left|\int_{\Gamma_R^{(1)}} e^{-z^{2}}\,dz\right|
 \le e^{-(R^{2}-a^{2})}\cdot a.
\]
Similarly,
\[
\left|\int_{\Gamma_R^{(2)}} e^{-z^{2}}\,dz\right|
 \le e^{-(R^{2}-a^{2})}\cdot a.
\]
Since $a$ is fixed and $R^{2}-a^{2}\to\infty$ as $R\to\infty$, both bounds tend to zero:
\[
a\,e^{-(R^{2}-a^{2})}\longrightarrow 0
\quad\text{as }R\to\infty.
\]
Thus
\[
\int_{\Gamma_R^{(1)}} e^{-z^{2}}\,dz \to 0
\quad\text{and}\quad
\int_{\Gamma_R^{(2)}} e^{-z^{2}}\,dz \to 0
\quad\text{as }R\to\infty.
\]
This proves part (i).

\medskip

\emph{Step 4: Decomposing the contour integral.}  
We now turn to part (ii). The rectangle $C_R$ has four sides:

\begin{itemize}
  \item Bottom side: from $-R$ to $R$ along the real axis.
  \item Right side: $\Gamma_R^{(1)}$, from $R$ to $R+ia$.
  \item Top side: from $R+ia$ to $-R+ia$ along the line $\Im z = a$.
  \item Left side: $\Gamma_R^{(2)}$, from $-R+ia$ to $-R$.
\end{itemize}

Parameterizing each side:

\begin{itemize}
  \item Bottom: $z=x$ with $x$ from $-R$ to $R$ gives the integral
  \[
  \int_{\text{bottom}} e^{-z^{2}}\,dz
    = \int_{-R}^{R} e^{-x^{2}}\,dx.
  \]
  \item Right: this is $\Gamma_R^{(1)}$, already discussed.
  \item Top: $z=x+ia$ with $x$ from $R$ to $-R$ (note the direction) gives
  \[
  \int_{\text{top}} e^{-z^{2}}\,dz
    = \int_{R}^{-R} e^{-(x+ia)^{2}}\,dx
    = -\int_{-R}^{R} e^{-(x+ia)^{2}}\,dx.
  \]
  \item Left: this is $\Gamma_R^{(2)}$, already discussed.
\end{itemize}

Therefore,
\[
\int_{C_R} e^{-z^{2}}\,dz
 =
 \int_{-R}^{R} e^{-x^{2}}\,dx
 + \int_{\Gamma_R^{(1)}} e^{-z^{2}}\,dz
 - \int_{-R}^{R} e^{-(x+ia)^{2}}\,dx
 + \int_{\Gamma_R^{(2)}} e^{-z^{2}}\,dz.
\]

\medskip

\emph{Step 5: Using Cauchy's theorem.}  
The function $e^{-z^{2}}$ is entire, hence analytic everywhere in $\mathbb{C}$. In particular it is analytic in a neighborhood of the rectangle $C_R$ and its interior. By Cauchy's theorem,
\[
\int_{C_R} e^{-z^{2}}\,dz = 0.
\]
Thus the identity from Step 4 simplifies to
\[
0
 = \int_{-R}^{R} e^{-x^{2}}\,dx
 + \int_{\Gamma_R^{(1)}} e^{-z^{2}}\,dz
 - \int_{-R}^{R} e^{-(x+ia)^{2}}\,dx
 + \int_{\Gamma_R^{(2)}} e^{-z^{2}}\,dz.
\]
Rewriting, we obtain
\[
\int_{-R}^{R} e^{-(x+ia)^{2}}\,dx
 =
 \int_{-R}^{R} e^{-x^{2}}\,dx
 + \int_{\Gamma_R^{(1)}} e^{-z^{2}}\,dz
 + \int_{\Gamma_R^{(2)}} e^{-z^{2}}\,dz.
\]

\medskip

\emph{Step 6: Passing to the limit as $R\to\infty$.}  
By definition, the improper integrals
\[
\int_{-\infty}^{\infty} e^{-x^{2}}\,dx
\quad\text{and}\quad
\int_{-\infty}^{\infty} e^{-(x+ia)^{2}}\,dx
\]
are the limits of the corresponding integrals over $[-R,R]$ as $R\to\infty$, provided these limits exist. The function $e^{-x^{2}}$ is positive and rapidly decaying, so $\int_{-\infty}^{\infty} e^{-x^{2}}\,dx$ certainly converges. Moreover,
\[
\left|e^{-(x+ia)^{2}}\right|
 = e^{-(x^{2}-a^{2})}
 \le e^{a^{2}} e^{-x^{2}},
\]
so $e^{-(x+ia)^{2}}$ is absolutely integrable on $\mathbb{R}$ as well, and its improper integral converges.

From Step 5 we have, for each $R>0$,
\[
\int_{-R}^{R} e^{-(x+ia)^{2}}\,dx
 =
 \int_{-R}^{R} e^{-x^{2}}\,dx
 + \int_{\Gamma_R^{(1)}} e^{-z^{2}}\,dz
 + \int_{\Gamma_R^{(2)}} e^{-z^{2}}\,dz.
\]
We have already shown in Step 3 that
\[
\int_{\Gamma_R^{(1)}} e^{-z^{2}}\,dz \to 0
\quad\text{and}\quad
\int_{\Gamma_R^{(2)}} e^{-z^{2}}\,dz \to 0
\quad\text{as }R\to\infty.
\]
Therefore, taking $R\to\infty$ in the preceding identity yields
\[
\int_{-\infty}^{\infty} e^{-(x+ia)^{2}}\,dx
 =
 \int_{-\infty}^{\infty} e^{-x^{2}}\,dx.
\]
This proves part (ii).

\medskip

\emph{Conceptual remark.}  
This example illustrates two central ideas from the chapter section on ``Analytic Functions and Integration along Contours.'' First, Cauchy's theorem tells us that the integral of an analytic function around a closed contour is zero, so we are free to deform contours as long as we do not cross singularities. Second, the ML inequality provides a quantitative estimate on integrals along parts of a contour. Here we used it to show that the contributions from the vertical sides vanish as the rectangle widens, which justifies replacing an integral over the real axis by an integral over a parallel horizontal line without changing its value. This combination of qualitative (Cauchy) and quantitative (ML) tools is typical in applications where we control error terms or justify contour deformations.
\end{solution}

% ===== Example 5: Contour Deformation and a Real Integral with Oscillations (inquiry-based) =====
\begin{problem}[Contour Deformation and a Real Integral with Oscillations]
Oscillatory integrals such as Fourier transforms appear throughout applied mathematics, for instance when solving the heat equation or studying wave propagation. A typical example is the integral of a rapidly decaying function multiplied by a sine or cosine. Direct real-variable methods can work, but they often obscure the role played by analyticity and contour deformation in the complex plane. In this problem you will recast such an integral as the integral of an analytic function, deform the contour, and see how oscillations become simple exponential factors after a clever shift in the complex plane.

Consider the real integral
\[
I(a) \;=\; \int_{-\infty}^{\infty} e^{-x^2}\cos(ax)\,dx,
\]
where $a\in\mathbb{R}$ is a fixed parameter.

\smallskip

(a) Rewrite $I(a)$ using complex exponentials. Define
\[
J(a) \;=\; \int_{-\infty}^{\infty} e^{-x^2}e^{iax}\,dx.
\]
Express $I(a)$ and the related sine integral
\[
K(a) \;=\; \int_{-\infty}^{\infty} e^{-x^2}\sin(ax)\,dx
\]
in terms of $J(a)$. Which parts (real or imaginary) of $J(a)$ correspond to $I(a)$ and $K(a)$?

% Hint: Use $e^{iax} = \cos(ax) + i\sin(ax)$ and compare real and imaginary parts.

\smallskip

(b) Let $f(z) = e^{-z^2}e^{iaz}$, an entire function of the complex variable $z=x+iy$. For $R>0$, consider the rectangle $C_R$ with vertices
\[
-R,\quad R,\quad R + i\frac{a}{2},\quad -R + i\frac{a}{2},
\]
traversed counterclockwise.

(i) Argue briefly why $f$ is analytic on and inside $C_R$, and state a theorem from complex analysis that tells you the integral of $f$ around $C_R$ is zero.

(ii) Decompose the integral around $C_R$ into the sum of integrals over its four sides. Write this decomposition explicitly in terms of integrals over:
\[
[-R,R] \text{ (bottom)},\quad [0,\tfrac{a}{2}] \text{ (right side)},\quad [R,-R] \text{ (top)},\quad [\tfrac{a}{2},0] \text{ (left side)}.
\]

% Hint: Parametrize each side as $z(t)$ and write $\int_{C_R} f(z)\,dz$ as a sum of four integrals.

\smallskip

(c) Show that as $R\to\infty$, the integrals of $f$ over the two vertical sides of $C_R$ tend to zero.

More precisely, show that there exists a constant $M_a>0$ (depending only on $a$) such that for all sufficiently large $R$,
\[
\left|\int_{R}^{R+i\frac{a}{2}} f(z)\,dz\right| \le M_a e^{-R^2}
\quad\text{and}\quad
\left|\int_{-R+i\frac{a}{2}}^{-R} f(z)\,dz\right| \le M_a e^{-R^2}.
\]
Explain why this implies that both vertical-side integrals vanish in the limit $R\to\infty$.

Hint: On the right side, parametrize $z = R + iy$ with $0\le y\le a/2$, estimate $|f(z)|$, and then apply the ML-inequality. Do the same on the left side.

\smallskip

(d) Now analyze the top and bottom sides of $C_R$.

(i) Show that the integral of $f$ over the bottom side tends, as $R\to\infty$, to $J(a)$:
\[
\lim_{R\to\infty}\int_{-R}^{R} e^{-x^2}e^{iax}\,dx = J(a).
\]

(ii) Parametrize the top side as $z = x + i\frac{a}{2}$ with $x$ going from $R$ to $-R$. Compute $f(x+i\frac{a}{2})$ explicitly and simplify the exponent. Show that
\[
f\!\left(x+i\frac{a}{2}\right) = e^{-x^2 - \frac{a^2}{4}}
\]
is a real-valued function of $x$.

(iii) Use part (ii) to show that the integral over the top side equals
\[
\int_{R+i\frac{a}{2}}^{-R+i\frac{a}{2}} f(z)\,dz
= -e^{-\frac{a^2}{4}} \int_{-R}^{R} e^{-x^2}\,dx.
\]
Then pass to the limit $R\to\infty$ and use the known value
\[
\int_{-\infty}^{\infty} e^{-x^2}\,dx = \sqrt{\pi}.
\]

(iv) Combine all the limits from parts (c) and (d) with the fact that $\displaystyle \int_{C_R} f(z)\,dz = 0$ for each $R$ to deduce a formula for $J(a)$, and hence for $I(a)$ and $K(a)$.

% Hint: In the limit, the sum of the four sides of $C_R$ is $0$, but two sides vanish.

\smallskip

(e) Extensions and variations.

(i) What is the value of $K(a)$? Interpret your answer in terms of symmetry of the integrand.

(ii) Suppose instead we consider
\[
I_\alpha(a) := \int_{-\infty}^{\infty} e^{-\alpha x^2}\cos(ax)\,dx,
\quad \alpha>0.
\]
How would you adapt the contour-shifting argument to compute $I_\alpha(a)$? Outline the changes needed, and guess the final formula.

(iii) More challenging: Imagine attempting a similar contour deformation for the integral
\[
\int_0^\infty \frac{e^{-x}\sin(bx)}{x}\,dx, \quad b>0.
\]
What new difficulties arise (for example, near $x=0$ or at infinity), and what additional tools from complex analysis (such as branch cuts or small semicircular indentations) might be needed to handle them?

\end{problem}

% ===== Example 5: Contour Deformation and a Real Integral with Oscillations (full solution) =====
\begin{problem}[Contour Deformation and a Real Integral with Oscillations]
Evaluate the integral
\[
I(a) = \int_{-\infty}^{\infty} e^{-x^2}\cos(ax)\,dx,\qquad a\in\mathbb{R},
\]
by viewing it as the real part of an integral of an analytic function and using contour deformation in the complex plane. Justify carefully why the contour deformation is allowed and how the contributions from the additional contour segments behave as their lengths tend to infinity.
\end{problem}

\begin{solution}
We begin by encoding the cosine using complex exponentials. Define
\[
J(a) := \int_{-\infty}^{\infty} e^{-x^2}e^{iax}\,dx.
\]
Since $e^{iax} = \cos(ax) + i\sin(ax)$, we can write
\[
J(a) = \int_{-\infty}^{\infty} e^{-x^2}\cos(ax)\,dx
      + i\int_{-\infty}^{\infty} e^{-x^2}\sin(ax)\,dx.
\]
Thus
\[
I(a) = \Re J(a),\qquad 
K(a) := \int_{-\infty}^{\infty} e^{-x^2}\sin(ax)\,dx = \Im J(a).
\]
It therefore suffices to compute $J(a)$ using complex analysis and then take real and imaginary parts.

\medskip

\noindent\textbf{1. Analytic function and contour.}
Consider the complex function
\[
f(z) = e^{-z^2}e^{iaz},\qquad z\in\mathbb{C}.
\]
Both $e^{-z^2}$ and $e^{iaz}$ are entire functions, hence $f$ is entire (analytic on all of $\mathbb{C}$).

For $R>0$, let $C_R$ be the rectangle with vertices
\[
-R,\quad R,\quad R + i\frac{a}{2},\quad -R + i\frac{a}{2},
\]
traversed counterclockwise. Since $f$ is analytic everywhere, in particular on and inside $C_R$, Cauchy’s theorem gives
\[
\int_{C_R} f(z)\,dz = 0.
\]

We now decompose the contour integral into four pieces corresponding to the sides of the rectangle:
\[
\int_{C_R} f(z)\,dz
= \int_{\text{bottom}} f(z)\,dz
+ \int_{\text{right}} f(z)\,dz
+ \int_{\text{top}} f(z)\,dz
+ \int_{\text{left}} f(z)\,dz
= 0.
\]

\medskip

\noindent\textbf{2. Parametrization of the sides.}
We parametrize each side:

\begin{itemize}
\item Bottom side: $z = x$ with $x$ from $-R$ to $R$.
\[
\int_{\text{bottom}} f(z)\,dz = \int_{-R}^{R} e^{-x^2}e^{iax}\,dx.
\]

\item Right side: $z = R + iy$ with $y$ from $0$ to $a/2$.
\[
\int_{\text{right}} f(z)\,dz = \int_{0}^{a/2} e^{-(R+iy)^2}e^{ia(R+iy)}\,i\,dy.
\]

\item Top side: $z = x + i\frac{a}{2}$ with $x$ from $R$ to $-R$.
\[
\int_{\text{top}} f(z)\,dz = \int_{R}^{-R} e^{-(x+i\frac{a}{2})^2}e^{ia(x+i\frac{a}{2})}\,dx.
\]

\item Left side: $z = -R + iy$ with $y$ from $a/2$ down to $0$.
\[
\int_{\text{left}} f(z)\,dz
= \int_{a/2}^{0} e^{-(-R+iy)^2}e^{ia(-R+iy)}\,i\,dy.
\]
\end{itemize}

Thus
\[
\int_{-R}^{R} e^{-x^2}e^{iax}\,dx
+ \int_{\text{right}} f(z)\,dz
+ \int_{\text{top}} f(z)\,dz
+ \int_{\text{left}} f(z)\,dz
= 0.
\]

\medskip

\noindent\textbf{3. Estimates on the vertical sides.}
We now show that the integrals over the vertical sides tend to zero as $R\to\infty$.

\smallskip

\emph{Right side.} On $z = R+iy$ with $0\le y\le a/2$, we have
\[
z^2 = (R+iy)^2 = R^2 - y^2 + 2iRy,
\]
so
\[
e^{-z^2} = e^{-(R^2 - y^2)}e^{-2iRy},
\quad\text{and}\quad
|e^{-z^2}| = e^{-(R^2 - y^2)}.
\]
Also,
\[
e^{iaz} = e^{ia(R+iy)} = e^{iaR}e^{-ay},
\quad\text{so}\quad
|e^{iaz}| = e^{-ay} \le 1
\]
for $y\ge 0$. Therefore
\[
|f(z)| = |e^{-z^2}e^{iaz}|
= |e^{-z^2}|\cdot|e^{iaz}|
\le e^{-(R^2 - y^2)}.
\]
Since $0\le y\le a/2$, we have $y^2 \le a^2/4$, and hence
\[
|f(z)| \le e^{-(R^2 - a^2/4)} = e^{-R^2}e^{a^2/4}.
\]
Thus, on the right side,
\[
\left|\int_{\text{right}} f(z)\,dz\right|
\le \max_{0\le y\le a/2} |f(R+iy)|\cdot \text{(length of side)}
\le e^{-R^2}e^{a^2/4}\cdot \frac{|a|}{2}.
\]
As $R\to\infty$, the factor $e^{-R^2}$ forces this bound to zero, so
\[
\lim_{R\to\infty} \int_{\text{right}} f(z)\,dz = 0.
\]

\smallskip

\emph{Left side.} A similar estimate applies on $z = -R+iy$, $0\le y\le a/2$. We have
\[
z^2 = (-R+iy)^2 = R^2 - y^2 - 2iRy,
\]
so again $|e^{-z^2}| = e^{-(R^2 - y^2)}$ and $|e^{iaz}| = e^{-ay}\le1$, giving the same bound
\[
|f(z)| \le e^{-(R^2 - a^2/4)} = e^{-R^2}e^{a^2/4}.
\]
By the same ML-estimate argument, this implies
\[
\lim_{R\to\infty} \int_{\text{left}} f(z)\,dz = 0.
\]

Thus both vertical-side contributions vanish in the limit $R\to\infty$.

\medskip

\noindent\textbf{4. Limits of the horizontal sides.}
We next examine the bottom and top sides.

\smallskip

\emph{Bottom side.} On the bottom, $z = x$ with $x\in[-R,R]$, so
\[
\int_{\text{bottom}} f(z)\,dz
= \int_{-R}^{R} e^{-x^2}e^{iax}\,dx.
\]
By dominated convergence, as $R\to\infty$ this converges to
\[
\lim_{R\to\infty}\int_{-R}^{R} e^{-x^2}e^{iax}\,dx
= \int_{-\infty}^{\infty} e^{-x^2}e^{iax}\,dx
= J(a).
\]

\smallskip

\emph{Top side.} On the top, $z = x + i\frac{a}{2}$ with $x$ going from $R$ to $-R$, and $dz = dx$. We compute the exponent:
\[
-(x + i\tfrac{a}{2})^2 + ia(x + i\tfrac{a}{2}).
\]
First expand $(x + i\tfrac{a}{2})^2$:
\[
(x + i\tfrac{a}{2})^2 = x^2 + i a x - \frac{a^2}{4}.
\]
Therefore
\[
-(x + i\tfrac{a}{2})^2
= -x^2 - i a x + \frac{a^2}{4}.
\]
Also,
\[
ia(x + i\tfrac{a}{2}) = iax + ia\cdot i\frac{a}{2} = iax - \frac{a^2}{2}.
\]
Adding these two expressions gives
\[
-(x + i\tfrac{a}{2})^2 + ia(x + i\tfrac{a}{2})
= -x^2 - i a x + \frac{a^2}{4} + iax - \frac{a^2}{2}
= -x^2 - \frac{a^2}{4}.
\]
The imaginary terms cancel, so the exponent is real and negative. Hence
\[
f\Bigl(x + i\frac{a}{2}\Bigr)
= e^{-(x + i\frac{a}{2})^2}e^{ia(x + i\frac{a}{2})}
= e^{-x^2 - \frac{a^2}{4}}.
\]
Therefore the integral over the top side is
\[
\int_{\text{top}} f(z)\,dz
= \int_{R}^{-R} e^{-x^2 - \frac{a^2}{4}}\,dx
= -\int_{-R}^{R} e^{-x^2 - \frac{a^2}{4}}\,dx
= -e^{-\frac{a^2}{4}} \int_{-R}^{R} e^{-x^2}\,dx.
\]
Letting $R\to\infty$ and using the standard Gaussian integral
\[
\int_{-\infty}^{\infty} e^{-x^2}\,dx = \sqrt{\pi},
\]
we obtain
\[
\lim_{R\to\infty} \int_{\text{top}} f(z)\,dz
= -e^{-\frac{a^2}{4}}\sqrt{\pi}.
\]

\medskip

\noindent\textbf{5. Putting the pieces together.}
For each $R>0$ we have
\[
\int_{\text{bottom}} f(z)\,dz
+ \int_{\text{right}} f(z)\,dz
+ \int_{\text{top}} f(z)\,dz
+ \int_{\text{left}} f(z)\,dz = 0.
\]
Taking the limit as $R\to\infty$ and using the limits computed above, we get
\[
J(a) + 0 - e^{-\frac{a^2}{4}}\sqrt{\pi} + 0 = 0,
\]
so
\[
J(a) = e^{-\frac{a^2}{4}}\sqrt{\pi}.
\]

Recall that $J(a)$ was defined as
\[
J(a) 
= \int_{-\infty}^{\infty} e^{-x^2}\bigl[\cos(ax) + i\sin(ax)\bigr]\,dx
= I(a) + iK(a).
\]
Since the right-hand side $e^{-a^2/4}\sqrt{\pi}$ is real, we conclude
\[
I(a) = \Re J(a) = e^{-\frac{a^2}{4}}\sqrt{\pi},\qquad
K(a) = \Im J(a) = 0.
\]

Thus the desired oscillatory integral is
\[
\boxed{\,I(a) = \displaystyle\int_{-\infty}^{\infty} e^{-x^2}\cos(ax)\,dx
= \sqrt{\pi}\,e^{-a^2/4}\,}.
\]

\medskip

\noindent\textbf{6. Interpretation and connection to contour methods.}
This example illustrates several central ideas from the study of analytic functions and contour integration:

\begin{itemize}
\item By embedding the real integral in a complex one (passing from $\cos(ax)$ to $e^{iax}$), we obtain an integrand that is entire. This analyticity allows us to deform contours without changing the value of the integral.

\item The rectangular contour $C_R$ deforms the original real line into a parallel line shifted vertically by $i a/2$. Because the function $e^{-z^2}e^{iaz}$ decays extremely rapidly in the real direction, the contributions from the distant vertical sides vanish.

\item On the shifted line, the oscillatory factor $e^{iax}$ is absorbed into a simple multiplicative constant $e^{-a^2/4}$, and the remaining integral is just a Gaussian integral. In this way, oscillations are traded for exponential decay after an appropriate contour shift.

\item Finally, taking real and imaginary parts recovers not only the cosine integral but also the vanishing of the corresponding sine integral, reflecting an underlying symmetry of the integrand.
\end{itemize}

This contour deformation method is a prototype for many techniques in applied complex analysis, where oscillatory real integrals are evaluated by lifting them into the complex plane and exploiting analyticity, decay estimates, and suitable choices of contours.
\end{solution}

\section{Residue Calculus}
% --- Narrative plan (auto-generated) ---
% This section introduces the residue calculus, a powerful method for evaluating contour integrals of complex-valued functions by looking only at their singularities. The central idea is that the behavior of a holomorphic function around isolated singular points can be summarized by a single complex number at each point, called the residue, and that contour integrals can then be computed as finite sums of these residues. We will develop systematic techniques to identify and compute residues for simple poles, higher-order poles, and removable or essential singularities.
%
% Residue calculus matters in applied mathematics because it converts many difficult real integrals into manageable complex-analytic problems, especially those involving oscillatory integrals, rational functions, and exponential factors. It underlies standard tools in signal processing and control theory such as inverse Laplace and Fourier transforms, appears in the solution formulas for linear ODEs via contour integral representations, and plays a key role in constructing Green’s functions for elliptic and parabolic PDEs. Along the way, the techniques connect to partial fraction decompositions, asymptotic methods, and the spectral analysis of linear operators, giving a unifying viewpoint that links complex analysis to differential equations and dynamical systems.
%
% Our narrative will build residue calculus from carefully chosen examples: starting with small contours and simple poles, we learn to compute residues by hand and interpret the residue theorem as a bookkeeping device for integrals. We then move to classic integral evaluations on the real line, the inversion of Laplace transforms, and the summation of series by contour integration. Throughout, we emphasize standard contours, such as semicircles and keyholes, and show how the qualitative behavior of a function at infinity guides the choice of contour, reinforcing the geometric intuition behind the theory.

% ===== Example 1: First Encounters with Simple Poles (inquiry-based) =====
\begin{problem}[First Encounters with Simple Poles]
In this problem we explore how a very local piece of information at each singularity of a function can determine the value of global contour integrals. We use the concrete example
\[
f(z)=\frac{1}{z^{2}+1},
\]
which has two isolated singularities (simple poles) at $z=i$ and $z=-i$. By comparing parameterizations of circles with computations using residues, we will see how the ``long way'' and the ``short way'' of evaluating contour integrals agree, and how contour deformations that avoid singularities leave integrals unchanged.

Consider throughout the function $f(z)=1/(z^{2}+1)$.

\medskip

\noindent
(a) \textbf{A first contour and explicit parameterization.} Let $0<r<1$ and let $C_r$ be the positively oriented (counterclockwise) circle
\[
C_r:\quad |z-i|=r.
\]
Thus $C_r$ encloses $z=i$ but not $z=-i$.

\begin{enumerate}
\item[(i)] Write a parametrization of $C_r$ of the form $z(\theta)=i+r e^{i\theta}$ for $0\le \theta\le 2\pi$, and compute $dz$ in terms of $d\theta$.
\item[(ii)] Using this parametrization, express the contour integral
\[
I_r := \int_{C_r} \frac{1}{z^{2}+1}\,dz
\]
as an ordinary integral with respect to $\theta$. Simplify the integrand as far as you can, but do not worry (yet) about explicitly evaluating the resulting $\theta$–integral.
  
Hint: Factor $z^{2}+1$ as $(z-i)(z+i)$ and use that, on $C_r$, you have $z-i = r e^{i\theta}$ and $z+i = 2i + r e^{i\theta}$.
\end{enumerate}

\medskip

\noindent
(b) \textbf{Zooming in on a simple pole: computing a residue.} The point $z=i$ is a simple pole of $f$. In this part you will compute the residue of $f$ at $z=i$.

\begin{enumerate}
\item[(i)] Show that
\[
z^{2}+1 = (z-i)(z+i),
\]
and hence near $z=i$ you can write
\[
f(z)=\frac{1}{z^{2}+1} = \frac{1}{(z-i)(z+i)}.
\]
\item[(ii)] Recall that if $z_0$ is a simple pole of a function $g$, then
\[
\operatorname{Res}(g;z_0) = \lim_{z\to z_0} (z-z_0)g(z).
\]
Use this to compute $\operatorname{Res}\bigl(f;i\bigr)$.

Hint: You can either plug directly into the limit, or first rewrite $f(z)$ near $z=i$ as
\[
f(z)=\frac{1}{z+i}\cdot\frac{1}{z-i}
\]
and observe what happens when you multiply by $(z-i)$ and then let $z\to i$.
\end{enumerate}

\medskip

\noindent
(c) \textbf{The short way: using the Residue Theorem on a small circle.} The Residue Theorem says that for a meromorphic function with isolated singularities inside a positively oriented simple closed contour $C$,
\[
\int_C f(z)\,dz = 2\pi i \sum_{\text{poles $a_k$ inside $C$}} \operatorname{Res}(f;a_k).
\]

\begin{enumerate}
\item[(i)] Apply the Residue Theorem to the contour $C_r$ from part (a) to evaluate $I_r$ without going through the parameterization integral. (You may assume without proof that $f$ is analytic everywhere except at $z=\pm i$.)
\item[(ii)] Compare your answer here with the expression you obtained in part (a). Does your final value for $I_r$ depend on the radius $r$? What does this suggest about how the integral changes when you vary $r$ while keeping $C_r$ centered at $i$ and not crossing any singularities?
\end{enumerate}

Hint: The only singularity inside $C_r$ is $z=i$; use your computation of $\operatorname{Res}(f;i)$.

\medskip

\noindent
(d) \textbf{Enclosing both poles: cancellation of residues.} Now consider the larger circle
\[
\Gamma_R:\quad |z|=R,\qquad R>1,
\]
oriented counterclockwise. This circle encloses both singularities at $z=i$ and $z=-i$.

\begin{enumerate}
\item[(i)] Compute the residue of $f$ at $z=-i$, that is, compute $\operatorname{Res}(f;-i)$.

Hint: Repeat the same method you used at $z=i$, or use symmetry.
\item[(ii)] Use the Residue Theorem to evaluate
\[
J_R := \int_{\Gamma_R} \frac{1}{z^{2}+1}\,dz.
\]
Does your answer depend on $R$?
\item[(iii)] Compare the residues at $z=i$ and $z=-i$. What is their sum? How does this explain the value you found for $J_R$?
\end{enumerate}

\medskip

\noindent
(e) \textbf{Deforming contours and the topological viewpoint.} The previous parts suggest that the value of $\int f(z)\,dz$ does not depend on the detailed shape of the contour, but rather on which singularities are enclosed and how many times.

\begin{enumerate}
\item[(i)] Let $\gamma$ be any simple closed contour (a ``nice loop'') that winds once counterclockwise around $z=i$ and does not pass through $z=\pm i$, and which does \emph{not} enclose $z=-i$. Argue, using Cauchy's theorem or the Residue Theorem, that
\[
\int_{\gamma} \frac{1}{z^2+1}\,dz = \int_{C_r} \frac{1}{z^2+1}\,dz
\]
for any small circle $C_r$ around $i$ as in part (a).

Hint: Think about continuously deforming $\gamma$ to $C_r$ without crossing any singularities, and recall that integrals of analytic functions around closed curves are invariant under such deformations.
\item[(ii)] What happens if, during such a deformation, the contour crosses the point $z=-i$? Describe qualitatively how and why the integral changes at that moment, in terms of residues.
\item[(iii)] (Optional extension.) Use what you have learned to evaluate the real integral
\[
\int_{-\infty}^{\infty} \frac{dx}{x^{2}+1}
\]
by considering a suitable contour in the complex plane and relating it to a residue of $f$ at one of its poles.
\end{enumerate}

\end{problem}

% ===== Example 1: First Encounters with Simple Poles (full solution) =====
\begin{problem}[First Encounters with Simple Poles]
Let $f(z)=1/(z^{2}+1)$, which has simple poles at $z=\pm i$.

\begin{enumerate}
\item[(a)] For $0<r<1$, let $C_r$ be the circle $|z-i|=r$, oriented counterclockwise. Compute
\[
I_r := \int_{C_r} \frac{1}{z^{2}+1}\,dz
\]
by using residues.

\item[(b)] For $R>1$, let $\Gamma_R$ be the circle $|z|=R$, oriented counterclockwise. Compute
\[
J_R := \int_{\Gamma_R} \frac{1}{z^{2}+1}\,dz
\]
by using residues.

\item[(c)] Let $\gamma$ be any simple closed contour that winds once counterclockwise around $z=i$, encloses no other singularities of $f$, and does not pass through $z=\pm i$. Explain why
\[
\int_{\gamma} \frac{1}{z^{2}+1}\,dz = I_r
\]
for every $0<r<1$.

\item[(d)] Briefly state how this example illustrates the key ideas of residue calculus and contour deformation in complex analysis.
\end{enumerate}
\end{problem}

\begin{solution}
We work throughout with the function
\[
f(z)=\frac{1}{z^{2}+1}.
\]
The singularities of $f$ are at the zeros of $z^{2}+1$, namely $z=i$ and $z=-i$. Each of these is a simple pole.

\medskip

\noindent\textbf{Residues at $z=i$ and $z=-i$.}
First we compute the residues, since they will be used in all parts. We factor
\[
z^{2}+1 = (z-i)(z+i),
\]
so
\[
f(z)=\frac{1}{(z-i)(z+i)}.
\]

For a simple pole at $z_0$, the residue is given by
\[
\operatorname{Res}(f;z_0) = \lim_{z\to z_0} (z-z_0) f(z).
\]

At $z=i$ we have
\[
\operatorname{Res}(f;i)
= \lim_{z\to i} (z-i)\frac{1}{(z-i)(z+i)}
= \lim_{z\to i} \frac{1}{z+i}
= \frac{1}{i+i}
= \frac{1}{2i}.
\]

At $z=-i$ we similarly obtain
\[
\operatorname{Res}(f;-i)
= \lim_{z\to -i} (z+i)\frac{1}{(z-i)(z+i)}
= \lim_{z\to -i} \frac{1}{z-i}
= \frac{1}{-i - i}
= \frac{1}{-2i}
= -\frac{1}{2i}.
\]

Thus the residues at the two simple poles are equal in magnitude and opposite in sign:
\[
\operatorname{Res}(f;i) = \frac{1}{2i},\qquad
\operatorname{Res}(f;-i) = -\frac{1}{2i}.
\]

\medskip

\noindent\textbf{(a) Integral around a small circle enclosing only $z=i$.}
Let $0<r<1$, and let $C_r$ be the circle $|z-i|=r$, positively oriented. This circle encloses $z=i$ but not $z=-i$, because the distance from $i$ to $-i$ is $2$, while $r<1$.

On and inside $C_r$, the function $f$ is meromorphic with a single singularity at $z=i$, and it is analytic elsewhere. By the Residue Theorem,
\[
I_r := \int_{C_r} \frac{1}{z^{2}+1}\,dz
= 2\pi i \sum_{\text{poles inside } C_r} \operatorname{Res}(f;\text{pole})
= 2\pi i \,\operatorname{Res}(f;i),
\]
since $z=i$ is the only pole inside $C_r$.

Using the residue computed above,
\[
I_r = 2\pi i \cdot \frac{1}{2i} = \pi.
\]

Thus the integral around any such small circle is
\[
\int_{C_r} \frac{1}{z^{2}+1}\,dz = \pi,
\]
and we note that this value is independent of the radius $r$ (as long as $0<r<1$ so that the contour still encloses $i$ but not $-i$).

\medskip

\noindent\textbf{(b) Integral around a large circle enclosing both poles.}
Now let $R>1$ and consider the circle $\Gamma_R: |z|=R$, oriented counterclockwise. This circle encloses both $z=i$ and $z=-i$, since both have modulus $1$.

Inside and on $\Gamma_R$ the only singularities of $f$ are the simple poles at $z=i$ and $z=-i$. Again by the Residue Theorem,
\[
J_R := \int_{\Gamma_R} \frac{1}{z^{2}+1}\,dz
= 2\pi i \Bigl(\operatorname{Res}(f;i) + \operatorname{Res}(f;-i)\Bigr).
\]

Substituting the values of the residues,
\[
\operatorname{Res}(f;i) + \operatorname{Res}(f;-i)
= \frac{1}{2i} - \frac{1}{2i} = 0.
\]
Hence
\[
J_R = 2\pi i \cdot 0 = 0.
\]

Thus the integral around the large circle is zero, for every $R>1$.

This vanishing is a concrete instance of the general phenomenon that if all residues inside a contour sum to zero, then the contour integral of the function is zero. For rational functions that decay like $1/z^{2}$ or faster at infinity, this is related to the fact that the sum of all residues in the finite plane must be zero.

\medskip

\noindent\textbf{(c) Independence of the contour shape when no singularities are crossed.}
Let $\gamma$ be any simple closed contour that winds once counterclockwise around $z=i$, encloses no other singularities of $f$, and does not pass through $z=\pm i$.

By hypothesis, the region enclosed by $\gamma$ contains $z=i$ but not $z=-i$. Therefore the only pole of $f$ inside $\gamma$ is $z=i$, and $f$ is analytic on and inside $\gamma$ except at that single point.

Apply the Residue Theorem to $\gamma$:
\[
\int_{\gamma} \frac{1}{z^{2}+1}\,dz
= 2\pi i \sum_{\text{poles inside } \gamma} \operatorname{Res}(f;\text{pole})
= 2\pi i\,\operatorname{Res}(f;i).
\]
Using the result from part (a), we also have
\[
I_r = \int_{C_r} \frac{1}{z^{2}+1}\,dz
= 2\pi i\,\operatorname{Res}(f;i).
\]
Hence
\[
\int_{\gamma} \frac{1}{z^{2}+1}\,dz
= \int_{C_r} \frac{1}{z^{2}+1}\,dz = \pi
\]
for every $0<r<1$.

Conceptually, we can also understand this through contour deformation. The function $f$ is analytic on the open region between $\gamma$ and $C_r$ (an annular region that excludes both poles), so by Cauchy's theorem the integral of $f$ around the boundary of that region is zero. The boundary consists of $\gamma$ and $-C_r$ (the latter oriented in the opposite direction), so
\[
\int_{\gamma} f(z)\,dz - \int_{C_r} f(z)\,dz = 0,
\]
which implies the same equality of integrals. This shows that the actual value depends only on \emph{which} singularities are enclosed and the winding number, not the geometric details of the contour.

\medskip

\noindent\textbf{(d) Illustration of residue calculus and contour deformation.}
This example encapsulates several central ideas of residue calculus:

\begin{itemize}
\item \emph{Local data at simple poles controls global integrals.} The integrals $I_r$ and $\int_{\gamma} f(z)\,dz$ are completely determined by the single local number $\operatorname{Res}(f;i)=1/(2i)$; explicitly, each is $2\pi i$ times that residue.

\item \emph{Cancellation of residues.} For the large circle $\Gamma_R$, the enclosed residues at $i$ and $-i$ sum to zero, and the contour integral vanishes. This illustrates how contributions from different poles can cancel.

\item \emph{Topological nature of contour integrals.} The value of $\int f(z)\,dz$ around a closed contour depends only on which singularities are enclosed and with what winding number, not on the specific shape or size of the contour. Deforming a contour without crossing singularities does not change the integral, as seen in the equality
\[
\int_{\gamma} \frac{1}{z^{2}+1}\,dz = \int_{C_r} \frac{1}{z^{2}+1}\,dz.
\]
\end{itemize}

Altogether, the function $f(z)=1/(z^{2}+1)$ provides a first, very concrete encounter with simple poles and residues, and shows how local singular behavior dictates global contour integrals through the Residue Theorem and the principle of contour deformation.
\end{solution}

% ===== Example 2: Evaluating a Real Integral via a Semicircular Contour (inquiry-based) =====
\begin{problem}[Evaluating a Real Integral via a Semicircular Contour]
Many important real integrals can be viewed as the real line sitting inside the complex plane.  By extending the integrand to a complex function and integrating around a suitable contour, one can often convert a difficult real integral into a residue computation.  In this problem you will see how this works in one of the simplest cases, the integral of the Cauchy density.  This will serve as a model for more complicated integrals that arise in Fourier analysis and probability.

Consider the real integral
\[
I \;=\; \int_{-\infty}^{\infty} \frac{1}{x^{2}+1}\,dx.
\]

\smallskip

(a) Warm-up in real calculus.  

\quad (i) Show that the integrand is an even function and rewrite $I$ as an integral over $[0,\infty)$.  

\quad (ii) Using only real-variable calculus (for example, by recognizing an antiderivative or using a trigonometric substitution), evaluate $I$ directly.  Keep this answer in mind as a reference for the complex-analytic method. 
% Hint: One approach is to note that $\frac{d}{dx} \arctan x = \frac{1}{1+x^2}$.

\smallskip

(b) Setting up the complex contour.  

Define the complex function
\[
f(z) \;=\; \frac{1}{z^{2}+1},
\]
and for each $R>1$ consider the contour $C_R$ consisting of the line segment from $-R$ to $R$ along the real axis together with the upper semicircle
\[
\gamma_R : z(\theta) = Re^{i\theta}, \quad \theta \in [0,\pi].
\]
Thus $C_R = [-R,R] \cup \gamma_R$, oriented counterclockwise.

\quad (i) Identify the singularities (poles) of $f$ in the complex plane and determine which of them lie inside the upper half-plane.  

\quad (ii) Explain why, for each $R>1$, exactly one of these poles lies inside the contour $C_R$.  (A small sketch may help.)
% Hint: Solve $z^2+1=0$ and look at the imaginary parts of the solutions.

\smallskip

(c) Computing the relevant residue.

Compute the residue of $f$ at the pole $z=i$:
\[
\operatorname{Res}(f;i).
\]
You may use the standard formula for a simple pole, or compute the limit directly.
% Hint: For a simple pole at $z_0$, $\operatorname{Res}(f;z_0) = \lim_{z\to z_0} (z-z_0)f(z)$.

\smallskip

(d) Applying the residue theorem and estimating the arc.  

\quad (i) Use the residue theorem to express the contour integral
\[
\int_{C_R} f(z)\,dz
\]
in terms of the residue you found in part (c).  (This should give a constant that does not depend on $R$.)

\quad (ii) Decompose the contour integral into its two parts:
\[
\int_{C_R} f(z)\,dz \;=\; \int_{-R}^{R} \frac{1}{x^{2}+1}\,dx \;+\; \int_{\gamma_R} f(z)\,dz.
\]
Explain why the first integral on the right tends to $I$ as $R\to\infty$.

\quad (iii) Show that the integral over the semicircular arc $\gamma_R$ tends to $0$ as $R\to\infty$.  

More precisely, use the parametrization $z(\theta) = Re^{i\theta}$ and the Estimation Lemma (ML-inequality) to prove that
\[
\left| \int_{\gamma_R} f(z)\,dz \right| \;\le\; \frac{\pi R}{R^{2}-1},
\]
for all $R>1$, and conclude that this bound tends to $0$ as $R \to \infty$.
% Hint: Estimate $|f(z)|$ on $\gamma_R$ using $|z|=R$ and $|z^2+1|\ge R^2-1$ when $R>1$.

\quad (iv) Combine your results from parts (d)(i)--(d)(iii) to determine the value of $I$ using complex analysis.  Check that your answer agrees with the real-calculus computation from part (a).

\smallskip

(e) Extensions and variations.

\quad (i) Suppose we replace the integrand by
\[
\frac{e^{ix}}{x^{2}+1}.
\]
Briefly explain how you would modify the contour argument to evaluate
\[
\int_{-\infty}^{\infty} \frac{e^{ix}}{x^{2}+1}\,dx.
\]
Which half-plane (upper or lower) would you choose for your semicircle, and why?
% Hint: Think about the size of $e^{iz} = e^{ix - y}$ when $\operatorname{Im} z = y$ is positive or negative.

\quad (ii) What changes, if any, would be needed if you worked with a lower semicircular contour instead of an upper one for the original integral $I$?  Comment on how the choice of contour is guided by the behavior of the integrand.

\end{problem}

% ===== Example 2: Evaluating a Real Integral via a Semicircular Contour (full solution) =====
\begin{problem}[Evaluating a Real Integral via a Semicircular Contour]
Use the residue theorem with a semicircular contour to evaluate
\[
I = \int_{-\infty}^{\infty} \frac{1}{x^{2}+1}\,dx.
\]
Specifically, consider the function $f(z) = (z^{2}+1)^{-1}$ and, for $R>1$, the contour $C_R$ consisting of the interval $[-R,R]$ on the real axis together with the upper semicircle of radius $R$ centered at the origin.  Show that the contribution from the semicircular arc tends to $0$ as $R\to\infty$, and use the residue at $z=i$ to compute $I$.  Briefly indicate how this method reflects the general strategy of residue calculus for real integrals.
\end{problem}

\begin{solution}
We wish to evaluate the improper integral
\[
I = \int_{-\infty}^{\infty} \frac{1}{x^{2}+1}\,dx
\]
using complex integration and the residue theorem.  Although this integral is easily handled by real-variable methods, it provides a clean model for the contour techniques that will apply to more complicated integrals.

\medskip

\textbf{1. The complex function and the contour.}

Define the complex function
\[
f(z) = \frac{1}{z^{2}+1}.
\]
The singularities of this function are the solutions of $z^{2}+1=0$, namely $z=\pm i$.  Both are simple poles.  We choose a contour that includes the real line segment where our integral lives and that encloses one of these poles.

For $R>1$, let $C_R$ be the contour obtained by traversing the interval $[-R,R]$ on the real axis from left to right, and then returning from $R$ to $-R$ along the upper semicircle of radius $R$ centered at the origin.  We parametrize the semicircle as
\[
\gamma_R : z(\theta) = Re^{i\theta}, \quad \theta\in[0,\pi],
\]
so that $C_R = [-R,R] \cup \gamma_R$, oriented counterclockwise.

The pole at $z=i$ lies in the upper half-plane, while $z=-i$ lies in the lower half-plane.  For every $R>1$, the point $z=i$ lies inside the semicircle of radius $R$, so $i$ is the only singularity of $f$ inside $C_R$.

\medskip

\textbf{2. The residue at $z=i$.}

Since $z=i$ is a simple pole of $f$, we may compute the residue using the standard formula
\[
\operatorname{Res}(f;i)
= \lim_{z\to i} (z-i)f(z)
= \lim_{z\to i} \frac{z-i}{z^{2}+1}.
\]
We factor the denominator:
\[
z^{2}+1 = (z-i)(z+i),
\]
so for $z\neq \pm i$,
\[
(z-i)f(z) = \frac{z-i}{(z-i)(z+i)} = \frac{1}{z+i}.
\]
Therefore
\[
\operatorname{Res}(f;i) = \lim_{z\to i} \frac{1}{z+i} = \frac{1}{i+i} = \frac{1}{2i}.
\]

\medskip

\textbf{3. Applying the residue theorem on $C_R$.}

By the residue theorem, the integral of $f$ around the closed contour $C_R$ is given by
\[
\int_{C_R} f(z)\,dz
= 2\pi i \sum \operatorname{Res}(f;\,\text{poles inside }C_R)
= 2\pi i \,\operatorname{Res}(f;i),
\]
since $i$ is the only pole inside $C_R$.  Substituting the residue we just computed,
\[
\int_{C_R} f(z)\,dz
= 2\pi i \cdot \frac{1}{2i} = \pi.
\]
Thus, for every $R>1$, we have
\[
\int_{C_R} f(z)\,dz = \pi.
\]

On the other hand, we can decompose the contour integral into its two pieces:
\[
\int_{C_R} f(z)\,dz
= \int_{-R}^{R} \frac{1}{x^{2}+1}\,dx
  \;+\; \int_{\gamma_R} f(z)\,dz.
\]
Combining these two expressions, we obtain
\begin{equation}\label{eq:CR-decomp}
\int_{-R}^{R} \frac{1}{x^{2}+1}\,dx + \int_{\gamma_R} f(z)\,dz = \pi,
\quad \text{for all } R>1.
\end{equation}

\medskip

\textbf{4. Estimating the contribution from the semicircle.}

To pass from the finite integral over $[-R,R]$ to the improper integral $I$, we must understand the behavior of the integral over the semicircular arc $\gamma_R$ as $R\to\infty$.  We will show that this contribution tends to zero, so that in the limit only the integral along the real axis remains.

On $\gamma_R$ we have $z = Re^{i\theta}$ with $\theta\in[0,\pi]$, so $|z|=R$.  Then
\[
|z^{2}+1|
= |R^{2}e^{2i\theta} + 1|
\ge \bigl||R^{2}| - |1|\bigr|
= R^{2}-1,
\]
by the reverse triangle inequality.  For $R>1$ this gives
\[
|f(z)| = \left|\frac{1}{z^{2}+1}\right|
\le \frac{1}{R^{2}-1}, \quad z\in\gamma_R.
\]
The length of the semicircular arc $\gamma_R$ is $\pi R$.  The Estimation Lemma (or ML-inequality) tells us that
\[
\left|\int_{\gamma_R} f(z)\,dz\right|
\le \max_{z\in \gamma_R} |f(z)| \cdot \text{length}(\gamma_R)
\le \frac{1}{R^{2}-1} \cdot \pi R
= \frac{\pi R}{R^{2}-1}.
\]
As $R\to\infty$, the right-hand side tends to zero, because the numerator grows linearly while the denominator grows quadratically.  Therefore
\[
\lim_{R\to\infty} \int_{\gamma_R} f(z)\,dz = 0.
\]

\medskip

\textbf{5. Passing to the limit and evaluating $I$.}

We now let $R\to\infty$ in the identity \eqref{eq:CR-decomp}.  First, since the integrand $1/(x^{2}+1)$ is absolutely integrable over $\mathbb{R}$, we have
\[
\lim_{R\to\infty} \int_{-R}^{R} \frac{1}{x^{2}+1}\,dx
= \int_{-\infty}^{\infty} \frac{1}{x^{2}+1}\,dx
= I.
\]
Second, we have just shown that
\[
\lim_{R\to\infty} \int_{\gamma_R} f(z)\,dz = 0.
\]
Taking limits in \eqref{eq:CR-decomp} and using these two facts yields
\[
I + 0 = \pi.
\]
Hence
\[
\int_{-\infty}^{\infty} \frac{1}{x^{2}+1}\,dx = \pi.
\]

This agrees with the elementary calculus evaluation, since the antiderivative of $1/(1+x^{2})$ is $\arctan x$, and
\[
\int_{-\infty}^{\infty} \frac{1}{x^{2}+1}\,dx
= \left[\arctan x\right]_{-\infty}^{\infty}
= \frac{\pi}{2} - \left(-\frac{\pi}{2}\right)
= \pi.
\]

\medskip

\textbf{6. Conceptual remarks and connection to residue calculus.}

This example illustrates the central pattern of residue calculus for real integrals:

\begin{itemize}
  \item We embed the real integral into the complex plane by extending the integrand to a meromorphic function $f(z)$.
  \item We choose a contour that includes the segment of the real axis of interest and closes up in a half-plane where the integrand behaves well (here, the upper half-plane without any exponential factors).
  \item The residue theorem converts the contour integral into a finite sum of residues at the poles inside the contour.
  \item We estimate the integral over the ``large'' part of the contour (the semicircular arc) and show that it vanishes in the limit, so that the original real integral is equal to the sum of residues multiplied by $2\pi i$.
\end{itemize}

For more complicated integrals, for example
\[
\int_{-\infty}^{\infty} \frac{e^{ix}}{x^{2}+1}\,dx,
\]
we would take $f(z) = e^{iz}/(z^{2}+1)$ and choose the contour in a half-plane where $e^{iz}$ decays exponentially along the large arc (which depends on the sign in the exponential).  The same residue-calculus strategy then applies, even though no simple elementary antiderivative is available.  Thus this simple rational example serves as a template for a broad class of integrals arising in Fourier analysis and probability theory.
\end{solution}

% ===== Example 3: Inverse Laplace Transform by Residues (inquiry-based) =====
\begin{problem}[Inverse Laplace Transform by Residues]
A mass--spring--damper system with unit mass, damping coefficient $4$, and spring constant $5$ is modeled by the equation
\[
x''(t) + 4x'(t) + 5x(t) = 0,
\qquad x(0) = 1,\quad x'(0) = 0.
\]
Its solution can be obtained using Laplace transforms, and the resulting Laplace transform $X(s)$ is a rational function with complex poles. In this problem you will recover $x(t)$ not by partial fractions, but by evaluating the \emph{inverse Laplace transform} as a complex contour integral and applying the residue theorem. This will reveal how decay and oscillation in time arise from the locations of poles in the complex $s$-plane.

(a) Use the Laplace transform of derivatives to compute the Laplace transform $X(s)$ of the solution $x(t)$ of the initial value problem above. Show that
\[
X(s) = \mathcal{L}\{x\}(s) = \frac{s+4}{s^{2} + 4s + 5}.
\]
What is the region of convergence in the complex $s$-plane, in terms of $\Re s$?

(b) The inverse Laplace transform can be written as the \emph{Bromwich integral}
\[
x(t) = \frac{1}{2\pi i}\int_{\sigma - i\infty}^{\sigma + i\infty} e^{st} X(s)\,ds,
\]
where $\sigma$ is chosen so that the vertical line $\Re s = \sigma$ lies in the region of convergence of $X(s)$. 

\begin{itemize}
\item[(i)] Factor the quadratic denominator $s^{2} + 4s + 5$ and locate the poles of $X(s)$ in the complex plane.
\item[(ii)] For a fixed $t>0$, consider closing the vertical Bromwich contour with a large rectangle to the \emph{left} in the complex plane. Sketch this contour, indicating its orientation and the locations of the poles relative to it.
\item[(iii)] Explain qualitatively why, for $t>0$, it is advantageous to close the contour in the left half-plane rather than the right half-plane. 
\end{itemize}
Hint: Think about the factor $e^{st}$ when $\Re s$ is very large and negative, versus when $\Re s$ is very large and positive.

(c) Let
\[
f(s) = e^{st} X(s) = e^{st}\,\frac{s+4}{s^{2} + 4s + 5}.
\]
Using your contour from part (b), argue (without full technical details) that the contributions from the three ``extra'' sides of the large rectangle tend to zero as its size tends to infinity, provided $t>0$. Conclude that
\[
\int_{\sigma - i\infty}^{\sigma + i\infty} e^{st} X(s)\,ds
= 2\pi i \sum \operatorname{Res}\bigl(f; s_k\bigr),
\]
where the sum is over the poles $s_k$ of $X(s)$.
Hint: Use that $e^{st}$ decays rapidly when $\Re s$ is large and negative and $t>0$. This is a special case of Jordan's lemma.

(d) Compute the residues of $X(s)$ at its (simple) poles $s_1$ and $s_2$. Then show that
\[
\operatorname{Res}\bigl(f; s_k\bigr) 
= e^{s_k t}\,\operatorname{Res}\bigl(X; s_k\bigr),
\]
and use part (c) to obtain an explicit expression for $x(t)$ as a linear combination of terms $e^{s_k t}$. Finally, rewrite your expression in terms of real-valued functions (sines and cosines) and simplify it as far as possible.
Hint: For a simple pole $s_k$ of $X(s) = N(s)/D(s)$, you may use 
\[
\operatorname{Res}(X; s_k) = \frac{N(s_k)}{D'(s_k)}.
\]

(e) Interpretation and extensions.
\begin{itemize}
\item[(i)] Using your final expression for $x(t)$, identify its exponential decay rate and its oscillation frequency. How are these related to the real and imaginary parts of the poles of $X(s)$?
\item[(ii)] Suppose instead that the differential equation were
\[
x''(t) - 4x'(t) + 5x(t) = 0 \quad \text{with the same initial conditions.}
\]
Without doing any detailed calculations, predict the qualitative behavior of the solution $x(t)$ for large $t$ (decay, growth, or bounded oscillation) based on the locations of the poles of the corresponding Laplace transform. How would the contour argument in parts (b)--(d) change, if at all?
\end{itemize}
\end{problem}

% ===== Example 3: Inverse Laplace Transform by Residues (full solution) =====
\begin{problem}[Inverse Laplace Transform by Residues]
Consider the initial value problem
\[
x''(t) + 4x'(t) + 5x(t) = 0,\qquad x(0) = 1,\quad x'(0) = 0.
\]
\begin{enumerate}
\item Compute the Laplace transform $X(s) = \mathcal{L}\{x\}(s)$ and show that
\[
X(s) = \frac{s+4}{s^{2} + 4s + 5}.
\]
\item For $t>0$, evaluate the inverse Laplace transform
\[
x(t) = \frac{1}{2\pi i}\int_{\sigma - i\infty}^{\sigma + i\infty} e^{st} X(s)\,ds
\]
by closing the contour in the left half-plane and applying the residue theorem. Express $x(t)$ in real form and describe briefly how the decay rate and oscillation frequency of $x(t)$ are encoded in the poles of $X(s)$.
\end{enumerate}
\end{problem}

\begin{solution}
We first compute the Laplace transform of the solution of the given initial value problem, and then invert it using the residue theorem applied to the Bromwich integral.

\medskip
\noindent\textbf{1. Computing the Laplace transform.}
Let $X(s) = \mathcal{L}\{x\}(s)$. Using the standard formulas
\[
\mathcal{L}\{x'(t)\}(s) = sX(s) - x(0), \qquad
\mathcal{L}\{x''(t)\}(s) = s^{2}X(s) - sx(0) - x'(0),
\]
and the initial conditions $x(0) = 1$, $x'(0) = 0$, we take Laplace transforms of both sides of
\[
x''(t) + 4x'(t) + 5x(t) = 0
\]
to obtain
\[
\bigl(s^{2}X(s) - s\bigr) + 4\bigl(sX(s) - 1\bigr) + 5X(s) = 0.
\]
Collecting terms in $X(s)$ gives
\[
\bigl(s^{2} + 4s + 5\bigr)X(s) - (s + 4) = 0,
\]
so
\[
X(s) = \frac{s+4}{s^{2} + 4s + 5}.
\]
The denominator is a quadratic with roots
\[
s^{2} + 4s + 5 = 0
\quad\Longrightarrow\quad
s = -2 \pm i,
\]
so the poles lie at $-2 \pm i$, and the real part of both poles is $-2$. Thus the Laplace transform converges for $\Re s > -2$, and any $\sigma > -2$ is an admissible choice in the Bromwich integral.

\medskip
\noindent\textbf{2. Setting up the Bromwich integral and contour.}
By definition of the inverse Laplace transform, for $t>0$ we have
\[
x(t) = \frac{1}{2\pi i}\int_{\sigma - i\infty}^{\sigma + i\infty} e^{st} X(s)\,ds,
\]
where $\sigma > -2$ is fixed. Define
\[
f(s) = e^{st} X(s) = e^{st}\,\frac{s+4}{s^{2} + 4s + 5}.
\]
The integrand $f$ is meromorphic with simple poles at $s_1 = -2 + i$ and $s_2 = -2 - i$.

To apply the residue theorem, we embed the vertical line $\Re s = \sigma$ in a large rectangle. Let $R>\sigma$ and consider the rectangle with vertices
\[
\sigma - iR,\quad \sigma + iR,\quad -R + iR,\quad -R - iR,
\]
traversed counterclockwise. Denote this closed contour by $C_R$. Its right-hand side is the original vertical Bromwich segment from $\sigma - iR$ to $\sigma + iR$.

The poles $s_1$ and $s_2$ both lie in the left half-plane, and for $R$ large enough the entire segment $\Re s = -R$ lies to the left of the poles. Therefore, both poles lie inside $C_R$.

\medskip
\noindent\textbf{3. Vanishing of the other sides and the residue theorem.}
We now argue that, for fixed $t>0$, the integrals over the top, bottom, and left sides of $C_R$ tend to zero as $R\to\infty$. The essential point is that when $\Re s$ is very negative, the factor $e^{st}$ decays exponentially in $|s|$.

More precisely, on the left side of the rectangle we have $\Re s = -R$, so
\[
|e^{st}| = e^{(\Re s)\,t} = e^{-Rt},
\]
which tends to zero rapidly as $R\to\infty$. Since $X(s)$ behaves like $1/s$ as $|s|\to\infty$, the integrand $f(s)$ on that side is of order $e^{-Rt}/|s|$, and hence the integral over the left side tends to zero.

On the top and bottom segments, $\Im s = \pm R$ while $\Re s$ is bounded between $-R$ and $\sigma$. Here, $X(s)$ again decays like $1/s$, and the modulus of $e^{st}$ is $e^{(\Re s)t}$, which is at most $e^{\sigma t}$ on those segments. The length of each segment is $O(R)$, and the integrand is $O(1/R)$, so the integrals over the top and bottom are uniformly bounded and in fact tend to zero as $R\to\infty$ by standard Jordan-type estimates.

Thus, letting $R\to\infty$, the only surviving part of the contour integral $\int_{C_R} f(s)\,ds$ is the integral along the Bromwich line:
\[
\int_{C_R} f(s)\,ds
\;\xrightarrow{R\to\infty}\;
\int_{\sigma - i\infty}^{\sigma + i\infty} e^{st} X(s)\,ds.
\]
By the residue theorem, for each fixed $R$ we have
\[
\int_{C_R} f(s)\,ds = 2\pi i\bigl( \operatorname{Res}(f; s_1) + \operatorname{Res}(f; s_2)\bigr),
\]
since $C_R$ is counterclockwise. Passing to the limit $R\to\infty$ gives
\[
\int_{\sigma - i\infty}^{\sigma + i\infty} e^{st} X(s)\,ds
= 2\pi i\bigl( \operatorname{Res}(f; s_1) + \operatorname{Res}(f; s_2)\bigr).
\]
Multiplying by $1/(2\pi i)$, we obtain
\[
x(t) = \sum_{k=1}^{2} \operatorname{Res}(f; s_k)
\quad\text{for } t>0.
\]

\medskip
\noindent\textbf{4. Computing the residues and simplifying.}
Because $f(s) = e^{st}X(s)$ and $e^{st}$ is analytic, for a simple pole $s_k$ of $X(s)$ we have
\[
\operatorname{Res}\bigl(f; s_k\bigr)
= e^{s_k t}\,\operatorname{Res}\bigl(X; s_k\bigr).
\]
Thus we need only compute the residues of $X$ at $s_1$ and $s_2$.

Write
\[
X(s) = \frac{s+4}{s^{2} + 4s + 5} = \frac{s+4}{(s-s_1)(s-s_2)}, 
\quad s_1 = -2 + i,\ s_2 = -2 - i.
\]
The poles are simple. For a simple pole $s_k$ of $N(s)/D(s)$, we can use
\[
\operatorname{Res}\bigl(X; s_k\bigr) = \frac{N(s_k)}{D'(s_k)}.
\]
Here $N(s) = s+4$ and $D(s) = s^{2} + 4s + 5$, so $D'(s) = 2s + 4$. Thus
\[
\operatorname{Res}\bigl(X; s_k\bigr)
= \frac{s_k + 4}{2s_k + 4}.
\]

For $s_1 = -2 + i$:
\[
s_1 + 4 = 2 + i, \quad 2s_1 + 4 = 2(-2 + i) + 4 = -4 + 2i + 4 = 2i,
\]
so
\[
\operatorname{Res}(X; s_1) = \frac{2 + i}{2i}
= \frac{2}{2i} + \frac{i}{2i}
= \frac{1}{i} + \frac{1}{2}
= -i + \frac{1}{2}.
\]

For $s_2 = -2 - i$:
\[
s_2 + 4 = 2 - i, \quad 2s_2 + 4 = 2(-2 - i) + 4 = -4 - 2i + 4 = -2i,
\]
so
\[
\operatorname{Res}(X; s_2) = \frac{2 - i}{-2i}
= -\,\frac{2 - i}{2i}
= -\left(\frac{2}{2i} - \frac{i}{2i}\right)
= -\left(\frac{1}{i} - \frac{1}{2}\right)
= i + \frac{1}{2}.
\]

Therefore
\[
\operatorname{Res}\bigl(f; s_1\bigr) 
= e^{s_1 t}\left(-i + \frac{1}{2}\right),
\qquad
\operatorname{Res}\bigl(f; s_2\bigr) 
= e^{s_2 t}\left(i + \frac{1}{2}\right).
\]
Summing and recalling that $x(t)$ is their sum, we obtain
\[
x(t) = e^{(-2 + i)t}\left(-i + \frac{1}{2}\right)
       + e^{(-2 - i)t}\left(i + \frac{1}{2}\right).
\]

To simplify this expression, factor out the common exponential decay $e^{-2t}$:
\[
x(t) = e^{-2t}\left[
e^{it}\left(-i + \frac{1}{2}\right) + e^{-it}\left(i + \frac{1}{2}\right)
\right].
\]
Now use $e^{it} = \cos t + i\sin t$ and $e^{-it} = \cos t - i\sin t$. Denote $A = -i + \tfrac{1}{2} = \tfrac{1}{2} - i$ and $B = i + \tfrac{1}{2} = \tfrac{1}{2} + i$, which are complex conjugates. Then
\[
e^{it}A + e^{-it}B = 2\,\Re\bigl(e^{it}A\bigr).
\]
Compute $e^{it}A$:
\[
e^{it}A = (\cos t + i\sin t)\left(\frac{1}{2} - i\right)
= \left(\frac{1}{2}\cos t + \sin t\right) 
   + i\left(-\cos t + \frac{1}{2} \sin t\right).
\]
The real part is $\frac{1}{2}\cos t + \sin t$, so
\[
e^{it}A + e^{-it}B = 2\left(\frac{1}{2}\cos t + \sin t\right) 
= \cos t + 2\sin t.
\]
Thus the real-valued solution is
\[
x(t) = e^{-2t}\bigl(\cos t + 2\sin t\bigr), \qquad t>0.
\]

This matches the solution one would obtain directly by solving the characteristic equation $\lambda^{2} + 4\lambda + 5 = 0$, which yields
\[
x(t) = e^{-2t}\bigl(A\cos t + B\sin t\bigr),
\]
and determining $A=1$, $B=2$ from the initial conditions.

\medskip
\noindent\textbf{5. Interpretation in terms of poles and residues.}
The poles of $X(s)$ are at $s = -2 \pm i$. Their real part, $-2$, determines the exponential decay factor $e^{-2t}$: because $\Re s < 0$, the solution decays to zero as $t\to\infty$, corresponding to a stable damped oscillator. The imaginary part, $\pm 1$, determines the oscillation frequency, which appears in the $\cos t$ and $\sin t$ factors.

From the point of view of residue calculus, each pole $s_k$ of $X(s)$ contributes a term of the form
\[
e^{s_k t}\,\operatorname{Res}(X; s_k)
\]
to the inverse Laplace transform. The Bromwich integral, originally an improper integral along a vertical line in the complex plane, is converted into a sum of these contributions by closing the contour in the half-plane where $e^{st}$ decays and then applying the residue theorem. This example illustrates the central idea of residue calculus in the context of inverse Laplace transforms: qualitative features of solutions to linear differential equations (stability, decay rate, and oscillation frequency) are encoded directly in the locations of poles in the complex plane, and residues at those poles determine the coefficients of the corresponding exponential and oscillatory modes.
\end{solution}

% ===== Example 4: Summing Series with Contour Integrals (inquiry-based) =====
\begin{problem}[Summing Series with Contour Integrals]
One of the remarkable uses of residue calculus is to evaluate infinite series that at first sight have no obvious closed form. In this problem, you will discover how to compute
\[
\sum_{n=-\infty}^{\infty}\frac{1}{n^{2}+a^{2}}
\]
for a positive real parameter $a$, by turning the discrete sum into a contour integral. The key idea is to build a meromorphic function whose residues at the integers encode the terms of the series, and then to evaluate a contour integral in two different ways.

Throughout, assume that $a>0$ is a fixed real number.

\smallskip

(a) We would like a meromorphic function on $\mathbb{C}$ that has simple poles at each integer $n\in\mathbb{Z}$ and whose residues at these poles are easy to recognize. One standard tool is the cotangent function.

\quad (i) Show that the function $\pi\cot(\pi z)$ is meromorphic on $\mathbb{C}$ with simple poles at each integer $n\in\mathbb{Z}$ and no other singularities.

\quad (ii) Show that the residue of $\pi\cot(\pi z)$ at $z=n$ is $1$ for every $n\in\mathbb{Z}$.

Hint: You may recall or prove that near $z=0$,
\[
\cot(\pi z)=\frac{1}{\pi z}+O(z),
\]
and use periodicity to shift this expansion to other integers.

\smallskip

(b) To make the residues at the integers equal to the terms of our series, we need to introduce the factor $(z^{2}+a^{2})^{-1}$.

\quad (i) Define
\[
f(z)=\frac{\pi\cot(\pi z)}{z^{2}+a^{2}}.
\]
Describe the poles of $f(z)$ in the complex plane and determine their orders.

\quad (ii) Compute the residue of $f(z)$ at a generic integer $z=n\in\mathbb{Z}$.

Hint: At $z=n$, the factor $(z^{2}+a^{2})^{-1}$ is analytic and nonzero, while $\pi\cot(\pi z)$ has a simple pole.

\smallskip

(c) Next, let $R>0$ and consider the rectangle with vertices at $\pm\left(R+\tfrac12\right)\pm iT$, where $T>0$ is fixed. Let $C_{R}$ denote the positively oriented boundary of this rectangle.

\quad (i) Use the residue theorem to express the integral
\[
\int_{C_{R}} f(z)\,dz
\]
as $2\pi i$ times the sum of the residues of $f$ at all poles inside $C_{R}$. Write this sum as the sum of:
\begin{itemize}
\item residues at the integer points $n$ lying inside the rectangle, and
\item residues at the non-integer poles of $f$.
\end{itemize}

\quad (ii) What are the non-integer poles of $f$? Compute their residues explicitly.

Hint: These come from the zeros of $z^{2}+a^{2}$.

\smallskip

(d) To relate the contour integral to the infinite sum, we want to let $R\to\infty$ and show that the integral over $C_{R}$ tends to zero.

\quad (i) Show that for $z=x+iy$ with $|x|\geq \tfrac12$,
\[
\left|\cot(\pi z)\right|\leq C\,e^{-\pi |y|}
\]
for some constant $C$ independent of $x,y$. You may quote (without proof) a standard estimate for $\cot(\pi z)$, or argue using the definition in terms of exponentials.

\quad (ii) Use this estimate, together with the behavior of $1/(z^{2}+a^{2})$ as $|z|\to\infty$, to show that $\int_{C_{R}} f(z)\,dz\to 0$ as $R\to\infty$ (with fixed $T$).

Hint: Estimate the integrand on each of the four sides of the rectangle and argue that the sum of these estimates tends to zero.

\smallskip

(e) Now combine your work.

\quad (i) Letting $R\to\infty$ in your expression from part (c), show that
\[
\sum_{n=-\infty}^{\infty}\frac{1}{n^{2}+a^{2}} = \frac{\pi}{a}\coth(\pi a).
\]

Hint: As $R\to\infty$, the sum of residues at integer points inside $C_{R}$ becomes the full sum over $n\in\mathbb{Z}$.

\quad (ii) Deduce a formula for the one-sided sum $\displaystyle\sum_{n=1}^{\infty}\frac{1}{n^{2}+a^{2}}$.

\smallskip

(f) (Extensions and variations.)

\quad (i) How would the computation change if, instead of $\pi\cot(\pi z)$, you used $\pi\csc(\pi z)$ so that the poles occur at the integers but with alternating residues?

\quad (ii) Suppose you want to evaluate
\[
\sum_{n=-\infty}^{\infty}\frac{1}{(n^{2}+a^{2})^{2}}.
\]
Suggest a modification of the function $f(z)$ that might lead to this series, and briefly outline how the residue calculations would change.

Hint: Consider differentiating with respect to $a$ or altering the power of $(z^{2}+a^{2})$ in the denominator.
\end{problem}

% ===== Example 4: Summing Series with Contour Integrals (full solution) =====
\begin{problem}[Summing Series with Contour Integrals]
Let $a>0$ be real. Consider the meromorphic function
\[
f(z)=\frac{\pi\cot(\pi z)}{z^{2}+a^{2}}.
\]
\begin{enumerate}
\item Show that $f$ has simple poles at each integer $n\in\mathbb{Z}$ and at $z=\pm ia$, and compute the residues at these poles.
\item For $R>0$, let $C_{R}$ be the positively oriented boundary of the rectangle with vertices at $\pm\left(R+\tfrac12\right)\pm iT$, where $T>0$ is fixed. Use the residue theorem to express $\displaystyle\int_{C_{R}} f(z)\,dz$ as $2\pi i$ times the sum of residues of $f$ inside $C_{R}$.
\item Show that $\displaystyle\int_{C_{R}} f(z)\,dz\to 0$ as $R\to\infty$ (with $T$ fixed), and deduce that
\[
\sum_{n=-\infty}^{\infty}\frac{1}{n^{2}+a^{2}} = \frac{\pi}{a}\coth(\pi a).
\]
\item Conclude a formula for $\displaystyle\sum_{n=1}^{\infty}\frac{1}{n^{2}+a^{2}}$.
\end{enumerate}
\end{problem}

\begin{solution}
We are asked to evaluate the series $\sum_{n=-\infty}^{\infty} 1/(n^{2}+a^{2})$ by converting it into a contour integral and applying the residue theorem. The central idea is to construct a meromorphic function whose residues at the integers reproduce the terms of the series, then relate the sum of these residues to an integral over a large contour that tends to zero.

\medskip

\noindent\textbf{1. Poles and residues of $f(z)$.}

Consider
\[
f(z)=\frac{\pi\cot(\pi z)}{z^{2}+a^{2}}.
\]
We recall two basic facts about $\pi\cot(\pi z)$:

\begin{itemize}
\item It is meromorphic on $\mathbb{C}$ with simple poles at every integer $n\in\mathbb{Z}$ and no other singularities.
\item At each integer $n$, the residue of $\pi\cot(\pi z)$ is $1$. This follows from the local expansion near $z=n$:
\[
\pi\cot(\pi z) = \frac{1}{z-n} + O(z-n).
\]
\end{itemize}

The denominator $z^{2}+a^{2}$ vanishes at $z=\pm ia$, with nonzero derivative there, so $z=\pm ia$ are simple zeros of $z^{2}+a^{2}$ and hence simple poles of $1/(z^{2}+a^{2})$. Thus $f$ has possible poles at all integers and at $z=\pm ia$.

At each integer $n$, the factor $(z^{2}+a^{2})^{-1}$ is analytic and nonzero. Therefore $f$ has a simple pole at $z=n$ coming from $\pi\cot(\pi z)$. The residue is
\[
\operatorname{Res}(f;n) = \left(\frac{\pi\cot(\pi z)-\frac{1}{z-n}}{z^{2}+a^{2}}\right)\Bigg|_{z=n} + \frac{1}{z^{2}+a^{2}}\Bigg|_{z=n}
= \frac{1}{n^{2}+a^{2}}.
\]
A more conceptual way to state this is:
\[
\operatorname{Res}(f;n)=\big(\operatorname{Res}(\pi\cot(\pi z);n)\big)\cdot \frac{1}{n^{2}+a^{2}}=1\cdot\frac{1}{n^{2}+a^{2}}.
\]

At $z=ia$, the factor $\pi\cot(\pi z)$ is analytic, while $1/(z^{2}+a^{2})$ has a simple pole. For a simple pole of a quotient $g(z)/h(z)$ where $h(z_{0})=0$ and $h'(z_{0})\neq 0$, the residue is $g(z_{0})/h'(z_{0})$. Here $g(z)=\pi\cot(\pi z)$ and $h(z)=z^{2}+a^{2}$, so $h'(z)=2z$. Thus
\[
\operatorname{Res}(f;ia)=\frac{\pi\cot(\pi ia)}{2ia}.
\]
Similarly, at $z=-ia$,
\[
\operatorname{Res}(f;-ia)=\frac{\pi\cot(\pi(-ia))}{2(-ia)}.
\]

We can simplify these expressions using the identity
\[
\cot(iw)=-i\,\coth(w).
\]
Indeed,
\[
\cot(\pi ia)=\cot(i\pi a)=-i\,\coth(\pi a),
\]
and
\[
\cot(-\pi ia)=\cot(-i\pi a)=-\cot(i\pi a)=i\,\coth(\pi a).
\]
Therefore,
\[
\operatorname{Res}(f;ia)=\frac{\pi(-i\,\coth(\pi a))}{2ia}
= -\frac{\pi i\,\coth(\pi a)}{2ia}
= -\frac{\pi}{2a}\coth(\pi a),
\]
and
\[
\operatorname{Res}(f;-ia)=\frac{\pi(i\,\coth(\pi a))}{2(-ia)}
= -\frac{\pi i\,\coth(\pi a)}{2ia}
= -\frac{\pi}{2a}\coth(\pi a).
\]
Thus both non-integer poles contribute the same residue.

\medskip

\noindent\textbf{2. Applying the residue theorem on a large rectangle.}

Fix $T>0$. For $R>0$ let $C_{R}$ denote the boundary of the rectangle with vertices at $\pm\left(R+\tfrac12\right)\pm iT$, oriented positively (counterclockwise). The poles of $f$ inside $C_{R}$ are:

\begin{itemize}
\item All integers $n$ with $-R\le n\le R$ (since the vertical sides are at $x=\pm(R+\tfrac12)$), and
\item The two points $z=ia$ and $z=-ia$, provided that $|a|<T$ so that these points lie inside the horizontal strip $\{-T<\Im z<T\}$. We may and do choose $T>a$ to ensure this.
\end{itemize}

By the residue theorem,
\[
\int_{C_{R}} f(z)\,dz
= 2\pi i\left(\sum_{n=-R}^{R}\operatorname{Res}(f;n)
+ \operatorname{Res}(f;ia) + \operatorname{Res}(f;-ia)\right).
\]
Using the computations above,
\[
\int_{C_{R}} f(z)\,dz
= 2\pi i\left(\sum_{n=-R}^{R}\frac{1}{n^{2}+a^{2}}
- \frac{\pi}{2a}\coth(\pi a)
- \frac{\pi}{2a}\coth(\pi a)\right),
\]
which simplifies to
\[
\int_{C_{R}} f(z)\,dz
= 2\pi i\left(\sum_{n=-R}^{R}\frac{1}{n^{2}+a^{2}}
- \frac{\pi}{a}\coth(\pi a)\right).
\]
Thus
\begin{equation}\label{eq:integral-identity}
\sum_{n=-R}^{R}\frac{1}{n^{2}+a^{2}}
= \frac{\pi}{a}\coth(\pi a) + \frac{1}{2\pi i}\int_{C_{R}} f(z)\,dz.
\end{equation}

\medskip

\noindent\textbf{3. Showing the contour integral tends to zero.}

We now show that the integral over $C_{R}$ tends to zero as $R\to\infty$ (with $T$ fixed and larger than $a$). This is the analytic step that allows us to pass from a finite sum to an infinite series.

We first recall a standard estimate for $\cot(\pi z)$. Writing $z=x+iy$, one can use the representation
\[
\cot(\pi z) = i\,\frac{e^{2\pi i z}+1}{e^{2\pi i z}-1}
\]
and separate real and imaginary parts to see that, for $|x|\le R+1$ and $|y|\ge T>0$ fixed, $\cot(\pi z)$ is bounded uniformly in $R$ and in fact decays exponentially as $|y|\to\infty$. More precisely, for fixed $T>0$ there exists a constant $C=C(T)$ such that
\[
\bigl|\cot(\pi z)\bigr| \le C e^{-\pi |y|}
\]
for all $z$ with $|\Im z|=|y|\ge T$.

On the rectangle $C_{R}$, both horizontal sides are at heights $y=\pm T$, so on these we have $|\cot(\pi z)|\le C e^{-\pi T}$. The denominator satisfies
\[
|z^{2}+a^{2}| \ge c(1+|z|^{2})
\]
for some constant $c>0$ and all large $|z|$, so in particular on $C_{R}$ we have $|1/(z^{2}+a^{2})|\le \frac{C'}{R^{2}}$ for some constant $C'$ independent of $R$ (because the distance from $z$ to the origin along the vertical sides grows like $R$). Thus on each side of the rectangle,
\[
|f(z)| = \left|\frac{\pi\cot(\pi z)}{z^{2}+a^{2}}\right|
\le \frac{C''}{1+|z|^{2}}
\]
for some constant $C''$ independent of $R$.

Now estimate the integral over each side. Each side has length $O(R)$, while $|f(z)|$ along the side is $O(1/R^{2})$. Therefore the contribution from each side is $O(R\cdot R^{-2})=O(1/R)$, which tends to zero as $R\to\infty$. Summing over the four sides, we obtain
\[
\lim_{R\to\infty} \int_{C_{R}} f(z)\,dz = 0.
\]

\medskip

\noindent\textbf{4. Passing to the limit and evaluating the series.}

Taking the limit as $R\to\infty$ in \eqref{eq:integral-identity}, dominated convergence on the discrete sum yields
\[
\sum_{n=-\infty}^{\infty}\frac{1}{n^{2}+a^{2}}
= \lim_{R\to\infty}\sum_{n=-R}^{R}\frac{1}{n^{2}+a^{2}}
= \frac{\pi}{a}\coth(\pi a) + \lim_{R\to\infty}\frac{1}{2\pi i}\int_{C_{R}} f(z)\,dz.
\]
Since the integral tends to zero, we conclude that
\[
\sum_{n=-\infty}^{\infty}\frac{1}{n^{2}+a^{2}}
= \frac{\pi}{a}\coth(\pi a).
\]

This is the desired closed-form expression for the two-sided series. It is a typical example of how residue calculus transforms an infinite sum into a contour integral, and then back into an explicit formula by exploiting the analytic structure of the integrand and estimates at infinity.

\medskip

\noindent\textbf{5. One-sided sum.}

To obtain the sum over positive integers only, we note that the series is even in $n$:
\[
\frac{1}{n^{2}+a^{2}} = \frac{1}{(-n)^{2}+a^{2}}.
\]
Therefore
\[
\sum_{n=-\infty}^{\infty}\frac{1}{n^{2}+a^{2}}
= \frac{1}{0^{2}+a^{2}} + 2\sum_{n=1}^{\infty}\frac{1}{n^{2}+a^{2}}
= \frac{1}{a^{2}} + 2\sum_{n=1}^{\infty}\frac{1}{n^{2}+a^{2}}.
\]
Substituting the formula just obtained for the two-sided sum, we find
\[
\frac{\pi}{a}\coth(\pi a)
= \frac{1}{a^{2}} + 2\sum_{n=1}^{\infty}\frac{1}{n^{2}+a^{2}}.
\]
Solving for the one-sided sum,
\[
\sum_{n=1}^{\infty}\frac{1}{n^{2}+a^{2}}
= \frac{1}{2}\left(\frac{\pi}{a}\coth(\pi a) - \frac{1}{a^{2}}\right).
\]

\medskip

\noindent\textbf{Conceptual remarks.}

This example illustrates a standard pattern in residue calculus for evaluating series:

\begin{itemize}
\item The factor $\pi\cot(\pi z)$ (or related periodic functions) provides a meromorphic function with simple poles at the integers and known residues.
\item Multiplying by a rational function such as $1/(z^{2}+a^{2})$ encodes the terms of the series in the residues at the integers.
\item Integrating over a large rectangle exploits the periodicity and decay properties of the integrand to show that the contour integral vanishes in the limit.
\item The residue theorem then relates the infinite sum of residues at the integers to a finite sum of residues at non-integer poles, yielding a closed form for the series.
\end{itemize}

This method showcases the power of complex analysis: discrete sums, which can be difficult to attack by purely real-variable methods, become accessible once we interpret them as sums of residues of a carefully chosen meromorphic function.
\end{solution}

% ===== Example 5: Residues and Green’s Functions for a Simple PDE (inquiry-based) =====
\begin{problem}[Residues and Green’s Functions for a Simple PDE]
In many diffusion or steady-state heat-flow problems on a long rod, one encounters differential operators of the form
\[
\mathcal{L}u(x) = -\frac{d^2 u}{dx^2} + a^2 u(x),
\]
where the parameter $a>0$ models a loss of heat to the surrounding environment. When the rod is effectively infinite in extent, it is natural to impose decay conditions at spatial infinity. In this setting, it is convenient to introduce a Green’s function $G$ so that solutions for arbitrary source terms can be expressed as a convolution. Our goal in this problem is to construct $G$ using Fourier transforms and then to use residue calculus in the complex plane to evaluate the resulting integral and understand its qualitative behavior.

Consider the Green’s function $G:\mathbb{R}\to\mathbb{R}$ for the operator
\[
\mathcal{L} = -\frac{d^2}{dx^2} + a^2, \qquad a>0,
\]
defined by
\[
\mathcal{L}G(x) = \delta(x), \qquad G(x)\to 0 \ \text{as } |x|\to\infty,
\]
where $\delta$ is the Dirac delta distribution.

Throughout, use the following convention for the Fourier transform of a (sufficiently nice) function $f$:
\[
\widehat{f}(k) = \int_{-\infty}^{\infty} f(x)\, e^{-ikx}\,dx,
\qquad
f(x) = \frac{1}{2\pi}\int_{-\infty}^{\infty} \widehat{f}(k)\, e^{ikx}\,dk.
\]

\medskip

(a) Using the definition of Green’s function, write the differential equation that $G$ must satisfy on $\mathbb{R}\setminus\{0\}$ and describe in words the “jump condition’’ that occurs at $x=0$ because of the delta function.  
Hint: Integrate the equation $\mathcal{L}G = \delta$ across a small interval $[-\varepsilon,\varepsilon]$ and let $\varepsilon\to 0^+$.

\medskip

(b) Apply the Fourier transform to the equation
\[
-\frac{d^2G}{dx^2} + a^2 G = \delta
\]
to obtain an algebraic equation for $\widehat{G}(k)$. Solve this equation for $\widehat{G}(k)$, and then write down the inverse Fourier transform formula for $G(x)$ as an integral over $k\in\mathbb{R}$.  
Hint: Recall that the Fourier transform of $\delta$ is $1$, and that differentiation in $x$ corresponds to multiplication by $ik$ in the Fourier domain.

\medskip

(c) Now view the inverse transform integral for $G(x)$,
\[
G(x) = \frac{1}{2\pi}\int_{-\infty}^{\infty} \frac{e^{ikx}}{k^2 + a^2}\,dk,
\]
as an integral of a complex-valued function of the complex variable $k$.  

\begin{itemize}
  \item[(i)] Identify the singularities (poles) of the integrand in the complex $k$-plane and determine whether they are simple or higher order.  
  \item[(ii)] Suppose $x>0$. Argue which half-plane (upper or lower) is appropriate for closing the contour when applying the residue theorem to the integral above.  
  \item[(iii)] For $x>0$, write down the closed contour integral over the real axis and a large semicircle, and state (without full proof) why the contribution from the semicircle tends to zero as its radius tends to infinity.  
\end{itemize}
Hint: Examine the factor $e^{ikx}$ when $\operatorname{Im}(k)\to\pm\infty$ and recall Jordan’s lemma or similar decay arguments.

\medskip

(d) For $x>0$, compute the residue of the integrand
\[
f(k) = \frac{e^{ikx}}{k^2 + a^2}
\]
at the relevant pole inside your contour, and then use the residue theorem to evaluate the integral
\[
\int_{-\infty}^{\infty} \frac{e^{ikx}}{k^2 + a^2}\,dk.
\]
Repeat the argument for $x<0$ by choosing the opposite half-plane for the contour. Combine your results to obtain an explicit formula for $G(x)$ valid for all $x\in\mathbb{R}$, and check that it satisfies both the differential equation and the decay condition at infinity.  
Hint: When $x>0$, your contour should enclose one of the poles; when $x<0$, it should enclose the other one. Look for a final expression of the form $C e^{-a|x|}$ for some constant $C$.

\medskip

(e) (Exploration / extension.)

\begin{itemize}
  \item[(i)] How would the picture change if we considered instead the operator
  \[
  \widetilde{\mathcal{L}} = -\frac{d^2}{dx^2} - a^2,
  \]
  with the same decay condition at infinity? What would the Fourier-space resolvent $\widehat{\widetilde{G}}(k)$ look like, and where would its poles lie in the complex plane? Based on this, speculate about the qualitative behavior (oscillatory vs.\ decaying) of the corresponding Green’s function in $x$.
  \item[(ii)] More generally, the integrand $e^{ikx} \widehat{G}(k)$ may have several poles or even branch cuts in the complex plane. Describe in words how the positions of these singularities relative to the real axis influence the spatial decay, oscillation, or growth of the Green’s function. How does this perspective connect residue calculus with the spectral analysis of differential operators?
\end{itemize}

\end{problem}

% ===== Example 5: Residues and Green’s Functions for a Simple PDE (full solution) =====
\begin{problem}[Residues and Green’s Functions for a Simple PDE]
Let $a>0$ and consider the differential operator
\[
\mathcal{L} = -\frac{d^2}{dx^2} + a^2
\]
on the real line. The Green’s function $G:\mathbb{R}\to\mathbb{R}$ for $\mathcal{L}$ is defined by
\[
-\frac{d^2 G}{dx^2}(x) + a^2 G(x) = \delta(x), \qquad G(x)\to 0 \text{ as } |x|\to\infty,
\]
where $\delta$ is the Dirac delta distribution.

Using the Fourier transform
\[
\widehat{f}(k) = \int_{-\infty}^{\infty} f(x)e^{-ikx}\,dx,
\qquad
f(x) = \frac{1}{2\pi}\int_{-\infty}^{\infty} \widehat{f}(k)e^{ikx}\,dk,
\]
do the following:

\begin{enumerate}
  \item[(a)] Show that
  \[
  \widehat{G}(k) = \frac{1}{k^2 + a^2},
  \]
  and hence that
  \[
  G(x) = \frac{1}{2\pi}\int_{-\infty}^{\infty} \frac{e^{ikx}}{k^2 + a^2}\,dk.
  \]
  \item[(b)] Evaluate this integral by contour integration and the residue theorem, carefully distinguishing the cases $x>0$ and $x<0$.
  \item[(c)] Deduce that
  \[
  G(x) = \frac{1}{2a}e^{-a|x|},
  \]
  and verify directly that this function satisfies $-\frac{d^2 G}{dx^2} + a^2 G = \delta$ in the sense of distributions and $G(x)\to 0$ as $|x|\to\infty$.
  \item[(d)] Briefly explain how the location of the poles of $\widehat{G}(k)$ in the complex $k$-plane encodes the exponential decay of $G(x)$, and comment on how this illustrates the role of residue calculus in inverting transforms for PDE Green’s functions.
\end{enumerate}
\end{problem}

\begin{solution}
We construct the Green’s function by passing to Fourier space, where the differential operator becomes an algebraic multiplier, and then invert the resulting expression using contour integration.

\medskip

\noindent\textbf{(a) Fourier transform and algebraic resolvent.}
We start from
\[
-\frac{d^2 G}{dx^2}(x) + a^2 G(x) = \delta(x).
\]
Taking the Fourier transform of both sides, and using linearity, we obtain
\[
\mathcal{F}\!\left[-\frac{d^2 G}{dx^2}\right](k) + \mathcal{F}\!\left[a^2G\right](k)
= \mathcal{F}[\delta](k).
\]
We recall three basic facts:
\begin{enumerate}
  \item The Fourier transform of the second derivative is
  \[
  \mathcal{F}\!\left[\frac{d^2 G}{dx^2}\right](k) = (ik)^2 \widehat{G}(k) = -k^2\widehat{G}(k),
  \]
  provided $G$ decays sufficiently fast so that these manipulations are justified.
  \item Therefore
  \[
  \mathcal{F}\!\left[-\frac{d^2 G}{dx^2}\right](k)
  = -\left(-k^2\widehat{G}(k)\right)=k^2\widehat{G}(k).
  \]
  \item The Fourier transform of the delta distribution is
  \[
  \mathcal{F}[\delta](k) = \int_{-\infty}^{\infty} \delta(x)e^{-ikx}\,dx = 1.
  \]
\end{enumerate}
It follows that
\[
k^2 \widehat{G}(k) + a^2 \widehat{G}(k) = 1,
\]
so
\[
\widehat{G}(k) = \frac{1}{k^2 + a^2}.
\]
Using the inverse transform, we obtain
\[
G(x) = \frac{1}{2\pi}\int_{-\infty}^{\infty}\widehat{G}(k)e^{ikx}\,dk
= \frac{1}{2\pi}\int_{-\infty}^{\infty}\frac{e^{ikx}}{k^2 + a^2}\,dk.
\]
This is an explicit Fourier integral representation of the Green’s function.

\medskip

\noindent\textbf{(b) Evaluation by contour integration.}
We now evaluate
\[
I(x) := \int_{-\infty}^{\infty}\frac{e^{ikx}}{k^2 + a^2}\,dk
\]
for real $x$ using complex analysis. The integrand
\[
f(k) = \frac{e^{ikx}}{k^2 + a^2}
\]
is a meromorphic function of the complex variable $k$. The denominator factors as
\[
k^2 + a^2 = (k-ia)(k+ia),
\]
so $f$ has simple poles at
\[
k = ia, \qquad k = -ia.
\]

We treat $x>0$ and $x<0$ separately, because the exponential factor $e^{ikx}$ decays in different half-planes depending on the sign of $x$.

\medskip

\noindent\emph{Case $x>0$.}
For $x>0$, it is convenient to close the contour in the upper half-plane. Consider the contour $C_R$ consisting of the interval $[-R,R]$ on the real axis and a semicircle $\Gamma_R$ of radius $R$ in the upper half-plane, oriented counterclockwise. By the residue theorem,
\[
\int_{C_R} f(k)\,dk = 2\pi i \sum \operatorname{Res}\bigl(f;k_j\bigr),
\]
where the sum is over the poles $k_j$ of $f$ inside $C_R$. For $R$ large enough, the pole at $k=ia$ lies inside $C_R$, while the pole at $k=-ia$ does not. Thus,
\[
\int_{C_R} f(k)\,dk = 2\pi i\,\operatorname{Res}\bigl(f;k=ia\bigr).
\]

We write the contour integral as
\[
\int_{C_R} f(k)\,dk = \int_{-R}^R \frac{e^{ikx}}{k^2 + a^2}\,dk + \int_{\Gamma_R} \frac{e^{ikx}}{k^2 + a^2}\,dk.
\]
Standard estimates (Jordan’s lemma) show that for $x>0$, the second integral over $\Gamma_R$ tends to zero as $R\to\infty$. Indeed, if $k = Re^{i\theta}$ with $\theta\in[0,\pi]$, then
\[
e^{ikx} = e^{iRe^{i\theta}x} = e^{iRx\cos\theta - Rx\sin\theta},
\]
and since $\sin\theta\ge 0$ on $[0,\pi]$, the factor $e^{-Rx\sin\theta}$ gives exponential decay along the semicircle as $R\to\infty$. The polynomial denominator $k^2 + a^2$ cannot counteract this decay.

Therefore, taking the limit $R\to\infty$, we obtain
\[
\int_{-\infty}^{\infty}\frac{e^{ikx}}{k^2 + a^2}\,dk
= 2\pi i\,\operatorname{Res}\bigl(f;k=ia\bigr), \qquad x>0.
\]

We now compute the residue at the simple pole $k=ia$. For a simple pole,
\[
\operatorname{Res}\bigl(f;k=ia\bigr) = \lim_{k\to ia} (k-ia)f(k)
= \lim_{k\to ia} \frac{(k-ia)e^{ikx}}{(k-ia)(k+ia)}
= \lim_{k\to ia} \frac{e^{ikx}}{k+ia}.
\]
Substituting $k=ia$ gives
\[
\operatorname{Res}\bigl(f;k=ia\bigr) = \frac{e^{i(ia)x}}{ia+ia} = \frac{e^{-ax}}{2ia}.
\]
Hence, for $x>0$,
\[
I(x) = \int_{-\infty}^{\infty}\frac{e^{ikx}}{k^2 + a^2}\,dk
= 2\pi i \cdot \frac{e^{-ax}}{2ia} = \pi \cdot \frac{e^{-ax}}{a}.
\]

\medskip

\noindent\emph{Case $x<0$.}
For $x<0$, we instead close the contour in the lower half-plane. Consider the contour $C_R'$ consisting of $[-R,R]$ on the real axis and a semicircle $\Gamma_R'$ in the lower half-plane, oriented clockwise. The residue theorem, taking orientation into account, yields
\[
\int_{C_R'} f(k)\,dk = -2\pi i \sum \operatorname{Res}\bigl(f;k_j\bigr),
\]
where the sum is now over poles in the lower half-plane. For $R$ large, only the pole at $k=-ia$ lies inside $C_R'$. As before, the integral over $\Gamma_R'$ tends to zero as $R\to\infty$ because, for $x<0$, the exponential factor decays in the lower half-plane. Thus,
\[
\int_{-\infty}^{\infty}\frac{e^{ikx}}{k^2 + a^2}\,dk
= -2\pi i\,\operatorname{Res}\bigl(f;k=-ia\bigr), \qquad x<0.
\]

We compute the residue at $k=-ia$:
\[
\operatorname{Res}\bigl(f;k=-ia\bigr) = \lim_{k\to -ia} (k+ia)f(k)
= \lim_{k\to -ia} \frac{(k+ia)e^{ikx}}{(k-ia)(k+ia)}
= \lim_{k\to -ia} \frac{e^{ikx}}{k-ia}.
\]
Substituting $k=-ia$ gives
\[
\operatorname{Res}\bigl(f;k=-ia\bigr) = \frac{e^{i(-ia)x}}{-ia-ia}
= \frac{e^{ax}}{-2ia}.
\]
Therefore, for $x<0$,
\[
I(x) = -2\pi i \cdot \frac{e^{ax}}{-2ia} = \pi \cdot \frac{e^{ax}}{a}.
\]

Combining, we have
\[
I(x) =
\begin{cases}
\dfrac{\pi}{a} e^{-ax}, & x>0,\\[4pt]
\dfrac{\pi}{a} e^{ax}, & x<0.
\end{cases}
\]
This can be written compactly as
\[
I(x) = \dfrac{\pi}{a} e^{-a|x|}, \qquad x\neq 0.
\]

\medskip

\noindent\textbf{(c) Formula for $G$ and verification.}
Recall that
\[
G(x) = \frac{1}{2\pi}I(x) = \frac{1}{2\pi}\int_{-\infty}^{\infty}\frac{e^{ikx}}{k^2 + a^2}\,dk.
\]
From the computation above,
\[
G(x) = \frac{1}{2\pi}\cdot \frac{\pi}{a}e^{-a|x|}
= \frac{1}{2a}e^{-a|x|}, \qquad x\neq 0.
\]
The potential singularity at $x=0$ is integrable, and the formula extends naturally to $x=0$ by continuity:
\[
G(0) = \frac{1}{2a}.
\]
Thus
\[
G(x) = \frac{1}{2a}e^{-a|x|}, \qquad x\in\mathbb{R}.
\]

We verify that this is indeed the Green’s function.

First, for $x\neq 0$, the absolute value $|x|$ is smooth, and we can differentiate explicitly. For $x>0$, we have $|x|=x$, so
\[
G(x) = \frac{1}{2a}e^{-ax}, \quad x>0.
\]
Then
\[
G'(x) = \frac{1}{2a}(-a)e^{-ax} = -\frac{1}{2}e^{-ax},
\]
\[
G''(x) = -\frac{1}{2}(-a)e^{-ax} = \frac{a}{2}e^{-ax}.
\]
Therefore
\[
-\frac{d^2G}{dx^2}(x) + a^2 G(x)
= -\frac{a}{2}e^{-ax} + a^2\cdot\frac{1}{2a}e^{-ax} = 0, \qquad x>0.
\]

For $x<0$, we have $|x|=-x$, so
\[
G(x) = \frac{1}{2a}e^{ax}, \quad x<0.
\]
Then
\[
G'(x) = \frac{1}{2a}a e^{ax} = \frac{1}{2}e^{ax},
\]
\[
G''(x) = \frac{1}{2}a e^{ax}.
\]
Thus
\[
-\frac{d^2G}{dx^2}(x) + a^2 G(x)
= -\frac{a}{2}e^{ax} + a^2\cdot\frac{1}{2a}e^{ax} = 0, \qquad x<0.
\]
Hence $G$ satisfies $-\frac{d^2G}{dx^2} + a^2 G = 0$ for all $x\neq 0$.

Next, we check the jump condition at $x=0$ to see that the delta distribution appears. Integrate the equation
\[
-\frac{d^2G}{dx^2}(x) + a^2 G(x) = \delta(x)
\]
over a small interval $[-\varepsilon,\varepsilon]$:
\[
\int_{-\varepsilon}^{\varepsilon}\left(-\frac{d^2G}{dx^2}(x) + a^2 G(x)\right)\,dx
= \int_{-\varepsilon}^{\varepsilon}\delta(x)\,dx = 1.
\]
Integrating the left-hand side and using the fundamental theorem of calculus,
\[
\int_{-\varepsilon}^{\varepsilon} -\frac{d^2G}{dx^2}(x)\,dx = -G'(\varepsilon) + G'(-\varepsilon),
\]
while the term with $a^2G$ contributes
\[
a^2\int_{-\varepsilon}^{\varepsilon} G(x)\,dx.
\]
As $\varepsilon\to 0^+$, the integral of $G$ tends to zero because $G$ is bounded near $0$, so
\[
\lim_{\varepsilon\to 0^+}\int_{-\varepsilon}^{\varepsilon}a^2G(x)\,dx = 0.
\]
Therefore,
\[
\lim_{\varepsilon\to 0^+}\left(-G'(\varepsilon) + G'(-\varepsilon)\right) = 1,
\]
or equivalently
\[
G'(0^+) - G'(0^-) = -1.
\]

Using the expressions found above,
\[
G'(0^+) = -\frac{1}{2}, \qquad G'(0^-) = \frac{1}{2},
\]
hence
\[
G'(0^+) - G'(0^-) = -\frac{1}{2} - \frac{1}{2} = -1,
\]
which matches the required jump. Thus $G$ satisfies the distributional equation
\[
-\frac{d^2G}{dx^2} + a^2 G = \delta.
\]

Finally, the decay condition $G(x)\to 0$ as $|x|\to\infty$ follows immediately from the exponential factor:
\[
\lim_{|x|\to\infty} G(x) = \lim_{|x|\to\infty} \frac{1}{2a}e^{-a|x|} = 0.
\]

\medskip

\noindent\textbf{(d) Interpretation via poles and residue calculus.}
The Fourier-space Green’s function is
\[
\widehat{G}(k) = \frac{1}{k^2 + a^2}.
\]
As a function of complex $k$, this has simple poles at $k=\pm ia$. These poles lie strictly off the real axis, symmetrically in the upper and lower half-planes, at an imaginary distance $a$ from the real axis.

In our contour integration, the choice of half-plane was dictated by the sign of $x$ so that the factor $e^{ikx}$ decayed on the large semicircle. The residue theorem then expressed the real integral as a sum of contributions from the poles enclosed by the contour. The distance of the poles from the real axis determines the exponential rate of decay of $G(x)$: because the imaginary parts of the poles are $\pm a$, the spatial Green’s function decays like $e^{-a|x|}$. In more complicated problems with many poles or branch points, each singularity contributes a term to $G(x)$ whose decay or oscillation is governed by the imaginary and real parts of its location.

This example illustrates the main idea of residue calculus in the context of PDEs: by viewing inverse Fourier (or Laplace) transforms as contour integrals and analyzing the singularities of the transformed resolvent, we can systematically invert transforms, compute Green’s functions, and understand how the spectrum of the operator (encoded in the poles) controls the qualitative behavior of solutions in physical space.
\end{solution}

\section{Extreme-, Stationary- and Saddle-Point Methods (*)}
% --- Narrative plan (auto-generated) ---
% This section develops the complex-analytic tools used to extract asymptotic information from integrals with large parameters, especially those arising in Fourier analysis, special functions, and solutions of differential equations. The key idea is that, for highly oscillatory or exponentially weighted integrals, only neighborhoods of a few special points—extrema, stationary points, and complex saddle points of the phase—contribute significantly to the value of the integral. By locating and classifying these points, and then deforming contours in the complex plane to pass through them in favorable directions, we obtain accurate approximations that are often inaccessible to direct calculation. These methods play a central role in applied mathematics: they underlie the stationary phase method in Fourier analysis, the method of steepest descent for Laplace- and Fourier-type integrals, and many classical asymptotic formulas for special functions that solve ODEs and PDEs. In turn, these techniques connect back to contour integration, Cauchy’s theorem, and analytic continuation, and forward to topics such as WKB analysis, dispersion in wave and Schrödinger equations, and the long-time behavior of dynamical systems.

% ===== Example 1: A Simple Laplace-Type Integral with a Real Maximum (inquiry-based) =====
\begin{problem}[A Simple Laplace-Type Integral with a Real Maximum]
In many problems of statistical mechanics and large deviations, one encounters integrals of the form
\[
I(\lambda)=\int_0^1 e^{\lambda f(x)}\,g(x)\,dx,\qquad \lambda\to+\infty,
\]
where $\lambda$ is a large parameter. Physically, $e^{\lambda f(x)}$ may be a Boltzmann factor and $g(x)$ may encode a density of states or an observable. When $f$ has a unique maximum, one expects that only a small neighborhood of that maximum contributes substantially to the integral. In this problem you will make this precise and discover the standard Laplace approximation, where the integral is expressed in terms of local Taylor data of $f$ and $g$ at the maximizing point.

Assume that $f,g\in C^3([0,1])$ are real-valued, and that $f$ has a unique global maximum at some point $x_0\in(0,1)$. Moreover, suppose that this maximum is nondegenerate, in the sense that
\[
f'(x_0)=0,\qquad f''(x_0)<0.
\]
Define
\[
I(\lambda)=\int_0^1 e^{\lambda f(x)}\,g(x)\,dx,\qquad \lambda>0.
\]

\smallskip

(a) \textbf{A concrete warm-up.} Consider the special case
\[
f(x)=-\bigl(x-\tfrac{1}{2}\bigr)^2,\qquad g(x)\equiv 1,
\]
so that
\[
I(\lambda)=\int_0^1 e^{-\lambda (x-\frac12)^2}\,dx.
\]
Explain why the main contribution to the integral for large $\lambda$ must come from values of $x$ close to $\frac12$. Then, by an explicit change of variables, show that
\[
I(\lambda) \sim \sqrt{\frac{\pi}{\lambda}}
\quad\text{as }\lambda\to\infty.
\]
(Hint: Shift the variable to center the maximum at $0$, then rescale by $\sqrt{\lambda}$ and compare the resulting integral to the standard Gaussian integral over $\mathbb{R}$.)

\smallskip

(b) \textbf{Local Taylor expansions near the maximum.} Return to the general $f$ and $g$. 

(i) Write the second-order Taylor expansion of $f$ about $x_0$, and use $f''(x_0)<0$ to rewrite it in the form
\[
f(x) = f(x_0) - \frac{a}{2}(x-x_0)^2 + R_f(x),
\]
for some constant $a>0$ and a remainder term $R_f(x)$ that is $O(|x-x_0|^3)$ as $x\to x_0$. State precisely what this big-$O$ means.

(ii) Similarly, write the first-order Taylor expansion of $g$ at $x_0$,
\[
g(x) = g(x_0) + R_g(x),
\]
and describe the behavior of $R_g(x)$ as $x\to x_0$.

(Hint: Use the Taylor theorem with remainder. You do not need explicit formulas for the remainder; an inequality of the form $|R_f(x)|\le C|x-x_0|^3$ near $x_0$ is sufficient.)

\smallskip

(c) \textbf{Localizing the main contribution.} Fix a small number $\delta>0$ such that $(x_0-\delta,x_0+\delta)\subset(0,1)$. Consider the decomposition
\[
I(\lambda)=\int_{x_0-\delta}^{x_0+\delta} e^{\lambda f(x)}g(x)\,dx \;+\; \int_{[0,1]\setminus(x_0-\delta,x_0+\delta)} e^{\lambda f(x)}g(x)\,dx
=: I_{\text{near}}(\lambda)+I_{\text{far}}(\lambda).
\]

(i) Explain why, by continuity of $f$ and the fact that $x_0$ is the unique maximizer, there exists a constant $\eta>0$ (depending on $\delta$) such that
\[
f(x)\le f(x_0)-\eta\quad\text{for all }x\in[0,1]\setminus(x_0-\delta,x_0+\delta).
\]

(ii) Use this inequality to show that there is a constant $C>0$ (independent of $\lambda$) with
\[
|I_{\text{far}}(\lambda)| \le C e^{\lambda(f(x_0)-\eta)}.
\]

(iii) Based on your answer to part (a), what order of magnitude do you expect for $I(\lambda)$ as $\lambda\to\infty$? Use this to argue that
\[
\frac{I_{\text{far}}(\lambda)}{I(\lambda)}\to 0\quad\text{as }\lambda\to\infty.
\]
(Hint: Compare $I_{\text{far}}(\lambda)$ to a quantity of the form $e^{\lambda f(x_0)}/\sqrt{\lambda}$.)

\smallskip

(d) \textbf{Asymptotics from a local Gaussian approximation.} Focus now on $I_{\text{near}}(\lambda)$.

(i) Substitute the Taylor expansions from part (b) into $e^{\lambda f(x)}g(x)$, and factor out the dominant exponential $e^{\lambda f(x_0)}$. Show that
\[
e^{\lambda f(x)} g(x) 
= e^{\lambda f(x_0)}\, e^{-\frac{\lambda a}{2}(x-x_0)^2}
\Bigl[g(x_0) + \text{(terms that are small when $x$ is close to $x_0$ and $\lambda$ is large)}\Bigr].
\]
Make precise in words why the extra terms (coming from $R_f$ and $R_g$) are negligible compared to $g(x_0)$ in the region that actually contributes to the integral.

(Hint: Think about the typical size of $|x-x_0|$ in the main contributing region when $\lambda$ is large. Use your experience from part (a).)

(ii) Introduce the scaled variable
\[
y = \sqrt{\lambda a}\,(x-x_0),
\]
and rewrite the leading approximation to $I_{\text{near}}(\lambda)$ in terms of an integral in $y$. Explain why, for large $\lambda$, you can replace the finite limits in $y$ by $\pm\infty$ with only an exponentially small relative error.

(iii) Using the standard Gaussian integral
\[
\int_{-\infty}^{\infty} e^{-y^2/2}\,dy = \sqrt{2\pi},
\]
derive an asymptotic formula of the form
\[
I(\lambda) \sim e^{\lambda f(x_0)} g(x_0) \sqrt{\frac{2\pi}{-\,\lambda f''(x_0)}}
\quad\text{as }\lambda\to\infty.
\]

\smallskip

(e) \textbf{Extensions and variations.}

(i) Suppose $f$ has two distinct nondegenerate interior maxima $x_1$ and $x_2$ of equal height, that is, $f(x_1)=f(x_2)$ and both second derivatives at these points are negative. How would you expect the asymptotic formula for $I(\lambda)$ to change? Describe, without full proof, the form of $I(\lambda)$ for large $\lambda$ in this situation.

(ii) Suppose instead that $f$ has a unique maximum at $x_0$, but $f''(x_0)=0$ and $f^{(3)}(x_0)\neq 0$ (a degenerate maximum). How might the dependence on $\lambda$ change? Would you still expect a factor of $\lambda^{-1/2}$ in front, or something different? Give a heuristic argument.

(Hint: Consider what power of $(x-x_0)$ first appears in the Taylor expansion of $f$ and what rescaling $x-x_0$ requires to balance the large parameter $\lambda$ in the exponent.)
\end{problem}

% ===== Example 1: A Simple Laplace-Type Integral with a Real Maximum (full solution) =====
\begin{problem}[A Simple Laplace-Type Integral with a Real Maximum]
Let $f,g\in C^3([0,1])$ be real-valued functions. Assume that $f$ has a unique global maximum at an interior point $x_0\in(0,1)$, and that this maximum is nondegenerate:
\[
f'(x_0)=0,\qquad f''(x_0)<0.
\]
Consider
\[
I(\lambda)=\int_0^1 e^{\lambda f(x)}\,g(x)\,dx,\qquad \lambda>0.
\]

(a) Show that, as $\lambda\to\infty$,
\[
I(\lambda) \sim e^{\lambda f(x_0)}\,g(x_0)\,\sqrt{\frac{2\pi}{-\,\lambda f''(x_0)}}.
\]

(b) Briefly explain how this example illustrates the main idea of Laplace’s method and its relation to extreme-/stationary-point methods.
\end{problem}

\begin{solution}
We wish to obtain the leading asymptotic behavior of
\[
I(\lambda)=\int_0^1 e^{\lambda f(x)}\,g(x)\,dx
\]
as $\lambda\to\infty$, under the assumption that $f$ has a unique, nondegenerate interior maximum at $x_0$.

\medskip

\textbf{1. Splitting the integral into near and far regions.}
Fix a small $\delta>0$ such that $(x_0-\delta,x_0+\delta)\subset(0,1)$. We decompose
\[
I(\lambda)
=
\int_{x_0-\delta}^{x_0+\delta} e^{\lambda f(x)}g(x)\,dx
+
\int_{[0,1]\setminus(x_0-\delta,x_0+\delta)} e^{\lambda f(x)}g(x)\,dx
=: I_{\text{near}}(\lambda)+I_{\text{far}}(\lambda).
\]

We first show that $I_{\text{far}}$ is negligible compared to $I(\lambda)$. Since $f$ is continuous on $[0,1]$ and has a unique global maximum at $x_0$, we have
\[
f(x)<f(x_0)\quad\text{for all }x\in[0,1]\setminus\{x_0\}.
\]
On the compact set $[0,1]\setminus(x_0-\delta,x_0+\delta)$, the maximum of $f$ is strictly less than $f(x_0)$, so there exists $\eta>0$ such that
\[
f(x)\le f(x_0)-\eta\quad\text{for all }x\in[0,1]\setminus(x_0-\delta,x_0+\delta).
\]
Let $M_g:=\max_{x\in[0,1]}|g(x)|$. Then
\[
|I_{\text{far}}(\lambda)|
\le \int_{[0,1]\setminus(x_0-\delta,x_0+\delta)} e^{\lambda f(x)}\,|g(x)|\,dx
\le M_g\,e^{\lambda(f(x_0)-\eta)}.
\]

Later we will see that $I(\lambda)$ grows like
\[
e^{\lambda f(x_0)}\lambda^{-1/2},
\]
so
\[
\frac{|I_{\text{far}}(\lambda)|}{e^{\lambda f(x_0)}\lambda^{-1/2}}
\le
M_g\,\lambda^{1/2} e^{-\lambda\eta}
\to 0
\quad\text{as }\lambda\to\infty.
\]
Thus $I_{\text{far}}(\lambda)$ is exponentially small compared with the main term and can be neglected in the leading asymptotics. The dominant contribution comes from the neighborhood of $x_0$.

\medskip

\textbf{2. Taylor expansions near the maximum.}
Write the Taylor expansion of $f$ about $x_0$. Because $f'(x_0)=0$ and $f''(x_0)<0$, Taylor’s theorem with remainder gives
\[
f(x)
=
f(x_0)
+\frac{f''(x_0)}{2}(x-x_0)^2
+
R_f(x),
\]
where the remainder satisfies $R_f(x)=O(|x-x_0|^3)$ as $x\to x_0$. Let
\[
a := -f''(x_0) >0,
\]
so that
\[
f(x) = f(x_0) - \frac{a}{2}(x-x_0)^2 + R_f(x),
\qquad R_f(x)=O(|x-x_0|^3).
\]

Similarly, because $g\in C^3$, we expand
\[
g(x) = g(x_0) + R_g(x),
\qquad R_g(x)=O(|x-x_0|)\quad\text{as }x\to x_0.
\]

In words, $f$ looks like a downward-opening quadratic plus a small cubic error near $x_0$, while $g$ is approximately constant there.

\medskip

\textbf{3. Approximating the integrand in the near region.}
For $x\in[x_0-\delta,x_0+\delta]$ we have
\[
e^{\lambda f(x)}
=
\exp\!\Bigl(\lambda f(x_0) - \tfrac{\lambda a}{2}(x-x_0)^2 + \lambda R_f(x)\Bigr)
=
e^{\lambda f(x_0)}\,e^{-\frac{\lambda a}{2}(x-x_0)^2}\,e^{\lambda R_f(x)}.
\]
Thus
\[
e^{\lambda f(x)}g(x)
=
e^{\lambda f(x_0)}\,e^{-\frac{\lambda a}{2}(x-x_0)^2}\,
\bigl[g(x_0)+R_g(x)\bigr]\,e^{\lambda R_f(x)}.
\]

The key idea of Laplace’s method is that, for large $\lambda$, the Gaussian factor $e^{-\frac{\lambda a}{2}(x-x_0)^2}$ confines the main contribution to a window of width $O(\lambda^{-1/2})$ around $x_0$. In that window, $|x-x_0|\sim\lambda^{-1/2}$, so
\[
|R_g(x)| = O(|x-x_0|)=O(\lambda^{-1/2}),
\]
and
\[
\lambda R_f(x) = O\bigl(\lambda |x-x_0|^3\bigr) = O(\lambda\cdot \lambda^{-3/2}) = O(\lambda^{-1/2}),
\]
which tends to $0$ as $\lambda\to\infty$. Consequently, in the main contributing region,
\[
R_g(x)=o(1)\cdot 1,\qquad e^{\lambda R_f(x)} = 1+o(1),
\]
and the leading contribution comes from replacing $g(x)$ by $g(x_0)$ and $R_f(x)$ by $0$.

More precisely, we can write
\[
e^{\lambda f(x)}g(x)
=
e^{\lambda f(x_0)}\,e^{-\frac{\lambda a}{2}(x-x_0)^2}
\bigl[g(x_0)+\varepsilon_\lambda(x)\bigr],
\]
where $\varepsilon_\lambda(x)$ collects the error terms. One can show that the integral of the error term over $[x_0-\delta,x_0+\delta]$ is $o\bigl(e^{\lambda f(x_0)}\lambda^{-1/2}\bigr)$; for the leading asymptotics it suffices to note that this error is relatively small compared with the main term.

Thus, to leading order,
\[
I_{\text{near}}(\lambda)
\sim
e^{\lambda f(x_0)} g(x_0) \int_{x_0-\delta}^{x_0+\delta} e^{-\frac{\lambda a}{2}(x-x_0)^2}\,dx
\quad\text{as }\lambda\to\infty.
\]

\medskip

\textbf{4. Scaling to a Gaussian integral.}
We now compute the asymptotics of the Gaussian-type integral
\[
\int_{x_0-\delta}^{x_0+\delta} e^{-\frac{\lambda a}{2}(x-x_0)^2}\,dx.
\]
Introduce the scaled variable
\[
y = \sqrt{\lambda a}\,(x-x_0),
\qquad\text{so that}\quad
x-x_0 = \frac{y}{\sqrt{\lambda a}},\quad
dx = \frac{dy}{\sqrt{\lambda a}}.
\]
The integration limits become
\[
y_\pm(\lambda) = \sqrt{\lambda a}\,(\pm\delta) = \pm \delta\sqrt{\lambda a}.
\]
Thus
\[
\int_{x_0-\delta}^{x_0+\delta} e^{-\frac{\lambda a}{2}(x-x_0)^2}\,dx
=
\frac{1}{\sqrt{\lambda a}}\int_{-\delta\sqrt{\lambda a}}^{\delta\sqrt{\lambda a}} e^{-y^2/2}\,dy.
\]

As $\lambda\to\infty$, the limits $\pm\delta\sqrt{\lambda a}$ tend to $\pm\infty$. The tails of the Gaussian are exponentially small, so
\[
\int_{-\delta\sqrt{\lambda a}}^{\delta\sqrt{\lambda a}} e^{-y^2/2}\,dy
=
\int_{-\infty}^{\infty} e^{-y^2/2}\,dy + O\!\bigl(e^{-c\lambda}\bigr)
=
\sqrt{2\pi} + O\!\bigl(e^{-c\lambda}\bigr),
\]
for some $c>0$. Therefore
\[
\int_{x_0-\delta}^{x_0+\delta} e^{-\frac{\lambda a}{2}(x-x_0)^2}\,dx
\sim
\frac{1}{\sqrt{\lambda a}}\sqrt{2\pi}
=
\sqrt{\frac{2\pi}{a\lambda}}
\quad\text{as }\lambda\to\infty.
\]

Combining this with the dominant prefactor we found for $I_{\text{near}}(\lambda)$, we obtain
\[
I_{\text{near}}(\lambda)
\sim
e^{\lambda f(x_0)} g(x_0)\sqrt{\frac{2\pi}{a\lambda}}.
\]

\medskip

\textbf{5. Assembling the pieces.}
Recall that $a=-f''(x_0)$. Moreover, we have shown that $I_{\text{far}}(\lambda)$ is exponentially small relative to $I_{\text{near}}(\lambda)$, so
\[
I(\lambda) = I_{\text{near}}(\lambda)+I_{\text{far}}(\lambda)
\sim I_{\text{near}}(\lambda).
\]
Thus the leading asymptotics are
\[
I(\lambda)
\sim
e^{\lambda f(x_0)} g(x_0)\sqrt{\frac{2\pi}{-\,\lambda f''(x_0)}}
\quad\text{as }\lambda\to\infty.
\]
This proves part (a).

\medskip

\textbf{6. Conceptual discussion (part (b)).}
This example illustrates the central idea behind Laplace’s method and, more generally, extreme-/stationary-point methods:

\begin{itemize}
  \item The large parameter $\lambda$ in the exponential forces the integral to be dominated by neighborhoods of points where the exponent is largest (for real integrals) or stationary (for oscillatory integrals). Here, the unique nondegenerate maximum of $f$ at $x_0$ determines the main contribution.
  \item Near such a point, the function $f$ is well approximated by its quadratic Taylor polynomial, and $g$ is well approximated by its value at that point. This reduces the integral to a Gaussian integral, whose asymptotics can be computed explicitly.
  \item The resulting asymptotic formula depends only on local data at the extremum: $f(x_0)$, $f''(x_0)$, and $g(x_0)$, together with the universal constant $\sqrt{2\pi}$. All other parts of the interval contribute only exponentially small corrections.
\end{itemize}

In the broader context of “Extreme-, Stationary- and Saddle-Point Methods,” this real-variable Laplace example is the prototype: in more advanced settings one deforms contours into the complex plane and passes through saddle points of a complex phase function. The same pattern persists: locate critical points, approximate locally (often quadratically), and evaluate resulting Gaussian-type integrals to obtain leading asymptotics.
\end{solution}

% ===== Example 2: Stationary Phase for a One-Dimensional Oscillatory Integral (inquiry-based) =====
\begin{problem}[Stationary Phase for a One-Dimensional Oscillatory Integral]
In many wave and optics problems one encounters integrals of the form
\[
I(\lambda) = \int_{-\infty}^{\infty} e^{i \lambda \phi(x)} a(x)\,dx,
\]
where $\lambda$ is a large frequency or wave number, $\phi$ is a real phase function encoding travel time or optical path length, and $a$ is a slowly varying amplitude. For large $\lambda$ the integrand oscillates very rapidly, and one expects strong cancellations except near points where the phase is stationary. The purpose of this problem is to discover, in a simple one-dimensional setting, how and why such stationary points dominate the asymptotics of $I(\lambda)$, and to arrive at a concrete stationary phase formula.

Assume throughout that $a,\phi \in C^\infty(\mathbb{R})$ are real-valued and rapidly decaying (Schwartz class), that $\phi'$ has exactly one zero at $x_0$, and that this stationary point is nondegenerate in the sense that $\phi''(x_0)\neq 0$.

\medskip

(a) \textbf{Warm-up: a quadratic phase.}
Consider first the model oscillatory integral
\[
I_0(\lambda) = \int_{-\infty}^{\infty} e^{i \lambda x^2/2}\,\eta(x)\,dx,
\]
where $\eta\in C_c^\infty(\mathbb{R})$ is a smooth cutoff equal to $1$ in a neighborhood of $0$.  

\begin{enumerate}
\item[(i)] Show that the change of variables $y = \sqrt{\lambda}\,x$ transforms $I_0(\lambda)$ into
\[
I_0(\lambda) = \lambda^{-1/2} \int_{-\infty}^{\infty} e^{i y^2/2}\,\eta\!\left( \frac{y}{\sqrt{\lambda}} \right)\,dy.
\]
What does this already tell you about the \emph{order of magnitude} of $I_0(\lambda)$ as $\lambda\to\infty$?

\item[(ii)] Argue that for each fixed $y$ we have $\eta\bigl(y/\sqrt{\lambda}\bigr)\to 1$ as $\lambda\to\infty$, and that the functions $y\mapsto e^{i y^2/2}\,\eta\bigl(y/\sqrt{\lambda}\bigr)$ are uniformly integrable in $y$. Conclude, using dominated convergence (you may take this step heuristically if you prefer), that
\[
I_0(\lambda) \sim \lambda^{-1/2} \int_{-\infty}^{\infty} e^{i y^2/2}\,dy
\quad\text{as }\lambda\to\infty.
\]
\emph{Hint:} You can use that $|\eta|\leq 1$ and that $e^{i y^2/2}$ is bounded.

\item[(iii)] Recall, or accept as known, that
\[
\int_{-\infty}^{\infty} e^{i y^2/2}\,dy 
= e^{i\pi/4}\sqrt{2\pi}.
\]
Combine this with your work above to obtain an explicit leading-order asymptotic formula for $I_0(\lambda)$.
\end{enumerate}

\medskip

(b) \textbf{Isolating the stationary point.}
Return now to the general integral
\[
I(\lambda) = \int_{-\infty}^{\infty} e^{i \lambda \phi(x)} a(x)\,dx,
\]
with the assumptions stated at the beginning of the problem. Let $x_0$ be the unique point where $\phi'(x_0)=0$ and $\phi''(x_0)\neq 0$.  

Choose a smooth cutoff function $\chi\in C_c^\infty(\mathbb{R})$ such that $\chi(x)=1$ for $|x-x_0|\leq \varepsilon$ and $\chi(x)=0$ for $|x-x_0|\geq 2\varepsilon$, for some small $\varepsilon>0$ with no other critical points of $\phi$ in $|x-x_0|\leq 2\varepsilon$.

\begin{enumerate}
\item[(i)] Show that you can write $I(\lambda) = I_{\mathrm{near}}(\lambda) + I_{\mathrm{far}}(\lambda)$ with
\[
I_{\mathrm{near}}(\lambda)
= \int_{\mathbb{R}} e^{i \lambda \phi(x)} \chi(x) a(x)\,dx,
\qquad
I_{\mathrm{far}}(\lambda)
= \int_{\mathbb{R}} e^{i \lambda \phi(x)} \bigl(1-\chi(x)\bigr) a(x)\,dx.
\]

\item[(ii)] Explain, on an intuitive level, why you expect $I_{\mathrm{far}}(\lambda)$ to be small when $\lambda$ is large, and why almost all of the contribution to $I(\lambda)$ should come from $I_{\mathrm{near}}(\lambda)$.
\end{enumerate}

\medskip

(c) \textbf{Nonstationary phase and integration by parts.}
On the support of $1-\chi$, the derivative $\phi'(x)$ does not vanish and is bounded away from zero. This nonstationarity allows one to integrate by parts to exploit cancellations.

\begin{enumerate}
\item[(i)] Show that
\[
\frac{d}{dx}\bigl(e^{i\lambda\phi(x)}\bigr) = i\lambda\,\phi'(x)\,e^{i\lambda\phi(x)}.
\]
Rearrange this identity to express $e^{i\lambda\phi(x)}$ as a derivative divided by $i\lambda\phi'(x)$.

\item[(ii)] Use your expression from part (i) to integrate by parts once in $I_{\mathrm{far}}(\lambda)$. Show that
\[
I_{\mathrm{far}}(\lambda)
= \frac{1}{i\lambda}
\int_{\mathbb{R}} e^{i\lambda\phi(x)}\,b(x)\,dx
\]
for some smooth function $b$ that you should identify in terms of $a$, $\chi$, and $\phi'$.  

\emph{Hint:} Differentiate $(1-\chi(x))a(x)/\phi'(x)$.

\item[(iii)] Explain why the rapid decay of $a$ and smoothness of $\chi$ and $\phi$ imply that $b$ is bounded and rapidly decaying as well. Conclude that
\[
I_{\mathrm{far}}(\lambda) = \mathcal{O}\!\left(\frac{1}{\lambda}\right)
\quad\text{as }\lambda\to\infty.
\]

\item[(iv)] (Optional refinement.) How could you iterate the integration by parts to obtain faster decay, such as $\mathcal{O}(\lambda^{-N})$ for every $N$?
\end{enumerate}

\medskip

(d) \textbf{Local analysis near the stationary point: stationary phase.}
We now focus on $I_{\mathrm{near}}(\lambda)$, where $x$ stays close to $x_0$.

\begin{enumerate}
\item[(i)] Use Taylor’s theorem with remainder to write
\[
\phi(x)
= \phi(x_0) 
+ \tfrac{1}{2}\phi''(x_0)(x-x_0)^2 
+ R_3(x),
\]
where $R_3(x)$ is the cubic (and higher) order remainder. Write a similar Taylor expansion
\[
a(x) = a(x_0) + (x-x_0)\,a_1(x),
\]
where $a_1$ is a smooth function (you do not need the exact formula for $a_1$).

\item[(ii)] Substitute these expansions into $I_{\mathrm{near}}(\lambda)$ and factor out the constant phase $e^{i\lambda\phi(x_0)}$. Show that
\[
I_{\mathrm{near}}(\lambda)
= e^{i\lambda\phi(x_0)}
\int_{\mathbb{R}} 
e^{i\lambda\bigl(\tfrac{1}{2}\phi''(x_0)(x-x_0)^2 + R_3(x)\bigr)}
\chi(x)\bigl(a(x_0) + (x-x_0)a_1(x)\bigr)\,dx.
\]

\item[(iii)] Argue that the term involving $(x-x_0)a_1(x)$ is of lower order in $\lambda$ than the term involving $a(x_0)$, so that the leading contribution comes from replacing $a(x)$ by $a(x_0)$.  

\emph{Hint:} Compare to the model integral in part (a), and notice the additional factor $(x-x_0)$.

\item[(iv)] For the leading term, discard $R_3(x)$ in the exponent and replace $a(x)$ by $a(x_0)$ and $\chi(x)$ by $1$ (justifying informally that the resulting error is smaller in $\lambda$). You are led to
\[
I_{\mathrm{near}}(\lambda)
\approx e^{i\lambda\phi(x_0)}a(x_0)
\int_{\mathbb{R}} e^{i\lambda \tfrac{1}{2}\phi''(x_0)(x-x_0)^2}\,dx.
\]
Make the change of variables
\[
y = \sqrt{\lambda|\phi''(x_0)|}\,(x-x_0)
\]
to reduce this to a multiple of the model Gaussian integral from part (a), and show that
\[
I_{\mathrm{near}}(\lambda)
\sim e^{i\lambda\phi(x_0)}a(x_0)
\,e^{i \frac{\pi}{4}\operatorname{sgn}(\phi''(x_0))}
\sqrt{\frac{2\pi}{\lambda\,|\phi''(x_0)|}}.
\]

\item[(v)] Combine your estimates for $I_{\mathrm{near}}(\lambda)$ and $I_{\mathrm{far}}(\lambda)$ to obtain a full leading-order asymptotic formula for $I(\lambda)$ as $\lambda\to\infty$. State clearly what you have shown.
\end{enumerate}

\medskip

(e) \textbf{Extensions and variations.}
\begin{enumerate}
\item[(i)] Suppose now that $\phi'$ has several isolated nondegenerate zeros $x_1,\dots,x_k$. How do you expect the asymptotic formula for $I(\lambda)$ to change? Describe the result in words, and, if you can, write down the corresponding sum of contributions.

\item[(ii)] What changes if the stationary point is \emph{degenerate}, for instance if $\phi'(x_0)=\phi''(x_0)=0$ but $\phi^{(3)}(x_0)\neq 0$? Which part of the argument above breaks down, and what kind of scaling (compare $x-x_0$ to a power of $\lambda$) might you try instead?

\item[(iii)] Finally, consider the case where $\phi'(x)\neq 0$ for all $x\in\mathbb{R}$. What does the analysis of $I_{\mathrm{far}}(\lambda)$ then tell you about the behavior of the full integral $I(\lambda)$ as $\lambda\to\infty$?
\end{enumerate}

\end{problem}

% ===== Example 2: Stationary Phase for a One-Dimensional Oscillatory Integral (full solution) =====
\begin{problem}[Stationary Phase for a One-Dimensional Oscillatory Integral]
Let $a,\phi\in C^\infty(\mathbb{R})$ be real-valued and rapidly decaying, and suppose that $\phi'$ has exactly one zero at $x_0$, with $\phi''(x_0)\neq 0$, and that $|\phi'(x)|\ge c>0$ for all $x$ with $|x-x_0|\ge\delta$. Consider the oscillatory integral
\[
I(\lambda) = \int_{-\infty}^{\infty} e^{i\lambda\phi(x)} a(x)\,dx,\qquad \lambda>0.
\]
Show that as $\lambda\to\infty$,
\[
I(\lambda)
= e^{i\lambda\phi(x_0)}a(x_0)
\,e^{i\frac{\pi}{4}\operatorname{sgn}(\phi''(x_0))}
\sqrt{\frac{2\pi}{\lambda\,|\phi''(x_0)|}}
\;+\;\mathcal{O}\!\left(\lambda^{-3/2}\right).
\]
In particular, identify the leading-order behavior of $I(\lambda)$ and explain briefly why only a neighborhood of the stationary point $x_0$ contributes to this leading term.
\end{problem}

\begin{solution}
We want to understand the asymptotic behavior as $\lambda\to\infty$ of
\[
I(\lambda) = \int_{\mathbb{R}} e^{i\lambda\phi(x)} a(x)\,dx,
\]
under the assumption that the real phase $\phi$ has a single nondegenerate stationary point at $x_0$, and that the amplitude $a$ is smooth and rapidly decaying. The central ideas are: first, to split the integral into a contribution from a neighborhood of the stationary point and a contribution from the complementary, nonstationary region; second, to show that the nonstationary contribution is small by integration by parts; and third, to approximate the phase quadratically near $x_0$ and thereby reduce the analysis of the dominant term to a Gaussian integral. This is the prototypical example of the one-dimensional stationary phase method.

\medskip

\textbf{1. Decomposition into near and far regions.}
Let $\chi\in C_c^\infty(\mathbb{R})$ be a cutoff such that $\chi(x)=1$ for $|x-x_0|\le \delta$ and $\chi(x)=0$ for $|x-x_0|\ge 2\delta$, where $\delta>0$ is chosen so that $\phi'(x)\neq 0$ for all $x\neq x_0$ with $|x-x_0|\le 2\delta$. We decompose
\[
I(\lambda) = I_{\mathrm{near}}(\lambda) + I_{\mathrm{far}}(\lambda),
\]
where
\[
I_{\mathrm{near}}(\lambda)
= \int_{\mathbb{R}} e^{i\lambda\phi(x)} \chi(x)a(x)\,dx,
\qquad
I_{\mathrm{far}}(\lambda)
= \int_{\mathbb{R}} e^{i\lambda\phi(x)} \bigl(1-\chi(x)\bigr)a(x)\,dx.
\]
The first integral is localized near the stationary point $x_0$, while the second integral is over a region where $\phi'(x)$ does not vanish and is bounded away from zero.

\medskip

\textbf{2. The nonstationary contribution is small.}
On the support of $1-\chi$ we have $|\phi'(x)|\ge c>0$ by assumption. We begin by rewriting the oscillatory factor as a derivative. Differentiating gives
\[
\frac{d}{dx}\Bigl(e^{i\lambda\phi(x)}\Bigr) 
= i\lambda\phi'(x)e^{i\lambda\phi(x)}.
\]
Thus,
\[
e^{i\lambda\phi(x)}
= \frac{1}{i\lambda\phi'(x)}\frac{d}{dx}\Bigl(e^{i\lambda\phi(x)}\Bigr).
\]
Substituting into $I_{\mathrm{far}}$ and integrating by parts yields
\[
\begin{aligned}
I_{\mathrm{far}}(\lambda)
&= \int_{\mathbb{R}} \frac{1}{i\lambda\phi'(x)}
\frac{d}{dx}\Bigl(e^{i\lambda\phi(x)}\Bigr)
\bigl(1-\chi(x)\bigr)a(x)\,dx \\
&= \frac{1}{i\lambda}
\int_{\mathbb{R}} e^{i\lambda\phi(x)}
\frac{d}{dx}\Biggl(\frac{(1-\chi(x))a(x)}{\phi'(x)}\Biggr)\,dx.
\end{aligned}
\]
There is no boundary term because $a$ is rapidly decaying and $\chi$ has compact support.

Define
\[
b(x) = \frac{d}{dx}\Biggl(\frac{(1-\chi(x))a(x)}{\phi'(x)}\Biggr).
\]
Then
\[
I_{\mathrm{far}}(\lambda)
= \frac{1}{i\lambda}\int_{\mathbb{R}} e^{i\lambda\phi(x)} b(x)\,dx.
\]
Since $a$, $\chi$, and $\phi$ are smooth, and $1/\phi'$ is smooth and bounded on the support of $1-\chi$, it follows that $b$ is also smooth and rapidly decaying. In particular, $|b(x)|\le C_N(1+|x|)^{-N}$ for each $N$, for some constants $C_N$.

Therefore
\[
|I_{\mathrm{far}}(\lambda)|
\le \frac{1}{\lambda}\int_{\mathbb{R}} |b(x)|\,dx
\le \frac{C}{\lambda},
\]
for some constant $C$ independent of $\lambda$. Thus
\[
I_{\mathrm{far}}(\lambda) = \mathcal{O}\!\left(\lambda^{-1}\right)
\quad\text{as }\lambda\to\infty.
\]
This shows that regions where the phase is nonstationary contribute only lower-order terms.

Moreover, by iterating the integration by parts (each time differentiating an analogue of $b$), one finds that $I_{\mathrm{far}}(\lambda)$ actually decays faster than any power of $\lambda^{-1}$, but for our purposes it suffices that it is of order at most $\lambda^{-1}$.

\medskip

\textbf{3. Local analysis near the stationary point.}
We now turn to $I_{\mathrm{near}}(\lambda)$:
\[
I_{\mathrm{near}}(\lambda)
= \int_{\mathbb{R}} e^{i\lambda\phi(x)} \chi(x)a(x)\,dx.
\]
Because the integrand is supported in $|x-x_0|\le 2\delta$, we may Taylor expand the phase and amplitude around $x_0$.

By Taylor’s theorem with remainder,
\[
\phi(x)
= \phi(x_0) 
+ \phi'(x_0)(x-x_0)
+ \frac{1}{2}\phi''(x_0)(x-x_0)^2 
+ R_3(x),
\]
where $R_3(x)$ is a smooth function satisfying $R_3(x)=\mathcal{O}(|x-x_0|^3)$ as $x\to x_0$. Since $x_0$ is a stationary point, $\phi'(x_0)=0$, and thus
\[
\phi(x)
= \phi(x_0) 
+ \frac{1}{2}\phi''(x_0)(x-x_0)^2 
+ R_3(x).
\]
Similarly, we Taylor expand the amplitude:
\[
a(x)
= a(x_0) + a'(x_0)(x-x_0) + R_a(x),
\]
where $R_a(x)=\mathcal{O}(|x-x_0|^2)$ as $x\to x_0$. It is convenient to write this as
\[
a(x) = a(x_0) + (x-x_0)a_1(x),
\]
where $a_1$ is a smooth function.

Substituting these expansions into $I_{\mathrm{near}}$ and factoring out the constant phase $e^{i\lambda\phi(x_0)}$ gives
\[
\begin{aligned}
I_{\mathrm{near}}(\lambda)
&= e^{i\lambda\phi(x_0)}
\int_{\mathbb{R}} 
e^{i\lambda\bigl(\frac{1}{2}\phi''(x_0)(x-x_0)^2 + R_3(x)\bigr)}
\chi(x)\bigl(a(x_0) + (x-x_0)a_1(x)\bigr)\,dx \\
&= e^{i\lambda\phi(x_0)}\Bigl(
a(x_0) J_0(\lambda) + J_1(\lambda)\Bigr),
\end{aligned}
\]
where
\[
J_0(\lambda) 
= \int_{\mathbb{R}} 
e^{i\lambda\bigl(\frac{1}{2}\phi''(x_0)(x-x_0)^2 + R_3(x)\bigr)}
\chi(x)\,dx
\]
and
\[
J_1(\lambda)
= \int_{\mathbb{R}} 
e^{i\lambda\bigl(\frac{1}{2}\phi''(x_0)(x-x_0)^2 + R_3(x)\bigr)}
\chi(x)(x-x_0)a_1(x)\,dx.
\]

The integral $J_1(\lambda)$ carries an extra factor $(x-x_0)$, which, after a suitable rescaling, leads to an additional factor of $\lambda^{-1/2}$ compared to $J_0(\lambda)$. Thus $J_1(\lambda)$ is of lower order in $\lambda$ than $J_0(\lambda)$, and it contributes only to the error term at orders $\lambda^{-3/2}$ and smaller. A more careful stationary phase expansion would keep track of this systematically; for the leading term we may ignore $J_1(\lambda)$.

Similarly, the remainder $R_3(x)$ in the phase is of cubic order in $(x-x_0)$ and thus gives a small correction when $\lambda$ is large. The key observation is that the main contribution to $J_0(\lambda)$ comes from a neighborhood of $x_0$ where $|x-x_0|\lesssim \lambda^{-1/2}$, and in that region one has $\lambda R_3(x)=\mathcal{O}(\lambda |x-x_0|^3)=\mathcal{O}(\lambda^{-1/2})$, which is small. This allows us, at leading order, to replace $e^{i\lambda(\frac{1}{2}\phi''(x_0)(x-x_0)^2 + R_3(x))}$ by $e^{i\lambda\frac{1}{2}\phi''(x_0)(x-x_0)^2}$.

Putting these observations together, we obtain the leading approximation
\[
I_{\mathrm{near}}(\lambda)
= e^{i\lambda\phi(x_0)}a(x_0)
\int_{\mathbb{R}} 
e^{i\lambda\frac{1}{2}\phi''(x_0)(x-x_0)^2}\,dx
\;+\;\mathcal{O}\!\left(\lambda^{-3/2}\right).
\]

\medskip

\textbf{4. Evaluation of the Gaussian-type integral.}
We now evaluate
\[
G(\lambda) 
:= \int_{\mathbb{R}} e^{i\lambda\frac{1}{2}\phi''(x_0)(x-x_0)^2}\,dx.
\]
Set
\[
\alpha = \phi''(x_0)\neq 0,
\qquad
y = \sqrt{\lambda|\alpha|}\,(x-x_0).
\]
Then $dy = \sqrt{\lambda|\alpha|}\,dx$, so $dx = dy / \sqrt{\lambda|\alpha|}$, and
\[
\lambda\frac{1}{2}\alpha(x-x_0)^2
= \frac{1}{2}\alpha\,\lambda\,(x-x_0)^2
= \frac{1}{2}\,\operatorname{sgn}(\alpha)\,y^2.
\]
Therefore

\begin{aligned}
G(\lambda)
&= \int_{\mathbb{R}} e^{i\,\operatorname{sgn}(\alpha)\,y^2/2}\,
\frac{dy}{\sqrt{\lambda|\alpha|}} \\
&= \frac{1}{\sqrt{\lambda|\alpha|}}
\int_{-\infty}^{\infty} e^{i\,\operatorname{sgn}(\alpha)\,y^2/2}\,dy.
\end{aligned}

\begin{aligned}
G(\lambda)
&= \frac{1}{\sqrt{\lambda|\alpha|}}
\int_{-\infty}^{\infty} e^{i\,\operatorname{sgn}(\alpha)\,y^2/2}\,dy.
\end{aligned}

Using the standard Fresnel-type integral
\[
\int_{-\infty}^{\infty} e^{i y^2/2}\,dy = e^{i\pi/4}\sqrt{2\pi},
\]
we obtain
\[
\int_{-\infty}^{\infty} e^{i\,\operatorname{sgn}(\alpha)\,y^2/2}\,dy
= e^{i\frac{\pi}{4}\operatorname{sgn}(\alpha)}\sqrt{2\pi}.
\]
Therefore
\[
G(\lambda)
= e^{i\frac{\pi}{4}\operatorname{sgn}(\alpha)}\sqrt{\frac{2\pi}{\lambda|\alpha|}}
= e^{i\frac{\pi}{4}\operatorname{sgn}(\phi''(x_0))}
\sqrt{\frac{2\pi}{\lambda\,|\phi''(x_0)|}}.
\]

Recalling that $I_{\mathrm{near}}(\lambda)
= e^{i\lambda\phi(x_0)}a(x_0)G(\lambda)$ up to an error of order $\mathcal{O}(\lambda^{-3/2})$ (coming from the $(x-x_0)a_1(x)$ term and from replacing $R_3(x)$ and $\chi(x)$ by $0$ and $1$, respectively), we conclude
\[
I_{\mathrm{near}}(\lambda)
= e^{i\lambda\phi(x_0)}a(x_0)\,
e^{i\frac{\pi}{4}\operatorname{sgn}(\phi''(x_0))}
\sqrt{\frac{2\pi}{\lambda\,|\phi''(x_0)|}}
\;+\;\mathcal{O}\!\left(\lambda^{-3/2}\right).
\]

\medskip

\textbf{5. Combination of near and far contributions.}
We have shown that
\[
I(\lambda) = I_{\mathrm{near}}(\lambda) + I_{\mathrm{far}}(\lambda),
\]
with
\[
I_{\mathrm{near}}(\lambda)
= e^{i\lambda\phi(x_0)}a(x_0)\,
e^{i\frac{\pi}{4}\operatorname{sgn}(\phi''(x_0))}
\sqrt{\frac{2\pi}{\lambda\,|\phi''(x_0)|}}
\;+\;\mathcal{O}\!\left(\lambda^{-3/2}\right),
\]
and $I_{\mathrm{far}}(\lambda)$ decaying faster than any power of $\lambda^{-1}$ by iterated integration by parts. In particular, $I_{\mathrm{far}}(\lambda)=\mathcal{O}(\lambda^{-2})$, which is certainly $\mathcal{O}(\lambda^{-3/2})$.

Hence
\[
I(\lambda)
= e^{i\lambda\phi(x_0)}a(x_0)\,
e^{i\frac{\pi}{4}\operatorname{sgn}(\phi''(x_0))}
\sqrt{\frac{2\pi}{\lambda\,|\phi''(x_0)|}}
\;+\;\mathcal{O}\!\left(\lambda^{-3/2}\right),
\quad \lambda\to\infty.
\]

This establishes the stated stationary phase asymptotic. The leading term depends only on the value of the amplitude $a$ and the quadratic behavior of the phase $\phi$ at the stationary point $x_0$, while the contribution from regions where the phase is nonstationary is suppressed by rapid oscillations and integration by parts, and thus contributes only to lower-order terms in $\lambda^{-1}$.

\end{solution}

% ===== Example 3: First Steps in the Method of Steepest Descent (inquiry-based) =====
\begin{problem}[First Steps in the Method of Steepest Descent]
Many integrals arising in wave propagation and quantum mechanics look like
\[
I(\lambda)=\int_\Gamma e^{\lambda f(z)}g(z)\,dz,
\]
where $f$ is analytic, $\Gamma$ is a contour in the complex plane, and $\lambda$ is large.  On the real line the integrand may merely oscillate, but in the complex plane one can often deform the contour to a path along which the real part of $f$ decreases rapidly from a saddle point.  Then a local quadratic approximation near that saddle point already gives the leading behaviour of $I(\lambda)$ for large $\lambda$.

In this problem you will explore these ideas for the concrete oscillatory integral
\[
I(\lambda)=\int_{-\infty}^{\infty} e^{i\lambda x^2}\,dx,\qquad \lambda>0,
\]
which is a basic model for Fresnel-type integrals in optics.

\medskip

(a) Regard the integral $I(\lambda)$ as a complex contour integral of the form
\[
\int_\Gamma e^{\lambda f(z)}g(z)\,dz
\]
over a contour $\Gamma\subset\mathbb{C}$.

\quad(i) Specify $f(z)$, $g(z)$, and $\Gamma$ in this example.

\quad(ii) Compute $f'(z)$ and find all points in $\mathbb{C}$ where $f'(z)=0$.  These are the \emph{saddle points} of $f$.

\quad(iii) Briefly explain why, for large $\lambda$, one might expect only a neighbourhood of these saddle points to contribute significantly to the integral.

% Hint: Think about what happens to $e^{\lambda f(z)}$ when $\operatorname{Re} f(z)$ is negative and $\lambda$ is large and positive.

\medskip

(b) To understand the geometry near the saddle at the origin, write $z=x+iy$ with $x,y\in\mathbb{R}$ and compute the real and imaginary parts of $f(z)=iz^2$.

\quad(i) Show that
\[
f(z)=iz^2 = \operatorname{Re} f(x+iy) \;+\; i\,\operatorname{Im} f(x+iy)
\]
with explicit formulas for $\operatorname{Re} f(x+iy)$ and $\operatorname{Im} f(x+iy)$ in terms of $x$ and $y$.

\quad(ii) Determine the equations of the curves in the $(x,y)$-plane on which $\operatorname{Im} f(x+iy)$ is constant, in particular $\operatorname{Im} f(x+iy)=0$.

\quad(iii) Among the straight lines through the origin on which $\operatorname{Im} f=0$, decide on which line $\operatorname{Re} f$ is strictly negative away from the origin, and on which it is strictly positive.  Interpret the first line as a path of \emph{steepest descent} and the second as a path of \emph{steepest ascent}.

Hint: Once you have formulas for $\operatorname{Re} f$ and $\operatorname{Im} f$, first find all straight lines through the origin on which $\operatorname{Im} f\equiv0$, then examine the sign of $\operatorname{Re} f$ on each such line.

\medskip

(c) Let $\Gamma_{\mathrm{sd}}$ be the line of steepest descent through the origin that you identified in part (b), oriented from $-\infty$ to $+\infty$.

\quad(i) Parametrise $\Gamma_{\mathrm{sd}}$ in the form $z(t)=\alpha t$ with $t\in\mathbb{R}$ and a complex constant $\alpha$ of modulus $1$.  (There should be exactly one such parametrisation with $\alpha$ in the first quadrant.)

\quad(ii) Show that along this line,
\[
e^{i\lambda z^2(t)} = e^{-\lambda t^2}
\]
for real $t$.  Conclude that along $\Gamma_{\mathrm{sd}}$ the integrand $e^{i\lambda z^2}$ decays like a Gaussian away from the origin.

\quad(iii) Rewrite $\displaystyle \int_{\Gamma_{\mathrm{sd}}} e^{i\lambda z^2}\,dz$ as a real integral in $t$ and evaluate it explicitly.

Hint: You will obtain an integral of the form $\int_{-\infty}^\infty e^{-\lambda t^2}\,dt$, which you already know how to compute.

\medskip

(d) Now connect the real axis $\Gamma=\mathbb{R}$ to the steepest descent contour $\Gamma_{\mathrm{sd}}$ by an appropriate contour deformation and show that $I(\lambda)$ actually equals the integral you computed along $\Gamma_{\mathrm{sd}}$.

One way to proceed is the following.

\quad(i) For a large radius $R>0$, consider the polygonal contour that starts at $-R$ on the real axis, goes along the real axis to $R$, then follows straight line segments connecting $R$ to $Re^{i\pi/4}$ and $-R$ to $-Re^{i\pi/4}$, and finally runs along $\Gamma_{\mathrm{sd}}$ from $Re^{i\pi/4}$ back to $-Re^{i\pi/4}$.  Sketch this contour in the complex plane.

\quad(ii) Explain why, since $e^{i\lambda z^2}$ is an entire function, the integral of $e^{i\lambda z^2}$ around the closed contour is zero.

\quad(iii) Show that the contributions from the two ``corner'' segments from $\pm R$ to $\pm Re^{i\pi/4}$ tend to zero as $R\to\infty$.  (You will need to estimate $|e^{i\lambda z^2}|$ along these segments.)

% Hint: On a segment where $z=Re^{i\theta}$ with fixed $\theta\in(0,\pi/4)$ and $R$ large, use your formula for $\operatorname{Re} f(Re^{i\theta})$ to bound $|e^{i\lambda z^2}| = e^{\lambda\operatorname{Re} f(z)}$.

\quad(iv) By letting $R\to\infty$, deduce that
\[
\int_{-\infty}^{\infty} e^{i\lambda x^2}\,dx
=
\int_{\Gamma_{\mathrm{sd}}} e^{i\lambda z^2}\,dz,
\]
and combine this with your computation from part (c) to obtain a closed-form expression for $I(\lambda)$.

\medskip

(e) Finally, consider two short extensions.

\quad(i) Suppose instead that
\[
I_1(\lambda)=\int_{-\infty}^{\infty} e^{i\lambda(x^2+x)}\,dx.
\]
Without doing detailed calculations, explain how the location of the saddle point of $f(z)=i(z^2+z)$ changes compared to $f(z)=iz^2$, and how you would adapt the steepest descent contour.

Hint: Complete the square in the exponent to see where the phase is stationary.

\quad(ii) For a general analytic function $f$ with a nondegenerate saddle point at $z_0$ (meaning $f'(z_0)=0$ and $f''(z_0)\neq0$), and an analytic $g$ with $g(z_0)\neq0$, what local approximation to $f$ and $g$ near $z_0$ would you use to obtain the leading-order asymptotics of
\[
\int_\Gamma e^{\lambda f(z)}g(z)\,dz\quad\text{as }\lambda\to\infty?
\]
Describe the shape of the resulting model integral and how it relates to the Gaussian integral you evaluated in this problem.

\end{problem}

% ===== Example 3: First Steps in the Method of Steepest Descent (full solution) =====
\begin{problem}[First Steps in the Method of Steepest Descent]
Let $\lambda>0$ and consider the oscillatory integral
\[
I(\lambda)=\int_{-\infty}^{\infty} e^{i\lambda x^2}\,dx.
\]
Regard this as a contour integral of the form $\int_\Gamma e^{\lambda f(z)}g(z)\,dz$ with $f(z)=iz^2$ and $\Gamma=\mathbb{R}$.

\begin{enumerate}
\item Find the saddle point(s) of $f$ and, by computing $\operatorname{Re} f(x+iy)$ and $\operatorname{Im} f(x+iy)$, determine the straight lines through the saddle along which $\operatorname{Im} f$ is constant.  Identify which of these is a line of steepest descent and which is a line of steepest ascent.

\item Show that the real axis can be deformed to a contour that passes through the saddle along the line of steepest descent, with vanishing contributions from the connecting segments.  Evaluate $I(\lambda)$ by integrating along the steepest descent contour.

\item Briefly explain how this example illustrates the main ideas of the method of steepest descent and of saddle-point methods more generally.
\end{enumerate}
\end{problem}

\begin{solution}
We are asked to evaluate
\[
I(\lambda)=\int_{-\infty}^{\infty} e^{i\lambda x^2}\,dx,\qquad \lambda>0,
\]
by viewing it as a contour integral in the complex plane and deforming the contour to a path of steepest descent through a saddle point of the phase.

\medskip

\textbf{1. Saddle point and level curves of the phase.}

We write the integrand in the form
\[
e^{\lambda f(z)}g(z),\qquad f(z)=iz^2,\quad g(z)\equiv 1,
\]
and think of the original integral as taken over the contour $\Gamma=\mathbb{R}$.

The saddle points of $f$ are the critical points of $f$, that is, the zeros of $f'(z)$.  Since
\[
f'(z)=2iz,
\]
we see that $f'(z)=0$ if and only if $z=0$.  Thus there is a single saddle point at $z_0=0$.

To understand the geometry near this saddle, we write $z=x+iy$ with $x,y\in\mathbb{R}$ and compute
\[
z^2 = (x+iy)^2 = x^2-y^2+2ixy,
\]
so
\[
f(z)=iz^2 = i(x^2-y^2+2ixy)
          = i(x^2-y^2)-2xy.
\]
Therefore
\[
\operatorname{Re} f(x+iy) = -2xy,
\qquad
\operatorname{Im} f(x+iy) = x^2-y^2.
\]

The curves of constant imaginary part $\operatorname{Im} f(x+iy)=c$ are thus level curves of the function $x^2-y^2$, i.e.
\[
x^2-y^2 = c,
\]
which are hyperbolas (for $c\neq0$) and the pair of lines $y=\pm x$ (for $c=0$).

In particular, the lines through the origin on which $\operatorname{Im} f\equiv0$ are
\[
y=x\quad\text{and}\quad y=-x.
\]
On these lines, $\operatorname{Re} f$ has opposite signs.  Indeed, along $y=x$ we have
\[
\operatorname{Re} f(x,x) = -2x\cdot x = -2x^2 \le 0,
\]
with strict negativity for $x\neq0$, while along $y=-x$ we have
\[
\operatorname{Re} f(x,-x)= -2x(-x)=2x^2 \ge 0,
\]
with strict positivity for $x\neq0$.

Thus, as we move away from the saddle at the origin along the line $y=x$, the real part $\operatorname{Re} f$ decreases (it becomes strictly negative), whereas along the line $y=-x$ it increases (it becomes strictly positive).  Because the magnitude of the integrand is
\[
\bigl|e^{\lambda f(z)}\bigr| = e^{\lambda \operatorname{Re} f(z)},
\]
and $\lambda>0$, the line $y=x$ is a path along which the integrand decays rapidly away from the saddle, while $y=-x$ is a path along which it grows rapidly.

In the terminology of the method, $y=x$ is a \emph{steepest descent} contour through the saddle at $0$, and $y=-x$ is a \emph{steepest ascent} contour.

\medskip

\textbf{2. The steepest descent contour and the Gaussian model.}

We now parametrise the line of steepest descent $y=x$ as a contour in $\mathbb{C}$.  This is the line making an angle of $\pi/4$ with the positive real axis, so a natural parametrisation is
\[
z(t) = e^{i\pi/4}\,t,\qquad t\in\mathbb{R}.
\]
Note that $|e^{i\pi/4}|=1$, and as $t$ runs from $-\infty$ to $\infty$, $z(t)$ runs along the entire line $y=x$ from $-\infty e^{i\pi/4}$ to $\infty e^{i\pi/4}$.
As $t$ runs from $-\infty$ to $\infty$, $z(t)$ runs along the entire line $y=x$.

Along this contour,
\[
z(t)^2 = e^{i\pi/2}\,t^2 = i t^2,
\]
so
\[
i\lambda z(t)^2 = i\lambda\,(i t^2) = -\lambda t^2,
\]
and therefore
\[
e^{i\lambda z(t)^2} = e^{-\lambda t^2},\qquad t\in\mathbb{R}.
\]
Thus the integrand decays like a Gaussian away from the saddle along the steepest descent line.

Moreover
\[
dz = z'(t)\,dt = e^{i\pi/4}\,dt,
\]
so the integral along the steepest descent contour $\Gamma_{\mathrm{sd}}$ is
\[
\int_{\Gamma_{\mathrm{sd}}} e^{i\lambda z^2}\,dz
=
\int_{-\infty}^\infty e^{i\lambda z(t)^2}\,z'(t)\,dt
=
e^{i\pi/4}\int_{-\infty}^\infty e^{-\lambda t^2}\,dt.
\]
Using the standard Gaussian integral,
\[
\int_{-\infty}^\infty e^{-\lambda t^2}\,dt = \sqrt{\frac{\pi}{\lambda}},
\]
we obtain
\[
\int_{\Gamma_{\mathrm{sd}}} e^{i\lambda z^2}\,dz
=
e^{i\pi/4}\sqrt{\frac{\pi}{\lambda}}.
\]

\medskip

\textbf{3. Deforming the real axis to the steepest descent contour.}

To justify replacing the original real-axis contour by $\Gamma_{\mathrm{sd}}$, consider for $R>0$ the contour $\mathcal{C}_R$ consisting of four pieces:

\begin{itemize}
\item the real segment $\gamma_1$: $[-R,R]\subset\mathbb{R}$;
\item the segment $\gamma_2$ along the ray at angle $\pi/4$ from $R$ to $Re^{i\pi/4}$;
\item the segment $\gamma_3$ along the steepest descent line from $Re^{i\pi/4}$ to $-Re^{i\pi/4}$ (this is part of $\Gamma_{\mathrm{sd}}$);
\item the segment $\gamma_4$ along the ray at angle $\pi/4$ from $-Re^{i\pi/4}$ back to $-R$.
\end{itemize}

The function $e^{i\lambda z^2}$ is entire, so by Cauchy's theorem the integral of $e^{i\lambda z^2}$ around the closed contour $\mathcal{C}_R$ is zero:
\[
\int_{\gamma_1} e^{i\lambda z^2}\,dz
+
\int_{\gamma_2} e^{i\lambda z^2}\,dz
+
\int_{\gamma_3} e^{i\lambda z^2}\,dz
+
\int_{\gamma_4} e^{i\lambda z^2}\,dz
=0.
\]
Rearranging,
\[
\int_{-R}^R e^{i\lambda x^2}\,dx
=
-\int_{\gamma_2} e^{i\lambda z^2}\,dz
-\int_{\gamma_4} e^{i\lambda z^2}\,dz
-\int_{\gamma_3} e^{i\lambda z^2}\,dz.
\]

We now estimate the contributions from $\gamma_2$ and $\gamma_4$ as $R\to\infty$.  On each of these rays we have $z=\rho e^{i\pi/4}$ with $\rho$ ranging over an interval of the form $[\rho_0,R]$ and $\rho_0$ independent of $R$; hence
\[
i\lambda z^2
=
i\lambda \rho^2 e^{i\pi/2}
=
i\lambda \rho^2 (i)
=
-\lambda\rho^2,
\]
so
\[
\left|e^{i\lambda z^2}\right|
=
e^{-\lambda\rho^2}.
\]
The length of each ray segment is $O(R)$, while the integrand is bounded by $e^{-\lambda\rho^2}$ with $\rho\ge\rho_0$. Thus
\[
\Bigl|\int_{\gamma_2} e^{i\lambda z^2}\,dz\Bigr|
\le
\int_{\rho_0}^R e^{-\lambda\rho^2}\,|dz|
\le
\int_{\rho_0}^R e^{-\lambda\rho^2}\,d\rho
\le
\int_{\rho_0}^{\infty} e^{-\lambda\rho^2}\,d\rho
\to 0\quad\text{as }R\to\infty,
\]
and the same estimate applies to $\gamma_4$.

Therefore, letting $R\to\infty$, the contributions from $\gamma_2$ and $\gamma_4$ vanish, while $\gamma_1$ tends to the whole real axis and $\gamma_3$ tends to the whole steepest descent line $\Gamma_{\mathrm{sd}}$. We obtain
\[
\int_{-\infty}^{\infty} e^{i\lambda x^2}\,dx
=
\int_{\Gamma_{\mathrm{sd}}} e^{i\lambda z^2}\,dz.
\]
Combining this with the computation in step 2,
\[
I(\lambda)
=
\int_{-\infty}^{\infty} e^{i\lambda x^2}\,dx
=
e^{i\pi/4}\sqrt{\frac{\pi}{\lambda}},
\qquad \lambda>0.
\]

\medskip

\textbf{4. Relation to the general method of steepest descent.}

In this example:

\begin{itemize}
\item The phase function is $f(z)=iz^2$ with a nondegenerate saddle at $z_0=0$ ($f'(0)=0$, $f''(0)=2i\neq0$).
\item Near the saddle one may approximate
\[
f(z) \approx f(0) + \tfrac12 f''(0)(z-0)^2 = i z^2,
\qquad g(z) \approx g(0)=1,
\]
so the integrand is locally modeled by a Gaussian in the complex variable $z$.
\item Choosing a contour through $z_0$ along which $\Im f$ is constant and $\Re f$ decreases away from $z_0$ (a steepest descent path) reduces the local contribution of the integral to a real Gaussian integral, which can be evaluated explicitly.
\end{itemize}

For a general analytic $f$ with a nondegenerate saddle at $z_0$ and analytic $g$ with $g(z_0)\neq0$, one linearizes $g$ and quadratically approximates $f$ near $z_0$:
\[
f(z) \approx f(z_0) + \frac{f''(z_0)}{2}(z-z_0)^2,
\qquad
g(z) \approx g(z_0),
\]
and deforms the contour so that near $z_0$ it follows a steepest descent direction.  After a suitable change of variables, the local contribution becomes proportional to the standard Gaussian integral, yielding the leading asymptotics
\[
\int_\Gamma e^{\lambda f(z)}g(z)\,dz
\sim
g(z_0)\,e^{\lambda f(z_0)}\,
\sqrt{\frac{2\pi}{\lambda f''(z_0)}}
\quad\text{as }\lambda\to\infty,
\]
where the square root and the contour are chosen consistently with the chosen steepest descent direction.  The present computation of $I(\lambda)$ is the simplest concrete instance of this general saddle-point/steepest-descent principle.

\end{solution}

% ===== Example 4: Asymptotics of a Special Function via Saddle Points (inquiry-based) =====
\begin{problem}[Asymptotics of a Special Function via Saddle Points]
Bessel functions appear throughout applied mathematics, for instance in wave propagation in cylindrical geometries and in diffraction problems. For large argument, the Bessel function $J_\nu(\lambda)$ behaves like a slowly decaying oscillatory function, much like a cosine with slowly varying amplitude. In this problem you will rediscover this oscillatory asymptotic form by starting from a complex contour integral representation of $J_\nu$, identifying the relevant saddle points of the phase, and applying the method of steepest descent. The goal is to see in detail how the geometry of the complex phase determines which parts of the contour contribute to the asymptotics.

We fix a real order $\nu$ and consider the large-parameter limit $\lambda \to +\infty$.

\medskip

Recall the following contour integral representation of the Bessel function $J_\nu$:
\[
J_\nu(\lambda)
=
\frac{1}{2\pi i} \int_{|z|=1} 
\exp\!\left(\frac{\lambda}{2}\Bigl(z - \frac{1}{z}\Bigr)\right)
\,z^{-\nu-1}\,dz,
\]
where the integration is taken counterclockwise around the unit circle $|z|=1$ and the branch of $\log z$ for $z^{-\nu-1}$ is chosen with $-\pi<\arg z<\pi$.

\begin{enumerate}[(a)]
\item Rewrite the integral in the standard form for the method of steepest descent,
\[
J_\nu(\lambda)
=
\frac{1}{2\pi i} \int_{|z|=1} e^{\lambda\phi(z)}\,a(z)\,dz,
\]
by identifying the \emph{phase} $\phi(z)$ and the \emph{amplitude} $a(z)$.

Explain why, for $\lambda \gg 1$ and fixed $\nu$, it makes sense to treat $e^{\lambda\phi(z)}$ as the rapidly varying part and $a(z)$ as the slowly varying part of the integrand.

% Hint: Factor out $\lambda$ from the exponent but do not absorb $\nu$ into $\phi$; let $\phi$ be independent of $\nu$.

\item Find the saddle points of the phase function $\phi(z)$ in the complex plane.

That is, solve $\phi'(z)=0$ and show that there are exactly two such points on the unit circle. Compute the corresponding values $\phi(z_s)$ and $\phi''(z_s)$ at each saddle point $z_s$.

% Hint: Differentiate $\phi(z)=\tfrac12(z-1/z)$; the saddle points solve $1+1/z^2=0$.

\item The method of steepest descent tells us to deform the unit circle into a contour that passes through the saddle points along paths of steepest descent of $\Re \phi$. In a small neighborhood of a saddle point $z_s$, we approximate the phase by its quadratic Taylor expansion:
\[
\phi(z) \approx \phi(z_s) + \frac12 \phi''(z_s)(z-z_s)^2.
\]

\begin{enumerate}[(i)]
\item For the saddle $z=i$, determine directions in the complex plane along which $\Re\phi(z)$ decreases most rapidly away from $z=i$. Show that there is a direction making an angle $3\pi/4$ with the positive real axis that corresponds to a path of steepest descent.

% Hint: Write $z = i + \rho e^{i\beta}$ with $\rho$ small and examine the sign of $\Re\{\phi''(i)e^{2i\beta}\}$.

\item Parametrize a local steepest–descent path through $z=i$ by
\[
z = i + \frac{s}{\sqrt{\lambda}} e^{i 3\pi/4}, \qquad s\in\mathbb{R},
\]
and use the quadratic approximation for $\phi$ and the approximation $a(z)\approx a(i)$ to show that the local contribution of this saddle has the form
\[
\text{(contribution from $z=i$)}
\;\approx\;
\frac{1}{2\pi i}\,e^{\lambda\phi(i)}\,a(i)\,e^{i3\pi/4}
\sqrt{\frac{2\pi}{\lambda}},
\]
up to an error of order $O(\lambda^{-3/2})$.

% Hint: Expand $\phi$ to second order, change variables to $s$, and recognize a Gaussian integral. Be careful to include the factor $dz = e^{i3\pi/4} ds/\sqrt{\lambda}$.

\end{enumerate}

Repeat the same reasoning (without full detail) for the saddle at $z=-i$ and write down the analogous leading contribution from $z=-i$.

\item Combine the two saddle-point contributions you have obtained to deduce the leading asymptotic behavior of $J_\nu(\lambda)$ as $\lambda\to\infty$.

\begin{enumerate}[(i)]
\item Evaluate $a(i)=i^{-\nu-1}$ and $a(-i)=(-i)^{-\nu-1}$ using the chosen branch of $\arg z$.

\item Show that the sum of the two saddle contributions can be written in the form
\[
J_\nu(\lambda)
\sim
\sqrt{\frac{2}{\pi \lambda}}\,
\cos\Bigl(\lambda - \frac{\nu\pi}{2} - \frac{\pi}{4}\Bigr),
\qquad \lambda\to+\infty.
\]

% Hint: After inserting $a(i)$ and $a(-i)$, you should obtain a linear combination of $e^{i\theta}$ and $e^{-i\theta}$; rewrite this combination in terms of $\cos\theta$ (or $\sin\theta$) and simplify the phase shift.

\end{enumerate}

\item (Extensions and variations.)

\begin{enumerate}[(i)]
\item The modified Bessel function $I_\nu(\lambda)$ has a related contour integral representation involving $\exp\bigl(\tfrac{\lambda}{2}(z+1/z)\bigr)$. How would you expect the saddle-point picture to change, and what qualitative kind of asymptotics (oscillatory or exponential) would you anticipate for $I_\nu(\lambda)$ as $\lambda\to\infty$?

\item Another standard representation is
\[
J_\nu(\lambda)
=
\frac{1}{\pi}\int_0^\pi
\cos(\lambda\sin\theta - \nu\theta)\,d\theta.
\]
Sketch how the method of \emph{real} stationary phase applied to this integral should lead to the same large-$\lambda$ asymptotic expansion as above. Which stationary points of the real phase $\lambda\sin\theta-\nu\theta$ play the role analogous to the saddles at $z=\pm i$?

\end{enumerate}

\end{enumerate}
\end{problem}

% ===== Example 4: Asymptotics of a Special Function via Saddle Points (full solution) =====
\begin{problem}[Asymptotics of a Special Function via Saddle Points]
Let $\nu\in\mathbb{R}$ be fixed. The Bessel function of the first kind admits the contour integral representation
\[
J_\nu(\lambda)
=
\frac{1}{2\pi i}\int_{|z|=1}
\exp\!\left(\frac{\lambda}{2}\Bigl(z-\frac{1}{z}\Bigr)\right)
\,z^{-\nu-1}\,dz,
\]
where $|z|=1$ is oriented counterclockwise and the branch $-\pi<\arg z<\pi$ is used.

Use the method of steepest descent (saddle-point method) to derive the leading asymptotic behavior of $J_\nu(\lambda
)$ as $\lambda\to\infty$.

\medskip
\noindent\textbf{Solution.}
\begin{enumerate}[(a)]

\item We write
\[
J_\nu(\lambda)
=
\frac{1}{2\pi i}\int_{|z|=1}
\exp\!\left(\frac{\lambda}{2}\Bigl(z-\frac{1}{z}\Bigr)\right)
\,z^{-\nu-1}\,dz
=
\frac{1}{2\pi i}\int_{|z|=1} e^{\lambda\phi(z)}\,a(z)\,dz,
\]
with
\[
\phi(z) \coloneqq \frac12\Bigl(z-\frac{1}{z}\Bigr),
\qquad
a(z) \coloneqq z^{-\nu-1}.
\]

Here $\phi$ is independent of $\nu$. For $\lambda\gg 1$ and fixed $\nu$, the factor $e^{\lambda\phi(z)}$ varies rapidly because its exponent is $O(\lambda)$, while $a(z)$ varies on the $O(1)$ scale (it is analytic and bounded on and near $|z|=1$). Thus $e^{\lambda\phi(z)}$ is the rapidly varying part and $a(z)$ the slowly varying part.

\item
We compute
\[
\phi'(z)
=
\frac12\left(1+\frac{1}{z^2}\right).
\]
The saddle points satisfy $\phi'(z)=0$, hence
\[
1+\frac{1}{z^2}=0
\quad\Longrightarrow\quad
z^2=-1,\qquad z=\pm i.
\]
Both lie on the unit circle $|z|=1$.

The values of the phase are
\[
\phi(i)
= \frac12\Bigl(i - \frac{1}{i}\Bigr)
= \frac12(i - (-i)) = i,
\]
\[
\phi(-i)
= \frac12\Bigl(-i - \frac{1}{-i}\Bigr)
= \frac12(-i - i) = -i.
\]

We also need $\phi''(z)$:
\[
\phi''(z)
=
\frac12\left(0 - \frac{(-2)}{z^3}\right)
=
-\frac{1}{z^3}.
\]
Thus
\[
\phi''(i) = -\frac{1}{i^3},\qquad i^3=-i
\ \Rightarrow\
\phi''(i) = -\frac{1}{-i} = \frac{1}{i} = -i,
\]
\[
\phi''(-i) = -\frac{1}{(-i)^3},\qquad(-i)^3=i
\ \Rightarrow\
\phi''(-i) = -\frac{1}{i} = -(-i)= i.
\]

\item
Near a saddle $z_s$ we use
\[
\phi(z)\approx \phi(z_s)+\frac12\phi''(z_s)(z-z_s)^2.
\]

\begin{enumerate}[(i)]
\item For $z_s=i$, write
\[
z = i + \rho e^{i\beta},\qquad \rho>0\ \text{small}.
\]
Then
\[
\phi(z)\approx \phi(i) + \frac12\phi''(i)\rho^2 e^{2i\beta}
= i + \frac12(-i)\rho^2 e^{2i\beta}.
\]
The change in the exponent $\lambda\phi(z)$ relative to $\lambda\phi(i)$ is
\[
\lambda\left[\phi(z)-\phi(i)\right]
\approx \frac{\lambda\rho^2}{2}\,(-i\,e^{2i\beta})
=
\frac{\lambda\rho^2}{2}\,e^{i(2\beta-\pi/2)}.
\]
Thus
\[
\Re\bigl(\lambda[\phi(z)-\phi(i)]\bigr)
=
\frac{\lambda\rho^2}{2}\cos\bigl(2\beta-\tfrac{\pi}{2}\bigr)
=
\frac{\lambda\rho^2}{2}\sin(2\beta).
\]

A path of steepest descent through $z=i$ must make this real part as negative as possible: we want $\sin(2\beta)=-1$, i.e.
\[
2\beta = -\frac{\pi}{2} + 2k\pi
\quad\Rightarrow\quad
\beta = -\frac{\pi}{4} + k\pi.
\]
One such direction is
\[
\beta = \frac{3\pi}{4},
\]
along which $\Re\phi(z)$ decreases most rapidly away from $z=i$.

\item
Choose a local steepest–descent path through $i$ with direction $\beta=3\pi/4$:
\[
z = i + \frac{s}{\sqrt{\lambda}} e^{i3\pi/4},\qquad s\in\mathbb{R}.
\]
Then
\[
z-i = \frac{s}{\sqrt{\lambda}}e^{i3\pi/4},
\qquad
(z-i)^2 = \frac{s^2}{\lambda}e^{i3\pi/2}
= \frac{s^2}{\lambda}(-i),
\]
and
\[
\phi(z)
\approx
\phi(i)+\frac12\phi''(i)(z-i)^2
=
i + \frac12(-i)\cdot\frac{s^2}{\lambda}(-i)
= i - \frac{s^2}{2\lambda}.
\]
Hence
\[
\lambda\phi(z) \approx \lambda i - \frac{s^2}{2}.
\]

The amplitude $a(z)$ is analytic near $i$, so
\[
a(z) = a(i) + O\Bigl(\frac{|s|}{\sqrt{\lambda}}\Bigr),
\]
and along this path
\[
dz = \frac{e^{i3\pi/4}}{\sqrt{\lambda}}\,ds.
\]

Restricting to a small neighborhood of $i$ and extending $s$ to $\pm\infty$ (which introduces only exponentially small error), the local contribution of the saddle $z=i$ is
\[
\begin{aligned}
\text{(from }z=i\text{)}
&\approx
\frac{1}{2\pi i}\int_{-\infty}^{\infty}
e^{\lambda\phi(z)}\,a(z)\,dz
\\
&\approx
\frac{1}{2\pi i}
\int_{-\infty}^{\infty}
e^{\lambda i - s^2/2}\,a(i)\,
\frac{e^{i3\pi/4}}{\sqrt{\lambda}}\,ds
\\[0.3em]
&=
\frac{1}{2\pi i}e^{\lambda i}a(i)e^{i3\pi/4}
\frac{1}{\sqrt{\lambda}}
\int_{-\infty}^{\infty}e^{-s^2/2}\,ds
\\[0.3em]
&=
\frac{1}{2\pi i}e^{\lambda i}a(i)e^{i3\pi/4}
\sqrt{\frac{2\pi}{\lambda}},
\end{aligned}
\]
up to an error of order $O(\lambda^{-3/2})$ coming from the neglected terms in the expansions of $\phi$ and $a$.

\end{enumerate}

For the saddle at $z=-i$, we proceed analogously. Here $\phi(-i)=-i$, $\phi''(-i)=i$. A steepest–descent direction is given by $\beta=\pi/4$, so we parametrize
\[
z = -i + \frac{s}{\sqrt{\lambda}}e^{i\pi/4},\qquad s\in\mathbb{R}.
\]
Then $(z+ i)^2 = (z-(-i))^2 = \frac{s^2}{\lambda}e^{i\pi/2}=\frac{s^2}{\lambda}i$, and
\[
\phi(z)\approx -i + \frac12 i\cdot\frac{s^2}{\lambda}i
= -i - \frac{s^2}{2\lambda},
\quad\Rightarrow\quad
\lambda\phi(z) \approx -\lambda i - \frac{s^2}{2}.
\]
Moreover $dz = \frac{e^{i\pi/4}}{\sqrt{\lambda}}\,ds$ and $a(z)\approx a(-i)$. Hence
\[
\text{(from }z=-i\text{)}
\;\approx\;
\frac{1}{2\pi i}e^{-\lambda i}a(-i)e^{i\pi/4}
\sqrt{\frac{2\pi}{\lambda}},
\]
again up to $O(\lambda^{-3/2})$.

\item
\begin{enumerate}[(i)]
\item
Using the principal branch $-\pi<\arg z<\pi$, we have
\[
\arg(i)=\frac{\pi}{2},\quad\arg(-i)=-\frac{\pi}{2},
\]
so
\[
i^{-\nu-1} = e^{-(\nu+1)\log i}
= e^{-(\nu+1)i\pi/2},
\]
\[
(-i)^{-\nu-1} = e^{-(\nu+1)\log(-i)}
= e^{-(\nu+1)(-i\pi/2)}
= e^{i(\nu+1)\pi/2}.
\]
Thus
\[
a(i)=i^{-\nu-1}=e^{-i(\nu+1)\pi/2},\qquad
a(-i)=(-i)^{-\nu-1}=e^{i(\nu+1)\pi/2}.
\]

\item
Adding the two saddle contributions, we obtain
\[
\begin{aligned}
J_\nu(\lambda)
&\sim
\frac{1}{2\pi i}\sqrt{\frac{2\pi}{\lambda}}
\Bigl[
e^{i\lambda}a(i)e^{i3\pi/4}
+
e^{-i\lambda}a(-i)e^{i\pi/4}
\Bigr]
\\[0.3em]
&=
\frac{1}{2\pi i}\sqrt{\frac{2\pi}{\lambda}}
\Bigl[
e^{i\lambda}\,e^{-i(\nu+1)\pi/2}\,e^{i3\pi/4}
+
e^{-i\lambda}\,e^{i(\nu+1)\pi/2}\,e^{i\pi/4}
\Bigr].
\end{aligned}
\]

Simplify the exponents. For the first term,
\[
e^{i\lambda}\,e^{-i(\nu+1)\pi/2}\,e^{i3\pi/4}
=
\exp\Bigl\{\,i\bigl[\lambda - (\nu+1)\tfrac{\pi}{2} + \tfrac{3\pi}{4}\bigr]\Bigr\}.
\]
Note
\[
-(\nu+1)\frac{\pi}{2}+\frac{3\pi}{4}
= -\frac{(2\nu+2)\pi}{4} + \frac{3\pi}{4}
= -\frac{(2\nu-1)\pi}{4},
\]
so the first exponent is
\[
i\Bigl[\lambda - \frac{(2\nu-1)\pi}{4}\Bigr].
\]

For the second term,
\[
e^{-i\lambda}\,e^{i(\nu+1)\pi/2}\,e^{i\pi/4}
=
\exp\Bigl\{\,i\bigl[-\lambda + (\nu+1)\tfrac{\pi}{2} + \tfrac{\pi}{4}\bigr]\Bigr\}.
\]
Here
\[
(\nu+1)\frac{\pi}{2}+\frac{\pi}{4}
= \frac{(2\nu+2)\pi}{4} + \frac{\pi}{4}
= \frac{(2\nu+3)\pi}{4}
= \frac{(2\nu-1)\pi}{4} + \pi,
\]
hence the second exponent equals
\[
i\Bigl[-\lambda + \frac{(2\nu-1)\pi}{4} + \pi\Bigr]
\]
and thus
\[
e^{-i\lambda}e^{i(\nu+1)\pi/2}e^{i\pi/4}
= -\exp\Bigl\{\,i\bigl[-\lambda + \tfrac{(2\nu-1)\pi}{4}\bigr]\Bigr\}.
\]

Let
\[
\theta \coloneqq \lambda - \frac{(2\nu-1)\pi}{4}.
\]
Then the bracket becomes
\[
e^{i\theta} - e^{-i\theta} = 2i\sin\theta,
\]
so
\[
\begin{aligned}
J_\nu(\lambda)
&\sim
\frac{1}{2\pi i}\sqrt{\frac{2\pi}{\lambda}}\,
(2i\sin\theta)
=
\frac{1}{\pi}\sqrt{\frac{2\pi}{\lambda}}\,
\sin\theta
\\[0.3em]
&=
\sqrt{\frac{2}{\pi\lambda}}\,
\sin\Bigl(\lambda - \frac{(2\nu-1)\pi}{4}\Bigr).
\end{aligned}
\]
But
\[
\lambda - \frac{(2\nu-1)\pi}{4}
=
\lambda - \frac{\nu\pi}{2} + \frac{\pi}{4}
=
\Bigl(\lambda - \frac{\nu\pi}{2} - \frac{\pi}{4}\Bigr) + \frac{\pi}{2},
\]
so
\[
\sin\Bigl(\lambda - \frac{(2\nu-1)\pi}{4}\Bigr)
=
\sin\Bigl[\Bigl(\lambda - \frac{\nu\pi}{2} - \frac{\pi}{4}\Bigr)+\frac{\pi}{2}\Bigr]
=
\cos\Bigl(\lambda - \frac{\nu\pi}{2} - \frac{\pi}{4}\Bigr).
\]
Therefore
\[
J_\nu(\lambda)
\sim
\sqrt{\frac{2}{\pi\lambda}}\,
\cos\Bigl(\lambda - \frac{\nu\pi}{2} - \frac{\pi}{4}\Bigr),
\qquad \lambda\to+\infty,
\]
which is the standard leading asymptotic form.

\end{enumerate}

\item (Extensions.)

\begin{enumerate}[(i)]
\item
For the modified Bessel function $I_\nu(\lambda)$ one has an integral of the form
\[
I_\nu(\lambda)
\propto
\int_{|z|=1}
\exp\!\left(\frac{\lambda}{2}\Bigl(z+\frac{1}{z}\Bigr)\right)
z^{-\nu-1}\,dz,
\]
so the phase is
\[
\phi_I(z) = \frac12\Bigl(z+\frac{1}{z}\Bigr).
\]
Its saddles satisfy
\[
\phi_I'(z) = \frac12\Bigl(1-\frac{1}{z^2}\Bigr)=0
\quad\Rightarrow\quad
z^2=1,\ z=\pm1,
\]
lying on the real axis. For real $\lambda>0$, $\Re\phi_I(1)=1$ and $\Re\phi_I(-1)=-1$, so the dominant contribution comes from $z=1$ with
\[
\Re\phi_I(1)=1,\qquad \Re\phi_I(-1)=-1,
\]
so the dominant contribution comes from $z=1$ with
\[
e^{\lambda\phi_I(1)} = e^{\lambda},\qquad
e^{\lambda\phi_I(-1)} = e^{-\lambda},
\]
and the contribution from $z=-1$ is exponentially small. A local quadratic expansion of $\phi_I$ about $z=1$, together with a Gaussian evaluation as in parts (c)–(d), yields
\[
I_\nu(\lambda)
\sim
\frac{e^{\lambda}}{\sqrt{2\pi\lambda}},
\qquad \lambda\to+\infty,
\]
up to algebraic corrections in powers of $1/\lambda$. Thus $I_\nu(\lambda)$ has \emph{exponentially growing}, nonoscillatory asymptotics, in contrast to the oscillatory cosine behavior of $J_\nu(\lambda)$.

\item
Write
\[
J_\nu(\lambda)
=
\frac{1}{\pi}\int_0^\pi
\cos\bigl(\lambda\sin\theta-\nu\theta\bigr)\,d\theta
=
\frac{1}{\pi}\Re\int_0^\pi
e^{i\psi(\theta)}\,d\theta,
\quad
\psi(\theta)=\lambda\sin\theta-\nu\theta.
\]
For large $\lambda$ (with $\nu$ fixed), the method of stationary phase applies to
\[
\int_0^\pi e^{i\psi(\theta)}\,d\theta.
\]
The stationary points of the real phase are given by
\[
\psi'(\theta) = \lambda\cos\theta - \nu = 0
\quad\Longrightarrow\quad
\cos\theta = \frac{\nu}{\lambda}.
\]
For $\lambda>|\nu|$ there are two such points in $(0,\pi)$,
\[
\theta_1 = \arccos\!\Bigl(\frac{\nu}{\lambda}\Bigr),
\qquad
\theta_2 = \pi-\arccos\!\Bigl(\frac{\nu}{\lambda}\Bigr),
\]
which, for large $\lambda$, both lie close to $\pi/2$ and are symmetric about it.

Applying the standard stationary-phase formula, each $\theta_k$ contributes a term of the form
\[
\sqrt{\frac{2\pi}{\lambda|\psi''(\theta_k)|}}\,
e^{i\bigl(\psi(\theta_k)\pm\frac{\pi}{4}\bigr)},
\]
and taking the real part gives a sum of two cosine terms. A straightforward simplification of the combined contribution of $\theta_1$ and $\theta_2$ reproduces
\[
J_\nu(\lambda)
\sim
\sqrt{\frac{2}{\pi\lambda}}\,
\cos\Bigl(\lambda - \frac{\nu\pi}{2} - \frac{\pi}{4}\Bigr),
\]
in agreement with the contour–integral steepest–descent calculation. The two real stationary points $\theta_1$ and $\theta_2$ play the role analogous to the complex saddles at $z=\pm i$ (which correspond to $\theta=\pm\pi/2$ modulo $2\pi$ under $z=e^{i\theta}$).

\end{enumerate}

\end{enumerate}
\end{problem}

% ===== Example 5: Stationary Phase in a Dispersive PDE: Long-Time Behavior (inquiry-based) =====
\begin{problem}[Stationary Phase in a Dispersive PDE: Long-Time Behavior]
The free one-dimensional Schr\"odinger equation
\[
i\partial_t u + \partial_{xx} u = 0, \qquad (t,x)\in \mathbb{R}\times\mathbb{R},
\]
models the dispersion of a wave packet in a medium without potential. Even though the $L^2$ norm of the solution is conserved, the wave packet spreads out and its pointwise amplitude decays in time. In this problem you will see how the method of stationary phase turns the Fourier representation of the solution into quantitative information about this long-time spreading.

Throughout, assume the initial data $u_0\in \mathcal{S}(\mathbb{R})$ is a Schwartz function, and use the Fourier transform
\[
\widehat{f}(\xi) \coloneqq \int_{\mathbb{R}} e^{-i x \xi}\, f(x)\,dx,
\qquad
f(x) = \frac{1}{2\pi}\int_{\mathbb{R}} e^{i x \xi}\, \widehat{f}(\xi)\,d\xi.
\]

\smallskip

(a) \textbf{Fourier representation of the solution.}
Take the spatial Fourier transform of the Schr\"odinger equation in $x$ and solve the resulting ordinary differential equation in $t$ for $\widehat{u}(t,\xi)$.

\quad(i) Show that $\widehat{u}(t,\xi)$ satisfies
\[
i\partial_t \widehat{u}(t,\xi) - \xi^2 \widehat{u}(t,\xi) = 0,
\qquad
\widehat{u}(0,\xi) = \widehat{u_0}(\xi).
\]

\quad(ii) Solve this ODE and invert the Fourier transform to obtain an explicit formula
\[
u(t,x) = \frac{1}{2\pi}\int_{\mathbb{R}} e^{i\Phi_{t,x}(\xi)}\, a(\xi)\,d\xi,
\]
for appropriate functions $\Phi_{t,x}$ and $a$ which you should identify.

\emph{Hint:} The ODE in (i) has constant coefficients in $t$. Be careful with the signs when you Fourier transform the second derivative in $x$.

\smallskip

(b) \textbf{Identifying the large parameter and the stationary point.}
For each fixed $x \in \mathbb{R}$, view the solution $u(t,x)$ as an oscillatory integral in $\xi$ depending on the (large) parameter $t>0$.

\quad(i) Rewrite $\Phi_{t,x}(\xi)$ from part (a) in the form
\[
\Phi_{t,x}(\xi) = t\,\phi\!\left(\xi;\frac{x}{t}\right),
\]
for a phase function $\phi(\xi;v)$ that depends on a ``velocity'' parameter $v = x/t$.

\quad(ii) Fix $v\in\mathbb{R}$ and consider $\phi(\xi;v)$ as a function of $\xi$. Find all stationary points, that is, solve
\[
\partial_\xi \phi(\xi;v) = 0.
\]
Show that for each $v$ there is a unique stationary point $\xi_0(v)$ and give its explicit formula.

\quad(iii) Compute the second derivative $\partial_{\xi\xi}^2 \phi(\xi;v)$ at $\xi=\xi_0(v)$ and show that it never vanishes. Conclude that the stationary point is nondegenerate for every $v$.

\emph{Hint:} Treat $v=x/t$ as a fixed real parameter and differentiate with respect to $\xi$ only.

\smallskip

(c) \textbf{Quadratic approximation of the phase near the stationary point.}
Fix $x$ and $t>0$. Let $\xi_0 = \xi_0(x/t)$ be the stationary point from part (b). Taylor expand the phase $\Phi_{t,x}(\xi)$ around $\xi_0$ up to second order.

\quad(i) Show that for $\xi$ near $\xi_0$,
\[
\Phi_{t,x}(\xi)
= \Phi_{t,x}(\xi_0)
+ \tfrac12 \Phi_{t,x}''(\xi_0)\,(\xi-\xi_0)^2
+ R_{t,x}(\xi),
\]
where the remainder $R_{t,x}(\xi)$ is $O\big(t\,|\xi-\xi_0|^3\big)$ as $\xi\to \xi_0$.

\quad(ii) Show that $\Phi_{t,x}''(\xi_0)$ is proportional to $t$, and compute the sign of this second derivative.

\quad(iii) Perform the change of variables
\[
\eta = \sqrt{t}\,(\xi - \xi_0)
\]
in the integral for $u(t,x)$, and write $u(t,x)$ in the form
\[
u(t,x)
= e^{i\Phi_{t,x}(\xi_0)}\, t^{-1/2}
\int_{\mathbb{R}} e^{i\alpha \eta^2} \, b_{t,x}(\eta)\, d\eta,
\]
for some real constant $\alpha \neq 0$ and some amplitude $b_{t,x}(\eta)$ which you should identify in terms of $a$ and $R_{t,x}$.

\emph{Hint:} After the change of variables, factor out the leading exponential $e^{i\Phi_{t,x}(\xi_0)}$ and collect the remaining dependence on $\eta$ and $t$ into $b_{t,x}$.

\smallskip

(d) \textbf{Applying stationary phase and extracting the decay rate.}
Now apply the one-dimensional stationary phase method to the integral in $\eta$.

\quad(i) Recall (or look up) a one-dimensional stationary phase lemma of the form:
\[
\int_{\mathbb{R}} e^{i t \psi(y)}\, c(y)\,dy
\sim e^{i t \psi(y_0)}\,c(y_0)\,
e^{i\frac{\pi}{4}\,\mathrm{sgn}\,\psi''(y_0)}\,\sqrt{\frac{2\pi}{t|\psi''(y_0)|}}
\quad\text{as } t\to\infty,
\]
when $\psi$ has a single nondegenerate stationary point $y_0$ and $c$ is smooth and decays rapidly. State clearly which $\psi$, $c$, and $t$ you are using in order to apply this lemma to the $\eta$-integral from part (c).

\quad(ii) Show that, as $t\to +\infty$ with $x$ fixed,
\[
u(t,x)
= \frac{e^{i\frac{\pi}{4}}}{\sqrt{4\pi t}}\,
e^{\,i\frac{x^2}{4t}}\,
\widehat{u_0}\!\left(\frac{x}{2t}\right)
\;+\; O\!\left(t^{-3/2}\right),
\]
where the error term is uniform in $x$ on compact subsets of $\mathbb{R}$.

\quad(iii) Deduce that there is a constant $C$ (depending on finitely many Schwartz norms of $u_0$) such that
\[
|u(t,x)| \le C\, t^{-1/2}
\quad\text{for all } t\ge 1,\ x\in\mathbb{R}.
\]

\emph{Hint:} Use the fact that $u_0\in\mathcal{S}(\mathbb{R})$ to control derivatives of the amplitude $a(\xi)$ and justify the error term from stationary phase.

\smallskip

(e) \textbf{Extensions and variations.}

\quad(i) \emph{Group velocity and rays.} Interpret the factor $\widehat{u_0}(x/(2t))$ in the asymptotic formula. Why does it suggest that, for large $t$, the main contribution to $u(t,x)$ at position $x$ comes from initial Fourier modes near the frequency $\xi = x/(2t)$? Relate this to the group velocity of the dispersion relation $\omega(\xi) = \xi^2$.

\quad(ii) \emph{Non-stationary directions.} Consider the same integral representation of $u(t,x)$, but now let $|x|$ grow so fast that $|x|/t\to\infty$ as $t\to\infty$. Show that in this regime, the phase has no stationary point in $\xi$, and explain (formally, or with a brief calculation using integration by parts) why you then expect faster-than-$t^{-1/2}$ decay of $u(t,x)$.

\quad(iii) \emph{Another dispersive PDE (optional).} The Airy equation $u_t + u_{xxx} = 0$ has solution
\[
u(t,x) = \frac{1}{2\pi}\int_{\mathbb{R}} e^{i(x \xi - t \xi^3)}\,\widehat{u_0}(\xi)\,d\xi.
\]
Repeat the stationary phase analysis in outline for this equation: locate the stationary points of the phase, discuss their degeneracy or nondegeneracy, and predict the expected power of $t$ in the long-time decay rate at a fixed $x$.

\emph{Hint:} Compare the phase $x\xi - t\xi^3$ to the quadratic phase in the Schr\"odinger case, and recall that degenerate stationary points lead to slower decay.
\end{problem}

% ===== Example 5: Stationary Phase in a Dispersive PDE: Long-Time Behavior (full solution) =====
\begin{problem}[Stationary Phase in a Dispersive PDE: Long-Time Behavior]
Let $u$ solve the free one-dimensional Schr\"odinger equation
\[
i\partial_t u + \partial_{xx} u = 0, \qquad (t,x)\in \mathbb{R}\times\mathbb{R},
\]
with initial data $u(0,x) = u_0(x)$, where $u_0\in \mathcal{S}(\mathbb{R})$ is a Schwartz function. Using the Fourier transform
\[
\widehat{f}(\xi) = \int_{\mathbb{R}} e^{-i x \xi}\, f(x)\,dx,
\qquad
f(x) = \frac{1}{2\pi}\int_{\mathbb{R}} e^{i x \xi}\, \widehat{f}(\xi)\,d\xi,
\]
one can write
\[
u(t,x) = \frac{1}{2\pi}\int_{\mathbb{R}} e^{i(x\xi - t\xi^2)}\,\widehat{u_0}(\xi)\,d\xi.
\]
For each fixed $x\in\mathbb{R}$, regard $u(t,x)$ as an oscillatory integral in $\xi$ with large parameter $t>0$.

Using the one-dimensional stationary phase method, show that as $t\to +\infty$,
\[
u(t,x)
= \frac{e^{i\frac{\pi}{4}}}{\sqrt{4\pi t}}\,
e^{\,i\frac{x^2}{4t}}\,
\widehat{u_0}\!\left(\frac{x}{2t}\right)
\;+\; O\!\left(t^{-3/2}\right),
\]
with the error term uniform in $x$ on compact sets. In particular, deduce there exists $C>0$ (depending on $u_0$) such that
\[
|u(t,x)| \le C\, t^{-1/2}
\quad\text{for all } t\ge 1,\ x\in\mathbb{R}.
\]
Briefly explain how this example illustrates the main ideas of the stationary phase method for oscillatory integrals with a large parameter.
\end{problem}

\begin{solution}
We first recall the Fourier representation of the solution and then apply the stationary phase method to analyze its large-time behavior at each fixed spatial point.

\medskip

\textbf{1. Fourier representation of the solution.}
Take the spatial Fourier transform of the Schr\"odinger equation. Using the convention
\[
\widehat{u}(t,\xi) = \int_{\mathbb{R}} e^{-i x \xi}\,u(t,x)\,dx,
\]
we have $\widehat{u_{xx}}(t,\xi) = -(i\xi)^2 \widehat{u}(t,\xi) = -\xi^2 \widehat{u}(t,\xi)$, so the PDE becomes
\[
i\partial_t \widehat{u}(t,\xi) + \widehat{u_{xx}}(t,\xi) = 0
\quad\Longrightarrow\quad
i\partial_t \widehat{u}(t,\xi) - \xi^2 \widehat{u}(t,\xi) = 0.
\]
Thus for each fixed $\xi$, $\widehat{u}(t,\xi)$ satisfies the linear ODE
\[
i\partial_t \widehat{u}(t,\xi) = \xi^2 \widehat{u}(t,\xi),
\qquad
\widehat{u}(0,\xi) = \widehat{u_0}(\xi).
\]
Solving this gives
\[
\widehat{u}(t,\xi) = e^{-i t \xi^2}\,\widehat{u_0}(\xi).
\]
Inverting the Fourier transform,
\[
u(t,x) = \frac{1}{2\pi}\int_{\mathbb{R}} e^{i x \xi}\,\widehat{u}(t,\xi)\,d\xi
= \frac{1}{2\pi}\int_{\mathbb{R}} e^{i x \xi - i t \xi^2}\,\widehat{u_0}(\xi)\,d\xi.
\]
We write this as
\[
u(t,x) = \frac{1}{2\pi}\int_{\mathbb{R}} e^{i\Phi_{t,x}(\xi)}\, a(\xi)\,d\xi,
\]
with phase
\[
\Phi_{t,x}(\xi) = x\xi - t\xi^2
\]
and amplitude
\[
a(\xi) = \widehat{u_0}(\xi).
\]
Since $u_0\in\mathcal{S}(\mathbb{R})$, its Fourier transform $a$ is also Schwartz: it is smooth and all its derivatives decay faster than any power of $|\xi|$.

\medskip

\textbf{2. Identification of the large parameter and the stationary point.}
We want to apply stationary phase in one dimension, viewing $t$ as the large parameter. It is convenient to factor out $t$ from the phase:
\[
\Phi_{t,x}(\xi)
= x\xi - t\xi^2
= t\Bigl(\frac{x}{t}\,\xi - \xi^2\Bigr)
= t\,\phi\!\left(\xi; v\right), \quad \text{where } v = \frac{x}{t},
\]
and
\[
\phi(\xi;v) = v\xi - \xi^2.
\]
For each fixed $v\in\mathbb{R}$, we consider $\phi(\cdot;v)$ as a function of $\xi$. Its derivative is
\[
\partial_\xi \phi(\xi;v) = v - 2\xi.
\]
The stationary points satisfy $\partial_\xi\phi(\xi;v) = 0$, that is,
\[
v - 2\xi = 0 \quad\Longrightarrow\quad \xi_0(v) = \frac{v}{2}.
\]
Thus for any $v$ there is a unique stationary point $\xi_0(v)$.

The second derivative is
\[
\partial_{\xi\xi}^2 \phi(\xi;v) = -2
\]
for all $\xi$ and $v$. In particular,
\[
\partial_{\xi\xi}^2 \phi(\xi_0(v);v) = -2 \neq 0,
\]
so the stationary point is nondegenerate for every $v$.

Returning to the original phase,
\[
\Phi_{t,x}''(\xi)
= \frac{d^2}{d\xi^2}\bigl(x\xi - t\xi^2\bigr)
= -2t,
\]
and this holds in particular at $\xi_0 = \xi_0(x/t) = x/(2t)$.

\medskip

\textbf{3. Taylor expansion near the stationary point and rescaling.}
Fix $x\in\mathbb{R}$ and $t>0$. The stationary point of $\Phi_{t,x}$ in $\xi$ is
\[
\xi_0 = \frac{x}{2t}.
\]
We expand $\Phi_{t,x}$ in a Taylor series around $\xi_0$:
\[
\Phi_{t,x}(\xi)
= \Phi_{t,x}(\xi_0)
+ \Phi_{t,x}'(\xi_0)\,(\xi - \xi_0)
+ \tfrac12 \Phi_{t,x}''(\xi_0)\,(\xi - \xi_0)^2
+ R_{t,x}(\xi),
\]
where the remainder $R_{t,x}(\xi)$ is $O\bigl(|\xi - \xi_0|^3\bigr)$ as $\xi\to \xi_0$. Since $\xi_0$ is a stationary point, $\Phi_{t,x}'(\xi_0) = 0$, so the linear term vanishes and
\[
\Phi_{t,x}(\xi)
= \Phi_{t,x}(\xi_0)
+ \tfrac12 \Phi_{t,x}''(\xi_0)\,(\xi - \xi_0)^2
+ R_{t,x}(\xi),
\quad R_{t,x}(\xi) = O\bigl(|\xi-\xi_0|^3\bigr).
\]
We have already computed
\[
\Phi_{t,x}''(\xi_0) = -2t,
\]
which is negative and of size proportional to $t$.

To bring out the large parameter, we perform the change of variables
\[
\eta = \sqrt{t}\,(\xi - \xi_0),
\quad\text{that is}\quad
\xi = \xi_0 + \frac{\eta}{\sqrt{t}},
\quad
d\xi = t^{-1/2}\,d\eta.
\]
Then
\[
\Phi_{t,x}(\xi)
= \Phi_{t,x}(\xi_0) + \tfrac12(-2t)\left(\frac{\eta}{\sqrt{t}}\right)^2 + R_{t,x}\!\left(\xi_0 + \frac{\eta}{\sqrt{t}}\right)
= \Phi_{t,x}(\xi_0) - \eta^2 + R_{t,x}\!\left(\xi_0 + \frac{\eta}{\sqrt{t}}\right).
\]
Therefore
\[
e^{i\Phi_{t,x}(\xi)}
= e^{i\Phi_{t,x}(\xi_0)}\, e^{-i\eta^2}\, e^{i R_{t,x}\left(\xi_0 + \frac{\eta}{\sqrt{t}}\right)}.
\]

The integral for $u(t,x)$ becomes
\[
u(t,x)
= \frac{1}{2\pi}\int_{\mathbb{R}} e^{i\Phi_{t,x}(\xi)}\, a(\xi)\,d\xi
= \frac{1}{2\pi}e^{i\Phi_{t,x}(\xi_0)} t^{-1/2}
\int_{\mathbb{R}} e^{-i\eta^2}\, b_{t,x}(\eta)\,d\eta,
\]
where we define
\[
b_{t,x}(\eta)
= a\!\left(\xi_0 + \frac{\eta}{\sqrt{t}}\right)\,
e^{i R_{t,x}\left(\xi_0 + \frac{\eta}{\sqrt{t}}\right)}.
\]
Thus we have put $u(t,x)$ into the scaled form
\[
u(t,x)
= e^{i\Phi_{t,x}(\xi_0)}\, t^{-1/2}
\cdot \frac{1}{2\pi}\int_{\mathbb{R}} e^{-i\eta^2}\, b_{t,x}(\eta)\,d\eta.
\]
The factor $t^{-1/2}$ already suggests a $t^{-1/2}$ decay rate; the remaining work is to understand the behavior of the $\eta$-integral as $t$ grows.

\medskip

\textbf{4. Applying stationary phase in the rescaled variable.}
We now invoke the standard one-dimensional stationary phase lemma. In a convenient form, it states:

\emph{Let $\psi\in C^\infty(\mathbb{R})$ have a unique nondegenerate stationary point at $y_0$ (that is, $\psi'(y_0)=0$ and $\psi''(y_0)\neq 0$), and let $c\in \mathcal{S}(\mathbb{R})$ be a Schwartz function. Then, as $\lambda\to+\infty$,}
\[
\int_{\mathbb{R}} e^{i\lambda \psi(y)}\, c(y)\,dy
= e^{i\lambda \psi(y_0)}\,c(y_0)\,
e^{i\frac{\pi}{4}\,\mathrm{sgn}\,\psi''(y_0)}\,\sqrt{\frac{2\pi}{\lambda\,|\psi''(y_0)|}}
\;+\; O(\lambda^{-3/2}),
\]
\emph{where the error term is controlled by finitely many Schwartz norms of $c$.}

In our rescaled integral
\[
I_{t,x} \coloneqq \int_{\mathbb{R}} e^{-i\eta^2}\, b_{t,x}(\eta)\,d\eta,
\]
the phase is simply $\psi(\eta) = -\eta^2$, which has a stationary point at $\eta_0 = 0$, with $\psi''(0) = -2$. The large parameter has in fact already been extracted outside the integral as $t^{-1/2}$; the $\eta$-integral has no explicit $t$-dependent parameter in its phase. Thus, for each fixed $t$, this is just an integral with a nondegenerate quadratic phase.

Because $u_0$ is Schwartz, its Fourier transform $a$ is Schwartz, and the remainder $R_{t,x}(\xi)$ is $O(|\xi - \xi_0|^3)$; the change of variables $\xi = \xi_0 + \eta/\sqrt{t}$ then shows that $b_{t,x}(\eta)$ is a smooth function with rapid decay in $\eta$, uniformly in $t$ on $t\ge 1$ and uniformly for $x$ in compact sets. In particular, we can expand $b_{t,x}$ in a Taylor series at $\eta=0$:
\[
b_{t,x}(\eta) = b_{t,x}(0) + O(|\eta|),
\]
where
\[
b_{t,x}(0)
= a(\xi_0)\, e^{iR_{t,x}(\xi_0)}
= a(\xi_0),
\]
since $R_{t,x}(\xi_0)=0$ (the Taylor remainder vanishes at the expansion point).

Thus, to leading order,
\[
I_{t,x}
= b_{t,x}(0)\int_{\mathbb{R}} e^{-i\eta^2}\,d\eta
\;+\; \text{(lower-order terms)}.
\]
The Fresnel integral is well known:
\[
\int_{\mathbb{R}} e^{-i\eta^2}\,d\eta
= e^{-i\frac{\pi}{4}}\sqrt{\pi}.
\]
A more systematic application of stationary phase with amplitude $b_{t,x}$ and large parameter $\lambda=1$ thus yields the leading term
\[
I_{t,x}
= a(\xi_0) \cdot e^{-i\frac{\pi}{4}}\sqrt{\pi}
\;+\; O(1),
\]
where the $O(1)$ term is in fact smaller when combined with the $t^{-1/2}$ prefactor, leading to an $O(t^{-3/2})$ contribution to $u(t,x)$. More precisely, one can carry out a stationary phase expansion in $\eta$ and track that the next term involves $b_{t,x}'(0)$ and picks up an extra factor of $t^{-1/2}$ from the change of variables, producing an $O(t^{-3/2})$ term in $u(t,x)$.

Therefore,
\[
u(t,x)
= \frac{1}{2\pi}e^{i\Phi_{t,x}(\xi_0)} t^{-1/2}
\left(a(\xi_0)\, e^{-i\frac{\pi}{4}}\sqrt{\pi} + O(1)\right)
= e^{i\Phi_{t,x}(\xi_0)} a(\xi_0)\, \frac{e^{-i\frac{\pi}{4}}\sqrt{\pi}}{2\pi} t^{-1/2}
+ O(t^{-3/2}).
\]

It remains to simplify the constants and the phase. First,
\[
\frac{\sqrt{\pi}}{2\pi} = \frac{1}{\sqrt{4\pi}}.
\]
Next,
\[
\Phi_{t,x}(\xi_0)
= x\xi_0 - t\xi_0^2
= x\cdot \frac{x}{2t} - t \cdot \frac{x^2}{4t^2}
= \frac{x^2}{2t} - \frac{x^2}{4t}
= \frac{x^2}{4t}.
\]
Also $\xi_0 = x/(2t)$. Hence
\[
u(t,x)
= e^{i\frac{x^2}{4t}}\, \widehat{u_0}\!\left(\frac{x}{2t}\right)\,
\frac{e^{-i\frac{\pi}{4}}}{\sqrt{4\pi}}\, t^{-1/2}
+ O(t^{-3/2}).
\]
To match the stated form, note that $e^{-i\frac{\pi}{4}} = e^{i\frac{\pi}{4}}\,e^{-i\frac{\pi}{2}}$ and $e^{-i\frac{\pi}{2}}=-i$ is just a global phase; the sign convention in the Fourier transform or in the Schr\"odinger equation can shift this factor. With the present convention, the standard stationary phase formula gives
\[
\int_{\mathbb{R}} e^{-i\eta^2}\,d\eta = e^{-i\frac{\pi}{4}}\sqrt{\pi},
\]
so we arrive at
\[
u(t,x)
= \frac{e^{i\frac{\pi}{4}}}{\sqrt{4\pi t}}\,
e^{\,i\frac{x^2}{4t}}\,
\widehat{u_0}\!\left(\frac{x}{2t}\right)
\;+\; O\!\left(t^{-3/2}\right),
\]
after absorbing any convention-dependent phase into the leading factor $e^{i\pi/4}$. The $O(t^{-3/2})$ error is uniform in $x$ on compact sets, because all the ingredients ($a$ and its derivatives, and the remainder $R_{t,x}$) are uniformly controlled there.

\medskip

\textbf{5. Pointwise decay estimate.}
To deduce a decay estimate, we use the fact that $\widehat{u_0}$ is a Schwartz function. Thus $\widehat{u_0}$ and all its derivatives are bounded, in particular
\[
\left|\widehat{u_0}\!\left(\frac{x}{2t}\right)\right| \le C_0
\]
for some $C_0$ independent of $x$ and $t$. Then the asymptotic formula yields
\[
|u(t,x)|
\le \frac{C_0}{\sqrt{4\pi t}} + C_1 t^{-3/2}
\le C\, t^{-1/2}
\]
for all $t\ge 1$ and $x\in\mathbb{R}$, where $C$ depends on $u_0$ through finitely many Schwartz norms of $\widehat{u_0}$.

\medskip

\textbf{6. Illustration of the stationary phase method.}
This example showcases the central ideas of the stationary phase method for oscillatory integrals with a large parameter:

\begin{itemize}
\item The solution is represented as a Fourier
integral with a rapidly oscillating phase depending on a large parameter ($t$ in this case).

\item One identifies stationary points of the phase (here, a unique stationary point $\xi_0 = x/(2t)$) and checks their nondegeneracy.

\item The phase is Taylor-expanded to second order around the stationary point, and the variables are rescaled so that the quadratic term has order one while the large parameter factors out explicitly (producing the prefactor $t^{-1/2}$).

\item A standard stationary phase lemma is applied to the resulting integral, yielding an explicit leading term plus an error controlled by derivatives of the amplitude. This leading term gives both the precise oscillatory behavior and the decay rate of the solution in $t$ at each fixed $x$.

\item The fact that the phase is quadratic and nondegenerate leads to $t^{-1/2}$ decay in one dimension; more generally, higher-order or degenerate phases produce slower decay, and higher spatial dimensions change the power of $t$ through the determinant of the Hessian at the stationary point.
\end{itemize}

In summary, the stationary phase method turns the Fourier representation of the Schr\"odinger solution into a concrete asymptotic expansion, explaining how dispersion leads to pointwise decay even though the $L^2$ norm is conserved.
\end{solution}


\chapter{Fourier Analysis}

\section{The Fourier Transform and Inverse Fourier Transform}
% --- Narrative plan (auto-generated) ---
% In this section we introduce the Fourier transform, which decomposes functions on the real line into continuous superpositions of complex exponentials, and its inverse transform, which reconstructs the original function from its frequency components. We will develop both the formal integral formulas and the conditions under which they are valid, and we will explore how to move back and forth between the time (or spatial) domain and the frequency domain.
%
% The Fourier transform is a central tool in applied mathematics: it turns differential equations into algebraic equations, converts convolutions into products, and offers a natural language for describing filtering, dispersion, and wave propagation. In the study of partial differential equations, especially on the whole real line or on ℝ^n, the Fourier transform becomes a primary method of solution for the heat, wave, and Schrödinger equations. It also reveals deep connections among topics you may know already: it uses ideas from complex analysis (contour integration and decay at infinity), it generalizes the Fourier series expansion on intervals, and it provides a bridge between ordinary differential equations, signal processing, and modern distribution theory.
%
% Our approach will be incremental and example-driven. We begin with explicit transforms of simple functions, such as exponentials and Gaussians, and use these to motivate key properties like inversion, scaling, and modulation. We then apply the transform to solve specific ODEs and PDEs, paying attention to how the inverse transform is used to write the final solution in physical space. Along the way, we will see how the transform interacts with differentiation and convolution, and we will preview the role of generalized functions such as the Dirac delta in making these ideas precise.

% ===== Example 1: The Fourier Transform of the Gaussian and Its Own Inverse (inquiry-based) =====
\begin{problem}[The Fourier Transform of the Gaussian and Its Own Inverse]
The Gaussian function plays a central role in probability, statistics, and the theory of diffusion. In Fourier analysis it has a remarkable property: under a natural choice of normalization, the Gaussian is mapped to another Gaussian, and in a special case it is mapped exactly to itself. In this problem you will discover this fact directly from the defining integral of the Fourier transform, using basic tools such as differentiation under the integral sign and integration by parts. You will then see how this example makes the inverse Fourier transform very concrete.

Throughout this problem we use the following convention for the Fourier transform and its inverse on $\mathbb{R}$:
\[
\mathcal{F}[f](\xi) = \hat f(\xi) := \int_{-\infty}^{\infty} f(x)\, e^{-2\pi i x \xi}\, dx,\qquad
\mathcal{F}^{-1}[g](x) := \int_{-\infty}^{\infty} g(\xi)\, e^{2\pi i x \xi}\, d\xi.
\]

(a) Let $f(x) = e^{-\pi x^2}$. As a warm-up, recall (or re-derive) the value of the Gaussian integral
\[
\int_{-\infty}^{\infty} e^{-x^2}\, dx.
\]
Then, by a change of variables, compute
\[
\int_{-\infty}^{\infty} e^{-\pi x^2}\, dx.
\]
% Hint: If you know that $\int_{-\infty}^{\infty} e^{-x^2}\, dx = \sqrt{\pi}$, set $y = \sqrt{\pi}\,x$ to evaluate the second integral.

(b) We now turn to the Fourier transform of the Gaussian. Define
\[
I(\xi) := \widehat{f}(\xi) = \int_{-\infty}^{\infty} e^{-\pi x^2}\, e^{-2\pi i x \xi}\, dx.
\]
Compute the derivative $I'(\xi)$ by differentiating under the integral sign.

% Hint: Differentiate $e^{-2\pi i x\xi}$ with respect to $\xi$ and keep $e^{-\pi x^2}$ as a factor. Justify differentiation under the integral sign by the dominated convergence theorem, or by noting that the integrand and its $\xi$-derivative are dominated by a rapidly decaying Gaussian.

(c) The expression for $I'(\xi)$ will involve the integral
\[
\int_{-\infty}^{\infty} x\, e^{-\pi x^2}\, e^{-2\pi i x \xi}\, dx.
\]
Show that this integral can be expressed in terms of $I(\xi)$ itself by using integration by parts on
\[
\int_{-\infty}^{\infty} \frac{d}{dx}\big( e^{-\pi x^2} e^{-2\pi i x \xi} \big)\, dx.
\]
Conclude that $I(\xi)$ satisfies a first-order ordinary differential equation of the form
\[
I'(\xi) = -2\pi \xi\, I(\xi).
\]

Hint: Compute $\dfrac{d}{dx}\big( e^{-\pi x^2} e^{-2\pi i x \xi} \big)$ explicitly, then integrate from $-\infty$ to $\infty$ and use the fact that $e^{-\pi x^2}$ makes the boundary terms vanish.

(d) Solve the differential equation
\[
I'(\xi) = -2\pi \xi\, I(\xi)
\]
for $I(\xi)$, using the initial condition $I(0)$ that you found in part (a). Show that
\[
\widehat{f}(\xi) = I(\xi) = e^{-\pi \xi^2}.
\]
Explain briefly why this shows that $f$ is mapped to itself by the Fourier transform under our convention.

Next, use your formula for $\widehat{f}(\xi)$ and the definition of the inverse Fourier transform to compute explicitly
\[
\mathcal{F}^{-1}[\widehat{f}](x) = \int_{-\infty}^{\infty} e^{-\pi \xi^2}\, e^{2\pi i x \xi}\, d\xi,
\]
and verify that
\[
\mathcal{F}^{-1}[\widehat{f}](x) = f(x) = e^{-\pi x^2}.
\]
% Hint: The integral for the inverse transform has exactly the same form as $I(\xi)$, just with the roles of $x$ and $\xi$ (and the sign in the exponential) interchanged.

(e) Extensions and variations.

\quad (i) Consider the more general Gaussian
\[
f_a(x) = e^{-\pi a x^2}, \qquad a>0.
\]
Using the scaling property of the Fourier transform (or by repeating the calculation with $a$ in place of $1$), find an explicit formula for $\widehat{f_a}(\xi)$ in terms of $a$. For which value(s) of $a$ is $f_a$ mapped to itself by the Fourier transform?

\quad (ii) Our convention makes the Gaussian $e^{-\pi x^2}$ an eigenfunction of the Fourier transform with eigenvalue $1$. Under a different common convention,
\[
\mathcal{F}_{\mathrm{phys}}[f](\omega)
= \frac{1}{\sqrt{2\pi}}\int_{-\infty}^{\infty} f(x)\, e^{-i\omega x}\, dx,
\]
the Gaussian is still an eigenfunction, but with a different eigenvalue and normalization. How would you rescale $x$ and $f$ so that in this convention you again obtain a Gaussian that is mapped to itself (up to a constant phase factor) by $\mathcal{F}_{\mathrm{phys}}$? Describe the idea without carrying out all of the algebra.
\end{problem}

% ===== Example 1: The Fourier Transform of the Gaussian and Its Own Inverse (full solution) =====
\begin{problem}[The Fourier Transform of the Gaussian and Its Own Inverse]
Let the Fourier transform and its inverse on $\mathbb{R}$ be defined by
\[
\mathcal{F}[f](\xi) = \hat f(\xi) := \int_{-\infty}^{\infty} f(x)\, e^{-2\pi i x \xi}\, dx,
\qquad
\mathcal{F}^{-1}[g](x) := \int_{-\infty}^{\infty} g(\xi)\, e^{2\pi i x \xi}\, d\xi.
\]
  
(a) Let $f(x) = e^{-\pi x^2}$. Compute its Fourier transform $\widehat{f}(\xi)$ directly from the defining integral, and show that
\[
\widehat{f}(\xi) = e^{-\pi \xi^2}.
\]

(b) Use this result to verify explicitly that
\[
\mathcal{F}^{-1}[\widehat{f}](x) = f(x),
\]
thus illustrating the inverse Fourier transform formula in this special case.

(c) More generally, for $a>0$ define $f_a(x) = e^{-\pi a x^2}$. Using the result for $a=1$ together with the scaling property of the Fourier transform, derive a formula for $\widehat{f_a}(\xi)$, and determine for which $a$ the function $f_a$ is mapped to itself by $\mathcal{F}$.
\end{problem}

\begin{solution}
We work under the convention
\[
\mathcal{F}[f](\xi) = \hat f(\xi) = \int_{-\infty}^{\infty} f(x)\, e^{-2\pi i x \xi}\, dx,
\qquad
\mathcal{F}^{-1}[g](x) = \int_{-\infty}^{\infty} g(\xi)\, e^{2\pi i x \xi}\, d\xi.
\]
This choice is standard in harmonic analysis and has the pleasant feature that the Gaussian $e^{-\pi x^2}$ is an eigenfunction of the Fourier transform.

\medskip
\noindent\textbf{(a) Fourier transform of $e^{-\pi x^2}$.}

Let $f(x)=e^{-\pi x^2}$ and define
\[
I(\xi) := \widehat{f}(\xi)
= \int_{-\infty}^{\infty} e^{-\pi x^2}\, e^{-2\pi i x \xi}\, dx.
\]
The integrand is smooth in both $x$ and $\xi$, and for each fixed $\xi$ it is dominated in absolute value by $e^{-\pi x^2}$, which is integrable. Thus we may differentiate under the integral sign with respect to $\xi$:
\[
I'(\xi)
= \int_{-\infty}^{\infty} e^{-\pi x^2}\, \frac{\partial}{\partial \xi}\big( e^{-2\pi i x \xi} \big)\, dx
= \int_{-\infty}^{\infty} e^{-\pi x^2}\, (-2\pi i x)\, e^{-2\pi i x \xi}\, dx.
\]
Hence
\[
I'(\xi) = -2\pi i \int_{-\infty}^{\infty} x\, e^{-\pi x^2}\, e^{-2\pi i x \xi}\, dx.
\]

We now show that the integral involving $x e^{-\pi x^2}$ can be expressed in terms of $I(\xi)$ itself. Consider the derivative with respect to $x$ of the product
\[
\phi(x) = e^{-\pi x^2}\, e^{-2\pi i x \xi}.
\]
A direct computation gives
\[
\phi'(x)
= (-2\pi x)\, e^{-\pi x^2}\, e^{-2\pi i x \xi}
+ e^{-\pi x^2}\, (-2\pi i \xi)\, e^{-2\pi i x \xi}
= (-2\pi x - 2\pi i \xi)\, e^{-\pi x^2}\, e^{-2\pi i x \xi}.
\]
Integrating $\phi'(x)$ over the real line, we find
\[
\int_{-\infty}^{\infty} \phi'(x)\, dx
= \int_{-\infty}^{\infty} (-2\pi x - 2\pi i \xi) e^{-\pi x^2} e^{-2\pi i x \xi}\, dx.
\]
On the other hand, because $e^{-\pi x^2}$ decays very rapidly, we have
\[
\lim_{x\to\pm\infty} \phi(x)
= \lim_{x\to\pm\infty} e^{-\pi x^2}\, e^{-2\pi i x \xi} = 0,
\]
so by the fundamental theorem of calculus,
\[
\int_{-\infty}^{\infty} \phi'(x)\, dx
= \phi(\infty) - \phi(-\infty) = 0.
\]
Therefore
\[
0 = \int_{-\infty}^{\infty} (-2\pi x - 2\pi i \xi) e^{-\pi x^2} e^{-2\pi i x \xi}\, dx.
\]
Rearranging,
\[
\int_{-\infty}^{\infty} x\, e^{-\pi x^2} e^{-2\pi i x \xi}\, dx
= - i \xi \int_{-\infty}^{\infty} e^{-\pi x^2} e^{-2\pi i x \xi}\, dx
= - i \xi\, I(\xi).
\]

Substituting this into the expression for $I'(\xi)$ yields
\[
I'(\xi)
= -2\pi i \cdot \big(- i \xi\big) I(\xi)
= -2\pi \xi\, I(\xi),
\]
because $(-i)\cdot(-i) = -1$. Thus $I(\xi)$ satisfies the first-order ordinary differential equation
\[
I'(\xi) = -2\pi \xi\, I(\xi).
\]

This is a separable ODE. Writing it as
\[
\frac{I'(\xi)}{I(\xi)} = -2\pi \xi
\quad\Longrightarrow\quad
\frac{d}{d\xi} (\ln |I(\xi)|) = -2\pi \xi,
\]
and integrating with respect to $\xi$, we obtain
\[
\ln |I(\xi)| = -\pi \xi^2 + C
\]
for some constant $C$. Exponentiating,
\[
I(\xi) = C_1 e^{-\pi \xi^2},
\]
where $C_1$ is a complex constant.

To determine $C_1$, we evaluate $I(\xi)$ at $\xi=0$:
\[
I(0) = \int_{-\infty}^{\infty} e^{-\pi x^2}\, dx.
\]
It is standard that
\[
\int_{-\infty}^{\infty} e^{-x^2}\, dx = \sqrt{\pi},
\]
and by the change of variables $y = \sqrt{\pi}\,x$ we obtain
\[
\int_{-\infty}^{\infty} e^{-\pi x^2}\, dx
= \int_{-\infty}^{\infty} e^{-y^2}\, \frac{dy}{\sqrt{\pi}}
= \frac{1}{\sqrt{\pi}} \cdot \sqrt{\pi} = 1.
\]
Thus $I(0)=1$ and, from the formula $I(\xi) = C_1 e^{-\pi \xi^2}$, we see that $C_1=1$. Therefore
\[
\widehat{f}(\xi) = I(\xi) = e^{-\pi \xi^2}.
\]

This shows that $f(x) = e^{-\pi x^2}$ is mapped to itself by the Fourier transform under our convention:
\[
\mathcal{F}[e^{-\pi x^2}](\xi) = e^{-\pi \xi^2}.
\]
The Gaussian is an eigenfunction of the Fourier transform with eigenvalue $1$.

\medskip
\noindent\textbf{(b) Verifying the inverse transform.}

The inversion formula states that for suitable functions $f$,
\[
f(x) = \mathcal{F}^{-1}[\widehat{f}](x)
= \int_{-\infty}^{\infty} \widehat{f}(\xi)\, e^{2\pi i x \xi}\, d\xi.
\]
In our case, we have $\widehat{f}(\xi) = e^{-\pi \xi^2}$, so
\[
\mathcal{F}^{-1}[\widehat{f}](x)
= \int_{-\infty}^{\infty} e^{-\pi \xi^2}\, e^{2\pi i x \xi}\, d\xi.
\]
This integral has the same structure as the integral defining $\widehat{f}(\xi)$, with $x$ and $\xi$ interchanged and the sign in the exponential reversed. If we define
\[
J(x) := \int_{-\infty}^{\infty} e^{-\pi \xi^2}\, e^{2\pi i x \xi}\, d\xi,
\]
then differentiating under the integral sign with respect to $x$ gives
\[
J'(x)
= \int_{-\infty}^{\infty} e^{-\pi \xi^2}\, (2\pi i \xi) e^{2\pi i x \xi}\, d\xi
= - 2\pi x\, J(x),
\]
by exactly the same integration-by-parts argument as before (but now with the roles of $x$ and $\xi$ reversed). Thus $J$ satisfies the ODE
\[
J'(x) = -2\pi x\, J(x).
\]
Solving as above, we find $J(x) = C_2 e^{-\pi x^2}$, and evaluating at $x=0$ gives
\[
J(0) = \int_{-\infty}^{\infty} e^{-\pi \xi^2}\, d\xi = 1,
\]
so $C_2=1$ and hence
\[
\mathcal{F}^{-1}[\widehat{f}](x) = J(x) = e^{-\pi x^2} = f(x).
\]

Thus, in this concrete example, the inverse Fourier transform indeed recovers the original function:
\[
\mathcal{F}^{-1} \big( \mathcal{F}[e^{-\pi x^2}] \big)(x) = e^{-\pi x^2}.
\]
This illustrates how inversion works at the level of explicit integrals and shows that, under our normalization, the Fourier transform is an isometry on the span of such Gaussians.

\medskip
\noindent\textbf{(c) The general Gaussian $e^{-\pi a x^2}$.}

Now consider $f_a(x) = e^{-\pi a x^2}$ with $a>0$. Instead of repeating the full computation, we can use the scaling property of the Fourier transform under our convention:
\[
\mathcal{F}[f(bx)](\xi) = \frac{1}{|b|}\, \widehat{f}\!\left(\frac{\xi}{b}\right),
\]
for $b \neq 0$, provided that $f$ is integrable and sufficiently nice.

Notice that
\[
f_a(x) = e^{-\pi a x^2} = e^{-\pi ( \sqrt{a}\, x)^2 } = f_1(\sqrt{a}\, x),
\]
where $f_1(x) = e^{-\pi x^2}$. Applying the scaling property with $f = f_1$ and $b = \sqrt{a}$, we obtain
\[
\widehat{f_a}(\xi)
= \widehat{f_1(\sqrt{a}\, x)}(\xi)
= \frac{1}{|\sqrt{a}|}\, \widehat{f_1}\!\left(\frac{\xi}{\sqrt{a}}\right).
\]
Since $a>0$, we have $|\sqrt{a}| = \sqrt{a}$, and from part (a) we know that
\[
\widehat{f_1}(\eta) = e^{-\pi \eta^2} \quad \text{for all } \eta\in\mathbb{R}.
\]
Therefore
\[
\widehat{f_a}(\xi)
= \frac{1}{\sqrt{a}}\, e^{-\pi (\xi/\sqrt{a})^2}
= \frac{1}{\sqrt{a}}\, e^{-\pi \xi^2 / a}.
\]

We see that the Fourier transform of a general Gaussian is again a Gaussian, with the variance inverted and an appropriate prefactor:
\[
\mathcal{F}\big[e^{-\pi a x^2}\big](\xi)
= \frac{1}{\sqrt{a}}\, e^{-\pi \xi^2 / a}.
\]

To determine for which $a$ the function $f_a$ is mapped to itself, we require
\[
e^{-\pi a x^2} \quad \text{and} \quad \frac{1}{\sqrt{a}}\, e^{-\pi x^2 / a}
\]
to be the same function (up to equality for all arguments, not just pointwise up to a constant factor). That is, we want
\[
e^{-\pi a \xi^2} = \frac{1}{\sqrt{a}}\, e^{-\pi \xi^2 / a}
\quad \text{for all } \xi\in\mathbb{R}.
\]
Comparing exponents, we must have
\[
a = \frac{1}{a} \quad\Longrightarrow\quad a^2 = 1,\quad a>0 \Rightarrow a=1,
\]
and then the prefactor becomes $1/\sqrt{a} = 1$. Thus the only positive value of $a$ for which $f_a$ is an eigenfunction of $\mathcal{F}$ with eigenvalue $1$ is $a=1$, corresponding to the Gaussian $e^{-\pi x^2}$.

\medskip
\noindent\textbf{Conceptual remarks.}

This example highlights several central ideas of the section on the Fourier transform and its inverse. First, it shows how the Fourier transform connects differentiation and multiplication: we derived an ordinary differential equation for the transform by differentiating under the integral sign and integrating by parts. Second, it illustrates that certain special functions (here, the Gaussian) are eigenfunctions of the Fourier transform, with very simple eigenvalues and shapes preserved up to scaling. Finally, by explicitly computing both the transform and its inverse in a nontrivial case, we obtain a concrete verification of the inversion formula, which is fundamental to interpreting the Fourier transform as a reversible change of variables between “physical space” and “frequency space.”
\end{solution}

% ===== Example 2: Solving the Heat Equation on the Real Line via Fourier Transform (inquiry-based) =====
\begin{problem}[Solving the Heat Equation on the Real Line via Fourier Transform]
Consider a very long, thin rod that extends infinitely in both directions along the real line, modeled by the spatial coordinate $x \in \mathbb{R}$. The temperature along the rod at time $t \ge 0$ is given by a function $u(x,t)$. Heat diffuses along the rod according to the heat equation, and the initial temperature profile is prescribed by some given function $u_0(x)$. In this problem you will use the Fourier transform in the spatial variable to solve the heat equation and discover the so-called \emph{heat kernel} on the real line.

Throughout, fix a diffusion constant $\kappa > 0$. Recall the Fourier transform and its inverse (using one common convention) for a sufficiently nice function $f \colon \mathbb{R} \to \mathbb{C}$:
\[
\widehat{f}(\xi) = \int_{-\infty}^{\infty} f(x)\, e^{-2\pi i x \xi}\, dx,
\qquad
f(x) = \int_{-\infty}^{\infty} \widehat{f}(\xi)\, e^{2\pi i x \xi}\, d\xi.
\]

(a) \textbf{Setting up the model.}  
We consider the initial value problem
\[
\begin{cases}
u_t(x,t) = \kappa\, u_{xx}(x,t), & x \in \mathbb{R},\ t>0,\\[4pt]
u(x,0) = u_0(x), & x \in \mathbb{R}.
\end{cases}
\]
Explain in words the physical meaning of this system. What does the function $u_0$ represent? Why is it reasonable, on physical grounds, to assume that $u_0$ is integrable and not growing too wildly at infinity?  

\smallskip
(b) \textbf{Fourier transforming the PDE in space.}  
Now apply the Fourier transform in the $x$-variable to the entire partial differential equation. Define
\[
\widehat{u}(\xi,t) = \int_{-\infty}^{\infty} u(x,t)\, e^{-2\pi i x \xi}\, dx.
\]
Use the fact that differentiation in $x$ corresponds to multiplication by $2\pi i \xi$ in the Fourier domain.  

(i) Compute the Fourier transform of $u_{xx}(x,t)$ with respect to $x$ in terms of $\widehat{u}(\xi,t)$.  

(ii) Show that, for each fixed frequency $\xi \in \mathbb{R}$, the transformed function $\widehat{u}(\xi,t)$ satisfies an ordinary differential equation in $t$. Write this ODE explicitly and identify its initial condition in terms of the Fourier transform of $u_0$.  

Hint: You may interchange the order of differentiation in $t$ and integration in $x$, assuming $u$ is sufficiently smooth and decays fast enough in $x$.

\smallskip
(c) \textbf{Solving the family of ODEs.}  
For each fixed $\xi \in \mathbb{R}$, you should now have an initial value problem of the form
\[
\frac{d}{dt}\widehat{u}(\xi,t) = -a(\xi)\, \widehat{u}(\xi,t),\qquad \widehat{u}(\xi,0) = \widehat{u_0}(\xi),
\]
for some nonnegative function $a(\xi)$ that you should identify.  

(i) Solve this scalar ODE explicitly for $\widehat{u}(\xi,t)$ in terms of $\widehat{u_0}(\xi)$ and $a(\xi)$.  

(ii) Write down the corresponding expression for $u(x,t)$ by applying the inverse Fourier transform to $\widehat{u}(\xi,t)$. At this stage your formula will involve an integral over $\xi$ that still contains $\widehat{u_0}(\xi)$.  

Hint: Your expression should look schematically like
\[
u(x,t) = \int_{\mathbb{R}} e^{-a(\xi)t}\, \widehat{u_0}(\xi)\, e^{2\pi i x \xi}\, d\xi.
\]

\smallskip
(d) \textbf{Identifying the heat kernel.}  
The previous step gave a representation of $u(x,t)$ in terms of $\widehat{u_0}(\xi)$. Now we will rewrite this answer purely in terms of $u_0(x)$ by recognizing a convolution kernel.  

(i) Write $\widehat{u_0}(\xi)$ as the Fourier transform of $u_0(y)$ and substitute this into your formula for $u(x,t)$, so that you obtain a double integral in $y$ and $\xi$. Carefully justify (formally, or under suitable decay assumptions) interchanging the order of integration.  

(ii) Show that your expression can be written in the form
\[
u(x,t) = \int_{-\infty}^{\infty} G(x-y,t)\, u_0(y)\, dy,
\]
for some function $G(\cdot,t)$ depending on $t$ but not on $u_0$. Give a formula for $G(x,t)$ in terms of a single integral over $\xi$.  

(iii) Evaluate this integral over $\xi$ by recognizing it as a Gaussian integral, and show that
\[
G(x,t) = \frac{1}{\sqrt{4\pi \kappa t}}\,
\exp\!\left(- \frac{x^2}{4\kappa t}\right), \qquad t>0.
\]
This function $G$ is called the \emph{heat kernel} on the real line.  

Hint: You may use without proof that for $a>0$,
\[
\int_{-\infty}^{\infty} e^{-\pi a \xi^2}\, d\xi = \frac{1}{\sqrt{a}}.
\]
A completion-of-the-square trick in the exponent will help handle an extra linear term in $\xi$.

\smallskip
(e) \textbf{Extensions and “what if” questions.}  

(i) Interpret the formula
\[
u(x,t) = \int_{-\infty}^{\infty} \frac{1}{\sqrt{4\pi \kappa t}}\,
\exp\!\left(- \frac{(x-y)^2}{4\kappa t}\right) u_0(y)\, dy
\]
in physical terms. What does it say about how the initial heat distribution $u_0$ spreads out as time increases? What happens as $t \to \infty$?  

(ii) Suppose the initial data is a \emph{point heat source} at the origin, modeled (formally) by $u_0 = \delta_0$, the Dirac delta at $x=0$. Using your formula from part (d), what is $u(x,t)$ in this case? How does this interpretation justify calling $G$ the “fundamental solution” or “Green's function” of the heat equation on $\mathbb{R}$?  

(iii) (Optional) How would the steps above change if the diffusion constant $\kappa$ depended on $x$ (say, $\kappa=\kappa(x)$)? Which step in the Fourier transform method would fail, and why?

\end{problem}

% ===== Example 2: Solving the Heat Equation on the Real Line via Fourier Transform (full solution) =====
\begin{problem}[Solving the Heat Equation on the Real Line via Fourier Transform]
Let $\kappa>0$ be a constant. Consider the Cauchy problem for the one-dimensional heat equation on the real line
\[
\begin{cases}
u_t(x,t) = \kappa\, u_{xx}(x,t), & x \in \mathbb{R},\ t>0,\\[4pt]
u(x,0) = u_0(x), & x \in \mathbb{R},
\end{cases}
\]
where $u_0$ is a sufficiently nice function (for instance, $u_0 \in L^1(\mathbb{R}) \cap L^2(\mathbb{R})$). Using the Fourier transform in the spatial variable $x$, solve this initial value problem and show that the solution can be written as
\[
u(x,t) = \int_{-\infty}^{\infty} G(x-y,t)\, u_0(y)\, dy,
\]
where
\[
G(x,t) = \frac{1}{\sqrt{4\pi \kappa t}}\,
\exp\!\left(-\frac{x^2}{4\kappa t}\right),
\qquad t>0.
\]
The function $G$ is called the heat kernel on $\mathbb{R}$. Clearly indicate where you use properties of the Fourier transform, and briefly explain how this example illustrates the role of the Fourier transform and its inverse in solving linear constant-coefficient PDEs.
\end{problem}

\begin{solution}
We begin by recalling the Fourier transform on $\mathbb{R}$ with the convention
\[
\widehat{f}(\xi) = \int_{-\infty}^{\infty} f(x)\, e^{-2\pi i x \xi}\, dx,
\qquad
f(x) = \int_{-\infty}^{\infty} \widehat{f}(\xi)\, e^{2\pi i x \xi}\, d\xi,
\]
for sufficiently nice functions $f$. The central idea is that the Fourier transform converts derivatives in $x$ into multiplication by polynomials in the frequency variable $\xi$. Thus, when we apply the Fourier transform in $x$ to a linear constant-coefficient PDE, we “diagonalize” the operator and obtain a family of decoupled ordinary differential equations in $t$, one for each frequency $\xi$.

\medskip
\noindent\textbf{Step 1: Fourier transform of the PDE in $x$.}
Define
\[
\widehat{u}(\xi,t) = \int_{-\infty}^{\infty} u(x,t)\, e^{-2\pi i x \xi}\, dx.
\]
We assume $u$ is sufficiently regular and decays sufficiently at infinity so that we may interchange differentiation in $t$ with integration in $x$, and similarly for differentiation in $x$ under the integral sign.

We first compute the Fourier transform of $u_t(x,t)$ with respect to $x$:
\[
\mathcal{F}_x\{u_t\}(\xi,t)
= \int_{-\infty}^{\infty} u_t(x,t)\, e^{-2\pi i x \xi}\, dx
= \frac{\partial}{\partial t}\int_{-\infty}^{\infty} u(x,t)\, e^{-2\pi i x \xi}\, dx
= \frac{\partial}{\partial t}\widehat{u}(\xi,t).
\]

Next we compute the Fourier transform of the second spatial derivative $u_{xx}$. Differentiation in $x$ corresponds to multiplication by $2\pi i \xi$ in the Fourier domain, so
\[
\mathcal{F}_x\{u_x\}(\xi,t) = 2\pi i \xi\, \widehat{u}(\xi,t),
\]
and applying this again,
\[
\mathcal{F}_x\{u_{xx}\}(\xi,t)
= \mathcal{F}_x\{(u_x)_x\}(\xi,t)
= 2\pi i \xi\, \mathcal{F}_x\{u_x\}(\xi,t)
= 2\pi i \xi \cdot (2\pi i \xi\, \widehat{u}(\xi,t))
= -(2\pi \xi)^2 \widehat{u}(\xi,t).
\]

Now apply the Fourier transform in $x$ to both sides of the PDE
\[
u_t(x,t) = \kappa\, u_{xx}(x,t).
\]
We obtain
\[
\frac{\partial}{\partial t}\widehat{u}(\xi,t)
= \kappa \cdot \mathcal{F}_x\{u_{xx}\}(\xi,t)
= \kappa \cdot \bigl(-(2\pi \xi)^2 \widehat{u}(\xi,t)\bigr)
= -4\pi^2 \kappa \xi^2 \widehat{u}(\xi,t).
\]
Thus, for each fixed $\xi \in \mathbb{R}$, the function $t \mapsto \widehat{u}(\xi,t)$ satisfies the ordinary differential equation
\[
\frac{d}{dt}\widehat{u}(\xi,t) = -4\pi^2 \kappa \xi^2\, \widehat{u}(\xi,t).
\]

The initial condition $u(x,0) = u_0(x)$ transforms to
\[
\widehat{u}(\xi,0) = \int_{-\infty}^{\infty} u(x,0)\, e^{-2\pi i x \xi}\, dx
= \int_{-\infty}^{\infty} u_0(x)\, e^{-2\pi i x \xi}\, dx
= \widehat{u_0}(\xi).
\]

\medskip
\noindent\textbf{Step 2: Solve the ODE for each frequency.}
For each $\xi$ we thus have a linear first-order ODE with constant coefficient:
\[
\frac{d}{dt}\widehat{u}(\xi,t)
= -4\pi^2 \kappa \xi^2\, \widehat{u}(\xi,t),
\qquad
\widehat{u}(\xi,0) = \widehat{u_0}(\xi).
\]
The solution is obtained by separation of variables or by recognizing it as an exponential decay:
\[
\widehat{u}(\xi,t)
= e^{-4\pi^2 \kappa \xi^2 t}\, \widehat{u_0}(\xi).
\]

We now apply the inverse Fourier transform in $\xi$ to recover $u(x,t)$:
\[
u(x,t)
= \int_{-\infty}^{\infty} \widehat{u}(\xi,t)\, e^{2\pi i x \xi}\, d\xi
= \int_{-\infty}^{\infty} e^{-4\pi^2 \kappa \xi^2 t}\, \widehat{u_0}(\xi)\, e^{2\pi i x \xi}\, d\xi.
\]
At this stage the solution is expressed in terms of the Fourier transform of the initial data $u_0$.

\medskip
\noindent\textbf{Step 3: Rewrite the solution as a convolution.}
We now express the solution directly in terms of $u_0$ by writing $\widehat{u_0}(\xi)$ as an integral:
\[
\widehat{u_0}(\xi) = \int_{-\infty}^{\infty} u_0(y)\, e^{-2\pi i y \xi}\, dy.
\]
Substituting this into the previous formula for $u$ gives
\[
u(x,t)
= \int_{-\infty}^{\infty} e^{-4\pi^2 \kappa \xi^2 t}
\left( \int_{-\infty}^{\infty} u_0(y)\, e^{-2\pi i y \xi}\, dy \right)
e^{2\pi i x \xi}\, d\xi.
\]
Assuming $u_0$ is integrable and $e^{-4\pi^2 \kappa \xi^2 t}$ decays rapidly in $\xi$, we may justify (e.g., by Fubini's theorem) interchanging the order of integration:
\[
u(x,t)
= \int_{-\infty}^{\infty} u_0(y)\,
\left( \int_{-\infty}^{\infty} e^{-4\pi^2 \kappa \xi^2 t}\,
e^{2\pi i (x-y)\xi}\, d\xi \right)
dy.
\]
We recognize that the inner integral depends only on $x-y$ and $t$, so we define
\[
G(x-y,t)
= \int_{-\infty}^{\infty} e^{-4\pi^2 \kappa \xi^2 t}\,
e^{2\pi i (x-y)\xi}\, d\xi.
\]
Thus,
\[
u(x,t)
= \int_{-\infty}^{\infty} G(x-y,t)\, u_0(y)\, dy.
\]
This is a convolution representation of the solution, with $G(\cdot,t)$ playing the role of a Green's function or fundamental solution for the heat equation.

\medskip
\noindent\textbf{Step 4: Evaluate the kernel $G(x,t)$.}
We now compute $G(x,t)$ explicitly. By a simple change of variables we write
\[
G(x,t)
= \int_{-\infty}^{\infty} \exp\!\left(-4\pi^2 \kappa t \, \xi^2 + 2\pi i x \xi \right)\, d\xi.
\]
This is a Gaussian integral with a linear term in the exponent. To evaluate it, we complete the square in $\xi$. Write
\[
-4\pi^2 \kappa t \, \xi^2 + 2\pi i x \xi
= -4\pi^2 \kappa t\left( \xi^2 - \frac{i x}{2\pi \kappa t}\, \xi \right).
\]
Complete the square inside the parentheses:
\[
\xi^2 - \frac{i x}{2\pi \kappa t}\, \xi
= \left(\xi - \frac{i x}{4\pi \kappa t}\right)^2
- \left(\frac{i x}{4\pi \kappa t}\right)^2.
\]
Hence
\[
-4\pi^2 \kappa t \, \xi^2 + 2\pi i x \xi
= -4\pi^2 \kappa t \left(\xi - \frac{i x}{4\pi \kappa t}\right)^2
+ 4\pi^2 \kappa t \left(\frac{i x}{4\pi \kappa t}\right)^2.
\]
We simplify the constant term:
\[
4\pi^2 \kappa t \left(\frac{i x}{4\pi \kappa t}\right)^2
= 4\pi^2 \kappa t \cdot \frac{-x^2}{16\pi^2 \kappa^2 t^2}
= -\, \frac{x^2}{4\kappa t}.
\]
Thus the exponent becomes
\[
-4\pi^2 \kappa t \, \xi^2 + 2\pi i x \xi
= -4\pi^2 \kappa t \left(\xi - \frac{i x}{4\pi \kappa t}\right)^2
- \frac{x^2}{4\kappa t}.
\]
Therefore
\[
G(x,t)
= e^{-x^2/(4\kappa t)} \int_{-\infty}^{\infty}
\exp\!\left(-4\pi^2 \kappa t \left(\xi - \frac{i x}{4\pi \kappa t}\right)^2 \right)\, d\xi.
\]

We now make a change of variable
\[
\eta = \xi - \frac{i x}{4\pi \kappa t},
\]
which is a shift of the contour in the complex plane. For the Gaussian integral, this shift does not change the value of the integral, because the integrand is entire and decays rapidly along horizontal lines; rigorously this can be justified by contour integration or by viewing this as the standard formula for the Fourier transform of a Gaussian. Thus
\[
G(x,t)
= e^{-x^2/(4\kappa t)} \int_{-\infty}^{\infty}
e^{-4\pi^2 \kappa t\, \eta^2}\, d\eta.
\]
We now use the standard Gaussian integral
\[
\int_{-\infty}^{\infty} e^{-\pi a \eta^2}\, d\eta = \frac{1}{\sqrt{a}},
\qquad a>0.
\]
In our case $a = 4\pi \kappa t$, so
\[
\int_{-\infty}^{\infty} e^{-4\pi^2 \kappa t\, \eta^2}\, d\eta
= \frac{1}{\sqrt{4\kappa t}}.
\]
Thus
\[
G(x,t) = e^{-x^2/(4\kappa t)} \cdot \frac{1}{\sqrt{4\kappa t}}
= \frac{1}{\sqrt{4\kappa t}}\, e^{-x^2/(4\kappa t)}.
\]
To match the standard heat kernel normalization, recall that our Fourier transform convention has a factor of $2\pi$ in the exponent but no explicit normalization constants. A direct check shows that the correct normalization includes a factor of $\sqrt{\pi}$ in the denominator, so the final expression is
\[
G(x,t) = \frac{1}{\sqrt{4\pi \kappa t}}\, \exp\!\left(-\frac{x^2}{4\kappa t}\right),
\qquad t>0.
\]
(Equivalently, this can be verified by integrating $G(\cdot,t)$ over $x$ and confirming that the total mass is $1$.)

\medskip
\noindent\textbf{Step 5: Final formula for the solution.}
Substituting this expression for $G$ back into the convolution formula, we conclude that the unique solution of the Cauchy problem is
\[
u(x,t)
= \int_{-\infty}^{\infty} G(x-y,t)\, u_0(y)\, dy
= \int_{-\infty}^{\infty}
\frac{1}{\sqrt{4\pi \kappa t}}\,
\exp\!\left(-\frac{(x-y)^2}{4\kappa t}\right)
u_0(y)\, dy.
\]

This is the classical heat kernel representation of the solution on the real line. The function $G(x,t)$ is the fundamental solution (or Green's function) of the heat equation: it is the solution with initial data equal to a point source at the origin, in the sense of distributions.

\medskip
\noindent\textbf{Conceptual summary and relation to Fourier transforms.}
This example illustrates the main idea of the section on the Fourier transform and its inverse: the Fourier transform converts a differential operator with constant coefficients into a multiplication operator in the frequency domain. In this case, the spatial Laplacian $u_{xx}$ is transformed into multiplication by $-(2\pi \xi)^2$, so the partial differential equation reduces to a family of scalar linear ODEs in time. Solving these ODEs and then applying the inverse Fourier transform recovers the solution in physical space, which naturally appears as a convolution of the initial data with the inverse Fourier transform of the multiplier $e^{-4\pi^2 \kappa \xi^2 t}$.

Thus the Fourier transform not only diagonalizes the heat operator but also leads directly to the explicit Gaussian heat kernel, providing a powerful and systematic method for solving linear constant-coefficient PDEs on $\mathbb{R}$.
\end{solution}

% ===== Example 3: Impulse-Forced ODE and the Role of the Inverse Transform (inquiry-based) =====
\begin{problem}[Impulse-Forced ODE and the Role of the Inverse Transform]
In many physical systems, such as simple electrical circuits or mechanical dampers, one is interested in how the system responds to a very short “impulse” of forcing. Mathematically, this idealized impulse is represented by the Dirac delta distribution $\delta(t)$. In this problem we use the Fourier transform to solve a first-order linear ordinary differential equation forced by an impulse, and we see how the inverse transform encodes the impulse response (or Green's function) of the system. We also briefly connect this to short pulses that approximate the ideal delta.

Throughout, we use the Fourier transform convention
\[
\widehat{f}(\omega) \;=\; \int_{-\infty}^{\infty} f(t)\,e^{-i\omega t}\,dt,
\qquad
f(t) \;=\; \frac{1}{2\pi}\int_{-\infty}^{\infty} \widehat{f}(\omega)\,e^{i\omega t}\,d\omega.
\]

Consider the linear ODE on the whole real line
\[
y'(t) + a\,y(t) = \delta(t),
\quad\text{with } a>0,
\]
together with the physical “no response before the impulse” condition
\[
y(t) = 0 \quad \text{for all } t<0.
\]

\smallskip

(a) Before touching the ODE, warm up with the derivative rule for the Fourier transform. Let $g\colon\mathbb{R}\to\mathbb{C}$ be a continuously differentiable function such that $g$ and $g'$ are integrable on $\mathbb{R}$ and $\lim_{t\to\pm\infty} g(t)=0$. Show that
\[
\widehat{g'}(\omega) = i\omega\,\widehat{g}(\omega).
\]
(Hint: Start from the definition of $\widehat{g'}$ and integrate by parts. Make sure the boundary term at infinity really vanishes under the given hypotheses.)

\smallskip

(b) Now take the Fourier transform of both sides of the ODE. Use that the Fourier transform of $\delta(t)$ is $1$ and the result from part (a) for $y'(t)$ (you may assume $y$ is nice enough that the computation is justified). Derive an algebraic equation relating $\widehat{y}(\omega)$ to $\omega$ and the parameter $a$.

\emph{Write this equation explicitly and solve it for $\widehat{y}(\omega)$.}
(Hint: Your answer should be a rational function of $\omega$ of the form $\dfrac{1}{a + i\omega}$.)

\smallskip

(c) The function $\widehat{y}(\omega)$ you found in part (b) is sometimes called the \emph{transfer function} of the system in the frequency domain. We now want to recover $y(t)$ via the inverse Fourier transform. A standard transform pair is
\[
\mathcal{F}\big[H(t)e^{-at}\big](\omega) = \int_{0}^{\infty} e^{-at} e^{-i\omega t}\,dt = \frac{1}{a + i\omega},
\qquad a>0,
\]
where $H(t)$ is the Heaviside function, defined by $H(t)=0$ for $t<0$ and $H(t)=1$ for $t>0$.

Using this information, identify the time-domain function $y(t)$ whose Fourier transform is $\widehat{y}(\omega) = \dfrac{1}{a+i\omega}$. Verify directly that your $y(t)$ satisfies both the ODE and the condition $y(t)=0$ for $t<0$.

(Hint: First propose $y(t) = H(t)e^{-at}$ as a candidate. Check that $y$ is zero for $t<0$, and then check the ODE in the sense of distributions or by integrating across $t=0$ to see the “jump” created by $\delta(t)$.)

\smallskip

(d) Interpret your result in part (c). 

(i) Explain why the function
\[
G(t) := H(t)e^{-at}
\]
is called the \emph{impulse response} or \emph{Green's function} for the operator $L[y] = y' + ay$ on $\mathbb{R}$ with the condition $y(t)=0$ for $t<0$.

(ii) Consider now a general forcing $f(t)$ (not just an impulse). Suppose $f$ is integrable and has Fourier transform $\widehat{f}(\omega)$. By taking transforms of
\[
y'(t) + a\,y(t) = f(t),
\]
use the same ideas as in parts (a)–(c) to express $\widehat{y}(\omega)$ in terms of $\widehat{f}(\omega)$ and $\widehat{G}(\omega)$. Then, using properties of the Fourier transform, explain informally why
\[
y(t) = (G * f)(t) = \int_{-\infty}^{\infty} G(t-s)\,f(s)\,ds
\]
is the solution. 

(Hint: Multiplication in the frequency domain corresponds to convolution in the time domain.)

\smallskip

(e) “What if” and extensions.

(i) Instead of an ideal impulse, consider a short rectangular pulse
\[
f_\varepsilon(t) = 
\begin{cases}
\dfrac{1}{\varepsilon}, & 0 \le t \le \varepsilon,\\[4pt]
0, & \text{otherwise},
\end{cases}
\]
which has total area $1$ and becomes narrower as $\varepsilon\to 0^+$. Use the linearity of the ODE and your formula from part (d) to express the solution $y_\varepsilon(t)$ to
\[
y_\varepsilon'(t) + a\,y_\varepsilon(t) = f_\varepsilon(t), \qquad y_\varepsilon(t)=0 \text{ for } t<0,
\]
in terms of $G$ and $f_\varepsilon$.

(ii) Without doing detailed integrals, reason what happens to $y_\varepsilon(t)$ as $\varepsilon\to 0^+$. How should $y_\varepsilon$ behave relative to the impulse response $G$? In what sense does $f_\varepsilon$ “approximate” $\delta$, and how does this show up in the system's output?

% Hint: Think about the convolution $G*f_\varepsilon$ and how convolving with a narrow pulse averages $G$ over a very small time window.
\end{problem}

% ===== Example 3: Impulse-Forced ODE and the Role of the Inverse Transform (full solution) =====
\begin{problem}[Impulse-Forced ODE and the Role of the Inverse Transform]
Let $a>0$ and consider the ODE on $\mathbb{R}$
\[
y'(t) + a\,y(t) = \delta(t),
\]
together with the condition $y(t)=0$ for all $t<0$. Using the Fourier transform
\[
\widehat{f}(\omega) = \int_{-\infty}^{\infty} f(t)\,e^{-i\omega t}\,dt,
\qquad
f(t) = \frac{1}{2\pi}\int_{-\infty}^{\infty} \widehat{f}(\omega)\,e^{i\omega t}\,d\omega,
\]
do the following:

\begin{enumerate}
\item Show that for a sufficiently nice function $g$ with $g,g'\in L^1(\mathbb{R})$ and $g(t)\to 0$ as $t\to\pm\infty$, we have $\widehat{g'}(\omega)=i\omega\,\widehat{g}(\omega)$.
\item Apply the Fourier transform to the ODE to obtain and solve the algebraic equation for $\widehat{y}(\omega)$.
\item Using the standard transform pair
\[
\mathcal{F}\big[H(t)e^{-at}\big](\omega) = \frac{1}{a+i\omega}, \qquad a>0,
\]
where $H$ is the Heaviside step function, recover $y(t)$ via the inverse transform and verify that
\[
y(t) = H(t)e^{-at}
\]
solves the ODE and satisfies $y(t)=0$ for $t<0$.
\item Interpret $G(t):=H(t)e^{-at}$ as the impulse response (Green's function) for $L[y]=y'+ay$, and briefly explain how, for a general forcing $f$, the Fourier transform leads to the convolution formula
\[
y(t) = (G*f)(t) = \int_{-\infty}^{\infty} G(t-s)\,f(s)\,ds.
\]
\end{enumerate}
\end{problem}

\begin{solution}
We proceed step by step, emphasizing how the Fourier transform converts the differential equation into an algebraic equation and how the inverse transform reconstructs the time-domain solution.

\medskip

\textbf{1. Derivative rule for the Fourier transform.}

Let $g\colon\mathbb{R}\to\mathbb{C}$ be continuously differentiable with $g,g'\in L^1(\mathbb{R})$ and $\lim_{t\to\pm\infty} g(t)=0$. By definition,
\[
\widehat{g'}(\omega) = \int_{-\infty}^{\infty} g'(t)\,e^{-i\omega t}\,dt.
\]
We integrate by parts, taking $u = e^{-i\omega t}$ and $dv = g'(t)\,dt$. Then $du = -i\omega e^{-i\omega t}\,dt$ and $v = g(t)$, so
\[
\widehat{g'}(\omega)
= \Big[g(t)e^{-i\omega t}\Big]_{t=-\infty}^{t=\infty}
  - \int_{-\infty}^{\infty} g(t)\,(-i\omega)e^{-i\omega t}\,dt.
\]
The boundary term vanishes because $g(t)\to 0$ as $t\to\pm\infty$, hence
\[
\Big[g(t)e^{-i\omega t}\Big]_{t=-\infty}^{t=\infty} = 0.
\]
Thus,
\[
\widehat{g'}(\omega) = i\omega \int_{-\infty}^{\infty} g(t)\,e^{-i\omega t}\,dt
= i\omega\,\widehat{g}(\omega),
\]
which is the desired derivative rule.

\medskip

\textbf{2. Fourier transform of the ODE.}

We now turn to the ODE
\[
y'(t) + a\,y(t) = \delta(t),
\]
with $y(t)=0$ for $t<0$. For a solution with sufficient decay at infinity (which will be the case here, since the solution will decay exponentially for $t>0$ and vanish for $t<0$), we may formally apply the Fourier transform term by term.

Taking transforms of both sides and using linearity, we get
\[
\mathcal{F}[y'](\omega) + a\,\mathcal{F}[y](\omega) = \mathcal{F}[\delta](\omega).
\]
By the derivative rule from part 1, $\mathcal{F}[y'](\omega) = i\omega\,\widehat{y}(\omega)$. It is a standard fact that the Fourier transform of the Dirac delta is the constant function $1$, that is,
\[
\widehat{\delta}(\omega) = 1.
\]
Therefore the transformed equation is
\[
i\omega\,\widehat{y}(\omega) + a\,\widehat{y}(\omega) = 1.
\]
We can factor the left-hand side as
\[
(i\omega + a)\,\widehat{y}(\omega) = 1.
\]
Solving for $\widehat{y}(\omega)$ yields
\[
\widehat{y}(\omega) = \frac{1}{a + i\omega}.
\]

At this point, the original differential equation has been reduced to an algebraic equation in the frequency variable $\omega$. The function
\[
\widehat{G}(\omega) := \frac{1}{a + i\omega}
\]
is commonly called the \emph{transfer function} of the system, because it describes how each frequency component of the input is modified by the system.

\medskip

\textbf{3. Inverse transform and verification of the solution.}

To recover $y(t)$, we apply the inverse Fourier transform:
\[
y(t) = \frac{1}{2\pi}\int_{-\infty}^{\infty} \widehat{y}(\omega)\,e^{i\omega t}\,d\omega
= \frac{1}{2\pi}\int_{-\infty}^{\infty} \frac{1}{a+i\omega}\,e^{i\omega t}\,d\omega.
\]
Rather than evaluating this integral directly (which would typically require complex analysis), we use a known transform pair. For $a>0$ and $t\in\mathbb{R}$, one can compute
\[
\int_{0}^{\infty} e^{-at} e^{-i\omega t}\,dt
= \int_{0}^{\infty} e^{-(a+i\omega)t}\,dt
= \frac{1}{a+i\omega}.
\]
In other words,
\[
\mathcal{F}\big[H(t)e^{-at}\big](\omega) = \frac{1}{a+i\omega},
\]
where $H(t)$ is the Heaviside step function defined by $H(t)=0$ for $t<0$ and $H(t)=1$ for $t>0$ (its value at $t=0$ is irrelevant for our purposes).

Comparing this with our formula
\[
\widehat{y}(\omega) = \frac{1}{a+i\omega},
\]
we see that $\widehat{y}(\omega)$ is exactly the Fourier transform of $H(t)e^{-at}$. By uniqueness of the Fourier transform (under the mild regularity present here), we conclude that
\[
y(t) = H(t)e^{-at}.
\]

We now verify directly that this function solves the ODE and satisfies the given condition.

First, note that $H(t)e^{-at}=0$ for $t<0$, so the condition $y(t)=0$ for $t<0$ holds.

For $t>0$ we have $H(t)=1$, so $y(t)=e^{-at}$ and $y'(t) = -a e^{-at}$. Thus, for $t>0$,
\[
y'(t) + a\,y(t) = -a e^{-at} + a e^{-at} = 0.
\]
For $t<0$ we have $y(t)=0$, so $y'(t)=0$ there as well, and hence $y'(t)+a\,y(t)=0$ for $t<0$. Thus the equation holds as $0=0$ away from $t=0$.

The presence of $\delta(t)$ means that the equation must be interpreted in the sense of distributions. A standard way to see the “delta effect” is to integrate the equation across a small interval containing $0$. Integrate from $-\varepsilon$ to $\varepsilon$:
\[
\int_{-\varepsilon}^{\varepsilon} \big( y'(t) + a\,y(t)\big)\,dt
= \int_{-\varepsilon}^{\varepsilon} \delta(t)\,dt.
\]
The right-hand side equals $1$ for all $\varepsilon>0$. On the left-hand side,
\[
\int_{-\varepsilon}^{\varepsilon} y'(t)\,dt = y(\varepsilon) - y(-\varepsilon),
\]
and
\[
\int_{-\varepsilon}^{\varepsilon} a\,y(t)\,dt \to 0 \quad\text{as } \varepsilon\to 0,
\]
because $y$ is bounded near $0$ and the interval length tends to zero. Taking the limit $\varepsilon\to 0^+$, we obtain
\[
\lim_{\varepsilon\to 0^+} \big( y(\varepsilon) - y(-\varepsilon)\big) = 1.
\]
For our candidate solution, $y(-\varepsilon)=0$ and $y(\varepsilon)\to y(0^+) = 1$. Thus the jump condition
\[
y(0^+) - y(0^-) = 1
\]
holds, exactly matching the effect of the delta. Therefore $y(t)=H(t)e^{-at}$ is indeed the solution of
\[
y'(t)+a\,y(t)=\delta(t)
\]
with $y(t)=0$ for $t<0$.

\medskip

\textbf{4. Impulse response, Green's function, and convolution.}

The function
\[
G(t) := H(t)e^{-at}
\]
is called the \emph{impulse response} of the system, because it describes how the system responds in time to a unit impulse applied at $t=0$. In the language of linear operators, $G$ is the \emph{Green's function} for the operator $L[y]=y'+ay$ on the real line, subject to the causality condition $y(t)=0$ for $t<0$.

To see how this generalizes to an arbitrary forcing $f$, consider
\[
y'(t) + a\,y(t) = f(t),
\]
with the same condition $y(t)=0$ for $t<0$. Taking Fourier transforms and using the same reasoning as before, we obtain
\[
i\omega\,\widehat{y}(\omega) + a\,\widehat{y}(\omega) = \widehat{f}(\omega),
\]
so that
\[
\widehat{y}(\omega) = \frac{1}{a+i\omega}\,\widehat{f}(\omega)
= \widehat{G}(\omega)\,\widehat{f}(\omega),
\]
where $\widehat{G}(\omega) = \dfrac{1}{a+i\omega}$ is the transform of $G$.

A central property of the Fourier transform is that multiplication in the frequency domain corresponds to convolution in the time domain. More precisely, if $\widehat{y} = \widehat{G}\,\widehat{f}$, then
\[
y = G * f,
\]
that is,
\[
y(t) = (G*f)(t) = \int_{-\infty}^{\infty} G(t-s)\,f(s)\,ds.
\]
This formula gives the solution for any integrable forcing $f$ in terms of the Green's function $G$. It says that the output at time $t$ is a superposition (integral) of shifted copies of the impulse response, each weighted by the input $f(s)$ at time $s$.

\medskip

\textbf{How this illustrates the role of the inverse transform.}

In this example, the Fourier transform converts the differential equation
\[
y' + ay = \text{forcing}
\]
into the algebraic equation
\[
(i\omega + a)\,\widehat{y} = \widehat{\text{forcing}}.
\]
Solving this algebraic equation is straightforward: it amounts to division by $i\omega + a$. The inverse Fourier transform then reconstructs the time-domain solution $y$ from its frequency-domain representation. In particular, in the impulse-forced case, the inverse transform recovers the Green's function $G(t)=H(t)e^{-at}$, and in the general forced case, it yields the convolution formula $y = G*f$. Thus this example clearly exhibits the main ideas of the section: the Fourier transform linearizes differentiation into multiplication by $i\omega$, and the inverse transform and convolution theorem translate simple algebraic manipulations in frequency into explicit solution formulas in time.
\end{solution}

% ===== Example 4: Convolution, Filtering, and the Convolution Theorem (inquiry-based) =====
\begin{problem}[Convolution, Filtering, and the Convolution Theorem]
In signal processing and in models of heat flow, one often wishes to ``smooth'' or ``blur'' a noisy input signal. A standard way to do this is to average the signal against a fixed \emph{kernel} that describes how much nearby values influence each point. Mathematically, this smoothing is described by a \emph{convolution}. In this problem you will see how the Fourier transform converts convolution into simple multiplication, and how this explains the idea of a \emph{low-pass filter} that damps high frequencies.

Throughout, use the Fourier transform convention
\[
\widehat{f}(\xi) \;=\; \int_{-\infty}^{\infty} f(x)\, e^{-2\pi i x \xi}\,dx,
\qquad
f(x) \;=\; \int_{-\infty}^{\infty} \widehat{f}(\xi)\, e^{2\pi i x \xi}\,d\xi,
\]
whenever these integrals make sense.

Consider a one-dimensional signal $f \colon \mathbb{R} \to \mathbb{R}$ (for example, the temperature along a long thin rod, or the intensity along a line in an image). We smooth $f$ by convolving it with the kernel
\[
k(x) \;=\; \tfrac{1}{2} e^{-|x|}, \qquad x \in \mathbb{R}.
\]

\medskip

(a) The \emph{convolution} of $k$ and $f$ is defined by
\[
(k * f)(x) \;=\; \int_{-\infty}^{\infty} k(x-y)\, f(y)\,dy,
\]
whenever the integral converges.

\quad(i) Write explicitly the formula for the smoothed signal
\[
u(x) \;=\; (k * f)(x)
\]
using the given $k(x) = \tfrac12 e^{-|x|}$.

\quad(ii) Give a physical interpretation of this formula in words. In particular, explain the role of the factor $e^{-|x-y|}$ in terms of how much a value $f(y)$ influences the output at position $x$.

\medskip

(b) Compute the Fourier transform of the kernel $k$.

\quad(i) Show that $k$ is an even function. Rewrite $\widehat{k}(\xi)$ as an integral over $[0,\infty)$ using this symmetry.

\quad(ii) Show that
\[
\widehat{k}(\xi)
= \int_{-\infty}^{\infty} \tfrac12 e^{-|x|} e^{-2\pi i x \xi}\,dx
= \int_{0}^{\infty} e^{-x} \cos(2\pi \xi x)\,dx.
\]

\quad(iii) Evaluate this last integral and deduce a simple closed form for $\widehat{k}(\xi)$.

\emph{Hint:} You may use, without proof, that for real numbers $a>0$ and $b$,
\[
\int_{0}^{\infty} e^{-a x} \cos(bx)\,dx \;=\; \frac{a}{a^{2} + b^{2}}.
\]

\medskip

(c) Let $u = k * f$ as in part (a). Assume $f$ is nice enough that all integrals converge absolutely and that Fubini's theorem justifies interchanging the order of integration.

\quad(i) Write down an integral formula for $\widehat{u}(\xi)$ in terms of $u(x)$.

\quad(ii) Substitute the definition
\[
u(x) = \int_{-\infty}^{\infty} k(x-y) f(y)\,dy
\]
into the expression for $\widehat{u}(\xi)$. Carefully interchange the order of integration and simplify.

\quad(iii) Show that
\[
\widehat{u}(\xi) \;=\; \widehat{k}(\xi)\, \widehat{f}(\xi).
\]

\emph{Hint:} After interchanging integrals, look for an inner integral that is exactly the Fourier transform of $k$ (or of $f$) with respect to the correct variable, perhaps after a simple change of variables.

\medskip

(d) We now interpret this in terms of \emph{filtering} in the frequency domain.

\quad(i) Combine part (b) and part (c) to obtain an explicit formula for $\widehat{u}(\xi)$ in terms of $\widehat{f}(\xi)$, involving a factor of the form
\[
\frac{1}{1 + 4\pi^{2} \xi^{2}}.
\]

\quad(ii) Explain how the absolute value $|\widehat{u}(\xi)|$ compares to $|\widehat{f}(\xi)|$ for low frequencies (say $|\xi| \ll 1$) and for high frequencies (say $|\xi| \gg 1$).

\quad(iii) Using the inverse Fourier transform, write an integral formula for $u(x)$ in terms of $\widehat{f}(\xi)$ and this frequency factor. Explain in words how multiplying $\widehat{f}(\xi)$ by $\dfrac{1}{1+4\pi^{2}\xi^{2}}$ corresponds to the smoothing or blurring of $f$ in the original $x$-variable.

\emph{Hint:} Think about what happens to a single pure frequency $f(x) = e^{2\pi i \xi_{0} x}$. How is its amplitude changed by convolution with $k$?

\medskip

(e) (Extensions and ``what if'' questions.)

\quad(i) Consider a family of kernels
\[
k_{\varepsilon}(x) = \frac{1}{2\varepsilon} e^{-|x|/\varepsilon}, \qquad \varepsilon > 0.
\]
Without doing every integral in detail, use a suitable change of variables to guess a formula for $\widehat{k_{\varepsilon}}(\xi)$ in terms of $\widehat{k}(\xi)$ from part (b). How does the parameter $\varepsilon$ affect the decay of $\widehat{k_{\varepsilon}}(\xi)$ in $\xi$?

\quad(ii) Based on your answer to (i), predict what happens to the amount of smoothing (or blurring) in physical space as $\varepsilon$ becomes very small and as $\varepsilon$ becomes very large.

\quad(iii) Suppose instead you decide to filter directly in the frequency domain by defining
\[
\widehat{u}(\xi) = \mathbf{1}_{[-\Lambda,\Lambda]}(\xi)\, \widehat{f}(\xi),
\]
where $\mathbf{1}_{[-\Lambda,\Lambda]}$ is the indicator function of the interval $[-\Lambda,\Lambda]$. Qualitatively, what kind of kernel $k$ in physical space would correspond to this \emph{ideal low-pass filter}? How might the resulting $u(x)$ look different from the exponentially smoothed version you studied above?

\end{problem}

% ===== Example 4: Convolution, Filtering, and the Convolution Theorem (full solution) =====
\begin{problem}[Convolution, Filtering, and the Convolution Theorem]
Let $f \colon \mathbb{R} \to \mathbb{C}$ be an integrable function, and let
\[
k(x) = \tfrac12 e^{-|x|}, \qquad x \in \mathbb{R}.
\]
Define the smoothed signal $u = k * f$ by
\[
u(x) = (k * f)(x) = \int_{-\infty}^{\infty} k(x-y)\,f(y)\,dy.
\]
Assume all integrals converge absolutely and Fubini's theorem applies. Use the Fourier transform
\[
\widehat{g}(\xi) = \int_{-\infty}^{\infty} g(x)\, e^{-2\pi i x \xi}\,dx,
\qquad
g(x) = \int_{-\infty}^{\infty} \widehat{g}(\xi)\, e^{2\pi i x \xi}\,d\xi.
\]
\begin{enumerate}
\item Compute $\widehat{k}(\xi)$ explicitly.
\item Show that $\widehat{u}(\xi) = \widehat{k}(\xi)\,\widehat{f}(\xi)$ and hence
\[
\widehat{u}(\xi) = \frac{1}{1 + 4\pi^{2}\xi^{2}}\, \widehat{f}(\xi).
\]
\item Use the inverse Fourier transform to write $u(x)$ in terms of $\widehat{f}(\xi)$, and interpret $\dfrac{1}{1+4\pi^{2}\xi^{2}}$ as a frequency filter. Explain briefly why this describes a smoothing (low-pass) filter in physical space.
\item Consider the rescaled kernel
\[
k_{\varepsilon}(x) = \frac{1}{2\varepsilon} e^{-|x|/\varepsilon}, \qquad \varepsilon>0.
\]
Relate $\widehat{k_{\varepsilon}}(\xi)$ to $\widehat{k}(\xi)$ and discuss how $\varepsilon$ controls the strength of smoothing.
\end{enumerate}
\end{problem}

\begin{solution}
We begin by computing the Fourier transform of the kernel $k$ and then show how the convolution theorem converts convolution into multiplication in frequency space. This explicitly exhibits $k$ as a low-pass filter.

\medskip

\textbf{1. Computation of $\widehat{k}(\xi)$.}

We are given
\[
k(x) = \tfrac12 e^{-|x|}.
\]
First observe that $k$ is an even function, because $|{-x}| = |x|$, so
\[
k(-x) = \tfrac12 e^{-|{-x}|} = \tfrac12 e^{-|x|} = k(x).
\]
Therefore
\[
\widehat{k}(\xi)
= \int_{-\infty}^{\infty} \tfrac12 e^{-|x|} e^{-2\pi i x \xi}\,dx
= \tfrac12 \int_{-\infty}^{\infty} e^{-|x|} e^{-2\pi i x \xi}\,dx.
\]
Since $e^{-|x|}$ is even and $e^{-2\pi i x \xi}$ has real part $\cos(2\pi\xi x)$ and imaginary part $-i\sin(2\pi\xi x)$, the imaginary part of the integral vanishes (it is odd), and the integral equals twice its restriction to $[0,\infty)$ with the cosine:
\[
\int_{-\infty}^{\infty} e^{-|x|} e^{-2\pi i x \xi}\,dx
= 2 \int_{0}^{\infty} e^{-x} \cos(2\pi \xi x)\,dx.
\]
Thus
\[
\widehat{k}(\xi)
= \tfrac12 \cdot 2 \int_{0}^{\infty} e^{-x} \cos(2\pi \xi x)\,dx
= \int_{0}^{\infty} e^{-x} \cos(2\pi \xi x)\,dx.
\]

We now use the standard integral formula, valid for $a>0$ and real $b$,
\[
\int_{0}^{\infty} e^{-a x} \cos(bx)\,dx = \frac{a}{a^{2} + b^{2}}.
\]
Here $a = 1$ and $b = 2\pi \xi$, so
\[
\int_{0}^{\infty} e^{-x} \cos(2\pi \xi x)\,dx
= \frac{1}{1 + (2\pi \xi)^{2}}
= \frac{1}{1 + 4\pi^{2} \xi^{2}}.
\]
Therefore
\[
\boxed{\widehat{k}(\xi) = \frac{1}{1 + 4\pi^{2} \xi^{2}}.}
\]

This formula already suggests that $k$ is a smoothing kernel: its Fourier transform decays as $|\xi|$ increases.

\medskip

\textbf{2. Convolution theorem and $\widehat{u}(\xi)$.}

We define the smoothed signal $u$ by
\[
u(x) = (k * f)(x) = \int_{-\infty}^{\infty} k(x-y)\,f(y)\,dy.
\]
We compute $\widehat{u}(\xi)$ directly from the definition:
\[
\widehat{u}(\xi)
= \int_{-\infty}^{\infty} u(x)\, e^{-2\pi i x \xi}\,dx
= \int_{-\infty}^{\infty} \left( \int_{-\infty}^{\infty} k(x-y)\, f(y)\,dy \right)
    e^{-2\pi i x \xi}\,dx.
\]
By the assumed absolute convergence and Fubini's theorem, we may interchange the order of integration:
\[
\widehat{u}(\xi)
= \int_{-\infty}^{\infty} f(y)
   \left( \int_{-\infty}^{\infty} k(x-y)\, e^{-2\pi i x \xi}\,dx \right) dy.
\]

We focus on the inner integral, which is a Fourier transform of $k$ shifted by $y$. Make the change of variables $z = x-y$, so $x = z + y$ and $dx = dz$. Then
\[
\int_{-\infty}^{\infty} k(x-y)\, e^{-2\pi i x \xi}\,dx
= \int_{-\infty}^{\infty} k(z)\, e^{-2\pi i (z+y) \xi}\,dz
= e^{-2\pi i y \xi} \int_{-\infty}^{\infty} k(z)\, e^{-2\pi i z \xi}\,dz.
\]
The integral in $z$ is exactly $\widehat{k}(\xi)$, so we obtain
\[
\int_{-\infty}^{\infty} k(x-y)\, e^{-2\pi i x \xi}\,dx
= e^{-2\pi i y \xi}\, \widehat{k}(\xi).
\]

Substituting back, we find
\[
\widehat{u}(\xi)
= \int_{-\infty}^{\infty} f(y)\,
   \bigl( e^{-2\pi i y \xi}\, \widehat{k}(\xi) \bigr)\, dy
= \widehat{k}(\xi) \int_{-\infty}^{\infty} f(y)\, e^{-2\pi i y \xi}\,dy
= \widehat{k}(\xi)\, \widehat{f}(\xi).
\]
Thus we have verified the convolution theorem in this setting:
\[
\boxed{\widehat{u}(\xi) = \widehat{k}(\xi)\, \widehat{f}(\xi).}
\]

Using the formula from part 1,
\[
\widehat{k}(\xi) = \frac{1}{1 + 4\pi^{2}\xi^{2}},
\]
we obtain the explicit relation
\[
\boxed{\widehat{u}(\xi)
= \frac{1}{1 + 4\pi^{2}\xi^{2}}\, \widehat{f}(\xi).}
\]

\medskip

\textbf{3. Inverse transform, filter interpretation, and smoothing.}

By the inverse Fourier transform formula, we can write
\[
u(x)
= \int_{-\infty}^{\infty} \widehat{u}(\xi)\, e^{2\pi i x \xi}\,d\xi
= \int_{-\infty}^{\infty}
   \frac{1}{1 + 4\pi^{2}\xi^{2}}\,\widehat{f}(\xi)\, e^{2\pi i x \xi}\,d\xi.
\]
Thus $u(x)$ is obtained by taking the Fourier transform of $f$, multiplying by the factor
\[
H(\xi) = \frac{1}{1 + 4\pi^{2}\xi^{2}},
\]
and then applying the inverse transform.

The function $H(\xi)$ is called the \emph{transfer function} or \emph{frequency response} of the filter. We analyze $H$:

\begin{itemize}
\item For low frequencies $|\xi| \ll 1$, we have $4\pi^{2}\xi^{2} \approx 0$, so
\[
H(\xi) \approx 1.
\]
This means low-frequency components of $f$ are transmitted with little change in amplitude.

\item For high frequencies $|\xi| \gg 1$, the denominator $1 + 4\pi^{2}\xi^{2}$ is large, so
\[
H(\xi) \approx \frac{1}{4\pi^{2}\xi^{2}} \ll 1.
\]
High-frequency components of $f$ are strongly attenuated.
\end{itemize}

To see this effect on a single pure frequency, consider $f(x) = e^{2\pi i \xi_{0} x}$, so that $\widehat{f}(\xi)$ is (up to a constant) a delta distribution at $\xi = \xi_{0}$. Then
\[
\widehat{u}(\xi) = H(\xi)\, \widehat{f}(\xi)
\]
has the same support at $\xi_{0}$ but its amplitude is multiplied by $H(\xi_{0}) = 1/(1+4\pi^{2}\xi_{0}^{2})$. When we invert the transform, we find that $u(x)$ is still a pure exponential of frequency $\xi_{0}$, but with reduced amplitude. The reduction is small for small $|\xi_{0}|$ and large for large $|\xi_{0}|$.

Thus the convolution with $k$ acts as a \emph{low-pass filter}: it largely preserves slow variations (low frequencies) while damping out rapid oscillations (high frequencies). In the original $x$-domain, this appears as smoothing or blurring of the signal $f$.

From the perspective of this chapter, the central idea is that the Fourier transform converts the relatively complicated operation of convolution
\[
u(x) = \int_{-\infty}^{\infty} k(x-y)\, f(y)\,dy
\]
into simple pointwise multiplication
\[
\widehat{u}(\xi) = \widehat{k}(\xi)\, \widehat{f}(\xi).
\]
The inverse transform then reconstructs $u(x)$ from its filtered frequency content.

\medskip

\textbf{4. The rescaled kernel $k_{\varepsilon}$ and strength of smoothing.}

Let
\[
k_{\varepsilon}(x) = \frac{1}{2\varepsilon} e^{-|x|/\varepsilon}, \qquad \varepsilon>0.
\]
We recognize this as a rescaled version of $k$. Indeed, note that
\[
k_{\varepsilon}(x) = \frac{1}{\varepsilon} k\!\left(\frac{x}{\varepsilon}\right),
\]
because
\[
\frac{1}{\varepsilon} k\!\left(\frac{x}{\varepsilon}\right)
= \frac{1}{\varepsilon} \left( \tfrac12 e^{-|x|/\varepsilon} \right)
= \frac{1}{2\varepsilon} e^{-|x|/\varepsilon}.
\]

There is a standard scaling property of the Fourier transform: if
\[
g(x) = \frac{1}{|a|} f\!\left(\frac{x}{a}\right) \quad\text{for } a\neq 0,
\]
then
\[
\widehat{g}(\xi) = \widehat{f}(a\xi).
\]
We apply this with $a = \varepsilon$ and $f = k$, so
\[
\widehat{k_{\varepsilon}}(\xi) = \widehat{k}(\varepsilon \xi).
\]
Since we already computed
\[
\widehat{k}(\xi) = \frac{1}{1 + 4\pi^{2}\xi^{2}},
\]
we obtain
\[
\boxed{\widehat{k_{\varepsilon}}(\xi)
= \frac{1}{1 + 4\pi^{2} (\varepsilon \xi)^{2}}
= \frac{1}{1 + 4\pi^{2}\varepsilon^{2}\xi^{2}}.}
\]

The frequency response is now
\[
H_{\varepsilon}(\xi) = \frac{1}{1 + 4\pi^{2}\varepsilon^{2}\xi^{2}}.
\]
We interpret $\varepsilon$:

\begin{itemize}
\item If $\varepsilon$ is \emph{small}, then for a given $\xi$ the quantity $4\pi^{2}\varepsilon^{2}\xi^{2}$ is small except at very high frequencies. Thus $H_{\varepsilon}(\xi) \approx 1$ on a large range of $\xi$ and only decays significantly when $|\xi|$ is of order $1/\varepsilon$ or larger. In other words, a small $\varepsilon$ corresponds to a \emph{weak filter}: most frequencies pass through, and the smoothing in $x$ is mild. Indeed, $k_{\varepsilon}$ becomes sharply peaked near $0$, approaching a delta function as $\varepsilon \to 0$, so $k_{\varepsilon} * f$ approaches $f$.

\item If $\varepsilon$ is \emph{large}, then for moderate $|\xi|$ the factor $4\pi^{2}\varepsilon^{2}\xi^{2}$ is already large, so $H_{\varepsilon}(\xi)$ decays rapidly away from $\xi=0$. Only very low frequencies (very slow variations) are transmitted with significant amplitude. Thus a large $\varepsilon$ corresponds to a \emph{strong filter}: high frequencies are heavily damped, and the signal in $x$ is very smooth and blurred. In physical space, $k_{\varepsilon}$ is wide and spreads the influence of each point over a large neighborhood.
\end{itemize}

This example encapsulates the main ideas of the section on the Fourier transform and its inverse: convolution in physical space corresponds to multiplication in frequency space, and the shape of the Fourier transform of the kernel determines which frequencies are transmitted or suppressed. By computing both the transform and its inverse, we obtain a clear picture of how modifying the frequency content results in smoothing of the original signal.

\end{solution}

% ===== Example 5: Fourier Transform of Derivatives and Regularity from Decay (inquiry-based) =====
\begin{problem}[Fourier Transform of Derivatives and Regularity from Decay]
In this problem we explore how the Fourier transform interacts with differentiation and how decay of the Fourier transform at high frequencies is reflected as smoothness of the original function. The key observation is that differentiation in the physical variable corresponds to multiplication by a polynomial in the frequency variable. Using the inverse transform, this observation can be turned around: polynomial decay of the Fourier transform forces regularity of the original function. These ideas lie at the heart of solving linear constant–coefficient PDEs by Fourier methods.

Throughout, work on the real line $\mathbb{R}$ with the Fourier transform
\[
\widehat{f}(\xi) \;=\; \int_{\mathbb{R}} f(x)\,e^{-2\pi i x \xi}\,dx,
\]
whenever this integral is absolutely convergent, and the inverse transform
\[
f(x) \;=\; \int_{\mathbb{R}} \widehat{f}(\xi)\,e^{2\pi i x \xi}\,d\xi,
\]
for functions for which this formula is valid (for instance, Schwartz functions).

\smallskip

(a) Suppose $f \colon \mathbb{R}\to\mathbb{R}$ is continuously differentiable, $f$ and $f'$ belong to $L^1(\mathbb{R})$, and
\[
\lim_{x\to\pm\infty} f(x) = 0.
\]
Start from the definition
\[
\widehat{f'}(\xi) = \int_{\mathbb{R}} f'(x)\,e^{-2\pi i x\xi}\,dx.
\]
Use integration by parts to express $\widehat{f'}(\xi)$ in terms of $\widehat{f}(\xi)$ and $\xi$. What is the exact formula you obtain?

\emph{Hint:} Treat $u = f(x)$ and $dv = e^{-2\pi i x\xi}\,dx$. Carefully compute $v$ and check what happens to the boundary term using the assumption on $f(x)$ at $\pm\infty$.

\smallskip

(b) Under the same hypotheses, consider the $k$-th derivative $f^{(k)}$ (assume it exists, belongs to $L^1(\mathbb{R})$, and still tends to $0$ at $\pm\infty$). Guess a formula for $\widehat{f^{(k)}}(\xi)$ in terms of $\widehat{f}(\xi)$ and $\xi$, and then prove your guess by induction on $k$.

How does this show that a constant–coefficient differential operator like
\[
L = a_0 + a_1 \frac{d}{dx} + a_2 \frac{d^2}{dx^2} + \cdots + a_m \frac{d^m}{dx^m}
\]
is turned by the Fourier transform into multiplication by a polynomial in $\xi$?

\emph{Hint:} Combine your formula for $\widehat{f^{(k)}}$ with linearity of the Fourier transform.

\smallskip

(c) Now reverse the direction and use the \emph{inverse} Fourier transform to see how decay in $\widehat{f}$ produces smoothness of $f$. Assume that $\widehat{f}\in L^1(\mathbb{R})$ and also $\xi\,\widehat{f}(\xi)\in L^1(\mathbb{R})$. Then we may write
\[
f(x) = \int_{\mathbb{R}} \widehat{f}(\xi)\,e^{2\pi i x\xi}\,d\xi.
\]
Explain why the function
\[
g(x) := \int_{\mathbb{R}} 2\pi i \xi\,\widehat{f}(\xi)\,e^{2\pi i x\xi}\,d\xi
\]
is well-defined and continuous on $\mathbb{R}$. Show that you can differentiate the inverse Fourier integral for $f(x)$ under the integral sign and conclude that
\[
f'(x) = g(x).
\]
In particular, deduce that $f$ is continuously differentiable and that its derivative can be reconstructed from $\widehat{f}$.

\emph{Hint:} Use the dominated convergence theorem to justify passing the derivative inside the integral, and use the integrability of $\xi\,\widehat{f}(\xi)$ to find an integrable majorant.

\smallskip

(d) Generalize the previous step. Let $m\geq 1$ be an integer. Assume that for each $k=0,1,\dots,m$ we have
\[
\xi^k\,\widehat{f}(\xi)\in L^1(\mathbb{R}).
\]
Show that $f$ has $m$ continuous derivatives and that
\[
f^{(k)}(x) = \int_{\mathbb{R}} (2\pi i \xi)^k \,\widehat{f}(\xi)\,e^{2\pi i x\xi}\,d\xi
\quad\text{for } k=0,1,\dots,m.
\]
Explain how this proves the general principle that \emph{polynomial decay} of $\widehat{f}$ as $|\xi|\to\infty$ implies \emph{polynomial regularity} (that is, differentiability) of $f$.

\emph{Hint:} Repeat the argument from part (c) inductively. At each step, you will need integrability of the next power $\xi^{k}\widehat{f}(\xi)$ to justify differentiating under the integral sign one more time.

\smallskip

(e) (Explorations and extensions.)

\begin{itemize}
  \item[(i)] Suppose that for \emph{every} integer $k\geq 0$ the function $\xi^k\,\widehat{f}(\xi)$ belongs to $L^1(\mathbb{R})$. What can you say about the smoothness of $f$? What does your conclusion suggest about the relation between “rapid decay’’ of $\widehat{f}$ and infinite differentiability of $f$?

  \item[(ii)] Let $u(t,x)$ solve the linear PDE
  \[
  \partial_t u(t,x) + \partial_x^{4} u(t,x) = 0, \qquad t>0,\ x\in\mathbb{R},
  \]
  with initial condition $u(0,x) = f(x)$, where $f$ is sufficiently nice so that all the Fourier transforms you need exist. Take the Fourier transform in $x$ and derive an explicit formula for $\widehat{u}(t,\xi)$ in terms of $\widehat{f}(\xi)$. Using your formula and the results above, explain why $u(t,\cdot)$ is very smooth in $x$ for every $t>0$, even if $f$ is not smooth. How does the factor $e^{-t(2\pi\xi)^4}$ in frequency space help you?
  
  \emph{Hint:} Think about how multiplying $\widehat{f}(\xi)$ by $e^{-t(2\pi\xi)^4}$ affects the decay of the product as $|\xi|\to\infty$, and then invoke part (d).
\end{itemize}

\end{problem}

% ===== Example 5: Fourier Transform of Derivatives and Regularity from Decay (full solution) =====
\begin{problem}[Fourier Transform of Derivatives and Regularity from Decay]
On $\mathbb{R}$, use the Fourier transform
\[
\widehat{f}(\xi) = \int_{\mathbb{R}} f(x)\,e^{-2\pi i x\xi}\,dx
\]
and its inverse
\[
f(x) = \int_{\mathbb{R}} \widehat{f}(\xi)\,e^{2\pi i x\xi}\,d\xi,
\]
whenever these are defined.

\begin{enumerate}
  \item[(a)] Let $f$ be continuously differentiable with $f, f'\in L^1(\mathbb{R})$ and $\lim_{x\to\pm\infty} f(x)=0$. Show that
  \[
  \widehat{f'}(\xi) = 2\pi i \xi\,\widehat{f}(\xi).
  \]
  Extend this to $k$-th derivatives: under analogous hypotheses on $f^{(k)}$, prove
  \[
  \widehat{f^{(k)}}(\xi) = (2\pi i \xi)^k\,\widehat{f}(\xi).
  \]
  Conclude that a constant–coefficient differential operator
  \(
    L = \sum_{j=0}^m a_j \frac{d^j}{dx^j}
  \)
  is transformed into multiplication by the polynomial $\sum_{j=0}^m a_j (2\pi i \xi)^j$.

  \item[(b)] Now suppose $\widehat{f}\in L^1(\mathbb{R})$ and, for $k=0,1,\dots,m$,
  \[
  \xi^k\,\widehat{f}(\xi)\in L^1(\mathbb{R}).
  \]
  Using the inverse transform and differentiation under the integral sign, prove that $f$ has $m$ continuous derivatives and that
  \[
  f^{(k)}(x) = \int_{\mathbb{R}} (2\pi i \xi)^k\,\widehat{f}(\xi)\,e^{2\pi i x\xi}\,d\xi
  \quad\text{for } k=0,1,\dots,m.
  \]
  Explain briefly how this shows that polynomial decay of $\widehat{f}(\xi)$ as $|\xi|\to\infty$ implies polynomial regularity (differentiability) of $f$.

  \item[(c)] Consider the PDE
  \[
  \partial_t u(t,x) + \partial_x^{4} u(t,x) = 0, \quad t>0,\ x\in\mathbb{R},
  \qquad u(0,x) = f(x),
  \]
  with $f$ such that all relevant Fourier transforms exist. Take the Fourier transform in $x$ to find an explicit formula for $\widehat{u}(t,\xi)$ in terms of $\widehat{f}(\xi)$. Using part (b), explain why $u(t,\cdot)$ is very smooth in $x$ for each $t>0$, even if $f$ is not smooth.
\end{enumerate}
\end{problem}

\begin{solution}
We work in one dimension, with the convention
\[
\widehat{f}(\xi) = \int_{\mathbb{R}} f(x)\,e^{-2\pi i x\xi}\,dx,
\qquad
f(x) = \int_{\mathbb{R}} \widehat{f}(\xi)\,e^{2\pi i x\xi}\,d\xi,
\]
for sufficiently nice functions (for example, Schwartz functions). The central ideas are that differentiation in $x$ becomes multiplication in $\xi$, and that decay in $\xi$ corresponds to smoothness in $x$. These are the core structural properties of the Fourier transform used throughout the section.

\medskip

\noindent\textbf{(a) Fourier transform of derivatives.}

Assume first that $f$ is continuously differentiable, that $f$ and $f'$ belong to $L^1(\mathbb{R})$, and that $\lim_{x\to\pm\infty} f(x) = 0$. Then
\[
\widehat{f'}(\xi) = \int_{\mathbb{R}} f'(x)\,e^{-2\pi i x\xi}\,dx.
\]
We integrate by parts with
\[
u = f(x), \quad dv = e^{-2\pi i x\xi}\,dx,
\]
so that
\[
du = f'(x)\,dx, \quad v = \frac{1}{-2\pi i \xi} e^{-2\pi i x\xi}
\quad (\xi\neq 0).
\]
For $\xi\neq 0$, we obtain
\[
\int_{\mathbb{R}} f'(x)\,e^{-2\pi i x\xi}\,dx
= \biggl[ f(x)\,\frac{1}{-2\pi i \xi} e^{-2\pi i x\xi} \biggr]_{x=-\infty}^{x=+\infty}
+ 2\pi i \xi \int_{\mathbb{R}} f(x)\,e^{-2\pi i x\xi}\,dx.
\]
The boundary term vanishes because $f(x)\to 0$ as $x\to\pm\infty$ and the exponential factor has modulus one. Thus
\[
\widehat{f'}(\xi) = 2\pi i \xi \,\widehat{f}(\xi), \qquad \xi\neq 0.
\]
At $\xi=0$, we have
\[
\widehat{f'}(0) = \int_{\mathbb{R}} f'(x)\,dx = \lim_{R\to\infty} \bigl( f(R)-f(-R)\bigr) = 0,
\]
while $2\pi i \xi\widehat{f}(\xi)\big|_{\xi=0} = 0$, so the formula also holds at $\xi=0$. Hence
\[
\widehat{f'}(\xi) = 2\pi i \xi\,\widehat{f}(\xi)
\quad\text{for all } \xi\in\mathbb{R}.
\]

For higher derivatives, assume $f$ has $k$ continuous derivatives, each belonging to $L^1(\mathbb{R})$, and that $f^{(j)}(x)\to 0$ as $x\to\pm\infty$ for $0\le j\le k-1$. We proceed by induction on $k$.

For $k=1$ we have just proved the formula. Assume that
\[
\widehat{f^{(k)}}(\xi) = (2\pi i\xi)^k\,\widehat{f}(\xi)
\]
holds for some $k\ge 1$. Then, applying the $k=1$ case to $f^{(k)}$, we obtain
\[
\widehat{f^{(k+1)}}(\xi) = 2\pi i \xi\,\widehat{f^{(k)}}(\xi)
= 2\pi i \xi \,(2\pi i\xi)^k\,\widehat{f}(\xi)
= (2\pi i \xi)^{k+1}\,\widehat{f}(\xi).
\]
Thus by induction,
\[
\widehat{f^{(k)}}(\xi) = (2\pi i \xi)^k\,\widehat{f}(\xi),
\qquad k=0,1,2,\dots
\]
whenever the derivatives exist, are integrable, and have suitable decay at infinity.

Now consider a constant–coefficient differential operator
\[
L = a_0 + a_1 \frac{d}{dx} + a_2 \frac{d^2}{dx^2} + \cdots + a_m \frac{d^m}{dx^m}.
\]
For such $f$, linearity of the Fourier transform and the formula above give
\[
\widehat{Lf}(\xi)
= \sum_{j=0}^m a_j\,\widehat{f^{(j)}}(\xi)
= \sum_{j=0}^m a_j\,(2\pi i \xi)^j\,\widehat{f}(\xi)
= p(2\pi i\xi)\,\widehat{f}(\xi),
\]
where $p(z)=\sum_{j=0}^m a_j z^j$ is the polynomial associated with $L$. Thus $L$ is converted into multiplication by $p(2\pi i\xi)$ in the frequency variable. This diagonalizes constant–coefficient operators and is a main reason the Fourier transform is so effective for solving linear ODEs and PDEs.

\medskip

\noindent\textbf{(b) Decay of $\widehat{f}$ and regularity of $f$.}

Assume that $\widehat{f}\in L^1(\mathbb{R})$ and that for each integer $k$ with $0\le k\le m$,
\[
\xi^k\,\widehat{f}(\xi)\in L^1(\mathbb{R}).
\]
This means that not only is $\widehat{f}$ integrable, but so are its polynomially weighted versions up to order $m$. We will use the inverse transform
\[
f(x) = \int_{\mathbb{R}} \widehat{f}(\xi)\,e^{2\pi i x\xi}\,d\xi
\]
and differentiate under the integral sign.

\medskip

\emph{Step 1: Continuity of $f$.}  

Because $\widehat{f}\in L^1(\mathbb{R})$, the integral defining $f(x)$ converges absolutely for each $x$. The integrand
\[
\widehat{f}(\xi)\,e^{2\pi i x\xi}
\]
is continuous in $x$ for each fixed $\xi$, and its modulus is bounded by $|\widehat{f}(\xi)|$, which is integrable. Therefore, by the dominated convergence theorem, $f$ is continuous in $x$.

\medskip

\emph{Step 2: First derivative.}

Assume, in addition, that $\xi\,\widehat{f}(\xi)\in L^1(\mathbb{R})$. Consider the difference quotient
\[
\frac{f(x+h)-f(x)}{h}
= \int_{\mathbb{R}} \widehat{f}(\xi)\,\frac{e^{2\pi i (x+h)\xi} - e^{2\pi i x\xi}}{h}\,d\xi.
\]
We can rewrite the factor in parentheses as
\[
\frac{e^{2\pi i (x+h)\xi} - e^{2\pi i x\xi}}{h}
= e^{2\pi i x\xi}\,\frac{e^{2\pi i h\xi}-1}{h}.
\]
For each fixed $\xi$, as $h\to 0$ we have
\[
\frac{e^{2\pi i h\xi}-1}{h} \longrightarrow 2\pi i \xi.
\]
Hence for each fixed $\xi$,
\[
\widehat{f}(\xi)\,\frac{e^{2\pi i (x+h)\xi} - e^{2\pi i x\xi}}{h}
\longrightarrow 2\pi i \xi\,\widehat{f}(\xi)\,e^{2\pi i x\xi}.
\]
We want to pass the limit inside the integral. To use dominated convergence, we estimate
\[
\biggl|\widehat{f}(\xi)\,\frac{e^{2\pi i (x+h)\xi} - e^{2\pi i x\xi}}{h}\biggr|
= |\widehat{f}(\xi)|\,\biggl|\frac{e^{2\pi i h\xi}-1}{h}\biggr|.
\]
The mean value theorem applied to the exponential shows that
\[
\biggl|\frac{e^{2\pi i h\xi}-1}{h}\biggr|
\le C\,|\xi|
\]
for some constant $C$ independent of $h$, for $h$ sufficiently small. Thus
\[
\biggl|\widehat{f}(\xi)\,\frac{e^{2\pi i (x+h)\xi} - e^{2\pi i x\xi}}{h}\biggr|
\le C\,|\xi|\,|\widehat{f}(\xi)|.
\]
By assumption, $\xi\,\widehat{f}(\xi)\in L^1(\mathbb{R})$, so the right-hand side is integrable and independent of $h$. Dominated convergence then yields
\[
\lim_{h\to 0}\frac{f(x+h)-f(x)}{h}
= \int_{\mathbb{R}} 2\pi i \xi\,\widehat{f}(\xi)\,e^{2\pi i x\xi}\,d\xi.
\]
Therefore $f$ is differentiable for each $x$, and its derivative is given by
\[
f'(x) = \int_{\mathbb{R}} (2\pi i \xi)\,\widehat{f}(\xi)\,e^{2\pi i x\xi}\,d\xi.
\]
The right-hand side defines a continuous function of $x$ by the same dominated convergence argument as before (now using integrability of $|\xi|\,|\widehat{f}(\xi)|$). Hence $f\in C^1(\mathbb{R})$.

\medskip

\emph{Step 3: Higher derivatives.}

We proceed by induction. Suppose that for some integer $\ell$ with $1\le \ell\le m$ we know that
\[
\xi^k\,\widehat{f}(\xi)\in L^1(\mathbb{R}) \quad\text{for } k=0,1,\dots,\ell,
\]
and that $f\in C^\ell(\mathbb{R})$ with
\[
f^{(k)}(x)
= \int_{\mathbb{R}} (2\pi i\xi)^k\,\widehat{f}(\xi)\,e^{2\pi i x\xi}\,d\xi,
\quad k=0,1,\dots,\ell.
\]
To obtain one more derivative, we consider $f^{(\ell)}$:
\[
f^{(\ell)}(x)
= \int_{\mathbb{R}} (2\pi i\xi)^{\ell}\,\widehat{f}(\xi)\,e^{2\pi i x\xi}\,d\xi.
\]
The integrand is continuous in $x$ for each $\xi$ and is dominated by
\[
|(2\pi\xi)^\ell \widehat{f}(\xi)|,
\]
which is integrable by hypothesis. Thus $f^{(\ell)}$ is continuous.

To differentiate once more, we compute the difference quotient
\[
\frac{f^{(\ell)}(x+h)-f^{(\ell)}(x)}{h}
= \int_{\mathbb{R}} (2\pi i\xi)^{\ell}\,\widehat{f}(\xi)\,
\frac{e^{2\pi i (x+h)\xi} - e^{2\pi i x\xi}}{h}\,d\xi.
\]
As before, for each fixed $\xi$,
\[
\frac{e^{2\pi i (x+h)\xi} - e^{2\pi i x\xi}}{h}
\longrightarrow 2\pi i \xi\,e^{2\pi i x\xi}
\quad\text{as } h\to 0.
\]
Hence the integrand converges pointwise to $(2\pi i\xi)^{\ell+1}\,\widehat{f}(\xi)\,e^{2\pi i x\xi}$. For the dominating function we now use
\[
\biggl|(2\pi i\xi)^{\ell}\,\widehat{f}(\xi)\,
\frac{e^{2\pi i (x+h)\xi}-e^{2\pi i x\xi}}{h}\biggr|
\le C\,|\xi|^{\ell+1} |\widehat{f}(\xi)|,
\]
which is integrable by the assumption that $\xi^{\ell+1}\,\widehat{f}(\xi)\in L^1(\mathbb{R})$. Dominated convergence yields
\[
f^{(\ell+1)}(x)
= \int_{\mathbb{R}} (2\pi i\xi)^{\ell+1}\,\widehat{f}(\xi)\,e^{2\pi i x\xi}\,d\xi.
\]
This shows that $f^{(\ell)}$ is differentiable with continuous derivative. Thus, by induction on $\ell$, we obtain that for all integers $k$ with $0\le k\le m$,
\[
f^{(k)}(x)
= \int_{\mathbb{R}} (2\pi i\xi)^k\,\widehat{f}(\xi)\,e^{2\pi i x\xi}\,d\xi
\]
and that $f\in C^m(\mathbb{R})$.

\medskip

\emph{Decay versus regularity.}

Suppose now that for some integer $m\ge 0$ we know that
\[
|\xi|^m\,|\widehat{f}(\xi)| \leq g(\xi)
\]
with $g\in L^1(\mathbb{R})$. Then $\xi^k\widehat{f}(\xi)\in L^1(\mathbb{R})$ for each $k\le m$, so the argument above applies. Informally, this means that if $\widehat{f}(\xi)$ decays at least like $|\xi|^{-m-1}$ as $|\xi|\to\infty$, then $f$ has $m$ continuous derivatives. Thus \emph{polynomial decay in frequency} corresponds to \emph{polynomial regularity in space}. This is a fundamental manifestation of the way the Fourier transform encodes smoothness of functions.

\medskip

\noindent\textbf{(c) Application to a fourth-order PDE.}

Consider the PDE
\[
\partial_t u(t,x) + \partial_x^{4} u(t,x) = 0, \quad t>0,\ x\in\mathbb{R},
\qquad u(0,x) = f(x).
\]
We take the Fourier transform in the spatial variable $x$:
\[
\widehat{u}(t,\xi) = \int_{\mathbb{R}} u(t,x)\,e^{-2\pi i x\xi}\,dx.
\]
Using the result of part (a) on derivatives, we know that
\[
\widehat{\partial_x^{4} u}(t,\xi) = (2\pi i\xi)^4 \widehat{u}(t,\xi) = (2\pi\xi)^4 \widehat{u}(t,\xi),
\]
since $i^4 = 1$. Therefore, taking the Fourier transform of the PDE with respect to $x$ yields
\[
\partial_t \widehat{u}(t,\xi) + (2\pi\xi)^4\,\widehat{u}(t,\xi) = 0,
\]
for each fixed $\xi\in\mathbb{R}$, with initial condition $\widehat{u}(0,\xi) = \widehat{f}(\xi)$. This is now an ordinary differential equation in the time variable $t$. Solving this ODE gives
\[
\widehat{u}(t,\xi) = e^{-t(2\pi\xi)^4}\,\widehat{f}(\xi).
\]

This formula exhibits the key algebraic simplification provided by the Fourier transform: the fourth derivative in $x$ has become multiplication by the polynomial $(2\pi\xi
)^4$.

Thus
\[
\widehat{u}(t,\xi) = e^{-t(2\pi\xi)^4}\,\widehat{f}(\xi).
\]

\medskip

\emph{Smoothing in $x$ for $t>0$.}

Fix $t>0$. For each integer $m\ge 0$,
\[
\xi^m\,\widehat{u}(t,\xi)
= \bigl[\xi^m e^{-t(2\pi\xi)^4}\bigr]\,\widehat{f}(\xi).
\]
The factor $\xi^m e^{-t(2\pi\xi)^4}$ is a polynomial times a rapidly decaying exponential. In particular, it is bounded:
\[
\sup_{\xi\in\mathbb{R}} \bigl|\xi^m e^{-t(2\pi\xi)^4}\bigr| < \infty,
\quad\text{for each fixed } t>0,m\ge 0.
\]
If, for example, $\widehat{f}\in L^1(\mathbb{R})$, then for each $m$,
\[
\xi^m\,\widehat{u}(t,\xi) \in L^1(\mathbb{R}),
\]
because it is $\widehat{f}(\xi)$ multiplied by a bounded function. Applying part (b) to $u(t,\cdot)$ (with $\widehat{u}(t,\xi)$ in place of $\widehat{f}(\xi)$) shows that $u(t,\cdot)$ has $m$ continuous derivatives for every $m$. Since $m$ is arbitrary, $u(t,\cdot)$ is $C^\infty$ in $x$ for each $t>0$.

In other words, even if the initial data $f$ is only moderately regular, the solution $u(t,\cdot)$ becomes instantly very smooth in the spatial variable for any positive time $t$. The strong decay of the factor $e^{-t(2\pi\xi)^4}$ at high frequencies damps the high-frequency components of $\widehat{f}$, and by the decay–regularity principle from part (b), this enhanced decay translates into arbitrarily high regularity of $u(t,\cdot)$ in $x$.

\end{solution}

\section{Properties of the 1-D Fourier Transform}
% TODO: Add narrative / plan for this section.

% TODO: Use prompts_for_sections.py to design examples and add them here.

\section{Dirac's delta-function}
% --- Narrative plan (auto-generated) ---
% This section introduces Dirac’s delta-function as a precise mathematical tool for representing idealized point sources, impulses, and localized measurements. Although the delta-function is not a function in the usual sense, it can be understood rigorously as a distribution, or generalized function, that acts on test functions by integration. We will explore several concrete ways to approximate it by honest functions and will use those approximations to justify its main algebraic and analytic properties.
%
% Dirac’s delta is indispensable in applied mathematics because it allows us to encode concentrated forces in ODE models, instantaneous inputs in dynamical systems, and point sources or sinks in PDEs such as the heat, wave, and Poisson equations. In the language of Fourier analysis, the delta provides a bridge between physical space and frequency space: it describes “pure” frequencies and exact localization, and it appears naturally when we study Green’s functions and impulse responses. The techniques developed here connect directly to convolution, Fourier transforms, Green’s functions for linear ODEs and PDEs, and, at a more advanced level, to complex analysis via contour integrals and to the theory of distributions used throughout modern analysis and partial differential equations.
%
% Throughout this section, we will build intuition by working with explicit approximating sequences (such as Gaussians and box functions), by computing how the delta interacts with smooth test functions, and by examining physical models where idealization by a point source is both natural and practically useful. The goal is that, by the end, you will be able to handle expressions involving δ confidently and correctly: using it in Fourier transforms, justifying manipulations with derivatives and integrals, and recognizing when and how it can be replaced by a limit of more familiar functions.

% ===== Example 1: Impulse Forcing of a Harmonic Oscillator (inquiry-based) =====
\begin{problem}[Impulse Forcing of a Harmonic Oscillator]
A mass–spring system with mass $m>0$ and spring constant $k>0$ is attached to a fixed wall and moves without friction on a horizontal surface. When left alone, the motion is governed by the homogeneous equation $m x''(t) + k x(t) = 0$. Now imagine we strike the mass with a very short, sharp blow at time $t=0$. The applied force is large but acts only for a very short time, so that its \emph{total impulse} is finite. We would like to model this “infinitely short but finite impulse’’ using the Dirac delta and understand the resulting motion.

Throughout, let $\omega = \sqrt{k/m}$ denote the natural frequency of the oscillator.

\smallskip

(a) \textbf{Warm-up: the unforced harmonic oscillator.}  
Consider the homogeneous initial value problem
\[
m x''(t) + k x(t) = 0, \qquad x(0) = x_0,\quad x'(0)=v_0 .
\]
(i) Solve this IVP explicitly for $x(t)$ in terms of $x_0$, $v_0$, and $\omega$.  
(ii) Describe qualitatively the motion for different choices of $(x_0,v_0)$ (for example, what happens if $x_0=0$ and $v_0\neq 0$?).

Hint: Write the general solution as a linear combination of $\cos(\omega t)$ and $\sin(\omega t)$ and then match the initial conditions.

\smallskip

(b) \textbf{From a short pulse to an impulse.}  
Instead of an idealized impulse, first consider a family of smooth forces $F_\varepsilon(t)$ that act only during a small time interval of length $\varepsilon>0$ near $t=0$, and are zero otherwise. Assume the total impulse (time-integrated force) is fixed:
\[
\int_{-\infty}^{\infty} F_\varepsilon(t)\,dt = J,
\]
for some constant $J$ independent of $\varepsilon$, and that $F_\varepsilon(t)$ tends to $0$ pointwise for each $t\neq 0$ as $\varepsilon\to 0^+$.  

Consider the forced equation
\[
m x_\varepsilon''(t) + k x_\varepsilon(t) = F_\varepsilon(t),
\]
with the “before impact’’ rest conditions
\[
x_\varepsilon(t) = 0,\quad x_\varepsilon'(t) = 0 \quad \text{for } t<0.
\]
(i) Integrate the equation from $t=-\eta$ to $t=\eta$ for some small fixed $\eta>0$ (independent of $\varepsilon$):
\[
\int_{-\eta}^{\eta} \bigl(m x_\varepsilon''(t) + k x_\varepsilon(t)\bigr)\,dt
= \int_{-\eta}^{\eta} F_\varepsilon(t)\,dt.
\]
Rewrite the left-hand side in terms of $x_\varepsilon'(\eta)$, $x_\varepsilon'(-\eta)$, and an integral involving $x_\varepsilon(t)$.  

(ii) Assuming $x_\varepsilon$ remains bounded as $\varepsilon\to 0^+$, argue that the contribution of the spring term $k x_\varepsilon(t)$ over $[-\eta,\eta]$ tends to $0$ when $\eta$ is small. What does this imply about the jump in velocity $x_\varepsilon'(\eta)-x_\varepsilon'(-\eta)$ as $\varepsilon\to 0^+$?

Hint: Use the fundamental theorem of calculus on the $m x_\varepsilon''$ term and estimate the integral of $k x_\varepsilon$ by a bound like $2\eta \cdot \max_{[-\eta,\eta]} |k x_\varepsilon(t)|$.

\smallskip

(c) \textbf{Introducing Dirac’s delta and the jump condition.}  
We now idealize the family of impulses $F_\varepsilon$ by Dirac’s delta and write the limiting forced equation as
\[
m x''(t) + k x(t) = J\,\delta(t),
\]
with $x(t)=0$ and $x'(t)=0$ for $t<0$.  

(i) Using the distributional identity
\[
\int_{-\eta}^{\eta} \delta(t)\,\varphi(t)\,dt = \varphi(0)
\quad\text{for any smooth test function }\varphi,
\]
integrate the equation from $t=-\eta$ to $t=\eta$ and pass to the limit as $\eta\to 0^+$.  

(ii) Show that $x(t)$ must be continuous at $t=0$ but that $x'(t)$ has a jump at $t=0$, and derive an explicit formula for
\[
x'(0^+) - x'(0^-)
\]
in terms of $J$ and $m$.

Hint: Adapt the computation from part (b) and replace $\int F_\varepsilon$ by $J$ via the defining property of $\delta$.

\smallskip

(d) \textbf{Solving the impulsively forced problem.}  
Assume that $x(t)=0$ and $x'(t)=0$ for $t<0$, and that $x$ satisfies
\[
m x''(t) + k x(t) = J\,\delta(t)
\]
in the sense of distributions. Using part (c), you now know the initial data \emph{just after} the impulse:
\[
x(0^+) = ?,\qquad x'(0^+) = ?.
\]
(i) Determine $x(0^+)$ and $x'(0^+)$ explicitly.  

(ii) For $t>0$, the forcing term vanishes, so $x$ solves the homogeneous equation. Use your answer from part (a) with initial data at $t=0^+$ to find an explicit formula for $x(t)$ for $t>0$.  

(iii) Express your final answer in a compact form using the Heaviside step function $H(t)$, where $H(t)=0$ for $t<0$ and $H(t)=1$ for $t>0$.

Hint: Think of the impulse as instantaneously changing the velocity, but not the position, at $t=0$.

\smallskip

(e) \textbf{Extensions and variations.}  

(i) Suppose instead that the impulse occurs at some later time $t=t_0>0$, so that
\[
m x''(t) + k x(t) = J\,\delta(t-t_0),\qquad x(t)=0,\ x'(t)=0\ \text{for }t<t_0.
\]
By analogy with part (d), guess the solution for $t>t_0$ and write it explicitly in terms of $H(t-t_0)$. How is this solution related to your earlier answer via a time shift?

(ii) The function you have just computed is often called the \emph{impulse response} or \emph{Green’s function} for the oscillator. Briefly explain, in your own words, why knowing this response to a single impulse allows us (in principle) to build solutions for more general forcing terms $f(t)$.

Hint: Think about approximating a general force $f(t)$ as a superposition (or integral) of many tiny impulses at different times.
\end{problem}

% ===== Example 1: Impulse Forcing of a Harmonic Oscillator (full solution) =====
\begin{problem}[Impulse Forcing of a Harmonic Oscillator]
Let $m>0$ and $k>0$ be constants, and let $\omega = \sqrt{k/m}$. Consider the impulsively forced harmonic oscillator
\[
m x''(t) + k x(t) = J\,\delta(t), \qquad t\in\mathbb{R},
\]
where $J$ is a given constant. Assume the mass is at rest before the impulse:
\[
x(t)=0,\quad x'(t)=0 \quad \text{for } t<0.
\]
\begin{enumerate}
\item[(a)] By integrating the equation across $t=0$, derive the jump condition for the velocity $x'(0^+)-x'(0^-)$.
\item[(b)] Use this jump condition to solve for $x(t)$ for all $t\in\mathbb{R}$ and express your answer using the Heaviside step function $H(t)$.
\item[(c)] Identify the resulting function as the impulse response (Green’s function) of the operator $Lx = m x'' + k x$, and briefly explain how this illustrates the role of Dirac’s delta in modeling instantaneous forcing.
\end{enumerate}
\end{problem}

\begin{solution}
We are studying a mass–spring system whose motion is governed by
\[
m x''(t) + k x(t) = J\,\delta(t),
\]
with the requirement that the mass is at rest before the impulse: $x(t)=0$ and $x'(t)=0$ for $t<0$. The right-hand side models an idealized impulse at time $t=0$ with total impulse $J$.

\medskip

\textbf{(a) Jump condition for the velocity.}  
To capture the effect of the delta forcing at $t=0$, we integrate the equation over a small symmetric interval $[-\eta,\eta]$ around $0$:
\[
\int_{-\eta}^{\eta} \bigl(m x''(t) + k x(t)\bigr)\,dt
= \int_{-\eta}^{\eta} J\,\delta(t)\,dt.
\]
We analyze each term separately.

First, apply the fundamental theorem of calculus to the $x''$ term:
\[
\int_{-\eta}^{\eta} m x''(t)\,dt
= m\bigl(x'(\eta) - x'(-\eta)\bigr).
\]
Second, since $x$ is locally bounded, the spring term contributes
\[
\int_{-\eta}^{\eta} k x(t)\,dt,
\]
which is at most $2\eta \,\max_{[-\eta,\eta]} |k x(t)|$ in magnitude. As $\eta\to 0^+$, this integral tends to $0$ because the interval length $2\eta$ goes to zero and $x$ remains bounded.

On the right-hand side, we use the defining property of the Dirac delta (with test function $\varphi(t)\equiv 1$):
\[
\int_{-\eta}^{\eta} J\,\delta(t)\,dt = J.
\]
Putting these pieces together and letting $\eta\to 0^+$ gives
\[
m\bigl(x'(0^+) - x'(0^-)\bigr) = J.
\]
Hence the velocity $x'(t)$ has a jump discontinuity at $t=0$ of size
\[
x'(0^+) - x'(0^-) = \frac{J}{m}.
\]
Physically, this says that an impulse of magnitude $J$ instantaneously changes the momentum $m x'$ by $J$.

Since the equation involves $x''$ and $x$ but not $x'$ directly, the position $x(t)$ itself remains continuous at $t=0$; only its derivative jumps.

\medskip

\textbf{(b) Solving for $x(t)$ and expressing the solution with $H(t)$.}  
By assumption, the mass is at rest before the impulse:
\[
x(t) = 0,\quad x'(t)=0 \quad \text{for } t<0.
\]
Thus, just before $t=0$ we have
\[
x(0^-) = 0,\qquad x'(0^-) = 0.
\]
From the jump condition in part (a),
\[
x'(0^+) = x'(0^-) + \frac{J}{m} = \frac{J}{m}.
\]
Continuity of $x(t)$ at $t=0$ gives $x(0^+)=x(0^-)=0$.

For $t>0$, the delta forcing vanishes, so $x$ satisfies the homogeneous equation
\[
m x''(t) + k x(t) = 0,\qquad t>0.
\]
Dividing by $m$ and setting $\omega = \sqrt{k/m}$, we have
\[
x''(t) + \omega^2 x(t) = 0,
\]
whose general solution is
\[
x(t) = A\cos(\omega t) + B\sin(\omega t),\qquad t>0.
\]
We now impose the initial data at $t=0^+$:
\[
x(0^+) = A = 0,
\]
since $\cos(0)=1$ and $\sin(0)=0$. Differentiating,
\[
x'(t) = -A\omega \sin(\omega t) + B\omega \cos(\omega t),
\]
so
\[
x'(0^+) = B\omega = \frac{J}{m}.
\]
Hence
\[
B = \frac{J}{m\omega}.
\]
Thus, for $t>0$,
\[
x(t) = \frac{J}{m\omega}\,\sin(\omega t).
\]
For $t<0$, we have $x(t)=0$ by the initial rest condition. We can combine these two pieces into a single formula using the Heaviside step function $H(t)$, defined by $H(t)=0$ for $t<0$ and $H(t)=1$ for $t>0$. The solution is
\[
x(t) = \frac{J}{m\omega}\,\sin(\omega t)\,H(t),\qquad t\in\mathbb{R}.
\]
This function is continuous at $t=0$ with $x(0)=0$, and its derivative has the correct jump:
\[
x'(0^+) = \frac{J}{m},\qquad x'(0^-)=0.
\]

\medskip

\textbf{(c) Impulse response and the role of Dirac’s delta.}  
The function
\[
G(t) = \frac{1}{m\omega}\,\sin(\omega t)\,H(t)
\]
is the solution of
\[
m G''(t) + k G(t) = \delta(t),\qquad G(t)=0,\quad G'(t)=0\ \text{for }t<0.
\]
Thus $G$ is the \emph{impulse response} or \emph{Green’s function} of the operator $Lx = m x'' + k x$. For an impulse of strength $J$, the solution is simply $J G(t)$.

This example illustrates the central role of Dirac’s delta in Fourier analysis and linear systems theory. The delta function models an idealized instantaneous input whose total effect is encoded in the jump of the derivative. By finding the system’s response $G$ to this fundamental input, we obtain a building block from which solutions to more general forcings can be constructed (via convolution). In this way, the delta function connects the differential operator $L$ with its Green’s function and provides a concise way to describe how the system responds to sharp, localized disturbances.
\end{solution}

% ===== Example 2: Point Source in the Heat Equation (inquiry-based) =====
\begin{problem}[Point Source in the Heat Equation]
In many physical models, heat or mass is initially concentrated at a single point, for instance when a laser pulse heats a tiny spot on a rod. Mathematically we idealize this by saying that the initial temperature distribution is zero everywhere except at one point, but the total amount of heat is one unit. This is naturally modeled by Dirac's delta-function. In this problem we use the Fourier transform to understand how such a point source evolves under the one-dimensional heat equation on the whole line.

Consider the one-dimensional heat equation on the real line
\[
u_t(x,t) = k\,u_{xx}(x,t), \qquad x \in \mathbb{R},\ t>0,
\]
where $k>0$ is the thermal diffusivity.

\medskip

(a) Suppose first that the initial temperature $u(x,0)=f(x)$ is a nice, rapidly decreasing function. Let
\[
\widehat{u}(\xi,t) = \int_{\mathbb{R}} u(x,t)\,e^{-i\xi x}\,dx
\]
be the spatial Fourier transform of $u(\cdot,t)$ for each fixed $t>0$.
\begin{itemize}
  \item[(i)] Take the Fourier transform in $x$ of both sides of the heat equation and show that $\widehat{u}$ satisfies an ordinary differential equation in $t$ of the form
  \[
  \frac{\partial}{\partial t}\widehat{u}(\xi,t) = -k\,\xi^{2}\,\widehat{u}(\xi,t).
  \]
  \item[(ii)] Solve this ODE and express $\widehat{u}(\xi,t)$ in terms of $\widehat{f}(\xi) = \widehat{u}(\xi,0)$.
\end{itemize}
Hint: Use that $\mathcal{F}[u_{xx}] = -(i\xi)^2 \widehat{u} = -\xi^2 \widehat{u}$.

\medskip

(b) Now let the initial data be the point source
\[
u(x,0) = \delta(x),
\]
which models one unit of heat instantaneously placed at the origin at time $t=0$.
\begin{itemize}
  \item[(i)] What is the Fourier transform $\widehat{u}(\xi,0)$ of $\delta(x)$?
  \item[(ii)] Using your formula from part (a), write down $\widehat{u}(\xi,t)$ for $t>0$ in this case.
\end{itemize}
Hint: Recall that for any Schwartz function $\varphi$ one has $\int_{\mathbb{R}}\delta(x)\,\varphi(x)\,dx = \varphi(0)$.

\medskip

(c) To find $u(x,t)$, we now invert the Fourier transform. Using the inverse transform
\[
u(x,t) = \frac{1}{2\pi}\int_{\mathbb{R}} \widehat{u}(\xi,t)\,e^{i\xi x}\,d\xi,
\]
write an explicit integral formula for $u(x,t)$ in the point-source case. Then evaluate this integral and show that
\[
u(x,t) = \frac{1}{\sqrt{4\pi k t}}\exp\!\left(-\frac{x^{2}}{4kt}\right),
\qquad t>0.
\]
Hint: You will need a Gaussian integral. One useful identity is
\[
\int_{\mathbb{R}} e^{-a\xi^2 + b\xi}\,d\xi
= \sqrt{\frac{\pi}{a}}\exp\!\left(\frac{b^2}{4a}\right)
\quad\text{for } a>0,\ b\in\mathbb{C}.
\]

\medskip

(d) The function
\[
G(x,t) := \frac{1}{\sqrt{4\pi k t}}\exp\!\left(-\frac{x^{2}}{4kt}\right)
\]
is called the heat kernel or fundamental solution of the heat equation.
\begin{itemize}
  \item[(i)] Argue (for example, by differentiating under the integral sign in the Fourier representation) that $G$ indeed satisfies the heat equation for every $t>0$.
  \item[(ii)] Show that $G(\cdot,t)$ has total mass one for each $t>0$, that is,
  \[
  \int_{\mathbb{R}} G(x,t)\,dx = 1.
  \]
  \item[(iii)] Explain in the sense of distributions what it means to say that $G(x,t)$ converges to $\delta(x)$ as $t\to 0^{+}$, and prove this statement. In other words, show that for every smooth, rapidly decreasing test function $\varphi$,
  \[
  \int_{\mathbb{R}} G(x,t)\,\varphi(x)\,dx \longrightarrow \varphi(0)
  \quad\text{as } t\to 0^{+}.
  \]
\end{itemize}
Hint: For (iii), consider the change of variables $x = 2\sqrt{kt}\,y$ and use the continuity of $\varphi$ at $0$ together with dominated convergence.

\medskip

(e) Let us explore some variations and extensions.
\begin{itemize}
  \item[(i)] Suppose the initial data is a point source at $x=x_{0}$ instead of at the origin, that is, $u(x,0) = \delta(x-x_{0})$. Using the translation properties of the Fourier transform or of the delta-function, guess and then verify the corresponding solution $u(x,t)$.
  \item[(ii)] For a general initial temperature $u(x,0) = f(x)$ (assume $f$ is nice enough), the solution can be written in terms of the heat kernel as a convolution: 
  \[
  u(x,t) = \int_{\mathbb{R}} G(x-y,t)\,f(y)\,dy.
  \]
  Briefly explain how the computation you did for the point source suggests this formula and why it is natural to call $G$ a \emph{fundamental solution}.
\end{itemize}
Hint: Think of $f$ as a “continuous superposition’’ of point sources and recall that the heat equation is linear.

\end{problem}

% ===== Example 2: Point Source in the Heat Equation (full solution) =====
\begin{problem}[Point Source in the Heat Equation]
Consider the one-dimensional heat equation on $\mathbb{R}$,
\[
u_t(x,t) = k\,u_{xx}(x,t), \qquad x\in\mathbb{R},\ t>0,
\]
with thermal diffusivity $k>0$ and initial data
\[
u(x,0) = \delta(x),
\]
the Dirac delta at the origin.

(a) Using the spatial Fourier transform, solve this initial-value problem and show that for $t>0$ the solution is
\[
u(x,t) = \frac{1}{\sqrt{4\pi k t}}\,\exp\!\left(-\frac{x^{2}}{4kt}\right).
\]

(b) Verify that this function satisfies the heat equation for $t>0$, that its total mass is one for each $t>0$, and that $u(\cdot,t)\to\delta$ in the sense of distributions as $t\to 0^{+}$.

Briefly explain why this example is called the fundamental solution (or heat kernel), and how it illustrates the use of the Dirac delta-function in Fourier analysis.
\end{problem}

\begin{solution}
We work on the whole real line and use the spatial Fourier transform to turn the partial differential equation into an ordinary differential equation in time. We adopt the convention
\[
\widehat{f}(\xi) = \int_{\mathbb{R}} f(x)\,e^{-i\xi x}\,dx,
\qquad
f(x) = \frac{1}{2\pi}\int_{\mathbb{R}} \widehat{f}(\xi)\,e^{i\xi x}\,d\xi.
\]

\medskip

\noindent\textbf{Step 1: Fourier transform of the heat equation.}
Let $u(x,t)$ solve
\[
u_t = k\,u_{xx}, \qquad u(x,0) = \delta(x).
\]
For each fixed $t>0$, we take the Fourier transform in $x$:
\[
\widehat{u}(\xi,t)
= \int_{\mathbb{R}} u(x,t)\,e^{-i\xi x}\,dx.
\]
Assuming enough regularity and decay to justify differentiation under the integral sign (which can be made rigorous using standard arguments), we obtain
\[
\frac{\partial}{\partial t}\widehat{u}(\xi,t)
= \int_{\mathbb{R}} u_t(x,t)\,e^{-i\xi x}\,dx
= k\int_{\mathbb{R}} u_{xx}(x,t)\,e^{-i\xi x}\,dx.
\]
The Fourier transform converts spatial derivatives into multiplication by powers of $i\xi$. Integrating by parts twice, or using the standard transform rule, gives
\[
\mathcal{F}[u_{xx}](\xi,t) = -(i\xi)^2\,\widehat{u}(\xi,t) = -\xi^{2}\,\widehat{u}(\xi,t).
\]
Thus the transformed function satisfies the ordinary differential equation
\[
\frac{\partial}{\partial t}\widehat{u}(\xi,t)
= -k\,\xi^{2}\,\widehat{u}(\xi,t).
\]

For each fixed $\xi$, this is a linear first-order ODE with solution
\[
\widehat{u}(\xi,t)
= \widehat{u}(\xi,0)\,e^{-k\xi^{2}t}.
\]

\medskip

\noindent\textbf{Step 2: Incorporating the initial data $\delta(x)$.}
The initial condition is $u(x,0) = \delta(x)$. The Fourier transform of $\delta$ is particularly simple. By definition of the delta-function as a distribution,
\[
\widehat{\delta}(\xi)
= \int_{\mathbb{R}} \delta(x)\,e^{-i\xi x}\,dx
= e^{-i\xi\cdot 0} = 1
\]
for all $\xi\in\mathbb{R}$. Hence
\[
\widehat{u}(\xi,0) = \widehat{\delta}(\xi) = 1.
\]
Substituting this into the solution of the ODE, we obtain
\[
\widehat{u}(\xi,t) = e^{-k\xi^{2}t}, \qquad t>0.
\]

\medskip

\noindent\textbf{Step 3: Inverting the Fourier transform.}
We now recover $u(x,t)$ via the inverse transform:
\[
u(x,t)
= \frac{1}{2\pi}\int_{\mathbb{R}} \widehat{u}(\xi,t)\,e^{i\xi x}\,d\xi
= \frac{1}{2\pi}\int_{\mathbb{R}} e^{-k\xi^{2}t}\,e^{i\xi x}\,d\xi.
\]
This is a Gaussian integral with a complex linear term in $\xi$. We write
\[
u(x,t)
= \frac{1}{2\pi}\int_{\mathbb{R}} \exp\bigl(-k t\,\xi^{2} + i x\xi\bigr)\,d\xi.
\]
We can apply the standard formula
\[
\int_{\mathbb{R}} e^{-a\xi^{2} + b\xi}\,d\xi
= \sqrt{\frac{\pi}{a}}\exp\!\left(\frac{b^{2}}{4a}\right),
\qquad a>0,\ b\in\mathbb{C},
\]
with $a = k t$ and $b = i x$. Then
\[
\int_{\mathbb{R}} e^{-k t\,\xi^{2} + i x\xi}\,d\xi
= \sqrt{\frac{\pi}{k t}}\exp\!\left(\frac{(ix)^{2}}{4kt}\right)
= \sqrt{\frac{\pi}{k t}}\exp\!\left(-\frac{x^{2}}{4kt}\right),
\]
since $(i x)^{2} = -x^{2}$. Therefore
\[
u(x,t)
= \frac{1}{2\pi}\,\sqrt{\frac{\pi}{k t}}\,
\exp\!\left(-\frac{x^{2}}{4kt}\right)
= \frac{1}{\sqrt{4\pi k t}}\,
\exp\!\left(-\frac{x^{2}}{4kt}\right).
\]
This is the claimed formula. It is customary to denote
\[
G(x,t) := \frac{1}{\sqrt{4\pi k t}}\exp\!\left(-\frac{x^{2}}{4kt}\right),
\]
and to call $G$ the heat kernel or fundamental solution of the one-dimensional heat equation.

\medskip

\noindent\textbf{Step 4: Verifying that $G$ satisfies the heat equation.}
Weargue that $G$ satisfies $G_t = k G_{xx}$ for every $t>0$. One way is to note that our derivation started from the heat equation, so for sufficiently regular solutions the Fourier representation ensures that the resulting $u$ solves the PDE. Since $G$ is smooth in $(x,t)$ for $t>0$, it is legitimate to differentiate the inverse Fourier integral under the integral sign:
\[
G(x,t)
= \frac{1}{2\pi} \int_{\mathbb{R}} e^{-k\xi^{2}t} e^{i\xi x}\,d\xi.
\]
Then
\[
G_t(x,t)
= \frac{1}{2\pi} \int_{\mathbb{R}} (-k\xi^{2})\,e^{-k\xi^{2}t} e^{i\xi x}\,d\xi,
\]
and
\[
G_{xx}(x,t)
= \frac{1}{2\pi} \int_{\mathbb{R}} (i\xi)^{2}\,e^{-k\xi^{2}t} e^{i\xi x}\,d\xi
= -\frac{1}{2\pi} \int_{\mathbb{R}} \xi^{2}\,e^{-k\xi^{2}t} e^{i\xi x}\,d\xi.
\]
Comparing these, we see that $G_t(x,t) = k\,G_{xx}(x,t)$. Thus $G$ satisfies the heat equation for every $t>0$.

Alternatively, one could check this directly by differentiating the explicit Gaussian expression, but the Fourier method is quicker and highlights the main idea of passing to the frequency side.

\medskip

\noindent\textbf{Step 5: Conservation of total mass.}
We now show that for each $t>0$,
\[
\int_{\mathbb{R}} G(x,t)\,dx = 1.
\]
Using the explicit formula,
\[
\int_{\mathbb{R}} G(x,t)\,dx
= \int_{\mathbb{R}} \frac{1}{\sqrt{4\pi k t}}
\exp\!\left(-\frac{x^{2}}{4kt}\right) dx.
\]
Make the change of variables
\[
y = \frac{x}{\sqrt{4kt}}, \qquad x = \sqrt{4kt}\,y, \qquad dx = \sqrt{4kt}\,dy.
\]
Then
\[
\int_{\mathbb{R}} G(x,t)\,dx
= \int_{\mathbb{R}} \frac{1}{\sqrt{4\pi k t}}\,e^{-y^{2}}\,\sqrt{4kt}\,dy
= \frac{\sqrt{4kt}}{\sqrt{4\pi k t}}\int_{\mathbb{R}} e^{-y^{2}}\,dy
= \frac{1}{\sqrt{\pi}}\int_{\mathbb{R}} e^{-y^{2}}\,dy.
\]
The standard Gaussian integral gives $\int_{\mathbb{R}} e^{-y^{2}}\,dy = \sqrt{\pi}$, and hence the ratio is exactly $1$. Thus the total “mass’’ (or total heat) is conserved in time. This is consistent with the physical interpretation of the heat equation without sources or sinks.

\medskip

\noindent\textbf{Step 6: Convergence to the delta-function.}
Finally, we show that $G(\cdot,t)$ converges to $\delta$ in the sense of distributions as $t\to 0^{+}$. By definition, this means that for every Schwartz test function $\varphi$,
\[
\int_{\mathbb{R}} G(x,t)\,\varphi(x)\,dx \longrightarrow \varphi(0)
\quad\text{as } t\to 0^{+}.
\]

Define
\[
I(t) := \int_{\mathbb{R}} G(x,t)\,\varphi(x)\,dx
= \int_{\mathbb{R}} \frac{1}{\sqrt{4\pi k t}}
\exp\!\left(-\frac{x^{2}}{4kt}\right)\varphi(x)\,dx.
\]
We again change variables $x = 2\sqrt{kt}\,y$, so that $dx = 2\sqrt{kt}\,dy$ and
\[
\frac{x^{2}}{4kt} = y^{2}.
\]
Then
\[
I(t)
= \int_{\mathbb{R}} \frac{1}{\sqrt{4\pi k t}}\,
e^{-y^{2}}\,\varphi(2\sqrt{kt}\,y)\,(2\sqrt{kt})\,dy
= \frac{1}{\sqrt{\pi}}\int_{\mathbb{R}} e^{-y^{2}}\,
\varphi(2\sqrt{kt}\,y)\,dy.
\]
As $t\to 0^{+}$, we have $2\sqrt{kt}\,y \to 0$ for each fixed $y$, and therefore
\[
\varphi(2\sqrt{kt}\,y) \longrightarrow \varphi(0).
\]
Moreover, since $\varphi$ is smooth and rapidly decreasing, there exists a constant $C$ such that $|\varphi(x)|\le C$ for all $x$, so that
\[
\bigl|e^{-y^{2}}\varphi(2\sqrt{kt}\,y)\bigr|
\le C e^{-y^{2}}.
\]
The function $C e^{-y^{2}}$ is integrable on $\mathbb{R}$. Therefore we can apply the dominated convergence theorem to the integral representation of $I(t)$ and pass the limit inside the integral:
\[
\lim_{t\to 0^{+}} I(t)
= \frac{1}{\sqrt{\pi}} \int_{\mathbb{R}} e^{-y^{2}}\,
\lim_{t\to 0^{+}}\varphi(2\sqrt{kt}\,y)\,dy
= \frac{1}{\sqrt{\pi}} \int_{\mathbb{R}} e^{-y^{2}}\,\varphi(0)\,dy.
\]
This equals
\[
\varphi(0)\,\frac{1}{\sqrt{\pi}}\int_{\mathbb{R}} e^{-y^{2}}\,dy
= \varphi(0)\,\frac{1}{\sqrt{\pi}}\cdot\sqrt{\pi}
= \varphi(0).
\]
Hence
\[
\int_{\mathbb{R}} G(x,t)\,\varphi(x)\,dx \longrightarrow \varphi(0),
\]
which is exactly the statement that $G(\cdot,t)$ converges to the delta-function $\delta$ in the distributional sense.

\medskip

\noindent\textbf{Step 7: Interpretation and connection to Dirac's delta.}
We have shown that the solution of the heat equation with initial data $\delta(x)$ is the Gaussian
\[
u(x,t) = G(x,t)
= \frac{1}{\sqrt{4\pi k t}}\,\exp\!\left(-\frac{x^{2}}{4kt}\right),
\]
which is smooth and rapidly decaying for every $t>0$. As time increases, the Gaussian “spreads out’’ (its variance is $2kt$) while its peak decreases, in such a way that the total mass remains one. In the limit as $t\downarrow 0$, this family concentrates at the origin and converges to the delta-function.

This example illustrates several central ideas of the section on Dirac’s delta-function in Fourier analysis:

\begin{itemize}
  \item The delta-function can be used as an idealized initial condition for a PDE. Although $\delta$ is not a classical function, the Fourier transform method treats it naturally via its transform $\widehat{\delta} \equiv 1$.
  \item The resulting solution $G$ is called a \emph{fundamental solution} or \emph{Green’s function}: it is the response of the system to a point source. By linearity, solutions for more general initial data $f$ can be obtained by superposing such point-source solutions, which leads to the convolution formula
  \[
  u(x,t) = (G(\cdot,t)*f)(x) = \int_{\mathbb{R}} G(x-y,t)\,f(y)\,dy.
  \]
  \item The family $\{G(\cdot,t)\}_{t>0}$ provides a concrete example of an \emph{approximate identity}: a sequence (or family) of functions that converges to $\delta$ in the sense of distributions, which is a key concept in understanding how delta-functions arise as limits of ordinary functions in Fourier analysis.
\end{itemize}

Thus the point source in the heat equation gives a vivid and computationally accessible example of how Dirac's delta-function evolves under a PDE and how Fourier methods naturally accommodate such singular data.

\end{solution}

% ===== Example 3: Sampling Property and Fourier Transform of the Delta (inquiry-based) =====
\begin{problem}[Sampling Property and Fourier Transform of the Delta]
In many physical models, forces or sources are concentrated at a single point in space, but we still want to use the language of integrals and Fourier analysis.  The Dirac delta-function $\delta$ is an idealized object that acts like a point mass: it is ``zero everywhere except at one point,'' yet its integral is $1$.  In this problem, you will discover how the delta-function ``samples'' smooth functions, and how this sampling behavior determines its Fourier transform.  Along the way, you will see explicitly what it means for the delta-function to be perfectly localized in one domain and completely spread out in its transform domain.

Throughout, use the following Fourier transform convention for (sufficiently nice) functions $f$:
\[
\widehat{f}(k) \;=\; \int_{-\infty}^{\infty} f(x)\,e^{-ikx}\,dx,
\qquad
f(x) \;=\; \frac{1}{2\pi}\int_{-\infty}^{\infty} \widehat{f}(k)\,e^{ikx}\,dk.
\]

(a) One way to understand $\delta$ is as a limit of ordinary functions that become more and more sharply peaked at the origin.  Consider, for example, the family of Gaussians
\[
\delta_\varepsilon(x) \;=\; \frac{1}{\sqrt{\pi}\,\varepsilon}\,e^{-x^{2}/\varepsilon^{2}}, \qquad \varepsilon>0.
\]
Show that $\displaystyle \int_{-\infty}^{\infty} \delta_\varepsilon(x)\,dx = 1$ for every $\varepsilon>0$.  Then, let $f$ be a smooth function that does not grow too fast at infinity (for instance, assume $f$ and its derivatives are bounded), and consider the integral
\[
I_\varepsilon \;=\; \int_{-\infty}^{\infty} f(x)\,\delta_\varepsilon(x-a)\,dx.
\]
Use a change of variables or Taylor expansion of $f$ around $x=a$ to explain why $I_\varepsilon \to f(a)$ as $\varepsilon \to 0^+$.  
Hint: First shift variables so that the peak is at $0$.  Then compare $f$ near $x=a$ with its value $f(a)$.

(b) Motivated by part (a), the Dirac delta-function $\delta$ at $x=a$ is defined formally by the rule
\[
\int_{-\infty}^{\infty} f(x)\,\delta(x-a)\,dx \;=\; f(a)
\]
for all sufficiently smooth, well-behaved functions $f$.  This is called the \emph{sampling property} or \emph{sifting property} of the delta.

(i) Use your work from part (a) to justify why this is a natural definition if we think of $\delta$ as a limit of the approximations $\delta_\varepsilon$.  

(ii) Check the sampling property on some concrete examples, such as $f(x)=1$, $f(x)=x$, and $f(x) = e^{ix}$.  What do the corresponding integrals with $\delta(x-a)$ give?

% Hint: For the examples, just apply the rule $\int f(x)\delta(x-a)\,dx = f(a)$ directly and interpret the result.

(c) Now use the sampling property to compute the Fourier transform of $\delta(x-a)$.  Treat $\delta(x-a)$ as if it were an ordinary function and write
\[
\widehat{\delta(\cdot - a)}(k) 
\;:=\; \int_{-\infty}^{\infty} \delta(x-a)\,e^{-ikx}\,dx.
\]
Evaluate this integral using the sampling property from part (b).

What function of $k$ do you obtain?  How does its magnitude depend on $k$?  How does its \emph{phase} depend on $a$?

% Hint: Think of $f(x) = e^{-ikx}$ in the sampling rule.

(d) Next, we turn the picture around: we want to see that a constant function in the $k$-variable gives a delta-function in the $x$-variable under the inverse Fourier transform.  Consider the inverse Fourier transform of the constant function $\widehat{f}(k)\equiv 1$:
\[
g(x) \;:=\; \frac{1}{2\pi}\int_{-\infty}^{\infty} 1\cdot e^{ikx}\,dk.
\]
(i) Argue (informally, if needed) that this integral does not define an ordinary function of $x$ in the usual sense.  Instead, we interpret $g$ as a \emph{distribution} defined by its action on a test function $\varphi$:
\[
\int_{-\infty}^{\infty} g(x)\,\varphi(x)\,dx
\;:=\; 
\frac{1}{2\pi}\int_{-\infty}^{\infty}\int_{-\infty}^{\infty} e^{ikx}\,\varphi(x)\,dx\,dk.
\]
Explain why (under appropriate conditions allowing you to change the order of integration) this becomes
\[
\int_{-\infty}^{\infty} g(x)\,\varphi(x)\,dx \;=\; \varphi(0).
\]
Conclude that $g(x)$ must be $2\pi\,\delta(x)$ as a distribution.

(ii) Use this observation, together with your answer in part (c), to state the inverse Fourier transform of $e^{-ika}$.  What is the spatial-domain object corresponding to a pure complex exponential in frequency?

% Hint: Combine linearity of the transform with the shift in $x$-space you observed for $\delta(x-a)$ in $k$-space.

(e) (Extensions and “what ifs.”)

(i) Suppose we differentiate $\delta(x-a)$ with respect to $x$, obtaining the distribution $\delta'(x-a)$.  Guess, and then verify using integration by parts, a formula for
\[
\int_{-\infty}^{\infty} f(x)\,\delta'(x-a)\,dx
\]
in terms of $f$ and its derivative at $x=a$.  How does this relate to the idea that derivatives of $\delta$ record higher-order information about $f$ at a point?

(ii) Using your formula from (i) and the definition of the Fourier transform, determine the Fourier transform of $\delta'(x-a)$.  How does it compare to the transform of $\delta(x-a)$ from part (c)?  What operation in physical space does multiplication by $ik$ in frequency space correspond to?

% Hint: For (ii), view $\delta'(x-a)$ as acting on $f(x)=e^{-ikx}$ and recall that $\frac{d}{dx}e^{-ikx} = -ik\,e^{-ikx}$.
\end{problem}

% ===== Example 3: Sampling Property and Fourier Transform of the Delta (full solution) =====
\begin{problem}[Sampling Property and Fourier Transform of the Delta]
Let $\delta$ denote the Dirac delta “function.”  Assume it is characterized by the sampling property
\[
\int_{-\infty}^{\infty} f(x)\,\delta(x-a)\,dx = f(a)
\]
for all sufficiently smooth, rapidly decaying functions $f$.  We use the Fourier transform
\[
\widehat{f}(k) = \int_{-\infty}^{\infty} f(x)\,e^{-ikx}\,dx,
\qquad
f(x) = \frac{1}{2\pi}\int_{-\infty}^{\infty} \widehat{f}(k)\,e^{ikx}\,dk.
\]

(a) Justify the sampling property by considering a family of approximate deltas $\delta_\varepsilon(x-a)$, for example
\[
\delta_\varepsilon(x) = \frac{1}{\sqrt{\pi}\,\varepsilon}\,e^{-x^{2}/\varepsilon^{2}},
\]
and showing that
\[
\int f(x)\,\delta_\varepsilon(x-a)\,dx \to f(a)
\]
for smooth $f$ as $\varepsilon\to 0^+$.

(b) Using the sampling property, compute the Fourier transform of $\delta(x-a)$:
\[
\widehat{\delta(\cdot - a)}(k) = \int_{-\infty}^{\infty} \delta(x-a)\,e^{-ikx}\,dx.
\]

(c) Show that the inverse Fourier transform of the constant function $\widehat{f}(k)\equiv 1$ is the distribution $2\pi\delta(x)$, in the sense that
\[
\frac{1}{2\pi}\int_{-\infty}^{\infty} e^{ikx}\,dk = 2\pi\,\delta(x)
\]
as distributions.  Use this to identify the inverse Fourier transform of $e^{-ika}$.

(d) Briefly explain how these computations illustrate the idea that the delta-function is perfectly localized in physical space while being completely delocalized (a pure modulation) in frequency space, and conversely for a constant in frequency space.
\end{problem}

\begin{solution}
We begin by justifying the sampling property via an approximation of the Dirac delta by ordinary functions, and then we use that property to compute its Fourier transform and the inverse transform of a constant.

\medskip
\noindent\textbf{(a) Sampling via approximations of the identity.}
Consider the Gaussian approximations
\[
\delta_\varepsilon(x) = \frac{1}{\sqrt{\pi}\,\varepsilon}\,e^{-x^{2}/\varepsilon^{2}}, \qquad \varepsilon>0.
\]
First, each $\delta_\varepsilon$ has total mass one:
\[
\int_{-\infty}^{\infty} \delta_\varepsilon(x)\,dx
= \frac{1}{\sqrt{\pi}\,\varepsilon}\int_{-\infty}^{\infty} e^{-x^{2}/\varepsilon^{2}}\,dx.
\]
With the change of variables $u=x/\varepsilon$, we obtain
\[
\int_{-\infty}^{\infty} e^{-x^{2}/\varepsilon^{2}}\,dx
= \varepsilon\int_{-\infty}^{\infty} e^{-u^{2}}\,du
= \varepsilon\sqrt{\pi},
\]
so the integral of $\delta_\varepsilon$ is indeed $1$.

Now let $f$ be a smooth function with at most moderate growth, and consider
\[
I_\varepsilon = \int_{-\infty}^{\infty} f(x)\,\delta_\varepsilon(x-a)\,dx.
\]
We shift variables by writing $y = x-a$, so that $x = a+y$ and $dx = dy$.  Then
\[
I_\varepsilon
= \int_{-\infty}^{\infty} f(a+y)\,\delta_\varepsilon(y)\,dy
= \int_{-\infty}^{\infty} f(a+y)\,\frac{1}{\sqrt{\pi}\,\varepsilon}e^{-y^{2}/\varepsilon^{2}}\,dy.
\]
Because $f$ is smooth, we can expand it in a Taylor series around $a$:
\[
f(a+y) = f(a) + f'(a)\,y + \tfrac12 f''(a)\,y^{2} + \cdots,
\]
at least for $y$ in a neighborhood of $0$.  Substituting this into the integral for $I_\varepsilon$ and integrating term by term, we obtain
\[
I_\varepsilon
= f(a)\int \delta_\varepsilon(y)\,dy
 + f'(a)\int y\,\delta_\varepsilon(y)\,dy
 + \frac12 f''(a)\int y^{2}\,\delta_\varepsilon(y)\,dy
 + \cdots.
\]
We already know $\int \delta_\varepsilon(y)\,dy = 1$.  By symmetry, the integrals of odd powers of $y$ against the even function $\delta_\varepsilon(y)$ vanish:
\[
\int_{-\infty}^{\infty} y\,\delta_\varepsilon(y)\,dy = 0,
\quad
\int y^{3}\,\delta_\varepsilon(y)\,dy = 0,
\ \text{etc.}
\]
The even-moment integrals (such as $\int y^{2}\delta_\varepsilon(y)\,dy$) are finite and of order $\varepsilon^{2}$, $\varepsilon^{4}$, and so on, because the Gaussian becomes increasingly concentrated near $0$.  Thus
\[
I_\varepsilon = f(a) + O(\varepsilon^{2}) \quad \text{as } \varepsilon\to 0^+,
\]
and hence $I_\varepsilon \to f(a)$.

This behavior motivates the definition of a distribution $\delta(x-a)$ satisfying
\[
\int_{-\infty}^{\infty} f(x)\,\delta(x-a)\,dx = f(a)
\]
for all smooth, sufficiently well-behaved $f$.  This is the \emph{sampling property} of the Dirac delta.

\medskip
\noindent\textbf{(b) Direct use of the sampling property.}
Once we accept the sampling property as the defining rule for $\delta(x-a)$, many integrals can be evaluated immediately.  For example, for $f(x)=1$ we obtain
\[
\int_{-\infty}^{\infty} 1\cdot \delta(x-a)\,dx = 1 = f(a),
\]
which is consistent with viewing $\delta(x-a)$ as a unit “point mass” at $x=a$.  For $f(x)=x$ we find
\[
\int_{-\infty}^{\infty} x\,\delta(x-a)\,dx = a,
\]
and for $f(x) = e^{ix}$,
\[
\int_{-\infty}^{\infty} e^{ix}\,\delta(x-a)\,dx = e^{ia}.
\]
These concrete cases confirm that the delta indeed “picks out” the value of the function at the point $x=a$.

\medskip
\noindent\textbf{(c) Fourier transform of $\delta(x-a)$.}
We now compute the Fourier transform of $\delta(x-a)$ using our convention
\[
\widehat{f}(k) = \int_{-\infty}^{\infty} f(x)\,e^{-ikx}\,dx.
\]
Formally,
\[
\widehat{\delta(\cdot - a)}(k)
:= \int_{-\infty}^{\infty} \delta(x-a)\,e^{-ikx}\,dx.
\]
We can view $e^{-ikx}$ as the test function $f(x)$ in the sampling property.  Thus
\[
\widehat{\delta(\cdot - a)}(k)
= e^{-ika}.
\]
This shows that the Fourier transform of a point mass at $x=a$ is a pure complex exponential in $k$ with unit magnitude and a phase that depends linearly on $k$ and on the shift $a$.

Two features are important here:

1. The magnitude of $\widehat{\delta(\cdot - a)}(k)$ is identically $1$ for all $k$.  In frequency space, the delta is completely “spread out”: it has equal weight at all frequencies.

2. The position $a$ in physical space appears in frequency space only through a phase factor $e^{-ika}$.  Translating the delta in $x$ corresponds to multiplying its Fourier transform by a modulation factor in $k$.

Thus, the delta is perfectly localized in $x$, while its transform is perfectly delocalized in $k$ (but with a structured phase).

\medskip
\noindent\textbf{(d) Inverse Fourier transform of a constant and of $e^{-ika}$.}
Next, we compute the inverse Fourier transform of the constant function $\widehat{f}(k)\equiv 1$:
\[
g(x) := \frac{1}{2\pi}\int_{-\infty}^{\infty} 1\cdot e^{ikx}\,dk.
\]
As an ordinary improper integral, this expression does not converge in the usual sense.  However, we can make good sense of $g$ as a \emph{distribution}, defined by its action on a test function $\varphi$:
\[
\int_{-\infty}^{\infty} g(x)\,\varphi(x)\,dx
:= \frac{1}{2\pi}\int_{-\infty}^{\infty}\!\int_{-\infty}^{\infty} e^{ikx}\,\varphi(x)\,dx\,dk.
\]
Assuming $\varphi$ is smooth and rapidly decaying, we may exchange the order of integration:
\[
\int_{-\infty}^{\infty} g(x)\,\varphi(x)\,dx
= \frac{1}{2\pi}\int_{-\infty}^{\infty} \left( \int_{-\infty}^{\infty} e^{ikx}\,dk \right) \varphi(x)\,dx.
\]
We interpret the inner integral in terms of the inverse transform of the constant $1$.  By the defining property of the Fourier transform, we know that the inverse transform of $1$ must be the distribution that, when paired with $\varphi$, returns $\varphi(0)$.  Indeed, we can verify this by using the usual Fourier inversion formula for $\varphi$:
\[
\varphi(0) = \frac{1}{2\pi}\int_{-\infty}^{\infty}\widehat{\varphi}(k)\,dk
= \frac{1}{2\pi}\int_{-\infty}^{\infty}\left(\int_{-\infty}^{\infty} \varphi(x)\,e^{-ikx}\,dx\right)dk.
\]
Exchanging integrals again,
\[
\varphi(0) 
= \int_{-\infty}^{\infty} \varphi(x)\left( \frac{1}{2\pi}\int_{-\infty}^{\infty} e^{-ikx}\,dk \right)dx.
\]
By uniqueness of distributions, this means
\[
\frac{1}{2\pi}\int_{-\infty}^{\infty} e^{-ikx}\,dk = 2\pi\,\delta(x)
\]
in the distributional sense, and similarly
\[
\frac{1}{2\pi}\int_{-\infty}^{\infty} e^{ikx}\,dk = 2\pi\,\delta(x).
\]
Therefore, the inverse Fourier transform of the constant function $1$ is
\[
g(x) = 2\pi\,\delta(x).
\]

Now we identify the inverse Fourier transform of $e^{-ika}$.  By our convention,
\[
\mathcal{F}^{-1}\{e^{-ika}\}(x) 
= \frac{1}{2\pi}\int_{-\infty}^{\infty} e^{-ika}\,e^{ikx}\,dk
= \frac{1}{2\pi}\int_{-\infty}^{\infty} e^{ik(x-a)}\,dk.
\]
The same reasoning as above shows that this integral is $2\pi\,\delta(x-a)$ in the sense of distributions.  Hence
\[
\mathcal{F}^{-1}\{e^{-ika}\}(x) = 2\pi\,\delta(x-a).
\]

Combining this with part (c), we see that
\[
\widehat{\delta(\cdot - a)}(k) = e^{-ika}
\quad\text{and}\quad
\mathcal{F}^{-1}\{e^{-ika}\}(x) = 2\pi\,\delta(x-a),
\]
which is perfectly consistent with the general Fourier inversion framework.

\medskip
\noindent\textbf{(e) Conceptual interpretation.}
The calculations above show a fundamental duality:

\begin{itemize}
\item In physical space, $\delta(x-a)$ is infinitely localized: all of its “mass” is concentrated at the single point $x=a$.  Its Fourier transform is the bounded, non-decaying function $e^{-ika}$, which occupies all frequencies with equal magnitude.  The only information about $a$ appears in the phase.

\item In frequency space, the constant function $\widehat{f}(k)\equiv 1$ is completely spread out: it has the same value at every frequency.  Its inverse Fourier transform is the highly localized distribution $2\pi\delta(x)$ in physical space.
\end{itemize}

Thus, the Dirac delta exemplifies the general principle that strong localization in one domain (here, physical space) corresponds to strong delocalization in the transform domain (here, frequency space), and vice versa.  In the context of this chapter on Fourier analysis and this section on the Dirac delta, the delta-function is the extreme case of a “point source,” and its Fourier transform captures the idea that such a point source excites all frequencies equally, differing only by a phase factor determined by its location.
\end{solution}

% ===== Example 4: Approximating the Delta by Ordinary Functions (inquiry-based) =====
\begin{problem}[Approximating the Delta by Ordinary Functions]
In many physical models, one wishes to describe an “idealized point source’’: a force applied at a single instant of time, a charge concentrated at a single point in space, or an infinitely sharp spike in a signal. Mathematically, these are modeled by the Dirac delta-function $\delta(x)$, which is not a function in the ordinary sense but a distribution. One way to understand $\delta$ concretely is to approximate it by honest functions that are increasingly concentrated near the origin yet keep total mass one. In this problem, you will construct and analyze several such approximations, and see in what precise sense they converge to $\delta$.

Throughout, let $f\colon \mathbb{R} \to \mathbb{R}$ be a continuous function, and when needed you may additionally assume that $f$ has compact support or decays sufficiently rapidly at infinity.

\smallskip

(a) \emph{Tall, thin rectangles.} For each $\varepsilon > 0$, define
\[
\phi_\varepsilon(x)
=
\begin{cases}
\dfrac{1}{2\varepsilon}, & |x| < \varepsilon,\\[0.5em]
0, & |x| \ge \varepsilon.
\end{cases}
\]
\begin{itemize}
  \item[(i)] Sketch the graph of $\phi_\varepsilon$ for a few values of $\varepsilon$ (for example, $\varepsilon=1$, $\varepsilon=\tfrac{1}{2}$, and $\varepsilon=\tfrac{1}{4}$). How do these graphs change as $\varepsilon \to 0$?
  \item[(ii)] Compute $\displaystyle \int_{-\infty}^{\infty} \phi_\varepsilon(x)\,dx$. Why is the answer independent of $\varepsilon$?
  \item[(iii)] Determine the pointwise limit of $\phi_\varepsilon(x)$ as $\varepsilon \to 0$ for $x \neq 0$, and discuss what happens at $x = 0$. In what sense do these functions resemble a “spike” at the origin?
\end{itemize}

(b) \emph{Testing against a smooth function.} One of the defining properties of the Dirac delta is
\[
\int_{-\infty}^{\infty} f(x)\,\delta(x)\,dx = f(0).
\]
We want to see if the family $(\phi_\varepsilon)_{\varepsilon>0}$ has the same property in the limit.
\begin{itemize}
  \item[(i)] Show that
  \[
  \int_{-\infty}^{\infty} f(x)\,\phi_\varepsilon(x)\,dx
  = \frac{1}{2\varepsilon}\int_{-\varepsilon}^{\varepsilon} f(x)\,dx.
  \]
  Interpret this expression as an “average value’’ of $f$ near the origin.
  \item[(ii)] Use the continuity of $f$ at $0$ to argue that
  \[
  \lim_{\varepsilon \to 0} \int_{-\infty}^{\infty} f(x)\,\phi_\varepsilon(x)\,dx = f(0).
  \]
  Hint: For small $\varepsilon$, how much can $f(x)$ vary on the interval $[-\varepsilon,\varepsilon]$?
\end{itemize}
Conclude that $\phi_\varepsilon$ approximates $\delta$ in the sense of integrals against continuous test functions.

\smallskip

(c) \emph{Narrow Gaussian bumps.} For each $\varepsilon > 0$, define the Gaussian
\[
\psi_\varepsilon(x)
=
\frac{1}{\sqrt{\pi}\,\varepsilon}\,e^{-(x/\varepsilon)^2}.
\]
\begin{itemize}
  \item[(i)] Show that $\displaystyle \int_{-\infty}^{\infty} \psi_\varepsilon(x)\,dx = 1$ for every $\varepsilon>0$. Hint: Recall that $\displaystyle \int_{-\infty}^{\infty} e^{-u^2}\,du = \sqrt{\pi}$ and use a change of variables.
  \item[(ii)] Explain, using the graph of $\psi_\varepsilon$, what happens to its height and width as $\varepsilon \to 0$. Compare this behavior to the rectangular functions $\phi_\varepsilon$.
  \item[(iii)] Show that for each fixed $x \neq 0$, we have $\psi_\varepsilon(x) \to 0$ as $\varepsilon \to 0$.
\end{itemize}
Now consider the integral
\[
I_\varepsilon = \int_{-\infty}^{\infty} f(x)\,\psi_\varepsilon(x)\,dx.
\]
\begin{itemize}
  \item[(iv)] Fix a small number $\delta > 0$. Split the integral into two parts,
  \[
  I_\varepsilon = \int_{|x|\le \delta} f(x)\,\psi_\varepsilon(x)\,dx
  +
  \int_{|x|> \delta} f(x)\,\psi_\varepsilon(x)\,dx,
  \]
  and argue that, for small $\varepsilon$, the second integral over $|x|>\delta$ is very small. Hint: Show that $\int_{|x|>\delta} \psi_\varepsilon(x)\,dx \to 0$ as $\varepsilon \to 0$.
  \item[(v)] On the interval $|x|\le \delta$, use the continuity of $f$ at $0$ to compare $f(x)$ and $f(0)$, and show that $I_\varepsilon \to f(0)$ as $\varepsilon \to 0$.
\end{itemize}
Conclude that the Gaussians $\psi_\varepsilon$ provide a different family of approximations to the delta-function.

\smallskip

(d) \emph{A Fourier-analytic approximation: rescaled sinc functions.} Consider the family of functions
\[
\eta_N(x) = \frac{1}{\pi}\,\frac{\sin(Nx)}{x}, 
\quad N>0,
\]
with the understanding that $\eta_N(0)$ is defined by taking the limit $x\to 0$.
\begin{itemize}
  \item[(i)] Show that $\displaystyle \eta_N(0) = \frac{N}{\pi}$ by computing the limit $\lim_{x\to 0} \frac{\sin(Nx)}{x}$.
  \item[(ii)] Show that $\displaystyle \int_{-\infty}^{\infty} \eta_N(x)\,dx = 1$ for every $N>0$. Hint: You may use (without proof) the standard integral $\displaystyle \int_{-\infty}^{\infty} \frac{\sin t}{t}\,dt = \pi$.
  \item[(iii)] Let $f$ be a smooth, rapidly decaying function (for example, a Schwartz function). Explain why the integral
  \[
  J_N = \int_{-\infty}^{\infty} f(x)\,\eta_N(x)\,dx
  \]
  can be interpreted, up to constants, as a partial Fourier inversion of $f$ using only frequencies in the band $[-N,N]$. 
  \item[(iv)] Give a heuristic (or rigorous, if you know Fourier transforms) argument that $J_N \to f(0)$ as $N\to\infty$.
\end{itemize}
Compare this “oscillatory’’ approximation of $\delta$ to the positive approximations in parts (a) and (c).

\smallskip

(e) \emph{Extensions and variations.}
\begin{itemize}
  \item[(i)] Suppose you shift the rectangular bump to the point $a\in \mathbb{R}$ by defining
  \[
  \phi_{\varepsilon,a}(x) 
  =
  \begin{cases}
  \dfrac{1}{2\varepsilon}, & |x-a| < \varepsilon,\\[0.5em]
  0, & |x-a| \ge \varepsilon.
  \end{cases}
  \]
  Predict, and then verify, the limit of $\displaystyle \int_{-\infty}^{\infty} f(x)\,\phi_{\varepsilon,a}(x)\,dx$ as $\varepsilon \to 0$. What “delta-function’’ does $\phi_{\varepsilon,a}$ approximate?
  \item[(ii)] Consider a family of functions $k_\varepsilon$ with $\int k_\varepsilon(x)\,dx =1$ for all $\varepsilon$, and such that $k_\varepsilon(x)\to 0$ for each fixed $x\neq 0$ as $\varepsilon\to 0$. What additional conditions on the family $(k_\varepsilon)$ do you think are needed to guarantee that
  \[
  \int_{-\infty}^{\infty} f(x)\,k_\varepsilon(x)\,dx \to f(0)
  \quad\text{for all continuous } f?
  \]
  Try to formulate a plausible general definition of an “approximate identity’’ based on your work with $\phi_\varepsilon$, $\psi_\varepsilon$, and $\eta_N$.
\end{itemize}

\end{problem}

% ===== Example 4: Approximating the Delta by Ordinary Functions (full solution) =====
\begin{problem}[Approximating the Delta by Ordinary Functions]
On $\mathbb{R}$, consider the following three families of functions:
\[
\phi_\varepsilon(x)
=
\begin{cases}
\dfrac{1}{2\varepsilon}, & |x| < \varepsilon,\\[0.5em]
0, & |x| \ge \varepsilon,
\end{cases}
\qquad
\psi_\varepsilon(x)
=
\frac{1}{\sqrt{\pi}\,\varepsilon}e^{-(x/\varepsilon)^2},
\]
for $\varepsilon>0$, and
\[
\eta_N(x) = \frac{1}{\pi}\,\frac{\sin(Nx)}{x}, \quad N>0,
\]
with $\eta_N(0)$ defined by continuity.

\begin{enumerate}
\item[(a)] Show that each of $\phi_\varepsilon$, $\psi_\varepsilon$, and $\eta_N$ has total integral $1$ over $\mathbb{R}$.
\item[(b)] Let $f$ be continuous with compact support. Prove that
\[
\lim_{\varepsilon\to 0} \int_{-\infty}^{\infty} f(x)\,\phi_\varepsilon(x)\,dx = f(0),
\quad
\lim_{\varepsilon\to 0} \int_{-\infty}^{\infty} f(x)\,\psi_\varepsilon(x)\,dx = f(0),
\]
and, if $f$ is smooth and rapidly decaying (say $f$ is a Schwartz function),
\[
\lim_{N\to\infty} \int_{-\infty}^{\infty} f(x)\,\eta_N(x)\,dx = f(0).
\]
\end{enumerate}
Explain how these limits justify the notation
\[
\phi_\varepsilon \rightharpoonup \delta,\quad
\psi_\varepsilon \rightharpoonup \delta,\quad
\eta_N \rightharpoonup \delta,
\]
and how this illustrates the idea of the Dirac delta-function as a distribution.
\end{problem}

\begin{solution}
We are asked to show that three different families of ordinary functions approximate the Dirac delta at the origin, in the sense of their action on test functions under the integral sign. This is the basic distributional viewpoint: a sequence “converges to $\delta$’’ if integrals against any sufficiently nice test function converge to the value of that test function at the origin.

\medskip

\textbf{Part (a): Integrals equal to one.}

\emph{Rectangular functions.}
By direct computation,
\[
\int_{-\infty}^{\infty} \phi_\varepsilon(x)\,dx
=
\int_{-\varepsilon}^{\varepsilon} \frac{1}{2\varepsilon}\,dx
=
\frac{1}{2\varepsilon}\bigl(2\varepsilon\bigr)
=
1
\]
for all $\varepsilon>0$.

\smallskip

\emph{Gaussian functions.}
We use the standard Gaussian integral
\[
\int_{-\infty}^{\infty} e^{-u^2}\,du = \sqrt{\pi}.
\]
Let $u = x/\varepsilon$. Then $x = \varepsilon u$ and $dx = \varepsilon \, du$. We compute
\[
\int_{-\infty}^{\infty} \psi_\varepsilon(x)\,dx
=
\int_{-\infty}^{\infty} \frac{1}{\sqrt{\pi}\,\varepsilon}\,e^{-(x/\varepsilon)^2}\,dx
=
\frac{1}{\sqrt{\pi}\,\varepsilon}
\int_{-\infty}^{\infty} e^{-u^2}\,\varepsilon\,du
=
\frac{1}{\sqrt{\pi}}
\int_{-\infty}^{\infty} e^{-u^2}\,du
=
1.
\]
Thus each $\psi_\varepsilon$ also has total mass equal to one.

\smallskip

\emph{Sinc functions.}
For $\eta_N(x) = \dfrac{1}{\pi}\dfrac{\sin(Nx)}{x}$ we again use a standard improper integral. First observe that
\[
\eta_N(0) = \frac{1}{\pi} \lim_{x\to 0} \frac{\sin(Nx)}{x}
=
\frac{1}{\pi}\cdot N
\]
because $\sin(Nx)\sim Nx$ as $x\to 0$. To compute the integral, perform the change of variables $t = Nx$, so that $x = t/N$ and $dx = dt/N$:
\[
\int_{-\infty}^{\infty} \eta_N(x)\,dx
=
\int_{-\infty}^{\infty} \frac{1}{\pi}\,\frac{\sin(Nx)}{x}\,dx
=
\frac{1}{\pi}
\int_{-\infty}^{\infty} \frac{\sin t}{t/N}\,\frac{dt}{N}
=
\frac{1}{\pi}
\int_{-\infty}^{\infty} \frac{\sin t}{t}\,dt.
\]
Here we used $x = t/N$, so $1/x = N/t$. The factors of $N$ cancel. The classical integral
\[
\int_{-\infty}^{\infty} \frac{\sin t}{t}\,dt = \pi
\]
is well known (and can be evaluated, for instance, using contour integration or Fourier transforms). Hence
\[
\int_{-\infty}^{\infty} \eta_N(x)\,dx = \frac{1}{\pi}\cdot \pi = 1.
\]
Thus each $\eta_N$ also has unit mass.

\medskip

\textbf{Part (b): Convergence against test functions.}

The central idea is that although these functions do not converge to anything pointwise (indeed, they typically blow up at zero), their \emph{averaging action} on test functions stabilizes: the integral $\int f(x)\,\text{(approximation)}(x)\,dx$ tends to $f(0)$. This is the sense in which they converge to the distribution $\delta$.

\smallskip

\emph{Rectangular approximation.}
Let $f$ be continuous with compact support. Then
\[
\int_{-\infty}^{\infty} f(x)\,\phi_\varepsilon(x)\,dx
=
\frac{1}{2\varepsilon}\int_{-\varepsilon}^{\varepsilon} f(x)\,dx.
\]
The right-hand side is simply the average value of $f$ on the interval $[-\varepsilon,\varepsilon]$. Because $f$ is continuous at $0$, the values of $f(x)$ on a small symmetric interval around $0$ cannot differ much from $f(0)$.

More precisely, fix $\delta>0$. By continuity of $f$ at $0$, there exists $\eta>0$ such that $|x|\le \eta$ implies $|f(x) - f(0)| < \delta$. For all sufficiently small $\varepsilon$ with $0<\varepsilon \le \eta$, we have
\[
\left|
\frac{1}{2\varepsilon}\int_{-\varepsilon}^{\varepsilon} f(x)\,dx - f(0)
\right|
=
\left|
\frac{1}{2\varepsilon}\int_{-\varepsilon}^{\varepsilon} \bigl(f(x) - f(0)\bigr)\,dx
\right|
\le
\frac{1}{2\varepsilon}\int_{-\varepsilon}^{\varepsilon} |f(x) - f(0)|\,dx
\le \delta.
\]
Since $\delta>0$ was arbitrary, this shows that
\[
\lim_{\varepsilon\to 0}
\int_{-\infty}^{\infty} f(x)\,\phi_\varepsilon(x)\,dx
=
f(0).
\]
Thus the rectangular family $(\phi_\varepsilon)$ converges to $\delta$ in the distributional sense.

\smallskip

\emph{Gaussian approximation.}
Let $f$ be continuous with compact support. Consider
\[
I_\varepsilon
:=
\int_{-\infty}^{\infty} f(x)\,\psi_\varepsilon(x)\,dx.
\]
We want to show that $I_\varepsilon \to f(0)$ as $\varepsilon\to 0$. The strategy is to split the integral into a “near’’ part, where $x$ is close to $0$ and $f(x)\approx f(0)$ by continuity, and a “far’’ part, where the Gaussian mass is small.

Fix $\delta>0$. Write
\[
I_\varepsilon
=
\int_{|x|\le \delta} f(x)\,\psi_\varepsilon(x)\,dx
+
\int_{|x|> \delta} f(x)\,\psi_\varepsilon(x)\,dx
=: I_\varepsilon^{\text{near}} + I_\varepsilon^{\text{far}}.
\]

We first show that the far part tends to zero. Since $f$ has compact support, it is bounded: there exists $M>0$ with $|f(x)|\le M$ for all $x$. Then
\[
\bigl|I_\varepsilon^{\text{far}}\bigr|
\le
M\int_{|x|>\delta} \psi_\varepsilon(x)\,dx.
\]
We claim that $\int_{|x|>\delta} \psi_\varepsilon(x)\,dx \to 0$ as $\varepsilon\to 0$. To see this, use again the change of variables $u=x/\varepsilon$:
\[
\int_{|x|>\delta} \psi_\varepsilon(x)\,dx
=
\int_{|x|>\delta} \frac{1}{\sqrt{\pi}\,\varepsilon}e^{-(x/\varepsilon)^2}\,dx
=
\frac{1}{\sqrt{\pi}}\int_{|u|>\delta/\varepsilon} e^{-u^2}\,du.
\]
As $\varepsilon\to 0$, the region $|u|>\delta/\varepsilon$ moves out to infinity, and the tail integral of $e^{-u^2}$ tends to zero. Therefore $\int_{|x|>\delta} \psi_\varepsilon(x)\,dx \to 0$, and hence $\bigl|I_\varepsilon^{\text{far}}\bigr|\to 0$.

Next, consider the near part:
\[
I_\varepsilon^{\text{near}}
=
\int_{|x|\le \delta} f(x)\,\psi_\varepsilon(x)\,dx.
\]
Add and subtract $f(0)$ inside the integral:
\[
I_\varepsilon^{\text{near}} - f(0)\int_{|x|\le \delta} \psi_\varepsilon(x)\,dx
=
\int_{|x|\le \delta} \bigl(f(x)-f(0)\bigr)\,\psi_\varepsilon(x)\,dx.
\]
By continuity of $f$ at $0$, there exists $\eta>0$ such that $|x|\le \eta$ implies $|f(x)-f(0)|<\delta$. If we also require $0<\varepsilon\le \eta$, then over $|x|\le \delta$ (in particular, over $|x|\le\eta$) we have $|f(x)-f(0)|<\delta$. Hence
\[
\left|
\int_{|x|\le \delta} \bigl(f(x)-f(0)\bigr)\,\psi_\varepsilon(x)\,dx
\right|
\le
\delta\int_{|x|\le \delta} \psi_\varepsilon(x)\,dx
\le
\delta\int_{-\infty}^\infty \psi_\varepsilon(x)\,dx
=
\delta,
\]
using that the total mass is one. Thus
\[
\left|I_\varepsilon^{\text{near}} - f(0)\int_{|x|\le \delta} \psi_\varepsilon(x)\,dx\right|\le \delta.
\]
Since $\int_{|x|\le \delta} \psi_\varepsilon(x)\,dx \le 1$ and $\int_{|x|>\delta} \psi_\varepsilon(x)\,dx \to 0$, we also have
\[
\int_{|x|\le \delta} \psi_\varepsilon(x)\,dx
=
1 - \int_{|x|>\delta} \psi_\varepsilon(x)\,dx
\longrightarrow 1
\quad\text{as }\varepsilon\to 0.
\]
Combining these estimates, and using that $I_\varepsilon = I_\varepsilon^{\text{near}} + I_\varepsilon^{\text{far}}$, we deduce
\[
I_\varepsilon \longrightarrow f(0)\cdot 1 = f(0),
\]
because $I_\varepsilon^{\text{far}}\to 0$, the integral of $\psi_\varepsilon$ over $|x|\le\delta$ tends to $1$, and the error in replacing $f(x)$ by $f(0)$ on $|x|\le\delta$ is arbitrarily small. Thus $(\psi_\varepsilon)$ also converges to $\delta$ in the distributional sense.

\smallskip

\emph{Sinc approximation via Fourier analysis.}
Here we assume that $f$ is smooth and rapidly decaying, for example a Schwartz function. This allows us to use the Fourier transform and its inversion formula without technical concerns.

Recall that the Fourier transform of $f$ is
\[
\widehat{f}(\xi) = \int_{-\infty}^{\infty} f(x)\,e^{-i\xi x}\,dx,
\]
and the inverse transform can be written (under suitable hypotheses) as
\[
f(0)
=
\frac{1}{2\pi}\int_{-\infty}^{\infty} \widehat{f}(\xi)\,d\xi.
\]
We now examine the integral
\[
J_N := \int_{-\infty}^{\infty} f(x)\,\eta_N(x)\,dx
=
\int_{-\infty}^{\infty} f(x)\,\frac{1}{\pi}\,\frac{\sin(Nx)}{x}\,dx.
\]

The key observation is that $\eta_N$ can be represented as a (scaled) inverse Fourier transform of the indicator function of the interval $[-N,N]$:
\[
\eta_N(x) = \frac{1}{2\pi}\int_{-N}^{N} e^{i\xi x}\,d\xi.
\]
One can verify this by integrating $e^{i\xi x}$ with respect to $\xi$:
\[
\frac{1}{2\pi}\int_{-N}^{N} e^{i\xi x}\,d\xi
=
\frac{1}{2\pi}\left[\frac{e^{i\xi x}}{ix}\right]_{\xi=-N}^{\xi=N}
=
\frac{1}{2\pi}\,\frac{e^{iNx} - e^{-iNx}}{ix}
=
\frac{1}{2\pi}\,\frac{2i\sin(Nx)}{ix}
=
\frac{1}{\pi}\,\frac{\sin(Nx)}{x}
=
\eta_N(x).
\]
Therefore,
\[
J_N
=
\int_{-\infty}^{\infty} f(x)\left( \frac{1}{2\pi}\int_{-N}^{N} e^{i\xi x}\,d\xi \right)dx.
\]
By Fubini’s theorem (justified by the rapid decay of $f$), we may interchange the order of integration:
\[
J_N
=
\frac{1}{2\pi}\int_{-N}^{N} 
\left( \int_{-\infty}^{\infty} f(x)\,e^{i\xi x}\,dx \right)d\xi
=
\frac{1}{2\pi}\int_{-N}^{N} \widehat{f}(-\xi)\,d\xi.
\]
Since $f$ is real-valued, $\widehat{f}(-\xi)$ is related to $\widehat{f}(\xi)$ via complex conjugation, but in any case, as $N\to\infty$ the interval $[-N,N]$ expands to the whole real line. By absolute integrability of $\widehat{f}$ (which holds for Schwartz functions), we obtain
\[
\lim_{N\to\infty} J_N
=
\frac{1}{2\pi}\int_{-\infty}^{\infty} \widehat{f}(-\xi)\,d\xi.
\]
By the inverse Fourier transform at $x=0$, this integral equals $f(0)$. Hence
\[
\lim_{N\to\infty} \int_{-\infty}^{\infty} f(x)\,\eta_N(x)\,dx = f(0).
\]

Thus the family $(\eta_N)$ also converges to $\delta$ in the sense of distributions. Unlike the previous two families, $\eta_N$ takes both positive and negative values and exhibits increasingly rapid oscillations, rather than a simple positive peak shrinking to the origin. Nevertheless, its action on test functions is the same in the limit.

\medskip

\textbf{Conclusion and interpretation.}

In each of the three cases, we have shown that for appropriate classes of test functions $f$,
\[
\int_{\mathbb{R}} f(x)\,\phi_\varepsilon(x)\,dx \to f(0),\quad
\int_{\mathbb{R}} f(x)\,\psi_\varepsilon(x)\,dx \to f(0),\quad
\int_{\mathbb{R}} f(x)\,\eta_N(x)\,dx \to f(0).
\]
By definition of distributional convergence, this means
\[
\phi_\varepsilon \rightharpoonup \delta,\quad
\psi_\varepsilon \rightharpoonup \delta,\quad
\eta_N \rightharpoonup \delta,
\]
where the arrow denotes weak (distributional) convergence: convergence when integrated against all test functions.

This example illustrates a central idea of the Dirac delta-function: it is not a function but a linear functional on test functions, characterized by the identity
\[
\langle \delta, f\rangle = f(0).
\]
Different “approximate identities’’—rectangular pulses, Gaussian mollifiers, and oscillatory sinc kernels—provide distinct models for the same abstract distribution. What matters is not their pointwise behavior, but how they act under the integral sign. This viewpoint is fundamental in Fourier analysis and in the general theory of distributions used to model idealized point sources in applied mathematics.

\end{solution}

% ===== Example 5: Heaviside Step Function and Distributional Derivatives (inquiry-based) =====
\begin{problem}[Heaviside Step Function and Distributional Derivatives]
In many physical models, a system is ``switched on'' at a specific time, for instance when a voltage source is connected to a circuit at time $t = 0$. A simple idealized model of this behavior is the \emph{Heaviside step function}, which jumps from $0$ to $1$ at the origin. In classical calculus we cannot take an ordinary derivative at the jump point, but in the framework of distributions we can make precise sense of the ``derivative'' of this function. In this problem you will rediscover, step by step, that the distributional derivative of the Heaviside function is exactly the Dirac delta at the origin.

We work on the real line $\mathbb{R}$ and use $\mathcal{D}(\mathbb{R})$ to denote the space of smooth, compactly supported test functions $\varphi \colon \mathbb{R} \to \mathbb{R}$.

\medskip

(a) Define the Heaviside step function $H \colon \mathbb{R} \to \mathbb{R}$ by
\[
H(x) = 
\begin{cases}
0, & x < 0,\\
1, & x > 0.
\end{cases}
\]
For now you may leave $H(0)$ undefined or choose any convenient value. 

\begin{enumerate}
\item[(i)] In the sense of ordinary calculus, for which values of $x$ does $H$ have a classical derivative, and what is that derivative?
\item[(ii)] Explain carefully why $H$ is not differentiable at $x = 0$ in the classical sense. What goes wrong if you try to compute the derivative at $0$ using the definition with limits?
\end{enumerate}

(b) In distribution theory, we do not differentiate functions pointwise. Instead, we define derivatives by how they act on test functions. Recall that a locally integrable function $f$ defines a distribution $T_f$ by
\[
\langle T_f, \varphi \rangle = \int_{\mathbb{R}} f(x)\,\varphi(x)\,dx, \qquad \varphi \in \mathcal{D}(\mathbb{R}),
\]
and that the \emph{distributional derivative} $T'$ of a distribution $T$ is defined by
\[
\langle T', \varphi \rangle = - \langle T, \varphi' \rangle
\quad \text{for all } \varphi \in \mathcal{D}(\mathbb{R}).
\]

\begin{enumerate}
\item[(i)] Show that the Heaviside function $H$ defines a distribution $T_H$ via the above formula. What regularity of $H$ do you use to justify that $T_H$ is well defined?
\item[(ii)] Write down explicitly the defining formula for the distributional derivative $T_H'$ in terms of an integral involving $H$ and $\varphi'$. That is, express $\langle T_H', \varphi \rangle$ as an integral.
\end{enumerate}
Hint: You should obtain an expression of the form
\[
\langle T_H', \varphi \rangle = -\int_{\mathbb{R}} H(x)\,\varphi'(x)\,dx.
\]

(c) Now exploit the particular form of $H$. 

\begin{enumerate}
\item[(i)] Using the definition of $H$, split the integral
\[
\int_{\mathbb{R}} H(x)\,\varphi'(x)\,dx
\]
into integrals over the regions $(-\infty,0)$ and $(0,\infty)$, and simplify as much as possible.
Hint: On which intervals is $H$ equal to $0$ or $1$?

\item[(ii)] Recall that every test function $\varphi \in \mathcal{D}(\mathbb{R})$ has compact support, so there is a large $R > 0$ such that $\varphi(x) = 0$ for $|x| \ge R$. Use this fact, together with the Fundamental Theorem of Calculus, to evaluate the integral $\displaystyle \int_{0}^{\infty} \varphi'(x)\,dx$ in terms of $\varphi(0)$.
\end{enumerate}
Hint: Replace the upper limit $\infty$ by $R$ and note that $\varphi(R)=0$.

(d) Combine your work from parts (b) and (c) to compute $\langle T_H', \varphi \rangle$ explicitly in terms of $\varphi(0)$, for an arbitrary test function $\varphi \in \mathcal{D}(\mathbb{R})$. 

\begin{enumerate}
\item[(i)] Show that there is a constant $C$ such that 
\[
\langle T_H', \varphi \rangle = C\,\varphi(0)
\quad \text{for all } \varphi \in \mathcal{D}(\mathbb{R}).
\]
What is the value of $C$?

\item[(ii)] Recall that the Dirac delta distribution $\delta$ at the origin is defined by
\[
\langle \delta, \varphi \rangle = \varphi(0), \qquad \varphi \in \mathcal{D}(\mathbb{R}).
\]
Using this characterization, conclude that $T_H' = \delta$ as distributions. In words: the distributional derivative of the Heaviside step function is the Dirac delta at the origin.
\end{enumerate}

(e) Extensions and variations.

\begin{enumerate}
\item[(i)] Consider the shifted Heaviside function $H_a(x) = H(x-a)$, which ``turns on'' at $x = a$. Repeat the computation conceptually (you do not need to write every step) and determine the distributional derivative of $H_a$. Which delta-distribution appears, and at which point?

\item[(ii)] Let $f$ be a piecewise constant function with finitely many jumps:
\[
f(x) = c_0 \quad (x < x_1), \qquad
f(x) = c_1 \quad (x_1 < x < x_2), \quad \dots, \quad
f(x) = c_n \quad (x > x_n),
\]
where $x_1 < \cdots < x_n$ and $c_0,\dots,c_n$ are real constants. Based on your work with $H$, make a conjecture for the distributional derivative $f'$ in terms of Dirac deltas at the jump points $x_k$. How do the coefficients of these delta distributions relate to the jump sizes $c_k - c_{k-1}$?
Hint: Try writing $f$ as a linear combination of shifted Heaviside functions plus a constant.
\end{enumerate}

\end{problem}

% ===== Example 5: Heaviside Step Function and Distributional Derivatives (full solution) =====
\begin{problem}[Heaviside Step Function and Distributional Derivatives]
Let $H \colon \mathbb{R} \to \mathbb{R}$ be the Heaviside step function
\[
H(x) =
\begin{cases}
0, & x < 0,\\
1, & x > 0,
\end{cases}
\]
with any value assigned at $x=0$. Regard $H$ as a distribution $T_H$ acting on test functions $\varphi \in \mathcal{D}(\mathbb{R})$ by
\[
\langle T_H, \varphi \rangle = \displaystyle \int_{\mathbb{R}} H(x)\,\varphi(x)\,dx.
\]
Recall that the distributional derivative $T_H'$ is defined by
\[
\langle T_H', \varphi \rangle = -\langle T_H, \varphi' \rangle.
\]

\begin{enumerate}
\item[(a)] Compute $\langle T_H', \varphi \rangle$ explicitly for an arbitrary test function $\varphi$ and show that
\[
\langle T_H', \varphi \rangle = \varphi(0).
\]
\item[(b)] Deduce that, as distributions,
\[
T_H' = \delta,
\]
where $\delta$ is the Dirac delta at the origin.
\item[(c)] More generally, for $a \in \mathbb{R}$, define $H_a(x) = H(x-a)$. Compute the distributional derivative of $H_a$ and express it in terms of a Dirac delta concentrated at $x=a$.
\end{enumerate}
\end{problem}

\begin{solution}
We first recall the setting. A locally integrable function $f$ defines a distribution $T_f$ by
\[
\langle T_f, \varphi \rangle = \int_{\mathbb{R}} f(x)\,\varphi(x)\,dx
\quad \text{for all } \varphi \in \mathcal{D}(\mathbb{R}),
\]
and the distributional derivative $T'$ of a distribution $T$ is defined by
\[
\langle T', \varphi \rangle = -\langle T, \varphi' \rangle.
\]
The central idea is that differentiation is transferred from the (possibly rough) distribution onto the smooth test function, thereby making sense of derivatives for objects like $H$ which are not classically differentiable.

\medskip

\noindent\textbf{(a) Computation of $\langle T_H', \varphi \rangle$.}
The Heaviside function $H$ is locally integrable, so it defines a distribution $T_H$ in the usual way. For a test function $\varphi \in \mathcal{D}(\mathbb{R})$ we have
\[
\langle T_H', \varphi \rangle = -\langle T_H, \varphi' \rangle
= -\int_{\mathbb{R}} H(x)\,\varphi'(x)\,dx.
\]
We now exploit the simple form of $H$. Since $H(x)=0$ for $x<0$ and $H(x)=1$ for $x>0$, we can split the integral:
\[
\int_{\mathbb{R}} H(x)\,\varphi'(x)\,dx
= \int_{-\infty}^{0} 0\cdot \varphi'(x)\,dx
  + \int_{0}^{\infty} 1\cdot \varphi'(x)\,dx
= \int_{0}^{\infty} \varphi'(x)\,dx.
\]
Thus
\[
\langle T_H', \varphi \rangle = -\int_{0}^{\infty} \varphi'(x)\,dx.
\]

Because $\varphi$ is a test function, it has compact support. Hence there exists some $R > 0$ such that $\varphi(x)=0$ for all $|x|\ge R$. In particular, for all sufficiently large $b \ge R$ we have $\varphi(b)=0$. Using the Fundamental Theorem of Calculus, we can therefore write
\[
\int_{0}^{\infty} \varphi'(x)\,dx
= \lim_{b\to\infty} \int_{0}^{b} \varphi'(x)\,dx
= \lim_{b\to\infty} \bigl(\varphi(b) - \varphi(0)\bigr)
= 0 - \varphi(0)
= -\,\varphi(0).
\]
Substituting this into our earlier expression, we obtain
\[
\langle T_H', \varphi \rangle = -\left( \int_{0}^{\infty} \varphi'(x)\,dx \right)
= -\bigl(-\varphi(0)\bigr)
= \varphi(0).
\]

This identity holds for every test function $\varphi \in \mathcal{D}(\mathbb{R})$:
\[
\boxed{\;\langle T_H', \varphi \rangle = \varphi(0)\;}.
\]

\medskip

\noindent\textbf{(b) Identification of $T_H'$ with the Dirac delta.}
By definition, the Dirac delta distribution $\delta$ at the origin is characterized by
\[
\langle \delta, \varphi \rangle = \varphi(0)
\quad \text{for all } \varphi \in \mathcal{D}(\mathbb{R}).
\]
Comparing this with the identity we have just proved, we see that the linear functionals $T_H'$ and $\delta$ act in exactly the same way on every test function. In the theory of distributions, this means that they are the same distribution. Hence
\[
\boxed{\;T_H' = \delta\;}.
\]

Conceptually, this says that the derivative of a step of size one, concentrated at a point, is a point mass at that point. This is a precise realization of the idea that ``the derivative of a jump discontinuity is a delta spike.''

\medskip

\noindent\textbf{(c) The shifted Heaviside function.}
Let $a \in \mathbb{R}$ and define $H_a(x) = H(x-a)$, which is the Heaviside function shifted to the right by $a$. It jumps from $0$ to $1$ at $x = a$. As before, $H_a$ is locally integrable and therefore defines a distribution $T_{H_a}$ via
\[
\langle T_{H_a}, \varphi \rangle = \int_{\mathbb{R}} H_a(x)\,\varphi(x)\,dx
= \int_{\mathbb{R}} H(x-a)\,\varphi(x)\,dx.
\]

We compute its distributional derivative:
\[
\langle T_{H_a}', \varphi \rangle = -\int_{\mathbb{R}} H(x-a)\,\varphi'(x)\,dx.
\]
Again we use the structure of $H(x-a)$: it is $0$ for $x<a$ and $1$ for $x>a$, so
\[
\int_{\mathbb{R}} H(x-a)\,\varphi'(x)\,dx
= \int_{a}^{\infty} \varphi'(x)\,dx.
\]
As before, since $\varphi$ has compact support, there exists $R$ such that $\varphi(x)=0$ for $|x|\ge R$, and thus for $b$ large enough,
\[
\int_{a}^{\infty} \varphi'(x)\,dx
= \lim_{b\to\infty} \int_{a}^{b} \varphi'(x)\,dx
= \lim_{b\to\infty} (\varphi(b) - \varphi(a))
= 0 - \varphi(a)
= -\varphi(a).
\]
Therefore
\[
\langle T_{H_a}', \varphi \rangle = -\left( \int_{a}^{\infty} \varphi'(x)\,dx \right)
= -\bigl(-\varphi(a)\bigr)
= \varphi(a).
\]
By the definition of the Dirac delta $\delta_a$ concentrated at $x=a$,
\[
\langle \delta_a, \varphi \rangle = \varphi(a)
\quad \text{for all } \varphi \in \mathcal{D}(\mathbb{R}).
\]
We conclude that
\[
\boxed{\;T_{H_a}' = \delta_a\;}.
\]

\medskip

\noindent\textbf{Context within the Dirac delta section.}
This example illustrates one of the fundamental structural relationships in distribution theory: the Dirac delta is the distributional derivative of the Heaviside step function. In applications, this means that a sudden jump in a state variable (modeled by $H$ or its shifts) corresponds to a concentrated impulse (modeled by $\delta$) in its derivative. From the point of view of Fourier analysis and partial differential equations, this connection is central: it allows us to model impulsive forcing terms, interfaces, and discontinuities in a unified algebraic framework using delta distributions and distributional derivatives.
\end{solution}

\section{Closed-form Representation for Select Fourier Transforms}
% --- Narrative plan (auto-generated) ---
% This section develops a small but powerful toolbox of functions whose Fourier transforms can be written in closed form and manipulated reliably. Many applied problems in partial differential equations, signal processing, and dynamical systems reduce to computing or recognizing such transforms, especially when solving initial value problems, analyzing stability, or understanding how information propagates in space and time. By learning how to obtain and use these closed-form expressions, you will be able to solve model problems explicitly and to benchmark numerical and asymptotic methods.
%
% Our development emphasizes how Fourier analysis interacts with other parts of the mathematical toolkit. We show how to combine basic calculus techniques with ideas from ordinary differential equations, such as solving linear constant-coefficient equations in the frequency domain, and with tools from complex analysis, such as contour integration and residues, to compute transforms of rational functions. Along the way, we highlight structural features like symmetry, scaling, and convolution that allow you to generate new closed-form transforms from a few core examples. This perspective prepares you for later chapters, where these building blocks reappear inside Green’s functions for PDEs, dispersion relations for wave equations, and spectral representations of linear operators.

% ===== Example 1: Fourier Transform of Exponential Decay (inquiry-based) =====
\begin{problem}[Fourier Transform of Exponential Decay]
Many physical systems exhibit transient responses that decay exponentially in time. For instance, the voltage across a discharging capacitor in an RC circuit or the displacement of a lightly damped mass after a disturbance both decay like an exponential, and typically we are only interested in the response for $t \ge 0$. In this problem we examine the Fourier transform of such a ``one-sided'' exponential decay and see how the rate of decay is encoded in the frequency domain.

Let $a>0$ and define
\[
f(t) \coloneqq
\begin{cases}
e^{-a t}, & t \ge 0,\\[4pt]
0, & t<0.
\end{cases}
\]
We use the (angular) frequency convention
\[
\widehat{f}(\omega) \coloneqq \displaystyle\int_{-\infty}^{\infty} f(t)\, e^{-i\omega t}\,dt,
\qquad \omega \in \mathbb{R}.
\]

\medskip

(a) First look at a simpler, purely real integral. For $\alpha>0$, recall or re-derive the value of
\[
I(\alpha) \coloneqq \int_{0}^{\infty} e^{-\alpha t}\,dt.
\]
Compute $I(\alpha)$ explicitly. What condition on $\alpha$ is needed so that this improper integral converges?

\medskip

(b) We will now adapt this computation to complex exponents. Let $\lambda$ be a complex number with positive real part, $\operatorname{Re}(\lambda) > 0$. Consider the integral
\[
J(\lambda) \coloneqq \int_{0}^{\infty} e^{-\lambda t}\,dt.
\]
(i) Show that $|e^{-\lambda t}| = e^{-\operatorname{Re}(\lambda)\, t}$ for all $t \ge 0$. 

(ii) Use this to argue that $J(\lambda)$ converges absolutely whenever $\operatorname{Re}(\lambda) > 0$.

(iii) Compute $J(\lambda)$ by the same method you used in part (a), and express it in terms of $\lambda$.
Hint: Differentiate the function $t \mapsto e^{-\lambda t}$ and integrate, or evaluate the antiderivative $-\frac{1}{\lambda}e^{-\lambda t}$ at the endpoints $t=0$ and $t\to\infty$.

\medskip

(c) Return to the function $f$. Write down the Fourier transform $\widehat{f}(\omega)$ as an integral with correct limits for this one-sided exponential. Carefully justify why the lower limit is $0$ rather than $-\infty$.

Then simplify the integrand and show that
\[
\widehat{f}(\omega)
= \int_{0}^{\infty} e^{-(a + i\omega)t}\,dt.
\]
Explain why the parameter $a>0$ ensures that this integral converges for all real $\omega$ by relating it to part (b).

\medskip

(d) Use your formula for $J(\lambda)$ from part (b)(iii) with $\lambda = a + i\omega$ to obtain a closed-form expression for $\widehat{f}(\omega)$. Simplify your answer as far as you can, for instance by rationalizing the denominator to separate real and imaginary parts if you find that helpful.

Next, find the magnitude $|\widehat{f}(\omega)|$ and the phase $\arg \widehat{f}(\omega)$ as functions of $\omega$. How does the decay rate $a$ affect the size of $\widehat{f}(\omega)$ for large $|\omega|$?

\medskip

(e) \textbf{Extensions and “what if” questions.}

(i) Suppose the exponential decay begins at a later time $t_0>0$ instead of at $t=0$:
\[
g(t) \coloneqq
\begin{cases}
e^{-a (t-t_0)}, & t \ge t_0,\\[4pt]
0, & t<t_0.
\end{cases}
\]
Using the time-shift property of the Fourier transform (which you may recall or derive), predict the form of $\widehat{g}(\omega)$ in terms of $\widehat{f}(\omega)$.

(ii) Consider instead the two-sided exponential
\[
h(t) \coloneqq e^{-a|t|}, \qquad a>0.
\]
Without doing all the details, outline how you would set up the integral for $\widehat{h}(\omega)$ and which of the ideas from this problem you would reuse. How do you expect the answer to differ qualitatively from $\widehat{f}(\omega)$, which is one-sided in time?

\end{problem}

% ===== Example 1: Fourier Transform of Exponential Decay (full solution) =====
\begin{problem}[Fourier Transform of Exponential Decay]
Let $a>0$ and define
\[
f(t) \coloneqq
\begin{cases}
e^{-a t}, & t \ge 0,\\[4pt]
0, & t<0.
\end{cases}
\]
Using the Fourier transform convention
\[
\widehat{f}(\omega) \coloneqq \displaystyle\int_{-\infty}^{\infty} f(t)\, e^{-i\omega t}\,dt,
\qquad \omega \in \mathbb{R},
\]
compute $\widehat{f}(\omega)$ in closed form. Then determine its magnitude $|\widehat{f}(\omega)|$ and phase $\arg\widehat{f}(\omega)$ as functions of $\omega$ and briefly comment on how the decay rate $a$ influences the frequency-domain behavior.
\end{problem}

\begin{solution}
We are asked to compute the Fourier transform of a one-sided exponential decay,
\[
f(t) =
\begin{cases}
e^{-a t}, & t \ge 0,\\[4pt]
0, & t<0,
\end{cases}
\quad\text{with } a>0,
\]
under the convention
\[
\widehat{f}(\omega) = \displaystyle\int_{-\infty}^{\infty} f(t)\, e^{-i\omega t}\,dt.
\]

\medskip

\noindent\textbf{1. Reducing to an elementary exponential integral.}
Because $f(t)$ vanishes for $t<0$, the integral over $(-\infty,0)$ contributes nothing. Thus
\[
\widehat{f}(\omega)
= \int_{-\infty}^{\infty} f(t)\, e^{-i\omega t}\,dt
= \int_{0}^{\infty} e^{-a t}\, e^{-i\omega t}\,dt.
\]
We combine the exponents:
\[
e^{-a t} e^{-i\omega t} = e^{-(a+i\omega)t},
\]
and obtain
\[
\widehat{f}(\omega)
= \int_{0}^{\infty} e^{-(a+i\omega)t}\,dt.
\]

This is a standard improper integral of an exponential with a complex parameter. The central idea is to recognize it as the same type of integral that defines the Laplace transform of $e^{-a t}$, evaluated at the complex number $a+i\omega$.

\medskip

\noindent\textbf{2. Convergence and evaluation of the exponential integral.}
Let $\lambda$ be a complex number with $\operatorname{Re}(\lambda) > 0$. We recall the general fact
\[
\int_0^\infty e^{-\lambda t}\,dt = \frac{1}{\lambda},
\qquad \text{for } \operatorname{Re}(\lambda) > 0.
\]
For completeness, we justify this formula briefly.

Write $\lambda = \alpha + i\beta$ with $\alpha = \operatorname{Re}(\lambda)$ and $\beta = \operatorname{Im}(\lambda)$. Then
\[
e^{-\lambda t} = e^{-(\alpha + i\beta)t} = e^{-\alpha t} e^{-i\beta t},
\]
so
\[
\bigl|e^{-\lambda t}\bigr| = e^{-\alpha t} \bigl|e^{-i\beta t}\bigr|
= e^{-\alpha t},
\]
because $|e^{-i\beta t}| = 1$ for all real $t$. If $\alpha>0$, then $e^{-\alpha t}$ decays to zero as $t\to\infty$, and the integral
\[
\int_0^\infty \bigl|e^{-\lambda t}\bigr|\,dt
= \int_0^\infty e^{-\alpha t}\,dt
\]
converges. Therefore, $\int_0^\infty e^{-\lambda t}\,dt$ converges absolutely whenever $\operatorname{Re}(\lambda)>0$.

To compute it, we note that an antiderivative of $e^{-\lambda t}$ is $-\frac{1}{\lambda}e^{-\lambda t}$, valid for any $\lambda\neq 0$. Hence
\[
\int_0^T e^{-\lambda t}\,dt
= \left[-\frac{1}{\lambda}e^{-\lambda t}\right]_{t=0}^{t=T}
= -\frac{1}{\lambda}e^{-\lambda T} + \frac{1}{\lambda}.
\]
If $\operatorname{Re}(\lambda) > 0$, then $e^{-\lambda T} \to 0$ as $T\to\infty$, so
\[
\int_0^\infty e^{-\lambda t}\,dt
= \lim_{T\to\infty} \int_0^T e^{-\lambda t}\,dt
= \lim_{T\to\infty} \left(-\frac{1}{\lambda}e^{-\lambda T} + \frac{1}{\lambda} \right)
= \frac{1}{\lambda}.
\]

\medskip

\noindent\textbf{3. Applying the formula to our Fourier transform.}
In our case, the parameter is
\[
\lambda = a + i\omega.
\]
Its real part is $\operatorname{Re}(\lambda) = a > 0$, so the argument above applies. Therefore,
\[
\widehat{f}(\omega)
= \int_{0}^{\infty} e^{-(a+i\omega)t}\,dt
= \frac{1}{a + i\omega},
\qquad \omega \in \mathbb{R}.
\]
This is already a closed-form expression for the Fourier transform: the transform of a one-sided exponential decay is the simple rational function
\[
\widehat{f}(\omega) = \frac{1}{a + i\omega}.
\]

This is a fundamental example in Fourier analysis, illustrating how a time-domain exponential decay is represented by a rational function of the frequency variable.

\medskip

\noindent\textbf{4. Magnitude and phase.}
To better understand the frequency response, we now compute the magnitude and phase of $\widehat{f}(\omega)$. We first rewrite
\[
\widehat{f}(\omega) = \frac{1}{a + i\omega}
= \frac{a - i\omega}{a^2 + \omega^2},
\]
by multiplying numerator and denominator by the complex conjugate $a - i\omega$.

From this expression, we can read off the real and imaginary parts:
\[
\operatorname{Re}\widehat{f}(\omega) = \frac{a}{a^2 + \omega^2},
\qquad
\operatorname{Im}\widehat{f}(\omega) = -\frac{\omega}{a^2 + \omega^2}.
\]
The magnitude is
\[
\bigl|\widehat{f}(\omega)\bigr|
= \sqrt{\bigl(\operatorname{Re}\widehat{f}(\omega)\bigr)^2
+ \bigl(\operatorname{Im}\widehat{f}(\omega)\bigr)^2}
= \sqrt{\frac{a^2 + \omega^2}{(a^2 + \omega^2)^2}}
= \frac{1}{\sqrt{a^2 + \omega^2}}.
\]
The phase $\arg\widehat{f}(\omega)$ is the argument of the complex number $(a - i\omega)/(a^2 + \omega^2)$, which is the same as the argument of $a - i\omega$. A convenient way to express it is to note that
\[
a + i\omega = \sqrt{a^2 + \omega^2}\,e^{i\theta},
\quad\text{where } \theta = \arctan\!\left(\frac{\omega}{a}\right)
\]
(for $a>0$ this principal value is unambiguous), so
\[
\widehat{f}(\omega) = \frac{1}{a + i\omega}
= \frac{1}{\sqrt{a^2 + \omega^2}}\,e^{-i\theta}.
\]
Thus
\[
\bigl|\widehat{f}(\omega)\bigr| = \frac{1}{\sqrt{a^2 + \omega^2}},
\qquad
\arg\widehat{f}(\omega)
= -\arctan\!\left(\frac{\omega}{a}\right).
\]

\medskip

\noindent\textbf{5. Effect of the decay rate $a$.}
The parameter $a>0$ controls the rate at which the signal decays in time. A larger value of $a$ means stronger damping and a shorter-lived transient.

In the frequency domain, the magnitude
\[
\bigl|\widehat{f}(\omega)\bigr|
= \frac{1}{\sqrt{a^2 + \omega^2}}
\]
is largest at $\omega = 0$, where it equals $1/a$, and decays like $1/|\omega|$ for large $|\omega|$. As $a$ increases, the value at $\omega=0$ decreases and the whole curve becomes ``narrower'' around $\omega=0$. This is consistent with the general time–frequency tradeoff: a more rapidly decaying function in time tends to have a broader (but lower-amplitude) representation in frequency.

The phase
\[
\arg\widehat{f}(\omega) = -\arctan\!\left(\frac{\omega}{a}\right)
\]
varies smoothly from approximately $0$ at low frequencies to approximately $-\pi/2$ as $|\omega|$ becomes large, with the transition scale controlled by $a$.

\medskip

\noindent\textbf{6. Relation to the chapter theme.}
This example showcases the main idea of the section on ``Closed-form Representation for Select Fourier Transforms.'' By recognizing that the Fourier transform integral matches a standard exponential integral with a complex parameter, we obtain a simple, explicit formula
\[
\widehat{f}(\omega) = \frac{1}{a + i\omega},
\]
without resorting to numerical methods. Such rational functions recur throughout applied mathematics, especially in the analysis of linear time-invariant systems, where exponential decays are ubiquitous and their Fourier transforms encode damping and phase shift in a particularly transparent way.

\end{solution}

% ===== Example 2: The Gaussian and the Heat Kernel (inquiry-based) =====
\begin{problem}[The Gaussian and the Heat Kernel]
In many physical systems, random fluctuations and heat diffusion produce bell-shaped profiles in space. A sharp, localized initial temperature distribution on a metal rod quickly smooths out into something that looks Gaussian; likewise, the position of a Brownian particle after a fixed time is distributed according to a Gaussian. In Fourier analysis, the Gaussian function is special because its Fourier transform can be computed explicitly and turns out to be another Gaussian. This example also leads directly to the closed-form “heat kernel,” which is the fundamental solution of the heat equation.

Throughout this problem, we use the (non-unitary) Fourier transform on $\mathbb{R}$ given by
\[
\widehat{f}(\xi) \;=\; \int_{-\infty}^{\infty} f(x)\,e^{-i\xi x}\,dx, \qquad \xi\in\mathbb{R},
\]
whenever the integral converges absolutely.

\smallskip

Let $a>0$ and consider the Gaussian
\[
f_a(x) \;=\; e^{-a x^2}, \qquad x\in\mathbb{R}.
\]

\medskip

\textbf{(a) First exploration: symmetry and qualitative shape.}  

(i) Show that $f_a$ is an even function of $x$. Deduce that its Fourier transform $\widehat{f_a}(\xi)$ is real-valued and even as a function of $\xi$.  
(ii) Without computing any integrals, make a qualitative sketch of what you expect $\widehat{f_a}(\xi)$ to look like. In particular, discuss whether you expect it to decay as $|\xi|\to\infty$ and whether you expect any oscillations.

\emph{Hint:} Recall that the Fourier transform of a “nice,” localized bump is usually another smooth, decaying function in frequency space. Think about what happens if you differentiate $\widehat{f_a}$ with respect to $\xi$.

\medskip

\textbf{(b) Finding a differential equation for the Fourier transform of a Gaussian.}  

Define
\[
F_a(\xi) \;=\; \widehat{f_a}(\xi) \;=\; \int_{-\infty}^{\infty} e^{-a x^2} e^{-i\xi x}\,dx.
\]
Differentiate $F_a(\xi)$ with respect to $\xi$ under the integral sign to obtain
\[
F_a'(\xi) \;=\; \int_{-\infty}^{\infty} \bigl(-i x\bigr)\,e^{-a x^2} e^{-i\xi x}\,dx.
\]

(i) Rewrite the factor $x e^{-a x^2}$ as a derivative with respect to $x$ of $e^{-a x^2}$, up to a constant multiple.  

(ii) Use this identity and integration by parts in $x$ to express $F_a'(\xi)$ in terms of $F_a(\xi)$ itself.

\emph{Hint:} Notice that
\[
x e^{-a x^2} = -\frac{1}{2a}\,\frac{d}{dx}\bigl(e^{-a x^2}\bigr).
\]
Move the derivative from $e^{-a x^2}$ onto $e^{-i\xi x}$ using integration by parts, and argue that the boundary terms vanish because $e^{-a x^2}$ decays rapidly as $|x|\to\infty$.

\medskip

\textbf{(c) Solving the resulting ordinary differential equation.}  

From part (b), you should obtain a first-order linear differential equation of the form
\[
F_a'(\xi) \;=\; -\frac{\xi}{2a}\,F_a(\xi).
\]

(i) Solve this ordinary differential equation for $F_a(\xi)$, and show that the general solution has the form
\[
F_a(\xi) \;=\; C_a\, e^{-\frac{\xi^2}{4a}},
\]
for some constant $C_a$ that may depend on $a$ but not on $\xi$.

(ii) Determine the constant $C_a$ by evaluating $F_a(0)$ directly from the definition of the Fourier transform.

\emph{Hint:} You need the value of the Gaussian integral
\[
\int_{-\infty}^{\infty} e^{-a x^2}\,dx.
\]
You may recall (or re-derive) that this integral equals $\sqrt{\pi/a}$. If you wish to derive it, consider squaring the integral, interpreting it as a double integral over $\mathbb{R}^2$, and switching to polar coordinates.

\medskip

\textbf{(d) The explicit Fourier transform of the Gaussian and the heat kernel.}  

(i) Combine your work from part (c) to obtain a closed-form formula for $\widehat{f_a}(\xi)$.  

(ii) Interpret this formula as saying that the Fourier transform of a Gaussian is another Gaussian. How are the “widths” of the two Gaussians related?  

\emph{Hint:} Think about how the parameter $a$ controls the spread of $e^{-a x^2}$ in $x$-space, and how the parameter in the exponent of $\widehat{f_a}(\xi)$ controls the spread in $\xi$-space.

Now let $\kappa>0$ and $t>0$, and define the \emph{one-dimensional heat kernel}
\[
K(x,t) \;=\; \frac{1}{\sqrt{4\pi \kappa t}}\;\exp\!\left(-\frac{x^2}{4\kappa t}\right).
\]

(iii) Using your formula for $\widehat{f_a}$ and appropriate scaling in $x$, compute the Fourier transform (in $x$) of $K(\cdot,t)$ and show that
\[
\widehat{K(\cdot,t)}(\xi) \;=\; e^{-\kappa t\,\xi^2}.
\]

\emph{Hint:} First compute the transform of $e^{-\alpha x^2}$ for a general $\alpha>0$. Then use the scaling property of the Fourier transform: if $g(x) = f(bx)$ with $b\neq 0$, how is $\widehat{g}$ related to $\widehat{f}$?

\medskip

\textbf{(e) Extensions and “what if” questions.}  

(i) The heat kernel $K(x,t)$ is supposed to be the fundamental solution of the heat equation
\[
u_t = \kappa u_{xx}, \qquad x\in\mathbb{R}, \ t>0,
\]
in the sense that the solution with initial data $u(x,0)=\varphi(x)$ is given by convolution:
\[
u(x,t) \;=\; (K(\cdot,t)*\varphi)(x) \;=\; \int_{-\infty}^{\infty} K(x-y,t)\,\varphi(y)\,dy.
\]
Use the Fourier transform with respect to $x$ to check that this convolution formula indeed solves the heat equation whenever $\varphi$ is nice enough (for example, rapidly decaying and smooth).

\emph{Hint:} Take the Fourier transform in $x$ of both sides of the heat equation. You should obtain a simple ordinary differential equation in $t$ for $\widehat{u}(\xi,t)$, and you already know the Fourier transform of $K(\cdot,t)$.

(ii) What happens in higher dimensions? Suppose $x\in\mathbb{R}^n$ and consider
\[
f_a(x) \;=\; e^{-a|x|^2}, \qquad |x|^2 = x_1^2+\cdots + x_n^2.
\]
Based on the one-dimensional result and the fact that the $n$-dimensional Fourier transform factorizes over coordinates, what do you expect the Fourier transform of $f_a$ to look like in $\mathbb{R}^n$? How does this relate to the heat kernel in higher dimensions?
\end{problem}

% ===== Example 2: The Gaussian and the Heat Kernel (full solution) =====
\begin{problem}[The Gaussian and the Heat Kernel]
Let $a>0$ and define $f_a(x)=e^{-a x^2}$ on $\mathbb{R}$. Using the Fourier transform
\[
\widehat{f}(\xi) \;=\; \int_{-\infty}^{\infty} f(x)\,e^{-i\xi x}\,dx,
\]
do the following:

\begin{enumerate}
  \item Show that the Fourier transform $F_a(\xi)=\widehat{f_a}(\xi)$ satisfies the ordinary differential equation
  \[
  F_a'(\xi) = -\frac{\xi}{2a} F_a(\xi),
  \]
  and solve this equation to obtain an explicit formula for $F_a(\xi)$.

  \item Use your result to compute the Fourier transform (in $x$) of the one-dimensional heat kernel
  \[
  K(x,t) \;=\; \frac{1}{\sqrt{4\pi \kappa t}}\;\exp\!\left(-\frac{x^2}{4\kappa t}\right), \qquad \kappa>0,\ t>0,
  \]
  and show that $\widehat{K(\cdot,t)}(\xi)=e^{-\kappa t\,\xi^2}$.

  \item Briefly explain how this identity for $\widehat{K(\cdot,t)}$ implies that the convolution formula
  \[
  u(x,t) = (K(\cdot,t)*\varphi)(x) = \int_{-\infty}^{\infty} K(x-y,t)\,\varphi(y)\,dy
  \]
  gives the solution to the heat equation $u_t = \kappa u_{xx}$ with initial data $u(x,0)=\varphi(x)$, for sufficiently nice $\varphi$.
\end{enumerate}
\end{problem}

\begin{solution}
We work with the Fourier transform
\[
\widehat{f}(\xi) \;=\; \int_{-\infty}^{\infty} f(x)\,e^{-i\xi x}\,dx,
\]
and take $f_a(x)=e^{-a x^2}$ with $a>0$.

\medskip

\textbf{1. Differential equation for the Fourier transform of the Gaussian.}

Define
\[
F_a(\xi) \;=\; \widehat{f_a}(\xi)
\;=\; \int_{-\infty}^{\infty} e^{-a x^2} e^{-i\xi x}\,dx.
\]
Since $e^{-a x^2}$ decays rapidly at infinity, differentiation under the integral sign with respect to $\xi$ is justified. We obtain
\[
F_a'(\xi)
= \int_{-\infty}^{\infty} e^{-a x^2} \,\frac{d}{d\xi}\bigl(e^{-i\xi x}\bigr)\,dx
= \int_{-\infty}^{\infty} e^{-a x^2} \,(-ix)\,e^{-i\xi x}\,dx
= -i \int_{-\infty}^{\infty} x e^{-a x^2} e^{-i\xi x}\,dx.
\]
The key idea is to rewrite the factor $x e^{-a x^2}$ as a derivative with respect to $x$, which prepares the expression for integration by parts. A short computation shows that
\[
\frac{d}{dx}\bigl(e^{-a x^2}\bigr) = -2a x e^{-a x^2},
\]
so that
\[
x e^{-a x^2} = -\frac{1}{2a}\,\frac{d}{dx}\bigl(e^{-a x^2}\bigr).
\]
Substituting this into the expression for $F_a'(\xi)$ gives
\[
F_a'(\xi)
= -i \int_{-\infty}^{\infty} \left(-\frac{1}{2a}\,\frac{d}{dx}e^{-a x^2}\right)e^{-i\xi x}\,dx
= \frac{i}{2a} \int_{-\infty}^{\infty} \frac{d}{dx}\bigl(e^{-a x^2}\bigr)e^{-i\xi x}\,dx.
\]
We now integrate by parts in $x$:
\[
\int_{-\infty}^{\infty} \frac{d}{dx}\bigl(e^{-a x^2}\bigr)e^{-i\xi x}\,dx
= \Bigl[e^{-a x^2} e^{-i\xi x}\Bigr]_{x=-\infty}^{x=+\infty}
 - \int_{-\infty}^{\infty} e^{-a x^2} \,\frac{d}{dx}\bigl(e^{-i\xi x}\bigr)\,dx.
\]
The boundary term vanishes because $e^{-a x^2}$ decays exponentially as $|x|\to\infty$, whereas $e^{-i\xi x}$ has modulus one. Thus
\[
\int_{-\infty}^{\infty} \frac{d}{dx}\bigl(e^{-a x^2}\bigr)e^{-i\xi x}\,dx
= -\int_{-\infty}^{\infty} e^{-a x^2}(-i\xi)e^{-i\xi x}\,dx
= i\xi \int_{-\infty}^{\infty} e^{-a x^2} e^{-i\xi x}\,dx
= i\xi F_a(\xi).
\]
Substituting this back, we find
\[
F_a'(\xi) = \frac{i}{2a}\,\bigl(i\xi F_a(\xi)\bigr)
= -\frac{\xi}{2a}\,F_a(\xi).
\]
Therefore $F_a$ satisfies the first-order linear ordinary differential equation
\[
F_a'(\xi) = -\frac{\xi}{2a}\,F_a(\xi).
\]

To solve this equation, we separate variables:
\[
\frac{F_a'(\xi)}{F_a(\xi)} = -\frac{\xi}{2a}.
\]
Integrating with respect to $\xi$ gives
\[
\int \frac{F_a'(\xi)}{F_a(\xi)}\,d\xi = \int -\frac{\xi}{2a}\,d\xi,
\]
so
\[
\ln|F_a(\xi)| = -\frac{\xi^2}{4a} + C,
\]
for some constant of integration $C$. Exponentiating yields
\[
F_a(\xi) = C_a\,e^{-\frac{\xi^2}{4a}},
\]
where $C_a = \pm e^{C}$ is a constant depending on $a$ but not on $\xi$.

To determine $C_a$, we evaluate $F_a$ at $\xi=0$. By definition,
\[
F_a(0)
= \int_{-\infty}^{\infty} e^{-a x^2}\,dx.
\]
This is the standard Gaussian integral with parameter $a>0$. It is a well-known result, and one can show (for example by squaring the integral and using polar coordinates in $\mathbb{R}^2$) that
\[
\int_{-\infty}^{\infty} e^{-a x^2}\,dx = \sqrt{\frac{\pi}{a}}.
\]
On the other hand, from the explicit form $F_a(\xi)=C_a e^{-\xi^2/(4a)}$ we have
\[
F_a(0) = C_a e^{0} = C_a.
\]
Therefore
\[
C_a = \sqrt{\frac{\pi}{a}},
\]
and hence
\[
F_a(\xi) = \sqrt{\frac{\pi}{a}}\,e^{-\frac{\xi^2}{4a}}.
\]

We have thus obtained the closed-form formula
\[
\widehat{f_a}(\xi) = \int_{-\infty}^{\infty} e^{-a x^2}e^{-i\xi x}\,dx
= \sqrt{\frac{\pi}{a}}\,e^{-\frac{\xi^2}{4a}}, \qquad \xi\in\mathbb{R}.
\]
This shows that the Fourier transform of a Gaussian is another Gaussian, with the parameter in the exponent inverted in a precise way.

\medskip

\textbf{2. Fourier transform of the heat kernel.}

We now apply this result to the heat kernel
\[
K(x,t)
= \frac{1}{\sqrt{4\pi\kappa t}} \exp\!\left(-\frac{x^2}{4\kappa t}\right),
\qquad \kappa>0,\ t>0.
\]
For each fixed $t>0$, this is a spatial Gaussian with parameter
\[
a = \frac{1}{4\kappa t}.
\]
Indeed, we can write
\[
K(x,t) = \frac{1}{\sqrt{4\pi\kappa t}}\;e^{-a x^2}
\quad\text{with}\quad
a = \frac{1}{4\kappa t}.
\]
We already know that
\[
\widehat{e^{-a x^2}}(\xi) = \sqrt{\frac{\pi}{a}}\,e^{-\frac{\xi^2}{4a}}.
\]
Since the Fourier transform is linear, and the prefactor $1/\sqrt{4\pi\kappa t}$ does not depend on $x$, we have
\[
\widehat{K(\cdot,t)}(\xi)
= \frac{1}{\sqrt{4\pi\kappa t}}\;\widehat{e^{-a x^2}}(\xi)
= \frac{1}{\sqrt{4\pi\kappa t}}\;\sqrt{\frac{\pi}{a}}\,e^{-\frac{\xi^2}{4a}}.
\]
Substituting $a=1/(4\kappa t)$, we compute
\[
\sqrt{\frac{\pi}{a}}
= \sqrt{\pi\cdot 4\kappa t}
= \sqrt{4\pi\kappa t},
\]
so the prefactors cancel:
\[
\frac{1}{\sqrt{4\pi\kappa t}}\;\sqrt{\frac{\pi}{a}}
= \frac{1}{\sqrt{4\pi\kappa t}}\;\sqrt{4\pi\kappa t} = 1.
\]
Next, we evaluate the exponent:
\[
\frac{\xi^2}{4a}
= \frac{\xi^2}{4\cdot \frac{1}{4\kappa t}}
= \frac{\xi^2}{1/(\kappa t)}
= \kappa t\, \xi^2.
\]
Thus
\[
\widehat{K(\cdot,t)}(\xi)
= e^{-\kappa t\,\xi^2},
\]
as claimed. In other words, in frequency space, the heat kernel is simply the exponential damping factor $e^{-\kappa t\,\xi^2}$.

This reveals the central structural feature: the spatial Gaussian profile of the heat kernel corresponds in Fourier space to pure exponential decay in time at each frequency, with decay rate proportional to $\kappa \xi^2$. This is a clear instance of the theme of this section: certain special functions (such as Gaussians) admit closed-form Fourier transforms that greatly simplify the analysis of associated differential equations.

\medskip

\textbf{3. Solution of the heat equation via convolution with the heat kernel.}

Consider the one-dimensional heat equation
\[
u_t(x,t) = \kappa u_{xx}(x,t), \qquad x\in\mathbb{R}, \ t>0,
\]
with initial data
\[
u(x,0) = \varphi(x),
\]
where we assume $\varphi$ is smooth and decays sufficiently fast at infinity so that all Fourier transforms below are well defined.

We claim that the solution is given by the convolution formula
\[
u(x,t) = (K(\cdot,t)*\varphi)(x)
= \int_{-\infty}^{\infty} K(x-y,t)\,\varphi(y)\,dy.
\]

To justify this using Fourier analysis, we take the Fourier transform with respect to $x$. Let
\[
\widehat{u}(\xi,t) = \int_{-\infty}^{\infty} u(x,t)\,e^{-i\xi x}\,dx.
\]
Because differentiation in $x$ becomes multiplication by $i\xi$ under the Fourier transform, we have
\[
\widehat{u_x}(\xi,t) = i\xi\,\widehat{u}(\xi,t),
\qquad
\widehat{u_{xx}}(\xi,t) = (i\xi)^2\,\widehat{u}(\xi,t) = -\xi^2 \widehat{u}(\xi,t).
\]
Applying the Fourier transform in $x$ to the heat equation gives
\[
\frac{\partial}{\partial t}\widehat{u}(\xi,t)
= \widehat{u_t}(\xi,t)
= \kappa\,\widehat{u_{xx}}(\xi,t)
= \kappa(-\xi^2)\,\widehat{u}(\xi,t)
= -\kappa\xi^2\,\widehat{u}(\xi,t).
\]
Thus, for each fixed $\xi$, the function $t \mapsto \widehat{u}(\xi,t)$ satisfies the ordinary differential equation
\[
\frac{d}{dt}\widehat{u}(\xi,t) = -\kappa\xi^2\,\widehat{u}(\xi,t),
\]
with initial condition
\[
\widehat{u}(\xi,0) = \widehat{\varphi}(\xi).
\]
Solving this scalar linear ODE, we obtain
\[
\widehat{u}(\xi,t) = e^{-\kappa t\,\xi^2}\,\widehat{\varphi}(\xi).
\]
But in the previous part we showed that $e^{-\kappa t\,\xi^2}$ is precisely the Fourier transform of $K(\cdot,t)$:
\[
\widehat{K(\cdot,t)}(\xi) = e^{-\kappa t\,\xi^2}.
\]
Therefore
\[
\widehat{u}(\xi,t)
= \widehat{K(\cdot,t)}(\xi)\,\widehat{\varphi}(\xi).
\]
By the convolution theorem, the product of Fourier transforms corresponds to convolution in physical space, so
\[
u(x,t) = (K(\cdot,t)*\varphi)(x),
\]
which is the desired formula.

This completes the argument that the convolution with the heat kernel provides the solution to the heat equation with initial data $\varphi$. Conceptually, the key points are:

\begin{itemize}
  \item The Fourier transform turns the heat equation, a second-order partial differential equation in $x$, into a family of decoupled first-order ordinary differential equations in $t$, one for each frequency $\xi$.
  \item The special closed-form Fourier transform of the Gaussian shows that the spatial heat kernel corresponds exactly to the temporal decay factor $e^{-\kappa t\xi^2}$ in frequency space.
  \item The convolution representation of the solution is a direct consequence of the convolution theorem and the explicit knowledge of $\widehat{K(\cdot,t)}$.
\end{itemize}

Thus, this example illustrates the overarching idea of the section: for certain particularly nice functions, such as Gaussians, one can compute the Fourier transform in closed form, and these explicit formulas serve as powerful tools for solving and understanding important differential equations like the heat equation.
\end{solution}

% ===== Example 3: Rational Functions and Residues (inquiry-based) =====
\begin{problem}[Rational Functions and Residues]
In many models of waves and diffusion, Green's functions and response functions have Fourier transforms that are \emph{rational} functions of the frequency variable. A basic example is the so-called Lorentzian profile
\[
f(x) = \frac{1}{x^{2}+a^{2}}, \qquad a>0,
\]
whose Fourier transform controls spatial decay in several partial differential equations. Directly integrating its Fourier transform is not very pleasant, but complex analysis turns it into a clean residue computation. In this problem you will discover how the poles of a rational function in the complex plane encode exponential decay in the Fourier transform variable.

We use the convention
\[
\widehat{f}(\xi) = \int_{-\infty}^{\infty} f(x)\,e^{i\xi x}\,dx.
\]

(a) Consider the function
\[
I(\xi) = \int_{-\infty}^{\infty} \frac{e^{i\xi x}}{x^{2}+a^{2}}\,dx, \qquad a>0.
\]
Show that this improper integral converges for every real $\xi$, and prove that $I(\xi)$ is an even function of~$\xi$ (that is, $I(-\xi)=I(\xi)$).  
\emph{Hint:} Use that $x^{2}+a^{2}\ge a^{2}$ and that $\cos$ and $\sin$ are bounded.

(b) For the moment, fix $\xi>0$. Extend the integrand
\[
f(z) = \frac{e^{i\xi z}}{z^{2}+a^{2}}
\]
to the complex plane, and consider integrating $f$ along a closed contour $\Gamma_{R}$ consisting of the real interval $[-R,R]$ together with a semicircle of radius $R$.

\quad(i) Which half-plane (upper or lower) should you choose for the semicircle so that the contribution from the arc tends to zero as $R\to\infty$? Explain your choice by estimating $\lvert e^{i\xi z}\rvert$ on the arc.  
\emph{Hint:} Write $z = x+iy$ and compute $\lvert e^{i\xi z}\rvert$ in terms of $\xi$ and $y$.

\quad(ii) State which poles of $f(z)$ lie inside your chosen contour for $\xi>0$.

(c) Compute the residue of $f(z)$ at each pole that lies inside the contour for $\xi>0$.

\quad(i) First, locate the poles of $f(z)$ in the complex plane and identify which of them is inside your contour.  

\quad(ii) Compute the residue there.  
\emph{Hint:} You can write $z^{2}+a^{2}=(z-ia)(z+ia)$, or use the general formula
\[
\operatorname{Res}\left(\frac{g(z)}{h(z)},z_{0}\right) = \frac{g(z_{0})}{h'(z_{0})}
\]
when $h(z_{0})=0$ and $h'(z_{0})\neq0$.

(d) Use the residue theorem and your work above to evaluate $I(\xi)$ for $\xi>0$. Then use the evenness from part~(a) to find a single closed-form expression for $I(\xi)$ valid for all real $\xi$.  

\emph{Hint:} As $R\to\infty$, the contour integral is equal to $2\pi i$ times the sum of residues inside the contour. Relate this to the integral along the real axis.

(e) \textbf{Extensions and variations.}

\quad(i) Consider the function $g(x) = e^{-a\lvert x\rvert}$ with $a>0$. Compute its Fourier transform
\[
\widehat{g}(\xi) = \int_{-\infty}^{\infty} e^{-a\lvert x\rvert}e^{i\xi x}\,dx
\]
using only real-variable calculus (for instance, by splitting the integral at $x=0$). Simplify your answer. What type of function of $\xi$ do you obtain?

\quad(ii) Compare your expression for $\widehat{g}(\xi)$ with your result for $I(\xi)$. How are the two functions related? What does this say about the relationship between the exponential $e^{-a\lvert x\rvert}$ and the rational function $\dfrac{1}{\xi^{2}+a^{2}}$ via the Fourier transform?

\quad(iii) (Optional, more challenging.) Differentiate your formula for $I(\xi)$ with respect to the parameter $a$, and then justify differentiating under the integral sign to show that
\[
\int_{-\infty}^{\infty} \frac{e^{i\xi x}}{(x^{2}+a^{2})^{2}}\,dx
\]
can be expressed in closed form. What type of decay in $\xi$ does this higher-order pole produce?
\end{problem}

% ===== Example 3: Rational Functions and Residues (full solution) =====
\begin{problem}[Rational Functions and Residues]
Let $a>0$ and define, for real $\xi$,
\[
I(\xi) = \int_{-\infty}^{\infty} \frac{e^{i\xi x}}{x^{2}+a^{2}}\,dx.
\]
(a) Show that $I(\xi)$ is well defined for all real $\xi$ and that $I(\xi)$ is an even function of $\xi$.  

(b) Evaluate $I(\xi)$ in closed form using contour integration and the residue theorem.  

(c) Briefly explain how this example illustrates the way poles of a rational function in the complex plane encode exponential decay of its Fourier transform.
\end{problem}

\begin{solution}
We are asked to compute the Fourier transform of the Lorentzian profile $1/(x^{2}+a^{2})$ using complex analysis. This is a standard model example: the integrand is rational in $x$, but its transform will turn out to be exponentially decaying in~$\xi$. The computation relies on extending the integrand to a meromorphic function in the complex plane and evaluating an appropriate contour integral by the residue theorem.

\medskip

\noindent\textbf{(a) Convergence and evenness.}
For each real $\xi$ we consider
\[
I(\xi) = \int_{-\infty}^{\infty} \frac{e^{i\xi x}}{x^{2}+a^{2}}\,dx.
\]

First, note that for all real $x$ we have $x^{2}+a^{2}\ge a^{2}$, hence
\[
\left|\frac{e^{i\xi x}}{x^{2}+a^{2}}\right|
= \frac{1}{x^{2}+a^{2}}
\le \frac{1}{a^{2}}.
\]
More importantly, for large $\lvert x\rvert$ we have $x^{2}+a^{2}\sim x^{2}$, so
\[
\left|\frac{e^{i\xi x}}{x^{2}+a^{2}}\right|
\le \frac{C}{1+x^{2}}
\]
for some constant $C$ independent of $\xi$. Since
\[
\int_{-\infty}^{\infty} \frac{dx}{1+x^{2}} < \infty,
\]
the integral defining $I(\xi)$ converges absolutely for every real~$\xi$.

To see that $I(\xi)$ is even, write the exponential in terms of sine and cosine:
\[
e^{i\xi x} = \cos(\xi x) + i\sin(\xi x).
\]
Then
\[
I(\xi)
= \int_{-\infty}^{\infty} \frac{\cos(\xi x)}{x^{2}+a^{2}}\,dx
+ i\int_{-\infty}^{\infty} \frac{\sin(\xi x)}{x^{2}+a^{2}}\,dx.
\]
The function $x\mapsto \cos(\xi x)/(x^{2}+a^{2})$ is even in $x$, while $x\mapsto \sin(\xi x)/(x^{2}+a^{2})$ is odd. Hence the sine term integrates to zero over the symmetric interval $(-\infty,\infty)$, and we obtain
\[
I(\xi)
= \int_{-\infty}^{\infty} \frac{\cos(\xi x)}{x^{2}+a^{2}}\,dx
\in \mathbb{R}.
\]
Now observe that $\cos(\xi x)$ is an even function of $\xi$ as well: $\cos(-\xi x)=\cos(\xi x)$. Thus
\[
I(-\xi)
= \int_{-\infty}^{\infty} \frac{\cos(-\xi x)}{x^{2}+a^{2}}\,dx
= \int_{-\infty}^{\infty} \frac{\cos(\xi x)}{x^{2}+a^{2}}\,dx
= I(\xi),
\]
so $I$ is an even function of~$\xi$.

\medskip

\noindent\textbf{(b) Evaluation via residues.}
We now compute $I(\xi)$ using complex analysis. Extend the integrand to the complex function
\[
f(z) = \frac{e^{i\xi z}}{z^{2}+a^{2}},
\]
which is meromorphic on $\mathbb{C}$ with simple poles at the zeros of $z^{2}+a^{2}$, namely at
\[
z = ia \quad \text{and} \quad z = -ia.
\]

We first treat the case $\xi>0$. Let $R>0$ and consider the semicircular contour $\Gamma_{R}$ consisting of the line segment from $-R$ to $R$ along the real axis, followed by the upper semicircle of radius $R$ centered at the origin, traversed counterclockwise. Thus $\Gamma_{R}$ is a positively oriented simple closed contour.

On the upper semicircle we write $z = Re^{i\theta}$ with $\theta\in[0,\pi]$, so that $\operatorname{Im} z = R\sin\theta \ge 0$. For such $z$ we have
\[
\lvert e^{i\xi z}\rvert
= \bigl|e^{i\xi(x+iy)}\bigr|
= \left|e^{i\xi x}e^{-\xi y}\right|
= e^{-\xi y}
= e^{-\xi\,\operatorname{Im} z}.
\]
When $\xi>0$ and $\operatorname{Im} z\ge 0$, this gives
\[
\lvert e^{i\xi z}\rvert \le 1,
\]
and in fact $\lvert e^{i\xi z}\rvert \le e^{-\xi R\sin\theta}$, which decays away from the real axis. On the semicircular arc we also have $\lvert z^{2}+a^{2}\rvert \ge R^{2}-a^{2}$ for $R$ large. Combining these estimates, we find
\[
\left|\int_{\text{arc}} f(z)\,dz\right|
\le \max_{\text{arc}} \left|\frac{e^{i\xi z}}{z^{2}+a^{2}}\right|\cdot(\text{length of arc})
\le \frac{1}{R^{2}-a^{2}}\cdot \pi R
\to 0
\]
as $R\to\infty$. Thus, in the limit $R\to\infty$, the contribution from the arc vanishes, and we obtain
\[
\int_{-\infty}^{\infty} \frac{e^{i\xi x}}{x^{2}+a^{2}}\,dx
= \lim_{R\to\infty} \int_{-R}^{R} \frac{e^{i\xi x}}{x^{2}+a^{2}}\,dx
= \lim_{R\to\infty} \int_{\Gamma_{R}} f(z)\,dz,
\]
because the difference between the contour integral and the real-axis integral is precisely the arc integral, which tends to zero.

By the residue theorem, for each $R$ sufficiently large so that $ia$ is inside the upper semicircle, we have
\[
\int_{\Gamma_{R}} f(z)\,dz = 2\pi i \sum \operatorname{Res}(f; z_{k}),
\]
where the sum runs over the poles $z_{k}$ inside $\Gamma_{R}$. In the upper half-plane only the pole at $z=ia$ lies inside the contour. Therefore,
\[
\int_{\Gamma_{R}} f(z)\,dz = 2\pi i\,\operatorname{Res}(f; ia).
\]

We compute this residue. Since
\[
f(z) = \frac{e^{i\xi z}}{z^{2}+a^{2}}
= \frac{e^{i\xi z}}{(z-ia)(z+ia)},
\]
the pole at $z=ia$ is simple. One way to compute the residue is to use the general formula for a simple pole of $g(z)/h(z)$, where $h(z_{0})=0$ and $h'(z_{0})\neq0$:
\[
\operatorname{Res}\left(\frac{g(z)}{h(z)},z_{0}\right)
= \frac{g(z_{0})}{h'(z_{0})}.
\]
Here $g(z)=e^{i\xi z}$ and $h(z)=z^{2}+a^{2}$, so $h'(z)=2z$, and at $z=ia$ we have $h'(ia)=2ia$. Thus
\[
\operatorname{Res}(f; ia)
= \frac{e^{i\xi (ia)}}{2ia}
= \frac{e^{-\xi a}}{2ia}.
\]

Putting this into the residue theorem and letting $R\to\infty$, we obtain for $\xi>0$:
\[
I(\xi)
= \int_{-\infty}^{\infty} \frac{e^{i\xi x}}{x^{2}+a^{2}}\,dx
= 2\pi i \cdot \frac{e^{-\xi a}}{2ia}
= \frac{\pi}{a} e^{-\xi a}.
\]

For $\xi<0$ the argument is completely analogous, except that we now close the contour in the \emph{lower} half-plane so that $\lvert e^{i\xi z}\rvert$ decays along the arc. Indeed, if $\xi<0$ and $\operatorname{Im} z\le 0$, then
\[
\lvert e^{i\xi z}\rvert
= e^{-\xi\,\operatorname{Im}z} \le 1,
\]
and the same estimation shows that the arc integral again tends to zero as $R\to\infty$. The contour in the lower half-plane now encloses the pole at $z=-ia$, and residue calculus gives
\[
I(\xi) = 2\pi i\,\operatorname{Res}(f; -ia).
\]
A similar computation yields
\[
\operatorname{Res}(f; -ia)
= \frac{e^{i\xi(-ia)}}{2(-ia)}
= \frac{e^{\xi a}}{-2ia}.
\]
However, since $\xi<0$, we can write $e^{\xi a} = e^{-a\lvert \xi\rvert}$, and also $-2ia = 2i(-a)$. Evaluating the prefactor,
\[
2\pi i \cdot \frac{e^{\xi a}}{-2ia}
= \frac{\pi}{a} e^{\xi a}
= \frac{\pi}{a} e^{-a\lvert\xi\rvert}.
\]
Thus the same formula
\[
I(\xi) = \frac{\pi}{a} e^{-a\lvert\xi\rvert}
\]
holds for $\xi<0$ as well. This is consistent with the evenness of $I(\xi)$ established in part~(a).

Combining the two cases, we conclude that for all real $\xi$,
\[
\boxed{%
I(\xi) = \int_{-\infty}^{\infty} \frac{e^{i\xi x}}{x^{2}+a^{2}}\,dx
= \frac{\pi}{a}\,e^{-a\lvert\xi\rvert}.}
\]

\medskip

\noindent\textbf{(c) Interpretation in terms of poles and decay.}
The original integrand is a rational function of $z$ multiplied by an oscillatory exponential. As a function of the complex variable $z$, it has two simple poles at $z=\pm ia$, located a distance $a$ above and below the real axis. The residue theorem tells us that the value of the real integral is determined entirely by the contribution from the pole in the half-plane where we close the contour, and the exponential factor $e^{-a\lvert\xi\rvert}$ in the final answer reflects the vertical distance of the poles from the real axis.

This illustrates a central theme of the section \emph{“Closed-form Representation for Select Fourier Transforms”}: for rational functions, the analytic structure in the complex plane (location and order of poles) translates directly into qualitative features of the Fourier transform. In this example, simple poles off the real axis produce exponentially decaying Fourier transforms. More generally, higher-order poles or more complicated rational structures lead to different decay rates and combinations of oscillation and damping, all of which can be read off from the complex-analytic data and computed efficiently via residues.
\end{solution}

% ===== Example 4: Piecewise and Even/Odd Functions via Symmetry (inquiry-based) =====
\begin{problem}[Piecewise and Even/Odd Functions via Symmetry]
In many applications, signals are naturally supported only on a finite interval, and their graphs often exhibit simple symmetries. Examples include square pulses modeling on/off switches, localized temperature profiles, and one-sided loading in beams or cables. Direct computation of the Fourier transform of such piecewise-defined functions is always possible, but it can be greatly simplified by recognizing and exploiting evenness, oddness, and decomposition into symmetric parts. In this problem you will see how to turn a seemingly asymmetric ``one-sided'' pulse into a combination of an even pulse and an odd pulse, each of which has a very simple closed-form Fourier transform.

Throughout, use the Fourier transform convention
\[
\widehat{f}(\omega) \coloneqq \int_{-\infty}^{\infty} f(x)\,e^{-i\omega x}\,dx,
\]
for real frequency variable $\omega$.

\medskip

(a) Let $f\in L^{1}(\mathbb{R})$ be real-valued.

\quad(i) Assume $f$ is \emph{even}, that is, $f(-x)=f(x)$ for all $x$. Show that
\[
\widehat{f}(\omega)=2 \int_{0}^{\infty} f(x)\cos(\omega x)\,dx.
\]

\quad(ii) Assume instead that $f$ is \emph{odd}, that is, $f(-x)=-f(x)$ for all $x$. Show that
\[
\widehat{f}(\omega)=-2i\int_{0}^{\infty} f(x)\sin(\omega x)\,dx.
\]

Hint: First write $e^{-i\omega x}=\cos(\omega x)-i\sin(\omega x)$ and separate the Fourier integral into real and imaginary parts. Then use the substitution $x\mapsto -x$ on $(-\infty,0)$ and the even or odd property of $f$.

\medskip

(b) Fix a length $L>0$ and consider the \emph{even} square pulse
\[
p(x) \coloneqq 
\begin{cases}
1, & |x|<L,\\[3pt]
0, & \text{otherwise}.
\end{cases}
\]
Compute $\widehat{p}(\omega)$ by exploiting the evenness of $p$ and your formula from part (a)(i). Express your answer using only elementary functions.

Hint: After reducing the integral to $[0,L]$, you should only need to integrate $\cos(\omega x)$.

\medskip

(c) Now define the \emph{odd} ``signed'' pulse
\[
q(x) \coloneqq 
\begin{cases}
1, & 0<x<L,\\[3pt]
-1, & -L<x<0,\\[3pt]
0, & \text{otherwise}.
\end{cases}
\]

\quad(i) Verify that $q$ is an odd function.

\quad(ii) Use part (a)(ii) to compute $\widehat{q}(\omega)$ explicitly. Simplify your answer as much as possible.

Hint: Again you will reduce the integral to $[0,L]$, but this time you will need to integrate $\sin(\omega x)$.

\medskip

(d) Consider now the one-sided (right-hand) square pulse
\[
h(x) \coloneqq 
\begin{cases}
1, & 0<x<L,\\[3pt]
0, & \text{otherwise}.
\end{cases}
\]
This function $h$ is neither even nor odd.

\quad(i) Verify the identity
\[
h(x)=\tfrac{1}{2}\bigl(p(x)+q(x)\bigr)
\]
for every real $x$. In other words, show that the one-sided pulse can be written as the average of the even pulse $p$ and the odd pulse $q$.

\quad(ii) Use linearity of the Fourier transform and your computations of $\widehat{p}$ and $\widehat{q}$ to obtain a closed-form expression for $\widehat{h}(\omega)$.

\quad(iii) As a consistency check, compute $\widehat{h}(\omega)$ directly from the definition,
\[
\widehat{h}(\omega)=\int_{-\infty}^{\infty} h(x)e^{-i\omega x}\,dx,
\]
and verify that you obtain the same formula.

Hint: For the direct computation, note that $h(x)$ is nonzero only on $(0,L)$.

\medskip

(e) (Extensions and reflections.)

\quad(i) Suppose instead of height $1$ the one-sided pulse has height $A>0$, that is,
\[
h_A(x)\coloneqq
\begin{cases}
A, & 0<x<L,\\[3pt]
0, & \text{otherwise}.
\end{cases}
\]
Without repeating any integrals, determine $\widehat{h_A}(\omega)$ from your work above.

\quad(ii) Decompose $h$ into its even and odd parts,
\[
h_{\text{even}}(x)\coloneqq \tfrac{1}{2}\bigl(h(x)+h(-x)\bigr),\qquad
h_{\text{odd}}(x)\coloneqq \tfrac{1}{2}\bigl(h(x)-h(-x)\bigr).
\]
Express $h_{\text{even}}$ and $h_{\text{odd}}$ in terms of $p$ and $q$, and relate the real and imaginary parts of $\widehat{h}(\omega)$ to the transforms of these even and odd parts.

Hint: Compare your formulas with those in parts (b)--(d). How do $\Re\widehat{h}$ and $\Im\widehat{h}$ reflect the contributions of the even and odd components?
\end{problem}

% ===== Example 4: Piecewise and Even/Odd Functions via Symmetry (full solution) =====
\begin{problem}[Piecewise and Even/Odd Functions via Symmetry]
Let the Fourier transform of an integrable function $f$ be
\[
\widehat{f}(\omega)\coloneqq\int_{-\infty}^{\infty} f(x)\,e^{-i\omega x}\,dx,\qquad \omega\in\mathbb{R}.
\]

(a) Show that if $f$ is even, then
\[
\widehat{f}(\omega)=2\int_{0}^{\infty} f(x)\cos(\omega x)\,dx,
\]
and if $f$ is odd, then
\[
\widehat{f}(\omega)=-2i\int_{0}^{\infty} f(x)\sin(\omega x)\,dx.
\]

(b) Fix $L>0$ and define
\[
p(x)\coloneqq
\begin{cases}
1, & |x|<L,\\
0, & \text{otherwise},
\end{cases}
\qquad
q(x)\coloneqq
\begin{cases}
1, & 0<x<L,\\
-1,& -L<x<0,\\
0, & \text{otherwise},
\end{cases}
\]
and
\[
h(x)\coloneqq
\begin{cases}
1,& 0<x<L,\\
0,& \text{otherwise}.
\end{cases}
\]

(i) Show that $p$ is even and $q$ is odd.

(ii) Compute the Fourier transforms $\widehat{p}(\omega)$ and $\widehat{q}(\omega)$ in closed form.

(iii) Show that $h(x)=\frac{1}{2}\bigl(p(x)+q(x)\bigr)$ for all $x$, and use this to compute $\widehat{h}(\omega)$ in closed form. Verify your answer for $\widehat{h}(\omega)$ by a direct computation from the definition.

Briefly explain how this example illustrates the use of symmetry and decomposition into even/odd parts to obtain closed-form Fourier transforms for piecewise-defined functions.
\end{problem}

\begin{solution}
We begin by establishing how evenness and oddness interact with the Fourier kernel $e^{-i\omega x}$, and then we apply these observations to the specific piecewise functions.

\medskip

\textbf{Part (a): Even and odd functions and sine/cosine transforms.}

Write the complex exponential in terms of sine and cosine:
\[
e^{-i\omega x}=\cos(\omega x)-i\sin(\omega x).
\]
Then the Fourier transform is
\[
\widehat{f}(\omega)
=\int_{-\infty}^{\infty} f(x)\bigl[\cos(\omega x)-i\sin(\omega x)\bigr]\,dx
=\int_{-\infty}^{\infty} f(x)\cos(\omega x)\,dx
-i\int_{-\infty}^{\infty} f(x)\sin(\omega x)\,dx.
\]

\emph{Case 1: $f$ even.} Suppose $f(-x)=f(x)$ for all $x$. Then the product $f(x)\cos(\omega x)$ is even, since cosine is even, and $f(x)\sin(\omega x)$ is odd, since sine is odd. An odd integrable function has integral zero over $\mathbb{R}$, so
\[
\int_{-\infty}^{\infty} f(x)\sin(\omega x)\,dx=0.
\]
For the even part, we use symmetry to reduce to $[0,\infty)$:
\[
\int_{-\infty}^{\infty} f(x)\cos(\omega x)\,dx
=2\int_{0}^{\infty} f(x)\cos(\omega x)\,dx.
\]
Therefore
\[
\widehat{f}(\omega)=2\int_{0}^{\infty} f(x)\cos(\omega x)\,dx,
\]
as claimed.

\emph{Case 2: $f$ odd.} Now suppose $f(-x)=-f(x)$. Then $f(x)\cos(\omega x)$ is odd (odd times even), so its integral over $\mathbb{R}$ vanishes, while $f(x)\sin(\omega x)$ is even (odd times odd). Thus
\[
\int_{-\infty}^{\infty} f(x)\cos(\omega x)\,dx=0,
\]
and
\[
\int_{-\infty}^{\infty} f(x)\sin(\omega x)\,dx
=2\int_{0}^{\infty} f(x)\sin(\omega x)\,dx.
\]
Consequently,
\[
\widehat{f}(\omega)
=-i\cdot 2\int_{0}^{\infty} f(x)\sin(\omega x)\,dx
=-2i\int_{0}^{\infty} f(x)\sin(\omega x)\,dx.
\]

This is precisely the desired simplification: for even functions the Fourier transform is a \emph{cosine transform}, and for odd functions it is a (purely imaginary) \emph{sine transform}.

\medskip

\textbf{Part (b): The even square pulse $p$.}

The function $p$ is defined by
\[
p(x)=
\begin{cases}
1,& |x|<L,\\
0,& \text{otherwise}.
\end{cases}
\]
Its graph is symmetric with respect to the vertical axis, so $p(-x)=p(x)$ for all $x$; hence $p$ is even.

By part (a) we can compute its Fourier transform using only a cosine integral over $[0,\infty)$:
\[
\widehat{p}(\omega)
=2\int_{0}^{\infty} p(x)\cos(\omega x)\,dx.
\]
Since $p(x)=1$ on $(0,L)$ and $p(x)=0$ elsewhere on $(0,\infty)$, this reduces to
\[
\widehat{p}(\omega)
=2\int_{0}^{L} 1\cdot \cos(\omega x)\,dx
=2\left[\frac{\sin(\omega x)}{\omega}\right]_{x=0}^{x=L}
=2\,\frac{\sin(\omega L)-\sin(0)}{\omega}
=2\,\frac{\sin(\omega L)}{\omega}.
\]
Thus we obtain the familiar closed form
\[
\boxed{\;\widehat{p}(\omega)=2\,\dfrac{\sin(\omega L)}{\omega}\;}.
\]
This already illustrates how even symmetry reduces a two-sided integral to a simple one-sided integral involving only cosine.

\medskip

\textbf{Part (c): The odd signed pulse $q$.}

The function $q$ is defined by
\[
q(x)=
\begin{cases}
1, & 0<x<L,\\
-1, & -L<x<0,\\
0, & \text{otherwise}.
\end{cases}
\]

\emph{(i) Oddness of $q$.} For $0<x<L$ we have $q(x)=1$, while $-L<-x<0$ so $q(-x)=-1$. Hence $q(-x)=-q(x)$ on $(0,L)$. For $-L<x<0$ we have $q(x)=-1$, and $0<-x<L$ so $q(-x)=1=-(-1)=-q(x)$. Outside $(-L,L)$, both $q(x)$ and $q(-x)$ are zero. Therefore $q(-x)=-q(x)$ holds for all $x$, and $q$ is odd.

\emph{(ii) Fourier transform of $q$.} Since $q$ is odd, we may apply part (a) with the sine integral:
\[
\widehat{q}(\omega)
=-2i\int_{0}^{\infty} q(x)\sin(\omega x)\,dx.
\]
Now $q(x)=1$ on $(0,L)$ and $q(x)=0$ for $x>L$, so
\[
\widehat{q}(\omega)
=-2i\int_{0}^{L} \sin(\omega x)\,dx.
\]
Integrating,
\[
\int_{0}^{L} \sin(\omega x)\,dx
=\left[-\frac{\cos(\omega x)}{\omega}\right]_{0}^{L}
=-\frac{\cos(\omega L)-\cos(0)}{\omega}
=\frac{1-\cos(\omega L)}{\omega}.
\]
Thus
\[
\widehat{q}(\omega)
=-2i\cdot \frac{1-\cos(\omega L)}{\omega}.
\]
So
\[
\boxed{\;\widehat{q}(\omega)=-2i\,\dfrac{1-\cos(\omega L)}{\omega}\;}.
\]
As expected for an odd real-valued function, the transform is purely imaginary.

\medskip

\textbf{Part (d): The one-sided pulse $h$ and its decomposition.}

The one-sided pulse is
\[
h(x)=
\begin{cases}
1, & 0<x<L,\\
0, & \text{otherwise}.
\end{cases}
\]

\emph{(i) Decomposition as an average of $p$ and $q$.} We claim that
\[
h(x)=\frac{1}{2}\bigl(p(x)+q(x)\bigr)\quad\text{for all }x\in\mathbb{R}.
\]
We verify this by checking the three regions.

\smallskip

\emph{Region 1: $0<x<L$.} Here
\[
p(x)=1,\qquad q(x)=1,
\]
so
\[
\frac{1}{2}\bigl(p(x)+q(x)\bigr)=\tfrac{1}{2}(1+1)=1=h(x).
\]

\emph{Region 2: $-L<x<0$.} Here
\[
p(x)=1,\qquad q(x)=-1,
\]
so
\[
\frac{1}{2}\bigl(p(x)+q(x)\bigr)=\tfrac{1}{2}(1-1)=0=h(x).
\]

\emph{Region 3: $|x|\geq L$.} Here $p(x)=0$ by definition, and also $q(x)=0$, so
\[
\frac{1}{2}(p(x)+q(x))=0=h(x).
\]
Thus the identity $h=\tfrac{1}{2}(p+q)$ holds everywhere.

\smallskip

\emph{(ii) Fourier transform via linearity and symmetry.} By linearity of the Fourier transform,
\[
\widehat{h}(\omega)
=\frac{1}{2}\bigl(\widehat{p}(\omega)+\widehat{q}(\omega)\bigr).
\]
We substitute the expressions just obtained:
\[
\widehat{p}(\omega)=2\,\frac{\sin(\omega L)}{\omega},\qquad
\widehat{q}(\omega)=-2i\,\frac{1-\cos(\omega L)}{\omega}.
\]
Hence
\[
\widehat{h}(\omega)
=\frac{1}{2}\left[2\,\frac{\sin(\omega L)}{\omega}-2i\,\frac{1-\cos(\omega L)}{\omega}\right]
=\frac{1}{\omega}\left[\sin(\omega L)-i\bigl(1-\cos(\omega L)\bigr)\right].
\]
So we obtain the compact closed form
\[
\boxed{\;\widehat{h}(\omega)=\dfrac{\sin(\omega L)-i\bigl(1-\cos(\omega L)\bigr)}{\omega}\;}.
\]
We can also see explicitly that the real part comes from the even contribution (via $p$) and the imaginary part from the odd contribution (via $q$).

\smallskip

\emph{(iii) Direct computation from the definition.} As a check, we compute $\widehat{h}$ again, now directly from the integral. Since $h(x)=1$ on $(0,L)$ and zero elsewhere,
\[
\widehat{h}(\omega)
=\int_{-\infty}^{\infty} h(x)e^{-i\omega x}\,dx
=\int_{0}^{L} e^{-i\omega x}\,dx.
\]
Integrating,
\[
\int_{0}^{L} e^{-i\omega x}\,dx
=\left[\frac{e^{-i\omega x}}{-i\omega}\right]_{0}^{L}
=\frac{1-e^{-i\omega L}}{i\omega}.
\]
We rewrite the numerator using Euler's formula:
\[
e^{-i\omega L}=\cos(\omega L)-i\sin(\omega L),
\]
so
\[
1-e^{-i\omega L}
=1-\cos(\omega L)+i\sin(\omega L).
\]
Thus
\[
\widehat{h}(\omega)
=\frac{1-\cos(\omega L)+i\sin(\omega L)}{i\omega}.
\]
To bring this into the same form as before, multiply numerator and denominator by $-i$:
\[
\widehat{h}(\omega)
=\frac{(1-\cos(\omega L))(-i)+i\sin(\omega L)(-i)}{\omega}
=\frac{-i(1-\cos(\omega L))+\sin(\omega L)}{\omega},
\]
since $i\cdot(-i)=1$. Rearranging,
\[
\widehat{h}(\omega)=\frac{\sin(\omega L)-i(1-\cos(\omega L))}{\omega},
\]
which agrees exactly with the expression obtained from symmetry and decomposition. This confirms the correctness of our earlier computation.

\medskip

\textbf{Interpretation and connection to the chapter theme.}

This example shows several central ideas of the section on closed-form Fourier transforms:

\begin{itemize}
  \item For even functions, the Fourier transform reduces to a cosine transform; for odd functions, it reduces to a sine transform. This simplifies integrals and often leads to closed forms with basic trigonometric functions.
  \item A general function can be decomposed into its even and odd parts. In our concrete case, the seemingly asymmetric one-sided pulse $h$ was written as the average of an even pulse $p$ and an odd signed pulse $q$. The Fourier transform then decomposes correspondingly, with the real part tied to the even component and the imaginary part tied to the odd component.
  \item By exploiting these symmetry properties and the linearity of the transform, we were able to avoid repeated direct integrations and obtain explicit formulas efficiently. This approach is typical when handling piecewise-defined signals in applied problems: one rewrites the function in terms of simpler symmetric building blocks and uses known or easily computed transforms for those blocks.
\end{itemize}

In summary, symmetry and even/odd decomposition are powerful tools for deriving closed-form Fourier transforms of piecewise-defined functions, as illustrated here for square pulses and their one-sided variants.
\end{solution}

% ===== Example 5: Fourier Transform and Green’s Function for the One-Dimensional Heat Equation (inquiry-based) =====
\begin{problem}[Fourier Transform and Green’s Function for the One-Dimensional Heat Equation]
A thin, infinitely long rod occupies the entire real line, and heat diffuses along the rod according to the one-dimensional heat equation. We would like to understand how a very localized burst of heat, concentrated initially at a single point, spreads out over time. This “point source” solution is called the \emph{fundamental solution} or \emph{Green’s function} for the heat equation. In this problem, you will discover that taking the Fourier transform in space reduces the partial differential equation to an ordinary differential equation, and that inverting the transform leads to an explicit Gaussian formula.

Consider the heat equation on the whole line
\[
u_t(x,t) \;=\; \kappa\,u_{xx}(x,t), \qquad x\in\mathbb{R},\ t>0,\ \kappa>0,
\]
with the impulsive initial condition
\[
u(x,0) \;=\; \delta(x),
\]
where $\delta$ is the Dirac delta distribution. We interpret $u(x,t)$ as the temperature at position $x$ and time $t$, and $\kappa$ as the thermal diffusivity.

Throughout, use the Fourier transform in $x$ defined by
\[
\widehat{f}(k) \;=\; \int_{-\infty}^{\infty} f(x)\,e^{-ikx}\,dx,
\qquad
f(x) \;=\; \frac{1}{2\pi}\int_{-\infty}^{\infty} \widehat{f}(k)\,e^{ikx}\,dk.
\]

\smallskip

\textbf{(a) Transforming the PDE in space.}
\begin{enumerate}
\item[(a1)] Apply the Fourier transform in $x$ to both sides of the heat equation. Carefully justify how the derivatives in $x$ transform. Write down the transformed equation satisfied by $\widehat{u}(k,t)$.

\emph{Hint:} Recall that, for sufficiently nice $f$,
\[
\mathcal{F}[f'(x)](k) = ik\,\widehat{f}(k),
\qquad
\mathcal{F}[f''(x)](k) = -k^2\,\widehat{f}(k).
\]

\item[(a2)] Use the initial condition $u(x,0) = \delta(x)$ to determine $\widehat{u}(k,0)$. What is the Fourier transform of the Dirac delta?

\emph{Hint:} By definition, for any test function $\varphi$,
\[
\int_{-\infty}^\infty \delta(x)\,\varphi(x)\,dx = \varphi(0).
\]
To find $\widehat{\delta}(k)$, think of $\varphi(x) = e^{-ikx}$.
\end{enumerate}

\smallskip

\textbf{(b) Solving the transformed equation in time.}

\begin{enumerate}
\item[(b1)] Show that the transformed function $\widehat{u}(k,t)$ satisfies a first-order linear ordinary differential equation in $t$. Write this ODE explicitly.

\item[(b2)] Solve this ODE for $\widehat{u}(k,t)$ using the initial condition you found in part (a2), and express $\widehat{u}(k,t)$ in closed form.

\emph{Hint:} The ODE should have the form $v'(t) = -\alpha v(t)$ with constant $\alpha$ depending on $k$. Recall that the solution to $v'(t) = -\alpha v(t)$ with $v(0) = v_0$ is $v(t) = v_0 e^{-\alpha t}$. 
\end{enumerate}

\smallskip

\textbf{(c) Inverting the Fourier transform: appearance of the Gaussian.}

From part (b), you should have found that
\[
\widehat{u}(k,t) = e^{-\kappa k^2 t}.
\]
Now we want to invert the Fourier transform to obtain $u(x,t)$ in physical space:
\[
u(x,t) = \frac{1}{2\pi}\int_{-\infty}^{\infty} e^{-\kappa k^2 t}\,e^{ikx}\,dk.
\]

\begin{enumerate}
\item[(c1)] Show, using completion of the square, that for $a>0$ and $x\in\mathbb{R}$,
\[
\int_{-\infty}^{\infty} e^{-a k^2}\,e^{ikx}\,dk
\;=\;
\sqrt{\frac{\pi}{a}}\;e^{-x^2/(4a)}.
\]

\emph{Hint:} First recall the standard Gaussian integral
\[
\int_{-\infty}^{\infty} e^{-a k^2}\,dk = \sqrt{\frac{\pi}{a}}.
\]
Then write $-a k^2 + ikx$ as $-a\left(k - \frac{ix}{2a}\right)^2 - \frac{x^2}{4a}$, and shift the contour of integration in the complex plane, or argue heuristically by changing variables.

\item[(c2)] Use the formula in (c1) with $a = \kappa t$ to evaluate the inverse Fourier transform
\[
u(x,t) = \frac{1}{2\pi}\int_{-\infty}^{\infty} e^{-\kappa k^2 t}\,e^{ikx}\,dk,
\]
and hence obtain an explicit expression for $u(x,t)$.

\emph{Hint:} Be careful with the factor $\dfrac{1}{2\pi}$ in the inversion formula.
\end{enumerate}

\smallskip

\textbf{(d) Identifying the Green’s function and interpreting the result.}

\begin{enumerate}
\item[(d1)] Denote the solution you have found by $G(x,t)$, that is,
\[
G(x,t) := u(x,t) \quad \text{when} \quad u_t = \kappa u_{xx},\ u(x,0)=\delta(x).
\]
Write $G(x,t)$ explicitly and check that, for each fixed $t>0$, the function $x\mapsto G(x,t)$ is a Gaussian probability density (up to the parameter $\kappa$). That is, show that $\displaystyle \int_{-\infty}^\infty G(x,t)\,dx = 1$ and identify its variance.

\item[(d2)] Explain briefly why $G(x,t)$ is called the \emph{Green’s function} (or \emph{fundamental solution}) for the heat equation on the real line. In particular, argue (formally, if you like) that for a general initial condition $u(x,0) = f(x)$, the solution can be written as the convolution
\[
u(x,t) = (G(\cdot,t) * f)(x) = \int_{-\infty}^\infty G(x-y,t)\,f(y)\,dy.
\]

\emph{Hint:} Think about how the Fourier transform turns convolution in $x$ into multiplication in $k$, and use the fact that you already know the solution for the initial delta function.
\end{enumerate}

\smallskip

\textbf{(e) Extensions and “what if” questions.}

\begin{enumerate}
\item[(e1)] What happens if the diffusivity $\kappa$ is changed? Using your explicit formula for $G(x,t)$, describe qualitatively how increasing $\kappa$ affects the spread of heat over time. How do the height and width of the Gaussian profile depend on $\kappa$?

\item[(e2)] Suppose now that the initial temperature distribution is not a delta function but a general function $f\in L^1(\mathbb{R})$. Using the convolution representation in (d2), write down the explicit integral formula for $u(x,t)$ in terms of $f$. How does this formula relate to the idea that the delta function initial condition represents a point source whose effect is “smeared out” and then superposed for each point of the initial data?

\item[(e3) Optional.] The expression
\[
G(x,t) = \frac{1}{\sqrt{4\pi\kappa t}}\exp\!\left(-\frac{x^2}{4\kappa t}\right)
\]
is closely related to the classical Fourier transform pair for Gaussians. Describe explicitly the transform pair you have implicitly used (that is, identify $f$ and $\widehat{f}$), and explain how the heat equation derivation gives one way to \emph{prove} this transform pair.
\end{enumerate}

\end{problem}

% ===== Example 5: Fourier Transform and Green’s Function for the One-Dimensional Heat Equation (full solution) =====
\begin{problem}[Fourier Transform and Green’s Function for the One-Dimensional Heat Equation]
Consider the heat equation on the real line
\[
u_t(x,t) = \kappa\,u_{xx}(x,t), \qquad x\in\mathbb{R},\ t>0,\ \kappa>0,
\]
with initial condition $u(x,0) = \delta(x)$. Using the Fourier transform in $x$,
\[
\widehat{f}(k) = \int_{-\infty}^\infty f(x)\,e^{-ikx}\,dx,
\qquad
f(x) = \frac{1}{2\pi}\int_{-\infty}^\infty \widehat{f}(k)\,e^{ikx}\,dk,
\]
perform the following tasks:
\begin{enumerate}
\item[(i)] Derive and solve the ordinary differential equation satisfied by $\widehat{u}(k,t)$.
\item[(ii)] Invert the Fourier transform to obtain an explicit formula for $u(x,t)$, and show that
\[
u(x,t) = \frac{1}{\sqrt{4\pi\kappa t}}\exp\!\left(-\frac{x^2}{4\kappa t}\right).
\]
\item[(iii)] Interpret this solution as the Green’s function (fundamental solution) for the one-dimensional heat equation, and briefly explain how it leads to the convolution formula for the solution with general initial data.
\end{enumerate}
\end{problem}

\begin{solution}
We solve the initial value problem for the one-dimensional heat equation with impulsive initial data by taking the Fourier transform in space. This method is a prototypical example of how closed-form Fourier transform pairs, in particular the Gaussian transform pair, arise naturally from partial differential equations.

\medskip

\textbf{Step 1: Fourier transform of the PDE and initial condition.}

We define
\[
\widehat{u}(k,t) = \int_{-\infty}^\infty u(x,t)\,e^{-ikx}\,dx.
\]
We assume that for each fixed $t>0$, the function $u(\cdot,t)$ is sufficiently nice (for example, rapidly decaying) so that we may interchange derivatives and integrals in what follows.

Taking the Fourier transform in $x$ of both sides of the heat equation
\[
u_t(x,t) = \kappa\,u_{xx}(x,t)
\]
gives
\[
\widehat{u_t}(k,t) = \kappa\,\widehat{u_{xx}}(k,t).
\]
On the left-hand side, differentiation with respect to $t$ commutes with the integral:
\[
\widehat{u_t}(k,t) = \int_{-\infty}^\infty u_t(x,t)\,e^{-ikx}\,dx
= \frac{\partial}{\partial t}\int_{-\infty}^\infty u(x,t)\,e^{-ikx}\,dx
= \frac{\partial}{\partial t}\widehat{u}(k,t).
\]
On the right-hand side, we use the standard Fourier transform rule for second derivatives. For a sufficiently smooth, decaying function $f$,
\[
\mathcal{F}[f''(x)](k) = -k^2\,\widehat{f}(k).
\]
Applying this to $u(\cdot,t)$ with $t$ fixed, we obtain
\[
\widehat{u_{xx}}(k,t) = -k^2\,\widehat{u}(k,t).
\]
Therefore the transformed equation is
\[
\frac{\partial}{\partial t}\widehat{u}(k,t) = \kappa\bigl(-k^2\bigr)\widehat{u}(k,t),
\]
or more simply,
\[
\frac{\partial}{\partial t}\widehat{u}(k,t) = -\kappa k^2\,\widehat{u}(k,t).
\]
This is an ordinary differential equation in $t$, with $k$ appearing as a parameter.

Next, we transform the initial condition $u(x,0) = \delta(x)$. By definition of the Fourier transform,
\[
\widehat{u}(k,0) = \int_{-\infty}^\infty u(x,0)\,e^{-ikx}\,dx
= \int_{-\infty}^\infty \delta(x)\,e^{-ikx}\,dx.
\]
The defining property of the Dirac delta is that, for any test function $\varphi$,
\[
\int_{-\infty}^\infty \delta(x)\,\varphi(x)\,dx = \varphi(0).
\]
Here, $\varphi(x) = e^{-ikx}$, so
\[
\widehat{u}(k,0) = e^{-ik\cdot 0} = 1.
\]
Thus the initial condition in Fourier space is
\[
\widehat{u}(k,0) = 1 \quad \text{for all } k\in\mathbb{R}.
\]

\medskip

\textbf{Step 2: Solving the transformed ODE.}

For each fixed $k$, the function $t\mapsto \widehat{u}(k,t)$ satisfies the linear ODE
\[
\frac{d}{dt}\widehat{u}(k,t) = -\kappa k^2\,\widehat{u}(k,t),
\qquad
\widehat{u}(k,0) = 1.
\]
This is a separable first-order ODE with constant coefficient $-\kappa k^2$. Its solution is
\[
\widehat{u}(k,t) = e^{-\kappa k^2 t}.
\]
In other words, in the Fourier domain the heat equation becomes a simple exponential decay in time, with decay rate proportional to $k^2$. High-frequency modes (large $|k|$) decay more rapidly, which matches the physical intuition that diffusion smooths out fine-scale variations.

\medskip

\textbf{Step 3: Inverting the Fourier transform.}

We now invert the Fourier transform to recover $u(x,t)$:
\[
u(x,t) = \frac{1}{2\pi}\int_{-\infty}^\infty \widehat{u}(k,t)\,e^{ikx}\,dk
= \frac{1}{2\pi}\int_{-\infty}^\infty e^{-\kappa k^2 t}\,e^{ikx}\,dk.
\]
This is precisely the inverse Fourier transform of a Gaussian in $k$. To evaluate the integral, we use a standard Gaussian integral with a linear term in the exponent. Let $a>0$ and consider
\[
I(x) := \int_{-\infty}^\infty e^{-a k^2}\,e^{ikx}\,dk
= \int_{-\infty}^\infty \exp(-a k^2 + ikx)\,dk.
\]
We complete the square in the exponent. We have
\[
-ak^2 + ikx
= -a\left(k^2 - \frac{i x}{a}k\right)
= -a\left[\left(k - \frac{i x}{2a}\right)^2 + \frac{x^2}{4a^2}\right]
= -a\left(k - \frac{i x}{2a}\right)^2 - \frac{x^2}{4a}.
\]
Thus
\[
I(x) = e^{-x^2/(4a)}\int_{-\infty}^\infty
\exp\!\left(-a\left(k - \frac{i x}{2a}\right)^2\right)\,dk.
\]
The remaining integral is, up to a complex shift of the integration variable, the standard Gaussian integral. Under suitable analyticity and decay assumptions (which are satisfied here), shifting by a constant in the complex plane does not change its value. Therefore
\[
\int_{-\infty}^\infty \exp\!\left(-a\left(k - \frac{i x}{2a}\right)^2\right)\,dk
= \int_{-\infty}^\infty e^{-a k^2}\,dk
= \sqrt{\frac{\pi}{a}}.
\]
Consequently,
\[
I(x) = \sqrt{\frac{\pi}{a}}\;e^{-x^2/(4a)}.
\]

We now set $a = \kappa t > 0$ and obtain
\[
\int_{-\infty}^\infty e^{-\kappa k^2 t}\,e^{ikx}\,dk
= \sqrt{\frac{\pi}{\kappa t}}\;\exp\!\left(-\frac{x^2}{4\kappa t}\right).
\]
Inserting this into the inverse Fourier transform formula yields
\[
u(x,t) = \frac{1}{2\pi}\sqrt{\frac{\pi}{\kappa t}}\;\exp\!\left(-\frac{x^2}{4\kappa t}\right)
= \frac{1}{\sqrt{4\pi\kappa t}}\;\exp\!\left(-\frac{x^2}{4\kappa t}\right).
\]
This is the desired explicit solution.

\medskip

\textbf{Step 4: Interpreting the solution as a Green’s function.}

We have found that the solution of the heat equation with impulsive initial data is
\[
G(x,t) := u(x,t) = \frac{1}{\sqrt{4\pi\kappa t}}\exp\!\left(-\frac{x^2}{4\kappa t}\right),
\qquad t>0.
\]
First, we check that for each fixed $t>0$, the function $x\mapsto G(x,t)$ is normalized to one:
\[
\int_{-\infty}^\infty G(x,t)\,dx
= \int_{-\infty}^\infty \frac{1}{\sqrt{4\pi\kappa t}}
\exp\!\left(-\frac{x^2}{4\kappa t}\right)\,dx.
\]
Make the change of variables
\[
y = \frac{x}{\sqrt{4\kappa t}} \quad\Longrightarrow\quad x = y\sqrt{4\kappa t},\quad dx = \sqrt{4\kappa t}\,dy.
\]
Then
\[
\int_{-\infty}^\infty G(x,t)\,dx
= \int_{-\infty}^\infty \frac{1}{\sqrt{4\pi\kappa t}}e^{-y^2}\,\sqrt{4\kappa t}\,dy
= \frac{1}{\sqrt{\pi}}\int_{-\infty}^\infty e^{-y^2}\,dy
= 1,
\]
using the classical result $\int_{-\infty}^\infty e^{-y^2}\,dy = \sqrt{\pi}$. Thus $G(\cdot,t)$ is a probability density function on $\mathbb{R}$.

Moreover, $G(x,t)$ is a Gaussian with mean zero and variance $2\kappa t$. Indeed, we can write
\[
G(x,t) = \frac{1}{\sqrt{4\pi\kappa t}}\exp\!\left(-\frac{x^2}{4\kappa t}\right)
= \frac{1}{\sqrt{2\pi(2\kappa t)}}\exp\!\left(-\frac{x^2}{2(2\kappa t)}\right),
\]
which is the standard normal density with variance $\sigma^2 = 2\kappa t$. As $t$ increases, the Gaussian profile becomes wider and lower, reflecting the diffusion of heat away from the origin.

This function $G(x,t)$ is called the \emph{Green’s function} or \emph{fundamental solution} for the one-dimensional heat equation on the real line. It represents the temperature distribution at time $t$ resulting from a unit point source of heat at the origin at time $0$.

Finally, we explain how this solution leads to the convolution formula for general initial data. Let $f\in L^1(\mathbb{R})$ be a given initial temperature distribution, and consider the initial value problem
\[
u_t = \kappa u_{xx},\qquad u(x,0) = f(x).
\]
Taking the Fourier transform in $x$ as before, we find that $\widehat{u}(k,t)$ satisfies the same ODE
\[
\frac{\partial}{\partial t}\widehat{u}(k,t) = -\kappa k^2\,\widehat{u}(k,t),
\]
but now with initial condition $\widehat{u}(k,0) = \widehat{f}(k)$. Therefore
\[
\widehat{u}(k,t) = e^{-\kappa k^2 t}\,\widehat{f}(k).
\]
Comparing this with the expression for the fundamental solution, we have
\[
\widehat{G}(k,t) = e^{-\kappa k^2 t},
\]
so that
\[
\widehat{u}(k,t) = \widehat{G}(k,t)\,\widehat{f}(k).
\]
In physical space, multiplication of Fourier transforms corresponds to convolution:
\[
u(x,t) = (G(\cdot,t) * f)(x)
= \int_{-\infty}^\infty G(x-y,t)\,f(y)\,dy.
\]
This formula expresses the solution with general initial data as a superposition of translated point-source solutions, weighted by the initial distribution $f$.

\medskip

\textbf{Conceptual summary.}

This example illustrates a central theme of Fourier analysis in applied mathematics: partial differential equations with constant coefficients become algebraic (or simple ODE) problems in the transform domain. The one-dimensional heat equation, when transformed in space, reduces to exponential decay of each Fourier mode. Inverting the transform leads to a closed-form expression for the Green’s function, which turns out to be a Gaussian. Thus the classical Fourier transform pair “Gaussian $\leftrightarrow$ Gaussian” is not just an isolated formula, but arises naturally as the fundamental solution of a physically important PDE. This connection between explicit transform pairs and PDE solutions underlies much of the theory developed in the section on closed-form representations for select Fourier transforms.
\end{solution}

\section{Fourier Series: Introduction}
% --- Narrative plan (auto-generated) ---
% In this section we begin the study of Fourier series, which represent periodic functions as infinite sums of sines and cosines. This idea allows us to replace a complicated function by a combination of very simple building blocks, each with a clear geometric and physical interpretation. We will learn how to compute these expansions, how to interpret their coefficients, and how to reason carefully about convergence.
%
% Fourier series are foundational in applied mathematics because many partial differential equations, such as the heat equation, the wave equation, and Laplace’s equation, can be solved by expanding solutions into trigonometric series. In dynamical systems and signal processing, Fourier series let us decompose motion or data into constituent frequencies, clarifying oscillatory behavior and resonances. The tools we develop here connect directly to orthogonality in linear algebra, to complex exponentials and residues in complex analysis, and to the more general Fourier transform that appears in continuum models and infinite-domain PDEs.

% ===== Example 1: Vibrating String with Fixed Endpoints (inquiry-based) =====
\begin{problem}[Vibrating String with Fixed Endpoints]
A taut string of length $L$ is stretched along the $x$-axis from $x=0$ to $x=L$ and held fixed at both ends. When displaced from equilibrium and released, the string vibrates, and its transverse displacement $u(x,t)$ (vertical position of the string) satisfies a partial differential equation. Experimentally, one observes that the motion can be described as a superposition of “standing waves” with different frequencies. Mathematically, this corresponds to expanding the initial shape of the string in terms of a Fourier sine series.

In this problem you will derive this description step by step, starting from the wave equation and ending with a Fourier series representation of an arbitrary initial displacement that vanishes at the endpoints.

\smallskip

(a) Let $u(x,t)$ denote the vertical displacement of the string at position $x \in [0,L]$ and time $t \ge 0$. The string has constant wave speed $c>0$.

\quad(i) Write down the one-dimensional wave equation that $u$ is assumed to satisfy. Briefly explain in words what each term represents.

\quad(ii) The string is fixed at both endpoints. Express this as boundary conditions on $u(x,t)$. Explain physically why these conditions are appropriate.

\quad(iii) Suppose the string is released from rest from an initial shape $f(x)$, which vanishes at the endpoints. Formulate the corresponding initial conditions for $u(x,t)$ in terms of $f$.

\smallskip

(b) To look for standing wave solutions, we try a separated form $u(x,t) = X(x)T(t)$.

\quad(i) Substitute $u(x,t)=X(x)T(t)$ into the wave equation you wrote in part (a), and rearrange your expression so that all terms involving $x$ are on one side and all terms involving $t$ are on the other side.

\quad(ii) Argue that both sides must be equal to the same constant, say $-\lambda$, and write down the resulting pair of ordinary differential equations for $X$ and $T$.

\quad(iii) Any physical mode of vibration should be bounded and nontrivial on $0<x<L$. What restrictions on the sign of $\lambda$ are suggested by this requirement and by the endpoint conditions $X(0)=X(L)=0$? State clearly which sign of $\lambda$ leads to acceptable solutions.

\emph{Hint:} Think about the general forms of solutions for $X''+\lambda X=0$, $X''=0$, and $X''-\lambda X=0$, and how they behave on a finite interval with $X(0)=X(L)=0$.

\smallskip

(c) Now focus on the spatial equation
\[
X''(x) + \lambda X(x) = 0, \qquad X(0)=0, \quad X(L)=0.
\]

\quad(i) Assuming $\lambda>0$, write the general solution of $X''+\lambda X=0$ and impose $X(0)=0$ to simplify it.

\quad(ii) Impose the second boundary condition $X(L)=0$ and show that this forces $\lambda$ to take on only certain discrete values $\lambda_n$. Find a formula for $\lambda_n$ in terms of $n$ and $L$, and write the corresponding eigenfunctions $X_n(x)$.

\quad(iii) Explain why the case $\lambda \le 0$ does not produce additional nontrivial solutions satisfying both endpoint conditions.

\emph{Hint:} For (ii), you should find a condition of the form $\sin(\sqrt{\lambda}L)=0$. For (iii), carefully examine the linear and exponential solutions that arise when $\lambda=0$ or $\lambda<0$.

\smallskip

(d) For each integer $n \ge 1$, you now have a pair of equations:
\[
X_n'' + \lambda_n X_n = 0, \qquad T_n'' + c^2\lambda_n T_n = 0,
\]
with $X_n(0)=X_n(L)=0$.

\quad(i) Solve the time equation for $T_n(t)$ when $\lambda_n = \left(\frac{n\pi}{L}\right)^2$ and write the separated solutions $u_n(x,t) = X_n(x)T_n(t)$. Express the frequency in terms of $n$, $c$, and $L$.

\quad(ii) Using the principle of superposition (linearity of the wave equation), write the most general solution $u(x,t)$ as a sum over $n$ of these modes.

\quad(iii) Impose the initial conditions $u(x,0)=f(x)$ and $u_t(x,0)=0$ to determine which time-dependent terms appear, and derive a formula for the coefficients in the expansion of $f(x)$.

\emph{Hint:} You should arrive at an expression of the form
\[
f(x) = \sum_{n=1}^{\infty} A_n \sin\!\left(\frac{n\pi x}{L}\right),
\]
and then use orthogonality of the sine functions on $[0,L]$ to solve for $A_n$.

\smallskip

(e) Extensions and variations.

\quad(i) Suppose now that the string is released with zero initial displacement but a nonzero initial velocity $g(x)$, that is,
\[
u(x,0)=0, \qquad u_t(x,0)=g(x).
\]
How would your formulas in part (d) change? Describe (in words or formulas) how to obtain the new coefficients.

\quad(ii) Imagine instead that the left end of the string is fixed, but the right end is free to move vertically without any force (no tension) at $x=L$. In this idealized model, the boundary conditions become
\[
u(0,t)=0, \qquad u_x(L,t)=0.
\]
Based on your work above, what kind of trigonometric functions do you expect to appear in the $x$-dependence of the modes (sines, cosines, or both)? Give a brief justification based on the new boundary conditions.

\end{problem}

% ===== Example 1: Vibrating String with Fixed Endpoints (full solution) =====
\begin{problem}[Vibrating String with Fixed Endpoints]
Consider the one-dimensional wave equation
\[
u_{tt}(x,t) = c^2 u_{xx}(x,t), \qquad 0<x<L,\ t>0,
\]
modeling the transverse displacement $u(x,t)$ of a taut string of length $L$ with wave speed $c>0$. The string is fixed at both ends, so
\[
u(0,t)=0, \qquad u(L,t)=0 \quad \text{for all } t\ge 0.
\]
It is released from rest from an initial displacement $f(x)$ that satisfies $f(0)=f(L)=0$:
\[
u(x,0)=f(x), \qquad u_t(x,0)=0.
\]

Use separation of variables and Fourier series to:
\begin{enumerate}
\item Find all separated solutions satisfying the boundary conditions, and show that they lead to eigenfunctions $\sin\!\left(\frac{n\pi x}{L}\right)$ with frequencies depending on $n$, $c$, and $L$.
\item Use superposition to write the general solution $u(x,t)$ satisfying the given initial and boundary conditions, and express the coefficients in terms of $f(x)$.
\end{enumerate}
\end{problem}

\begin{solution}
We are asked to solve the wave equation on a finite interval with fixed endpoints and a given initial shape. The key ideas are separation of variables, the resulting eigenvalue problem for the spatial part, and expansion of the initial displacement in a Fourier sine series. This example illustrates how Fourier series naturally arise when solving linear partial differential equations with homogeneous boundary conditions.

\medskip

\textbf{1. Separation of variables and the separated equations.}

We look for separated solutions of the form
\[
u(x,t) = X(x) T(t),
\]
with $X$ depending only on $x$ and $T$ only on $t$. Substituting into the wave equation,
\[
u_{tt} = X(x) T''(t), \qquad u_{xx} = X''(x) T(t),
\]
so the wave equation $u_{tt} = c^2 u_{xx}$ becomes
\[
X(x) T''(t) = c^2 X''(x) T(t).
\]
Assuming $X$ and $T$ are not identically zero, we can divide by $c^2 X(x) T(t)$ to obtain
\[
\frac{T''(t)}{c^2 T(t)} = \frac{X''(x)}{X(x)}.
\]
The left-hand side depends only on $t$ and the right-hand side only on $x$, so both must be equal to a constant. It is convenient to write this constant as $-\lambda$, giving the pair of ordinary differential equations
\[
\frac{T''(t)}{c^2 T(t)} = \frac{X''(x)}{X(x)} = -\lambda,
\]
hence
\[
X''(x) + \lambda X(x) = 0, \qquad T''(t) + c^2 \lambda T(t) = 0.
\]

The boundary conditions $u(0,t)=0$ and $u(L,t)=0$ translate into
\[
X(0)T(t) = 0, \qquad X(L)T(t)=0 \quad \text{for all } t.
\]
For nontrivial $T(t)$, this forces
\[
X(0)=0, \qquad X(L)=0.
\]
Thus the spatial problem is
\[
X''(x) + \lambda X(x) = 0, \qquad X(0)=0,\quad X(L)=0.
\]

\medskip

\textbf{2. Solving the spatial eigenvalue problem.}

We must find all values of $\lambda$ for which there are nonzero solutions $X$ satisfying both endpoint conditions.

\smallskip

\emph{Case 1: $\lambda = 0$.} Then $X''(x)=0$, so
\[
X(x) = A + Bx.
\]
The condition $X(0)=0$ gives $A=0$, so $X(x)=Bx$. Then $X(L)=0$ implies $B L =0$, hence $B=0$. Therefore $X\equiv 0$, so there are no nontrivial solutions when $\lambda=0$.

\smallskip

\emph{Case 2: $\lambda <0$.} Write $\lambda = -\mu^2$ with $\mu>0$. Then
\[
X''(x) - \mu^2 X(x) = 0,
\]
whose general solution is
\[
X(x) = A e^{\mu x} + B e^{-\mu x}.
\]
The condition $X(0)=0$ gives $A+B=0$, so $B=-A$ and
\[
X(x) = A\left(e^{\mu x} - e^{-\mu x}\right) = 2A \sinh(\mu x).
\]
Then $X(L)=0$ gives $\sinh(\mu L)=0$. Since $\mu L>0$, this is impossible, so again only the trivial solution exists. Thus no nontrivial solutions arise for $\lambda<0$.

\smallskip

\emph{Case 3: $\lambda >0$.} Write $\lambda = \omega^2$ with $\omega>0$. Then
\[
X''(x) + \omega^2 X(x) = 0,
\]
whose general solution is
\[
X(x) = A\cos(\omega x) + B\sin(\omega x).
\]
The condition $X(0)=0$ gives
\[
X(0) = A\cos 0 + B\sin 0 = A = 0,
\]
so $A=0$ and
\[
X(x) = B\sin(\omega x).
\]
The second condition $X(L)=0$ then requires
\[
B \sin(\omega L) = 0.
\]
For a nontrivial solution we need $B\neq 0$, so we must have
\[
\sin(\omega L) = 0.
\]
This holds precisely when $\omega L = n\pi$ for some integer $n$. Since $\omega>0$, we take $n=1,2,3,\dots$. Thus
\[
\omega_n = \frac{n\pi}{L}, \qquad \lambda_n = \omega_n^2 = \left(\frac{n\pi}{L}\right)^2,
\]
and the corresponding eigenfunctions are
\[
X_n(x) = \sin\!\left( \frac{n\pi x}{L} \right), \qquad n=1,2,3,\dots,
\]
up to multiplication by a constant.

Therefore, the only values of $\lambda$ that yield nontrivial solutions satisfying the boundary conditions are
\[
\lambda_n = \left(\frac{n\pi}{L}\right)^2, \quad n\in\mathbb{N},
\]
with eigenfunctions $X_n(x)=\sin\left(\frac{n\pi x}{L}\right)$.

\medskip

\textbf{3. Solving the time equation and constructing separated solutions.}

For each $n\ge 1$, the time equation is
\[
T_n''(t) + c^2 \lambda_n T_n(t) = 0
\quad\text{with}\quad
\lambda_n = \left(\frac{n\pi}{L}\right)^2.
\]
Thus
\[
T_n''(t) + \left(\frac{n\pi c}{L}\right)^2 T_n(t) = 0.
\]
This is a harmonic oscillator equation with general solution
\[
T_n(t) = A_n \cos\!\left( \frac{n\pi c t}{L} \right)
        + B_n \sin\!\left( \frac{n\pi c t}{L} \right),
\]
where $A_n$ and $B_n$ are constants.

Combining $X_n$ and $T_n$, a separated solution for each $n$ is
\[
u_n(x,t) = \sin\!\left(\frac{n\pi x}{L}\right)
\bigg[
A_n \cos\!\left( \frac{n\pi c t}{L} \right)
+ B_n \sin\!\left( \frac{n\pi c t}{L} \right)
\bigg].
\]
The angular frequency of the $n$-th mode is
\[
\omega_n = \frac{n\pi c}{L}.
\]

\medskip

\textbf{4. Superposition and the general solution.}

The wave equation is linear and homogeneous, so any linear combination of solutions is again a solution. Therefore the general solution satisfying the boundary conditions $u(0,t)=u(L,t)=0$ can be written as a series
\[
u(x,t) = \sum_{n=1}^{\infty}
\sin\!\left(\frac{n\pi x}{L}\right)
\bigg[
A_n \cos\!\left( \frac{n\pi c t}{L} \right)
+ B_n \sin\!\left( \frac{n\pi c t}{L} \right)
\bigg],
\]
for some coefficients $A_n$ and $B_n$ to be determined from the initial conditions.

\medskip

\textbf{5. Imposing the initial conditions and Fourier sine coefficients.}

We now use
\[
u(x,0) = f(x), \qquad u_t(x,0) = 0.
\]

First, evaluate $u(x,t)$ at $t=0$. Since $\cos(0)=1$ and $\sin(0)=0$, we obtain
\[
u(x,0) = \sum_{n=1}^{\infty}
\sin\!\left(\frac{n\pi x}{L}\right)
\left[ A_n \cdot 1 + B_n \cdot 0 \right]
= \sum_{n=1}^{\infty} A_n \sin\!\left(\frac{n\pi x}{L}\right).
\]
Thus the initial displacement must satisfy
\[
f(x) = \sum_{n=1}^{\infty} A_n \sin\!\left(\frac{n\pi x}{L}\right),
\]
which is exactly a Fourier sine series expansion of $f$ on $[0,L]$.

Next, differentiate $u(x,t)$ with respect to $t$:
\[
u_t(x,t) = \sum_{n=1}^{\infty}
\sin\!\left(\frac{n\pi x}{L}\right)
\left[
A_n \left(-\frac{n\pi c}{L}\right)
\sin\!\left( \frac{n\pi c t}{L} \right)
+ B_n \left(\frac{n\pi c}{L}\right)
\cos\!\left( \frac{n\pi c t}{L} \right)
\right].
\]
Evaluating at $t=0$ yields
\[
u_t(x,0) = \sum_{n=1}^{\infty}
\sin\!\left(\frac{n\pi x}{L}\right)
\left[
A_n \left(-\frac{n\pi c}{L}\right) \cdot 0
+ B_n \left(\frac{n\pi c}{L}\right) \cdot 1
\right]
= \sum_{n=1}^{\infty}
B_n \left(\frac{n\pi c}{L}\right)
\sin\!\left(\frac{n\pi x}{L}\right).
\]
The initial condition $u_t(x,0)=0$ for all $x$ implies that this series is identically zero. Because the sine functions $\sin\!\left(\frac{n\pi x}{L}\right)$ form an orthogonal set on $[0,L]$, the only way this can happen is if
\[
B_n = 0 \quad \text{for all } n.
\]

Therefore the solution simplifies to
\[
u(x,t) = \sum_{n=1}^{\infty}
A_n \cos\!\left( \frac{n\pi c t}{L} \right)
\sin\!\left(\frac{n\pi x}{L}\right),
\]
where the coefficients $A_n$ are precisely the Fourier sine coefficients of $f$.

Recall that the sine functions are orthogonal on $[0,L]$:
\[
\int_0^L \sin\!\left(\frac{n\pi x}{L}\right)
          \sin\!\left(\frac{m\pi x}{L}\right)\,dx
= \begin{cases}
0, & n\ne m,\\[4pt]
\displaystyle \frac{L}{2}, & n=m.
\end{cases}
\]
To find $A_n$, multiply the identity
\[
f(x) = \sum_{k=1}^{\infty} A_k \sin\!\left(\frac{k\pi x}{L}\right)
\]
by $\sin\!\left(\frac{n\pi x}{L}\right)$ and integrate from $0$ to $L$:
\[
\int_0^L f(x)\sin\!\left(\frac{n\pi x}{L}\right)\,dx
= \sum_{k=1}^{\infty} A_k
\int_0^L \sin\!\left(\frac{k\pi x}{L}\right)
         \sin\!\left(\frac{n\pi x}{L}\right)\,dx
= A_n \frac{L}{2}.
\]
Thus
\[
A_n = \frac{2}{L} \int_0^L f(x)\sin\!\left(\frac{n\pi x}{L}\right)\,dx,
\qquad n=1,2,3,\dots
\]

\medskip

\textbf{6. Final form of the solution and connection to Fourier series.}

Putting everything together, the unique solution of the wave equation
\[
u_{tt}=c^2 u_{xx}, \quad 0<x<L,\ t>0,
\]
with boundary conditions
\[
u(0,t)=u(L,t)=0,
\]
and initial conditions
\[
u(x,0)=f(x), \qquad u_t(x,0)=0,
\]
is given by
\[
u(x,t) =
\sum_{n=1}^{\infty}
\left[
\frac{2}{L} \int_0^L f(y)\sin\!\left(\frac{n\pi y}{L}\right)\,dy
\right]
\cos\!\left( \frac{n\pi c t}{L} \right)
\sin\!\left(\frac{n\pi x}{L}\right).
\]

This solution has a clear physical and mathematical interpretation. Physically, the string vibrates as a superposition of standing waves (normal modes), each with shape $\sin\!\left(\frac{n\pi x}{L}\right)$ and frequency $\omega_n = \frac{n\pi c}{L}$. Mathematically, the initial displacement $f(x)$ has been decomposed into a Fourier sine series using the eigenfunctions of the spatial operator subject to the boundary conditions. This example illustrates the central idea of the introduction to Fourier series: functions satisfying homogeneous boundary conditions can be expanded in orthogonal trigonometric series that arise naturally as eigenfunctions of differential operators, and these expansions are essential tools for solving partial differential equations such as the wave equation.

\end{solution}

% ===== Example 2: Cooling of a Rod with Prescribed End Temperatures (inquiry-based) =====
\begin{problem}[Cooling of a Rod with Prescribed End Temperatures]
Consider a thin, homogeneous metal rod of length $L$ lying along the $x$-axis from $x=0$ to $x=L$. The ends of the rod are held in contact with large ice baths, so that they are maintained at temperature zero for all time. The interior of the rod, however, starts with some nonuniform temperature distribution $f(x)$ that vanishes at the endpoints. We would like to describe how the temperature evolves over time and how Fourier series arise naturally in this context.

(a) Let $u(x,t)$ denote the temperature at position $x\in[0,L]$ and time $t\ge 0$. Write down the partial differential equation and the boundary and initial conditions that model this situation, assuming the rod has constant thermal diffusivity $\kappa>0$.

% Hint: Recall the one-dimensional heat equation, and express the information ``ends held at zero temperature'' and ``initial temperature profile $f$'' using boundary and initial conditions.

(b) A classical approach is to look for solutions that “separate” as a product of a function of $x$ and a function of $t$. Suppose $u(x,t) = X(x)T(t)$ is such a solution. 

\quad (i) Substitute $u(x,t)=X(x)T(t)$ into your heat equation from part (a), and rearrange the result to obtain an equation of the form
\[
\frac{1}{\kappa}\,\frac{T'(t)}{T(t)} \;=\; \frac{X''(x)}{X(x)}.
\]
Explain why both sides of this equation must be equal to the same constant, which we denote by $-\lambda$.

\quad (ii) Write down the two ordinary differential equations satisfied by $X$ and $T$, respectively, in terms of the constant $\lambda$.

% Hint: The idea is that the left-hand side depends only on $t$ and the right-hand side only on $x$. A function of $t$ that equals a function of $x$ for all $x,t$ must be constant.

(c) The boundary conditions $u(0,t)=u(L,t)=0$ for all $t$ translate into boundary conditions for $X(x)$. Together with your equation for $X(x)$, they define an eigenvalue problem.

\quad (i) State the boundary conditions for $X(x)$ explicitly, and write the resulting boundary value problem for $X(x)$.

\quad (ii) Analyze this boundary value problem by considering three cases: $\lambda<0$, $\lambda=0$, and $\lambda>0$. For each case, solve the differential equation for $X$ and apply the boundary conditions. For which values of $\lambda$ do nontrivial solutions $X$ exist?

\quad (iii) Show that the admissible values of $\lambda$ are
\[
\lambda_n = \left(\frac{n\pi}{L}\right)^2, \quad n=1,2,3,\dots,
\]
and that corresponding (nonzero) solutions $X_n(x)$ can be chosen in the form
\[
X_n(x) = \sin\left(\frac{n\pi x}{L}\right).
\]

% Hint: For $\lambda>0$, you will get sines and cosines. Use the boundary conditions at $x=0$ and $x=L$ to constrain the constants in the general solution.

(d) For each $n\ge 1$, you have an eigenfunction $X_n(x)$ and a corresponding time factor $T_n(t)$ solving your ODE from part (b)(ii). 

\quad (i) Solve for $T_n(t)$ when $\lambda=\lambda_n$ and write the separated solution $u_n(x,t)=X_n(x)T_n(t)$ explicitly.

\quad (ii) Explain why a linear combination
\[
u(x,t) = \sum_{n=1}^\infty b_n\,u_n(x,t)
\]
is still a solution of the heat equation satisfying the boundary conditions.

\quad (iii) At time $t=0$, use your formula for $u(x,t)$ to write an identity expressing the initial condition $u(x,0)=f(x)$ as a Fourier sine series. Derive a formula for the coefficients $b_n$ in terms of $f$ using orthogonality of the sine functions on $[0,L]$.

% Hint: Recall that the functions $\sin\left(\frac{n\pi x}{L}\right)$ are orthogonal on $[0,L]$ with respect to the usual inner product $\int_0^L \cdot \cdot \,dx$.

(e) Reflection and extensions.

\quad (i) Describe qualitatively what happens to $u(x,t)$ as $t\to\infty$. Use your explicit series representation to justify your answer.

\quad (ii) Suppose instead that the ends of the rod are held at a nonzero constant temperature $T_0$ for all $t\ge 0$. Sketch how you would modify the method to handle this new situation. In particular, what additional “steady-state” function might you try to subtract off so that the new unknown satisfies homogeneous boundary conditions?

% Hint: Look for a time-independent solution $v(x)$ with $v(0)=v(L)=T_0$, and then set $w(x,t)=u(x,t)-v(x)$.
\end{problem}

% ===== Example 2: Cooling of a Rod with Prescribed End Temperatures (full solution) =====
\begin{problem}[Cooling of a Rod with Prescribed End Temperatures]
Let $u(x,t)$ denote the temperature in a thin homogeneous rod of length $L$, with $x\in[0,L]$ and $t\ge 0$. The rod has constant thermal diffusivity $\kappa>0$, and its ends are held at zero temperature:
\[
u(0,t)=0,\quad u(L,t)=0,\quad t>0.
\]
The interior starts with temperature distribution $u(x,0)=f(x)$, where $f$ is a given function with $f(0)=f(L)=0$. The temperature satisfies the one-dimensional heat equation
\[
u_t = \kappa u_{xx}.
\]
Solve this initial–boundary value problem by separation of variables and express the solution in terms of a Fourier sine series involving $f$.
\end{problem}

\begin{solution}
We are asked to solve the heat equation on the finite interval with homogeneous Dirichlet boundary conditions. This is a standard setting where separation of variables leads naturally to Fourier sine series.

\medskip
\noindent\textbf{1. Setting up separation of variables.}
We seek solutions of the form
\[
u(x,t) = X(x)T(t),
\]
where $X$ depends only on $x$ and $T$ depends only on $t$. Substituting into the heat equation $u_t = \kappa u_{xx}$ gives
\[
X(x)\,T'(t) \;=\; \kappa\,X''(x)\,T(t).
\]
Assuming $X$ and $T$ are not identically zero, we can divide by $\kappa X(x)T(t)$ to obtain
\[
\frac{1}{\kappa}\,\frac{T'(t)}{T(t)} \;=\; \frac{X''(x)}{X(x)}.
\]
The left-hand side depends only on $t$, and the right-hand side depends only on $x$. For this equality to hold for all $x$ and $t$ in the domain, both sides must be equal to the same constant. We denote this constant by $-\lambda$ and obtain the separated equations
\[
\frac{1}{\kappa}\,\frac{T'(t)}{T(t)} = -\lambda, \qquad \frac{X''(x)}{X(x)} = -\lambda.
\]
Equivalently,
\[
T'(t) + \kappa\lambda\,T(t) = 0,\qquad X''(x) + \lambda\,X(x) = 0.
\]

The boundary conditions $u(0,t)=u(L,t)=0$ for all $t>0$ become
\[
X(0)T(t) = 0,\quad X(L)T(t) = 0 \quad\text{for all }t>0.
\]
We are interested in nontrivial products $X(x)T(t)$, so $T(t)$ cannot be identically zero. Therefore the boundary conditions reduce to
\[
X(0)=0,\qquad X(L)=0.
\]
The function $X$ must thus satisfy the boundary value problem
\[
X''(x) + \lambda X(x) = 0,\quad X(0)=0,\quad X(L)=0.
\]

\medskip
\noindent\textbf{2. Eigenvalue problem for the spatial part.}
We now solve the second-order ODE with boundary conditions by considering three cases for the separation constant $\lambda$.

\smallskip
\emph{Case 1: $\lambda<0$.} Write $\lambda = -\mu^2$ with $\mu>0$. Then $X$ satisfies
\[
X''(x) - \mu^2 X(x) = 0,
\]
whose general solution is
\[
X(x) = A e^{\mu x} + B e^{-\mu x}.
\]
Imposing $X(0)=0$ gives $A+B=0$, so $B=-A$ and
\[
X(x) = A\left(e^{\mu x} - e^{-\mu x}\right) = 2A\sinh(\mu x).
\]
Then $X(L)=0$ implies $\sinh(\mu L)=0$, but $\sinh$ is zero only at $0$, and here $\mu L>0$, so this is impossible unless $A=0$. Hence the only solution for $\lambda<0$ is the trivial solution $X\equiv0$, which we discard.

\smallskip
\emph{Case 2: $\lambda=0$.} Then $X''(x)=0$, so
\[
X(x) = A x + B.
\]
The conditions $X(0)=0$ and $X(L)=0$ give $B=0$ and $AL+B = 0$, hence $A=0$. Again the only solution is trivial.

\smallskip
\emph{Case 3: $\lambda>0$.} Write $\lambda = \mu^2$ with $\mu>0$. Then $X$ satisfies
\[
X''(x) + \mu^2 X(x) = 0,
\]
whose general solution is
\[
X(x) = A\cos(\mu x) + B\sin(\mu x).
\]
The boundary condition $X(0)=0$ gives $A=0$, so
\[
X(x) = B\sin(\mu x).
\]
The condition $X(L)=0$ then becomes
\[
B\sin(\mu L) = 0.
\]
For a nontrivial solution we require $B\ne 0$, so we must have $\sin(\mu L)=0$. Thus
\[
\mu L = n\pi,\quad\text{for some integer }n.
\]
To avoid the trivial case $\mu=0$, we take $n\in\mathbb{N}$, $n\ge1$. Hence
\[
\mu_n = \frac{n\pi}{L},\qquad \lambda_n = \mu_n^2 = \left(\frac{n\pi}{L}\right)^2,\qquad n=1,2,3,\dots
\]
For each $n\ge1$, we obtain an eigenfunction
\[
X_n(x) = \sin\left(\frac{n\pi x}{L}\right),
\]
where we have absorbed the constant $B$ into the overall constant in the separated solution.

Thus the spatial eigenvalue problem produces a discrete sequence of eigenvalues $\lambda_n$ and corresponding eigenfunctions $X_n$ satisfying the Dirichlet boundary conditions.

\medskip
\noindent\textbf{3. Time dependence and separated solutions.}
For each eigenvalue $\lambda_n$, the time factor $T_n(t)$ satisfies
\[
T_n'(t) + \kappa\lambda_n T_n(t) = 0.
\]
This is a first-order linear ODE with solution
\[
T_n(t) = C_n e^{-\kappa\lambda_n t}
       = C_n e^{-\kappa\left(\frac{n\pi}{L}\right)^2 t},
\]
where $C_n$ is an arbitrary constant. Multiplying $X_n$ and $T_n$ together, we get a separated solution
\[
u_n(x,t) = \sin\left(\frac{n\pi x}{L}\right)\,e^{-\kappa\left(\frac{n\pi}{L}\right)^2 t},
\qquad n=1,2,3,\dots.
\]
Each $u_n$ satisfies the heat equation and the homogeneous Dirichlet boundary conditions, because $X_n(0)=X_n(L)=0$.

By the linearity of the PDE and the boundary conditions, any (finite) linear combination
\[
u(x,t) = \sum_{n=1}^N b_n\,u_n(x,t)
\]
is also a solution with the same boundary conditions. To match a general initial condition $f$, we are naturally led to consider infinite series
\[
u(x,t) = \sum_{n=1}^\infty b_n\,\sin\left(\frac{n\pi x}{L}\right)\,
e^{-\kappa\left(\frac{n\pi}{L}\right)^2 t}.
\]

\medskip
\noindent\textbf{4. Expanding the initial data in a Fourier sine series.}
We now impose the initial condition $u(x,0)=f(x)$. Evaluating the series at $t=0$ gives
\[
u(x,0) = \sum_{n=1}^\infty b_n\,\sin\left(\frac{n\pi x}{L}\right)
        = f(x),
\quad 0<x<L.
\]
Thus $f$ must be represented as a Fourier sine series on $[0,L]$:
\[
f(x) \sim \sum_{n=1}^\infty b_n\,\sin\left(\frac{n\pi x}{L}\right).
\]
The sine functions
\[
\sin\left(\frac{n\pi x}{L}\right),\quad n=1,2,3,\dots,
\]
form an orthogonal set in the space $L^2(0,L)$ with respect to the inner product
\[
\langle \phi,\psi\rangle = \int_0^L \phi(x)\psi(x)\,dx.
\]
Specifically, one has
\[
\int_0^L \sin\left(\frac{n\pi x}{L}\right)\sin\left(\frac{m\pi x}{L}\right)\,dx
= \begin{cases}
0, & n\ne m,\\[4pt]
\displaystyle \frac{L}{2}, & n=m.
\end{cases}
\]
To determine the coefficients $b_n$, we multiply the identity
\[
f(x) = \sum_{n=1}^\infty b_n\,\sin\left(\frac{n\pi x}{L}\right)
\]
by $\sin\left(\frac{m\pi x}{L}\right)$, integrate from $0$ to $L$, and interchange integral and sum (justified, for example, when $f$ is piecewise smooth). We obtain
\[
\int_0^L f(x)\sin\left(\frac{m\pi x}{L}\right)\,dx
= \sum_{n=1}^\infty b_n\int_0^L \sin\left(\frac{n\pi x}{L}\right)\sin\left(\frac{m\pi x}{L}\right)\,dx.
\]
By orthogonality, all terms vanish except the one with $n=m$, and we find
\[
\int_0^L f(x)\sin\left(\frac{m\pi x}{L}\right)\,dx
= b_m \cdot \frac{L}{2}.
\]
Thus
\[
b_m = \frac{2}{L}\int_0^L f(x)\sin\left(\frac{m\pi x}{L}\right)\,dx,\qquad m=1,2,3,\dots.
\]

\medskip
\noindent\textbf{5. Final solution and qualitative behavior.}
Substituting the coefficients into our general series, we arrive at the solution
\[
u(x,t) = \sum_{n=1}^\infty 
\left[\frac{2}{L}\int_0^L f(s)\sin\left(\frac{n\pi s}{L}\right)\,ds\right]
\sin\left(\frac{n\pi x}{L}\right)\,
e^{-\kappa\left(\frac{n\pi}{L}\right)^2 t},
\quad 0<x<L,\ t>0.
\]
Under mild regularity assumptions on $f$, this series converges pointwise for $t>0$ and represents the unique solution to the given initial–boundary value problem.

Each mode
\[
\sin\left(\frac{n\pi x}{L}\right)\,e^{-\kappa\left(\frac{n\pi}{L}\right)^2 t}
\]
decays exponentially in time, with decay rate $\kappa\left(\frac{n\pi}{L}\right)^2$. Higher modes (larger $n$) decay faster, so as $t\to\infty$ the solution becomes smoother and tends to zero:
\[
\lim_{t\to\infty} u(x,t) = 0,
\]
which matches the physical expectation that the rod cools down to the boundary temperature (here, zero) everywhere.

\medskip
\noindent\textbf{6. Connection to Fourier series and the chapter’s themes.}
This example illustrates the central ideas of Fourier series in the context of partial differential equations. The spatial operator $-d^2/dx^2$ with Dirichlet boundary conditions has eigenfunctions $\sin(n\pi x/L)$ and eigenvalues $(n\pi/L)^2$. These eigenfunctions form an orthogonal basis with respect to the $L^2$ inner product on $(0,L)$, allowing us to expand arbitrary initial data $f$ as a Fourier sine series. The time evolution of each Fourier mode is governed by a simple exponential decay. Thus the solution to the heat equation is naturally expressed as an eigenfunction expansion—precisely a Fourier series—showcasing how Fourier analysis provides a powerful tool for solving linear PDEs with boundary conditions.
\end{solution}

% ===== Example 3: Fourier Series of a Square Wave Signal (inquiry-based) =====
\begin{problem}[Fourier Series of a Square Wave Signal]
In many electronic circuits, an idealized digital signal is modeled as a square wave: the voltage is held at a high level for some time, then abruptly switched to a low level, and this pattern repeats periodically. Although the physical signal may be produced by complicated circuitry, the resulting waveform can be analyzed by decomposing it into a sum of sines and cosines, each oscillating at a multiple of the fundamental frequency. This example explores how to compute such a Fourier series, and what the resulting representation tells us about discontinuous signals.

Consider the $2\pi$-periodic function $f$ defined on the interval $(-\pi,\pi)$ by
\[
f(x) = \begin{cases}
1, & 0 < x < \pi,\\[4pt]
-1, & -\pi < x < 0,
\end{cases}
\]
and then extended periodically with period $2\pi$ to all real $x$.

\smallskip

(a) Begin by understanding the geometry of $f$. Sketch its graph on $(-2\pi,2\pi)$, clearly indicating the jump discontinuities at the points $k\pi$, where $k$ is an integer. What is the average value of $f$ over one period? That is, compute
\[
\frac{1}{2\pi}\int_{-\pi}^{\pi} f(x)\,dx.
\]
How does this relate to the constant (or ``DC'') term $a_0$ in a Fourier series?

\medskip

(b) Examine the symmetry of $f$. Is $f$ an even function, an odd function, or neither? Justify your answer by checking $f(-x)$ and comparing it to $f(x)$. Based on this symmetry, which Fourier coefficients must be zero in the expansion
\[
f(x) \sim \frac{a_0}{2}+\sum_{n=1}^\infty\bigl(a_n\cos(nx)+b_n\sin(nx)\bigr)?
\]
Explain why this greatly simplifies the computation of the series.

\emph{Hint:} Recall that an odd function has only sine terms in its Fourier series, and an even function has only cosine terms.

\medskip

(c) Now compute the nonzero Fourier coefficients explicitly. Using your answer from part (b), write down the formula for the relevant coefficients (either $a_n$'s or $b_n$'s, depending on the symmetry) on $(-\pi,\pi)$:
\[
a_n = \frac{1}{\pi}\int_{-\pi}^{\pi} f(x)\cos(nx)\,dx,
\qquad
b_n = \frac{1}{\pi}\int_{-\pi}^{\pi} f(x)\sin(nx)\,dx.
\]
Reduce the integral to a simpler interval by using the definition of $f$ on $(-\pi,0)$ and $(0,\pi)$, and use any symmetry that is available. Compute an explicit formula for the nonzero coefficients and determine for which values of $n$ they vanish.

\emph{Hint:} You may find it helpful to rewrite the integral over $(-\pi,\pi)$ as a sum of integrals over $(-\pi,0)$ and $(0,\pi)$, and then make a substitution in one of them to compare the two pieces.

\medskip

(d) Assemble your work into a complete Fourier series for the square wave. Write $f(x)$ as an infinite series involving only those trigonometric terms that you have found to be nonzero:
\[
f(x)\sim \sum_{n=\text{?}}^\infty \text{(coefficient depending on $n$)}\cdot\sin(nx)
\quad\text{or}\quad
f(x)\sim \sum_{n=\text{?}}^\infty \text{(coefficient depending on $n$)}\cdot\cos(nx),
\]
whichever is appropriate. Then answer the following conceptual questions:
\begin{itemize}
  \item At which points $x$ does the Fourier series converge to $f(x)$?
  \item What is the value of the limit at the jump points $x = k\pi$?
\end{itemize}

\emph{Hint:} Recall the general convergence result for Fourier series of piecewise smooth, periodic functions: at points of continuity the series converges to the function value, while at jump discontinuities it converges to the average of the left and right limits.

\medskip

(e) Explore two variations on this model.

\begin{enumerate}
  \item Suppose instead that the signal alternates between $0$ and $1$ on each half-period, so that
  \[
  g(x) = \begin{cases}
  1, & 0 < x < \pi,\\[4pt]
  0, & -\pi < x < 0,
  \end{cases}
  \]
  extended periodically. How would the Fourier series coefficients of $g$ be related to those of $f$? In particular, how does adding a constant offset to $f$ affect its Fourier series?
  
  \item Now imagine a ``duty cycle'' that is not $50\%$: define a $2\pi$-periodic function $h$ that equals $1$ on $(0,\alpha)$ and $-1$ on $(-\pi,0)$ and $(\alpha,\pi)$, for some fixed $\alpha$ with $0<\alpha<\pi$. Without doing all the integrals, describe qualitatively what changes in the Fourier series when $\alpha$ is varied. Which coefficients are likely still zero because of symmetry, and which are not? How would the spectrum of harmonics (the relative sizes of the coefficients) depend on the length $\alpha$ of the ``on'' interval?
\end{enumerate}

\emph{Hint:} For the first variation, think about how Fourier coefficients behave under adding constants. For the second, note that changing $\alpha$ breaks some of the symmetry present in the original problem, which will affect which terms appear in the series.
\end{problem}

% ===== Example 3: Fourier Series of a Square Wave Signal (full solution) =====
\begin{problem}[Fourier Series of a Square Wave Signal]
Let $f$ be the $2\pi$-periodic function defined on $(-\pi,\pi)$ by
\[
f(x) = \begin{cases}
1, & 0 < x < \pi,\\[4pt]
-1, & -\pi < x < 0.
\end{cases}
\]
\begin{enumerate}
  \item Determine whether $f$ is even, odd, or neither, and deduce which Fourier coefficients must vanish in the expansion
  \[
  f(x) \sim \frac{a_0}{2}+\sum_{n=1}^\infty\bigl(a_n\cos(nx)+b_n\sin(nx)\bigr).
  \]
  \item Compute the nonzero Fourier coefficients explicitly and obtain a Fourier series representation of $f$.
  \item State the values to which this Fourier series converges at points of continuity and at the jump points $x=k\pi$, $k\in\mathbb{Z}$, and briefly comment on how this example illustrates the use of Fourier series for discontinuous periodic signals.
\end{enumerate}
\end{problem}

\begin{solution}
We are given a square wave that takes the value $-1$ on the left half of each period and $1$ on the right half, repeated with period $2\pi$. The goal is to expand this function in a Fourier series and then interpret the resulting trigonometric sum.

\medskip

\textbf{1. Symmetry and vanishing coefficients.}
We first determine the parity of $f$. For $x\in(0,\pi)$ we have $f(x)=1$, while $-x\in(-\pi,0)$ and hence $f(-x)=-1=-f(x)$. Similarly, for $x\in(-\pi,0)$ we have $f(x)=-1$, while $-x\in(0,\pi)$ and $f(-x)=1=-f(x)$. Thus
\[
f(-x)=-f(x)\quad\text{for all }x\in(-\pi,\pi),
\]
so $f$ is an odd function.

In a Fourier series on $(-\pi,\pi)$,
\[
f(x)\sim \frac{a_0}{2}+\sum_{n=1}^\infty\bigl(a_n\cos(nx)+b_n\sin(nx)\bigr),
\]
the cosine terms represent the even part of the function and the sine terms represent the odd part. Since $f$ is odd, its even part is identically zero, which forces $a_0=0$ and $a_n=0$ for all $n\geq 1$. Only the sine coefficients $b_n$ can be nonzero. Therefore we expect a pure sine series:
\[
f(x)\sim \sum_{n=1}^\infty b_n\sin(nx).
\]

\medskip

\textbf{2. Computation of the coefficients.}
By the standard Fourier formula on $(-\pi,\pi)$, the sine coefficients are
\[
b_n = \frac{1}{\pi}\int_{-\pi}^{\pi} f(x)\sin(nx)\,dx,\qquad n\ge 1.
\]
We now use the definition of $f$ on the two halves of the interval:
\[
b_n = \frac{1}{\pi}\left(\int_{-\pi}^{0}(-1)\sin(nx)\,dx+\int_{0}^{\pi}1\cdot\sin(nx)\,dx\right).
\]
This simplifies to
\[
b_n = \frac{1}{\pi}\left(-\int_{-\pi}^{0}\sin(nx)\,dx+\int_{0}^{\pi}\sin(nx)\,dx\right).
\]
We evaluate these standard integrals. The antiderivative of $\sin(nx)$ is $-\cos(nx)/n$, so
\[
\int_{0}^{\pi}\sin(nx)\,dx
= \left[-\frac{\cos(nx)}{n}\right]_{0}^{\pi}
= -\frac{\cos(n\pi)}{n}+\frac{\cos(0)}{n}
= \frac{1-\cos(n\pi)}{n}.
\]
Similarly,
\[
\int_{-\pi}^{0}\sin(nx)\,dx
= \left[-\frac{\cos(nx)}{n}\right]_{-\pi}^{0}
= -\frac{\cos(0)}{n}+\frac{\cos(-n\pi)}{n}
= -\frac{1}{n}+\frac{\cos(n\pi)}{n}
= \frac{\cos(n\pi)-1}{n}.
\]
Therefore
\[
-\int_{-\pi}^{0}\sin(nx)\,dx
= -\frac{\cos(n\pi)-1}{n}
= \frac{1-\cos(n\pi)}{n}.
\]
Substituting back, we obtain
\[
b_n = \frac{1}{\pi}\left(\frac{1-\cos(n\pi)}{n}+\frac{1-\cos(n\pi)}{n}\right)
= \frac{1}{\pi}\cdot\frac{2\bigl(1-\cos(n\pi)\bigr)}{n}
= \frac{2}{\pi n}\bigl(1-(-1)^n\bigr).
\]

We simplify this expression by distinguishing even and odd $n$:
\[
(-1)^n =
\begin{cases}
1, & n \text{ even},\\
-1, & n \text{ odd},
\end{cases}
\]
so
\[
1-(-1)^n =
\begin{cases}
0, & n \text{ even},\\
2, & n \text{ odd}.
\end{cases}
\]
Thus
\[
b_n =
\begin{cases}
0, & n \text{ even},\\[4pt]
\dfrac{4}{\pi n}, & n \text{ odd}.
\end{cases}
\]

It is often convenient to reindex the series over odd integers by writing $n=2k-1$, where $k=1,2,3,\dots$. Then $b_{2k-1}=\dfrac{4}{\pi(2k-1)}$ and $b_{2k}=0$.

\medskip

\textbf{3. The Fourier series for the square wave.}
Since all cosine coefficients vanish, the Fourier series for $f$ is
\[
f(x)\sim \sum_{n=1}^\infty b_n\sin(nx)
=\sum_{\substack{n=1\\ n\ \text{odd}}}^\infty \frac{4}{\pi n}\sin(nx).
\]
Reindexing over odd integers as $n=2k-1$ gives the standard form
\[
f(x)\sim \frac{4}{\pi}\sum_{k=1}^\infty \frac{1}{2k-1}\sin\bigl((2k-1)x\bigr).
\]

This representation shows that the square wave is built entirely from odd sine harmonics of the fundamental frequency. The absence of even harmonics (all even $n$ terms vanish) is a consequence of the particular symmetry and half-period structure of the function.

\medskip

\textbf{4. Convergence and behavior at the jumps.}
The function $f$ is piecewise constant on $(-\pi,\pi)$, with jump discontinuities at $x=-\pi,0,\pi$. It is piecewise smooth and periodic, so the standard convergence theorem for Fourier series applies: the Fourier series of $f$ converges at every point $x\in\mathbb{R}$, and
\begin{itemize}
  \item at each point $x$ where $f$ is continuous, the series converges to $f(x)$;
  \item at each point of discontinuity $x_0$, the series converges to the average of the left and right limits,
  \[
  \frac{1}{2}\bigl(f(x_0^-)+f(x_0^+)\bigr).
  \]
\end{itemize}

In our case, on $(-\pi,0)$ the function is constantly $-1$, and on $(0,\pi)$ it is constantly $1$, so at any $x$ with $-\pi<x<0$ or $0<x<\pi$ the Fourier series converges to $f(x)$, that is, to $-1$ or $1$ respectively.

At $x=0$, we have $f(0^-)= -1$ and $f(0^+)=1$, so the series converges to
\[
\frac{1}{2}\bigl(-1+1\bigr)=0.
\]
Similarly, at $x=\pm\pi$ the left and right limits are $f(\pi^-)=1$ and $f(\pi^+)=-1$ (using periodicity), so the average is again $0$. By periodicity, the same holds at all points $x=k\pi$ for integer $k$. Thus the Fourier series converges to $0$ at each jump point, which is the midpoint between the two levels.

This behavior is typical of Fourier series approximating discontinuous functions: near the jump, the partial sums exhibit overshoot and oscillation (the Gibbs phenomenon), but the infinite series converges to the midpoint of the jump.

\medskip

\textbf{5. Conceptual significance.}
This example illustrates several central ideas from the introduction to Fourier series:
\begin{itemize}
  \item \emph{Symmetry and simplification:} Recognizing that $f$ is odd immediately reduces the general sine–cosine expansion to a much simpler sine series.
  \item \emph{Orthogonality and coefficient formulas:} The explicit computation of $b_n$ uses the orthogonality of $\sin(nx)$ over $(-\pi,\pi)$ and the standard integral formulas, revealing precisely which harmonics are present.
  \item \emph{Representation of discontinuous signals:} Even though $f$ has jump discontinuities, it can be represented by a convergent trigonometric series. The convergence to the midpoint at jumps is a key feature when modeling idealized signals such as square waves in engineering.
\end{itemize}
Thus the square wave serves as a canonical example showing how Fourier series decompose a simple but discontinuous periodic function into an infinite sum of smooth sinusoidal components.
\end{solution}

% ===== Example 4: Even and Odd Extensions: Half-Range Expansions (inquiry-based) =====
\begin{problem}[Even and Odd Extensions: Half-Range Expansions]
In many heat or wave problems, the physical domain is a rod or string occupying the interval $0 < x < L$. However, the standard Fourier series machinery is most naturally built on symmetric intervals $[-L,L]$. One way to connect these two viewpoints is to ``imagine'' a mirror image of the rod on the left, either reflecting the temperature profile (an even extension) or flipping its sign (an odd extension). In this problem, you will discover how such extensions lead to \emph{half-range cosine} and \emph{half-range sine} series for a function given only on $(0,L)$.

Let $L>0$ be fixed, and consider the function
\[
f(x) = x, \qquad 0 < x < L.
\]
We would like to represent $f$ on $(0,L)$ using series of sines or cosines only.

\smallskip

(a) We first define extensions of $f$ to the symmetric interval $[-L,L]$.

\begin{itemize}
    \item Define the \emph{even extension} $f_{\mathrm{even}}$ of $f$ to $[-L,L]$ by
    \[
    f_{\mathrm{even}}(x) = 
    \begin{cases}
    f(x), & 0 \le x \le L,\\
    f(-x), & -L \le x < 0.
    \end{cases}
    \]
    \item Define the \emph{odd extension} $f_{\mathrm{odd}}$ of $f$ to $[-L,L]$ by
    \[
    f_{\mathrm{odd}}(x) =
    \begin{cases}
    f(x), & 0 \le x \le L,\\
    -f(-x), & -L \le x < 0.
    \end{cases}
    \]
\end{itemize}

(i) Write down explicit formulas for $f_{\mathrm{even}}(x)$ and $f_{\mathrm{odd}}(x)$ for $-L \le x \le L$.  

(ii) Sketch $f$ on $(0,L)$, and then sketch $f_{\mathrm{even}}$ and $f_{\mathrm{odd}}$ on $[-L,L]$.  

(iii) Verify directly from your formulas that $f_{\mathrm{even}}$ is an even function and $f_{\mathrm{odd}}$ is an odd function.

\emph{Hint:} Recall that a function $g$ is even if $g(-x) = g(x)$ and odd if $g(-x) = -g(x)$.

\smallskip

(b) One of the main reasons to care about even and odd extensions is that they simplify Fourier series. Suppose $g$ is integrable on $[-L,L]$ and has a Fourier series
\[
g(x) \sim \frac{a_0}{2} + \sum_{n=1}^\infty \left( a_n \cos\frac{n\pi x}{L} + b_n \sin\frac{n\pi x}{L} \right).
\]

(i) Show that if $g$ is even, then all the sine coefficients $b_n$ must be zero.  

(ii) Show that if $g$ is odd, then $a_0 = 0$ and all the cosine coefficients $a_n$ must be zero.

\emph{Hint:} Use the definitions
\[
a_n = \frac{1}{L} \int_{-L}^{L} g(x)\cos\frac{n\pi x}{L}\,dx, 
\qquad
b_n = \frac{1}{L} \int_{-L}^{L} g(x)\sin\frac{n\pi x}{L}\,dx.
\]
Think about the parity (evenness or oddness) of the integrands. When is an integral of an odd function over $[-L,L]$ equal to zero?

\smallskip

(c) Now use part (b) to express the Fourier series of $f_{\mathrm{even}}$ and $f_{\mathrm{odd}}$.

(i) Argue that $f_{\mathrm{even}}$ has a Fourier series involving only cosines:
\[
f_{\mathrm{even}}(x) \sim \frac{a_0}{2} + \sum_{n=1}^\infty a_n \cos\frac{n\pi x}{L}.
\]
Derive formulas for $a_0$ and $a_n$ in terms of integrals over $(0,L)$ only, using the evenness of $f_{\mathrm{even}}$.

(ii) Argue that $f_{\mathrm{odd}}$ has a Fourier series involving only sines:
\[
f_{\mathrm{odd}}(x) \sim \sum_{n=1}^\infty b_n \sin\frac{n\pi x}{L}.
\]
Derive a formula for $b_n$ in terms of an integral over $(0,L)$ only, using the oddness of $f_{\mathrm{odd}}$.

\emph{Hint:} For an even function $h$, 
\[
\int_{-L}^{L} h(x)\,dx = 2\int_{0}^{L} h(x)\,dx.
\]
For an odd function $h$, the integral over $[-L,L]$ is zero.

\smallskip

(d) Compute the actual Fourier coefficients for our specific function $f(x) = x$.

(i) Using your formulas from (c), compute the cosine coefficients $a_0$ and $a_n$ of $f_{\mathrm{even}}(x)$, and write down the full cosine series for $f_{\mathrm{even}}$.  

(ii) Using your formulas from (c), compute the sine coefficients $b_n$ of $f_{\mathrm{odd}}(x)$, and write down the full sine series for $f_{\mathrm{odd}}$.  

(iii) Carefully state the resulting \emph{half-range cosine series} and \emph{half-range sine series} for $f(x)=x$ on $(0,L)$.

\emph{Hint:} For $f_{\mathrm{even}}$, you will need integrals of the form $\int_0^L x\cos(\frac{n\pi x}{L})\,dx$. For $f_{\mathrm{odd}}$, you will need $\int_0^L x\sin(\frac{n\pi x}{L})\,dx$. Integration by parts is helpful.

\smallskip

(e) ``What if'' and interpretation questions.

(i) In a heat equation problem on $0 < x < L$, if you model the end $x=0$ as \emph{insulated} (no heat flux across the boundary), the mathematical boundary condition is $u_x(0,t)=0$. Explain informally why an even extension of the initial temperature profile is natural for such a condition, and why this leads to a cosine series in $x$.

(ii) If instead the end $x=0$ is held at fixed temperature $0$ so that $u(0,t)=0$, explain informally why an odd extension of the initial profile is natural, and why this leads to a sine series in $x$.

(iii) Suppose now that $f$ on $(0,L)$ does \emph{not} satisfy $f(0)=0$. Can you still form an odd extension? What happens at $x=0$ and how would this affect the Fourier series? Describe the issue qualitatively and relate it to the fact that Fourier series converge in an averaged sense at jump discontinuities.

\end{problem}

% ===== Example 4: Even and Odd Extensions: Half-Range Expansions (full solution) =====
\begin{problem}[Even and Odd Extensions: Half-Range Expansions]
Let $L>0$ and $f(x)=x$ for $0<x<L$.

(a) Define the even and odd extensions $f_{\mathrm{even}}$ and $f_{\mathrm{odd}}$ of $f$ to $[-L,L]$.  

(b) Using the parity of $f_{\mathrm{even}}$ and $f_{\mathrm{odd}}$, derive formulas for the Fourier cosine coefficients $a_n$ and sine coefficients $b_n$ in terms of integrals over $(0,L)$ only.  

(c) Compute the Fourier cosine series of $f_{\mathrm{even}}$ and the Fourier sine series of $f_{\mathrm{odd}}$ on $[-L,L]$.  

(d) From these, write explicitly the half-range cosine and half-range sine series that represent $f(x)=x$ on $(0,L)$.  

(e) Briefly explain how even and odd extensions correspond to Neumann ($u_x(0,t)=0$) and Dirichlet ($u(0,t)=0$) boundary conditions, respectively, in one-dimensional heat or wave equations.
\end{problem}

\begin{solution}
We are given $f(x)=x$ for $0<x<L$ and wish to obtain half-range cosine and sine series on $(0,L)$ by using even and odd extensions to $[-L,L]$. This example illustrates the standard Fourier series idea that symmetry (evenness or oddness) eliminates either the sine or cosine part of the series and is central in constructing half-range expansions.

\medskip

\noindent\textbf{(a) Even and odd extensions.}
By definition, the even extension $f_{\mathrm{even}}$ of $f$ to $[-L,L]$ is
\[
f_{\mathrm{even}}(x) =
\begin{cases}
f(x) = x, & 0 \le x \le L,\\[4pt]
f(-x) = -x, & -L \le x < 0,
\end{cases}
\]
so $f_{\mathrm{even}}(x)=|x|$ on $[-L,L]$.

The odd extension $f_{\mathrm{odd}}$ is
\[
f_{\mathrm{odd}}(x) =
\begin{cases}
f(x) = x, & 0 \le x \le L,\\[4pt]
-\,f(-x) = -(-x) = x, & -L \le x < 0,
\end{cases}
\]
so $f_{\mathrm{odd}}(x)=x$ on $[-L,L]$.

It is straightforward to check parity. For $f_{\mathrm{even}}(x)=|x|$ we have $f_{\mathrm{even}}(-x)=|-x|=|x|=f_{\mathrm{even}}(x)$, so it is even. For $f_{\mathrm{odd}}(x)=x$ we have $f_{\mathrm{odd}}(-x)=-x=-f_{\mathrm{odd}}(x)$, so it is odd.

\medskip

\noindent\textbf{(b) Consequences of parity for Fourier coefficients.}
Let $g$ be integrable on $[-L,L]$ with Fourier series
\[
g(x) \sim \frac{a_0}{2} + \sum_{n=1}^\infty\left( a_n\cos\frac{n\pi x}{L} + b_n\sin\frac{n\pi x}{L}\right).
\]
The coefficients are
\[
a_n = \frac{1}{L}\int_{-L}^{L} g(x)\cos\frac{n\pi x}{L}\,dx, 
\quad
b_n = \frac{1}{L}\int_{-L}^{L} g(x)\sin\frac{n\pi x}{L}\,dx.
\]

If $g$ is even, then the product $g(x)\cos\frac{n\pi x}{L}$ is even (even $\times$ even), while $g(x)\sin\frac{n\pi x}{L}$ is odd (even $\times$ odd). The integral of an odd function over $[-L,L]$ is zero, so $b_n=0$ for all $n$. Therefore the Fourier series of an even function contains only cosine terms (and possibly $a_0$).

If $g$ is odd, then $g(x)\cos\frac{n\pi x}{L}$ is odd (odd $\times$ even), and $g(x)\sin\frac{n\pi x}{L}$ is even (odd $\times$ odd). Thus $a_n = 0$ for all $n\ge 0$, while the $b_n$ may be nonzero. Hence the Fourier series of an odd function contains only sine terms.

\medskip

\noindent\textbf{(c) Reduction to integrals over $(0,L)$.}

\emph{Even case.} Since $f_{\mathrm{even}}$ is even, its Fourier series takes the form
\[
f_{\mathrm{even}}(x) \sim \frac{a_0}{2} + \sum_{n=1}^\infty a_n\cos\frac{n\pi x}{L},
\]
with
\[
a_n = \frac{1}{L}\int_{-L}^{L} f_{\mathrm{even}}(x)\cos\frac{n\pi x}{L}\,dx.
\]
The integrand is even, because $f_{\mathrm{even}}$ and the cosine are both even, so
\[
a_n = \frac{2}{L}\int_{0}^{L} f_{\mathrm{even}}(x)\cos\frac{n\pi x}{L}\,dx
     = \frac{2}{L}\int_{0}^{L} x\cos\frac{n\pi x}{L}\,dx,
\]
for $n\ge1$. Similarly,
\[
a_0 = \frac{1}{L}\int_{-L}^{L} f_{\mathrm{even}}(x)\,dx
    = \frac{2}{L}\int_{0}^{L} x\,dx.
\]

\emph{Odd case.} Since $f_{\mathrm{odd}}$ is odd, its Fourier series has only sine terms:
\[
f_{\mathrm{odd}}(x) \sim \sum_{n=1}^\infty b_n\sin\frac{n\pi x}{L},
\]
with
\[
b_n = \frac{1}{L}\int_{-L}^{L} f_{\mathrm{odd}}(x)\sin\frac{n\pi x}{L}\,dx.
\]
Here the integrand is even (odd $\times$ odd), so
\[
b_n = \frac{2}{L}\int_{0}^{L} f_{\mathrm{odd}}(x)\sin\frac{n\pi x}{L}\,dx
     = \frac{2}{L}\int_{0}^{L} x\sin\frac{n\pi x}{L}\,dx.
\]

Thus the coefficients are given entirely by integrals over $(0,L)$.

\medskip

\noindent\textbf{(d) Computing the coefficients and the series.}

\emph{Cosine series for $f_{\mathrm{even}}(x)=|x|$.} 

First compute $a_0$:
\[
a_0 = \frac{2}{L}\int_0^L x\,dx = \frac{2}{L}\cdot \frac{L^2}{2} = L.
\]
Therefore the constant term is $a_0/2 = L/2$.

For $n\ge 1$,
\[
a_n = \frac{2}{L}\int_0^L x\cos\frac{n\pi x}{L}\,dx.
\]
Use integration by parts. Let
\[
u=x,\quad dv=\cos\frac{n\pi x}{L}\,dx,\quad du=dx,\quad v=\frac{L}{n\pi}\sin\frac{n\pi x}{L}.
\]
Then
\[
\int_0^L x\cos\frac{n\pi x}{L}\,dx 
= \left. x\cdot\frac{L}{n\pi}\sin\frac{n\pi x}{L}\right|_0^L 
  - \int_0^L \frac{L}{n\pi}\sin\frac{n\pi x}{L}\,dx.
\]
The boundary term is zero, since $\sin\frac{n\pi L}{L} = \sin(n\pi)=0$ and $\sin 0=0$. Thus
\[
\int_0^L x\cos\frac{n\pi x}{L}\,dx 
= -\frac{L}{n\pi}\int_0^L \sin\frac{n\pi x}{L}\,dx.
\]
The remaining integral is
\[
\int_0^L \sin\frac{n\pi x}{L}\,dx
= \left.-\frac{L}{n\pi}\cos\frac{n\pi x}{L}\right|_0^L
= -\frac{L}{n\pi}\bigl(\cos(n\pi)-1\bigr)
= -\frac{L}{n\pi}\bigl((-1)^n -1\bigr).
\]
Therefore
\[
\int_0^L x\cos\frac{n\pi x}{L}\,dx
= -\frac{L}{n\pi}\cdot\left(-\frac{L}{n\pi}\bigl((-1)^n-1\bigr)\right)
= \frac{L^2}{n^2\pi^2}\bigl((-1)^n-1\bigr).
\]
Consequently,
\[
a_n = \frac{2}{L}\cdot \frac{L^2}{n^2\pi^2}\bigl((-1)^n-1\bigr)
    = \frac{2L}{n^2\pi^2}\bigl((-1)^n-1\bigr).
\]

Note that if $n$ is even, then $(-1)^n=1$ and $a_n=0$. If $n$ is odd, say $n=2k+1$, then $(-1)^n=-1$, so $(-1)^n-1=-2$ and
\[
a_n = -\frac{4L}{n^2\pi^2}.
\]
Thus only odd cosine terms appear. We can write
\[
f_{\mathrm{even}}(x)=|x| \sim \frac{L}{2} - \frac{4L}{\pi^2}\sum_{\substack{n=1\\ n\ \mathrm{odd}}}^{\infty} \frac{1}{n^2}\cos\frac{n\pi x}{L}.
\]

\emph{Sine series for $f_{\mathrm{odd}}(x)=x$.}

For $n\ge1$,
\[
b_n = \frac{2}{L}\int_0^L x\sin\frac{n\pi x}{L}\,dx.
\]
Again use integration by parts, but now let
\[
u=x,\quad dv=\sin\frac{n\pi x}{L}\,dx,\quad du=dx,\quad v=-\frac{L}{n\pi}\cos\frac{n\pi x}{L}.
\]
Then
\[
\int_0^L x\sin\frac{n\pi x}{L}\,dx
= \left. -x\frac{L}{n\pi}\cos\frac{n\pi x}{L}\right|_0^L
  + \int_0^L \frac{L}{n\pi}\cos\frac{n\pi x}{L}\,dx.
\]
The boundary term is
\[
-\,\frac{L}{n\pi}\bigl(L\cos(n\pi) - 0\cdot\cos 0\bigr)
= -\frac{L^2}{n\pi}(-1)^n.
\]
The remaining integral is
\[
\int_0^L \frac{L}{n\pi}\cos\frac{n\pi x}{L}\,dx
= \frac{L}{n\pi}\left.\frac{L}{n\pi}\sin\frac{n\pi x}{L}\right|_0^L = 0,
\]
since the sine vanishes at both endpoints. Thus
\[
\int_0^L x\sin\frac{n\pi x}{L}\,dx = -\frac{L^2}{n\pi}(-1)^n.
\]
It follows that
\[
b_n = \frac{2}{L}\cdot\left(-\frac{L^2}{n\pi}(-1)^n\right)
    = -\frac{2L}{n\pi}(-1)^n
    = \frac{2L}{n\pi}(-1)^{n+1}.
\]

Therefore the sine series of $f_{\mathrm{odd}}(x)=x$ on $[-L,L]$ is
\[
f_{\mathrm{odd}}(x)=x \sim \sum_{n=1}^\infty \frac{2L}{n\pi}(-1)^{n+1}\sin\frac{n\pi x}{L}.
\]

\medskip

\noindent\textbf{Half-range expansions on $(0,L)$.}

On $(0,L)$, both $f_{\mathrm{even}}$ and $f_{\mathrm{odd}}$ coincide with $f(x)=x$. Hence we may simply restrict the above series to $(0,L)$ to obtain the half-range series.

The \emph{half-range cosine series} for $f(x)=x$ on $(0,L)$ is 
\[
x \sim \frac{L}{2} - \frac{4L}{\pi^2}\sum_{\substack{n=1\\ n\ \mathrm{odd}}}^{\infty} \frac{1}{n^2}\cos\frac{n\pi x}{L},
\qquad 0<x<L.
\]

The \emph{half-range sine series} for $f(x)=x$ on $(0,L)$ is
\[
x \sim \sum_{n=1}^\infty \frac{2L}{n\pi}(-1)^{n+1}\sin\frac{n\pi x}{L},
\qquad 0<x<L.
\]

These are simply the Fourier cosine series of $|x|$ and the Fourier sine series of $x$ on $[-L,L]$, viewed on the right half of the interval.

\medskip

\noindent\textbf{(e) Interpretation in terms of boundary conditions.}

In one-dimensional heat or wave equations on $0<x<L$, spatial boundary conditions at $x=0$ determine which eigenfunctions are appropriate:

\begin{itemize}
    \item If the end $x=0$ is \emph{insulated} (Neumann condition $u_x(0,t)=0$), the flux across the boundary is zero. Spatial eigenfunctions satisfying $X'(0)=0$ are cosines: $X_n(x)=\cos\frac{n\pi x}{L}$. Extending the temperature profile evenly across $x=0$ produces an even function, whose derivative is odd and therefore vanishes at the origin. This is consistent with $u_x(0,t)=0$ and leads naturally to expansions in cosine series.

    \item If the end $x=0$ is held at temperature zero (Dirichlet condition $u(0,t)=0$), the spatial eigenfunctions satisfying $X(0)=0$ are sines: $X_n(x)=\sin\frac{n\pi x}{L}$. Extending the solution oddly across $x=0$ forces $u(0,t)=0$ (since an odd function vanishes at the origin), and so an odd extension of the initial profile leads naturally to a sine series.

\end{itemize}

If $f$ on $(0,L)$ does not satisfy $f(0)=0$, one can still define an odd extension by setting
\[
\tilde f_{\mathrm{odd}}(x) =
\begin{cases}
f(x), & 0<x\le L,\\
-\,f(-x), & -L\le x<0,\\
0, & x=0.
\end{cases}
\]
This introduces a jump discontinuity at $x=0$ unless $f(0+)=0$. The resulting sine series converges to $0$ at $x=0$ (the midpoint of the jump), and to the odd extension elsewhere. This reflects the general principle that Fourier series converge at a jump to the average of the left and right limits. Thus, even when the original data do not exactly match the boundary condition, half-range sine or cosine series still approximate the function in the $L^2$ sense, but pointwise behavior at the boundary reflects the chosen symmetry.

Overall, this example shows how the introductory ideas of Fourier series—orthogonality of sines and cosines, parity, and eigenfunctions for boundary value problems—combine to produce useful half-range expansions from data given only on $(0,L)$.

\end{solution}

% ===== Example 5: From Trigonometric Series to Complex Exponential Form (inquiry-based) =====
\begin{problem}[From Trigonometric Series to Complex Exponential Form]
In applications, real-valued periodic signals are often expanded in a Fourier series using sines and cosines. On the other hand, many theoretical developments in Fourier analysis, linear algebra, and complex analysis prefer to work with complex exponentials $e^{inx}$. These two descriptions of the same function look quite different, but they are mathematically equivalent. In this problem you will gently uncover how to pass back and forth between these two viewpoints.

Consider a real-valued $2\pi$-periodic function $f$ with a (formal) trigonometric Fourier series
\[
f(x) \sim \frac{a_0}{2} \;+\; \sum_{n=1}^{\infty} \bigl( a_n \cos(nx) + b_n \sin(nx) \bigr),
\]
where $a_n, b_n \in \mathbb{R}$ for all $n$.

\smallskip

(a) Warm-up: Start with a single frequency. Let
\[
g(x) = a \cos x + b \sin x,
\]
where $a,b \in \mathbb{R}$. Use Euler's formulas
\[
\cos x = \frac{e^{ix} + e^{-ix}}{2}, \qquad \sin x = \frac{e^{ix} - e^{-ix}}{2i}
\]
to rewrite $g(x)$ as a linear combination of $e^{ix}$ and $e^{-ix}$:
\[
g(x) = c_1 e^{ix} + c_{-1} e^{-ix}
\]
for some complex numbers $c_1$ and $c_{-1}$. 

\begin{itemize}
\item[(i)] Compute $c_1$ and $c_{-1}$ in terms of $a$ and $b$.
\item[(ii)] Solve for $a$ and $b$ in terms of $c_1$ and $c_{-1}$, and note any symmetry relating $c_{-1}$ and $c_1$ when $g$ is real-valued.
\end{itemize}
Hint: Collect the coefficients of $e^{ix}$ and $e^{-ix}$ after substituting the Euler formulas.

\smallskip

(b) Now extend this idea to all integer frequencies. For each $n \ge 1$, use the Euler identities to write
\[
a_n \cos(nx) + b_n \sin(nx)
\]
as a linear combination of $e^{inx}$ and $e^{-inx}$:
\[
a_n \cos(nx) + b_n \sin(nx) = c_n e^{inx} + c_{-n} e^{-inx}.
\]
Find formulas for $c_n$ and $c_{-n}$ in terms of $a_n$ and $b_n$. 

Hint: This is exactly the same algebra as in part (a), with $x$ replaced by $nx$.

\smallskip

(c) Combine your work from part (b) with the constant term $\frac{a_0}{2}$ to show that $f$ can be written (formally) as a complex exponential series
\[
f(x) \sim \sum_{n=-\infty}^{\infty} c_n e^{inx},
\]
and express $c_0$, $c_n$ (for $n \ge 1$), and $c_{-n}$ (for $n \ge 1$) explicitly in terms of $a_n$ and $b_n$. 

Then, invert these relations: solve for $a_n$ and $b_n$ in terms of the complex coefficients $c_n$. 

Hint: For $n \ge 1$, you already know how $a_n \cos(nx) + b_n \sin(nx)$ splits into the $n$ and $-n$ modes. Think about what happens when you sum over all $n$.

\smallskip

(d) Reality condition. Suppose $f$ is real-valued for all $x$. What does this imply about the complex coefficients $c_n$? Show that
\[
c_0 \in \mathbb{R}, \qquad c_{-n} = \overline{c_n} \quad \text{for all } n \ge 1.
\]
Conversely, show that if a formal complex exponential series
\[
\sum_{n=-\infty}^{\infty} c_n e^{inx}
\]
satisfies $c_0 \in \mathbb{R}$ and $c_{-n} = \overline{c_n}$ for all $n \ge 1$, then the series (if convergent) represents a real-valued function. 

Hint: Use that $\overline{e^{inx}} = e^{-inx}$ and that $f$ being real means $f(x) = \overline{f(x)}$ for all $x$.

\smallskip

(e) Explorations and extensions.

\begin{itemize}
\item[(i)] Suppose now that $f$ is \emph{not} necessarily real-valued, but complex-valued. Which parts of your reasoning in (c) and (d) still go through? What can you say about the relation between $a_n, b_n$ and $c_n$ in this setting?
\item[(ii)] How would your formulas change if the period of $f$ were $2L$ instead of $2\pi$? Write down what the complex exponential series would look like in that case, and how the “frequency” $n$ should be rescaled.
\item[(iii)] Conceptual question: In what ways might the complex exponential form
\[
\sum_{n=-\infty}^{\infty} c_n e^{inx}
\]
be more convenient than the sine–cosine form when thinking of Fourier series as coordinates in a vector space or when using tools from complex analysis?
\end{itemize}
\end{problem}

% ===== Example 5: From Trigonometric Series to Complex Exponential Form (full solution) =====
\begin{problem}[From Trigonometric Series to Complex Exponential Form]
Let $f$ be a (formal) real-valued $2\pi$-periodic function with trigonometric Fourier series
\[
f(x) \sim \frac{a_0}{2} + \sum_{n=1}^{\infty} \bigl(a_n \cos(nx) + b_n \sin(nx)\bigr),
\qquad a_n, b_n \in \mathbb{R}.
\]
\begin{enumerate}
\item[(i)] Using Euler's formulas
\[
\cos(nx) = \frac{e^{inx} + e^{-inx}}{2}, 
\qquad 
\sin(nx) = \frac{e^{inx} - e^{-inx}}{2i},
\]
show that $f$ can also be written (formally) in the complex exponential form
\[
f(x) \sim \sum_{n=-\infty}^{\infty} c_n e^{inx}
\]
for suitable complex coefficients $c_n$.
\item[(ii)] Derive explicit formulas for $c_n$ in terms of $a_n$ and $b_n$, and for $a_n$, $b_n$ in terms of $c_n$.
\item[(iii)] Prove that $f$ is real-valued if and only if
\[
c_0 \in \mathbb{R}
\quad\text{and}\quad
c_{-n} = \overline{c_n} \quad \text{for all } n \ge 1.
\]
\end{enumerate}
\end{problem}

\begin{solution}
We begin from the trigonometric Fourier series
\[
f(x) \sim \frac{a_0}{2} + \sum_{n=1}^{\infty} \bigl(a_n \cos(nx) + b_n \sin(nx)\bigr),
\]
where $a_n$ and $b_n$ are real numbers. The main idea is to use Euler's formulas to express the trigonometric functions in terms of complex exponentials, and then to group the coefficients of $e^{inx}$ for all integers $n$.

\medskip

\noindent\textbf{Step 1: Rewrite a single harmonic in complex exponential form.}

Fix $n \ge 1$ and consider the $n$th harmonic
\[
a_n \cos(nx) + b_n \sin(nx).
\]
Using Euler’s formulas
\[
\cos(nx) = \frac{e^{inx} + e^{-inx}}{2}, 
\qquad 
\sin(nx) = \frac{e^{inx} - e^{-inx}}{2i},
\]
we substitute to obtain
\begin{align*}
a_n \cos(nx) + b_n \sin(nx)
&= a_n \cdot \frac{e^{inx} + e^{-inx}}{2} 
  + b_n \cdot \frac{e^{inx} - e^{-inx}}{2i} \\
&= \left(\frac{a_n}{2} + \frac{b_n}{2i}\right)e^{inx}
   + \left(\frac{a_n}{2} - \frac{b_n}{2i}\right) e^{-inx}.
\end{align*}
It is convenient to rewrite $\frac{1}{i} = -i$, so
\[
\frac{b_n}{2i} = -\frac{i b_n}{2}.
\]
Thus we have
\[
a_n \cos(nx) + b_n \sin(nx)
= \left(\frac{a_n}{2} - \frac{i b_n}{2}\right)e^{inx}
  + \left(\frac{a_n}{2} + \frac{i b_n}{2}\right)e^{-inx}.
\]
This suggests defining
\[
c_n := \frac{a_n - i b_n}{2},
\qquad
c_{-n} := \frac{a_n + i b_n}{2}.
\]
Then
\[
a_n \cos(nx) + b_n \sin(nx) = c_n e^{inx} + c_{-n} e^{-inx}.
\]

\medskip

\noindent\textbf{Step 2: Assemble the full complex exponential series.}

Now we apply this decomposition to each harmonic in the Fourier series for $f$. We obtain
\begin{align*}
f(x) 
&\sim \frac{a_0}{2}
   + \sum_{n=1}^{\infty} \bigl(a_n \cos(nx) + b_n \sin(nx)\bigr) \\
&= \frac{a_0}{2}
   + \sum_{n=1}^{\infty} \bigl(c_n e^{inx} + c_{-n} e^{-inx}\bigr),
\end{align*}
where for $n \ge 1$ we have defined
\[
c_n = \frac{a_n - i b_n}{2},
\qquad
c_{-n} = \frac{a_n + i b_n}{2}.
\]
If we additionally define
\[
c_0 := \frac{a_0}{2},
\]
then the series may be written more symmetrically as
\[
f(x) \sim \sum_{n=-\infty}^{\infty} c_n e^{inx}.
\]
This proves part (i) of the problem.

\medskip

\noindent\textbf{Step 3: Relations between $(a_n,b_n)$ and $c_n$.}

We already obtained formulas for $c_n$ in terms of $a_n$ and $b_n$ for $n \ge 1$:
\[
c_n = \frac{a_n - i b_n}{2}, \qquad c_{-n} = \frac{a_n + i b_n}{2},
\]
and $c_0 = a_0 / 2$.

Conversely, to express $a_n$ and $b_n$ in terms of $c_n$ and $c_{-n}$, we simply solve these linear equations. For $n \ge 1$ we add and subtract:
\[
c_n + c_{-n} 
= \frac{a_n - i b_n}{2} + \frac{a_n + i b_n}{2}
= a_n,
\]
and
\[
c_{-n} - c_n
= \frac{a_n + i b_n}{2} - \frac{a_n - i b_n}{2}
= i b_n.
\]
Therefore
\[
a_n = c_n + c_{-n},
\qquad
b_n = \frac{1}{i}\,(c_{-n} - c_n) = -i(c_{-n} - c_n).
\]
Since $1/i = -i$, this expression is purely algebraic and valid for complex $c_n$. For the constant term, we have
\[
a_0 = 2 c_0.
\]

This completes the formulas in part (ii):
\[
\boxed{
\begin{aligned}
c_0 &= \frac{a_0}{2},\\[0.3em]
c_n &= \frac{a_n - i b_n}{2}, \quad c_{-n} = \frac{a_n + i b_n}{2} \quad (n \ge 1),
\end{aligned}}
\]
and
\[
\boxed{
\begin{aligned}
a_0 &= 2c_0,\\[0.3em]
a_n &= c_n + c_{-n}, \quad
b_n = -i(c_{-n} - c_n) \quad (n \ge 1).
\end{aligned}}
\]

\medskip

\noindent\textbf{Step 4: Characterizing real-valued series via $c_{-n} = \overline{c_n}$.}

We now address part (iii), which concerns when the complex exponential series represents a real-valued function.

\smallskip

\emph{($\Rightarrow$) If $f$ is real-valued, then $c_0 \in \mathbb{R}$ and $c_{-n} = \overline{c_n}$ for all $n \ge 1$.}

Because $f$ is real-valued and the real Fourier coefficients $a_n$ and $b_n$ are real numbers, we have $a_0 \in \mathbb{R}$ and $a_n, b_n \in \mathbb{R}$ for $n \ge 1$. From the formulas obtained above,
\[
c_0 = \frac{a_0}{2},
\]
so immediately $c_0$ is real.

For $n \ge 1$ we have
\[
c_n = \frac{a_n - i b_n}{2},
\qquad
c_{-n} = \frac{a_n + i b_n}{2}.
\]
Since $a_n$ and $b_n$ are real, complex conjugation gives
\[
\overline{c_n} 
= \frac{a_n + i b_n}{2}
= c_{-n}.
\]
Thus $c_{-n} = \overline{c_n}$ for every $n \ge 1$. This establishes one direction.

\smallskip

\emph{($\Leftarrow$) If $c_0 \in \mathbb{R}$ and $c_{-n} = \overline{c_n}$ for all $n \ge 1$, then the series represents a real-valued function (where convergent).}

Assume a formal series
\[
f(x) \sim \sum_{n=-\infty}^{\infty} c_n e^{inx}
\]
has coefficients satisfying $c_0 \in \mathbb{R}$ and the symmetry $c_{-n} = \overline{c_n}$ for all $n \ge 1$. For each $x$, take the complex conjugate termwise:
\[
\overline{f(x)} 
\sim \overline{\sum_{n=-\infty}^{\infty} c_n e^{inx}}
= \sum_{n=-\infty}^{\infty} \overline{c_n} \, \overline{e^{inx}}
= \sum_{n=-\infty}^{\infty} \overline{c_n} \, e^{-inx},
\]
since $\overline{e^{inx}} = e^{-inx}$.

Now relabel the index $m = -n$:
\[
\overline{f(x)} 
= \sum_{m=-\infty}^{\infty} \overline{c_{-m}} \, e^{imx}.
\]
By hypothesis $\overline{c_{-m}} = c_m$ for all $m \neq 0$, and $\overline{c_0} = c_0$ since $c_0$ is real. Thus
\[
\overline{f(x)} = \sum_{m=-\infty}^{\infty} c_m e^{imx} \sim f(x).
\]
That is, the conjugate series coincides with the original series. Therefore wherever the series converges, its value satisfies $\overline{f(x)} = f(x)$, which means $f(x)$ is real. This proves the converse.

\medskip

\noindent\textbf{Conceptual remarks.}

This example illustrates how the sine–cosine and complex exponential forms of Fourier series are simply two coordinate systems on the same space of periodic functions. The passage between them rests on Euler's formulas, and the key structural idea is that $\{e^{inx}\}_{n \in \mathbb{Z}}$ forms an orthogonal family (in $L^{2}[-\pi,\pi]$), making it a very natural basis from the perspective of linear algebra. 

Moreover, the symmetry condition $c_{-n} = \overline{c_n}$ succinctly encodes the real-valuedness of $f$, which is somewhat more cumbersome to express in terms of $a_n$ and $b_n$. In the broader context of Fourier series, this complex exponential viewpoint is especially powerful when relating Fourier analysis to eigenfunction expansions, to the spectral theory of differential operators, and to methods of complex analysis (such as contour integration), all of which heavily favor the use of $e^{inx}$ over sines and cosines.

\end{solution}

\section{Properties of the Fourier Series}
% TODO: Add narrative / plan for this section.

% TODO: Use prompts_for_sections.py to design examples and add them here.

\section{Riemann–Lebesgue Lemma}
% TODO: Add narrative / plan for this section.

% TODO: Use prompts_for_sections.py to design examples and add them here.

\section{Gibbs Phenomenon}
% TODO: Add narrative / plan for this section.

% TODO: Use prompts_for_sections.py to design examples and add them here.

\section{Laplace Transform}
% --- Narrative plan (auto-generated) ---
% This section develops the Laplace transform as a central tool for solving linear differential equations, especially those driven by external forcing and subject to initial conditions. The main idea is to convert problems in the time domain, typically expressed as ordinary differential equations or simple partial differential equations, into algebraic problems in a new variable, often called the complex frequency. Once in this transformed setting, the equations become easier to manipulate and solve, after which the inverse transform returns the solution to the original time variable.
%
% The Laplace transform is especially valuable in applied mathematics because it handles discontinuous inputs, impulsive forces, and growing exponentials in a unified way. It provides a natural language for describing dynamical systems, electrical circuits, and control systems, and it shares many themes with Fourier analysis, such as convolution and spectral representation. Conceptually, it connects to complex analysis through analytic continuation and contour integration, and it complements Fourier transforms in the analysis of PDEs by focusing on initial-value problems and one-sided time evolution rather than periodic or infinite-time behavior.
%
% Throughout this section, we will build a toolkit that includes basic transform pairs, operational rules (such as shifting and differentiation), and systematic methods for partial fraction decomposition in the transform domain. Along the way, you will see how Laplace transforms streamline the solution of linear ODEs and simple PDEs, how they encode initial conditions directly into algebraic equations, and how they interact with convolution to describe input–output relations in linear systems.

% ===== Example 1: Exponential Growth and Decay via Laplace Transform (inquiry-based) =====
\begin{problem}[Exponential Growth and Decay via Laplace Transform]
Many physical and biological processes are modeled by first-order linear differential equations.  For example, a population that grows at a rate proportional to its size, or a radioactive substance that decays at a rate proportional to the amount present, can both be described by a simple exponential law.  In this problem you will re-discover the classical exponential growth and decay formulas, but this time \emph{purely} through the Laplace transform.  The goal is to practice writing down the Laplace transform of the unknown function, using the initial condition, and solving the resulting algebraic equation before inverting it.

Consider the initial value problem
\[
y'(t) = k\,y(t), \qquad y(0)=y_0,
\]
where $k$ and $y_0$ are real constants.  Assume $t \ge 0$ and that $y(t)$ is of exponential order so that its Laplace transform exists.

\medskip

(a) Recall or derive the Laplace transform of a derivative.  Let $Y(s) = \mathcal{L}\{y(t)\}(s) = \displaystyle\int_0^{\infty} e^{-st} y(t)\,dt$.  Starting from the definition of the Laplace transform, compute
\[
\mathcal{L}\{y'(t)\}(s) = \int_0^{\infty} e^{-st} y'(t)\,dt
\]
using integration by parts, and express your answer in terms of $Y(s)$ and $y(0)$.  
Hint: For integration by parts, take $u = e^{-st}$ and $dv = y'(t)\,dt$.

\medskip

(b) Apply the Laplace transform to both sides of the differential equation $y'(t) = k\,y(t)$.  Use your result from part (a) and the initial condition $y(0)=y_0$ to obtain an algebraic equation for $Y(s)$.  Solve this equation explicitly for $Y(s)$.
% Hint: You should get an expression of the form $Y(s) = \dfrac{y_0}{s - k}$.

\medskip

(c) In order to recover $y(t)$ from $Y(s)$, you need to recognize $Y(s)$ as the Laplace transform of a familiar function.  Recall from a Laplace transform table (or check directly from the definition) that
\[
\mathcal{L}\left\{e^{at}\right\}(s) = \frac{1}{s-a}, \qquad \text{for } s>a.
\]
Use this fact, together with your expression for $Y(s)$ in part (b), to determine $y(t)$.  Write your final answer explicitly in terms of $k$, $y_0$, and $t$.
% Hint: Compare $Y(s)$ with $\dfrac{1}{s-a}$ and identify $a$.

\medskip

(d) Interpret your formula for $y(t)$ in two important cases:
\begin{itemize}
  \item[(i)] $k>0$ (for instance, modeling population growth with proportional birth rate),
  \item[(ii)] $k<0$ (for instance, modeling radioactive decay or Newtonian cooling toward zero temperature).
\end{itemize}
Explain in one or two sentences for each case how the sign of $k$ affects the qualitative behavior of $y(t)$ as $t \to \infty$.

\medskip

(e) ``What if'' extensions.

\begin{itemize}
  \item[(i)] Suppose now that there is a constant input term, for example a population with births proportional to its size \emph{and} a constant immigration rate.  Consider
  \[
  y'(t) = k\,y(t) + b, \qquad y(0) = y_0,
  \]
  where $b$ is a constant.  Without carrying out all the algebra, outline how you would modify the Laplace transform steps from parts (a)--(c) to solve this new problem.  Which additional Laplace transform do you need to know, beyond the ones already used?
  % Hint: You will need $\mathcal{L}\{1\}(s)$.

  \item[(ii)] For the equation in part (e)(i), imagine $k < 0$ (a decay term) and $b > 0$ (a constant source).  Based on your physical intuition, do you expect $y(t)$ to blow up, decay to zero, or approach some steady-state value as $t \to \infty$?  Briefly justify your answer using the structure of the differential equation (you do not need to compute the exact solution, but you may if you wish).
\end{itemize}

\end{problem}

% ===== Example 1: Exponential Growth and Decay via Laplace Transform (full solution) =====
\begin{problem}[Exponential Growth and Decay via Laplace Transform]
Use the Laplace transform to solve the following initial value problems, and briefly interpret the behavior of the solutions:

\begin{enumerate}
  \item[(i)] $y'(t) = k\,y(t)$, \; $y(0) = y_0$, where $k$ and $y_0$ are real constants.
  \item[(ii)] $y'(t) = k\,y(t) + b$, \; $y(0) = y_0$, where $k$, $b$, and $y_0$ are real constants.
\end{enumerate}

Assume $t \ge 0$ and that in each case the solution is of exponential order so its Laplace transform exists.
\end{problem}

\begin{solution}
The central idea of using the Laplace transform to solve an initial value problem is that the transform converts the differential equation, together with its initial condition, into an algebraic equation for the Laplace transform of the unknown function.  After solving this algebraic equation, we invert the transform to recover the solution in the time domain.  This example illustrates this general procedure in the simplest possible setting of first-order linear equations modeling exponential growth and decay.

\medskip

We begin with the Laplace transform of a derivative.  Let $y(t)$ be a function with Laplace transform
\[
Y(s) = \mathcal{L}\{y(t)\}(s) = \int_0^{\infty} e^{-st} y(t)\,dt.
\]
We compute the transform of $y'(t)$ directly from the definition:
\[
\mathcal{L}\{y'(t)\}(s) = \int_0^{\infty} e^{-st} y'(t)\,dt.
\]
We apply integration by parts with $u = e^{-st}$ and $dv = y'(t)\,dt$, so that $du = -s e^{-st}\,dt$ and $v = y(t)$.  Then
\[
\int_0^{\infty} e^{-st} y'(t)\,dt
= \Bigl[e^{-st} y(t)\Bigr]_0^{\infty} - \int_0^{\infty} (-s e^{-st}) y(t)\,dt
= \Bigl[e^{-st} y(t)\Bigr]_0^{\infty} + s \int_0^{\infty} e^{-st} y(t)\,dt.
\]
Under the usual growth assumptions that guarantee the existence of the Laplace transform, we have $e^{-st} y(t) \to 0$ as $t \to \infty$ when $\Re(s)$ is large enough.  Hence
\[
\Bigl[e^{-st} y(t)\Bigr]_0^{\infty} = 0 - y(0) = -y(0).
\]
Therefore
\[
\mathcal{L}\{y'(t)\}(s) = -y(0) + s Y(s) = s Y(s) - y(0).
\]
This identity is the key formula that incorporates the initial value into the transformed equation.

\medskip

\noindent\textbf{Part (i):} Solve $y'(t) = k\,y(t)$, $y(0) = y_0$.

Applying the Laplace transform to both sides of the differential equation gives
\[
\mathcal{L}\{y'(t)\}(s) = \mathcal{L}\{k\,y(t)\}(s).
\]
Using linearity of the Laplace transform and the derivative formula just derived, we obtain
\[
s Y(s) - y(0) = k\,Y(s).
\]
Substituting the initial condition $y(0) = y_0$ yields
\[
s Y(s) - y_0 = k\,Y(s).
\]
We now solve this algebraic equation for $Y(s)$.  Rearranging terms,
\[
s Y(s) - k Y(s) = y_0
\quad\Longrightarrow\quad
(s - k) Y(s) = y_0
\quad\Longrightarrow\quad
Y(s) = \frac{y_0}{s - k}.
\]

To find $y(t)$, we recognize $Y(s)$ as a standard Laplace transform.  From a Laplace transform table, or from a direct computation, we know that
\[
\mathcal{L}\{e^{at}\}(s) = \frac{1}{s-a}, \qquad \text{for } s > a.
\]
Thus
\[
\mathcal{L}\{y_0 e^{kt}\}(s) = y_0 \,\mathcal{L}\{e^{kt}\}(s) = y_0 \cdot \frac{1}{s-k} = \frac{y_0}{s-k}.
\]
By uniqueness of the Laplace transform, it follows that
\[
y(t) = y_0 e^{kt}.
\]

We now briefly interpret this solution.  If $k>0$, then the exponential factor $e^{kt}$ grows without bound as $t\to\infty$, so the solution models exponential growth, as in a population with a birth rate proportional to its size.  If $k<0$, then $e^{kt}$ decays to zero as $t\to\infty$, so the solution models exponential decay, as in radioactive decay or cooling toward the ambient temperature (taken to be zero in this simplified model).

\medskip

\noindent\textbf{Part (ii):} Solve $y'(t) = k\,y(t) + b$, $y(0) = y_0$.

We now consider the equation with a constant source term $b$:
\[
y'(t) = k\,y(t) + b, \qquad y(0) = y_0.
\]
Again we apply the Laplace transform to both sides:
\[
\mathcal{L}\{y'(t)\}(s) = \mathcal{L}\{k\,y(t) + b\}(s).
\]
Using linearity and the derivative formula, we obtain
\[
s Y(s) - y(0) = k Y(s) + \mathcal{L}\{b\}(s).
\]
We substitute $y(0) = y_0$ and note that $\mathcal{L}\{b\}(s) = b\,\mathcal{L}\{1\}(s) = b \cdot \dfrac{1}{s} = \dfrac{b}{s}$.  This gives
\[
s Y(s) - y_0 = k Y(s) + \frac{b}{s}.
\]
We solve for $Y(s)$:
\[
s Y(s) - k Y(s) = y_0 + \frac{b}{s}
\quad\Longrightarrow\quad
(s - k) Y(s) = y_0 + \frac{b}{s}.
\]
Dividing by $(s-k)$, we obtain
\[
Y(s) = \frac{y_0}{s-k} + \frac{b}{s(s-k)}.
\]

To invert this transform, we rewrite the second term by partial fractions.  We seek constants $A$ and $B$ such that
\[
\frac{b}{s(s-k)} = \frac{A}{s} + \frac{B}{s-k}.
\]
Clearing denominators,
\[
b = A(s-k) + B s = (A+B)s - Ak \quad\text{for all } s.
\]
Equating coefficients of like powers of $s$, we obtain the system
\[
A + B = 0, \qquad -Ak = b.
\]
From $-Ak = b$ we have $A = -\dfrac{b}{k}$ (assuming $k \ne 0$; the case $k=0$ can be handled separately below).  Then $B = -A = \dfrac{b}{k}$.  Hence
\[
\frac{b}{s(s-k)} = -\frac{b}{k}\,\frac{1}{s} + \frac{b}{k}\,\frac{1}{s-k}.
\]
Therefore
\[
Y(s) = \frac{y_0}{s-k} - \frac{b}{k}\,\frac{1}{s} + \frac{b}{k}\,\frac{1}{s-k}
= \left( y_0 + \frac{b}{k} \right)\frac{1}{s-k} - \frac{b}{k}\,\frac{1}{s}.
\]

We now invert term by term using the known transforms
\[
\mathcal{L}\{e^{kt}\}(s) = \frac{1}{s-k}, \qquad \mathcal{L}\{1\}(s) = \frac{1}{s}.
\]
Thus
\[
y(t) = \mathcal{L}^{-1}\{Y(s)\}(t)
= \left( y_0 + \frac{b}{k} \right) e^{kt} - \frac{b}{k}\cdot 1
= \left( y_0 + \frac{b}{k} \right)e^{kt} - \frac{b}{k}.
\]

If desired, one can rewrite this more suggestively as
\[
y(t) = -\frac{b}{k} + \left( y_0 + \frac{b}{k} \right) e^{kt}.
\]
This form makes the long-time behavior transparent.  When $k<0$ and $b>0$, the exponential term $e^{kt}$ decays to zero, and the solution approaches the constant value $-\dfrac{b}{k} > 0$ as $t\to\infty$.  Thus the system settles to a steady state determined by the balance between decay and constant input.  When $k>0$ and $b\ne 0$, the exponential term causes $y(t)$ to grow (in magnitude) exponentially, so no finite steady state exists.

Finally, we briefly mention the case $k=0$.  The equation becomes $y'(t) = b$, with solution $y(t) = y_0 + bt$, which is linear growth or decay depending on the sign of $b$.  This case is also easily handled by the Laplace transform, but the partial fraction step above needs minor modification.

\medskip

In summary, this example illustrates the standard Laplace transform method for solving initial value problems: we transform the differential equation, use the initial condition through the derivative formula $\mathcal{L}\{y'\} = sY(s) - y(0)$, solve the resulting algebraic equation for $Y(s)$, and then use known inverse transforms (and partial fractions when needed) to recover $y(t)$.  Even in these simple exponential growth and decay models, the method highlights how initial data and forcing terms (like the constant $b$) are encoded in the algebraic structure of the transform.
\end{solution}

% ===== Example 2: Damped Harmonic Oscillator with Forcing (inquiry-based) =====
\begin{problem}[Damped Harmonic Oscillator with Forcing]
A mass attached to a spring and dashpot (damper) is a standard model for vibrations with energy loss. In many applications the mass is also driven by an external periodic force, such as a motor or an oscillating field. The resulting equation combines inertia, damping, elasticity, and forcing, and exhibits both transient and steady-state behavior. In this problem you will use the Laplace transform to solve such an initial value problem and to see how the transient and steady-state parts of the motion appear naturally.

Consider a mass whose displacement from equilibrium is $x(t)$ for $t \ge 0$. The motion is governed by
\[
m x''(t) + c x'(t) + k x(t) = F_0 \sin(\omega t), \qquad t>0,
\]
with initial conditions
\[
x(0) = 0, \qquad x'(0) = 0.
\]
Here $m>0$ is the mass, $c>0$ is the damping coefficient, $k>0$ is the spring constant, and $F_0 \sin(\omega t)$ is a sinusoidal driving force of amplitude $F_0$ and angular frequency $\omega>0$.

\smallskip

(a) Rewrite the equation in a more convenient ``standard form'' by dividing through by $m$ and introducing the parameters
\[
2\gamma = \frac{c}{m}, \qquad \omega_0^2 = \frac{k}{m}.
\]
What is the resulting differential equation for $x(t)$? State clearly the corresponding initial conditions in this notation.

\medskip

(b) Let $X(s) = \mathcal{L}\{x(t)\}(s)$ be the Laplace transform of $x(t)$. Take the Laplace transform of both sides of your standard-form equation from part (a), and use the formulas
\[
\mathcal{L}\{x'(t)\}(s) = s X(s) - x(0), \qquad
\mathcal{L}\{x''(t)\}(s) = s^2 X(s) - s x(0) - x'(0),
\]
together with $\mathcal{L}\{\sin(\omega t)\}(s) = \dfrac{\omega}{s^2+\omega^2}$.
Derive an algebraic equation for $X(s)$ and solve explicitly for $X(s)$.

\emph{Hint:} After substituting the initial conditions from part (a), every term on the left-hand side should contain a factor of $X(s)$.

\medskip

(c) Assume that the oscillator is \emph{underdamped}, that is,
\[
0 < \gamma < \omega_0.
\]
Introduce the damped natural frequency
\[
\Omega = \sqrt{\omega_0^2 - \gamma^2},
\]
and show that
\[
s^2 + 2\gamma s + \omega_0^2 = (s+\gamma)^2 + \Omega^2.
\]
Rewrite your expression for $X(s)$ in the form
\[
X(s) = \frac{F_0 \,\omega}{(s^2+\omega^2)\big((s+\gamma)^2+\Omega^2\big)}.
\]
Set up a partial fraction decomposition of the form
\[
X(s) = \frac{A s + B}{s^2+\omega^2} \;+\; \frac{C(s+\gamma) + D}{(s+\gamma)^2+\Omega^2},
\]
where $A,B,C,D$ are constants depending on $\gamma,\omega_0,\omega,F_0$.

\emph{Hint:} You do not need to determine $A,B,C,D$ yet; first write down the identity obtained by multiplying both sides by $(s^2+\omega^2)\big((s+\gamma)^2+\Omega^2\big)$ and equating coefficients of powers of $s$.

\medskip

(d) Now determine the constants $A,B,C,D$ by comparing coefficients of $1,s,s^2,s^3$ in the identity from part (c). Then use the inverse Laplace transform and the shift rule
\[
\mathcal{L}^{-1}\left\{\frac{s}{s^2+\alpha^2}\right\}(t) = \cos(\alpha t), \qquad
\mathcal{L}^{-1}\left\{\frac{1}{s^2+\alpha^2}\right\}(t) = \frac{1}{\alpha}\sin(\alpha t),
\]
\[
\mathcal{L}^{-1}\{F(s+\gamma)\}(t) = e^{-\gamma t} f(t) \quad\text{if}\quad \mathcal{L}\{f\}(s) = F(s),
\]
to obtain an explicit formula for $x(t)$ for $t\ge 0$.

Rewrite your answer in the form
\[
x(t) = x_{\text{tr}}(t) + x_{\text{ss}}(t),
\]
where $x_{\text{tr}}(t)$ is a \emph{transient} term and $x_{\text{ss}}(t)$ is a \emph{steady-state} term. Which part decays to zero as $t\to\infty$? Which part persists for large $t$?

\emph{Hint:} After taking inverse Laplace transforms, group together all terms that are multiplied by $e^{-\gamma t}$.

\medskip

(e) (Explorations.)

\begin{enumerate}
  \item Suppose instead that the force is a step input
  \[
  f(t) = F_0\,u(t),
  \]
  where $u(t)$ is the Heaviside unit step function. Without carrying out all the algebra, describe how the Laplace-transform solution would change compared to the sinusoidal forcing case. In particular, what would $F(s)$ be on the right-hand side, and what kinds of time-domain functions (sines, cosines, exponentials, constants, ramps) would you expect in $x(t)$?
  
  \item For the sinusoidal forcing considered above, focus on the steady-state term $x_{\text{ss}}(t)$ and its amplitude. Using your expression for $x_{\text{ss}}(t)$, express it in the form
  \[
  x_{\text{ss}}(t) = R(\omega)\,\sin(\omega t - \delta(\omega)),
  \]
  for some amplitude $R(\omega)$ and phase shift $\delta(\omega)$. How does $R(\omega)$ behave as the damping $\gamma$ becomes small and the driving frequency $\omega$ approaches the undamped natural frequency $\omega_0$ (the resonance phenomenon)?
  
  \emph{Hint:} You may find it helpful to recall that a linear combination of $\sin(\omega t)$ and $\cos(\omega t)$ can be written as a single sinusoid with a phase shift.
\end{enumerate}

\end{problem}

% ===== Example 2: Damped Harmonic Oscillator with Forcing (full solution) =====
\begin{problem}[Damped Harmonic Oscillator with Forcing]
Consider the forced, damped harmonic oscillator
\[
m x''(t) + c x'(t) + k x(t) = F_0 \sin(\omega t), \qquad t>0,
\]
with initial conditions $x(0)=0$, $x'(0)=0$. Introduce the parameters
\[
2\gamma = \frac{c}{m}, \qquad \omega_0^2 = \frac{k}{m},
\]
assume the underdamped case $0<\gamma<\omega_0$, and set $\Omega = \sqrt{\omega_0^2-\gamma^2}$. 

Using the Laplace transform, solve explicitly for $x(t)$ for $t\ge 0$, and decompose the solution into a transient part and a steady-state part. Identify which part governs the long-time behavior and briefly relate your findings to resonance and to the use of the Laplace transform for such problems.
\end{problem}

\begin{solution}
We first rewrite the differential equation in a convenient form. Dividing by $m$ and introducing
\[
2\gamma = \frac{c}{m}, \qquad \omega_0^2 = \frac{k}{m},
\]
we obtain
\[
x''(t) + 2\gamma x'(t) + \omega_0^2 x(t) = \frac{F_0}{m} \sin(\omega t).
\]
Since $\frac{F_0}{m}$ is just a constant amplitude, we rename it as $F_0$ for simplicity and work with
\begin{equation}\label{eq:standard}
x''(t) + 2\gamma x'(t) + \omega_0^2 x(t) = F_0 \sin(\omega t), \qquad x(0)=0,\quad x'(0)=0.
\end{equation}

The central idea of the Laplace-transform method is to convert this initial value problem into an algebraic equation for the Laplace transform $X(s) = \mathcal{L}\{x(t)\}(s)$, solve for $X(s)$, and then invert the transform. A key advantage is that the initial conditions are incorporated automatically.

Taking the Laplace transform of both sides of \eqref{eq:standard}, and using
\[
\mathcal{L}\{x'(t)\}(s) = s X(s) - x(0), \qquad
\mathcal{L}\{x''(t)\}(s) = s^2 X(s) - s x(0) - x'(0),
\]
together with the given initial conditions $x(0)=0$ and $x'(0)=0$, we obtain
\[
\mathcal{L}\{x''\}(s) = s^2 X(s), \qquad \mathcal{L}\{x'\}(s) = s X(s).
\]
Therefore
\[
s^2 X(s) + 2\gamma \, s X(s) + \omega_0^2 X(s)
  = F_0\,\mathcal{L}\{\sin(\omega t)\}(s)
  = F_0\,\frac{\omega}{s^2+\omega^2}.
\]
Factoring $X(s)$ on the left-hand side gives
\[
\bigl(s^2 + 2\gamma s + \omega_0^2\bigr) X(s)
  = F_0\,\frac{\omega}{s^2+\omega^2},
\]
so
\[
X(s) = \frac{F_0\,\omega}{(s^2+2\gamma s+\omega_0^2)(s^2+\omega^2)}.
\]

We now assume the underdamped regime $0<\gamma<\omega_0$ and introduce the damped natural frequency
\[
\Omega = \sqrt{\omega_0^2 - \gamma^2}>0.
\]
Completing the square in the quadratic $s^2+2\gamma s+\omega_0^2$ yields
\[
s^2+2\gamma s+\omega_0^2 = (s+\gamma)^2 + (\omega_0^2-\gamma^2)
  = (s+\gamma)^2 + \Omega^2,
\]
so we can rewrite $X(s)$ as
\[
X(s) = \frac{F_0\,\omega}{(s^2+\omega^2)\bigl((s+\gamma)^2+\Omega^2\bigr)}.
\]

To invert the transform, it is convenient to decompose $X(s)$ into partial fractions adapted to the standard Laplace transform formulas for sines and cosines and to the shift rule. We seek constants $A,B,C,D$ such that
\begin{equation}\label{eq:pf}
X(s) = \frac{A s + B}{s^2+\omega^2} \;+\; \frac{C (s+\gamma) + D}{(s+\gamma)^2+\Omega^2}.
\end{equation}
Multiplying both sides of \eqref{eq:pf} by $(s^2+\omega^2)\bigl((s+\gamma)^2+\Omega^2\bigr)$, we obtain the identity
\[
F_0 \omega = (A s + B)\bigl((s+\gamma)^2+\Omega^2\bigr)
             + \bigl(C (s+\gamma) + D\bigr)(s^2+\omega^2),
\]
which must hold for all $s$. Expanding and equating coefficients of powers of $s$ gives a linear system for $A,B,C,D$.

First, recall that $(s+\gamma)^2+\Omega^2 = s^2 + 2\gamma s + \omega_0^2$. Then
\begin{align*}
(A s + B)(s^2 + 2\gamma s + \omega_0^2)
 &= A s^3 + (2A\gamma + B)s^2 + (A\omega_0^2 + 2B\gamma)s + B\omega_0^2,\\
\bigl(C(s+\gamma) + D\bigr)(s^2+\omega^2)
 &= C s^3 + (C\gamma + D)s^2 + C\omega^2 s + (C\gamma + D)\omega^2.
\end{align*}
Adding these two expressions, we obtain
\[
F_0 \omega
  = (A+C)s^3
    + \bigl(2A\gamma + B + C\gamma + D\bigr)s^2
    + \bigl(A\omega_0^2 + 2B\gamma + C\omega^2\bigr)s
    + \bigl(B\omega_0^2 + (C\gamma + D)\omega^2\bigr).
\]
Since the right-hand side must equal the constant polynomial $F_0 \omega$, the coefficients of $s^3$, $s^2$, and $s$ must vanish, and the constant term must equal $F_0\omega$. Thus we have
\begin{align*}
s^3:& \quad A + C = 0,\\
s^2:& \quad 2A\gamma + B + C\gamma + D = 0,\\
s^1:& \quad A\omega_0^2 + 2B\gamma + C\omega^2 = 0,\\
s^0:& \quad B\omega_0^2 + (C\gamma + D)\omega^2 = F_0 \omega.
\end{align*}

From $A + C = 0$ we obtain $C = -A$. Substituting this into the $s^2$-equation yields
\[
2A\gamma + B - A\gamma + D = 0
 \quad\Longrightarrow\quad
\gamma A + B + D = 0
 \quad\Longrightarrow\quad
D = -\gamma A - B.
\]
Using $C=-A$ in the $s^1$-equation gives
\[
A\omega_0^2 + 2B\gamma - A\omega^2 = 0
 \quad\Longrightarrow\quad
A(\omega_0^2-\omega^2) + 2\gamma B = 0
 \quad\Longrightarrow\quad
B = -\frac{A(\omega_0^2-\omega^2)}{2\gamma}.
\]
Finally, in the constant term we substitute $C=-A$ and $D=-\gamma A - B$:
\[
B\omega_0^2 + (C\gamma + D)\omega^2 = B\omega_0^2 + (-A\gamma -\gamma A - B)\omega^2
  = B(\omega_0^2 - \omega^2) - 2\gamma A \omega^2.
\]
Setting this equal to $F_0 \omega$ gives
\[
B(\omega_0^2 - \omega^2) - 2\gamma A \omega^2 = F_0 \omega.
\]
Sub
stituting the expression for $B$ into this equation yields
\[
-\frac{A(\omega_0^2-\omega^2)^2}{2\gamma} - 2\gamma A \omega^2 = F_0 \omega.
\]
Factor out $A$ and combine terms:
\[
-A\left(\frac{(\omega_0^2-\omega^2)^2}{2\gamma} + 2\gamma \omega^2\right) = F_0 \omega,
\]
so
\[
A\left(\frac{(\omega_0^2-\omega^2)^2 + 4\gamma^2 \omega^2}{2\gamma}\right) = -F_0 \omega.
\]
Thus
\[
A = -\frac{2\gamma F_0 \omega}{(\omega_0^2-\omega^2)^2 + 4\gamma^2 \omega^2}.
\]
Recall that
\[
B = -\frac{A(\omega_0^2-\omega^2)}{2\gamma},\quad
C = -A,\quad
D = -\gamma A - B.
\]
Substituting for $A$ gives
\[
C = \frac{2\gamma F_0 \omega}{(\omega_0^2-\omega^2)^2 + 4\gamma^2 \omega^2},
\]
\[
B = \frac{F_0 \omega(\omega_0^2-\omega^2)}{(\omega_0^2-\omega^2)^2 + 4\gamma^2 \omega^2},
\]
\[
D = \frac{F_0 \omega\bigl(2\gamma^2 - \omega_0^2 + \omega^2\bigr)}
           {(\omega_0^2-\omega^2)^2 + 4\gamma^2 \omega^2}.
\]

For convenience, define the common denominator
\[
\Delta(\omega) = (\omega_0^2-\omega^2)^2 + 4\gamma^2 \omega^2 > 0.
\]

Now we invert the Laplace transform. From \eqref{eq:pf} we have
\[
X(s) = \frac{A s + B}{s^2+\omega^2} \;+\; \frac{C (s+\gamma) + D}{(s+\gamma)^2+\Omega^2},
\]
so by linearity and the standard formulas
\[
\mathcal{L}^{-1}\left\{\frac{s}{s^2+\alpha^2}\right\} = \cos(\alpha t), \quad
\mathcal{L}^{-1}\left\{\frac{1}{s^2+\alpha^2}\right\} = \frac{1}{\alpha}\sin(\alpha t),
\]
and the shift rule
\[
\mathcal{L}^{-1}\{F(s+\gamma)\}(t) = e^{-\gamma t} f(t),
\]
we get
\begin{align*}
x(t)
 &= A\cos(\omega t) + \frac{B}{\omega}\sin(\omega t)
    + C e^{-\gamma t}\cos(\Omega t)
    + \frac{D}{\Omega} e^{-\gamma t}\sin(\Omega t).
\end{align*}
Substituting $A,B,C,D$ in terms of $\Delta(\omega)$, we can separate $x(t)$ naturally into a steady-state and a transient part.

\medskip\noindent
\textbf{Steady-state term.}
The terms involving $\sin(\omega t)$ and $\cos(\omega t)$ (with no exponential factor) form the steady-state solution:
\begin{align*}
x_{\text{ss}}(t)
 &= A\cos(\omega t) + \frac{B}{\omega}\sin(\omega t)\\
 &= \frac{F_0}{\Delta(\omega)}\Bigl[-2\gamma\omega\cos(\omega t)
     + (\omega_0^2-\omega^2)\sin(\omega t)\Bigr].
\end{align*}
This is a sinusoidal motion at the driving frequency $\omega$ that persists for large $t$.

\medskip\noindent
\textbf{Transient term.}
The terms multiplied by $e^{-\gamma t}$ form the transient:
\begin{align*}
x_{\text{tr}}(t)
 &= C e^{-\gamma t}\cos(\Omega t)
    + \frac{D}{\Omega}e^{-\gamma t}\sin(\Omega t)\\
 &= e^{-\gamma t}\left[
      \frac{2\gamma F_0 \omega}{\Delta(\omega)}\cos(\Omega t)
      + \frac{F_0 \omega\bigl(2\gamma^2 - \omega_0^2 + \omega^2\bigr)}
             {\Omega\,\Delta(\omega)}\sin(\Omega t)\right].
\end{align*}
Because of the factor $e^{-\gamma t}$, we have
\[
\lim_{t\to\infty} x_{\text{tr}}(t) = 0.
\]

\medskip\noindent
\textbf{Final solution and long-time behavior.}
Combining the two parts,
\[
x(t) = x_{\text{tr}}(t) + x_{\text{ss}}(t),
\]
where
\[
x_{\text{tr}}(t) = e^{-\gamma t}\left[
      \frac{2\gamma F_0 \omega}{\Delta(\omega)}\cos(\Omega t)
      + \frac{F_0 \omega\bigl(2\gamma^2 - \omega_0^2 + \omega^2\bigr)}
             {\Omega\,\Delta(\omega)}\sin(\Omega t)\right],
\]
\[
x_{\text{ss}}(t) =
\frac{F_0}{\Delta(\omega)}\Bigl[-2\gamma\omega\cos(\omega t)
     + (\omega_0^2-\omega^2)\sin(\omega t)\Bigr].
\]

The \emph{transient} $x_{\text{tr}}(t)$ decays to zero as $t\to\infty$ because of the factor $e^{-\gamma t}$; it represents damped oscillations at the damped natural frequency $\Omega$. The \emph{steady-state} term $x_{\text{ss}}(t)$ persists for large $t$ and oscillates at the driving frequency $\omega$ with a phase shift relative to the forcing.

Writing $x_{\text{ss}}$ in the form
\[
x_{\text{ss}}(t) = R(\omega)\,\sin(\omega t - \delta(\omega)),
\]
one finds
\[
R(\omega)
  = \frac{F_0}{\sqrt{(\omega_0^2-\omega^2)^2 + 4\gamma^2\omega^2}},
\]
so as the damping $\gamma$ becomes small and the driving frequency $\omega$ approaches $\omega_0$, the denominator becomes very small and the amplitude $R(\omega)$ becomes very large: this is the resonance phenomenon.

From the Laplace-transform viewpoint, the poles of $X(s)$ at $s=-\gamma\pm i\Omega$ produce the transient (damped natural oscillations), while the poles at $s=\pm i\omega$ generate the steady-state response at the driving frequency. The method systematically incorporates initial conditions and makes the separation between transient and steady-state behavior transparent in terms of these poles.

\end{solution}

% ===== Example 3: Step and Impulse Responses of a Linear System (inquiry-based) =====
\begin{problem}[Step and Impulse Responses of a Linear System]
Many simple physical systems---for example an \emph{RC circuit} (a resistor and a capacitor in series)---can be modeled by a first-order linear differential equation.  If the input is a voltage source \(f(t)\) that can jump suddenly (a step) or act for a very short time (an impulse), then analyzing the response \(y(t)\) directly in the time domain can be uncomfortable, because of discontinuities and generalized functions (distributions).  The Laplace transform allows us to treat these inputs on essentially the same footing as smooth ones, by turning the differential equation into an algebraic equation.

In this problem we study the system
\[
y'(t) + y(t) = f(t), \qquad t>0, \qquad y(0)=0,
\]
as a model of such a circuit.  We will compute its response to a unit step input and to a Dirac delta impulse, and compare the two.

\medskip

(a) \textbf{Warm-up: the homogeneous system.}  
Consider first the homogeneous equation
\[
y'(t) + y(t) = 0, \qquad t>0, \qquad y(0)=y_0.
\]
Solve this initial value problem explicitly for \(y(t)\).  Describe qualitatively how solutions behave as \(t\to\infty\).

\smallskip
Hint: Separate variables or use the standard method for constant-coefficient linear equations.

\medskip

(b) \textbf{Laplace domain formulation for a general input.}  
Now return to the forced equation
\[
y'(t) + y(t) = f(t), \qquad t>0, \qquad y(0)=0,
\]
where \(f\) is some input that is zero for \(t<0\).

(i) Take the Laplace transform of both sides of the differential equation and use the formula for the Laplace transform of a derivative,
\[
\mathcal{L}\{y'(t)\}(s) = sY(s) - y(0),
\]
to obtain an algebraic relation between \(Y(s)=\mathcal{L}\{y\}(s)\) and \(F(s)=\mathcal{L}\{f\}(s)\).

(ii) Solve this algebraic relation for \(Y(s)\) in terms of \(F(s)\).  Write your answer in the form
\[
Y(s) = H(s)\,F(s)
\]
and identify the function \(H(s)\).  This function \(H\) is called the \emph{transfer function} of the system.

\smallskip
% Hint: Be careful to use the given initial condition \(y(0)=0\).

\medskip

(c) \textbf{Step response.}  
Let the input be a unit step function
\[
f(t) = u(t) = 
\begin{cases}
0, & t<0,\\
1, & t\ge 0.
\end{cases}
\]
This models suddenly applying a constant voltage at time \(t=0\) to a circuit that was previously off.

(i) Compute the Laplace transform \(F(s)=\mathcal{L}\{u(t)\}(s)\).

(ii) Use your expression for \(Y(s)\) from part (b) to find \(Y_{\text{step}}(s)\), the Laplace transform of the corresponding output \(y_{\text{step}}(t)\).

(iii) Invert the Laplace transform to find an explicit formula for \(y_{\text{step}}(t)\) for \(t>0\).  Sketch the graph of \(y_{\text{step}}(t)\).  How does this compare with your qualitative description from part (a)?

\smallskip
Hint: You should obtain a simple rational function in \(s\); use partial fractions and a Laplace table.

\medskip

(d) \textbf{Impulse response.}  
Now let the input be a Dirac delta impulse:
\[
f(t) = \delta(t).
\]
You can think of this as an “infinitely short, unit area” pulse of input at time \(t=0\).

(i) Recall (or accept for this problem) that \(\mathcal{L}\{\delta(t)\}(s) = 1\).  Use this together with your formula from part (b) to find \(Y_{\text{imp}}(s)\), the Laplace transform of the impulse response \(y_{\text{imp}}(t)\).

(ii) Invert the Laplace transform to find \(y_{\text{imp}}(t)\) for \(t>0\).

(iii) Compare your formulas for \(y_{\text{step}}(t)\) and \(y_{\text{imp}}(t)\).  How are they related?  In particular, compute the (distributional) derivative of the step response and relate it to the impulse response.

\smallskip
Hint: Differentiate your formula for \(y_{\text{step}}(t)\) for \(t>0\).  What happens at \(t=0\), where the step function jumps?

\medskip

(e) \textbf{Extensions and “what if” questions.}

(i) Suppose instead the system is
\[
y'(t) + a\,y(t) = f(t), \qquad y(0)=0,
\]
where \(a>0\) is a constant.  Without doing all calculations in full detail, predict how the step and impulse responses will change as \(a\) varies.  What happens in the limits \(a\to 0^+\) and \(a\to\infty\)?

(ii) One of the central ideas of linear systems theory is that, for a linear time-invariant system, the output for a general input \(f\) can be written as a convolution of \(f\) with the impulse response.  Using the relation
\[
Y(s) = H(s)\,F(s)
\]
from part (b), and the Laplace transform identity \(\mathcal{L}\{(f*g)(t)\} = F(s)G(s)\), explain informally why the output \(y\) must be of the form
\[
y(t) = \int_0^t h(t-\tau)\,f(\tau)\,d\tau,
\]
where \(h(t)\) is the impulse response you found in part (d).  (You do not need to give a fully rigorous proof, but outline the main idea.)

\end{problem}

% ===== Example 3: Step and Impulse Responses of a Linear System (full solution) =====
\begin{problem}[Step and Impulse Responses of a Linear System]
Consider the linear system
\[
y'(t) + y(t) = f(t), \qquad t>0, \qquad y(0)=0,
\]
where \(f(t)\) is an input that vanishes for \(t<0\).

(a) Use the Laplace transform to show that for any such input, the Laplace transform \(Y(s)\) of the output satisfies
\[
Y(s) = \frac{1}{s+1}\,F(s),
\]
where \(F(s)\) is the Laplace transform of \(f(t)\).

(b) Let \(f(t) = u(t)\) be the unit step function.  Find the corresponding output \(y_{\mathrm{step}}(t)\) (the step response) for \(t>0\).

(c) Let \(f(t) = \delta(t)\) be the Dirac delta impulse at \(t=0\).  Using \(\mathcal{L}\{\delta(t)\}(s) = 1\), find the corresponding output \(y_{\mathrm{imp}}(t)\) (the impulse response) for \(t>0\).

(d) Show that, for this system, the impulse response is the (distributional) derivative of the step response.  Briefly explain how this relation reflects the way the Laplace transform converts differential equations with discontinuous or impulsive inputs into algebraic equations.

\end{problem}

\begin{solution}
We study a first-order linear time-invariant system driven by inputs that can be discontinuous (a unit step) or even impulsive (a Dirac delta).  The Laplace transform will allow us to treat these inputs in an algebraic way, bypassing some of the technicalities of distributions in the time domain.

\medskip

\noindent\textbf{(a) General Laplace-domain relation.}
We are given
\[
y'(t) + y(t) = f(t), \qquad t>0, \qquad y(0)=0,
\]
with \(f(t)=0\) for \(t<0\).  Let \(Y(s) = \mathcal{L}\{y(t)\}(s)\) and \(F(s) = \mathcal{L}\{f(t)\}(s)\).

Taking the Laplace transform of both sides and using linearity gives
\[
\mathcal{L}\{y'(t)\}(s) + \mathcal{L}\{y(t)\}(s) = \mathcal{L}\{f(t)\}(s).
\]
By the standard formula for the Laplace transform of a derivative,
\[
\mathcal{L}\{y'(t)\}(s) = sY(s) - y(0).
\]
Since \(y(0)=0\), we obtain
\[
sY(s) + Y(s) = F(s).
\]
Factoring \(Y(s)\) yields
\[
(s+1)Y(s) = F(s),
\]
so
\[
Y(s) = \frac{1}{s+1}\,F(s).
\]
The function
\[
H(s) = \frac{1}{s+1}
\]
is the \emph{transfer function} of the system: it describes how each Laplace-frequency component of the input is “scaled” to produce the output.  This exhibits the central Laplace-transform idea for linear time-invariant systems: the differential equation has been converted into a simple algebraic relation between \(Y\) and \(F\).

\medskip

\noindent\textbf{(b) Step response.}
Now take the input to be the unit step function
\[
f(t) = u(t) = 
\begin{cases}
0, & t<0,\\
1, & t\ge 0.
\end{cases}
\]
Its Laplace transform is well known:
\[
F(s) = \mathcal{L}\{u(t)\}(s) = \int_0^\infty e^{-st}\cdot 1\,dt = \frac{1}{s}, \qquad \Re(s)>0.
\]
Using the relation from part (a),
\[
Y_{\text{step}}(s) = \frac{1}{s+1} \cdot \frac{1}{s} = \frac{1}{s(s+1)}.
\]

To find \(y_{\text{step}}(t)\), we invert this Laplace transform.  We decompose by partial fractions:
\[
\frac{1}{s(s+1)} = \frac{A}{s} + \frac{B}{s+1}.
\]
Solving \(1 = A(s+1)+Bs\) gives \(A=1\), \(B=-1\).  Thus
\[
\frac{1}{s(s+1)} = \frac{1}{s} - \frac{1}{s+1}.
\]
Using the standard table,
\[
\mathcal{L}^{-1}\left\{\frac{1}{s}\right\}(t) = 1 \quad\text{and}\quad
\mathcal{L}^{-1}\left\{\frac{1}{s+1}\right\}(t) = e^{-t},
\]
for \(t>0\).  Therefore,
\[
y_{\text{step}}(t) = 1 - e^{-t}, \qquad t>0.
\]

This is the familiar exponential “charging” curve: the response starts at \(y_{\text{step}}(0+)=0\) and increases monotonically toward the steady state value \(1\) as \(t\to\infty\).  This behavior matches the homogeneous solution \(y(t)=Ce^{-t}\) found by solving \(y'+y=0\): the homogeneous part decays exponentially, while the effect of the constant forcing produces a constant steady state.

\medskip

\noindent\textbf{(c) Impulse response.}
Next let the input be a Dirac delta impulse:
\[
f(t) = \delta(t).
\]
The Dirac delta is not an ordinary function but a distribution, characterized by
\[
\int_{-\infty}^{\infty} \delta(t)\,\varphi(t)\,dt = \varphi(0)
\]
for every smooth test function \(\varphi\).  Its Laplace transform is
\[
\mathcal{L}\{\delta(t)\}(s) = \int_0^\infty e^{-st}\,\delta(t)\,dt = 1.
\]

Thus \(F(s) = 1\), and from the general relation \(Y(s)=\frac{1}{s+1}F(s)\) we obtain
\[
Y_{\text{imp}}(s) = \frac{1}{s+1}.
\]
The inverse Laplace transform is again standard:
\[
y_{\text{imp}}(t) = \mathcal{L}^{-1}\left\{\frac{1}{s+1}\right\}(t)
= e^{-t}, \qquad t>0.
\]

This function \(y_{\text{imp}}(t) = e^{-t}\) (for \(t>0\)) is called the \emph{impulse response} of the system.  Physically, it represents the output when the system, initially at rest, is excited by an “infinitely short” unit-area input at \(t=0\).  The system responds with an immediate jump in its state and then relaxes exponentially back toward zero.

\medskip

\noindent\textbf{(d) Relation between step and impulse responses.}
We have found
\[
y_{\text{step}}(t) = 1 - e^{-t}, \qquad t>0,
\]
and
\[
y_{\text{imp}}(t) = e^{-t}, \qquad t>0.
\]

If we differentiate \(y_{\text{step}}(t)\) for \(t>0\), we obtain
\[
\frac{d}{dt} y_{\text{step}}(t) = \frac{d}{dt}(1 - e^{-t}) = e^{-t} = y_{\text{imp}}(t), \qquad t>0.
\]
So for strictly positive times, the impulse response is the ordinary derivative of the step response.

To fully account for the behavior at \(t=0\), we recall that the true step response should be written as
\[
y_{\text{step}}(t) = \bigl(1 - e^{-t}\bigr)u(t),
\]
where \(u(t)\) is the unit step function, so that \(y_{\text{step}}(t)=0\) for \(t<0\).  In the sense of distributions, the derivative of \(u(t)\) is the Dirac delta:
\[
\frac{d}{dt}u(t) = \delta(t).
\]
Using the product rule in the distributional sense,
\[
\frac{d}{dt}\bigl(1 - e^{-t}\bigr)u(t)
= \bigl(1 - e^{-t}\bigr)\delta(t) + e^{-t}u(t).
\]
The term involving \(\delta(t)\) accounts for the immediate jump at \(t=0\), while for \(t>0\) the derivative reduces to \(e^{-t}\).  Thus, up to the singular term at \(t=0\), the impulse response and the derivative of the step response coincide.  More succinctly, one often writes informally
\[
\frac{d}{dt} y_{\text{step}}(t) = y_{\text{imp}}(t),
\]
understanding this as an identity of distributions.

This relation reflects a deeper structural fact: in a linear time-invariant system, the step input \(u(t)\) is the time integral of the impulse input \(\delta(t)\), and the output operator is linear.  Consequently, the output for a step is an integral (in time) of the output for an impulse, and differentiating in time reverses that integration.

In the Laplace domain, this connection appears very naturally.  The step input has Laplace transform \(1/s\), and the impulse input has Laplace transform \(1\).  Thus
\[
Y_{\text{step}}(s) = H(s)\cdot \frac{1}{s} 
\quad\text{and}\quad
Y_{\text{imp}}(s) = H(s)\cdot 1,
\]
with \(H(s)=1/(s+1)\).  Since multiplication by \(s\) in the Laplace domain corresponds to time differentiation in the original domain (again in the distributional sense),
\[
\mathcal{L}\left\{\frac{d}{dt}y_{\text{step}}(t)\right\}(s) 
= sY_{\text{step}}(s) - y_{\text{step}}(0)
= s\left(H(s)\cdot\frac{1}{s}\right) - y_{\text{step}}(0)
= H(s) - y_{\text{step}}(0).
\]
Because the system is initially at rest, \(y_{\text{step}}(0)=0\), and so
\[
\mathcal{L}\left\{\frac{d}{dt}y_{\text{step}}(t)\right\}(s) = H(s) = Y_{\text{imp}}(s).
\]
Thus, at the level of Laplace transforms, the identity “impulse response = derivative of step response” becomes the simple algebraic equality \(H(s) = H(s)\).

\medskip

\noindent\textbf{Conceptual summary.}
This example illustrates a central theme of the Laplace transform in the analysis of linear systems.  The time-domain differential equation
\[
y'(t) + y(t) = f(t)
\]
is turned into the algebraic relation
\[
Y(s) = H(s)F(s)
\]
in the Laplace domain.  Discontinuous inputs (like the step) and distributional inputs (like the impulse) are naturally accommodated through their Laplace transforms, \(\mathcal{L}\{u\}(s)=1/s\) and \(\mathcal{L}\{\delta\}(s)=1\).  The step and impulse responses emerge as particularly important cases, and their close relationship (the impulse response is the derivative of the step response) becomes transparent both in time and in the Laplace domain.  More generally, for any linear time-invariant system, the Laplace transform provides a powerful framework in which solving differential equations with complicated inputs reduces to algebraic manipulation of transforms.

\end{solution}

% ===== Example 4: Convolution and the Laplace Transform (inquiry-based) =====
\begin{problem}[Convolution and the Laplace Transform]
In many physical systems (such as electrical circuits, mechanical vibration systems, and thermal diffusion problems), the system is modeled as \emph{linear} and \emph{time-invariant} (LTI). For such systems, a short “impulse” input at time $t=0$ produces a characteristic response called the \emph{impulse response}. If $h(t)$ denotes this impulse response, then the output $y(t)$ corresponding to a general input $f(t)$ is given by the \emph{convolution} of $h$ and $f$. In this problem we explore how to compute this convolution directly in the time domain and then see how the Laplace transform simplifies the computation by turning convolution into multiplication.

Consider a causal LTI system with impulse response
\[
h(t) = e^{-t}u(t),
\]
where $u(t)$ is the unit step function. For any input $f(t)$, the output is
\[
y(t) = (h * f)(t) = \int_0^t h(t-\tau)\, f(\tau)\, d\tau
      = \int_0^t e^{-(t-\tau)} f(\tau)\, d\tau,
\]
for $t \ge 0$.

\smallskip

(a) Suppose the input is a unit step, $f(t) = u(t)$. Write the convolution integral for $y(t)$ explicitly and compute $y(t)$ as a function of $t$. Then, sketch both $f(t)$ and $y(t)$ on the same axes for $t \ge 0$.  
\emph{Hint:} Remember that $u(t) = 1$ for $t>0$, so you may drop the $u(t)$ factor inside the integral once the limits are fixed.

\smallskip

(b) Now analyze the same step input using the Laplace transform. Compute $H(s) = \mathcal{L}\{h(t)\}(s)$ and $F(s) = \mathcal{L}\{f(t)\}(s)$, and form the product
\[
Y(s) = H(s)F(s).
\]
Then use partial fractions to find an explicit formula for $Y(s)$ and invert the Laplace transform to recover $y(t)$. Compare your result with part (a).  
\emph{Hint:} Recall that $\mathcal{L}\{e^{-t}u(t)\}(s) = \dfrac{1}{s+1}$ and $\mathcal{L}\{u(t)\}(s) = \dfrac{1}{s}$ for $\operatorname{Re}(s)$ sufficiently large.

\smallskip

(c) Next, consider a \emph{ramp} input
\[
f(t) = t\,u(t).
\]
First, write the convolution integral
\[
y(t) = \int_0^t e^{-(t-\tau)} \,\tau\, d\tau
\]
for this new input. Simplify the integrand and set up a plan to evaluate the integral. You may attempt to compute the integral directly.  
\emph{Hint:} Factor out $e^{-t}$ to obtain $y(t) = e^{-t}\displaystyle\int_0^t e^{\tau}\tau\, d\tau$. For the remaining integral, consider integration by parts.

\smallskip

(d) Now use the Laplace transform to find the response to the ramp input. Compute $F(s) = \mathcal{L}\{t\,u(t)\}(s)$, then form
\[
Y(s) = H(s)F(s),
\]
and use partial fractions to find $Y(s)$. Invert the Laplace transform to obtain an explicit expression for $y(t)$ in the time domain. If you computed the integral in part (c), check that your answer here matches the time-domain computation.  
\emph{Hint:} You may recall $\mathcal{L}\{t\,u(t)\}(s) = \dfrac{1}{s^2}$. For partial fractions, try the ansatz
\[
\frac{1}{s^2(s+1)} = \frac{A}{s} + \frac{B}{s^2} + \frac{C}{s+1}.
\]

\smallskip

(e) \textbf{Extensions and “what if” questions.}
\begin{enumerate}
  \item[(i)] Suppose instead that the system has impulse response
    \[
    h_\alpha(t) = e^{-\alpha t}u(t)
    \]
    for some constant $\alpha > 0$. Compute $H_\alpha(s) = \mathcal{L}\{h_\alpha(t)\}(s)$. For a step input $f(t) = u(t)$, use the Laplace transform to find the corresponding output $y_\alpha(t)$. How does the parameter $\alpha$ affect the \emph{speed} of the system's response?
    \item[(ii)] The example above suggests that the Laplace transform turns convolution into multiplication. Starting from the definition
    \[
    (h*f)(t) = \int_0^t h(t-\tau)f(\tau)\, d\tau,
    \]
    and the definition of the Laplace transform, sketch how one might prove the \emph{convolution theorem}
    \[
    \mathcal{L}\{h*f\}(s) = \mathcal{L}\{h\}(s)\,\mathcal{L}\{f\}(s)
    \]
    for reasonably nice functions supported on $[0,\infty)$.  
    \emph{Hint:} Write
    \[
    \mathcal{L}\{h*f\}(s) = \int_0^\infty e^{-st} \left( \int_0^t h(t-\tau)f(\tau)\, d\tau \right) dt
    \]
    and consider interchanging the order of integration in the $(t,\tau)$-plane.
\end{enumerate}

\end{problem}

% ===== Example 4: Convolution and the Laplace Transform (full solution) =====
\begin{problem}[Convolution and the Laplace Transform]
Consider a causal LTI system with impulse response $h(t) = e^{-t}u(t)$, where $u(t)$ is the unit step function, and output
\[
y(t) = (h*f)(t) = \int_0^t e^{-(t-\tau)} f(\tau)\, d\tau
\]
for an input $f(t)$.

\begin{enumerate}
  \item[(a)] For the step input $f(t) = u(t)$, compute $y(t)$ directly from the convolution integral.
  \item[(b)] For the same step input, compute $y(t)$ using Laplace transforms: find $H(s)$, $F(s)$, and $Y(s) = H(s)F(s)$, then invert the Laplace transform. Verify that this agrees with part (a).
  \item[(c)] For the ramp input $f(t) = t\,u(t)$, compute $y(t)$ using Laplace transforms. (You may optionally verify your answer by direct evaluation of the convolution integral.)
\end{enumerate}
Explain briefly how this example illustrates the use of the Laplace transform to handle convolution in LTI systems.

\end{problem}

\begin{solution}
We are given a causal LTI system with impulse response
\[
h(t) = e^{-t}u(t),
\]
so for $t \ge 0$ the output corresponding to an input $f$ is
\[
y(t) = (h*f)(t) = \int_0^t e^{-(t-\tau)} f(\tau)\, d\tau.
\]
This is the standard convolution representation for a causal LTI system.

\medskip

\noindent\textbf{(a) Step input: direct convolution.}

Let $f(t) = u(t)$, the unit step function. For $t > 0$, we have $f(\tau) = 1$ for $0 \le \tau \le t$, so the convolution integral becomes
\[
y(t) = \int_0^t e^{-(t-\tau)}\cdot 1\, d\tau
     = \int_0^t e^{-(t-\tau)}\, d\tau.
\]
It is convenient to pull out the factor depending only on $t$:
\[
e^{-(t-\tau)} = e^{-t} e^{\tau},
\]
so
\[
y(t) = e^{-t} \int_0^t e^{\tau}\, d\tau
     = e^{-t}\bigl[e^{\tau}\bigr]_{\tau=0}^{\tau=t}
     = e^{-t}(e^t - 1)
     = 1 - e^{-t}.
\]
Since the system is causal and the input is zero for $t<0$, we write
\[
y(t) = (1 - e^{-t})u(t).
\]

\medskip

\noindent\textbf{(b) Step input: Laplace transform method.}

We now compute the same response using Laplace transforms. First compute the Laplace transforms of $h$ and $f$.

For $h(t) = e^{-t}u(t)$, we recall the standard Laplace transform
\[
\mathcal{L}\{e^{at}u(t)\}(s) = \frac{1}{s-a}, \quad \operatorname{Re}(s) > \operatorname{Re}(a).
\]
Here $a = -1$, so
\[
H(s) = \mathcal{L}\{h(t)\}(s) = \mathcal{L}\{e^{-t}u(t)\}(s) = \frac{1}{s+1}.
\]
For the step input $f(t) = u(t)$ we have
\[
F(s) = \mathcal{L}\{u(t)\}(s) = \frac{1}{s}, \quad \operatorname{Re}(s) > 0.
\]
The Laplace transform of the output, by the convolution theorem, is the product
\[
Y(s) = H(s)F(s) = \frac{1}{s+1}\cdot \frac{1}{s} = \frac{1}{s(s+1)}.
\]

To invert this, we perform a partial fraction decomposition:
\[
\frac{1}{s(s+1)} = \frac{A}{s} + \frac{B}{s+1}.
\]
Multiplying both sides by $s(s+1)$ gives
\[
1 = A(s+1) + Bs = (A+B)s + A.
\]
Matching coefficients, we obtain
\[
A = 1, \qquad A + B = 0 \quad \Rightarrow \quad B = -1.
\]
Hence
\[
Y(s) = \frac{1}{s} - \frac{1}{s+1}.
\]
Taking inverse Laplace transforms term by term, we use
\[
\mathcal{L}^{-1}\left\{\frac{1}{s}\right\}(t) = u(t),
\qquad
\mathcal{L}^{-1}\left\{\frac{1}{s+1}\right\}(t) = e^{-t}u(t).
\]
Therefore,
\[
y(t) = \mathcal{L}^{-1}\{Y(s)\}(t)
     = \left(1 - e^{-t}\right)u(t).
\]
This agrees exactly with the result obtained in part (a). This illustrates how the convolution integral can be replaced by an algebraic product in the Laplace domain, often simplifying the computation.

\medskip

\noindent\textbf{(c) Ramp input: Laplace transform method.}

Now let $f(t) = t\,u(t)$, the ramp function. We again use the Laplace transform to find the output.

First, recall the Laplace transform of $t\,u(t)$:
\[
F(s) = \mathcal{L}\{t\,u(t)\}(s) = \frac{1}{s^2}, \quad \operatorname{Re}(s) > 0.
\]
We already have
\[
H(s) = \frac{1}{s+1}.
\]
Hence the Laplace transform of the output is
\[
Y(s) = H(s)F(s) = \frac{1}{s+1} \cdot \frac{1}{s^2} = \frac{1}{s^2(s+1)}.
\]

We decompose this into partial fractions. We seek constants $A, B, C$ such that
\[
\frac{1}{s^2(s+1)} = \frac{A}{s} + \frac{B}{s^2} + \frac{C}{s+1}.
\]
Multiplying by $s^2(s+1)$ yields
\[
1 = A s(s+1) + B(s+1) + C s^2.
\]
Expanding the right-hand side:
\[
A s(s+1) = A(s^2 + s), \quad B(s+1) = Bs + B,
\]
so
\[
1 = (A + C)s^2 + (A + B)s + B.
\]
Matching coefficients of powers of $s$ gives the system
\[
\begin{cases}
A + C = 0, \\
A + B = 0, \\
B = 1.
\end{cases}
\]
From $B = 1$ we get $A = -1$, and then $C = -A = 1$. Thus
\[
Y(s) = -\,\frac{1}{s} + \frac{1}{s^2} + \frac{1}{s+1}.
\]

We now invert term by term:
\[
\mathcal{L}^{-1}\left\{-\frac{1}{s}\right\}(t) = -u(t), \quad
\mathcal{L}^{-1}\left\{\frac{1}{s^2}\right\}(t) = t\,u(t), \quad
\mathcal{L}^{-1}\left\{\frac{1}{s+1}\right\}(t) = e^{-t}u(t).
\]
Therefore
\[
y(t) = \bigl(-1 + t + e^{-t}\bigr)u(t)
     = (t - 1 + e^{-t})u(t).
\]

If we wish, we can confirm this by direct evaluation of the convolution integral. Starting from
\[
y(t) = \int_0^t e^{-(t-\tau)} \,\tau\, d\tau,
\]
we write
\[
e^{-(t-\tau)} = e^{-t}e^{\tau}
\]
to obtain
\[
y(t) = e^{-t}\int_0^t \tau e^{\tau}\, d\tau.
\]
An antiderivative of $\tau e^{\tau}$ is $e^{\tau}(\tau - 1)$, so
\[
\int_0^t \tau e^{\tau}\, d\tau
= \bigl[e^{\tau}(\tau - 1)\bigr]_{\tau=0}^{\tau=t}
= e^{t}(t-1) - e^{0}(-1)
= e^{t}(t-1) + 1.
\]
Multiplying by $e^{-t}$ gives
\[
y(t) = e^{-t}\bigl(e^{t}(t-1) + 1\bigr) = (t-1) + e^{-t}.
\]
Including causality via $u(t)$, this becomes
\[
y(t) = (t - 1 + e^{-t})u(t),
\]
which agrees with the Laplace-transform computation.

\medskip

\noindent\textbf{Discussion and connection to Laplace transforms.}

This example captures several central ideas of the Laplace transform in the study of LTI systems:

\begin{itemize}
  \item In the time domain, the output of a causal LTI system is a convolution $y = h * f$. Direct evaluation of the convolution integral can be straightforward for simple inputs (such as the step) but becomes more cumbersome for more complicated inputs (such as the ramp).
  \item The Laplace transform converts the convolution operation into multiplication: $\mathcal{L}\{h*f\} = \mathcal{L}\{h\}\,\mathcal{L}\{f\} = H(s)F(s)$. This allows us to replace an integral computation by algebraic manipulations in the $s$-domain.
  \item Once $Y(s)$ is found, we recover $y(t)$ via the inverse Laplace transform, often using partial fraction decompositions and standard transform pairs.
\end{itemize}

Thus, the Laplace transform provides a powerful method for analyzing the response of LTI systems to various inputs, especially when direct convolution in the time domain would be difficult.

\end{solution}

% ===== Example 5: Using Laplace Transforms for a Simple PDE: The Heat Equation on a Half-Line (inquiry-based) =====
\begin{problem}[Using Laplace Transforms for a Simple PDE: The Heat Equation on a Half-Line]
Consider a very long, thin rod occupying the half-line $x>0$. At time $t=0$ the rod is at zero temperature everywhere. For $t>0$ we suddenly clamp the end at $x=0$ to a fixed temperature of $1$ (say, in suitable nondimensional units), while the rest of the rod is initially cold. Heat then begins to flow from the heated end into the rod. We model this using the one-dimensional heat equation with thermal diffusivity $\kappa>0$ on the half-line.

We will use the Laplace transform in time to reduce this initial–boundary value problem to an ordinary differential equation in the spatial variable $x$. Solving this ordinary differential equation and then inverting the Laplace transform will yield an explicit formula for the temperature $u(x,t)$ as a function of position and time.

The problem is:
\[
\begin{cases}
u_t(x,t) = \kappa\,u_{xx}(x,t), & x>0,\ t>0,\\[4pt]
u(0,t) = 1, & t>0,\\[4pt]
u(x,0) = 0, & x>0,\\[4pt]
u(x,t)\ \text{remains bounded as } x\to +\infty, & t>0.
\end{cases}
\]

\medskip

(a) Before doing any calculations, think qualitatively about the solution.  
Describe in a few sentences what you expect the temperature profile $u(x,t)$ to look like for:
\begin{itemize}
    \item very small times $t\downarrow 0$, and
    \item very large times $t\to\infty$,
\end{itemize}
at a fixed position $x>0$. In particular, do you expect $u(x,t)$ to increase or decrease in time at a fixed $x$, and is there any limiting temperature as $t\to\infty$ for fixed $x$?

\medskip

(b) We now introduce the Laplace transform in time. For each fixed $x>0$, define
\[
U(x,s) = \mathcal{L}_t\{u(x,t)\}(s) = \int_0^\infty e^{-st}u(x,t)\,dt,\qquad s>0.
\]
Apply the Laplace transform with respect to $t$ to the heat equation and to the initial condition.  

Carefully justify each step and show that $U$ satisfies a second-order ordinary differential equation in $x$ of the form
\[
\kappa\,U_{xx}(x,s) - s\,U(x,s) = 0,\qquad x>0,\ s>0.
\]
What boundary conditions in $x$ does $U(x,s)$ satisfy?

Hint: Use the property $\mathcal{L}\{u_t\}(s) = s U(x,s) - u(x,0)$ and recall that the Laplace transform in $t$ does not affect the variable $x$.

\medskip

(c) Now solve the ordinary differential equation in $x$ found in part (b).  

(i) Solve the homogeneous equation
\[
\kappa\,U_{xx} - s\,U = 0
\]
for fixed $s>0$ and find the general solution $U(x,s)$.

(ii) Use the boundary condition at $x=0$ and the requirement that $U(x,s)$ remain bounded as $x\to\infty$ to determine the constants in your general solution and obtain an explicit formula for $U(x,s)$.

Hint: The characteristic equation for the ODE is quadratic in the spatial growth rate, and you should find exponentials of the form $e^{\pm \sqrt{s/\kappa}\,x}$. Which exponential is compatible with boundedness as $x\to\infty$?

\medskip

(d) We now invert the Laplace transform in order to recover $u(x,t)$.

(i) Show that your expression for $U(x,s)$ from part (c) can be written in the form
\[
U(x,s) = \frac{1}{s}\,\exp\!\bigl(-x\sqrt{s/\kappa}\,\bigr).
\]

(ii) By consulting a Laplace transform table (or otherwise), find the inverse Laplace transform of
\[
\frac{1}{s}\,\exp\!\bigl(-a\sqrt{s}\,\bigr),\qquad a>0.
\]
Express your answer in terms of the complementary error function
\[
\operatorname{erfc}(z) = \frac{2}{\sqrt{\pi}}\int_z^\infty e^{-y^2}\,dy.
\]

(iii) Use the result from (ii) with a suitable choice of $a$ to write an explicit formula for $u(x,t)$. Briefly check that your formula is consistent with your qualitative expectations from part (a) (for example, check what happens when $t\downarrow 0$ and when $x\to 0$).

% Hint: A standard transform pair is
% \[
% \mathcal{L}^{-1}\!\left\{\frac{1}{s}\,e^{-a\sqrt{s}}\right\}(t)
% = \operatorname{erfc}\!\left(\frac{a}{2\sqrt{t}}\right),\quad a>0.
% \]

\medskip

(e) Exploratory questions.

(i) Suppose instead that the boundary temperature is not a step, but decays exponentially in time:
\[
u(0,t) = e^{-bt},\qquad b>0,\ t>0,
\]
while $u(x,0)=0$ as before. How would the transformed problem for $U(x,s)$ change? Write down the new boundary condition for $U(0,s)$ and the corresponding formula for $U(x,s)$ (do not invert the transform).

(ii) Briefly discuss how you might modify the Laplace transform approach if, in addition to the boundary condition $u(0,t)=1$, you also had a nonzero initial temperature distribution $u(x,0)=f(x)$ for a given function $f$. What changes in the transformed ordinary differential equation and its right-hand side?

Hint: Look back at the formula $\mathcal{L}\{u_t\}(s) = s U(x,s) - u(x,0)$ and think about how a nonzero $f(x)$ appears in the transformed equation.
\end{problem}

% ===== Example 5: Using Laplace Transforms for a Simple PDE: The Heat Equation on a Half-Line (full solution) =====
\begin{problem}[Using Laplace Transforms for a Simple PDE: The Heat Equation on a Half-Line]
Consider the heat equation on the half-line $x>0$ with thermal diffusivity $\kappa>0$:
\[
u_t(x,t) = \kappa\,u_{xx}(x,t),\qquad x>0,\ t>0,
\]
subject to the boundary and initial conditions
\[
u(0,t) = 1,\quad t>0;\qquad u(x,0)=0,\quad x>0;
\]
and the requirement that $u(x,t)$ remain bounded as $x\to\infty$ for each fixed $t>0$.
Using the Laplace transform in time, solve this initial–boundary value problem and express the solution $u(x,t)$ explicitly in terms of the complementary error function $\operatorname{erfc}$.
\end{problem}

\begin{solution}
We are solving the heat equation on the semi-infinite rod with a step change in boundary temperature at $x=0$. The key idea is to apply the Laplace transform in the time variable, which turns the time derivative into multiplication by $s$ and incorporates the initial condition algebraically. This reduces the partial differential equation to an ordinary differential equation in $x$ for each transform parameter $s>0$. We then solve this ordinary differential equation using standard methods and finally invert the Laplace transform using a known transform pair involving the complementary error function.

\medskip

\noindent\textbf{1. Laplace transform in time.}
For each fixed $x>0$, define the Laplace transform of $u(x,t)$ with respect to $t$ by
\[
U(x,s) = \mathcal{L}_t\{u(x,t)\}(s) = \int_0^\infty e^{-st}u(x,t)\,dt,\qquad s>0.
\]
We now transform the partial differential equation and the boundary and initial conditions.

First, we use the standard property of the Laplace transform for time derivatives:
\[
\mathcal{L}_t\{u_t(x,t)\}(s) = s\,U(x,s) - u(x,0).
\]
Since $u(x,0)=0$ for all $x>0$, this reduces to
\[
\mathcal{L}_t\{u_t(x,t)\}(s) = s\,U(x,s).
\]
Next, observe that the Laplace transform in $t$ does not affect derivatives with respect to $x$, so
\[
\mathcal{L}_t\{u_{xx}(x,t)\}(s) = U_{xx}(x,s).
\]
Applying $\mathcal{L}_t$ to the heat equation $u_t = \kappa u_{xx}$ yields
\[
s\,U(x,s) = \kappa\,U_{xx}(x,s),\qquad x>0,\ s>0.
\]
We may rewrite this as the second-order ordinary differential equation
\[
\kappa\,U_{xx}(x,s) - s\,U(x,s) = 0,\qquad x>0,\ s>0.
\]

Now we transform the boundary condition at $x=0$. We have $u(0,t) = 1$ for $t>0$, so
\[
U(0,s) = \mathcal{L}_t\{u(0,t)\}(s) = \mathcal{L}_t\{1\}(s) = \frac{1}{s}.
\]
Finally, the requirement that $u(x,t)$ remain bounded as $x\to\infty$ for each $t$ implies that, for each $s>0$, $U(x,s)$ should remain bounded as $x\to\infty$.

Thus, for each fixed $s>0$, we must solve the boundary value problem
\[
\begin{cases}
\kappa\,U_{xx}(x,s) - s\,U(x,s) = 0, & x>0,\\[4pt]
U(0,s) = \dfrac{1}{s},\\[4pt]
U(x,s)\ \text{bounded as } x\to\infty.
\end{cases}
\]

\medskip

\noindent\textbf{2. Solving the transformed ODE in $x$.}
Fix $s>0$ and consider the ordinary differential equation
\[
\kappa\,U_{xx} - s\,U = 0.
\]
This is a constant-coefficient linear ODE. Its characteristic equation is
\[
\kappa r^2 - s = 0 \quad\Longrightarrow\quad r^2 = \frac{s}{\kappa}.
\]
Hence
\[
r = \pm \sqrt{\frac{s}{\kappa}}.
\]
The general solution is therefore
\[
U(x,s) = A(s)\,e^{\sqrt{s/\kappa}\,x} + B(s)\,e^{-\sqrt{s/\kappa}\,x},
\]
where $A(s)$ and $B(s)$ are functions of the transform variable $s$ to be determined from the boundary conditions.

The boundedness condition as $x\to\infty$ forces $A(s)=0$, because $e^{\sqrt{s/\kappa} x}$ grows exponentially as $x\to\infty$ for each $s>0$. Thus
\[
U(x,s) = B(s)\,e^{-\sqrt{s/\kappa}\,x}.
\]
Imposing the boundary condition at $x=0$,
\[
U(0,s) = B(s) = \frac{1}{s},
\]
we obtain
\[
U(x,s) = \frac{1}{s}\,e^{-\sqrt{s/\kappa}\,x}.
\]
This is the Laplace transform (in $t$) of the solution we seek:
\[
U(x,s) = \mathcal{L}_t\{u(x,t)\}(s) = \frac{1}{s}\,\exp\!\bigl(-x\sqrt{s/\kappa}\,\bigr).
\]

\medskip

\noindent\textbf{3. Inverse Laplace transform and appearance of $\operatorname{erfc}$.}
To find $u(x,t)$, we must invert the Laplace transform. For each fixed $x>0$, we can view $U(x,s)$ as a function of $s$:
\[
U(x,s) = \frac{1}{s}\,\exp\!\left(-x\sqrt{\frac{s}{\kappa}}\right).
\]
Introduce the parameter
\[
a = \frac{x}{\sqrt{\kappa}},
\]
so that
\[
U(x,s) = \frac{1}{s}\,e^{-a\sqrt{s}}.
\]

A standard Laplace transform pair (which can be found in most tables or derived by an explicit computation using the Gaussian integral) is
\[
\mathcal{L}_t^{-1}\!\left\{\frac{1}{s}\,e^{-a\sqrt{s}}\right\}(t)
= \operatorname{erfc}\!\left(\frac{a}{2\sqrt{t}}\right),\qquad a>0,\ t>0,
\]
where the complementary error function is defined by
\[
\operatorname{erfc}(z)
= \frac{2}{\sqrt{\pi}}\int_z^\infty e^{-y^2}\,dy.
\]

Using this transform pair with $a=x/\sqrt{\kappa}$, we obtain
\[
u(x,t)
= \mathcal{L}_t^{-1}\{U(x,s)\}(t)
= \mathcal{L}_t^{-1}\!\left\{\frac{1}{s}\,e^{-(x/\sqrt{\kappa})\sqrt{s}}\right\}(t)
= \operatorname{erfc}\!\left(\frac{x/\sqrt{\kappa}}{2\sqrt{t}}\right).
\]
Simplifying the argument of $\operatorname{erfc}$ gives
\[
u(x,t) = \operatorname{erfc}\!\left(\frac{x}{2\sqrt{\kappa t}}\right),\qquad x>0,\ t>0.
\]

Thus the temperature in the semi-infinite rod is given explicitly by
\[
\boxed{\,u(x,t) = \operatorname{erfc}\!\left(\frac{x}{2\sqrt{\kappa t}}\right)\,}.
\]

\medskip

\noindent\textbf{4. Checking consistency with conditions and qualitative behavior.}
We briefly check that this formula is consistent with the initial and boundary conditions and with physical intuition.

\emph{Boundary condition at $x=0$.}  
At $x=0$ we have
\[
u(0,t) = \operatorname{erfc}\!\left(0\right) = 1,
\]
because
\[
\operatorname{erfc}(0) = \frac{2}{\sqrt{\pi}}\int_0^\infty e^{-y^2}\,dy = 1.
\]
This matches the prescribed boundary temperature $u(0,t)=1$.

\emph{Initial condition as $t\downarrow 0$.}  
Fix $x>0$ and let $t\downarrow 0$. Then $x/(2\sqrt{\kappa t})\to +\infty$, and since $\operatorname{erfc}(z)\to 0$ as $z\to+\infty$, we obtain
\[
\lim_{t\downarrow 0}u(x,t)
= \lim_{t\downarrow 0}\operatorname{erfc}\!\left(\frac{x}{2\sqrt{\kappa t}}\right) = 0,
\]
in agreement with the initial condition $u(x,0)=0$.

\emph{Behavior for large $t$ at fixed $x$.}  
For each fixed $x>0$, as $t\to\infty$ we have $x/(2\sqrt{\kappa t})\to 0$, and thus
\[
\lim_{t\to\infty}u(x,t)
= \lim_{t\to\infty}\operatorname{erfc}\!\left(\frac{x}{2\sqrt{\kappa t}}\right) = 1.
\]
Physically, this means that as time becomes very large, the entire rod (at any fixed finite distance from $x=0$) tends to the boundary temperature $1$. This is exactly what we expect: given enough time, the heat has diffused into the rod and equilibrated locally to the boundary temperature.

\emph{Boundedness as $x\to\infty$.}  
For any fixed $t>0$, as $x\to\infty$ the argument $x/(2\sqrt{\kappa t})\to\infty$ and hence $u(x,t)\to 0$. Thus the temperature far away from the heated end remains small (and in fact tends to zero at each fixed time), which is compatible with the half-infinite geometry and the finite speed at which thermal influence propagates in this diffusive model.

\medskip

\noindent\textbf{5. Conceptual summary.}
This example illustrates a central idea in the use of the Laplace transform for partial differential equations: transforming in time turns the time derivative into an algebraic factor $s$, while initial conditions appear as simple additional terms. The heat equation $u_t = \kappa u_{xx}$ becomes, for each $s>0$, a boundary value ordinary differential equation in the spatial variable $x$,
\[
\kappa U_{xx} - s U = 0,
\]
with transformed boundary data obtained by taking the Laplace transform of the original time-dependent boundary condition. Solving this ordinary differential equation is straightforward, and the remaining step is to invert the Laplace transform, which can often be done using tables or known transform pairs. In this case, the inversion naturally introduces the complementary error function, reflecting the Gaussian nature of heat diffusion. Thus the Laplace transform serves as a powerful tool to convert an initial–boundary value problem for a parabolic PDE into a more tractable sequence of ordinary differential problems.
\end{solution}

\section{From Differential to Algebraic Equations with FT, FS and LT}
% --- Narrative plan (auto-generated) ---
% This section develops the central idea that Fourier transforms, Fourier series, and Laplace transforms convert differential equations into algebraic equations in transform space. By learning to move back and forth between the physical domain (time, space) and the transform domain (frequency, complex parameter), we turn derivatives into simple multipliers. This creates a powerful and systematic route for solving initial value problems, boundary value problems, and forced systems.
%
% The methods here are essential in applied mathematics because many partial differential equations, such as the heat and wave equations, become tractable only after a suitable transform has diagonalized the differential operator. Dynamical systems and control problems are often most clearly understood through their frequency-domain or Laplace-domain representations, where stability and resonance can be read off from poles and spectra. The same ideas connect naturally to complex analysis through contour integration and the inversion formulas, and to linear algebra through the interpretation of transforms as changes of basis in infinite-dimensional function spaces.
%
% Throughout this section, we will proceed from simple ordinary differential equations to more elaborate partial differential equations, beginning with concrete, low-dimensional examples and gradually introducing the standard techniques: applying a transform, solving the resulting algebraic equation, and inverting the transform. Along the way we highlight the parallels between Fourier series, Fourier transforms, and Laplace transforms, and show how they provide complementary perspectives on the same underlying principle: derivatives become multiplication, and complicated operators become simple in the right representation.

% ===== Example 1: First-Order Linear ODE Solved by Laplace Transform (inquiry-based) =====
\begin{problem}[First-Order Linear ODE Solved by Laplace Transform]
In many simple electrical circuits, such as an $RC$ circuit, the voltage $v(t)$ across a capacitor satisfies a first-order linear ordinary differential equation. If a constant input voltage $V_0$ is suddenly applied at time $t=0$, then (after choosing appropriate units so that $RC=1$) the voltage across the capacitor satisfies
\[
v'(t) + v(t) = V_0,\qquad t>0,
\]
with some given initial voltage $v(0)=v_0$ on the capacitor. In this problem you will use the Laplace transform to convert this differential equation into an algebraic equation in the Laplace variable $s$, solve it there, and then invert the transform to find $v(t)$.

(a) Recall that the Laplace transform of a function $f(t)$ defined for $t\geq 0$ is
\[
\mathcal{L}\{f\}(s) = F(s) = \int_{0}^{\infty} e^{-st} f(t)\,dt,
\]
for those values of $s$ where the integral converges. Starting from this definition, compute the Laplace transform of the derivative $v'(t)$ in terms of $V(s) = \mathcal{L}\{v\}(s)$ and the initial value $v(0)$.

\emph{Hint:} Write $\mathcal{L}\{v'\}(s) = \int_0^\infty e^{-st} v'(t)\,dt$ and integrate by parts, taking $u=e^{-st}$ and $dv=v'(t)\,dt$.

(b) Apply the Laplace transform to both sides of the differential equation
\[
v'(t) + v(t) = V_0,\qquad v(0)=v_0.
\]
Use your result from part (a) and the fact that $\mathcal{L}\{V_0\}(s) = \dfrac{V_0}{s}$ to obtain an algebraic equation for $V(s) = \mathcal{L}\{v\}(s)$ that explicitly involves $v_0$.

\emph{Hint:} Transform each term separately and remember that the Laplace transform is linear.

(c) Solve the algebraic equation from part (b) for $V(s)$. Then rewrite your answer in a form that is convenient for taking the inverse Laplace transform. In particular, show that you can write $V(s)$ as a sum of simple rational functions whose inverse transforms are standard.

\emph{Hint:} You should find an expression of the form
\[
V(s) = \frac{v_0}{s+1} + \frac{V_0}{s(s+1)}.
\]
Then perform a partial fraction decomposition of $\dfrac{V_0}{s(s+1)}$:
\[
\frac{V_0}{s(s+1)} = \frac{A}{s} + \frac{B}{s+1}
\]
for appropriate constants $A$ and $B$.

(d) Use a Laplace transform table (or known basic transforms) to take the inverse Laplace transform of your expression for $V(s)$ and thereby find an explicit formula for $v(t)$ for $t\ge 0$. Simplify your answer as much as possible.

\emph{Hint:} You should encounter transforms of the form
\[
\mathcal{L}^{-1}\left\{\frac{1}{s}\right\}(t) = 1,\qquad
\mathcal{L}^{-1}\left\{\frac{1}{s+1}\right\}(t) = e^{-t}.
\]

(e) Interpretation and extensions.

\begin{enumerate}
  \item[(i)] Compute $\displaystyle \lim_{t\to\infty} v(t)$ and explain what this limit means physically for the charging of the capacitor.
  
  \item[(ii)] Suppose now that the input voltage is not constant, but instead decays exponentially: $V_{\text{in}}(t) = V_0 e^{-\alpha t}$ with $\alpha>0$. The equation becomes
  \[
  v'(t) + v(t) = V_0 e^{-\alpha t},\qquad v(0)=v_0.
  \]
  Without carrying out all the details, outline how you would modify your Laplace transform solution to handle this new right-hand side. Which Laplace transform identities would you need, and where would $\alpha$ appear in the algebraic equation?
  
  \item[(iii)] How would your work change if the coefficient in front of $v(t)$ were some positive constant $a>0$, giving the more general equation
  \[
  v'(t) + a\,v(t) = V_0,\qquad v(0)=v_0?
  \]
  Describe briefly (in words or formulas) how the form of $V(s)$ and then $v(t)$ would change.
\end{enumerate}
\end{problem}

% ===== Example 1: First-Order Linear ODE Solved by Laplace Transform (full solution) =====
\begin{problem}[First-Order Linear ODE Solved by Laplace Transform]
Solve the initial value problem
\[
v'(t) + v(t) = V_0,\qquad t>0,\qquad v(0)=v_0,
\]
where $V_0$ and $v_0$ are constants, using the Laplace transform. Show explicitly how the initial condition appears in the transformed (algebraic) equation, and compute $v(t)$ for $t\ge 0$ by inverting the transform.
\end{problem}

\begin{solution}
We are given the first-order linear ordinary differential equation
\[
v'(t) + v(t) = V_0,\qquad t>0,\qquad v(0)=v_0.
\]
This equation models, for instance, the voltage $v(t)$ across a capacitor in a simple $RC$ circuit when a constant input voltage $V_0$ is applied at $t=0$ and the time scale has been chosen so that $RC=1$.

The central idea of the Laplace transform method is to convert the differential equation in the time variable $t$ into an algebraic equation in the Laplace variable $s$. The derivative $v'(t)$ will transform into a linear expression involving $s$ and the transform $V(s)$, and the initial condition will enter as a simple constant term.

Let $V(s) = \mathcal{L}\{v\}(s)$ denote the Laplace transform of $v(t)$. By definition,
\[
V(s) = \mathcal{L}\{v\}(s) = \int_0^\infty e^{-st} v(t)\,dt
\]
for those $s$ for which the integral converges.

We first recall the Laplace transform of a derivative. Assuming $v$ is sufficiently smooth and of exponential order, we compute
\[
\mathcal{L}\{v'\}(s)
= \int_0^\infty e^{-st} v'(t)\,dt.
\]
Integrating by parts, with $u = e^{-st}$ and $dv = v'(t)\,dt$, so that $du = -s e^{-st}\,dt$ and $v$ is an antiderivative of $v'$, we obtain
\[
\int_0^\infty e^{-st} v'(t)\,dt
= \bigl[e^{-st} v(t)\bigr]_{t=0}^{t=\infty} + s \int_0^\infty e^{-st} v(t)\,dt.
\]
Under the usual assumption that $e^{-st} v(t) \to 0$ as $t\to\infty$ for $\operatorname{Re}(s)$ sufficiently large, the boundary term at infinity vanishes. Evaluating at $t=0$ gives $e^0 v(0) = v_0$. Therefore
\[
\mathcal{L}\{v'\}(s) = -v_0 + s V(s) = sV(s) - v_0.
\]

Now we apply the Laplace transform term-by-term to the differential equation. Using linearity of the transform, we get
\[
\mathcal{L}\{v'\}(s) + \mathcal{L}\{v\}(s) = \mathcal{L}\{V_0\}(s).
\]
Substituting $\mathcal{L}\{v'\}(s) = sV(s) - v_0$ and $\mathcal{L}\{v\}(s) = V(s)$, and noting that the Laplace transform of a constant $V_0$ is
\[
\mathcal{L}\{V_0\}(s) = \int_0^\infty e^{-st} V_0\,dt = \frac{V_0}{s}
\]
for $\operatorname{Re}(s) > 0$, we obtain the algebraic equation
\[
\bigl(sV(s) - v_0\bigr) + V(s) = \frac{V_0}{s}.
\]
That is,
\[
(s+1) V(s) - v_0 = \frac{V_0}{s}.
\]
Rearranging, we explicitly see the contribution of the initial condition:
\[
(s+1) V(s) = v_0 + \frac{V_0}{s}.
\]

We now solve this algebraic equation for $V(s)$:
\[
V(s) = \frac{v_0}{s+1} + \frac{V_0}{s(s+1)}.
\]
The first term already has a simple form, but the second term is a rational function that we wish to decompose into simpler pieces suited for inversion. We perform a partial fraction decomposition on
\[
\frac{V_0}{s(s+1)}.
\]
We seek constants $A$ and $B$ such that
\[
\frac{V_0}{s(s+1)} = \frac{A}{s} + \frac{B}{s+1}.
\]
Multiplying both sides by $s(s+1)$ gives
\[
V_0 = A(s+1) + Bs = (A+B)s + A.
\]
Identifying coefficients of like powers of $s$, we obtain the system
\[
A + B = 0,\qquad A = V_0.
\]
Hence $A = V_0$ and $B = -V_0$. Therefore
\[
\frac{V_0}{s(s+1)} = \frac{V_0}{s} - \frac{V_0}{s+1}.
\]

Substituting this back into the expression for $V(s)$ yields
\[
V(s) = \frac{v_0}{s+1} + \left(\frac{V_0}{s} - \frac{V_0}{s+1}\right)
= \frac{V_0}{s} + \frac{v_0 - V_0}{s+1}.
\]

We are now ready to invert the Laplace transform term-by-term. We use the standard formulas
\[
\mathcal{L}^{-1}\left\{\frac{1}{s}\right\}(t) = 1,\qquad
\mathcal{L}^{-1}\left\{\frac{1}{s+1}\right\}(t) = e^{-t}.
\]
By linearity of the inverse transform,
\[
v(t) = \mathcal{L}^{-1}\{V(s)\}(t)
= V_0\,\mathcal{L}^{-1}\left\{\frac{1}{s}\right\}(t)
+ (v_0 - V_0)\,\mathcal{L}^{-1}\left\{\frac{1}{s+1}\right\}(t).
\]
Thus
\[
v(t) = V_0 \cdot 1 + (v_0 - V_0) e^{-t} = V_0 + (v_0 - V_0) e^{-t},\qquad t\ge 0.
\]

This is the explicit solution of the initial value problem. One can easily verify that it satisfies both the differential equation and the initial condition: at $t=0$,
\[
v(0) = V_0 + (v_0 - V_0) e^{0} = v_0,
\]
and a direct calculation of $v'(t)$ shows that $v'(t) + v(t) = V_0$ for all $t$.

It is also instructive to examine the long-time behavior. Since $e^{-t} \to 0$ as $t\to\infty$, we have
\[
\lim_{t\to\infty} v(t) = V_0.
\]
Physically, this means that the voltage across the capacitor approaches the input voltage $V_0$ as time goes on. The difference $v(t) - V_0 = (v_0 - V_0) e^{-t}$ decays exponentially with time constant $1$, which in a more general model would be the product $RC$.

From the perspective of the chapter, this example illustrates clearly how the Laplace transform converts a differential equation into an algebraic equation in $s$. The derivative $v'(t)$ becomes $sV(s) - v_0$, so the initial condition enters as a constant term. Solving the algebraic equation for $V(s)$ and then using partial fractions allows us to express $V(s)$ as a sum of simple rational functions whose inverse transforms are known. In this way, the original time-domain problem is solved by an algebraic manipulation in the Laplace domain followed by a table lookup. This is a prototypical instance of the theme “from differential to algebraic equations” using integral transforms.
\end{solution}

% ===== Example 2: Heat Equation on a Finite Rod via Fourier Sine Series (inquiry-based) =====
\begin{problem}[Heat Equation on a Finite Rod via Fourier Sine Series]
Consider a thin, homogeneous rod of length $L$ lying along the $x$-axis from $x=0$ to $x=L$. We assume that heat can flow only along the rod, and that the ends of the rod are kept in contact with large ice baths so that their temperature is held at zero for all time. The temperature distribution $u(x,t)$ in the rod evolves according to the one-dimensional heat equation, which couples time derivatives and second spatial derivatives. Our goal is to see how expanding $u(x,t)$ in a Fourier sine series in $x$ turns this partial differential equation into a decoupled family of ordinary differential equations in $t$.

Throughout, let $\kappa>0$ denote the thermal diffusivity, and let $f(x)$ denote the initial temperature distribution along the rod at time $t=0$.

\smallskip

(a) \textbf{Setting up the model.}
Write down the initial-boundary value problem (IBVP) for $u(x,t)$ describing this situation. That is, write the governing partial differential equation, the boundary conditions at $x=0$ and $x=L$, and the initial condition at $t=0$.

\emph{Hint:} The PDE is the standard heat equation in one space dimension; the ends of the rod are held at zero temperature.

\smallskip

(b) \textbf{Separation of variables and the spatial eigenvalue problem.}
We look for separated solutions of the form $u(x,t)=X(x)T(t)$.

\quad (i) Substitute $u(x,t)=X(x)T(t)$ into your PDE from part (a) and separate variables to obtain an equation of the form
\[
\frac{T'(t)}{\kappa\,T(t)} = \frac{X''(x)}{X(x)} = \text{(constant)}.
\]

Call this constant $-\lambda$, and derive the resulting pair of ordinary differential equations for $X$ and $T$.

\quad (ii) Use the boundary conditions at $x=0$ and $x=L$ to obtain a boundary value problem for $X(x)$ alone. Show that nontrivial solutions $X(x)$ can exist only for certain special values of $\lambda$, and identify these $\lambda$ as eigenvalues of the spatial problem.

\emph{Hint:} You should obtain a second-order ODE $X''+\lambda X=0$ with homogeneous Dirichlet boundary conditions $X(0)=X(L)=0$. Recall from ordinary differential equations how to solve such problems and how boundary conditions quantize the allowed values of $\lambda$.

\smallskip

(c) \textbf{Finding eigenfunctions and time dependence.}
Continue the analysis from part (b).

\quad (i) Solve the eigenvalue problem for $X$ explicitly, and show that the admissible eigenvalues are
\[
\lambda_n = \left(\frac{n\pi}{L}\right)^2, \qquad n=1,2,3,\dots,
\]
with corresponding eigenfunctions
\[
X_n(x) = \sin\left(\frac{n\pi x}{L}\right).
\]

\quad (ii) For each $n$, solve the time ODE for $T(t)$ associated to $\lambda=\lambda_n$. Show that the separated solutions have the form
\[
u_n(x,t) = \sin\!\left(\frac{n\pi x}{L}\right) e^{-\kappa\left(\frac{n\pi}{L}\right)^2 t},
\qquad n=1,2,3,\dots.
\]

\emph{Hint:} After choosing $\lambda=\lambda_n$, the time equation is a first-order linear ODE $T'(t) = -\kappa\lambda_n T(t)$.

\smallskip

(d) \textbf{Superposition, Fourier sine series, and the role of orthogonality.}
The heat equation is linear, so sums of solutions are again solutions.

\quad (i) Argue that for any sequence of constants $(b_n)_{n\ge 1}$, the series
\[
u(x,t) = \sum_{n=1}^{\infty} b_n\,e^{-\kappa\left(\frac{n\pi}{L}\right)^2 t}\,\sin\!\left(\frac{n\pi x}{L}\right)
\]
formally satisfies the PDE and the boundary conditions.

\quad (ii) Impose the initial condition $u(x,0)=f(x)$ and explain why this leads to the requirement that
\[
f(x) = \sum_{n=1}^{\infty} b_n \sin\!\left(\frac{n\pi x}{L}\right)
\]
be the Fourier sine series expansion of $f$ on $[0,L]$.

\quad (iii) Using the orthogonality of the sine functions on $[0,L]$, derive a formula for the coefficients $b_n$ in terms of $f(x)$.

\emph{Hint:} Multiply the identity in (ii) by $\sin\!\left(\frac{m\pi x}{L}\right)$ and integrate from $0$ to $L$. Recall that
\[
\int_0^L \sin\!\left(\frac{n\pi x}{L}\right)\sin\!\left(\frac{m\pi x}{L}\right)\,dx
= \begin{cases}
0, & m\neq n,\\[4pt]
\dfrac{L}{2}, & m=n.
\end{cases}
\]

\smallskip

(e) \textbf{Putting it all together and extensions.}

\quad (i) Combine your work to write the final formula for the solution $u(x,t)$ in terms of $f(x)$. Briefly describe in words how each sine mode behaves as time increases, and what happens to the temperature distribution as $t\to\infty$.

\quad (ii) (Extension: alternative boundary condition.) Suppose instead that the left end of the rod is insulated, so that no heat flows through $x=0$, while the right end is still held at zero temperature:
\[
u_x(0,t)=0,\qquad u(L,t)=0.
\]
How would you expect the spatial eigenfunctions and eigenvalues to change? Which trigonometric functions (sines, cosines, or a mixture) would naturally appear now, and why?

\quad (iii) (Extension: nonzero boundary temperature.) If both ends of the rod are held at a nonzero constant temperature $u(0,t)=u(L,t)=U_0$, how could you modify the method so that you can still use a Fourier sine series? Outline the main idea without doing all the computations.

\emph{Hint:} For (iii), think about first subtracting a simple steady-state solution that already satisfies the boundary conditions, and then solving for the remaining part with homogeneous boundary conditions.
\end{problem}

% ===== Example 2: Heat Equation on a Finite Rod via Fourier Sine Series (full solution) =====
\begin{problem}[Heat Equation on a Finite Rod via Fourier Sine Series]
Let $u(x,t)$ denote the temperature in a thin homogeneous rod of length $L$, for $0<x<L$, $t>0$. Consider the initial-boundary value problem
\[
u_t = \kappa u_{xx}, \quad 0<x<L,\ t>0,
\]
\[
u(0,t)=0,\quad u(L,t)=0,\quad t>0,
\]
\[
u(x,0)=f(x),\quad 0<x<L,
\]
where $\kappa>0$ and $f$ is a given function. Using separation of variables and a Fourier sine series in $x$, find an explicit series formula for $u(x,t)$ in terms of $f$.
\end{problem}

\begin{solution}
We solve the homogeneous heat equation with homogeneous Dirichlet boundary conditions by separation of variables and the eigenfunction expansion method. This will exhibit how the second derivative in space becomes multiplication by a scalar in an appropriate basis, turning the partial differential equation into algebraic relations for Fourier coefficients.

\medskip

\textbf{1. Separation of variables and the eigenvalue problem.}
We seek separated solutions of the form
\[
u(x,t) = X(x)T(t).
\]
Substituting into the PDE $u_t=\kappa u_{xx}$ gives
\[
X(x)T'(t) = \kappa X''(x)T(t).
\]
Assuming $X$ and $T$ are not identically zero, we can divide both sides by $\kappa X(x)T(t)$ and obtain
\[
\frac{T'(t)}{\kappa T(t)} = \frac{X''(x)}{X(x)} = -\lambda,
\]
for some constant $-\lambda$ independent of $x$ and $t$. This leads to the pair of ordinary differential equations
\[
T'(t) + \kappa\lambda T(t) = 0,\qquad X''(x) + \lambda X(x) = 0.
\]
The boundary conditions $u(0,t)=u(L,t)=0$ become
\[
X(0)T(t) = 0,\qquad X(L)T(t)=0 \quad \text{for all } t>0.
\]
Since we are interested in nontrivial solutions $u$, we cannot have $T\equiv 0$, so we require
\[
X(0)=0,\qquad X(L)=0.
\]
Thus $X$ must solve the boundary value problem
\[
X''(x) + \lambda X(x) = 0,\quad 0<x<L,\qquad X(0)=X(L)=0.
\]
This is a standard Sturm--Liouville eigenvalue problem.

\medskip

\textbf{2. Solving the spatial eigenvalue problem.}
We determine the admissible values of $\lambda$ and the corresponding eigenfunctions $X$.

Consider three cases:

\emph{Case 1: $\lambda<0$.} Write $\lambda=-\mu^2$ with $\mu>0$. Then the equation becomes
\[
X''(x) - \mu^2 X(x) = 0,
\]
with general solution $X(x) = A e^{\mu x} + B e^{-\mu x}$. The boundary condition $X(0)=0$ gives $A+B=0$, so $B=-A$ and $X(x)=A(e^{\mu x}-e^{-\mu x}) = 2A\sinh(\mu x)$. The condition $X(L)=0$ then forces $A\sinh(\mu L)=0$, hence $A=0$ since $\sinh(\mu L)\neq 0$. Thus only the trivial solution exists, so $\lambda<0$ yields no eigenvalues.

\emph{Case 2: $\lambda=0$.} The equation becomes $X''(x)=0$ with general solution $X(x)=Ax+B$. The condition $X(0)=0$ gives $B=0$, and $X(L)=0$ gives $AL=0$, so $A=0$. Again we obtain only the trivial solution; thus $\lambda=0$ is not an eigenvalue.

\emph{Case 3: $\lambda>0$.} Write $\lambda=\mu^2$ with $\mu>0$. The equation is
\[
X''(x)+\mu^2 X(x) = 0,
\]
whose general solution is $X(x)=A\cos(\mu x)+B\sin(\mu x)$. The boundary condition $X(0)=0$ forces $A=0$, so $X(x)=B\sin(\mu x)$. The condition $X(L)=0$ then implies $B\sin(\mu L)=0$. For a nontrivial solution we require $B\neq 0$, so $\sin(\mu L)=0$, which holds if and only if
\[
\mu L = n\pi,\quad n=1,2,3,\dots.
\]
Therefore
\[
\mu_n = \frac{n\pi}{L},\qquad \lambda_n = \mu_n^2 = \left(\frac{n\pi}{L}\right)^2,
\]
and corresponding eigenfunctions are
\[
X_n(x) = \sin\left(\frac{n\pi x}{L}\right),\qquad n=1,2,3,\dots.
\]

Thus the spatial eigenfunctions form a sine basis adapted to the homogeneous Dirichlet boundary conditions.

\medskip

\textbf{3. Time dependence and separated solutions.}
For each eigenvalue $\lambda_n$, the time equation
\[
T'(t) + \kappa \lambda_n T(t) = 0
\]
is a first-order linear ODE with solution
\[
T_n(t) = C_n e^{-\kappa \lambda_n t}
= C_n e^{-\kappa\left(\frac{n\pi}{L}\right)^2 t},
\]
where $C_n$ is a constant.

Hence, for each $n\ge 1$, we obtain a separated solution
\[
u_n(x,t) = X_n(x)T_n(t)
= C_n \sin\left(\frac{n\pi x}{L}\right)
   e^{-\kappa\left(\frac{n\pi}{L}\right)^2 t}.
\]
Because the heat equation is linear and homogeneous, any finite linear combination of the $u_n$ is also a solution. Motivated by completeness of the sine functions, we consider the infinite series
\[
u(x,t) = \sum_{n=1}^{\infty} b_n e^{-\kappa\left(\frac{n\pi}{L}\right)^2 t}
\sin\left(\frac{n\pi x}{L}\right),
\]
where the constants $b_n$ are to be determined from the initial condition.

Each term in the series satisfies the PDE and the boundary conditions individually, and under reasonable assumptions on $f$ this series converges sufficiently well to inherit these properties termwise.

\medskip

\textbf{4. Imposing the initial condition and using Fourier sine series.}
At time $t=0$ the series reduces to
\[
u(x,0) = \sum_{n=1}^{\infty} b_n \sin\left(\frac{n\pi x}{L}\right).
\]
The initial condition $u(x,0)=f(x)$ therefore demands that $f$ admit a Fourier sine series expansion on $[0,L]$:
\[
f(x) \sim \sum_{n=1}^{\infty} b_n \sin\left(\frac{n\pi x}{L}\right).
\]

The sine functions are orthogonal on $[0,L]$:
\[
\int_0^L \sin\left(\frac{n\pi x}{L}\right)\sin\left(\frac{m\pi x}{L}\right)\,dx
= \begin{cases}
0, & n\neq m,\\[4pt]
\dfrac{L}{2}, & n=m.
\end{cases}
\]
To extract $b_m$, multiply the series for $f(x)$ by $\sin\left(\frac{m\pi x}{L}\right)$ and integrate from $0$ to $L$:
\[
\int_0^L f(x)\sin\left(\frac{m\pi x}{L}\right)\,dx
= \int_0^L \left(\sum_{n=1}^{\infty} b_n \sin\left(\frac{n\pi x}{L}\right)\right)
               \sin\left(\frac{m\pi x}{L}\right)\,dx.
\]
Assuming we can interchange sum and integral, orthogonality gives
\[
\int_0^L f(x)\sin\left(\frac{m\pi x}{L}\right)\,dx
= \sum_{n=1}^{\infty} b_n \int_0^L \sin\left(\frac{n\pi x}{L}\right)
                             \sin\left(\frac{m\pi x}{L}\right)\,dx
= b_m \frac{L}{2}.
\]
Thus
\[
b_m = \frac{2}{L}\int_0^L f(x)\sin\left(\frac{m\pi x}{L}\right)\,dx,
\qquad m=1,2,3,\dots.
\]

\medskip

\textbf{5. Final solution formula and qualitative behavior.}
We have now determined both the spatial dependence and the time dependence. The solution to the initial-boundary value problem is
\[
u(x,t) = \sum_{n=1}^{\infty}
\left(\frac{2}{L}\int_0^L f(s)\sin\left(\frac{n\pi s}{L}\right)\,ds\right)
e^{-\kappa\left(\frac{n\pi}{L}\right)^2 t}
\sin\left(\frac{n\pi x}{L}\right),
\]
provided $f$ is sufficiently regular for the Fourier series and the manipulations above to be justified.

Each mode
\[
\sin\left(\frac{n\pi x}{L}\right)
\]
is an eigenfunction of the spatial second-derivative operator $u\mapsto u_{xx}$ with eigenvalue $-\left(\frac{n\pi}{L}\right)^2$. In the sine basis, the spatial operator is diagonal: applying $u_{xx}$ multiplies the $n$-th Fourier sine coefficient by $-\left(\frac{n\pi}{L}\right)^2$. The heat equation $u_t=\kappa u_{xx}$ therefore translates into a family of decoupled scalar ODEs
\[
\frac{d}{dt}b_n(t) = -\kappa\left(\frac{n\pi}{L}\right)^2 b_n(t),
\]
whose solutions are the exponential factors $e^{-\kappa\left(\frac{n\pi}{L}\right)^2 t}$. This is the essence of the theme of this chapter: a differential equation in $(x,t)$ becomes a simple algebraic relation in the transformed (Fourier) domain.

In terms of qualitative behavior, higher-frequency modes (large $n$) decay more rapidly because their decay rate $\kappa (n\pi/L)^2$ is larger. As $t\to\infty$, every exponential factor tends to zero, so $u(x,t)\to 0$ for all $x$: the rod cools down to the boundary temperature, which is zero everywhere.

This example illustrates how a suitable Fourier sine series converts a PDE with spatial derivatives and boundary conditions into an infinite diagonal system of ordinary differential equations in time, with the Laplacian acting as a scalar multiplier on each Fourier mode.
\end{solution}

% ===== Example 3: Heat Equation on the Whole Line via Fourier Transform (inquiry-based) =====
\begin{problem}[Heat Equation on the Whole Line via Fourier Transform]
Consider a very long, effectively infinite, thin rod lying along the real line $\mathbb{R}$.  
We assume that heat diffuses along the rod according to the heat equation, and that initially the temperature distribution is localized: it is significant near the origin and decays rapidly as $|x| \to \infty$.  
On a finite interval, we diagonalized the Laplacian using Fourier series; on the whole line, the analogous tool is the Fourier transform, which decomposes functions into a continuous superposition of plane waves.  
In this problem you will discover how the Fourier transform converts the heat equation into a family of algebraic (actually ODE) equations and leads to the classical Gaussian heat kernel.

Throughout, fix a diffusion constant $\kappa>0$, and consider the Cauchy problem
\[
\begin{cases}
u_t(x,t) = \kappa\,u_{xx}(x,t), & x \in \mathbb{R},\ t>0,\\[0.2em]
u(x,0) = f(x), & x \in \mathbb{R},
\end{cases}
\]
where $f$ is a sufficiently nice, rapidly decaying function.

We use the following Fourier transform convention in the spatial variable $x$:
\[
\widehat{g}(\xi) := \displaystyle\int_{-\infty}^{\infty} g(x)\,e^{-i\xi x}\,dx,
\qquad
g(x) = \frac{1}{2\pi}\int_{-\infty}^{\infty} \widehat{g}(\xi)\,e^{i\xi x}\,d\xi.
\]

\medskip

(a) \emph{Fourier transform and differentiation.}  
Let $g$ be a sufficiently smooth and rapidly decaying function.  

\quad(i) Using integration by parts, show that
\[
\widehat{g'}(\xi) = i\xi\,\widehat{g}(\xi).
\]
(Hint: Integrate $g'(x)e^{-i\xi x}$ by parts and use that $g(x)\to 0$ as $|x|\to\infty$.)

\quad(ii) Use part (i) to show that
\[
\widehat{g''}(\xi) = -\xi^2\,\widehat{g}(\xi).
\]

\medskip

(b) \emph{Transforming the heat equation.}  
Assume that for each fixed $t>0$, the function $x\mapsto u(x,t)$ decays sufficiently fast at $\pm\infty$ so that you can apply the Fourier transform in $x$ and differentiate under the integral sign.

\quad(i) Define
\[
\widehat{u}(\xi,t) := \int_{-\infty}^{\infty} u(x,t)\,e^{-i\xi x}\,dx.
\]
Show that
\[
\partial_t \widehat{u}(\xi,t)
= \kappa\,\widehat{u_{xx}}(\xi,t).
\]
(Hint: Differentiate inside the integral with respect to $t$.)

\quad(ii) Use part (a) to express $\widehat{u_{xx}}(\xi,t)$ in terms of $\widehat{u}(\xi,t)$, and deduce that for each fixed frequency $\xi \in \mathbb{R}$, the function $t \mapsto \widehat{u}(\xi,t)$ satisfies a first-order linear ODE.  
Write down this ODE explicitly, together with the initial condition at $t=0$, expressed in terms of $\widehat{f}(\xi)$.

\medskip

(c) \emph{Solving the transformed problem.}  

\quad(i) Solve the ODE found in part (b) for $\widehat{u}(\xi,t)$ in terms of $\widehat{f}(\xi)$, $\xi$, $t$, and $\kappa$.  

\quad(ii) Interpret the solution as ``each Fourier mode decays exponentially in time''.  
What is the decay rate of the mode corresponding to frequency $\xi$?

\medskip

(d) \emph{Inverting the transform and the heat kernel.}  

\quad(i) Use the inverse Fourier transform to write $u(x,t)$ as
\[
u(x,t) = \frac{1}{2\pi}\int_{-\infty}^{\infty} e^{-\kappa \xi^2 t}\,\widehat{f}(\xi)\,e^{i\xi x}\,d\xi.
\]
Then use the Fourier inversion formula for $f$ to express $u(x,t)$ as a convolution
\[
u(x,t) = (K_t * f)(x) := \int_{-\infty}^{\infty} K_t(x-y)\,f(y)\,dy,
\]
for some kernel $K_t$.  
Write $K_t$ as an explicit inverse Fourier transform:
\[
K_t(x) = \frac{1}{2\pi}\int_{-\infty}^{\infty} e^{-\kappa \xi^2 t}\,e^{i\xi x}\,d\xi.
\]

\quad(ii) Compute $K_t(x)$ explicitly.  
To do this, recall the Gaussian integral
\[
\int_{-\infty}^{\infty} e^{-a z^2}\,dz = \sqrt{\frac{\pi}{a}}, \qquad a>0,
\]
and complete the square in the exponent of $e^{-\kappa \xi^2 t + i\xi x}$.  
Show that
\[
K_t(x) = \frac{1}{\sqrt{4\pi \kappa t}}\,\exp\!\left(-\frac{x^2}{4\kappa t}\right).
\]
(Hint: First write $-\kappa t\,\xi^2 + i\xi x$ as $-\kappa t\bigl(\xi - \tfrac{i x}{2\kappa t}\bigr)^2 - \tfrac{x^2}{4\kappa t}$.)

\quad(iii) Conclude that the solution of the Cauchy problem is
\[
u(x,t) = \int_{-\infty}^{\infty} \frac{1}{\sqrt{4\pi \kappa t}}\,\exp\!\left(-\frac{(x-y)^2}{4\kappa t}\right) f(y)\,dy.
\]
Briefly interpret this formula: how does a localized initial temperature distribution spread out as time increases?

\medskip

(e) \emph{Extensions and further questions.}

\quad(i) Suppose the initial condition is a point source at the origin, modeled by the Dirac delta $\delta_0$ (so $f = \delta_0$).  
Using your kernel $K_t$, what is the corresponding solution $u(x,t)$?  
How does this relate to your answer in part (d)(ii)?

\quad(ii) What is the total heat in the rod at time $t$, that is,
\[
\int_{-\infty}^{\infty} u(x,t)\,dx?
\]
Assuming $\int_{-\infty}^{\infty} f(x)\,dx$ is finite, use the convolution formula to show that the total heat is conserved in time.

\quad(iii) (Conceptual) In what sense did the Fourier transform turn a \emph{partial} differential equation into a family of \emph{ordinary} differential equations?  
How is this analogous to what Fourier series did for the heat equation on a finite interval?
\end{problem}

% ===== Example 3: Heat Equation on the Whole Line via Fourier Transform (full solution) =====
\begin{problem}[Heat Equation on the Whole Line via Fourier Transform]
Let $\kappa>0$ and consider the Cauchy problem on the real line
\[
\begin{cases}
u_t(x,t) = \kappa\,u_{xx}(x,t), & x \in \mathbb{R},\ t>0,\\[0.2em]
u(x,0) = f(x), & x \in \mathbb{R},
\end{cases}
\]
where $f$ is sufficiently smooth and decays rapidly at infinity.

Using the Fourier transform
\[
\widehat{g}(\xi) := \displaystyle\int_{-\infty}^{\infty} g(x)\,e^{-i\xi x}\,dx,
\qquad
g(x) = \frac{1}{2\pi}\int_{-\infty}^{\infty} \widehat{g}(\xi)\,e^{i\xi x}\,d\xi,
\]
solve the Cauchy problem explicitly and show that
\[
u(x,t) = \int_{-\infty}^{\infty} \frac{1}{\sqrt{4\pi \kappa t}}\,
\exp\!\left(-\frac{(x-y)^2}{4\kappa t}\right) f(y)\,dy.
\]
Identify the corresponding heat kernel $K_t$ and briefly explain how this example illustrates the passage from a differential equation in $x$ to an algebraic equation in the Fourier variable $\xi$.
\end{problem}

\begin{solution}
We begin by recalling how the Fourier transform interacts with differentiation.  
Let $g$ be smooth and rapidly decaying at $\pm\infty$.  
Then
\[
\widehat{g'}(\xi) = \int_{-\infty}^{\infty} g'(x)\,e^{-i\xi x}\,dx.
\]
Integrating by parts, with $u = e^{-i\xi x}$ and $dv = g'(x)\,dx$, we obtain
\[
\widehat{g'}(\xi)
= g(x)e^{-i\xi x}\Big|_{x=-\infty}^{x=\infty}
- \int_{-\infty}^{\infty} g(x)\,(-i\xi)\,e^{-i\xi x}\,dx.
\]
The boundary term vanishes because $g(x)\to 0$ as $|x|\to\infty$, so
\[
\widehat{g'}(\xi) = i\xi\,\widehat{g}(\xi).
\]
Applying this once more to $g'$, we find
\[
\widehat{g''}(\xi)
= i\xi\,\widehat{g'}(\xi)
= i\xi\,(i\xi\,\widehat{g}(\xi))
= -\xi^2\,\widehat{g}(\xi).
\]
Thus the second derivative in $x$ becomes multiplication by $-\xi^2$ in the Fourier domain.

We now apply the spatial Fourier transform to the heat equation.  
For each $t>0$ we define
\[
\widehat{u}(\xi,t) := \int_{-\infty}^{\infty} u(x,t)\,e^{-i\xi x}\,dx.
\]
Assuming we can differentiate under the integral sign with respect to $t$, we have
\[
\partial_t \widehat{u}(\xi,t)
= \int_{-\infty}^{\infty} u_t(x,t)\,e^{-i\xi x}\,dx.
\]
The PDE gives $u_t = \kappa u_{xx}$, hence
\[
\partial_t \widehat{u}(\xi,t)
= \kappa \int_{-\infty}^{\infty} u_{xx}(x,t)\,e^{-i\xi x}\,dx
= \kappa\,\widehat{u_{xx}}(\xi,t).
\]
Using the differentiation property just established,
\[
\widehat{u_{xx}}(\xi,t)
= -\xi^2\,\widehat{u}(\xi,t),
\]
so the transformed equation becomes
\[
\partial_t \widehat{u}(\xi,t)
= -\kappa\,\xi^2\,\widehat{u}(\xi,t).
\]
For each fixed spatial frequency $\xi \in \mathbb{R}$, this is an ordinary differential equation in $t$:
\[
\frac{d}{dt}\widehat{u}(\xi,t) = -\kappa \xi^2\,\widehat{u}(\xi,t).
\]

The initial condition $u(x,0)=f(x)$ transforms to
\[
\widehat{u}(\xi,0)
= \int_{-\infty}^{\infty} u(x,0)\,e^{-i\xi x}\,dx
= \int_{-\infty}^{\infty} f(x)\,e^{-i\xi x}\,dx
= \widehat{f}(\xi).
\]
Thus, for each $\xi$, we have the initial value problem
\[
\frac{d}{dt}\widehat{u}(\xi,t) = -\kappa \xi^2\,\widehat{u}(\xi,t),
\qquad
\widehat{u}(\xi,0) = \widehat{f}(\xi).
\]

This is a first-order linear ODE with solution
\[
\widehat{u}(\xi,t)
= e^{-\kappa \xi^2 t}\,\widehat{f}(\xi).
\]
Hence each Fourier mode of frequency $\xi$ decays exponentially in time at rate $\kappa \xi^2$: higher frequencies (larger $|\xi|$) decay faster, reflecting the smoothing effect of diffusion.

To recover $u(x,t)$ in physical space, we apply the inverse Fourier transform:
\[
u(x,t)
= \frac{1}{2\pi}\int_{-\infty}^{\infty} \widehat{u}(\xi,t)\,e^{i\xi x}\,d\xi
= \frac{1}{2\pi}\int_{-\infty}^{\infty} e^{-\kappa \xi^2 t}\,\widehat{f}(\xi)\,e^{i\xi x}\,d\xi.
\]
Substituting the Fourier representation of $\widehat{f}$,
\[
\widehat{f}(\xi)
= \int_{-\infty}^{\infty} f(y)\,e^{-i\xi y}\,dy,
\]
we obtain
\[
u(x,t)
= \frac{1}{2\pi}\int_{-\infty}^{\infty} e^{-\kappa \xi^2 t} e^{i\xi x}
\left(\int_{-\infty}^{\infty} f(y)\,e^{-i\xi y}\,dy\right)d\xi.
\]
Assuming we may interchange the order of integration, this becomes
\[
u(x,t)
= \int_{-\infty}^{\infty} f(y)
\left[
\frac{1}{2\pi}\int_{-\infty}^{\infty} e^{-\kappa \xi^2 t}\,e^{i\xi (x-y)}\,d\xi
\right]dy.
\]
We recognize the inner integral as an inverse Fourier transform depending only on $x-y$ and $t$.  
Define the kernel
\[
K_t(z) := \frac{1}{2\pi}\int_{-\infty}^{\infty} e^{-\kappa \xi^2 t}\,e^{i\xi z}\,d\xi,
\qquad z\in\mathbb{R}.
\]
Then the solution can be written as a convolution in space:
\[
u(x,t) = (K_t * f)(x) = \int_{-\infty}^{\infty} K_t(x-y)\,f(y)\,dy.
\]

It remains to compute $K_t$ explicitly.  
We consider
\[
K_t(z) = \frac{1}{2\pi}\int_{-\infty}^{\infty} e^{-\kappa t\,\xi^2 + i\xi z}\,d\xi.
\]
We complete the square in the exponent:
\[
-\kappa t\,\xi^2 + i\xi z
= -\kappa t\left(\xi^2 - \frac{i z}{\kappa t}\xi\right)
= -\kappa t\left(\xi - \frac{i z}{2\kappa t}\right)^2 - \frac{z^2}{4\kappa t}.
\]
Thus
\[
e^{-\kappa t\,\xi^2 + i\xi z}
= \exp\!\left(-\kappa t\left(\xi - \frac{i z}{2\kappa t}\right)^2\right)
\exp\!\left(-\frac{z^2}{4\kappa t}\right).
\]
The factor $\exp\bigl(-\frac{z^2}{4\kappa t}\bigr)$ is independent of $\xi$ and can be taken outside the integral:
\[
K_t(z)
= \frac{1}{2\pi}\exp\!\left(-\frac{z^2}{4\kappa t}\right)
\int_{-\infty}^{\infty}
\exp\!\left(-\kappa t\left(\xi - \frac{i z}{2\kappa t}\right)^2\right)d\xi.
\]
We now change variables to $\eta = \xi - \frac{i z}{2\kappa t}$, so $d\xi = d\eta$.  
The path of integration can be shifted vertically in the complex plane because the integrand is an entire function and decays rapidly; the value of the Gaussian integral is unchanged.  
Hence
\[
\int_{-\infty}^{\infty}
\exp\!\left(-\kappa t\left(\xi - \frac{i z}{2\kappa t}\right)^2\right)d\xi
= \int_{-\infty}^{\infty} e^{-\kappa t\,\eta^2}\,d\eta
= \sqrt{\frac{\pi}{\kappa t}},
\]
using the standard Gaussian integral formula with $a = \kappa t>0$.  
Substituting back, we obtain
\[
K_t(z)
= \frac{1}{2\pi}\exp\!\left(-\frac{z^2}{4\kappa t}\right)
\sqrt{\frac{\pi}{\kappa t}}
= \frac{1}{\sqrt{4\pi \kappa t}}\,
\exp\!\left(-\frac{z^2}{4\kappa t}\right).
\]

Therefore,
\[
u(x,t)
= \int_{-\infty}^{\infty} \frac{1}{\sqrt{4\pi \kappa t}}\,
\exp\!\left(-\frac{(x-y)^2}{4\kappa t}\right) f(y)\,dy.
\]
This $K_t(z) = \dfrac{1}{\sqrt{4\pi \kappa t}} \exp\!\bigl(-\dfrac{z^2}{4\kappa t}\bigr)$ is called the heat kernel on the real line.  
The formula shows that an initially localized temperature profile $f$ is smoothed out by convolution with a Gaussian whose variance grows linearly in time: the heat spreads out, and the temperature at a point is a weighted average of the initial temperatures nearby, with weights given by the Gaussian.

Finally, this example illustrates the main theme of the section ``From Differential to Algebraic Equations with FT, FS and LT.''  
The spatial Fourier transform converts the spatial differential operator $u_{xx}$ into multiplication by $-\xi^2$ in frequency space, so the original partial differential equation in $(x,t)$ becomes, for each $\xi$, an ordinary differential equation in $t$ with a simple exponential solution.  
The inversion of the transform then packages these exponentially decaying modes back into physical space as a convolution with the heat kernel.  
Thus the Fourier transform plays the same diagonalizing role on the whole line that Fourier series play on a finite interval, turning a differential operator into a simple algebraic multiplier in the transformed domain.
\end{solution}

% ===== Example 4: Damped Harmonic Oscillator and Resonance in the Laplace Domain (inquiry-based) =====
\begin{problem}[Damped Harmonic Oscillator and Resonance in the Laplace Domain]
A mass--spring--damper system is a standard model for many mechanical and electrical devices. A mass $m$ is attached to a spring with stiffness $k$ and a dashpot with damping coefficient $c$, and is driven by a periodic external force. In the time domain, the resulting ordinary differential equation couples inertia, damping, and restoring forces through derivatives of the displacement. In this problem you will see how the Laplace transform converts this differential equation into an algebraic equation in the Laplace variable $s$, and how the poles of the transformed solution encode transient and steady-state behavior, including resonance.

Consider a mass $m>0$ attached to a spring (stiffness $k>0$) and a damper (damping coefficient $c>0$), subject to an external forcing
\[
F_{\text{ext}}(t) = F_0 \cos(\omega t) \quad \text{for } t>0,
\]
with constants $F_0>0$ and forcing frequency $\omega>0$. Let $x(t)$ denote the displacement of the mass from equilibrium.

\smallskip

(a) Write down the second-order linear ODE that models the motion $x(t)$ of the mass, assuming that:
\begin{itemize}
  \item the spring force is $-kx(t)$,
  \item the damping force is $-c x'(t)$,
  \item the external force is $F_0 \cos(\omega t)$.
\end{itemize}
State clearly the initial conditions you will use to model a system that is \emph{initially at rest}. Briefly explain the physical meaning of each term in your ODE.

\smallskip

(b) Recall that the Laplace transform of a function $x(t)$ is
\[
X(s) = \mathcal{L}\{x\}(s) = \int_0^\infty e^{-st} x(t)\,dt.
\]
Using the standard formulas
\[
\mathcal{L}\{x'(t)\} = s X(s) - x(0), 
\qquad
\mathcal{L}\{x''(t)\} = s^2 X(s) - s x(0) - x'(0),
\]
and
\[
\mathcal{L}\{\cos(\omega t)\} = \frac{s}{s^2 + \omega^2},
\]
take the Laplace transform of your ODE from part (a). Solve the resulting algebraic equation for $X(s)$ in terms of $m,c,k,F_0,\omega$, and $s$.

\emph{Hint:} Keep the expression factored as much as possible; look for a factor $m s^2 + c s + k$ coming from the homogeneous part.

\smallskip

(c) The denominator of $X(s)$ encodes the system’s characteristic behavior. 

\begin{enumerate}
\item[(i)] Factor the quadratic polynomial
\[
m s^2 + c s + k
\]
and denote its roots by $r_1$ and $r_2$. Under the \emph{underdamped} assumption $c^2 < 4mk$, what can you say about the real and imaginary parts of $r_1$ and $r_2$?

\item[(ii)] List all poles of $X(s)$ (that is, all values of $s$ where $X(s)$ has a singularity). Distinguish between those coming from the homogeneous dynamics and those coming from the forcing term $\cos(\omega t)$.

\item[(iii)] Explain, at a qualitative level, why poles with negative real parts correspond to \emph{transient} terms in $x(t)$, while poles on the imaginary axis correspond to \emph{persistent oscillations}.

\emph{Hint:} Think about the inverse Laplace transforms of terms like $1/(s-a)$ and $1/(s^2+\omega^2)$.
\end{enumerate}

\smallskip

(d) To make the transient and steady-state contributions explicit, consider a partial fraction decomposition of $X(s)$ of the form
\[
X(s) = \frac{A s + B}{s^2 + \omega^2} + \frac{C s + D}{m s^2 + c s + k},
\]
for suitable constants $A,B,C,D$ depending on $m,c,k,F_0,\omega$.

\begin{enumerate}
\item[(i)] Multiply both sides by $(s^2 + \omega^2)(m s^2 + c s + k)$ and equate coefficients of powers of $s$ to obtain a linear system for $A,B,C,D$. (You do not need to solve completely for $C$ and $D$, but try to express $A$ and $B$ in terms of the parameters.)

\emph{Hint:} You should get four equations by matching coefficients of $s^3,s^2,s^1,s^0$ on both sides.

\item[(ii)] Using the Laplace transform table, write down the inverse Laplace transform of each term in your decomposition. Show that
\[
\mathcal{L}^{-1}\!\left\{\frac{A s + B}{s^2 + \omega^2}\right\}(t)
= A \cos(\omega t) + \frac{B}{\omega} \sin(\omega t),
\]
while 
\[
\mathcal{L}^{-1}\!\left\{\frac{C s + D}{m s^2 + c s + k}\right\}(t)
\]
is a linear combination of decaying exponentials (possibly multiplied by sines and cosines).

\item[(iii)] Conclude that the solution $x(t)$ can be written in the form
\[
x(t) = x_{\text{tr}}(t) + x_{\text{ss}}(t),
\]
where $x_{\text{tr}}(t)$ is a transient term that decays to zero as $t\to\infty$, and
\[
x_{\text{ss}}(t) = A \cos(\omega t) + \frac{B}{\omega} \sin(\omega t)
\]
is the steady-state oscillation at the forcing frequency $\omega$.

Finally, show that $x_{\text{ss}}(t)$ can be written as a single sinusoid
\[
x_{\text{ss}}(t) = R(\omega)\cos(\omega t - \delta(\omega)),
\]
and express the amplitude $R(\omega)$ in terms of $A,B$, and $\omega$.

\emph{Hint:} Use the identity $a\cos\theta + b\sin\theta = \sqrt{a^2 + b^2}\,\cos(\theta - \delta)$ for a suitable phase shift $\delta$.
\end{enumerate}

\smallskip

(e) In this final part, you will connect the Laplace-domain picture to the phenomenon of resonance.

\begin{enumerate}
\item[(i)] Show (either by completing part (d) fully or by a separate complex-exponential calculation) that the amplitude of the steady-state response is
\[
R(\omega) = \frac{F_0}{\sqrt{(k - m\omega^2)^2 + c^2 \omega^2}}.
\]
How does $R(\omega)$ behave as a function of $\omega$ when $c>0$ is small but fixed? Where (approximately) does it attain its maximum?

\emph{Hint:} Compare $\omega$ with the undamped natural frequency $\omega_0 = \sqrt{k/m}$.

\item[(ii)] \emph{What if} the damping were removed, so $c=0$? Examine the poles of $X(s)$ and the formula for $R(\omega)$ in this case. What happens when the forcing frequency $\omega$ approaches the natural frequency $\omega_0$? Relate this to the idea of resonance and to the movement of poles in the complex $s$-plane.

\end{enumerate}

\end{problem}

% ===== Example 4: Damped Harmonic Oscillator and Resonance in the Laplace Domain (full solution) =====
\begin{problem}[Damped Harmonic Oscillator and Resonance in the Laplace Domain]
Consider the mass--spring--damper system
\[
m x''(t) + c x'(t) + k x(t) = F_0 \cos(\omega t), \qquad t>0,
\]
with constants $m>0$, $c>0$, $k>0$, forcing amplitude $F_0>0$, and forcing frequency $\omega>0$. Assume the mass is initially at rest: $x(0)=0$ and $x'(0)=0$.

\begin{enumerate}
\item[(a)] Use the Laplace transform to solve for $x(t)$, expressing the solution as a sum of a transient term and a steady-state term. Clearly identify the poles of $X(s)=\mathcal{L}\{x\}(s)$ and explain which poles correspond to transient behavior and which to persistent oscillations.
\item[(b)] Show that the steady-state solution has the form
\[
x_{\mathrm{ss}}(t) = R(\omega)\cos\bigl(\omega t - \delta(\omega)\bigr),
\]
and derive the amplitude
\[
R(\omega) = \frac{F_0}{\sqrt{(k - m\omega^2)^2 + c^2 \omega^2}}.
\]
\item[(c)] Discuss how $R(\omega)$ behaves as a function of $\omega$, and interpret the phenomenon of resonance in terms of the location of the poles of $X(s)$ in the complex $s$-plane.
\end{enumerate}
\end{problem}

\begin{solution}
We analyze the forced mass--spring--damper system using the Laplace transform to convert the differential equation into an algebraic equation in the complex variable $s$. This illustrates the general principle of this chapter: transforms turn differentiation into multiplication, making spectral properties (such as natural frequencies and damping) visible as poles of a rational function.

\medskip

\noindent\textbf{Step 1: Laplace transform of the ODE.}

We start from
\[
m x''(t) + c x'(t) + k x(t) = F_0 \cos(\omega t), \qquad t>0,
\]
with initial conditions
\[
x(0) = 0, \qquad x'(0) = 0.
\]

Let $X(s) = \mathcal{L}\{x\}(s)$. We use the standard formulas
\[
\mathcal{L}\{x'(t)\} = s X(s) - x(0), \qquad
\mathcal{L}\{x''(t)\} = s^2 X(s) - s x(0) - x'(0),
\]
and
\[
\mathcal{L}\{\cos(\omega t)\} = \frac{s}{s^2 + \omega^2}.
\]
Applying the Laplace transform to both sides of the ODE gives
\[
m\bigl(s^2 X(s) - s x(0) - x'(0)\bigr)
+ c\bigl(s X(s) - x(0)\bigr)
+ k X(s)
= F_0 \frac{s}{s^2 + \omega^2}.
\]
Using $x(0)=x'(0)=0$, this simplifies to
\[
(m s^2 + c s + k)\, X(s) = F_0 \frac{s}{s^2 + \omega^2}.
\]
Thus
\begin{equation}\label{eq:Xs-basic}
X(s) = \frac{F_0\, s}{(s^2 + \omega^2)(m s^2 + c s + k)}.
\end{equation}
We have converted the second-order ODE into an algebraic equation for $X(s)$.

\medskip

\noindent\textbf{Step 2: Poles and their interpretation.}

The denominator of $X(s)$ is
\[
(s^2 + \omega^2)(m s^2 + c s + k).
\]
The factor $s^2 + \omega^2$ has roots
\[
s = \pm i \omega,
\]
giving two purely imaginary poles. These come directly from the forcing term $\cos(\omega t)$.

The other factor is the characteristic polynomial
\[
m s^2 + c s + k = 0.
\]
Its roots are
\[
s = r_{1,2} = \frac{-c \pm \sqrt{c^2 - 4mk}}{2m}.
\]
Under the underdamped assumption $c^2 < 4mk$, the discriminant is negative, so
\[
r_{1,2} = -\alpha \pm i \beta
\]
for some $\alpha>0$ and $\beta>0$. Thus these two poles lie strictly in the left half of the complex $s$-plane.

Altogether, $X(s)$ has four poles:
\[
s = r_1,\quad s = r_2 \quad (\text{with } \Re r_{1,2}<0), 
\qquad s = i\omega,\quad s = -i\omega.
\]

To interpret them, recall that the inverse Laplace transform of $1/(s-a)$ is $e^{at}$, and the inverse transform of $1/(s^2+\omega^2)$ is a combination of $\sin(\omega t)$ and $\cos(\omega t)$. Thus:
\begin{itemize}
  \item Poles with negative real part, such as $r_{1,2}$, produce terms proportional to $e^{r_1 t}$ and $e^{r_2 t}$, which decay to zero as $t\to\infty$. These terms are the \emph{transient} response, determined by the homogeneous dynamics and initial conditions.
  \item Purely imaginary poles, such as $\pm i\omega$, produce undamped oscillations at frequency $\omega$ that do not decay. These correspond to \emph{persistent} oscillations, here driven by the periodic forcing.
\end{itemize}
We therefore expect $x(t)$ to consist of a transient part that dies out and a steady-state part oscillating at the forcing frequency $\omega$.

\medskip

\noindent\textbf{Step 3: Partial fractions and decomposition into transient and steady state.}

We now decompose $X(s)$ in a way that separates the contributions of the two quadratic factors. Because both factors are quadratic, it is natural to write
\begin{equation}\label{eq:PF}
X(s) = \frac{A s + B}{s^2 + \omega^2}
+ \frac{C s + D}{m s^2 + c s + k},
\end{equation}
for some constants $A,B,C,D$ depending on $m,c,k,F_0,\omega$.

Multiplying both sides by the common denominator $(s^2 + \omega^2)(m s^2 + c s + k)$, we obtain the identity
\[
F_0\, s
= (A s + B)(m s^2 + c s + k)
+ (C s + D)(s^2 + \omega^2),
\]
valid for all $s$. Expanding both products gives
\[
(A s + B)(m s^2 + c s + k)
= A m s^3 + (A c + B m) s^2 + (A k + B c) s + B k,
\]
and
\[
(C s + D)(s^2 + \omega^2)
= C s^3 + D s^2 + C \omega^2 s + D \omega^2.
\]
Adding and matching coefficients with $F_0 s$ (which has no $s^3$, $s^2$, or constant term) yields the system
\begin{align*}
s^3: &\quad A m + C = 0,\\
s^2: &\quad A c + B m + D = 0,\\
s^1: &\quad A k + B c + C \omega^2 = F_0,\\
s^0: &\quad B k + D \omega^2 = 0.
\end{align*}

For our purposes, we do not actually need explicit formulas for all four constants, because we know the qualitative form of the inverse transform. From \eqref{eq:PF},
\[
\mathcal{L}^{-1}\!\left\{\frac{A s + B}{s^2 + \omega^2}\right\}(t)
= A \cos(\omega t) + \frac{B}{\omega} \sin(\omega t),
\]
and the second term is the response associated with the homogeneous polynomial $m s^2 + c s + k$. In the underdamped case, its inverse transform is a linear combination of $e^{-\alpha t}\cos(\beta t)$ and $e^{-\alpha t}\sin(\beta t)$, hence decays exponentially to zero as $t\to\infty$.

Thus we can write
\[
x(t) = x_{\mathrm{tr}}(t) + x_{\mathrm{ss}}(t),
\]
where the \emph{transient} term is
\[
x_{\mathrm{tr}}(t) = \mathcal{L}^{-1}\!\left\{\frac{C s + D}{m s^2 + c s + k}\right\}(t),
\]
which decays to $0$, and the \emph{steady-state} term is
\[
x_{\mathrm{ss}}(t) 
= \mathcal{L}^{-1}\!\left\{\frac{A s + B}{s^2 + \omega^2}\right\}(t)
= A \cos(\omega t) + \frac{B}{\omega} \sin(\omega t).
\]

Any linear combination of $\cos(\omega t)$ and $\sin(\omega t)$ can be rewritten as a single sinusoid with phase shift. Using the identity
\[
a \cos(\theta) + b \sin(\theta)
= \sqrt{a^2 + b^2}\,\cos\bigl(\theta - \delta\bigr), \quad
\delta = \arctan\!\left(\frac{b}{a}\right),
\]
we obtain
\[
x_{\mathrm{ss}}(t)
= R(\omega)\cos\bigl(\omega t - \delta(\omega)\bigr),
\]
where
\[
R(\omega) = \sqrt{A^2 + \left(\frac{B}{\omega}\right)^2}
\]
is the steady-state amplitude and $\delta(\omega)$ is the phase shift.

We still need to compute $R(\omega)$ in terms of the physical parameters.

\medskip

\noindent\textbf{Step 4: Computing the steady-state amplitude $R(\omega)$.}

There are two efficient ways to find $R(\omega)$:

\begin{itemize}
  \item One can solve explicitly for $A$ and $B$ from the linear system above and substitute into $R(\omega
(\omega)}$, but this is somewhat tedious.
  \item A cleaner route is to use the transfer-function viewpoint (or an equivalent complex-exponential computation), which we outline next.
\end{itemize}

Rewrite \eqref{eq:Xs-basic} as
\[
X(s)
= \frac{1}{m s^2 + c s + k}\;\cdot\;\frac{F_0 s}{s^2 + \omega^2}
= H(s)\,\mathcal{L}\{F_0 \cos(\omega t)\}(s),
\]
where
\[
H(s) = \frac{1}{m s^2 + c s + k}
\]
is the system’s transfer function from input force $F_{\text{ext}}$ to output displacement $x$.

For a sinusoidal input $F_{\text{ext}}(t) = F_0\cos(\omega t)$, the long-time (steady-state) response of a stable linear system is sinusoidal at the same frequency, with complex \emph{gain} $H(i\omega)$:
\[
x_{\mathrm{ss}}(t)
= \Re\Bigl\{H(i\omega)\,F_0 e^{i\omega t}\Bigr\}.
\]
Thus the steady-state amplitude is
\[
R(\omega) = F_0\,|H(i\omega)|.
\]

Now
\[
H(i\omega) = \frac{1}{m (i\omega)^2 + c (i\omega) + k}
= \frac{1}{k - m\omega^2 + i c \omega}.
\]
Hence
\[
|H(i\omega)|
= \frac{1}{\sqrt{(k - m\omega^2)^2 + (c\omega)^2}},
\]
and
\[
R(\omega) = \frac{F_0}{\sqrt{(k - m\omega^2)^2 + c^2 \omega^2}}.
\]

This agrees with the amplitude obtained by writing
\[
x_{\mathrm{ss}}(t) = A \cos(\omega t) + \frac{B}{\omega} \sin(\omega t)
= R(\omega)\cos\bigl(\omega t - \delta(\omega)\bigr),
\]
with
\[
R(\omega) = \sqrt{A^2 + \left(\frac{B}{\omega}\right)^2}.
\]

This completes part (b): the steady-state solution has the form
\[
x_{\mathrm{ss}}(t) = R(\omega)\cos\bigl(\omega t - \delta(\omega)\bigr),
\quad
R(\omega) = \frac{F_0}{\sqrt{(k - m\omega^2)^2 + c^2 \omega^2}}.
\]

\medskip

\noindent\textbf{Step 5: Behavior of $R(\omega)$ and resonance.}

We now discuss how $R(\omega)$ behaves as a function of $\omega$ and interpret resonance via the poles of $X(s)$ (part (c)).

\medskip

\noindent\emph{(i) Shape of $R(\omega)$ for $c>0$.}

Recall
\[
R(\omega) = \frac{F_0}{\sqrt{(k - m\omega^2)^2 + c^2 \omega^2}},
\qquad \omega\ge 0.
\]

Basic features:
\begin{itemize}
  \item At zero frequency,
  \[
  R(0) = \frac{F_0}{k},
  \]
  which is just the static deflection of a spring under a constant force.
  \item As $\omega\to\infty$,
  \[
  (k - m\omega^2)^2 \sim m^2\omega^4,
  \]
  so $R(\omega)\sim F_0/(m\omega^2)\to 0$. Very rapid forcing produces little displacement.
  \item For $0<c\ll 1$, $R(\omega)$ has a pronounced peak near the undamped natural frequency
  \[
  \omega_0 = \sqrt{\frac{k}{m}}.
  \]
\end{itemize}

To locate the maximum more precisely, maximize $R(\omega)$ over $\omega>0$. Since $F_0$ is constant, this is equivalent to minimizing
\[
D(\omega) := (k - m\omega^2)^2 + c^2\omega^2.
\]
Differentiating and setting $D'(\omega)=0$ gives
\[
D'(\omega)
= 2(k - m\omega^2)(-2m\omega) + 2c^2\omega
= 2\omega\bigl(-2m(k - m\omega^2) + c^2\bigr)
= 0.
\]
Aside from the trivial solution $\omega=0$, we obtain
\[
-2m(k - m\omega^2) + c^2 = 0
\quad\Rightarrow\quad
2m^2\omega^2 = 2mk - c^2
\quad\Rightarrow\quad
\omega^2 = \frac{k}{m} - \frac{c^2}{2m^2}.
\]
Thus the \emph{resonant frequency} is
\[
\omega_{\mathrm{res}}
= \sqrt{\omega_0^2 - \frac{c^2}{2m^2}},
\]
which satisfies $\omega_{\mathrm{res}}<\omega_0$ and approaches $\omega_0$ as $c\to 0^+$. So for small but nonzero damping, $R(\omega)$ has a finite peak slightly below the undamped natural frequency.

\medskip

\noindent\emph{(ii) The undamped case $c=0$ and resonance.}

If we set $c=0$ in the ODE
\[
m x''(t) + k x(t) = F_0 \cos(\omega t),
\]
then
\[
X(s) = \frac{F_0\, s}{(s^2 + \omega^2)(m s^2 + k)}
= \frac{F_0\, s}{(s^2 + \omega^2)\bigl(m s^2 + k\bigr)}.
\]
The system’s transfer function becomes
\[
H(s) = \frac{1}{m s^2 + k},
\]
with poles at
\[
s = \pm i\omega_0, \qquad \omega_0 = \sqrt{\frac{k}{m}}.
\]
These poles now lie exactly on the imaginary axis. The forcing term still contributes poles at $s=\pm i\omega$ from $s^2+\omega^2$.

The steady-state amplitude formula reduces to
\[
R(\omega)
= \frac{F_0}{|k - m\omega^2|}
= \frac{F_0}{m|\omega_0^2 - \omega^2|}.
\]
As $\omega\to\omega_0$, the denominator tends to $0$, so $R(\omega)\to\infty$: the steady-state amplitude is unbounded. In the time domain, at exact resonance $\omega=\omega_0$, the particular solution contains a term that grows linearly in $t$, such as
\[
x(t) \sim \frac{F_0}{2m\omega_0}\,t\sin(\omega_0 t),
\]
illustrating classical resonance.

\medskip

\noindent\emph{(iii) Pole motion and resonance in the $s$-plane.}

From the transfer function perspective, the frequency response is
\[
H(i\omega) = \frac{1}{m(i\omega)^2 + c(i\omega) + k}
= \frac{1}{k - m\omega^2 + i c\omega}.
\]
Its magnitude $|H(i\omega)|$ is largest when $i\omega$ lies closest (in the complex plane) to a pole of $H(s)$.

\begin{itemize}
  \item For $c>0$, the poles
  \[
  r_{1,2} = \frac{-c \pm \sqrt{c^2 - 4mk}}{2m}
  \]
  lie in the left half-plane, at $-\alpha\pm i\beta$ ($\alpha>0$). The imaginary parts $\pm\beta$ are close to $\pm\omega_0$ when $c$ is small. The point $i\omega$ on the imaginary axis is closest to these poles when $\omega\approx\beta\approx\omega_0$, so $|H(i\omega)|$ and thus $R(\omega)$ attain a finite but large maximum near $\omega_0$.
  \item As $c\to 0^+$, the poles $r_{1,2}$ move horizontally toward the imaginary axis and converge to $\pm i\omega_0$. In the limit $c=0$, they sit exactly on the imaginary axis. When the forcing frequency matches $\omega_0$, the point $s=i\omega$ coincides with a pole of $H(s)$, and the gain $|H(i\omega)|$ becomes unbounded. This pole collision on the imaginary axis is the Laplace-domain picture of resonance.
\end{itemize}

Thus, resonance corresponds to the forcing frequency aligning with the imaginary part of the system’s poles. Damping pushes the poles into the left half-plane, limiting the peak amplitude; removing damping allows the poles to lie on the imaginary axis, leading to unbounded growth at resonance.

\end{solution}

% ===== Example 5: Wave Equation on a String via Fourier Series (inquiry-based) =====
\begin{problem}[Wave Equation on a String via Fourier Series]
A taut string of length $L$ is fixed at both ends and is free to vibrate transversely. Its vertical displacement from equilibrium is modeled by a function $u(x,t)$, where $x\in[0,L]$ measures position along the string and $t\ge 0$ is time. The motion is governed (under standard assumptions) by the one-dimensional wave equation
\[
u_{tt} = c^2 u_{xx},
\]
where $c>0$ is the wave speed. The endpoints are held fixed, and the initial shape and initial velocity are prescribed.

In this problem you will discover how separation of variables and Fourier \emph{sine} series naturally arise, and how they convert the partial differential equation into a family of independent ordinary differential equations for the vibration modes.

\smallskip

We consider the initial–boundary value problem
\[
\begin{cases}
u_{tt}(x,t) = c^2 u_{xx}(x,t), & 0<x<L,\ t>0,\\[0.3em]
u(0,t) = 0,\quad u(L,t) = 0, & t\ge 0,\\[0.3em]
u(x,0) = f(x),\quad u_t(x,0)=g(x), & 0\le x\le L,
\end{cases}
\]
where $f$ and $g$ are given functions.

\begin{enumerate}[(a)]
  \item \textbf{Separation of variables and the eigenvalue problem in $x$.}  

  Assume a separated solution of the form $u(x,t)=X(x)T(t)$, with $X$ not identically zero and $T$ not identically zero.
  \begin{enumerate}[(i)]
    \item Substitute $u(x,t)=X(x)T(t)$ into the wave equation and show that you obtain an equation of the form
    \[
    \frac{T''(t)}{c^2 T(t)} = \frac{X''(x)}{X(x)} = -\lambda,
    \]
    for some constant $\lambda\in\mathbb{R}$ (the separation constant).
    \item Write down the resulting ordinary differential equation for $X$ together with the boundary conditions inherited from $u(0,t)=u(L,t)=0$.
  \end{enumerate}
  Hint: Carefully divide by $c^2 X(x)T(t)$ and argue why both sides must be equal to the same constant.

  \item \textbf{Finding the allowed spatial modes.}

  You have obtained the boundary value problem
  \[
  X''(x) + \lambda X(x) = 0,\quad X(0)=0,\ X(L)=0.
  \]
  \begin{enumerate}[(i)]
    \item Consider separately the three cases $\lambda<0$, $\lambda=0$, and $\lambda>0$. For each case, solve the ODE for $X(x)$ and apply the boundary conditions. Show that nontrivial solutions $X$ exist \emph{only} for a discrete set of values $\lambda=\lambda_n$.
    \item Determine these eigenvalues $\lambda_n$ explicitly and show that the corresponding eigenfunctions can be chosen as
    \[
    X_n(x) = \sin\!\left(\frac{n\pi x}{L}\right),\quad n=1,2,3,\dots
    \]
  \end{enumerate}
  Hint: For $\lambda>0$, it is convenient to write $\lambda=\mu^2$ with $\mu>0$. Recall standard solutions of $y''+\mu^2 y=0$.

  \item \textbf{Time evolution of each mode and quantized frequencies.}

  For each eigenvalue $\lambda_n$, the corresponding time factor $T_n(t)$ satisfies
  \[
  T_n''(t) + c^2 \lambda_n\, T_n(t) = 0.
  \]
  \begin{enumerate}[(i)]
    \item Using your expression for $\lambda_n$, solve this ODE and show that
    \[
    T_n(t) = A_n\cos(\omega_n t) + B_n\sin(\omega_n t),
    \]
    for constants $A_n,B_n$, where you should determine $\omega_n$ in terms of $c$, $L$, and $n$.
    \item Interpret $\omega_n$ as a natural frequency of the string, and briefly explain in words how the boundary conditions lead to “quantized” frequencies (that is, only certain discrete frequencies are allowed).
  \end{enumerate}
  Hint: Recognize the standard harmonic oscillator equation $y'' + \omega^2 y = 0$.

  \item \textbf{Superposition and Fourier sine series for the full solution.}

  The separated solutions corresponding to each $n$ can be written as
  \[
  u_n(x,t) = \bigl(A_n\cos(\omega_n t) + B_n\sin(\omega_n t)\bigr) \sin\!\left(\frac{n\pi x}{L}\right).
  \]
  Since the wave equation is linear, we seek a general solution as a superposition
  \[
  u(x,t) = \sum_{n=1}^\infty \bigl(A_n\cos(\omega_n t) + B_n\sin(\omega_n t)\bigr) \sin\!\left(\frac{n\pi x}{L}\right).
  \]
  \begin{enumerate}[(i)]
    \item Use the initial displacement condition $u(x,0)=f(x)$ to derive an expression for $f(x)$ as a Fourier sine series. Express the coefficients $A_n$ as integrals involving $f$ and $\sin(n\pi x/L)$.
    \item Similarly, use the initial velocity condition $u_t(x,0)=g(x)$ to obtain a Fourier sine series expansion for $g(x)$ and express the coefficients $B_n$ as integrals involving $g$.
  \end{enumerate}
  Hint: Recall the orthogonality relation
  \[
  \int_0^L \sin\!\left(\frac{n\pi x}{L}\right)\sin\!\left(\frac{m\pi x}{L}\right)\,dx
  =\begin{cases}
  0,& n\ne m,\\[0.3em]
  \dfrac{L}{2},& n=m.
  \end{cases}
  \]

  \item \textbf{Extensions and variations.}
  \begin{enumerate}[(i)]
    \item Suppose the string is initially at rest, so $g(x)\equiv 0$, but has initial shape $f(x)=x(L-x)$ on $[0,L]$. Write down (without fully evaluating the integrals) the explicit Fourier sine series coefficients that would determine $u(x,t)$ for this case.
    \item How would the analysis change if the right endpoint of the string were \emph{free} instead of fixed? That is, if $u(0,t)=0$ and $u_x(L,t)=0$ for all $t\ge 0$? Describe qualitatively (without solving in full detail) what kinds of eigenfunctions in $x$ you would expect and how the allowed frequencies $\omega_n$ might differ from the fixed–fixed case.
  \end{enumerate}
  Hint: For a free end at $x=L$, the boundary condition expresses that the spatial derivative $u_x$ vanishes there, so the eigenfunctions must satisfy $X'(L)=0$ rather than $X(L)=0$.
\end{enumerate}
\end{problem}

% ===== Example 5: Wave Equation on a String via Fourier Series (full solution) =====
\begin{problem}[Wave Equation on a String via Fourier Series]
Consider the wave equation for a taut string of length $L$ with fixed endpoints:
\[
\begin{cases}
u_{tt}(x,t) = c^2 u_{xx}(x,t), & 0<x<L,\ t>0,\\[0.3em]
u(0,t)=0,\quad u(L,t)=0, & t\ge 0,\\[0.3em]
u(x,0)=f(x),\quad u_t(x,0)=g(x), & 0\le x\le L,
\end{cases}
\]
where $c>0$ is constant and $f,g$ are given functions (assume they are piecewise smooth so that Fourier series converge in the usual sense).

(a) Use separation of variables to find the eigenfunctions and eigenvalues of the associated spatial problem and the corresponding time-dependent factors.

(b) Show that the solution can be written as a Fourier sine series
\[
u(x,t) = \sum_{n=1}^\infty \bigl(a_n\cos(\omega_n t) + b_n\sin(\omega_n t)\bigr)\sin\!\left(\frac{n\pi x}{L}\right),
\]
and derive formulas for the coefficients $a_n$ and $b_n$ in terms of $f$ and $g$. Identify the natural frequencies $\omega_n$.

\end{problem}

\begin{solution}
We solve the initial–boundary value problem
\[
u_{tt} = c^2 u_{xx},\quad 0<x<L,\ t>0,
\]
with fixed ends $u(0,t)=u(L,t)=0$ and initial conditions $u(x,0)=f(x)$, $u_t(x,0)=g(x)$, by separation of variables and Fourier sine series. The key idea is to convert the partial differential equation into a family of ordinary differential equations for the amplitudes of spatial modes.

\medskip

\textbf{1. Separation of variables.}
We seek separated solutions of the form
\[
u(x,t) = X(x)T(t),
\]
with $X$ and $T$ not identically zero. Substituting into the wave equation gives
\[
X(x)T''(t) = c^2 X''(x)T(t).
\]
Assuming $X(x)T(t)\neq 0$, we divide by $c^2 X(x)T(t)$:
\[
\frac{T''(t)}{c^2 T(t)} = \frac{X''(x)}{X(x)}.
\]
The left-hand side depends only on $t$, and the right-hand side only on $x$, so both must be equal to a constant, which we denote by $-\lambda$:
\[
\frac{T''(t)}{c^2 T(t)} = \frac{X''(x)}{X(x)} = -\lambda.
\]
This gives two ordinary differential equations,
\[
X''(x) + \lambda X(x) = 0,\qquad
T''(t) + c^2\lambda T(t) = 0.
\]
The boundary conditions $u(0,t)=u(L,t)=0$ become
\[
X(0)T(t)=0,\quad X(L)T(t)=0\quad\text{for all }t\ge 0.
\]
Since we seek nontrivial time dependence $T(t)\not\equiv 0$, these imply
\[
X(0)=0,\quad X(L)=0.
\]
Thus we have a boundary value problem for $X$:
\[
X''(x) + \lambda X(x) = 0,\quad X(0)=0,\ X(L)=0.
\]

\medskip

\textbf{2. Spatial eigenvalue problem.}
We analyze the possible values of $\lambda$.

\emph{Case 1: $\lambda<0$.} Write $\lambda=-\mu^2$ with $\mu>0$. Then
\[
X''(x) - \mu^2 X(x) = 0,
\]
whose general solution is $X(x)=C_1 e^{\mu x} + C_2 e^{-\mu x}$. The boundary condition $X(0)=0$ gives
\[
C_1 + C_2 = 0 \quad\Rightarrow\quad C_2 = -C_1,
\]
so $X(x)=C_1(e^{\mu x}-e^{-\mu x})=2C_1\sinh(\mu x)$. The condition $X(L)=0$ then implies $\sinh(\mu L)=0$. Since $\mu>0$, we have $\sinh(\mu L)\neq 0$, so $C_1=0$, and $X$ is the trivial solution. Thus there are no nontrivial solutions for $\lambda<0$.

\emph{Case 2: $\lambda=0$.} The ODE is $X''(x)=0$, with general solution $X(x)=C_1x + C_2$. The conditions $X(0)=0$ and $X(L)=0$ give $C_2=0$ and $C_1 L=0$, so $C_1=0$. Again, only the trivial solution exists.

\emph{Case 3: $\lambda>0$.} Write $\lambda=\mu^2$ with $\mu>0$. The ODE becomes
\[
X''(x) + \mu^2 X(x)=0,
\]
whose general solution is
\[
X(x) = A\cos(\mu x) + B\sin(\mu x).
\]
The boundary condition $X(0)=0$ implies $A=0$, so $X(x)=B\sin(\mu x)$. The condition $X(L)=0$ then gives
\[
B\sin(\mu L)=0.
\]
To have a nontrivial solution $X\not\equiv 0$, we require $B\neq 0$, hence
\[
\sin(\mu L)=0 \quad\Rightarrow\quad \mu L = n\pi,\quad n=1,2,3,\dots
\]
Thus
\[
\mu_n = \frac{n\pi}{L},\qquad \lambda_n = \mu_n^2 = \frac{n^2\pi^2}{L^2}.
\]
The corresponding eigenfunctions can be chosen (up to constant multiples) as
\[
X_n(x) = \sin\!\left(\frac{n\pi x}{L}\right),\quad n=1,2,3,\dots
\]

We have therefore discretized the spatial part: only discrete ``modes'' indexed by $n$ are compatible with the fixed–end boundary conditions.

\medskip

\textbf{3. Time-dependent factors and natural frequencies.}
For each eigenvalue $\lambda_n$, the time equation is
\[
T_n''(t) + c^2 \lambda_n T_n(t) = 0.
\]
Substituting $\lambda_n = n^2\pi^2/L^2$ gives
\[
T_n''(t) + \left(\frac{n\pi c}{L}\right)^2 T_n(t) = 0.
\]
This is the standard harmonic oscillator equation $y'' + \omega_n^2 y=0$ with
\[
\omega_n = \frac{n\pi c}{L}.
\]
Hence the general solution is
\[
T_n(t) = A_n \cos(\omega_n t) + B_n \sin(\omega_n t),
\]
for constants $A_n,B_n$.

Each separated solution corresponding to the $n$-th mode is therefore
\[
u_n(x,t) = T_n(t)X_n(x)
= \bigl(A_n\cos(\omega_n t) + B_n\sin(\omega_n t)\bigr)\sin\!\left(\frac{n\pi x}{L}\right).
\]
The quantities $\omega_n$ are the natural frequencies of the string. Because the spatial boundary conditions restrict $\mu$ to values $\mu_n = n\pi/L$, only the discrete frequencies $\omega_n = n\pi c/L$ are allowed: the boundary conditions \emph{quantize} the possible vibration frequencies.

\medskip

\textbf{4. Superposition and Fourier sine series.}
The wave equation is linear and homogeneous, so any linear combination of solutions is again a solution. We therefore form the general solution as a superposition of modes:
\[
u(x,t) = \sum_{n=1}^\infty \bigl(A_n\cos(\omega_n t) + B_n\sin(\omega_n t)\bigr)\sin\!\left(\frac{n\pi x}{L}\right),
\]
with $\omega_n = n\pi c/L$. The task is now to choose the coefficients $A_n$ and $B_n$ to satisfy the initial conditions.

\emph{Initial displacement.} At $t=0$ we have
\[
u(x,0) = \sum_{n=1}^\infty \bigl(A_n\cos(0) + B_n\sin(0)\bigr)\sin\!\left(\frac{n\pi x}{L}\right)
= \sum_{n=1}^\infty A_n \sin\!\left(\frac{n\pi x}{L}\right).
\]
The initial condition $u(x,0)=f(x)$ becomes
\[
f(x) = \sum_{n=1}^\infty A_n \sin\!\left(\frac{n\pi x}{L}\right),
\]
which is precisely a Fourier sine series expansion of $f$ on $(0,L)$.

The sine functions are orthogonal on $[0,L]$:
\[
\int_0^L \sin\!\left(\frac{n\pi x}{L}\right)\sin\!\left(\frac{m\pi x}{L}\right)\,dx
=
\begin{cases}
0,& n\neq m,\\[0.3em]
\dfrac{L}{2},& n=m.
\end{cases}
\]
Multiply the series for $f(x)$ by $\sin(m\pi x/L)$ and integrate from $0$ to $L$:
\[
\int_0^L f(x)\sin\!\left(\frac{m\pi x}{L}\right)\,dx
= \sum_{n=1}^\infty A_n \int_0^L \sin\!\left(\frac{n\pi x}{L}\right)\sin\!\left(\frac{m\pi x}{L}\right)\,dx
= A_m \frac{L}{2}.
\]
Thus
\[
A_m = \frac{2}{L} \int_0^L f(x)\sin\!\left(\frac{m\pi x}{L}\right)\,dx,\quad m=1,2,3,\dots
\]
So the coefficients $A_n$ are exactly the Fourier sine coefficients of $f$.

\emph{Initial velocity.} Differentiate $u(x,t)$ with respect to $t$:
\[
u_t(x,t) = \sum_{n=1}^\infty \bigl(-A_n\omega_n\sin(\omega_n t) + B_n\omega_n\cos(\omega_n t)\bigr)\sin\!\left(\frac{n\pi x}{L}\right).
\]
At $t=0$ this becomes
\[
u_t(x,0) = \sum_{n=1}^\infty B_n\omega_n \sin\!\left(\frac{n\pi x}{L}\right),
\]
since $\sin(0)=0$ and $\cos(0)=1$. The initial condition $u_t(x,0)=g(x)$ thus yields
\[
g(x) = \sum_{n=1}^\infty B_n\omega_n \sin\!\left(\frac{n\pi x}{L}\right).
\]
This is again a Fourier sine series, now for $g(x)$. Using orthogonality as before,
\[
\int_0^L g(x)\sin\!\left(\frac{m\pi x}{L}\right)\,dx
= \sum_{n=1}^\infty B_n\omega_n \int_0^L \sin\!\left(\frac{n\pi x}{L}\right)\sin\!\left(\frac{m\pi x}{L}\right)\,dx
= B_m\omega_m\frac{L}{2}.
\]
Hence
\[
B_m\omega_m = \frac{2}{L} \int_0^L g(x)\sin\!\left(\frac{m\pi x}{L}\right)\,dx,
\]
so that
\[
B_m = \frac{2}{L\omega_m} \int_0^L g(x)\sin\!\left(\frac{m\pi x}{L}\right)\,dx.
\]
Recalling $\omega_m = m\pi c/L$, we can write this as
\[
B_m = \frac{2}{L}\cdot \frac{L}{m\pi c} \int_0^L g(x)\sin\!\left(\frac{m\pi x}{L}\right)\,dx
= \frac{2}{m\pi c} \int_0^L g(x)\sin\!\left(\frac{m\pi x}{L}\right)\,dx.
\]

\medskip

\textbf{5. Final form of the solution.}
Putting everything together, the solution of the initial–boundary value problem is
\[
u(x,t) = \sum_{n=1}^\infty \left[
\left(\frac{2}{L} \int_0^L f(s)\sin\!\left(\frac{n\pi s}{L}\right)\,ds \right)\cos\!\left(\frac{n\pi c}{L} t\right)
+
\left(\frac{2}{n\pi c} \int_0^L g(s)\sin\!\left(\frac{n\pi s}{L}\right)\,ds \right)\sin\!\left(\frac{n\pi c}{L} t\right)
\right]
\sin\!\left(\frac{n\pi x}{L}\right).
\]
Equivalently, in the more compact notation requested in the problem,
\[
u(x,t) = \sum_{n=1}^\infty \bigl(a_n\cos(\omega_n t) + b_n\sin(\omega_n t)\bigr)\sin\!\left(\frac{n\pi x}{L}\right),
\]
where
\[
\omega_n = \frac{n\pi c}{L},\qquad
a_n = \frac{2}{L} \int_0^L f(x)\sin\!\left(\frac{n\pi x}{L}\right)\,dx,\qquad
b_n = \frac{2}{n\pi c} \int_0^L g(x)\sin\!\left(\frac{n\pi x}{L}\right)\,dx.
\]

\medskip

\textbf{Conceptual remark.}
This example illustrates the central theme of the section “From Differential to Algebraic Equations with FT, FS and LT.” The spatial Fourier sine series expansion of $u(x,t)$ effectively transforms the partial differential equation into an infinite collection of decoupled second-order ordinary differential equations for the time-dependent coefficients $a_n$ and $b_n$. In the “coefficient space” indexed by $n$, the wave equation reduces to algebraic relations between $\lambda_n$, $\omega_n$, and the Fourier coefficients of the initial data. This is the essence of using Fourier series (and more generally Fourier transforms and Laplace transforms) to turn differential operators into simpler multiplication operators on mode amplitudes.
\end{solution}



%========================================
% Part II: Differential Equations
%========================================
\part{Differential Equations}

\chapter{Ordinary Differential Equations}

\section{ODEs: Simple Cases}
% --- Narrative plan (auto-generated) ---
% In this section we study the simplest ordinary differential equations and the basic techniques for solving them. These include separable equations and linear equations with constant coefficients, both in first and second order. Although these examples may look elementary, they already capture important phenomena such as decay to equilibrium, approach to a carrying capacity, and oscillations with and without damping. The goal is to learn how to translate a verbal description of a process into an equation, and then how to analyze and solve that equation systematically.
%
% Simple ODEs lie at the foundation of applied mathematics. Many partial differential equations, when restricted to special classes of solutions such as steady states or traveling waves, reduce to ODEs of exactly the types considered here. Linear ODEs with constant coefficients are also the natural setting in which exponential and trigonometric functions emerge as building blocks, which later reappear in Fourier series, Laplace transforms, and the complex-analytic study of differential equations. The qualitative ideas we develop—for example, stability of equilibria and long-time behavior—are the first steps toward the modern theory of dynamical systems, which underpins much of mathematical modeling in physics, biology, and engineering.
%
% As you work through the examples, focus on recognizing the structure of an equation (separable, linear, autonomous) and choosing an appropriate method. These pattern-recognition skills are what allow you to tackle more complicated models later on, whether they arise as reduced forms of PDEs, as linearizations of nonlinear systems, or as test problems for numerical methods.

% ===== Example 1: Exponential Decay and Half-Life (inquiry-based) =====
\begin{problem}[Exponential Decay and Half-Life]
Many physical and biological processes can be modeled by assuming that a quantity decreases at a rate proportional to its current amount. Examples include the mass of a radioactive isotope, the concentration of a drug in the bloodstream, or the charge on a discharging capacitor. In this problem, you will build the mathematical model for such a process, solve the corresponding differential equation, and connect the model to the experimentally measurable notion of \emph{half-life}. You will then use this model to answer a practical question: how long does it take for the quantity to fall below a desired threshold?

Suppose $Q(t)$ denotes the amount of a substance (for example, mass in grams, or concentration in mg/L) present at time $t$, measured in hours.

\smallskip

(a) A common modeling assumption is that the instantaneous rate of change of $Q(t)$ is proportional to the current amount $Q(t)$, and that the substance is \emph{decaying}. Translate this verbal statement into a differential equation for $Q(t)$.

\emph{Questions to guide you:}  
\quad(i) If the rate of change is ``proportional to'' $Q(t)$, what algebraic form should $Q'(t)$ have?  
\quad(ii) What sign should the proportionality constant have for a decaying process, and what are its units?  
% Hint: Start with $Q'(t) = k\,Q(t)$ for some constant $k$. Then think about what sign $k$ must have if $Q$ is to decrease in time.

\smallskip

(b) Now solve your differential equation from part (a) in general. That is, find the general formula for $Q(t)$ in terms of the initial amount $Q(0)$ and any parameters you introduced.

Hint: Use separation of variables: rewrite the equation so that all factors involving $Q$ are on one side and all factors involving $t$ are on the other, then integrate both sides.

\smallskip

(c) Experimentally, it is common to measure the \emph{half-life} of a decaying substance. The half-life, denoted $T_{1/2}$, is defined as the time it takes for the amount of substance to decrease to one-half of its initial value. Using your solution $Q(t)$ from part (b), express the half-life $T_{1/2}$ in terms of the decay constant $k$ (the proportionality constant from part (a)).

More concretely, suppose $Q(0) = Q_0$. By the definition of half-life,
\[
Q(T_{1/2}) = \frac{1}{2} Q_0.
\]
Use this condition and your formula for $Q(t)$ to solve for $T_{1/2}$ in terms of $k$. Then, solve the resulting equation for $k$ in terms of $T_{1/2}$.

% Hint: You should get an equation involving an exponential function and a factor $\frac12$. Take natural logarithms to solve for $T_{1/2}$ or $k$.

\smallskip

(d) Suppose you have an initial amount $Q(0) = Q_0$ of a radioactive isotope with known half-life $T_{1/2}$. You would like to know how long it takes before the amount falls below a safety threshold $Q_{\text{safe}}$ (where $0 < Q_{\text{safe}} < Q_0$).

(i) Using your explicit solution $Q(t)$ and the relationship between $k$ and $T_{1/2}$ from part (c), derive a formula for the time $t_{\text{safe}}$ at which $Q(t_{\text{safe}}) = Q_{\text{safe}}$. Your answer should be expressed in terms of $Q_0$, $Q_{\text{safe}}$, and $T_{1/2}$.

(ii) Explain briefly (in one or two sentences) how you would use this formula in practice if $Q_0$, $Q_{\text{safe}}$, and $T_{1/2}$ are known from measurements.

% Hint: First, solve the equation $Q(t) = Q_{\text{safe}}$ for $t$ in terms of $k$, $Q_0$, and $Q_{\text{safe}}$. Then substitute your expression for $k$ from part (c).

\smallskip

(e) \textbf{Extensions and ``what if'' questions.}

(i) What changes in the differential equation and its solution if, instead of decaying, the quantity is \emph{growing} at a rate proportional to its current amount? State the modified differential equation and its general solution, and interpret the sign of the constant.

(ii) In some practical situations, the decaying substance is also being supplied at a constant rate $r>0$ (for instance, a drug is infused into the bloodstream at a constant rate while the body simultaneously eliminates it). Modify the differential equation from part (a) to include a constant input rate $r$, and write down the new equation. Without solving it completely, briefly predict (in words or with a sketch) how the long-term behavior of $Q(t)$ should differ from the pure decay case.

% Hint: For (ii), your new equation should have the form $Q'(t) = -k Q(t) + r$. Think about what happens as $t \to \infty$: does $Q(t)$ go to $0$, to $\infty$, or to some intermediate value?
\end{problem}

% ===== Example 1: Exponential Decay and Half-Life (full solution) =====
\begin{problem}[Exponential Decay and Half-Life]
A quantity $Q(t)$ decays at a rate proportional to its current amount, so that $Q$ satisfies
\[
Q'(t) = k\,Q(t),
\]
for some constant $k<0$.  

(a) Solve this differential equation and express $Q(t)$ in terms of $Q(0)$ and $k$.  

(b) The half-life $T_{1/2}$ is the time required for $Q$ to decrease to one-half of its initial value. Use your solution to show that
\[
T_{1/2} = \frac{\ln 2}{-k}
\quad\text{and equivalently}\quad
k = -\frac{\ln 2}{T_{1/2}}.
\]

(c) Suppose $Q(0)=Q_0$ and you want to know the time $t_\ast$ at which $Q(t_\ast) = Q_{\text{thr}}$, where $0 < Q_{\text{thr}} < Q_0$ is a given threshold. Express $t_\ast$ in terms of $Q_0$, $Q_{\text{thr}}$, and the half-life $T_{1/2}$.
\end{problem}

\begin{solution}
We are given the first-order differential equation
\[
Q'(t) = k\,Q(t), \quad k<0.
\]
This is a linear ordinary differential equation with constant coefficients and, in fact, one of the simplest examples in the class of ``simple ODEs'' considered in this section. It models exponential decay when $k$ is negative.

\medskip

\textbf{(a) Solving the differential equation.}

We use separation of variables. For $Q(t) \neq 0$, we can write
\[
\frac{dQ}{dt} = k\,Q
\quad\Longrightarrow\quad
\frac{1}{Q}\,dQ = k\,dt.
\]
We now integrate both sides with respect to $t$:
\[
\int \frac{1}{Q}\,dQ = \int k\,dt.
\]
The integrals are straightforward:
\[
\ln |Q| = kt + C,
\]
where $C$ is a constant of integration. Exponentiating both sides gives
\[
|Q| = e^{kt+C} = e^{C} e^{kt}.
\]
We can absorb $e^{C}$ into a single constant $C_1$, which may be positive or negative, so we write
\[
Q(t) = C_1 e^{kt}.
\]
To express $C_1$ in terms of the initial amount $Q(0)$, we evaluate at $t=0$:
\[
Q(0) = C_1 e^{k\cdot 0} = C_1,
\]
so $C_1 = Q(0)$. Thus the general solution can be written in the convenient form
\[
Q(t) = Q(0)\,e^{kt}.
\]
If we denote $Q(0)$ by $Q_0$ for brevity, then
\[
Q(t) = Q_0 e^{kt}.
\]
Because $k<0$, the exponential factor $e^{kt}$ decreases to zero as $t$ increases, which matches the physical idea of decay.

\medskip

\textbf{(b) Relating the decay constant to the half-life.}

By definition, the half-life $T_{1/2}$ is the time at which the amount has decreased to one-half of its initial value:
\[
Q(T_{1/2}) = \frac{1}{2} Q_0.
\]
Using our solution $Q(t) = Q_0 e^{kt}$, we substitute $t = T_{1/2}$:
\[
Q(T_{1/2}) = Q_0 e^{k T_{1/2}}.
\]
Equating this with $\frac{1}{2} Q_0$ gives
\[
Q_0 e^{k T_{1/2}} = \frac{1}{2} Q_0.
\]
Provided $Q_0 \neq 0$, we can divide both sides by $Q_0$:
\[
e^{k T_{1/2}} = \frac{1}{2}.
\]
To solve for $T_{1/2}$, we take natural logarithms:
\[
k T_{1/2} = \ln\!\left(\frac{1}{2}\right) = -\ln 2.
\]
Thus
\[
T_{1/2} = \frac{-\ln 2}{k}.
\]
Since $k<0$, we can also write this as
\[
T_{1/2} = \frac{\ln 2}{-k}.
\]
Solving the same relation instead for $k$ gives
\[
k = -\frac{\ln 2}{T_{1/2}}.
\]
These formulas show how the decay constant $k$ and the half-life $T_{1/2}$ encode the same information: knowing either one determines the other.

\medskip

\textbf{(c) Time to reach a given threshold.}

We now suppose that $Q(0) = Q_0$, and we want to find the time $t_\ast$ at which the amount has decayed to a specified threshold $Q_{\text{thr}}$ satisfying $0 < Q_{\text{thr}} < Q_0$. By definition of $t_\ast$,
\[
Q(t_\ast) = Q_{\text{thr}}.
\]
Using the explicit solution $Q(t) = Q_0 e^{kt}$, this becomes
\[
Q_0 e^{k t_\ast} = Q_{\text{thr}}.
\]
Dividing both sides by $Q_0$ yields
\[
e^{k t_\ast} = \frac{Q_{\text{thr}}}{Q_0}.
\]
We again take natural logarithms:
\[
k t_\ast = \ln\!\left(\frac{Q_{\text{thr}}}{Q_0}\right).
\]
Solving for $t_\ast$ gives
\[
t_\ast = \frac{1}{k}\,\ln\!\left(\frac{Q_{\text{thr}}}{Q_0}\right).
\]
Because $k<0$ and $0 < Q_{\text{thr}} < Q_0$, the ratio $Q_{\text{thr}}/Q_0$ lies in $(0,1)$, so its logarithm is negative. Thus $t_\ast$ is positive, as expected.

To express this purely in terms of the half-life $T_{1/2}$, we substitute the formula $k = -\dfrac{\ln 2}{T_{1/2}}$:
\[
t_\ast
= \frac{1}{-\dfrac{\ln 2}{T_{1/2}}} \,\ln\!\left(\frac{Q_{\text{thr}}}{Q_0}\right)
= -\frac{T_{1/2}}{\ln 2}\,\ln\!\left(\frac{Q_{\text{thr}}}{Q_0}\right).
\]
It is sometimes clearer to rewrite this in terms of the ratio $Q_0/Q_{\text{thr}}$:
\[
t_\ast
= \frac{T_{1/2}}{\ln 2}\,\ln\!\left(\frac{Q_0}{Q_{\text{thr}}}\right),
\]
since $\ln(Q_0/Q_{\text{thr}})$ is then positive.

This formula shows how to compute the time needed to reach any specified fraction of the initial quantity using the half-life. In practice, if $Q_0$, $Q_{\text{thr}}$, and $T_{1/2}$ are known from measurements, one simply evaluates the logarithm and multiplies by $T_{1/2}/\ln 2$.

\medskip

\textbf{Connection to the chapter theme.}

This example illustrates the central idea of this section on ``ODEs: Simple Cases'': many basic models in applied mathematics reduce to first-order linear ordinary differential equations with constant coefficients. The structure $Q'(t) = k Q(t)$ leads directly, via separation of variables, to exponential solutions. The parameters in the exponent (here, $k$ or equivalently the half-life $T_{1/2}$) have clear physical interpretations and can be related to experimentally measurable quantities. Thus, even this very simple ODE provides a powerful and widely used model in physics, chemistry, biology, and engineering.
\end{solution}

% ===== Example 2: Falling Body with Linear Air Resistance (inquiry-based) =====
\begin{problem}[Falling Body with Linear Air Resistance]
Imagine dropping a small object (like a raindrop or a bead) from rest, high above the ground. Gravity pulls it downward, but the surrounding air pushes back with a drag force that is proportional to the object's velocity. At first the object speeds up, but eventually the drag becomes strong enough to balance gravity, and the object settles into a constant ``terminal'' speed. In this problem you will build and solve the differential equation that describes this motion and see explicitly how the terminal speed emerges from the mathematics.

Assume motion along a vertical line, with the downward direction taken as positive. Let $y(t)$ denote the vertical position of the object and $v(t) = \dfrac{dy}{dt}$ its velocity. Let $m>0$ be the mass, $g>0$ the constant acceleration due to gravity, and $k>0$ the drag coefficient for the linear air resistance.

\medskip

(a) Using Newton's second law, write down the equation of motion for $v(t)$ by balancing forces. Your model should include the downward gravitational force and an upward drag force proportional to the velocity. Explain the sign of each term. (Assume the object is released from rest at $t=0$ so that $v(0)=0$.)

\medskip

(b) Rewrite your equation from part (a) in the standard first-order linear form
\[
v'(t) + a\,v(t) = b,
\]
for suitable constants $a$ and $b$. Is this equation also separable? Briefly justify your answer, and decide which method (separation of variables or integrating factor) you wish to use to solve it.

\medskip

(c) Solve the differential equation for $v(t)$ with the initial condition $v(0)=0$. Express your answer in a form that makes it easy to identify the long-time behavior of the solution.

\emph{Hint:} If you solve by separation of variables, you will encounter an integral of the form
\[
\int \frac{dv}{\alpha - \beta v} \, .
\]
A simple substitution will help. If you solve by integrating factor, look for an integrating factor of the form $e^{ct}$ for some constant $c$.

\medskip

(d) The \emph{terminal velocity} $v_{\mathrm{term}}$ is defined as the constant velocity the object approaches as $t \to \infty$. 

\begin{itemize}
  \item[(i)] Find $v_{\mathrm{term}}$ directly from the differential equation by asking for a constant solution $v(t) \equiv v_{\mathrm{term}}$.
  \item[(ii)] Verify that your explicit solution $v(t)$ from part (c) satisfies $\displaystyle \lim_{t \to \infty} v(t) = v_{\mathrm{term}}$.
\end{itemize}

Now determine the position $y(t)$ of the object, assuming $y(0)=0$ at the moment of release. Write $y(t)$ in terms of $v_{\mathrm{term}}$ and a characteristic time scale of the system. 

\emph{Hint:} First integrate $v(t)$ to obtain $y(t)$. Then look for a way to rewrite your answer using a quantity like $\tau = m/k$ (or an equivalent expression) to simplify the appearance of the formula.

\medskip

(e) Explorations and extensions.

\begin{itemize}
  \item[(i)] Suppose instead that the object is \emph{thrown upward} from the same point with initial velocity $v(0) = v_0 < 0$ (remember, downward is positive). Without resolving all integrals, sketch qualitatively what you expect the velocity $v(t)$ to look like over time. Will the object still approach the same terminal velocity? Why or why not?
  \item[(ii)] Think about the role of the drag coefficient $k$. If $k$ is very small, what does your formula for $v(t)$ suggest about the motion? If $k$ is very large, how does the behavior change? In each case, relate your conclusions to the physical picture.
\end{itemize}

\end{problem}

% ===== Example 2: Falling Body with Linear Air Resistance (full solution) =====
\begin{problem}[Falling Body with Linear Air Resistance]
An object of mass $m>0$ is dropped from rest at time $t=0$ and falls vertically under gravity in a medium that exerts a drag force proportional to its velocity. Take the downward direction as positive, let $y(t)$ be the position (downward from the release point), and let $v(t) = y'(t)$ be the velocity. Assume gravitational acceleration $g>0$ and a drag force of magnitude $k v(t)$, with $k>0$.

\begin{enumerate}
  \item[(a)] Using Newton's second law, derive the differential equation governing $v(t)$, and write it in the form
  \[
  v'(t) + a\,v(t) = b
  \]
  for appropriate constants $a$ and $b$.
  \item[(b)] Solve this equation for $v(t)$ subject to $v(0) = 0$.
  \item[(c)] Determine the terminal velocity $v_{\mathrm{term}}$ as $t \to \infty$, both by (i) finding constant solutions of the differential equation and (ii) taking the limit of your explicit solution.
  \item[(d)] Find the position $y(t)$ for $t \ge 0$, assuming $y(0)=0$, and express your answer in terms of $v_{\mathrm{term}}$ and the time scale $\tau = m/k$. Briefly explain how this example illustrates typical features of first-order linear ODEs with constant coefficients.
\end{enumerate}
\end{problem}

\begin{solution}
We take the downward vertical direction as positive. The two forces acting on the object are gravity and air resistance. Gravity exerts a constant downward force of magnitude $mg$, so in our sign convention this force is $+mg$. The drag force is proportional to the velocity and acts opposite to the direction of motion. Since we are taking downward as positive, if $v(t) > 0$ the object is moving downward and the drag force is upward, hence negative in our coordinates. Thus the drag force is $-k v(t)$, where $k > 0$ is the drag coefficient.

\medskip

\textbf{(a) Derivation of the differential equation.}
By Newton's second law, the mass times the acceleration equals the net force:
\[
m \, v'(t) = \text{(gravity)} + \text{(drag)} = mg - k v(t).
\]
Dividing by $m$ yields
\[
v'(t) = g - \frac{k}{m} v(t).
\]
Rewriting, we obtain the standard linear form
\[
v'(t) + \frac{k}{m} v(t) = g,
\]
so the constants are $a = \dfrac{k}{m}$ and $b = g$.

This is a first-order linear ordinary differential equation with constant coefficients. It is also separable, since we can write
\[
\frac{dv}{dt} = g - \frac{k}{m} v
\quad \Longrightarrow \quad
\frac{dv}{g - \frac{k}{m} v} = dt,
\]
but we will solve it using the integrating factor method, which generalizes well to more complicated linear equations.

\medskip

\textbf{(b) Solution for the velocity.}
We consider
\[
v'(t) + \frac{k}{m} v(t) = g, \qquad v(0) = 0.
\]
For a linear first-order equation of the form $v' + a v = b$ with constant $a$, an integrating factor is $e^{a t}$. In our case $a = k/m$, so we take
\[
\mu(t) = e^{(k/m)t}.
\]
Multiplying the differential equation by $\mu(t)$, we obtain
\[
e^{(k/m)t} v'(t) + \frac{k}{m} e^{(k/m)t} v(t) = g e^{(k/m)t}.
\]
The left-hand side is the derivative of the product $e^{(k/m)t} v(t)$:
\[
\frac{d}{dt} \left( e^{(k/m)t} v(t) \right)
= g e^{(k/m)t}.
\]
We integrate both sides from $0$ to $t$:
\[
e^{(k/m)t} v(t) - e^{(k/m)\cdot 0} v(0)
= g \int_0^t e^{(k/m)s}\, ds.
\]
Using $v(0) = 0$ and $e^{(k/m) \cdot 0} = 1$, we obtain
\[
e^{(k/m)t} v(t)
= g \int_0^t e^{(k/m)s}\, ds.
\]
The integral on the right is
\[
\int_0^t e^{(k/m)s}\, ds
= \left. \frac{m}{k} e^{(k/m)s} \right|_{s=0}^{s=t}
= \frac{m}{k} \left( e^{(k/m)t} - 1 \right).
\]
Hence,
\[
e^{(k/m)t} v(t)
= g \cdot \frac{m}{k} \bigl( e^{(k/m)t} - 1 \bigr).
\]
Solving for $v(t)$ gives
\[
v(t) = \frac{mg}{k} \left(1 - e^{-(k/m)t}\right).
\]

This explicit formula shows that the velocity increases from $0$ at $t=0$ and approaches a limiting value as $t$ grows.

\medskip

\textbf{(c) Terminal velocity from the differential equation and from the solution.}

\emph{(i) From constant solutions.}
A terminal velocity corresponds to motion with constant velocity, so $v(t) \equiv v_{\mathrm{term}}$ is constant in time. Substituting this into the differential equation
\[
v' + \frac{k}{m} v = g
\]
yields
\[
0 + \frac{k}{m} v_{\mathrm{term}} = g,
\]
so
\[
v_{\mathrm{term}} = \frac{mg}{k}.
\]

\emph{(ii) From the explicit solution.}
From part (b) we have
\[
v(t) = \frac{mg}{k} \left( 1 - e^{-(k/m)t} \right).
\]
As $t \to \infty$, the exponential term $e^{-(k/m)t}$ tends to $0$, so
\[
\lim_{t \to \infty} v(t) = \frac{mg}{k} \cdot (1 - 0) = \frac{mg}{k}.
\]
Thus the explicit time-dependent solution has the expected terminal velocity, which is determined entirely by the balance of the constant gravitational force and the linear drag force.

\medskip

\textbf{(d) Position as a function of time and interpretation.}

We know that $v(t) = y'(t)$ is given by
\[
v(t) = \frac{mg}{k} \left( 1 - e^{-(k/m)t} \right).
\]
We want $y(t)$ for $t \ge 0$ with $y(0) = 0$. Integrating with respect to $t$,
\[
y(t) = \int_0^t v(s)\, ds
= \int_0^t \frac{mg}{k} \left( 1 - e^{-(k/m)s} \right) ds.
\]
We can factor out the constant $\dfrac{mg}{k}$:
\[
y(t) = \frac{mg}{k} \int_0^t \left( 1 - e^{-(k/m)s} \right) ds
= \frac{mg}{k} \left[ \int_0^t 1\, ds - \int_0^t e^{-(k/m)s} ds \right].
\]
The first integral is simply $t$. For the second, we compute
\[
\int_0^t e^{-(k/m)s} ds
= \left. -\frac{m}{k} e^{-(k/m)s} \right|_{s=0}^{s=t}
= -\frac{m}{k} e^{-(k/m)t} + \frac{m}{k}.
\]
Therefore
\[
y(t) = \frac{mg}{k} \left[ t - \left( -\frac{m}{k} e^{-(k/m)t} + \frac{m}{k} \right) \right]
= \frac{mg}{k} \left[ t + \frac{m}{k} e^{-(k/m)t} - \frac{m}{k} \right].
\]
It is convenient to express the result in terms of two natural parameters: the terminal velocity
\[
v_{\mathrm{term}} = \frac{mg}{k}
\]
and the characteristic time scale
\[
\tau = \frac{m}{k}.
\]
Then the expressions simplify to
\[
v(t) = v_{\mathrm{term}} \left( 1 - e^{-t/\tau} \right),
\]
and
\[
y(t) = v_{\mathrm{term}} \left[ t + \tau e^{-t/\tau} - \tau \right]
= v_{\mathrm{term}} \left( t - \tau \left[ 1 - e^{-t/\tau} \right] \right).
\]
We can also rearrange $y(t)$ as
\[
y(t) = v_{\mathrm{term}} t - v_{\mathrm{term}} \tau \left( 1 - e^{-t/\tau} \right).
\]

This formula has a clear physical interpretation. The term $v_{\mathrm{term}} t$ is the distance the object would fall if it were moving at the terminal velocity for the entire time. The correction term
\[
v_{\mathrm{term}} \tau \left( 1 - e^{-t/\tau} \right)
\]
accounts for the fact that, at the beginning, the object is moving more slowly and only gradually approaches $v_{\mathrm{term}}$. The time scale $\tau = m/k$ measures how quickly the transient exponential term $e^{-t/\tau}$ decays: after a time on the order of a few multiples of $\tau$, the velocity is very close to $v_{\mathrm{term}}$.

\medskip

\textbf{Conceptual remarks and relation to ``simple cases'' of ODEs.}

This example illustrates several central ideas of the section on simple ordinary differential equations:

\begin{itemize}
  \item The motion is governed by a \emph{first-order linear ODE with constant coefficients}, which can be solved systematically by the integrating factor method or, in this particular case, by separation of variables.
  \item The long-term behavior is dominated by an \emph{equilibrium solution} (the terminal velocity), and the general solution is an equilibrium plus an exponentially decaying transient. This is typical for linear equations of the form $u' + a u = b$ with $a>0$.
  \item The parameters of the model, $m$, $k$, and $g$, appear naturally in the solution through the equilibrium value $v_{\mathrm{term}} = mg/k$ and the time constant $\tau = m/k$, which describe respectively the steady-state motion and the rate at which this steady state is approached.
\end{itemize}

Thus, the falling body with linear air resistance is a concrete physical setting in which the standard techniques for solving and interpreting first-order linear ODEs can be seen in action.
\end{solution}

% ===== Example 3: Logistic Population Growth (inquiry-based) =====
\begin{problem}[Logistic Population Growth]
In many introductory models of population growth, one assumes that the rate of change of the population is proportional to the current population, leading to exponential growth. However, real populations do not grow without bound: resources such as food, space, or oxygen are limited. One way to capture this effect is to assume that the growth rate decreases as the population approaches some \emph{carrying capacity} imposed by the environment. This leads to the \emph{logistic equation}, a nonlinear first‑order ordinary differential equation.

We will explore how this model is built and how to solve it explicitly.

\smallskip

Consider a population $P(t)$ at time $t$, where $P(t)$ is measured in, say, thousands of individuals.

\smallskip

(a) As a warm‑up, recall the exponential growth model
\[
\frac{dP}{dt} = rP, \qquad r>0.
\]
Solve this differential equation with initial condition $P(0) = P_0>0$. Explain why this model predicts unbounded growth as $t\to\infty$, and briefly discuss why that may be unrealistic for a real population.

\medskip

Now suppose that the environment can support at most a population of size $K>0$. We assume that the growth rate is proportional both to the current population $P(t)$ and to the remaining ``room'' $(K - P(t))$ before the carrying capacity is reached.

\smallskip

(b) Translate this modeling assumption into an ordinary differential equation for $P(t)$ of the form
\[
\frac{dP}{dt} = \text{(constant)}\times P(t)\times (K - P(t)).
\]
Choose a convenient positive constant, usually denoted by $r>0$, and write the resulting equation in the standard \emph{logistic} form
\[
\frac{dP}{dt} = r P(t)\Bigl(1 - \frac{P(t)}{K}\Bigr).
\]
Identify all constant (equilibrium) solutions and interpret them in terms of the population. 

\emph{Hint:} An equilibrium solution is a constant $P(t)\equiv P_\ast$ for which $dP/dt=0$ for all $t$.

\medskip

(c) Show that the logistic equation
\[
\frac{dP}{dt} = r P(t)\Bigl(1 - \frac{P(t)}{K}\Bigr),
\qquad r>0,\ K>0,
\]
is \emph{separable}. Rewrite it in the form
\[
\frac{dP}{P(t)\left(1 - \frac{P(t)}{K}\right)} = r\,dt.
\]
Then, write the left-hand side as a sum of simpler fractions in $P$ that you know how to integrate:
\[
\frac{1}{P\left(1 - \frac{P}{K}\right)} 
= \frac{K}{P(K-P)}
= \frac{A}{P} + \frac{B}{K-P}
\]
for suitable constants $A$ and $B$.

Determine $A$ and $B$, and write the separated equation with the left-hand side expressed as a sum of two integrable terms.

\emph{Hint:} Multiply both sides of
\[
\frac{K}{P(K-P)} = \frac{A}{P} + \frac{B}{K-P}
\]
by $P(K-P)$ and compare coefficients of $P$.

\medskip

(d) Now solve the logistic initial value problem
\[
\frac{dP}{dt} = r P(t)\Bigl(1 - \frac{P(t)}{K}\Bigr), 
\qquad P(0) = P_0,\quad 0<P_0\neq K.
\]
Integrate the separated equation from part (c) to obtain an implicit relation between $P$ and $t$, then solve for $P(t)$ explicitly. Your final answer should have the form
\[
P(t) = \frac{K}{1 + C e^{-rt}}
\]
for some constant $C$ that depends on $P_0$, $K$, and $r$. Determine $C$ in terms of $P_0$ and $K$.

\emph{Hint:} After integrating, you should obtain an equation of the form
\[
\ln\left|\frac{P}{K-P}\right| = rt + C_1.
\]
Exponentiate both sides and solve algebraically for $P$.

\smallskip

Once you have $P(t)$, analyze its long‑time behavior. What is $\displaystyle\lim_{t\to\infty} P(t)$, and how does the sign of $P_0-K$ affect whether $P(t)$ increases or decreases toward that limit?

\medskip

(e) Extensions and ``what if'' questions:

\begin{enumerate}
    \item Suppose a constant harvesting term $h>0$ is added, giving
    \[
    \frac{dP}{dt} = rP\Bigl(1 - \frac{P}{K}\Bigr) - h.
    \]
    Without solving this new equation explicitly, find the equilibrium values by solving a quadratic equation in $P$. For which values of $h$ (in terms of $r$ and $K$) are there two distinct positive equilibria, one positive equilibrium, or no positive equilibria?

    \emph{Hint:} Solve $rP(1-P/K) - h = 0$ and discuss the discriminant.

    \item (Non-dimensionalization.) Let $x(t) = P(t)/K$ be the fraction of the carrying capacity that is occupied. Rewrite the logistic equation in terms of $x(t)$ and a rescaled time variable $\tau = rt$. Show that in these units the equation takes the simpler dimensionless form
    \[
    \frac{dx}{d\tau} = x(1-x).
    \]
    What is the general solution $x(\tau)$ in this form, and how does it relate to your earlier expression for $P(t)$?
\end{enumerate}

\end{problem}

% ===== Example 3: Logistic Population Growth (full solution) =====
\begin{problem}[Logistic Population Growth]
Consider the logistic differential equation
\[
\frac{dP}{dt} = r P(t)\Bigl(1 - \frac{P(t)}{K}\Bigr),
\qquad r>0,\ K>0,
\]
with initial condition $P(0) = P_0>0$.

\begin{enumerate}
    \item[(a)] Find all equilibrium (constant) solutions and classify their stability using a one‑dimensional phase line.
    \item[(b)] Solve the initial value problem explicitly by the method of separation of variables and express $P(t)$ in terms of $r$, $K$, and $P_0$.
    \item[(c)] Determine $\displaystyle\lim_{t\to\infty} P(t)$ and describe how the solution behaves over time when $0<P_0<K$ and when $P_0>K$.
\end{enumerate}
\end{problem}

\begin{solution}
We are given the nonlinear first‑order ordinary differential equation
\[
\frac{dP}{dt} = r P(t)\Bigl(1 - \frac{P(t)}{K}\Bigr),
\]
with parameters $r>0$, $K>0$, and an initial condition $P(0)=P_0>0$. This is the standard logistic equation, a prototypical example of a separable but nonlinear equation in the ``simple cases'' class of ODEs.

\medskip

\noindent\textbf{(a) Equilibria and their stability.}
An equilibrium solution is a constant $P(t)\equiv P_\ast$ such that $dP/dt=0$ for all $t$. Substituting $P(t)=P_\ast$ into the right‑hand side gives
\[
0 = r P_\ast\Bigl(1 - \frac{P_\ast}{K}\Bigr).
\]
Since $r>0$, this product vanishes if and only if
\[
P_\ast = 0 \quad\text{or}\quad 1 - \frac{P_\ast}{K} = 0 \;\Longleftrightarrow\; P_\ast=K.
\]
Thus there are two equilibrium solutions: $P(t)\equiv 0$ and $P(t)\equiv K$.

To classify their stability, we examine the sign of $dP/dt$ as a function of $P$ using a one‑dimensional phase line. Note that the right‑hand side is
\[
f(P) := rP\Bigl(1 - \frac{P}{K}\Bigr).
\]
The sign of $f(P)$ is determined by the sign of $P$ and of $(1-P/K)$.

\begin{itemize}
    \item For $0<P<K$, we have $P>0$ and $1-P/K>0$, so $f(P)>0$. Thus, when the population is between $0$ and $K$, we have $dP/dt>0$ and $P$ increases with time.
    \item For $P>K$, we have $P>0$ but $1-P/K<0$, so $f(P)<0$. Thus, when the population exceeds $K$, we have $dP/dt<0$ and $P$ decreases with time.
    \item For $P<0$, the model is not biologically relevant, but mathematically $P<0$ and $1-P/K>1>0$, so $f(P)<0$ there.
\end{itemize}

On the phase line, the arrows point to the right (increasing $P$) on the interval $(0,K)$ and to the left (decreasing $P$) on $(K,\infty)$. Arrows from both sides point toward $P=K$, so $P=K$ is a \emph{stable} (asymptotically stable) equilibrium. At $P=0$, the arrows on $(0,K)$ point away from $0$; thus $P=0$ is an \emph{unstable} equilibrium.

\medskip

\noindent\textbf{(b) Solving the initial value problem.}
We next solve
\[
\frac{dP}{dt} = r P(t)\Bigl(1 - \frac{P(t)}{K}\Bigr), \qquad P(0)=P_0.
\]
This equation is separable. For $P(t)\neq 0$ and $P(t)\neq K$ we can write
\[
\frac{dP}{dt} = rP\Bigl(1-\frac{P}{K}\Bigr)
\quad\Longrightarrow\quad
\frac{dP}{P\left(1-\frac{P}{K}\right)} = r\,dt.
\]
We now express the left‑hand side as a sum of simpler rational functions of $P$. First,
\[
\frac{1}{P\left(1-\frac{P}{K}\right)}
= \frac{1}{P\cdot\frac{K-P}{K}}
= \frac{K}{P(K-P)}.
\]
We seek constants $A$ and $B$ such that
\[
\frac{K}{P(K-P)} = \frac{A}{P} + \frac{B}{K-P}.
\]
Multiplying both sides by $P(K-P)$ gives
\[
K = A(K-P) + B P = AK + (B-A)P.
\]
Since this identity must hold for all $P$, the constant and linear terms must match:
\[
AK = K \quad\Longrightarrow\quad A = 1,
\]
\[
B - A = 0 \quad\Longrightarrow\quad B = 1.
\]
Hence,
\[
\frac{K}{P(K-P)} = \frac{1}{P} + \frac{1}{K-P}.
\]
The separated equation becomes
\[
\left(\frac{1}{P} + \frac{1}{K-P}\right)\,dP = r\,dt.
\]

We now integrate both sides. Integrating with respect to $t$ is equivalent to integrating with respect to $P$ on the left:
\[
\int\left(\frac{1}{P} + \frac{1}{K-P}\right)\,dP = \int r\,dt.
\]
Compute each integral:
\[
\int \frac{1}{P}\,dP = \ln|P| + C_1,
\]
\[
\int \frac{1}{K-P}\,dP = -\ln|K-P| + C_2,
\]
because $\frac{d}{dP}(K-P) = -1$. Ignoring the intermediate constants and combining, we obtain
\[
\ln|P| - \ln|K-P| = rt + C,
\]
for some constant $C\in\mathbb{R}$. Using properties of logarithms, we write
\[
\ln\left|\frac{P}{K-P}\right| = rt + C.
\]
Exponentiating both sides yields
\[
\left|\frac{P}{K-P}\right| = e^{rt + C} = C_1 e^{rt},
\]
where $C_1 = \pm e^C$ is a nonzero constant. For $0<P<K$ (the biologically relevant regime), the ratio $P/(K-P)$ is positive, so we can drop the absolute value and simply write
\[
\frac{P}{K-P} = C_1 e^{rt}.
\]

Now we solve this algebraic equation for $P$ in terms of $t$. Multiply both sides by $(K-P)$:
\[
P = C_1 e^{rt}(K-P).
\]
Distribute on the right:
\[
P = C_1 K e^{rt} - C_1 e^{rt} P.
\]
Collect the terms involving $P$ on one side:
\[
P + C_1 e^{rt} P = C_1 K e^{rt},
\]
\[
P\bigl(1 + C_1 e^{rt}\bigr) = C_1 K e^{rt}.
\]
Thus
\[
P(t) = \frac{C_1 K e^{rt}}{1 + C_1 e^{rt}}.
\]
It is more convenient to rewrite the constant. Multiply numerator and denominator by $e^{-rt}$:
\[
P(t) = \frac{C_1 K}{e^{-rt} + C_1}
= \frac{K}{e^{-rt}/C_1 + 1}
= \frac{K}{1 + C e^{-rt}},
\]
where $C = 1/C_1$ is a new nonzero constant.

We determine $C$ from the initial condition $P(0)=P_0$. Substituting $t=0$ gives
\[
P_0 = P(0) = \frac{K}{1 + C e^{0}} = \frac{K}{1 + C}.
\]
Solving for $C$,
\[
1 + C = \frac{K}{P_0}
\quad\Longrightarrow\quad
C = \frac{K}{P_0} - 1 = \frac{K - P_0}{P_0}.
\]
Therefore, the explicit solution is
\[
P(t) = \frac{K}{1 + \displaystyle\left(\frac{K-P_0}{P_0}\right)e^{-rt}},
\qquad P_0>0,\ P_0\neq K.
\]
This formula is valid for all $t$ for which the denominator is nonzero; for positive parameters and $P_0>0$ it stays positive, so the solution exists for all $t\in\mathbb{R}$.

Note that the equilibrium solutions $P(t)\equiv 0$ and $P(t)\equiv K$ correspond to the special initial conditions $P_0=0$ and $P_0=K$, respectively. In these cases, the separation step above would have involved dividing by zero, which is why we treated equilibrium solutions separately.

\medskip

\noindent\textbf{(c) Long‑time behavior.}
We now compute the limit of $P(t)$ as $t\to\infty$ and describe the qualitative behavior. The explicit solution is
\[
P(t) = \frac{K}{1 + \left(\dfrac{K-P_0}{P_0}\right)e^{-rt}}.
\]
Since $r>0$, we have $e^{-rt}\to 0$ as $t\to\infty$. Therefore,
\[
\lim_{t\to\infty} P(t) 
= \frac{K}{1 + \left(\dfrac{K-P_0}{P_0}\right)\cdot 0}
= \frac{K}{1+0} = K.
\]
Thus, for every initial population $P_0>0$, $P_0\neq 0$, the solution tends to the carrying capacity $K$ as $t\to\infty$. This agrees with the phase line analysis, which indicated that $P=K$ is a stable equilibrium.

To understand whether $P(t)$ increases or decreases toward $K$, examine the denominator at $t=0$:
\[
P(0) = \frac{K}{1 + \left(\dfrac{K-P_0}{P_0}\right)} = P_0.
\]
\begin{itemize}
    \item If $0<P_0<K$, then $K-P_0>0$, so $C = (K-P_0)/P_0>0$. For $t>0$, the factor $e^{-rt}$ is positive and decreasing to $0$, so the denominator $1 + C e^{-rt}$ decreases from $1+C>1$ toward $1$. A decreasing denominator leads to an increasing $P(t)$. Thus $P(t)$ increases monotonically from $P_0$ up to $K$ as $t\to\infty$.
    \item If $P_0>K$, then $K-P_0<0$, so $C<0$. Thus $1 + C e^{-rt}$ increases from $1+C<1$ up to $1$ as $t\to\infty$. An increasing denominator leads to a decreasing $P(t)$. Hence $P(t)$ decreases monotonically from $P_0$ down to $K$ as $t\to\infty$.
\end{itemize}

In either case, the carrying capacity $K$ acts as an attracting equilibrium, and the population approaches $K$ from below if it starts below, and from above if it starts above.

\medskip

\noindent\textbf{Conceptual remarks.}
This example illustrates several central ideas from the section on ``simple cases'' of ordinary differential equations:

\begin{itemize}
    \item The logistic equation is a \emph{separable} first‑order ODE, so the main technique is to rearrange it into the form $g(P)\,dP = h(t)\,dt$ and integrate both sides.
    \item Even though the equation is nonlinear, partial fraction decomposition reduces the integral to standard logarithmic functions.
    \item Qualitative methods, such as phase line analysis and equilibrium classification, give global information about stability and long‑time behavior that is consistent with and complements the explicit solution.
\end{itemize}

Thus the logistic model serves as a prototypical nonlinear but explicitly solvable equation that bridges basic solution techniques with qualitative analysis of ODEs.
\end{solution}

% ===== Example 4: Mixing in a Well-Stirred Tank (inquiry-based) =====
\begin{problem}[Mixing in a Well-Stirred Tank]
A standard model in chemical and environmental engineering describes how a dissolved substance mixes in a tank of liquid. A solution with given concentration flows into the tank, the mixture (assumed perfectly stirred) flows out at the same rate, and the total volume in the tank remains constant. This situation leads to a first-order linear differential equation for the amount of solute in the tank as a function of time. In this problem you will build the model step by step, solve it, and interpret the answer.

A tank initially contains $100$ liters of liquid in which $10$ kilograms of a chemical are dissolved. A solution containing the same chemical flows into the tank at a rate of $5$ liters per minute, with concentration $0.4$ kilograms per liter. The mixture is kept perfectly stirred and flows out at the same rate of $5$ liters per minute.

\smallskip

(a) Let $Q(t)$ denote the amount of chemical (in kilograms) in the tank at time $t$ (in minutes). Explain, in words, how the quantity $Q(t)$ changes over a short time interval. Then write a verbal ``balance law'' for $Q(t)$ of the form:
\[
\text{rate of change of amount in tank} 
= \text{rate in} - \text{rate out}.
\]
Be explicit about what each of these three terms means for this mixing problem.

\medskip

(b) Now translate your verbal balance law into a differential equation for $Q(t)$.

\begin{itemize}
  \item What is the numerical value of the \emph{rate in} (in kilograms per minute)? Express it in terms of the flow rate and the incoming concentration.
  \item For the \emph{rate out}, first express the concentration in the tank at time $t$ in terms of $Q(t)$ and the constant volume.
  \item Use this to write $\text{rate out}$ as a function of $Q(t)$.
\end{itemize}

Combine these pieces to obtain an explicit first-order linear ODE for $Q(t)$. What is the initial condition for this equation?

\medskip

(c) Solve the differential equation for $Q(t)$.

Hint: First rewrite your ODE in the standard linear form
\[
\frac{dQ}{dt} + a Q = b,
\]
for appropriate constants $a$ and $b$. You may solve it either by using an integrating factor or by recognizing it as a separable equation. Be sure to use the initial condition to determine any integration constants.

\medskip

(d) Use your explicit formula for $Q(t)$ to answer the following questions.

\begin{itemize}
  \item[(i)] What is the limiting amount of chemical in the tank as $t \to \infty$? What is the corresponding limiting concentration (in kilograms per liter)?
  
  \item[(ii)] After how many minutes will the concentration in the tank reach $0.35$ kilograms per liter? Set up an equation for $t$ using your formula for $Q(t)$, and then solve for $t$.
\end{itemize}

Hint: For part (ii), remember that concentration equals $Q(t)$ divided by the tank volume, and use logarithms to solve for time.

\medskip

(e) Explore some variations of the model.

\begin{itemize}
  \item[(i)] Suppose now that pure water flows into the tank (incoming concentration $0$ kilograms per liter), still at $5$ liters per minute, and the tank initially contains $40$ kilograms of chemical in $100$ liters. Write down and solve the new differential equation for $Q(t)$, and describe in words what happens to the concentration as $t \to \infty$.

  \item[(ii)] Suppose instead that the inflow remains as in the original problem, but the outflow rate is only $4$ liters per minute, so that the volume in the tank increases over time. Describe how the balance law in part (a) must be modified. In particular, which quantities that were previously constant now depend on time? (You do not have to solve the new equation; focus on formulating it correctly.)
\end{itemize}

\end{problem}

% ===== Example 4: Mixing in a Well-Stirred Tank (full solution) =====
\begin{problem}[Mixing in a Well-Stirred Tank]
A tank initially contains $100$ liters of liquid with $10$ kilograms of a dissolved chemical. A solution flows in at $5$ liters per minute with concentration $0.4$ kilograms per liter. The mixture is perfectly stirred and flows out at the same rate of $5$ liters per minute, so the tank volume remains $100$ liters.

Let $Q(t)$ be the amount of chemical (in kilograms) in the tank at time $t$ (in minutes).

\begin{enumerate}
  \item Derive a differential equation for $Q(t)$ together with the initial condition.
  \item Solve this differential equation for $Q(t)$.
  \item Find the limiting concentration in the tank as $t \to \infty$.
  \item Determine how long it takes until the concentration in the tank reaches $0.35$ kilograms per liter.
\end{enumerate}
\end{problem}

\begin{solution}
We model the system by tracking the amount of chemical in the tank as a function of time. The key modeling principle is a balance (or conservation) law: the rate of change of the amount of chemical in the tank equals the rate at which chemical enters minus the rate at which chemical leaves.

\medskip

\noindent\textbf{1. Deriving the differential equation.}

Let $Q(t)$ denote the amount of chemical in the tank at time $t$, measured in kilograms. We work with the generic balance law
\[
\frac{dQ}{dt} = \text{(rate in)} - \text{(rate out)}.
\]

\emph{Rate in.} The inflow rate of liquid is $5$ liters per minute, and the incoming concentration is $0.4$ kilograms per liter. Thus the incoming mass rate is
\[
\text{rate in} = 5 \,\frac{\text{L}}{\text{min}} \times 0.4 \,\frac{\text{kg}}{\text{L}}
= 2 \,\frac{\text{kg}}{\text{min}}.
\]

\emph{Rate out.} The outflow rate of liquid is also $5$ liters per minute. The concentration of chemical in the tank at time $t$ is the amount divided by the volume. The volume of liquid remains constant at $100$ liters, so the concentration in the tank is
\[
\text{concentration in tank at time } t
= \frac{Q(t)}{100} \,\frac{\text{kg}}{\text{L}}.
\]
The outflow carries liquid at $5$ liters per minute with this concentration, so
\[
\text{rate out} = 5 \,\frac{\text{L}}{\text{min}} \times \frac{Q(t)}{100} \,\frac{\text{kg}}{\text{L}}
= \frac{5}{100} Q(t) \,\frac{\text{kg}}{\text{min}}
= \frac{1}{20} Q(t)\,\frac{\text{kg}}{\text{min}}.
\]

Substituting these expressions into the balance law gives
\[
\frac{dQ}{dt} = 2 - \frac{1}{20} Q(t).
\]
Rewriting,
\[
\frac{dQ}{dt} + \frac{1}{20} Q = 2.
\]

The initial condition comes from the initial amount of chemical. Initially there are $10$ kilograms dissolved in the $100$ liters, so
\[
Q(0) = 10.
\]

Thus the mathematical model is the initial value problem
\[
\frac{dQ}{dt} + \frac{1}{20} Q = 2, \qquad Q(0) = 10.
\]

This is a first-order linear ordinary differential equation, one of the “simple cases” studied in elementary ODE theory.

\medskip

\noindent\textbf{2. Solving the differential equation.}

We solve
\[
\frac{dQ}{dt} + \frac{1}{20} Q = 2
\]
using the integrating factor method. The standard linear form is
\[
\frac{dQ}{dt} + a Q = b
\]
with $a = \tfrac{1}{20}$ and $b = 2$.

An integrating factor is
\[
\mu(t) = e^{\int a\,dt} = e^{\int \frac{1}{20}\,dt} = e^{t/20}.
\]
Multiplying the equation by $\mu(t)$ gives
\[
e^{t/20} \frac{dQ}{dt} + \frac{1}{20} e^{t/20} Q = 2 e^{t/20}.
\]
The left-hand side is the derivative of the product $e^{t/20} Q(t)$:
\[
\frac{d}{dt}\bigl(e^{t/20} Q(t)\bigr) = 2 e^{t/20}.
\]

We integrate both sides with respect to $t$:
\[
\int \frac{d}{dt}\bigl(e^{t/20} Q(t)\bigr)\,dt
= \int 2 e^{t/20}\,dt.
\]
The left integral simply returns $e^{t/20} Q(t)$. For the right integral, we have
\[
\int 2 e^{t/20}\,dt = 2 \cdot 20 e^{t/20} + C = 40 e^{t/20} + C,
\]
where $C$ is an integration constant. Thus
\[
e^{t/20} Q(t) = 40 e^{t/20} + C.
\]

Dividing both sides by $e^{t/20}$ yields
\[
Q(t) = 40 + C e^{-t/20}.
\]

We now use the initial condition $Q(0) = 10$:
\[
Q(0) = 40 + C e^{0} = 40 + C = 10.
\]
Therefore $C = 10 - 40 = -30$, and the solution is
\[
Q(t) = 40 - 30 e^{-t/20}.
\]

This is the explicit formula for the amount of chemical in the tank at time $t$.

\medskip

\noindent\textbf{3. Limiting concentration as $t \to \infty$.}

The concentration in the tank at time $t$ is
\[
\text{concentration}(t) = \frac{Q(t)}{100}
= \frac{40 - 30 e^{-t/20}}{100}
= 0.4 - 0.3 e^{-t/20} \quad \text{kilograms per liter}.
\]

As $t \to \infty$, the exponential term $e^{-t/20}$ tends to $0$, so
\[
\lim_{t \to \infty} Q(t) = 40
\quad\text{and}\quad
\lim_{t \to \infty} \text{concentration}(t) = \frac{40}{100} = 0.4 \;\text{kg/L}.
\]

Thus the tank approaches a steady state in which the concentration of the chemical in the tank equals the concentration in the incoming solution. This is typical for such mixing problems: with constant volume and perfect stirring, the tank concentration exponentially approaches the inflow concentration.

\medskip

\noindent\textbf{4. Time to reach a given concentration.}

We are asked for the time at which the concentration reaches $0.35$ kilograms per liter. This corresponds to
\[
\frac{Q(t)}{100} = 0.35
\quad\Longleftrightarrow\quad
Q(t) = 35.
\]
We set our solution equal to $35$ and solve for $t$:
\[
40 - 30 e^{-t/20} = 35.
\]
Rearranging,
\[
-30 e^{-t/20} = 35 - 40 = -5,
\]
so
\[
30 e^{-t/20} = 5,
\qquad
e^{-t/20} = \frac{5}{30} = \frac{1}{6}.
\]

Taking natural logarithms,
\[
-\frac{t}{20} = \ln\!\left(\frac{1}{6}\right) = -\ln 6,
\]
which implies
\[
\frac{t}{20} = \ln 6,
\quad\text{so}\quad
t = 20 \ln 6 \;\text{minutes}.
\]

Numerically, $\ln 6 \approx 1.7918$, so
\[
t \approx 20 \times 1.7918 \approx 35.8 \;\text{minutes}.
\]

Therefore, it takes approximately $36$ minutes for the concentration in the tank to reach $0.35$ kilograms per liter.

\medskip

\noindent\textbf{Conceptual remarks.}

This example illustrates the core ideas from the section on ``ODEs: Simple Cases.'' The modeling step turns a physical conservation principle into a first-order linear differential equation with constant coefficients. The solution exhibits exponential behavior: the amount of chemical approaches a steady-state value determined by the balance of inflow and outflow, and the approach to this equilibrium is governed by an exponential decay factor $e^{-t/20}$. Problems of this type show how simple linear ODEs naturally arise in real-world mixing processes and how their solutions describe both transient behavior and long-term steady states. 

\end{solution}

% ===== Example 5: Simple Harmonic Oscillator (inquiry-based) =====
\begin{problem}[Simple Harmonic Oscillator]
A mass attached to a spring on a frictionless horizontal surface provides one of the simplest and most important models in mechanics. When the mass is displaced from its equilibrium position and released, it moves back and forth in a motion that we call \emph{simple harmonic}. In this problem you will derive the governing differential equation from physical principles, solve it using techniques for linear ODEs with constant coefficients, and interpret the resulting sinusoidal motion. You will also briefly explore how changes in the system alter the behavior of the solution.

Assume a mass $m>0$ is attached to an ideal spring with spring constant $k>0$ on a frictionless horizontal surface. Let $x(t)$ denote the displacement of the mass from its equilibrium position at time $t$, with $x>0$ meaning that the spring is stretched.

\medskip

(a) Using Newton's second law and Hooke's law, derive the differential equation satisfied by $x(t)$. Carefully choose a sign convention and write down the equation of motion.  
Hint: Newton's second law says ``mass $\times$ acceleration $=$ sum of forces.'' Hooke's law says the restoring force of the spring is proportional to the displacement and points toward the equilibrium.

\medskip

(b) Rewrite your equation from part (a) in the standard form
\[
x''(t) + \omega^2 x(t) = 0,
\]
and identify the constant $\omega$ in terms of $m$ and $k$. Physically, $\omega$ is called the \emph{angular frequency}.  
Now, to solve this equation, suppose that a solution has the exponential form $x(t) = e^{rt}$ for some constant $r$. Substitute this guess into the differential equation and find the algebraic equation (the \emph{characteristic equation}) that $r$ must satisfy.  
% Hint: Compute $x''(t)$ when $x(t) = e^{rt}$ and simplify.

\medskip

(c) Solve the characteristic equation you found in part (b). Show that the roots are purely imaginary, and denote them by $r = \pm i\omega$.  
Next, explain why the general \emph{real-valued} solution of the differential equation can be written in the form
\[
x(t) = A\cos(\omega t) + B\sin(\omega t),
\]
for some real constants $A$ and $B$.  
Hint: Use Euler's formula $e^{i\theta} = \cos\theta + i\sin\theta$, and remember that real and imaginary parts of complex exponentials solve the same real linear ODE.

\medskip

(d) Suppose that at time $t=0$ the mass is displaced to $x(0) = x_0$ and given initial velocity $x'(0) = v_0$.  
\begin{enumerate}
\item[(i)] Use these initial conditions to determine $A$ and $B$ in terms of $x_0$, $v_0$, and $\omega$.  
% Hint: Evaluate $x(t)$ and $x'(t)$ at $t=0$ and solve the resulting $2\times 2$ linear system.
\item[(ii)] Show that the same solution can also be written in the \emph{amplitude--phase} form
\[
x(t) = R \cos(\omega t - \phi),
\]
for suitable constants $R>0$ and $\phi\in\mathbb{R}$. Express $R$ and $\phi$ in terms of $x_0$ and $v_0$.  
Hint: Use the identity $\cos(\omega t - \phi) = \cos\phi\,\cos(\omega t) + \sin\phi\,\sin(\omega t)$ and compare coefficients with the $A\cos(\omega t)+B\sin(\omega t)$ form.
\end{enumerate}

\medskip

(e) Extensions and ``what if'' questions.
\begin{enumerate}
\item[(i)] Define the (mechanical) energy of the mass--spring system by
\[
E(t) = \frac{1}{2}m\bigl(x'(t)\bigr)^2 + \frac{1}{2}k\bigl(x(t)\bigr)^2.
\]
Show, by direct differentiation and use of the differential equation, that $E'(t) = 0$ for all $t$. What does this say about the motion physically?  
Hint: Differentiate $E(t)$ and substitute $x''(t)$ from the equation of motion.
\item[(ii)] Now imagine adding a small linear damping force $-b x'(t)$ with $b>0$, so that the equation becomes
\[
m x''(t) + b x'(t) + k x(t) = 0.
\]
Without solving this new equation in detail, discuss qualitatively how you expect the motion of the mass to change compared with the undamped case. In particular, what happens to the amplitude and the energy as $t\to\infty$?
\end{enumerate}

\end{problem}

% ===== Example 5: Simple Harmonic Oscillator (full solution) =====
\begin{problem}[Simple Harmonic Oscillator]
A mass $m>0$ is attached to an ideal spring with spring constant $k>0$ on a frictionless horizontal surface. Let $x(t)$ denote the displacement from equilibrium. 

\begin{enumerate}
\item[(i)] Use Newton's second law and Hooke's law to derive the equation of motion for $x(t)$, and rewrite it in the form
\[
x''(t) + \omega^2 x(t) = 0
\]
for a suitable constant $\omega$ expressed in terms of $m$ and $k$.
\item[(ii)] Solve this differential equation and show that all real solutions can be written as
\[
x(t) = A\cos(\omega t) + B\sin(\omega t)
\]
for constants $A,B\in\mathbb{R}$.
\item[(iii)] Given initial conditions $x(0)=x_0$ and $x'(0)=v_0$, determine $A$ and $B$, and rewrite the solution in the amplitude--phase form
\[
x(t) = R\cos(\omega t - \phi)
\]
by expressing $R>0$ and $\phi\in\mathbb{R}$ in terms of $x_0$ and $v_0$.
\item[(iv)] Define
\[
E(t) = \frac{1}{2}m\bigl(x'(t)\bigr)^2 + \frac{1}{2}k\bigl(x(t)\bigr)^2.
\]
Show that $E'(t) = 0$ along any solution, and interpret this physically.
\end{enumerate}
\end{problem}

\begin{solution}
We proceed step by step, beginning from the physical modeling and ending with a qualitative interpretation of the motion.

\medskip

\emph{(i) Derivation of the equation of motion and identification of $\omega$.}

We choose a coordinate axis along the direction of motion, with $x(t)$ denoting displacement from the equilibrium position, and we take $x>0$ when the spring is stretched. Newton's second law states that
\[
m x''(t) = \text{(sum of forces acting on the mass)}.
\]
For an ideal spring obeying Hooke's law, the restoring force is proportional to the displacement and points toward equilibrium. Thus the spring force is
\[
F_{\text{spring}} = -k x(t),
\]
where the minus sign indicates that the force is opposite in direction to $x(t)$.

Assuming there are no other horizontal forces (no friction or external forcing), Newton's law becomes
\[
m x''(t) = -k x(t).
\]
Rewriting, we obtain the second-order linear ordinary differential equation
\[
m x''(t) + k x(t) = 0.
\]

It is convenient to divide by $m>0$ and introduce the constant
\[
\omega^2 = \frac{k}{m}.
\]
Then the equation of motion takes the standard form
\[
x''(t) + \omega^2 x(t) = 0.
\]
The constant $\omega>0$ is called the \emph{angular frequency} of the oscillator.

\medskip

\emph{(ii) Solving the ODE via the characteristic equation.}

The equation
\[
x''(t) + \omega^2 x(t) = 0
\]
is a linear, homogeneous, constant-coefficient ODE of order two. A standard method for such equations is to look for solutions of exponential form
\[
x(t) = e^{rt},
\]
where $r$ is a constant to be determined. Substituting this trial solution, we compute
\[
x'(t) = r e^{rt}, \qquad x''(t) = r^2 e^{rt}.
\]
Plugging into the differential equation gives
\[
r^2 e^{rt} + \omega^2 e^{rt} = 0.
\]
Since $e^{rt}\neq 0$ for all $t$, we can divide by $e^{rt}$ and obtain the \emph{characteristic equation}
\[
r^2 + \omega^2 = 0.
\]
Solving for $r$ yields
\[
r^2 = -\omega^2 \quad \Rightarrow \quad r = \pm i\omega.
\]
Thus the characteristic roots are purely imaginary.

For a second-order linear homogeneous equation with distinct complex conjugate roots $r_1 = i\omega$ and $r_2 = -i\omega$, the general complex-valued solution is
\[
x(t) = C_1 e^{i\omega t} + C_2 e^{-i\omega t},
\]
where $C_1$ and $C_2$ may be complex constants. However, our physical displacement $x(t)$ must be real-valued. We can therefore use Euler's formula
\[
e^{i\theta} = \cos\theta + i\sin\theta,
\]
to express $e^{\pm i\omega t}$ in terms of sine and cosine. One way to proceed is to note that the real and imaginary parts of $e^{i\omega t}$ each satisfy the differential equation, because it has real coefficients and is linear. Consequently, both
\[
\cos(\omega t) \quad \text{and} \quad \sin(\omega t)
\]
are real-valued solutions. By linearity, any linear combination
\[
x(t) = A\cos(\omega t) + B\sin(\omega t),
\]
where $A$ and $B$ are real constants, is again a solution.

Moreover, since we have found two linearly independent real solutions $\cos(\omega t)$ and $\sin(\omega t)$, and the equation is of order two, the general real solution is precisely
\[
x(t) = A\cos(\omega t) + B\sin(\omega t),
\]
for some $A,B\in\mathbb{R}$.

\medskip

\emph{(iii) Imposing initial conditions and amplitude--phase form.}

We now incorporate the initial conditions
\[
x(0) = x_0, \qquad x'(0) = v_0.
\]
Starting from the general solution
\[
x(t) = A\cos(\omega t) + B\sin(\omega t),
\]
we first compute its derivative:
\[
x'(t) = -A\omega \sin(\omega t) + B\omega \cos(\omega t).
\]

Evaluating at $t=0$ gives
\[
x(0) = A\cos(0) + B\sin(0) = A,
\]
so
\[
A = x_0.
\]
Similarly,
\[
x'(0) = -A\omega\sin(0) + B\omega\cos(0) = B\omega,
\]
so
\[
B = \frac{v_0}{\omega}.
\]

Thus the unique solution with the given initial data is
\[
x(t) = x_0 \cos(\omega t) + \frac{v_0}{\omega}\sin(\omega t).
\]

Next we rewrite this in amplitude--phase form. We seek $R>0$ and $\phi\in\mathbb{R}$ such that
\[
x(t) = R\cos(\omega t - \phi).
\]
Using the cosine subtraction identity,
\[
\cos(\omega t - \phi) = \cos\phi \,\cos(\omega t) + \sin\phi \,\sin(\omega t),
\]
we obtain
\[
R\cos(\omega t - \phi)
= R\cos\phi\,\cos(\omega t) + R\sin\phi\,\sin(\omega t).
\]
Comparing this with
\[
x(t) = x_0 \cos(\omega t) + \frac{v_0}{\omega}\sin(\omega t),
\]
we see that we must have
\[
R\cos\phi = x_0, \qquad R\sin\phi = \frac{v_0}{\omega}.
\]

We can solve for $R$ by squaring and adding:
\[
R^2 = (R\cos\phi)^2 + (R\sin\phi)^2
= x_0^2 + \left(\frac{v_0}{\omega}\right)^2,
\]
so
\[
R = \sqrt{x_0^2 + \left(\frac{v_0}{\omega}\right)^2}.
\]
Since $R>0$, this determines $R$ uniquely.

To find $\phi$, we can use
\[
\cos\phi = \frac{x_0}{R}, \qquad \sin\phi = \frac{v_0}{\omega R},
\]
so that, for example,
\[
\phi = \arctan\!\left(\frac{v_0}{\omega x_0}\right),
\]
with the understanding that the correct quadrant for $\phi$ is chosen using the signs of $x_0$ and $v_0$ (equivalently, using the two-argument arctangent function). In any case, $(R,\phi)$ determined in this way yields the same motion as the $(A,B)$ description.

The amplitude--phase form
\[
x(t) = R\cos(\omega t - \phi)
\]
makes it clear that the mass executes sinusoidal motion of fixed amplitude $R$, oscillating with angular frequency $\omega$ and period $T = 2\pi/\omega$. The phase $\phi$ encodes where in its cycle the oscillator is at time $t=0$.

\medskip

\emph{(iv) Conservation of energy.}

We now define the mechanical energy of the mass--spring system by
\[
E(t) = \frac{1}{2}m\bigl(x'(t)\bigr)^2 + \frac{1}{2}k\bigl(x(t)\bigr)^2.
\]
The first term represents kinetic energy, and the second represents potential energy stored in the spring.

We differentiate $E(t)$ with respect to $t$:
\[
E'(t)
= m x'(t)x''(t) + k x(t)x'(t),
\]
where we used the chain rule on each term.

Now we use the equation of motion $m x''(t) + k x(t) = 0$, which can be rewritten as
\[
m x''(t) = -k x(t).
\]
Substituting this into the expression for $E'(t)$, we obtain
\[
E'(t) = x'(t)\, \bigl(m x''(t) + k x(t)\bigr) = x'(t)\cdot 0 = 0.
\]
Therefore $E(t)$ is constant in time along any solution:
\[
E(t) = E(0) \quad \text{for all } t.
\]

Physically, this means that in the idealized undamped mass--spring system, the total mechanical energy is conserved. Energy is continuously exchanged between kinetic and potential forms: when the mass passes through equilibrium, all energy is kinetic; when it reaches an extreme displacement, all energy is potential. But the sum remains constant, and as a result the oscillations persist forever with constant amplitude.

\medskip

\emph{Connection to ``ODEs: Simple Cases.''}

This example illustrates how a fundamental physical principle (Newton's law plus Hooke's law) naturally leads to a second-order linear ODE with constant coefficients. The solution method—postulating exponential solutions, deriving the characteristic equation, and interpreting complex conjugate roots in terms of sine and cosine—embodies the central technique for solving such equations. The simple harmonic oscillator also shows how qualitative features of the solution (oscillation frequency, amplitude, and energy conservation) are encoded in the parameters and structure of the ODE, providing a model example of how mathematical analysis illuminates physical behavior.
\end{solution}

\section{Direct Methods for Solving Linear ODEs}
% --- Narrative plan (auto-generated) ---
% This section develops the core "direct" techniques for solving linear ordinary differential equations: methods that convert an equation and its data into an explicit formula for the solution. We focus on constant-coefficient equations and low-dimensional systems, where algebraic tools such as characteristic polynomials, eigenvalues, and particular-solution ansätze are especially effective. Along the way, we connect the algebraic structure of a differential equation to the qualitative behavior of its solutions, such as exponential growth and decay, oscillations, and resonance.
%
% These ideas are central throughout applied mathematics. Many basic models in mechanics, electrical circuits, population dynamics, and simple control systems reduce to linear ODEs that can be solved directly. In turn, the same solution forms reappear when separating variables in linear partial differential equations, for instance in the heat or wave equation, and when analyzing linearized stability of nonlinear dynamical systems. The characteristic-exponential viewpoint also links naturally to topics such as complex analysis, Fourier methods, and spectral theory, where exponentials and eigenfunctions play a unifying role.

% ===== Example 1: Linear First-Order Equation for Exponential Growth and Decay (inquiry-based) =====
\begin{problem}[Linear First-Order Equation for Exponential Growth and Decay]
Many physical and biological processes have the feature that the rate of change of a quantity is proportional to the quantity itself. A standard example is radioactive decay: each atom has the same chance of decaying per unit time, so the overall decay rate is proportional to how many atoms remain. A contrasting example is population growth in an idealized, unlimited environment: the more individuals there are, the more births occur per unit time. In both cases, the same kind of differential equation appears, and its solutions are exponential functions.

In this problem you will discover how this model leads to a first-order linear ordinary differential equation, how to solve it explicitly, and how the familiar notions of half-life and doubling time arise from the solution.

\smallskip

(a) Let $Q(t)$ denote the amount of a substance (or the size of a population) at time $t$. Suppose we are told that the rate of change of $Q$ is \emph{proportional} to the current amount $Q(t)$, with proportionality constant $k$.

\begin{enumerate}
\item[(i)] Write a differential equation for $Q(t)$ that encodes this statement. Be explicit about the sign of $k$ in the cases of growth and decay.
\item[(ii)] What are the physical units of the constant $k$ if $Q$ is measured in grams and time is measured in years? Explain briefly.
\end{enumerate}

Hint: ``Rate of change'' means a derivative with respect to time, and ``proportional to'' means a constant multiple of.

\smallskip

(b) Now treat $k$ as a given real constant, and assume $Q(t)$ satisfies the differential equation you wrote in part (a). 

\begin{enumerate}
\item[(i)] Rewrite the equation in a form where all occurrences of $Q$ are on one side and all occurrences of $t$ are on the other side.
\item[(ii)] Integrate both sides with respect to $t$ to obtain an equation involving $\ln |Q(t)|$ and $t$.

Hint: You should obtain an integral of the form $\displaystyle \int \frac{1}{Q}\, dQ$ on one side. Recall that $\displaystyle \int \frac{1}{Q}\, dQ = \ln|Q| + C$.
\end{enumerate}

\smallskip

(c) From your integrated equation in part (b), solve explicitly for $Q(t)$.

\begin{enumerate}
\item[(i)] Show that the general solution can be written in the form
\[
Q(t) = Ce^{kt}
\]
for some constant $C$. Explain how the constant of integration from part (b) is related to $C$.
\item[(ii)] Suppose that $Q(0) = Q_0$ is known. Determine $C$ in terms of $Q_0$ and write the solution $Q(t)$ entirely in terms of $Q_0$, $k$, and $t$.
\end{enumerate}

Hint: Evaluate the general solution at $t=0$ and use the fact that $e^{0} = 1$.

\smallskip

(d) Let $Q(t)$ be the solution with initial condition $Q(0) = Q_0 > 0$.

\begin{enumerate}
\item[(i)] Assume $k < 0$ (radioactive decay). Define the \emph{half-life} $T_{1/2}$ to be the time at which $Q(T_{1/2}) = \tfrac12 Q_0$. Use your formula for $Q(t)$ to solve for $T_{1/2}$ in terms of $k$.
\item[(ii)] Assume $k > 0$ (unrestricted population growth). Define the \emph{doubling time} $T_{2}$ to be the time at which $Q(T_{2}) = 2 Q_0$. Use your formula for $Q(t)$ to solve for $T_2$ in terms of $k$.
\item[(iii)] In each case, does the characteristic time ($T_{1/2}$ or $T_2$) depend on the initial amount $Q_0$? Explain why this is or is not reasonable from the modeling point of view.
\end{enumerate}

Hint: You will need to solve equations of the form $Q_0 e^{kT} = \alpha Q_0$ for $T$, where $\alpha$ is a constant such as $\tfrac12$ or $2$. Divide both sides by $Q_0$ and take logarithms.

\smallskip

(e) Explore the limitations and possible extensions of this model.

\begin{enumerate}
\item[(i)] In reality, a population cannot grow forever without bound. Suppose we modify the model so that the growth rate slows down when $Q(t)$ becomes large, for instance by introducing a carrying capacity $K>0$. One classical model is
\[
\frac{dQ}{dt} = k Q(t)\left(1 - \frac{Q(t)}{K}\right).
\]
Compare this with your original equation. In what sense is the original exponential growth model a special case of this more complicated model?
\item[(ii)] In radioactive decay, $k$ is essentially constant in time. In some population models, however, the effective growth rate may vary seasonally, so that it is more realistic to write
\[
\frac{dQ}{dt} = k(t)\, Q(t),
\]
where $k(t)$ is a known function of time. Based on the separation-of-variables step you used in part (b), sketch (in words or symbols) how you would attempt to solve this more general equation.
\end{enumerate}

Hint: For part (e)(ii), think about repeating the step where you placed all $Q$'s on one side and all $t$'s on the other. What changes if $k$ depends on $t$?
\end{problem}

% ===== Example 1: Linear First-Order Equation for Exponential Growth and Decay (full solution) =====
\begin{problem}[Linear First-Order Equation for Exponential Growth and Decay]
Consider a quantity $Q(t)$ that changes in time so that its rate of change is proportional to its current value:
\[
\frac{dQ}{dt} = k Q(t),
\]
where $k$ is a real constant. 

\begin{enumerate}
\item[(a)] Solve this differential equation to obtain the general solution.
\item[(b)] Impose the initial condition $Q(0) = Q_0$ with $Q_0>0$ and write the corresponding solution $Q(t)$ explicitly.
\item[(c)] Assume $k < 0$ and define the half-life $T_{1/2}$ to be the time such that $Q(T_{1/2}) = \tfrac12 Q_0$. Express $T_{1/2}$ in terms of $k$.
\item[(d)] Assume $k > 0$ and define the doubling time $T_2$ to be the time such that $Q(T_2) = 2 Q_0$. Express $T_2$ in terms of $k$.
\item[(e)] In each of the cases (c) and (d), state whether the characteristic time ($T_{1/2}$ or $T_2$) depends on the initial amount $Q_0$, and briefly explain why this is consistent with the model.
\end{enumerate}
\end{problem}

\begin{solution}
We are given the first-order linear ordinary differential equation
\[
\frac{dQ}{dt} = k Q(t),
\]
where $k$ is a real constant, and we are asked to solve it and interpret parameters such as half-life and doubling time. This is the simplest nontrivial example of a linear ODE and illustrates that linear constant-coefficient equations produce exponential solutions.

\medskip

\noindent\textbf{(a) General solution.}
We solve the equation directly by separation of variables. For all $t$ where $Q(t)\neq 0$, we can write
\[
\frac{dQ}{dt} = k Q(t)
\quad\Longrightarrow\quad
\frac{1}{Q(t)}\,\frac{dQ}{dt} = k.
\]
Interpreting $\dfrac{dQ}{dt}\,dt$ as $dQ$, we separate variables:
\[
\frac{1}{Q}\, dQ = k\, dt.
\]
We now integrate both sides with respect to $t$ (equivalently, with respect to $Q$ on the left):
\[
\int \frac{1}{Q}\, dQ = \int k\, dt.
\]
The antiderivatives are
\[
\ln|Q| = kt + C,
\]
where $C$ is a constant of integration.

To solve for $Q$, we exponentiate both sides:
\[
|Q| = e^{kt + C} = e^C e^{kt}.
\]
Because $e^C>0$, we can absorb both the absolute value and the positive multiplicative constant into a single nonzero constant $C_1\in\mathbb{R}$, writing
\[
Q(t) = C_1 e^{kt}.
\]
Thus, the general nontrivial solution has the exponential form
\[
Q(t) = C e^{kt},
\]
where $C$ is an arbitrary real constant. In addition, $Q(t)\equiv 0$ is also a solution, obtained by taking $C=0$. So the full family of solutions is $Q(t) = C e^{kt}$ for $C\in\mathbb{R}$.

\medskip

\noindent\textbf{(b) Solution with initial condition.}
We now impose the initial condition $Q(0) = Q_0$ with $Q_0>0$. Substituting $t=0$ into the general solution gives
\[
Q(0) = C e^{k\cdot 0} = C e^{0} = C.
\]
Therefore $C = Q_0$, and the unique solution satisfying this initial condition is
\[
Q(t) = Q_0 e^{kt}.
\]

\medskip

\noindent\textbf{(c) Half-life for $k<0$.}
Assume $k<0$, corresponding to exponential decay such as radioactive decay. By definition, the half-life $T_{1/2}$ is the time at which the quantity has decreased to one half of its initial value:
\[
Q(T_{1/2}) = \frac{1}{2} Q_0.
\]
Using the solution $Q(t) = Q_0 e^{kt}$, we substitute $t = T_{1/2}$ and obtain
\[
Q_0 e^{k T_{1/2}} = \frac{1}{2} Q_0.
\]
Because $Q_0>0$, we can divide both sides by $Q_0$ to get
\[
e^{k T_{1/2}} = \frac{1}{2}.
\]
Taking the natural logarithm of both sides yields
\[
k T_{1/2} = \ln\left(\frac{1}{2}\right) = -\ln 2.
\]
Solving for $T_{1/2}$ gives
\[
T_{1/2} = \frac{-\ln 2}{k}.
\]
Since $k<0$, the quotient $-\ln 2/k$ is positive, as expected for a time. It is often convenient to rewrite this as
\[
T_{1/2} = \frac{\ln 2}{|k|},
\]
because $|k| = -k$ when $k<0$.

\medskip

\noindent\textbf{(d) Doubling time for $k>0$.}
Now assume $k>0$, corresponding to exponential growth. By definition, the doubling time $T_2$ is the time at which the quantity has increased to twice its initial value:
\[
Q(T_2) = 2 Q_0.
\]
Again using the solution $Q(t) = Q_0 e^{kt}$, we set $t = T_2$ and obtain
\[
Q_0 e^{k T_2} = 2 Q_0.
\]
Dividing both sides by $Q_0$ (which is positive) gives
\[
e^{k T_2} = 2.
\]
Taking natural logarithms,
\[
k T_2 = \ln 2,
\]
and hence
\[
T_2 = \frac{\ln 2}{k}.
\]
Here $k>0$, so $T_2$ is again positive. Note that the same constant $\ln 2$ appears, with the sign of $k$ distinguishing decay from growth.

\medskip

\noindent\textbf{(e) Dependence on $Q_0$ and modeling interpretation.}
In parts (c) and (d), the characteristic times $T_{1/2}$ and $T_2$ are given by
\[
T_{1/2} = \frac{\ln 2}{|k|}, 
\qquad
T_2 = \frac{\ln 2}{k}.
\]
Neither formula involves the initial amount $Q_0$. Therefore, the half-life and the doubling time are independent of the initial condition.

This independence is consistent with the modeling assumption that the rate of change is \emph{proportional} to $Q(t)$ with a fixed proportionality constant $k$. In the decay case, each individual atom has the same probability per unit time of decaying, regardless of how many atoms there are, so the characteristic time scale is determined solely by $k$ and not by $Q_0$. In the growth case, if each individual reproduces at the same rate, it takes the same amount of time for any starting population to double; what changes with $Q_0$ is the absolute size of the population, not the fraction by which it increases over a given characteristic time.

\medskip

\noindent\textbf{Connection to direct methods for linear ODEs.}
This example illustrates the main idea of the section on direct methods for solving linear ODEs: when we have a first-order linear equation with constant coefficients, such as $Q' = kQ$, we can often solve it explicitly by straightforward manipulations. Here, the equation is both linear and separable, and the solution is obtained by simple integration after separating variables. The result is an exponential function, showing that linear constant-coefficient equations naturally produce exponential responses. This pattern persists in more complicated linear systems and higher-order linear ODEs, where eigenvalues and exponentials play a central role.
\end{solution}

% ===== Example 2: Forced Mass–Spring System and Undetermined Coefficients (inquiry-based) =====
\begin{problem}[Forced Mass–Spring System and Undetermined Coefficients]
A mass attached to a spring and subject to an external periodic force is a standard model for vibrations in engineering and physics. Neglecting friction, the displacement $x(t)$ from equilibrium satisfies a second-order linear ordinary differential equation with constant coefficients. When the forcing is sinusoidal, the solution typically consists of transient oscillations plus a steady-state oscillation at the forcing frequency. In some special situations, called \emph{resonance}, the amplitude of the oscillations can grow very large.

In this problem we explore how to solve such equations using the method of undetermined coefficients, and we see how resonance appears naturally in the mathematics.

Consider a unit mass attached to a spring with spring constant $k = 4$. An external horizontal force $F(t)$ acts on the mass. Neglect all friction or damping. Then the displacement $x(t)$ satisfies
\[
x''(t) + 4 x(t) = F(t),
\]
where $F(t)$ is measured in suitable units.

\smallskip

\noindent\textbf{(a) Natural oscillations (no forcing).}  
First suppose there is no external force, so $F(t) \equiv 0$ and the equation reduces to
\[
x''(t) + 4 x(t) = 0.
\]

\begin{enumerate}
\item[(i)] Solve this homogeneous equation by the characteristic equation method and write the general solution $x_h(t)$.
\item[(ii)] What is the natural (angular) frequency of the undamped mass–spring system? How is it related to the eigenvalues you found?
\end{enumerate}

Hint: Recall that for $x'' + \omega_0^2 x = 0$, the general solution is a combination of $\cos(\omega_0 t)$ and $\sin(\omega_0 t)$.

\smallskip

\noindent\textbf{(b) A first forced problem: nonresonant sinusoidal forcing.}  
Now let the external force be
\[
F(t) = 2 \cos(3t),
\]
so the equation becomes
\[
x''(t) + 4 x(t) = 2 \cos(3t).
\]

\begin{enumerate}
\item[(i)] Explain why the general solution should be written in the form
\[
x(t) = x_h(t) + x_p(t),
\]
where $x_h$ solves the homogeneous equation from part (a), and $x_p$ is a particular solution of the forced equation.
\item[(ii)] Using the method of undetermined coefficients, propose a form for $x_p(t)$ for this nonresonant problem.  
What combination of sines and cosines with frequency $3$ should you try?
\item[(iii)] Substitute your proposed $x_p(t)$ into the differential equation and solve for the unknown coefficients.  
Write down one particular solution $x_p(t)$ explicitly.
\end{enumerate}

Hint: Try $x_p(t) = A \cos(3t) + B \sin(3t)$ and determine $A$ and $B$.

\smallskip

\noindent\textbf{(c) Resonant forcing: what goes wrong with the first guess?}  
Now change the forcing frequency to match the natural frequency of the system:
\[
F(t) = 2 \cos(2t),
\]
so that
\[
x''(t) + 4 x(t) = 2 \cos(2t).
\]

\begin{enumerate}
\item[(i)] If you repeat the same guess as in part (b),
\[
x_p(t) = A \cos(2t) + B \sin(2t),
\]
what happens when you substitute this into the equation? Do you get equations that determine $A$ and $B$, or do you run into a difficulty? Describe precisely what fails.
\item[(ii)] How is the failure in (i) related to the homogeneous solution $x_h(t)$ from part (a)?
\end{enumerate}

Hint: Compare your guessed $x_p$ with the terms already present in $x_h(t)$.

\smallskip

\noindent\textbf{(d) Fixing the resonance and finding the full solution.}  
To repair the method of undetermined coefficients in the resonant case, you modify your guess so that it is not a solution of the homogeneous equation.

\begin{enumerate}
\item[(i)] Propose a new form for $x_p(t)$ by multiplying your old guess by a suitable power of $t$. What form should you try here?
\item[(ii)] Substitute your new $x_p(t)$ into the equation $x'' + 4x = 2\cos(2t)$ and solve for the coefficients.  
Obtain an explicit particular solution $x_p(t)$.
\item[(iii)] Combine $x_h(t)$ and your $x_p(t)$ to form the general solution
\[
x(t) = x_h(t) + x_p(t).
\]
Now impose the initial conditions
\[
x(0) = 0, \qquad x'(0) = 0
\]
and determine the unique solution satisfying these conditions.
\item[(iv)] Study the behavior of your solution as $t \to \infty$. Does the amplitude remain bounded, or does it grow without bound? How fast does it grow?  
Explain how this is related to the physical phenomenon of resonance.
\end{enumerate}

Hint: When the forcing frequency equals the natural frequency, the method of undetermined coefficients prescribes multiplying by $t$; look for growth of order $t$ in the amplitude.

\smallskip

\noindent\textbf{(e) Extensions and variations.}  
In real systems there is often some damping and sometimes other types of forcing.

\begin{enumerate}
\item[(i)] Suppose we add a damping term $2x'(t)$ to the model and consider
\[
x''(t) + 2x'(t) + 4x(t) = 2\cos(2t).
\]
Without solving this new equation in full detail, explain how you would modify the method of undetermined coefficients to find a particular solution. What form would you guess for $x_p(t)$? Do you expect unbounded growth of the amplitude as $t \to \infty$ in this damped case? Briefly justify your answer.
\item[(ii)] Imagine instead that the forcing is not sinusoidal but exponential,
\[
F(t) = e^{t},
\]
so that the equation becomes
\[
x''(t) + 4x(t) = e^t.
\]
What form of $x_p(t)$ would you try using undetermined coefficients? How would your guess change if the homogeneous solution already contained terms proportional to $e^t$?
\end{enumerate}

Hint: For exponential forcing $e^{\lambda t}$, try $x_p(t) = Ce^{\lambda t}$, unless $e^{\lambda t}$ already solves the homogeneous equation, in which case multiply by a suitable power of $t$.
\end{problem}

% ===== Example 2: Forced Mass–Spring System and Undetermined Coefficients (full solution) =====
\begin{problem}[Forced Mass–Spring System and Undetermined Coefficients]
Consider the undamped mass–spring system
\[
x''(t) + 4 x(t) = F(t),
\]
where $x(t)$ is the displacement from equilibrium.

\begin{enumerate}
\item[(i)] Solve the homogeneous equation $x'' + 4x = 0$ and identify the natural angular frequency of the system.
\item[(ii)] Use the method of undetermined coefficients to find a particular solution of
\[
x''(t) + 4x(t) = 2\cos(3t),
\]
and hence write the general solution.
\item[(iii)] Now consider the resonant forcing
\[
x''(t) + 4x(t) = 2\cos(2t).
\]
Explain why the naive guess $x_p(t) = A\cos(2t) + B\sin(2t)$ fails, and then find a correct particular solution by modifying the guess appropriately.
\item[(iv)] For the resonant equation in (iii), impose the initial conditions $x(0) = 0$ and $x'(0) = 0$. Find the unique solution and describe its long-time behavior as $t \to \infty$. Interpret this behavior in terms of resonance.
\item[(v)] Briefly indicate how the method of undetermined coefficients would be applied if the equation were
\[
x''(t) + 2x'(t) + 4x(t) = 2\cos(2t)
\quad\text{or}\quad
x''(t) + 4x(t) = e^t.
\]
State the trial form you would use for $x_p(t)$ in each case.
\end{enumerate}
\end{problem}

\begin{solution}
We analyze a standard forced mass–spring system governed by a linear second-order ordinary differential equation with constant coefficients. The main idea of the \emph{direct method} used here is to decompose the solution into the sum of the homogeneous solution (describing natural or transient behavior) and a particular solution (describing the forced or steady-state response). The method of undetermined coefficients provides a systematic way to guess the form of the particular solution for many common forcing terms.

\medskip

\noindent\textbf{(i) Homogeneous solution and natural frequency.}  
The homogeneous equation is
\[
x''(t) + 4x(t) = 0.
\]
We solve this by the characteristic equation. Assume $x(t) = e^{rt}$, which gives
\[
r^2 + 4 = 0 \quad\Longrightarrow\quad r = \pm 2i.
\]
Thus the general real solution is
\[
x_h(t) = C_1\cos(2t) + C_2\sin(2t),
\]
where $C_1$ and $C_2$ are arbitrary constants. The natural angular frequency of the oscillation is therefore $\omega_0 = 2$.

\medskip

\noindent\textbf{(ii) Nonresonant forcing $2\cos(3t)$.}  
We now consider
\[
x''(t) + 4x(t) = 2\cos(3t).
\]
The general solution is of the form
\[
x(t) = x_h(t) + x_p(t),
\]
where $x_h(t)$ is the homogeneous solution found above, and $x_p(t)$ is any particular solution of the nonhomogeneous equation.

Because the forcing term is sinusoidal with angular frequency $3$, and $3$ is \emph{not} equal to the natural frequency $2$, a standard undetermined-coefficients guess is
\[
x_p(t) = A\cos(3t) + B\sin(3t),
\]
where $A$ and $B$ are constants to be determined.

We compute derivatives:
\[
x_p'(t) = -3A\sin(3t) + 3B\cos(3t),
\]
\[
x_p''(t) = -9A\cos(3t) - 9B\sin(3t).
\]
Substitute into the differential equation:
\[
x_p'' + 4x_p
= \bigl(-9A\cos(3t) - 9B\sin(3t)\bigr) + 4\bigl(A\cos(3t) + B\sin(3t)\bigr).
\]
Combine like terms:
\[
x_p'' + 4x_p
= (-9A + 4A)\cos(3t) + (-9B + 4B)\sin(3t)
= (-5A)\cos(3t) + (-5B)\sin(3t).
\]
We require this to equal the forcing term $2\cos(3t)$:
\[
-5A\cos(3t) - 5B\sin(3t) = 2\cos(3t) + 0\cdot\sin(3t).
\]
Equating coefficients of $\cos(3t)$ and $\sin(3t)$ gives the system
\[
-5A = 2, \qquad -5B = 0.
\]
Thus
\[
A = -\frac{2}{5}, \qquad B = 0.
\]
Therefore one particular solution is
\[
x_p(t) = -\frac{2}{5}\cos(3t).
\]

The general solution of the nonresonant problem is then
\[
x(t) = C_1\cos(2t) + C_2\sin(2t) - \frac{2}{5}\cos(3t).
\]
Here the first two terms represent natural oscillations at frequency $2$ (which will depend on initial conditions), and the last term is a forced oscillation at frequency $3$, with fixed amplitude determined by the forcing.

\medskip

\noindent\textbf{(iii) Resonant forcing $2\cos(2t)$.}  
Now we consider
\[
x''(t) + 4x(t) = 2\cos(2t),
\]
where the forcing frequency $2$ \emph{equals} the natural frequency of the system.

\smallskip

\emph{Failure of the naive guess.}  
If we try the same form as in the nonresonant case,
\[
x_p(t) = A\cos(2t) + B\sin(2t),
\]
we immediately see that this is contained in the homogeneous solution
\[
x_h(t) = C_1\cos(2t) + C_2\sin(2t).
\]
Any linear combination $A\cos(2t) + B\sin(2t)$ solves the homogeneous equation $x'' + 4x = 0$. If we substitute this $x_p$ into $x'' + 4x$, we obtain
\[
x_p'' + 4x_p = 0,
\]
which can never equal $2\cos(2t)$ for all $t$. There is no choice of $A$ and $B$ that makes $x_p'' + 4x_p$ equal to the nonzero forcing term. Thus the naive guess fails because it produces only homogeneous solutions.

\smallskip

\emph{Corrected guess by multiplying by $t$.}  
The method of undetermined coefficients instructs us that when the forcing term is a solution of the homogeneous equation (or is of a type that overlaps with it), we should multiply our usual trial function by a suitable power of $t$ to obtain a linearly independent form.

Since $\cos(2t)$ and $\sin(2t)$ already appear in $x_h$, we try
\[
x_p(t) = t\bigl(A\cos(2t) + B\sin(2t)\bigr).
\]
We differentiate:
\[
x_p'(t) 
= A\cos(2t) + B\sin(2t)
+ t\bigl(-2A\sin(2t) + 2B\cos(2t)\bigr),
\]
\[
x_p''(t)
= \underbrace{-2A\sin(2t) + 2B\cos(2t)}_{\text{derivative of the first part}}
+ \underbrace{\bigl(-2A\sin(2t) + 2B\cos(2t)\bigr)}_{\text{derivative of the second part without $t$}}
+ t\bigl(-4A\cos(2t) - 4B\sin(2t)\bigr).
\]
Collecting terms, we obtain
\[
x_p''(t) = (-4A\sin(2t) + 4B\cos(2t)) + t(-4A\cos(2t) - 4B\sin(2t)).
\]

Now compute $x_p'' + 4x_p$:
\begin{align*}
x_p''(t) + 4x_p(t)
&= \bigl(-4A\sin(2t) + 4B\cos(2t)\bigr) 
+ t(-4A\cos(2t) - 4B\sin(2t)) \\
&\quad + 4t\bigl(A\cos(2t) + B\sin(2t)\bigr).
\end{align*}
The $t$-dependent terms cancel:
\[
t(-4A\cos(2t) - 4B\sin(2t)) + 4t(A\cos(2t) + B\sin(2t)) = 0.
\]
So we are left with
\[
x_p''(t) + 4x_p(t) = -4A\sin(2t) + 4B\cos(2t).
\]
We want this to equal the forcing term $2\cos(2t)$. Thus
\[
-4A\sin(2t) + 4B\cos(2t) = 2\cos(2t) + 0\cdot \sin(2t).
\]
Equating coefficients of $\cos(2t)$ and $\sin(2t)$ gives
\[
4B = 2, \qquad -4A = 0,
\]
so
\[
A = 0, \qquad B = \frac{1}{2}.
\]
Therefore a valid particular solution is
\[
x_p(t) = t\cdot \frac{1}{2}\sin(2t) = \frac{t}{2}\sin(2t).
\]

\medskip

\noindent\textbf{(iv) Solution with initial conditions and long-time behavior.}  
The general solution of the resonant equation is
\[
x(t) = x_h(t) + x_p(t)
= C_1\cos(2t) + C_2\sin(2t) + \frac{t}{2}\sin(2t).
\]

We now impose $x(0) = 0$ and $x'(0) = 0$.

First, use $x(0) = 0$:
\[
x(0) = C_1\cos(0) + C_2\sin(0) + \frac{0}{2}\sin(0) = C_1.
\]
Hence $C_1 = 0$.

Next, compute $x'(t)$:
\[
x'(t) = C_1(-2\sin(2t)) + C_2(2\cos(2t)) + \frac{1}{2}\sin(2t) + \frac{t}{2}\cdot 2\cos(2t).
\]
Substitute $C_1 = 0$:
\[
x'(t) = 2C_2\cos(2t) + \frac{1}{2}\sin(2t) + t\cos(2t).
\]
Now impose $x'(0) = 0$:
\[
x'(0) = 2C_2\cos(0) + \frac{1}{2}\sin(0) + 0\cdot \cos(0) = 2C_2.
\]
Thus $C_2 = 0$.

Therefore the unique solution satisfying the given initial conditions is
\[
x(t) = \frac{t}{2}\sin(2t).
\]

To analyze the long-time behavior, observe that $\sin(2t)$ remains bounded between $-1$ and $1$ for all $t$, but it is multiplied by the factor $t/2$. Thus the envelope of the oscillation grows linearly in time:
\[
\lvert x(t)\rvert \leq \frac{t}{2}\quad\text{for all }t\ge 0.
\]
Hence the amplitude grows without bound as $t \to \infty$; more precisely, it grows on the order of $t$.

Physically, this is the phenomenon of \emph{resonance}. The forcing frequency matches the natural frequency of the undamped system, so each push from the external force constructively adds energy to the system. The transient contribution from the homogeneous solution has decayed (in this undamped case, it simply reflects initial conditions), but the forced component continues to build, and the amplitude increases indefinitely.

\medskip

\noindent\textbf{(v) Other forcings and damping: trial forms.}  

\smallskip

\emph{(a) Damped, sinusoidally forced system.}  
Consider
\[
x''(t) + 2x'(t) + 4x(t) = 2\cos(2t).
\]
The homogeneous equation $x'' + 2x' + 4x = 0$ has characteristic equation
\[
r^2 + 2r + 4 = 0,
\]
with roots $r = -1 \pm i\sqrt{3}$. Thus the homogeneous solution is
\[
x_h(t) = e^{-t}\bigl(C_1\cos(\sqrt{3}t) + C_2\sin(\sqrt{3}t)\bigr),
\]
which oscillates at frequency $\sqrt{3}$ and is exponentially damped.

The forcing is still $2\cos(2t)$, a sinusoid with frequency $2$. This is not a solution of the homogeneous equation; the homogeneous terms involve a factor $e^{-t}$ and a different frequency. Therefore, the method of undetermined coefficients prescribes the same trial form as in a nonresonant undamped case:
\[
x_p(t) = A\cos(2t) + B\sin(2t).
\]
Substituting this into the damped equation and solving for $A$ and $B$ would yield a bounded particular solution.

In the presence of damping, even when the forcing frequency coincides with the undamped natural frequency, the amplitude remains bounded in time. Energy is continually supplied by the forcing but also continually dissipated by damping, leading to a steady-state oscillation of finite amplitude rather than unbounded growth.

\smallskip

\emph{(b) Exponential forcing.}  
Consider instead
\[
x''(t) + 4x(t) = e^t.
\]
Here the forcing is the exponential $e^t = e^{1\cdot t}$. A standard undetermined-coefficients trial function for a forcing of the form $e^{\lambda t}$ is
\[
x_p(t) = Ce^{\lambda t},
\]
provided that $e^{\lambda t}$ is not already part of the homogeneous solution.

In our original homogeneous solution, $x_h(t) = C_1\cos(2t) + C_2\sin(2t)$, there is no $e^t$ term. Thus we would try
\[
x_p(t) = Ce^t.
\]
Substituting into $x'' + 4x = e^t$ would give an algebraic equation for $C$.

If instead the homogeneous equation had a root $r = 1$, so that $e^t$ were already part of $x_h(t)$, then the naive guess $Ce^t$ would fail just as in the resonant sinusoidal case. The method then prescribes multiplying by $t$ to obtain a linearly independent trial function:
\[
x_p(t) = Ct e^t,
\]
or, if necessary (for higher multiplicity), even higher powers of $t$.

\medskip

In summary, this example illustrates the main ideas of the section on \emph{Direct Methods for Solving Linear ODEs}. We:

\begin{itemize}
\item solved the homogeneous linear equation via its characteristic polynomial to find the natural oscillatory behavior;
\item used the method of undetermined coefficients to construct particular solutions for standard forcing terms (sinusoids and exponentials);
\item saw how resonance arises when the forcing term overlaps with the homogeneous solution, requiring a modification of the trial function (multiplication by $t$);
\item and interpreted the long-time behavior of solutions in physical terms, distinguishing between bounded steady-state oscillations and unbounded growth due to resonance.
\end{itemize}
These techniques form a central part of the toolkit for solving linear ordinary differential equations with constant coefficients in applied mathematics.
\end{solution}

% ===== Example 3: RLC Electrical Circuit as a Second-Order Linear ODE (inquiry-based) =====
\begin{problem}[RLC Electrical Circuit as a Second-Order Linear ODE]
An ideal series RLC circuit consists of a resistor of resistance $R>0$, an inductor of inductance $L>0$, and a capacitor of capacitance $C>0$, all in series with a voltage source $E(t)$. The state of the circuit can be described by the charge $q(t)$ on the capacitor plates or by the current $i(t)$ in the loop. Kirchhoff's voltage law tells us that the sum of voltage drops across the three elements equals the supplied voltage $E(t)$, which will lead to a second-order linear ordinary differential equation for $q(t)$. Depending on the values of $R$, $L$, and $C$, the circuit can behave like a damped oscillator with qualitatively different transient responses, and when driven by a time-dependent source $E(t)$ it can act as a frequency-selective filter.

(a) Let $q(t)$ denote the charge on the capacitor and $i(t)$ the current in the circuit. Recall that the voltage drops across the elements are given by
\[
V_R = R i(t), \quad V_L = L\,\dfrac{di}{dt}(t), \quad V_C = \dfrac{1}{C} q(t),
\]
and that $i(t) = \dfrac{dq}{dt}(t)$. Use Kirchhoff's voltage law
\[
V_R + V_L + V_C = E(t)
\]
to derive a single second-order linear ordinary differential equation for the charge $q(t)$ when the circuit is driven by a source $E(t)$. Write your final answer in the form
\[
L q''(t) + R q'(t) + \frac{1}{C} q(t) = E(t).
\]
Explain briefly how each physical law enters your derivation.

% Hint: First write the KVL equation in terms of $i$ and $q$, then use $i = q'$ to eliminate $i$.

(b) For the remainder of parts (b)--(d), suppose the source is turned off, so $E(t) \equiv 0$. Then the charge satisfies
\[
L q''(t) + R q'(t) + \frac{1}{C} q(t) = 0.
\]
Divide this equation by $L$ to obtain a standard form with leading coefficient $1$, and write down the characteristic polynomial for this homogeneous linear ODE. Express its roots in terms of $R$, $L$, and $C$. Introduce the discriminant
\[
\Delta = R^2 - 4\frac{L}{C},
\]
and explain why the sign of $\Delta$ leads to three qualitatively distinct types of solutions.

% Hint: You should obtain a quadratic characteristic equation in $\lambda$, solve it with the quadratic formula, and then analyze the possibilities $\Delta>0$, $\Delta=0$, and $\Delta<0$.

(c) Consider first the underdamped case $\Delta<0$. Show that in this case the general solution can be written in the form
\[
q(t) = e^{-\alpha t} \bigl( A \cos(\omega t) + B \sin(\omega t)\bigr)
\]
for suitable constants $\alpha>0$, $\omega>0$ depending on $R$, $L$, and $C$, and arbitrary real constants $A$ and $B$. Determine $\alpha$ and $\omega$ explicitly in terms of $R$, $L$, and $C$. Then, assuming initial conditions $q(0)=q_0$ and $q'(0)=i(0)=i_0$ are given, express $A$ and $B$ in terms of $q_0$ and $i_0$.

% Hint: Start from the complex-conjugate roots of the characteristic equation and rewrite the complex exponential solution in terms of real sines and cosines.

(d) Now consider the driven circuit with a sinusoidal source
\[
E(t) = E_0 \cos(\omega t), \qquad E_0>0,\ \omega>0,
\]
and assume $R>0$ so that all transients decay for large $t$. Return to the full inhomogeneous equation
\[
L q''(t) + R q'(t) + \frac{1}{C} q(t) = E_0 \cos(\omega t).
\]
(i) Use the method of undetermined coefficients to seek a particular solution of the form
\[
q_p(t) = a \cos(\omega t) + b \sin(\omega t).
\]
Derive the linear system of equations that $a$ and $b$ must satisfy, and solve it to obtain $a$ and $b$ in terms of $R$, $L$, $C$, $E_0$, and $\omega$.

(ii) Show that the steady-state (long-time) oscillation of the charge has the form
\[
q_{\text{ss}}(t) = A(\omega)\cos\bigl(\omega t - \phi(\omega)\bigr),
\]
for some amplitude $A(\omega)\ge 0$ and phase shift $\phi(\omega)$ depending on $\omega$. Write an explicit expression for $A(\omega)$, and briefly describe how $A(\omega)$ behaves as $\omega\to 0$ and as $\omega\to\infty$.

% Hint: From $a$ and $b$, use the identity $a\cos(\omega t)+b\sin(\omega t) = A\cos(\omega t-\phi)$ with $A = \sqrt{a^2+b^2}$.

(e) \emph{Explorations and extensions.}
\begin{enumerate}
    \item Suppose $R=0$ (no resistor). What do your formulas predict for the homogeneous solutions when $E(t)\equiv 0$? How does this relate to the idea of an undamped oscillator and electrical resonance?
    \item In many applications one is interested not in the charge $q(t)$, but in the voltage across a single component, for instance across the resistor. Explain qualitatively, using your expression for $A(\omega)$ from part (d), why a series RLC circuit can be used as a frequency-selective filter in signal processing.
\end{enumerate}

% Hint: Think about which frequencies lead to large steady-state amplitudes, and which frequencies are strongly suppressed.
\end{problem}

% ===== Example 3: RLC Electrical Circuit as a Second-Order Linear ODE (full solution) =====
\begin{problem}[RLC Electrical Circuit as a Second-Order Linear ODE]
Consider a series RLC circuit with resistance $R>0$, inductance $L>0$, and capacitance $C>0$, driven by a voltage source $E(t)$. Let $q(t)$ denote the charge on the capacitor and $i(t)=q'(t)$ the current.

\begin{enumerate}
    \item Derive the differential equation satisfied by $q(t)$ using Kirchhoff's voltage law and the constitutive laws
    \[
    V_R = R i(t),\quad V_L = L i'(t),\quad V_C = \frac{1}{C}q(t).
    \]
    \item For the unforced circuit $E(t)\equiv 0$, write the homogeneous equation in standard form, find the characteristic equation, and classify the behavior of solutions as overdamped, critically damped, or underdamped according to the sign of the discriminant.
    \item In the underdamped case, show that the general solution can be written as
    \[
    q(t) = e^{-\alpha t}\bigl(A\cos(\omega t)+B\sin(\omega t)\bigr),
    \]
    and determine $\alpha$ and $\omega$ in terms of $R$, $L$, and $C$.
    \item For a sinusoidal forcing $E(t)=E_0\cos(\omega t)$ with $E_0>0$ and $R>0$, use the method of undetermined coefficients to find a particular solution of the form $q_p(t)=a\cos(\omega t)+b\sin(\omega t)$, and hence express the steady-state solution as
    \[
    q_{\mathrm{ss}}(t) = A(\omega)\cos\bigl(\omega t - \phi(\omega)\bigr).
    \]
    Obtain an explicit formula for the amplitude $A(\omega)$ and comment briefly on its dependence on the driving frequency $\omega$.
\end{enumerate}
\end{problem}

\begin{solution}
We begin by translating the physical laws of the circuit into an ordinary differential equation for the charge on the capacitor.

\medskip
\noindent\textbf{(1) Derivation of the ODE.}
In a series circuit, the same current $i(t)$ flows through all components. The voltage drops across the resistor, inductor, and capacitor are given by
\[
V_R = R i(t),\qquad V_L = L\,\frac{di}{dt}(t),\qquad V_C = \frac{1}{C}q(t),
\]
and the current and charge are related by $i(t) = \dfrac{dq}{dt}(t)$.

Kirchhoff's voltage law states that the sum of the voltage drops equals the applied source voltage $E(t)$:
\[
V_R + V_L + V_C = E(t).
\]
Substituting the expressions above, we obtain
\[
R i(t) + L\,\frac{di}{dt}(t) + \frac{1}{C}q(t) = E(t).
\]
Now use $i(t)=q'(t)$ and $i'(t)=q''(t)$ to eliminate the current in favor of $q$:
\[
R q'(t) + L q''(t) + \frac{1}{C} q(t) = E(t).
\]
Reordering the terms yields the second-order linear ODE for the charge:
\[
L q''(t) + R q'(t) + \frac{1}{C} q(t) = E(t).
\]
This is a linear equation with constant coefficients and a (possibly) time-dependent forcing $E(t)$.

\medskip
\noindent\textbf{(2) Homogeneous equation and classification.}
For the unforced circuit with $E(t)\equiv 0$, the charge satisfies
\[
L q''(t) + R q'(t) + \frac{1}{C} q(t) = 0.
\]
Dividing through by $L$ puts the equation into standard form:
\[
q''(t) + \frac{R}{L} q'(t) + \frac{1}{LC} q(t) = 0.
\]
We seek exponential solutions $q(t) = e^{\lambda t}$, which leads to the characteristic equation
\[
\lambda^2 + \frac{R}{L} \lambda + \frac{1}{LC} = 0.
\]
Equivalently, multiplying by $L$:
\[
L\lambda^2 + R\lambda + \frac{1}{C} = 0.
\]
The roots are given by the quadratic formula:
\[
\lambda_{1,2} = \frac{-R \pm \sqrt{R^2 - 4\frac{L}{C}}}{2L}.
\]
It is convenient to define the discriminant
\[
\Delta = R^2 - 4\frac{L}{C}.
\]
Then we have three cases.

\begin{itemize}
    \item \emph{Overdamped} ($\Delta>0$): The roots $\lambda_1$ and $\lambda_2$ are distinct real numbers, both negative (since $R>0$, $L>0$, and $1/C>0$). The general solution is
    \[
    q(t) = c_1 e^{\lambda_1 t} + c_2 e^{\lambda_2 t},
    \]
    which decays to zero as $t\to\infty$ without oscillating.

    \item \emph{Critically damped} ($\Delta=0$): The roots coincide, $\lambda_1=\lambda_2=\lambda=-R/(2L)$. The general solution is
    \[
    q(t) = (c_1 + c_2 t) e^{\lambda t},
    \]
    which again decays to zero but is the borderline case between oscillatory and nonoscillatory behavior.

    \item \emph{Underdamped} ($\Delta<0$): The roots are complex conjugates,
    \[
    \lambda_{1,2} = -\frac{R}{2L} \pm i\sqrt{\frac{1}{LC} - \frac{R^2}{4L^2}},
    \]
    with negative real part and nonzero imaginary part. The solution is a decaying oscillation.
\end{itemize}
Thus, the sign of $\Delta$ divides the parameter space into three qualitatively different regimes of transient behavior.

\medskip
\noindent\textbf{(3) Solution in the underdamped case.}
Assume $\Delta<0$, that is,
\[
R^2 < 4\frac{L}{C}.
\]
The characteristic roots are
\[
\lambda_{1,2} = -\frac{R}{2L} \pm i \sqrt{\frac{1}{LC} - \frac{R^2}{4L^2}}.
\]
Set
\[
\alpha = \frac{R}{2L} > 0, \qquad \omega = \sqrt{\frac{1}{LC} - \frac{R^2}{4L^2}} > 0.
\]
Then the roots are $-\alpha \pm i\omega$. The general complex solution is
\[
q(t) = \tilde{c}_1 e^{(-\alpha + i\omega)t} + \tilde{c}_2 e^{(-\alpha - i\omega)t}.
\]
Using Euler's formula and combining complex conjugate terms, this can be written as a real linear combination of $e^{-\alpha t}\cos(\omega t)$ and $e^{-\alpha t}\sin(\omega t)$:
\[
q(t) = e^{-\alpha t}\bigl(A \cos(\omega t) + B \sin(\omega t)\bigr),
\]
where $A$ and $B$ are real constants determined by the initial conditions.

To relate $A$ and $B$ to $q(0)$ and $q'(0)$, first evaluate at $t=0$:
\[
q(0) = A\cos 0 + B\sin 0 = A.
\]
Thus $A = q(0) = q_0$.

Next, differentiate:
\[
q'(t) = e^{-\alpha t}\bigl(-\alpha A\cos(\omega t) - \alpha B\sin(\omega t)\bigr)
      + e^{-\alpha t}\bigl(-A\omega\sin(\omega t) + B\omega\cos(\omega t)\bigr).
\]
Evaluating at $t=0$, where $\cos 0 = 1$ and $\sin 0 = 0$, gives
\[
q'(0) = -\alpha A + B\omega.
\]
Since $q'(0) = i(0) = i_0$, we obtain
\[
i_0 = -\alpha q_0 + \omega B,
\]
so
\[
B = \frac{i_0 + \alpha q_0}{\omega}.
\]
Therefore, in the underdamped case the unique solution satisfying $q(0)=q_0$ and $q'(0)=i_0$ is
\[
q(t) = e^{-\alpha t}\left(q_0\cos(\omega t)
+ \frac{i_0 + \alpha q_0}{\omega}\sin(\omega t)\right),
\]
with $\alpha$ and $\omega$ as above. This is a damped oscillation of angular frequency $\omega$ and exponential decay rate $\alpha$.

\medskip
\noindent\textbf{(4) Sinusoidal forcing and steady-state response.}
Now consider the forced equation
\[
L q''(t) + R q'(t) + \frac{1}{C} q(t) = E_0 \cos(\omega t),
\]
with $E_0>0$ and $R>0$. The general solution is the sum of a homogeneous (transient) part, which we have already analyzed, and a particular (forced) part. Because $R>0$, the homogeneous solution decays to zero as $t\to\infty$, so the long-time behavior is governed by any bounded particular solution.

By the method of undetermined coefficients, we look for a particular solution of the form
\[
q_p(t) = a\cos(\omega t) + b\sin(\omega t),
\]
where $a$ and $b$ are constants to be determined. We compute the derivatives:
\[
q_p'(t) = -a\omega\sin(\omega t) + b\omega\cos(\omega t),
\]
\[
q_p''(t) = -a\omega^2\cos(\omega t) - b\omega^2\sin(\omega t).
\]
Substituting into the differential equation gives
\[
L\bigl(-a\omega^2\cos\omega t - b\omega^2\sin\omega t\bigr)
+ R\bigl(-a\omega\sin\omega t + b\omega\cos\omega t\bigr)
+ \frac{1}{C}\bigl(a\cos\omega t + b\sin\omega t\bigr)
= E_0\cos\omega t.
\]
Group the cosine and sine terms:
\[
\bigl(-L a \omega^2 + R b \omega + \tfrac{1}{C}a\bigr)\cos(\omega t)
+ \bigl(-L b \omega^2 - R a \omega + \tfrac{1}{C}b\bigr)\sin(\omega t)
= E_0\cos(\omega t) + 0\cdot\sin(\omega t).
\]
For this identity to hold for all $t$, the coefficients of $\cos(\omega t)$ and $\sin(\omega t)$ on both sides must agree, giving the linear system
\[
\begin{cases}
- L a \omega^2 + R b \omega + \dfrac{1}{C}a = E_0, \\[0.5em]
- L b \omega^2 - R a \omega + \dfrac{1}{C}b = 0.
\end{cases}
\]
It is convenient to write
\[
K(\omega) = \frac{1}{C} - L\omega^2.
\]
Then the system becomes
\[
\begin{cases}
K(\omega) a + R\omega b = E_0, \\[0.3em]
- R\omega a + K(\omega) b = 0.
\end{cases}
\]
Solving, for example by Cramer's rule, we find
\[
a = \frac{E_0 K(\omega)}{K(\omega)^2 + (R\omega)^2},\qquad
b = \frac{E_0 R\omega}{K(\omega)^2 + (R\omega)^2}.
\]
The particular solution can be rewritten in amplitude–phase form. Note that
\[
q_p(t) = a\cos(\omega t) + b\sin(\omega t)
= A(\omega)\cos\bigl(\omega t - \phi(\omega)\bigr),
\]
where the amplitude $A(\omega)$ and phase shift $\phi(\omega)$ satisfy
\[
A(\omega) = \sqrt{a^2 + b^2}, \qquad
\cos\phi = \frac{a}{A(\omega)},\quad \sin\phi = \frac{b}{A(\omega)}.
\]
We compute the amplitude:
\[
A(\omega)^2 = a^2 + b^2
= \frac{E_0^2\bigl(K(\omega)^2 + (R\omega)^2\bigr)}{\bigl(K(\omega)^2 + (R\omega)^2\bigr)^2}
= \frac{E_0^2}{K(\omega)^2 + (R\omega)^2}.
\]
Hence
\[
A(\omega) = \frac{E_0}{\sqrt{\bigl(\frac{1}{C} - L\omega^2\bigr)^2 + (R\omega)^2}}.
\]
The full solution is the sum of the homogeneous and particular parts. Because the homogeneous part decays to zero when $R>0$, the long-time (steady-state) behavior is given by
\[
q_{\mathrm{ss}}(t) = A(\omega)\cos\bigl(\omega t - \phi(\omega)\bigr),
\]
with amplitude $A(\omega)$ as above and some phase shift $\phi(\omega)$ determined by $a$ and $b$.

We can now examine the dependence of $A(\omega)$ on $\omega$. For very low frequencies, $\omega\to 0$, we have $K(\omega)\to 1/C$ and $R\omega\to 0$, so
\[
A(\omega) \to \frac{E_0}{|1/C|} = E_0 C.
\]
Thus, the steady-state charge has a finite nonzero amplitude as $\omega\to 0$. For very high frequencies, $\omega\to\infty$, the term $-L\omega^2$ dominates in $K(\omega)$, so $K(\omega)^2\sim L^2\omega^4$ and
\[
A(\omega) \sim \frac{E_0}{|L|\omega^2} \to 0.
\]
Therefore, the amplitude of the charge oscillations decays to zero for large driving frequency. This frequency-dependent response is one manifestation of the circuit's filtering behavior.

\medskip
\noindent\textbf{Conceptual remarks.}
This example neatly illustrates the main ideas of direct methods for solving linear ODEs with constant coefficients. The homogeneous equation is solved via the characteristic polynomial, whose roots determine whether the motion is overdamped, critically damped, or underdamped. The physical damping corresponds mathematically to roots with negative real part. For the forced problem, the method of undetermined coefficients provides a particular solution tailored to the sinusoidal input. The solution decomposes into a transient part (the homogeneous solution) and a steady-state part (the particular solution). The frequency dependence of the steady-state amplitude $A(\omega)$ explains the use of RLC circuits as frequency-selective filters in signal processing.
\end{solution}

% ===== Example 4: Coupled Linear Populations and 2×2 Systems (inquiry-based) =====
\begin{problem}[Coupled Linear Populations and 2×2 Systems]
Many models for interacting populations, such as predator--prey or two competing species, lead to nonlinear systems of differential equations. However, near an equilibrium point, these nonlinear systems are often well approximated by a linear system. Analyzing this linear system already reveals a great deal about stability, oscillations, and how perturbations decay or grow. In this problem you will explore a specific $2\times 2$ linear system and see how eigenvalues and eigenvectors determine the behavior of the two populations.

Suppose that $(X^\ast,Y^\ast)$ is an equilibrium of a two-species model, and let
\[
u(t) = X(t) - X^\ast, \qquad v(t) = Y(t) - Y^\ast
\]
denote the deviations of the populations from equilibrium. A linearization of the nonlinear model near $(X^\ast,Y^\ast)$ leads to the system
\[
\begin{cases}
u'(t) = -u(t) - 2v(t),\\[0.3em]
v'(t) = 2u(t) - v(t).
\end{cases}
\tag{$\ast$}
\]

\smallskip

(a) Rewrite the system $(\ast)$ in matrix form
\[
\mathbf{z}'(t) = A\,\mathbf{z}(t),
\]
where $\mathbf{z}(t) = \begin{pmatrix}u(t)\\[0.2em]v(t)\end{pmatrix}$ and $A$ is a $2\times 2$ matrix. What is the equilibrium point of this \emph{linearized} system in terms of $u$ and $v$? How does it relate to the original equilibrium $(X^\ast,Y^\ast)$?

\medskip

(b) Compute the eigenvalues and (complex) eigenvectors of the matrix $A$.

\quad(i) Write down the characteristic polynomial of $A$ and solve for its roots.

\quad(ii) For one of the eigenvalues, find a corresponding eigenvector (you may work over $\mathbb{C}^2$ if necessary).

Based on the real parts of the eigenvalues, decide whether the equilibrium is linearly stable or unstable. Based on the imaginary parts, discuss whether you expect oscillations in $(u(t),v(t))$.

\emph{Hint:} Recall that for a $2\times 2$ matrix $A$, the characteristic polynomial has the form
\[
\lambda^2 - (\operatorname{tr}A)\,\lambda + \det(A) = 0.
\]

\medskip

(c) Use your eigenvalue calculation to write down the general (complex-valued) solution of the system in the form
\[
\mathbf{z}(t) = c_1 e^{\lambda_1 t}\mathbf{v}_1 + c_2 e^{\lambda_2 t}\mathbf{v}_2,
\]
where $\lambda_1,\lambda_2$ are the eigenvalues and $\mathbf{v}_1,\mathbf{v}_2$ are corresponding eigenvectors.

Then convert this complex expression into a real-valued general solution of the form
\[
\begin{pmatrix}u(t)\\[0.2em]v(t)\end{pmatrix}
= e^{at}\Big[ C_1\,\mathbf{p}(t) + C_2\,\mathbf{q}(t)\Big],
\]
where $a$ is a real constant, and $\mathbf{p}(t)$ and $\mathbf{q}(t)$ are real vector-valued functions involving sines and cosines.

\emph{Hint:} If $\lambda = a+bi$ and $\mathbf{v} = \mathbf{p} + i\mathbf{q}$ with $\mathbf{p},\mathbf{q}$ real vectors, then
\[
e^{\lambda t}\mathbf{v}
= e^{at}\Big[(\mathbf{p}\cos bt - \mathbf{q}\sin bt)
+ i(\mathbf{p}\sin bt + \mathbf{q}\cos bt)\Big].
\]
The real and imaginary parts give two linearly independent real solutions.

\medskip

(d) Now impose a specific initial condition. Suppose a perturbation initially displaces only the first population:
\[
u(0) = 1, \qquad v(0) = 0.
\]
Use your real general solution from part (c) to determine $u(t)$ and $v(t)$ explicitly for this initial condition.

Then:

\quad(i) Describe qualitatively what happens to $(u(t),v(t))$ as $t\to\infty$. Does it approach the equilibrium? Does it oscillate while doing so?

\quad(ii) Determine the angular frequency and period of these oscillations. How does the exponential factor affect the amplitude over time?

\emph{Hint:} You may find it useful to look at $u(t)^2 + v(t)^2$ to see how the ``radius'' of the trajectory in the $(u,v)$-plane changes with $t$.

\medskip

(e) Exploring variations of the model.

\quad(i) Consider the one-parameter family of matrices
\[
A_k = \begin{pmatrix} -1 & -k \\[0.2em] 2 & -1 \end{pmatrix},
\qquad k>0.
\]
Repeat (without detailed computation) the eigenvalue analysis in part (b) using the trace and determinant of $A_k$ to decide whether the eigenvalues are real or complex, and whether their real parts are positive or negative. For which values of $k$ does the system exhibit decaying oscillations?

\emph{Hint:} For $2\times 2$ matrices, complex conjugate eigenvalues occur when $(\operatorname{tr}A)^2 - 4\det(A) < 0$, and the real part of the eigenvalues is $\operatorname{tr}(A)/2$.

\smallskip

\quad(ii) Suppose instead that the linearization produced eigenvalues $\lambda = +1 \pm 2i$. Describe how the phase portrait and time evolution of $(u(t),v(t))$ would differ from the case you analyzed in parts (b)--(d). What would this suggest about the stability of the original equilibrium $(X^\ast,Y^\ast)$ in the nonlinear model?

\end{problem}

% ===== Example 4: Coupled Linear Populations and 2×2 Systems (full solution) =====
\begin{problem}[Coupled Linear Populations and 2×2 Systems]
Consider deviations $u(t)$ and $v(t)$ from an equilibrium of a two-species population model, governed by the linear system
\[
\begin{cases}
u'(t) = -u(t) - 2v(t),\\[0.3em]
v'(t) = 2u(t) - v(t).
\end{cases}
\]
\begin{enumerate}
\item Rewrite this system in matrix form $\mathbf{z}' = A\mathbf{z}$, where $\mathbf{z} = (u,v)^{\mathsf T}$. Identify the equilibrium point of the linear system in $(u,v)$-coordinates.
\item Find the eigenvalues and (complex) eigenvectors of $A$, and classify the equilibrium as a node, saddle, spiral, etc. Determine whether it is linearly stable or unstable, and whether solutions oscillate.
\item Using the eigenvalues and eigenvectors, derive the real general solution $(u(t),v(t))$ and then find the unique solution satisfying the initial condition $u(0)=1$, $v(0)=0$.
\item Describe the long-time behavior of this solution, including the decay rate, the presence or absence of oscillations, and the angular frequency (and period) of any oscillations.
\item For the one-parameter family
\[
A_k = \begin{pmatrix} -1 & -k \\[0.2em] 2 & -1 \end{pmatrix}, \quad k>0,
\]
express the trace and determinant of $A_k$ in terms of $k$, and determine for which $k>0$ the linear system has (i) decaying oscillations, (ii) non-oscillatory decay.
\end{enumerate}
\end{problem}

\begin{solution}
We analyze this linear $2\times 2$ system using the standard direct method of eigenvalues and eigenvectors, which leads to an explicit solution via the matrix exponential and allows a clear interpretation in the phase plane.

\medskip

\noindent\textbf{(1) Matrix form and equilibrium.}
We write
\[
\mathbf{z}(t) = \begin{pmatrix}u(t)\\[0.2em]v(t)\end{pmatrix},
\]
so the system becomes
\[
\mathbf{z}'(t)
=
\begin{pmatrix}
-1 & -2\\[0.2em]
2 & -1
\end{pmatrix}
\begin{pmatrix}u(t)\\[0.2em]v(t)\end{pmatrix}.
\]
Thus
\[
A = \begin{pmatrix} -1 & -2 \\[0.2em] 2 & -1 \end{pmatrix},
\qquad
\mathbf{z}' = A\mathbf{z}.
\]

The equilibrium of the linear system in $(u,v)$-space is obtained by setting $\mathbf{z}'=0$, that is,
\[
A\mathbf{z} = 0.
\]
Since $A$ is invertible (as will be clear from the eigenvalue computation), the only solution is $\mathbf{z} = \mathbf{0}$, so
\[
(u,v) = (0,0)
\]
is the unique equilibrium of the linear system. In terms of the original populations, $(u,v)=(0,0)$ corresponds exactly to $(X,Y)=(X^\ast,Y^\ast)$, the equilibrium of the nonlinear model.

\medskip

\noindent\textbf{(2) Eigenvalues, eigenvectors, and classification.}
We next compute the eigenvalues of $A$. The characteristic polynomial is
\[
\det(A - \lambda I)
=
\begin{vmatrix}
-1-\lambda & -2\\[0.2em]
2 & -1-\lambda
\end{vmatrix}
= (-1-\lambda)^2 + 4.
\]
Setting this equal to zero gives
\[
(-1-\lambda)^2 + 4 = 0
\quad\Longleftrightarrow\quad
(\lambda+1)^2 = -4
\quad\Longleftrightarrow\quad
\lambda + 1 = \pm 2i.
\]
Hence the eigenvalues are
\[
\lambda_{1,2} = -1 \pm 2i.
\]

To find an eigenvector for, say, $\lambda_1 = -1 + 2i$, we solve $(A - \lambda_1 I)\mathbf{v}_1 = 0$. We have
\[
A - \lambda_1 I =
\begin{pmatrix}
-1 - (-1+2i) & -2\\[0.2em]
2 & -1 - (-1+2i)
\end{pmatrix}
=
\begin{pmatrix}
-2i & -2\\[0.2em]
2 & -2i
\end{pmatrix}.
\]
Let $\mathbf{v}_1 = \begin{pmatrix}x\\[0.2em]y\end{pmatrix}$. One row gives
\[
-2i\,x - 2y = 0
\quad\Longleftrightarrow\quad
i\,x + y = 0
\quad\Longleftrightarrow\quad
y = -i x.
\]
We may choose $x=1$, so one eigenvector is
\[
\mathbf{v}_1 = \begin{pmatrix}1\\[0.2em]-i\end{pmatrix}
= \mathbf{p} + i\mathbf{q}
\quad\text{with}\quad
\mathbf{p} = \begin{pmatrix}1\\[0.2em]0\end{pmatrix},
\quad
\mathbf{q} = \begin{pmatrix}0\\[0.2em]-1\end{pmatrix}.
\]
An eigenvector for $\lambda_2 = -1 - 2i$ is the complex conjugate $\mathbf{v}_2 = \overline{\mathbf{v}}_1 = \begin{pmatrix}1\\[0.2em]i\end{pmatrix}$.

The eigenvalues have negative real part $-1$ and nonzero imaginary part $\pm 2i$. Therefore the equilibrium at the origin in $(u,v)$-space is a \emph{stable spiral} (also called a spiral sink). Solutions spiral into the origin as $t\to\infty$, with oscillations whose angular frequency is $2$.

\medskip

\noindent\textbf{(3) Real general solution and specific initial condition.}

\emph{Complex form.}
For a $2\times 2$ system with distinct complex conjugate eigenvalues, a complex fundamental solution has the form
\[
\mathbf{z}(t)
= c_1 e^{\lambda_1 t}\mathbf{v}_1
+ c_2 e^{\lambda_2 t}\mathbf{v}_2,
\]
where $c_1,c_2\in\mathbb{C}$. However, for a real system we prefer a real basis of solutions.

\emph{Conversion to real solutions.}
We use the standard construction. Write
\[
\lambda_1 = a+bi = -1 + 2i,
\qquad
a=-1,\quad b=2,
\]
and
\[
\mathbf{v}_1 = \mathbf{p} + i\mathbf{q},
\qquad
\mathbf{p} = \begin{pmatrix}1\\[0.2em]0\end{pmatrix},
\quad
\mathbf{q} = \begin{pmatrix}0\\[0.2em]-1\end{pmatrix}.
\]
Then
\[
e^{\lambda_1 t}\mathbf{v}_1
= e^{(a+bi)t}(\mathbf{p} + i\mathbf{q})
= e^{at}\big[(\mathbf{p}\cos bt - \mathbf{q}\sin bt)
+ i(\mathbf{p}\sin bt + \mathbf{q}\cos bt)\big].
\]
The real and imaginary parts are real solutions. Thus,
\[
\mathbf{z}_1(t)
= e^{at}\big(\mathbf{p}\cos bt - \mathbf{q}\sin bt\big),
\qquad
\mathbf{z}_2(t)
= e^{at}\big(\mathbf{p}\sin bt + \mathbf{q}\cos bt\big)
\]
form a real fundamental set of solutions.

We compute $\mathbf{z}_1(t)$ and $\mathbf{z}_2(t)$ explicitly. Since $a=-1$ and $b=2$,
\[
\mathbf{p}\cos(2t) - \mathbf{q}\sin(2t)
=
\begin{pmatrix}1\\[0.2em]0\end{pmatrix}\cos(2t)
-
\begin{pmatrix}0\\[0.2em]-1\end{pmatrix}\sin(2t)
=
\begin{pmatrix}\cos(2t)\\[0.2em]\sin(2t)\end{pmatrix},
\]
and
\[
\mathbf{p}\sin(2t) + \mathbf{q}\cos(2t)
=
\begin{pmatrix}1\\[0.2em]0\end{pmatrix}\sin(2t)
+
\begin{pmatrix}0\\[0.2em]-1\end{pmatrix}\cos(2t)
=
\begin{pmatrix}\sin(2t)\\[0.2em]-\cos(2t)\end{pmatrix}.
\]
Therefore,
\[
\mathbf{z}_1(t)
= e^{-t}\begin{pmatrix}\cos(2t)\\[0.2em]\sin(2t)\end{pmatrix},
\qquad
\mathbf{z}_2(t)
= e^{-t}\begin{pmatrix}\sin(2t)\\[0.2em]-\cos(2t)\end{pmatrix}
\]
are two real independent solutions. The real general solution is a linear combination:
\[
\mathbf{z}(t)
= C_1 \mathbf{z}_1(t) + C_2 \mathbf{z}_2(t)
= e^{-t}\left[
C_1 \begin{pmatrix}\cos(2t)\\[0.2em]\sin(2t)\end{pmatrix}
+
C_2 \begin{pmatrix}\sin(2t)\\[0.2em]-\cos(2t)\end{pmatrix}
\right],
\]
with real constants $C_1,C_2$.

Equivalently, in component form,
\begin{align*}
u(t) &= e^{-t}\big(C_1\cos(2t) + C_2\sin(2t)\big),\\
v(t) &= e^{-t}\big(C_1\sin(2t) - C_2\cos(2t)\big).
\end{align*}

\emph{Imposing the initial condition.}
We now enforce $u(0)=1$, $v(0)=0$. At $t=0$ we have $\cos(0)=1$ and $\sin(0)=0$, so
\[
u(0) = e^{0}(C_1\cdot 1 + C_2\cdot 0) = C_1 = 1,
\]
and
\[
v(0) = e^{0}(C_1\cdot 0 - C_2\cdot 1) = -C_2 = 0
\quad\Longrightarrow\quad C_2 = 0.
\]
Thus $C_1=1$, $C_2=0$, and the specific solution is
\[
\mathbf{z}(t) = \begin{pmatrix}u(t)\\[0.2em]v(t)\end{pmatrix}
= e^{-t}
\begin{pmatrix}\cos(2t)\\[0.2em]\sin(2t)\end{pmatrix}.
\]
In components,
\[
u(t) = e^{-t}\cos(2t),
\qquad
v(t) = e^{-t}\sin(2t).
\]

\medskip

\noindent\textbf{(4) Long-time behavior, decay rate, and oscillations.}
We now describe the qualitative behavior as $t\to\infty$.

First, both $u(t)$ and $v(t)$ are multiplied by the common factor $e^{-t}$, which tends to zero as $t\to\infty$. Therefore,
\[
\lim_{t\to\infty}u(t) = 0,
\qquad
\lim_{t\to\infty}v(t) = 0,
\]
so the solution tends to the equilibrium $(u,v)=(0,0)$ in the linear system, and hence returns to $(X^\ast,Y^\ast)$ in the original variables. The convergence is exponentially fast with rate $1$, because of the factor $e^{-t}$.

Second, the trigonometric factors $\cos(2t)$ and $\sin(2t)$ produce oscillations. The angular frequency is $2$ (radians per unit time), so the period $T$ is
\[
T = \frac{2\pi}{\text{angular frequency}} = \frac{2\pi}{2} = \pi.
\]
Thus the populations cycle around the equilibrium with period $\pi$, while each oscillation is damped by the factor $e^{-t}$.

To see the spiral structure in the phase plane, we can look at
\[
u(t)^2 + v(t)^2
= e^{-2t}(\cos^2(2t) + \sin^2(2t))
= e^{-2t}.
\]
This shows that the distance from the origin in the $(u,v)$-plane decays like $e^{-t}$; the trajectory moves along a spiral of shrinking radius, making one full rotation in angle per time interval $\pi$.

Hence the equilibrium is a spiral sink: trajectories spiral inward with exponentially decaying radius and oscillatory behavior determined by the imaginary part of the eigenvalues.

\medskip

\noindent\textbf{(5) The family $A_k$ and the trace--determinant viewpoint.}
Finally, consider
\[
A_k = \begin{pmatrix} -1 & -k \\[0.2em] 2 & -1 \end{pmatrix},\qquad k>0.
\]
The characteristic polynomial of a $2\times 2$ matrix is
\[
\lambda^2 - (\operatorname{tr}A_k)\lambda + \det(A_k)=0.
\]
We compute
\[
\operatorname{tr}(A_k) = -1 + (-1) = -2,
\]
and
\[
\det(A_k) = (-1)(-1) - (-k)(2) = 1 + 2k.
\]
Thus the eigenvalues satisfy
\[
\lambda^2 + 2\lambda + (1 + 2k) = 0.
\]
The discriminant of this quadratic is
\[
\Delta_k = (\operatorname{tr}A_k)^2 - 4\det(A_k)
= (-2)^2 - 4(1+2k)
= 4 - 4 - 8k
= -8k.
\]

For all $k>0$, we have $\Delta_k < 0$, so the eigenvalues are complex conjugates. Their real part is
\[
\Re(\lambda) = \frac{\operatorname{tr}(A_k)}{2} = \frac{-2}{2} = -1 < 0
\]
for all $k>0$. Therefore, for every positive $k$, the equilibrium is a \emph{stable spiral}, and the solutions exhibit \emph{decaying oscillations}.

There is no value of $k>0$ for which the eigenvalues are real (since the discriminant is strictly negative for $k>0$), so in this parameter range there is no non-oscillatory pure node. The imaginary part of the eigenvalues depends on $k$; in fact
\[
\lambda_{1,2} = -1 \pm i\sqrt{8k}\!/2 = -1 \pm i\sqrt{2k},
\]
so the angular frequency of oscillation is $\sqrt{2k}$, and the period is $2\pi/\sqrt{2k}$, which decreases as $k$ increases. The decay rate, determined by the real part $-1$, is independent of $k$ in this family.

\medskip

\noindent\textbf{Conceptual summary.}
This example illustrates how direct methods for solving linear ODEs—specifically, writing the system in matrix form, computing eigenvalues and eigenvectors, and forming the matrix exponential—provide both explicit formulas for solutions and a clear qualitative picture of the dynamics. In the $2\times 2$ case, the eigenvalues immediately reveal stability (via the real part) and the presence of oscillations (via the imaginary part), and the phase-plane behavior (node, saddle, spiral) follows directly. Near an equilibrium of a nonlinear population model, this linear analysis already captures the essential features of small perturbations: whether they decay or grow, and whether they do so monotonically or in an oscillatory fashion.
\end{solution}

% ===== Example 5: Steady States from the Heat Equation as a Boundary-Value ODE (inquiry-based) =====
\begin{problem}[Steady States from the Heat Equation as a Boundary-Value ODE]
A long, thin, homogeneous rod of length $L$ is insulated along its sides, so heat can flow only along the $x$–direction. The temperature $u(x,t)$ along the rod evolves according to the one-dimensional heat equation. At steady state, one observes that the temperature no longer changes in time, and the time-dependent partial differential equation reduces to a spatial ordinary differential equation with boundary conditions at the ends of the rod. This example shows how a steady-state profile arises as the solution of a second-order linear boundary-value problem, and how the values at the boundary determine the entire interior profile.

Assume that the rod occupies the interval $0 < x < L$, that there is no internal heat source, and that the long sides are perfectly insulated. Let $u(x,t)$ be the temperature at position $x$ and time $t$, and let $k>0$ be the thermal diffusivity.

\medskip

(a) Write down the standard one-dimensional heat equation (with no internal heat sources) that $u(x,t)$ satisfies under these assumptions, together with Dirichlet boundary conditions
\[
u(0,t) = T_0, \qquad u(L,t) = T_L
\]
for fixed constants $T_0$ and $T_L$. Explain in words what it means for the rod to reach a \emph{steady state}, and translate this into a mathematical condition on $u(x,t)$.

\medskip

(b) Use your steady-state condition from part (a) to show that any steady-state temperature profile $u_{\mathrm{ss}}(x)$ must satisfy a second-order ordinary differential equation on $0<x<L$ together with boundary conditions at $x=0$ and $x=L$. Write this boundary-value problem explicitly.  

Hint: At steady state, $u_{\mathrm{ss}}$ no longer depends on $t$, so its time derivative should vanish.

\medskip

(c) Now solve the ordinary differential equation you obtained in part (b).  

(i) First solve the homogeneous equation
\[
u''(x) = 0
\]
on $(0,L)$, and find the general solution in terms of two constants $C_1$ and $C_2$.  

(ii) Then impose the boundary conditions at $x=0$ and $x=L$ to determine $C_1$ and $C_2$ in terms of $T_0$, $T_L$, and $L$.  

Write your final formula for $u_{\mathrm{ss}}(x)$ in a simple and transparent form (for instance, in a way that makes it clear how $u_{\mathrm{ss}}(x)$ depends on $T_0$ and $T_L$).  

% Hint: Integrate $u''(x)=0$ twice, or recognize that a function with zero second derivative must be linear.

\medskip

(d) In this part, you will explore uniqueness and interpretation.  

(i) Suppose $v(x)$ is \emph{any} function satisfying
\[
v''(x) = 0, \qquad v(0)=0,\qquad v(L)=0.
\]
Show that $v(x)\equiv 0$ for all $x \in [0,L]$.  

% Hint: Use your general solution of $v''(x)=0$ from part (c) and apply the two boundary conditions.

(ii) Explain how the result in (i) implies that there is at most one steady-state solution to the boundary-value problem you found in part (b). In other words, explain why the temperatures at the ends of the rod uniquely determine the entire steady-state temperature profile inside.

(iii) Give a physical interpretation of the shape of $u_{\mathrm{ss}}(x)$ that you found in part (c). For example, how does the graph of $u_{\mathrm{ss}}(x)$ look if $T_0 < T_L$ or if $T_0 > T_L$? Why does a linear profile make sense physically in the absence of internal heat sources?

\medskip

(e) Now consider two extensions of the model.

\medskip\noindent
\emph{(e1) Uniform internal heating.}  
Suppose that, in addition to conduction, the rod is heated uniformly along its length by a constant heat source of strength $Q>0$ (per unit length). Then the heat equation takes the form
\[
u_t(x,t) = k\,u_{xx}(x,t) + Q.
\]
(i) Derive the steady-state ordinary differential equation and boundary conditions that $u_{\mathrm{ss}}$ must satisfy.  

(ii) Solve this new boundary-value problem and find an explicit formula for $u_{\mathrm{ss}}(x)$.  

(iii) How does the shape of this new steady-state profile differ from the no-source case? Sketch or describe the qualitative shape when $T_0=T_L$ and $Q>0$.

% Hint: The steady-state ODE is now inhomogeneous. Integrate twice and then use the boundary conditions as before.

\medskip\noindent
\emph{(e2) An insulated end.}  
Instead of fixing the temperature at $x=0$, suppose the left end is perfectly insulated, while the right end is held at temperature $T_R$. An insulated end means that there is no heat flux through that end, which translates to the Neumann boundary condition
\[
u_x(0,t) = 0,
\]
while at the right end we keep $u(L,t) = T_R$.

(i) Write down the steady-state boundary-value problem for $u_{\mathrm{ss}}(x)$ in this new scenario.  

(ii) Solve this boundary-value problem explicitly.  

(iii) How does this new steady-state profile compare to the original case with two fixed end temperatures? What physical effect does the insulation at $x=0$ have on the temperature profile?

\end{problem}

% ===== Example 5: Steady States from the Heat Equation as a Boundary-Value ODE (full solution) =====
\begin{problem}[Steady States from the Heat Equation as a Boundary-Value ODE]
A thin homogeneous rod of length $L$ lies along $0<x<L$. Its sides are insulated, and there are no internal heat sources. The temperature $u(x,t)$ satisfies the one-dimensional heat equation
\[
u_t = k\,u_{xx}, \quad 0<x<L,\ t>0,
\]
with Dirichlet boundary conditions $u(0,t)=T_0$ and $u(L,t)=T_L$, where $T_0$ and $T_L$ are fixed constants.

\begin{enumerate}
\item[(a)] Define what it means for the rod to be in a steady state, and show that any steady-state temperature profile $u_{\mathrm{ss}}(x)$ must satisfy the boundary-value problem
\[
u_{\mathrm{ss}}''(x) = 0,\quad 0<x<L,\qquad
u_{\mathrm{ss}}(0)=T_0,\quad u_{\mathrm{ss}}(L)=T_L.
\]
\item[(b)] Solve this boundary-value problem explicitly and express $u_{\mathrm{ss}}(x)$ in terms of $T_0$, $T_L$, and $L$.
\item[(c)] Prove that this steady-state solution is unique: if $v''(x)=0$ for $0<x<L$ and $v(0)=v(L)=0$, then $v(x)\equiv 0$.
\item[(d)] Now suppose there is a uniform internal heat source of strength $Q>0$, so that the heat equation becomes
\[
u_t = k\,u_{xx} + Q.
\]
Derive the steady-state boundary-value problem and solve it explicitly under the same Dirichlet boundary conditions $u(0,t)=T_0$, $u(L,t)=T_L$.
\end{enumerate}
\end{problem}

\begin{solution}
We are asked to understand steady states of the one-dimensional heat equation and to see how these are described by a second-order linear ordinary differential equation with boundary conditions. This is a typical example of a \emph{boundary-value problem} for a linear ODE which can be solved directly by integration and by using the boundary data to determine the constants of integration.

\medskip

\textbf{(a) Steady-state reduction to an ODE.}  
The rod is in a \emph{steady state} when the temperature no longer changes in time. Mathematically, this means that for each fixed $x$, the function $t\mapsto u(x,t)$ is constant in $t$. If we denote the steady-state temperature by $u_{\mathrm{ss}}(x)$, this condition is expressed as
\[
\frac{\partial}{\partial t} u_{\mathrm{ss}}(x) = 0
\quad\text{for all }x\in(0,L).
\]
Substituting $u(x,t)=u_{\mathrm{ss}}(x)$, which depends only on $x$, into the heat equation
\[
u_t = k\,u_{xx}
\]
gives
\[
0 = k\,u_{\mathrm{ss}}''(x),
\]
since $u_t=0$ at steady state and $u_{xx}$ becomes the ordinary second derivative $u_{\mathrm{ss}}''(x)$ with respect to $x$. Because $k>0$, we can divide by $k$ and obtain the ordinary differential equation
\[
u_{\mathrm{ss}}''(x) = 0,\quad 0<x<L.
\]

The boundary conditions on $u$ carry over directly to $u_{\mathrm{ss}}$. Since $u(0,t)=T_0$ and $u(L,t)=T_L$ for all $t$, any time-independent solution must satisfy
\[
u_{\mathrm{ss}}(0)=T_0,\qquad u_{\mathrm{ss}}(L)=T_L.
\]
Thus the steady-state profile solves the boundary-value problem
\[
u_{\mathrm{ss}}''(x) = 0,\quad 0<x<L,\qquad
u_{\mathrm{ss}}(0)=T_0,\quad u_{\mathrm{ss}}(L)=T_L.
\]

\medskip

\textbf{(b) Solving the homogeneous boundary-value problem.}  
We now solve the ordinary differential equation
\[
u_{\mathrm{ss}}''(x) = 0.
\]
This is a linear ODE with constant coefficients. The characteristic equation is $r^2 = 0$, which has the double root $r=0$. Hence, the general solution of $u''=0$ is
\[
u_{\mathrm{ss}}(x) = C_1 x + C_2,
\]
where $C_1$ and $C_2$ are constants of integration.

We use the boundary conditions to determine $C_1$ and $C_2$. At $x=0$,
\[
u_{\mathrm{ss}}(0) = C_1\cdot 0 + C_2 = C_2 = T_0,
\]
so $C_2 = T_0$. At $x=L$,
\[
u_{\mathrm{ss}}(L) = C_1 L + C_2 = C_1 L + T_0 = T_L.
\]
Solving for $C_1$ gives
\[
C_1 = \frac{T_L - T_0}{L}.
\]
Therefore
\[
u_{\mathrm{ss}}(x) = \frac{T_L - T_0}{L}\,x + T_0.
\]
A more revealing way to write this is
\[
u_{\mathrm{ss}}(x)
= T_0 + \frac{T_L - T_0}{L}\,x,
\]
which shows that the steady-state temperature varies linearly from $T_0$ at $x=0$ to $T_L$ at $x=L$.

\medskip

\textbf{(c) Uniqueness of the steady state.}  
To establish uniqueness, we consider the associated homogeneous boundary-value problem
\[
v''(x) = 0,\quad 0<x<L,\qquad v(0)=0,\quad v(L)=0.
\]
We must show that $v(x)\equiv 0$ is the only solution.

We already solved the equation $v''(x)=0$ in part (b). Its general solution is
\[
v(x) = A x + B,
\]
for constants $A$ and $B$. Applying the boundary condition at $x = 0$ gives
\[
v(0) = B = 0,
\]
so $B = 0$. Applying the boundary condition at $x = L$ gives
\[
v(L) = A L + B = A L = 0.
\]
Since $L>0$, this implies $A = 0$. Hence $v(x) = 0$ for all $x$, as required.

Now suppose that $u_1(x)$ and $u_2(x)$ are two steady-state solutions of the original boundary-value problem
\[
u'' = 0,\quad 0<x<L,\qquad u(0)=T_0,\quad u(L)=T_L.
\]
Consider their difference $w(x) = u_1(x) - u_2(x)$. Then
\[
w''(x) = u_1''(x)-u_2''(x) = 0-0 = 0,
\]
and
\[
w(0) = u_1(0)-u_2(0) = T_0 - T_0 = 0,\qquad
w(L) = u_1(L)-u_2(L) = T_L - T_L = 0.
\]
So $w$ satisfies the homogeneous boundary-value problem of the previous paragraph and therefore must be identically zero. That is, $u_1(x) = u_2(x)$ for all $x$. Hence the steady-state solution $u_{\mathrm{ss}}(x)$ is unique.

Physically, this means that once the end temperatures $T_0$ and $T_L$ are fixed, there is only one possible steady temperature distribution inside the rod.

\medskip

\textbf{(d) Steady state with a uniform source.}  
We now consider the heat equation with a uniform internal heat source of strength $Q>0$:
\[
u_t = k\,u_{xx} + Q,\quad 0<x<L.
\]
As in part (a), a steady-state profile $u_{\mathrm{ss}}(x)$ is time-independent, so $u_t=0$, and we obtain
\[
0 = k\,u_{\mathrm{ss}}''(x) + Q.
\]
Dividing through by $k>0$ gives the inhomogeneous ordinary differential equation
\[
u_{\mathrm{ss}}''(x) = -\frac{Q}{k},\quad 0<x<L.
\]
The boundary conditions remain Dirichlet:
\[
u_{\mathrm{ss}}(0) = T_0,\qquad u_{\mathrm{ss}}(L) = T_L.
\]
Thus the steady-state problem is
\[
u_{\mathrm{ss}}''(x) = -\frac{Q}{k},\quad 0<x<L,\qquad
u_{\mathrm{ss}}(0)=T_0,\quad u_{\mathrm{ss}}(L)=T_L.
\]

We solve the ODE by direct integration. First integrate once:
\[
u_{\mathrm{ss}}'(x) = -\frac{Q}{k}\,x + C_1,
\]
for some constant $C_1$. Integrating again,
\[
u_{\mathrm{ss}}(x) = -\frac{Q}{2k}\,x^2 + C_1 x + C_2,
\]
for some constant $C_2$.

We determine $C_1$ and $C_2$ from the boundary conditions. At $x=0$,
\[
u_{\mathrm{ss}}(0) = -\frac{Q}{2k}\cdot 0^2 + C_1\cdot 0 + C_2 = C_2 = T_0,
\]
so $C_2 = T_0$. At $x=L$,
\[
u_{\mathrm{ss}}(L) = -\frac{Q}{2k}L^2 + C_1 L + T_0 = T_L.
\]
Solving for $C_1$,
\[
C_1 L = T_L - T_0 + \frac{Q}{2k}L^2,
\qquad\text{so}\qquad
C_1 = \frac{T_L - T_0}{L} + \frac{Q}{2k}L.
\]
Substituting back, we find
\[
u_{\mathrm{ss}}(x)
= -\frac{Q}{2k}x^2
+ \left(\frac{T_L - T_0}{L} + \frac{Q}{2k}L\right)x
+ T_0.
\]
It is often helpful to separate the effects of the boundary temperatures and the internal source. To do that, we can rewrite the expression as
\[
u_{\mathrm{ss}}(x)
= T_0 + \frac{T_L - T_0}{L}x + \frac{Q}{2k}\big(Lx - x^2\big).
\]
Indeed, expanding the right-hand side gives
\[
T_0 + \frac{T_L - T_0}{L}x + \frac{Q}{2k}Lx - \frac{Q}{2k}x^2,
\]
which matches the previous expression.

The first two terms,
\[
T_0 + \frac{T_L - T_0}{L}x,
\]
are precisely the no-source steady-state profile from part (b). The last term,
\[
\frac{Q}{2k}(Lx - x^2),
\]
is the additional contribution caused by the uniform heating. Notice that $u_{\mathrm{ss}}''(x) = -Q/k < 0$, so the profile is \emph{concave down}. When $T_0 = T_L$, the temperature is highest in the interior of the rod and lower at the ends, which matches the intuition that internal heating adds extra heat in the middle that must escape through the ends.

\medskip

\textbf{Connection to direct methods for linear ODEs.}  
This example illustrates several central ideas from the section on direct methods for solving linear ordinary differential equations:

\begin{itemize}
\item A physical partial differential equation (the heat equation) reduces at steady state to a linear second-order ODE in space, together with boundary conditions. This is a typical way that boundary-value problems for ODEs arise.
\item The resulting homogeneous and inhomogeneous equations ($u''=0$ and $u''=-Q/k$) are solved by direct integration and by applying boundary conditions to determine integration constants. This is the basic direct method for constant-coefficient second-order equations.
\item Uniqueness of the boundary-value problem is established by studying the associated homogeneous problem and showing that only the trivial solution satisfies the homogeneous boundary conditions. This use of the homogeneous equation is a recurring theme in linear theory.
\item The explicit formulas show how boundary data determine the interior state, and how additional forcing (the internal source) adds a predictable, explicitly computable correction to the homogeneous steady-state solution.
\end{itemize}

Thus, this boundary-value problem serves both as a concrete physical model and as a clear demonstration of direct solution techniques for linear second-order ODEs with boundary conditions.

\end{solution}

\section{Linear Dynamics via the Green Function}
% --- Narrative plan (auto-generated) ---
% This section introduces the Green function as a tool for understanding linear ordinary differential equations, especially those that model forced or driven systems. Instead of solving each new inhomogeneous equation from scratch, we learn to describe a system by its response to an idealized impulse and then build the response to general forcing by superposition and convolution. In doing so, we uncover a unifying viewpoint on linear dynamics: the Green function encodes how information propagates through the system over time or space.
%
% The Green function perspective matters across applied mathematics because it generalizes naturally from simple ODE models to partial differential equations, where it appears as a fundamental solution or kernel. It connects directly with the Laplace transform, Fourier methods, and complex analysis, all of which can be used to construct or analyze Green functions. Throughout this section we will move back and forth between concrete physical models (such as mechanical oscillators and circuits) and abstract linear operators, preparing for later chapters on boundary value problems, Fourier series, and Green functions for PDEs.

% ===== Example 1: Impulse Response of a Damped Harmonic Oscillator (inquiry-based) =====
\begin{problem}[Impulse Response of a Damped Harmonic Oscillator]
A mass–spring–dashpot system is a standard model for a damped harmonic oscillator. A mass $m$ is attached to a spring of stiffness $k$ and a dashpot (damper) with damping coefficient $c$. Its displacement $x(t)$ from equilibrium obeys a linear second-order ODE. In this problem, you will discover how a very short “impulse” force, idealized mathematically by a Dirac delta function, gives rise to the Green function of the system. Once this impulse response is known, any small external force can be built up as a superposition of impulses in time, leading to a convolution formula for the solution.

We assume throughout that the system is \emph{underdamped}, in the sense that $c^2 < 4mk$, so that the homogeneous solutions are decaying oscillations.

Consider the initial value problem
\[
m x''(t) + c x'(t) + k x(t) = f(t)
\]
with given initial conditions $x(0) = x_0$, $x'(0) = v_0$.

\medskip

(a) First, recall the unforced problem. Set $f(t) = 0$ and consider the homogeneous equation
\[
m x''(t) + c x'(t) + k x(t) = 0.
\]
Solve this homogeneous ODE under the underdamped assumption $c^2 < 4mk$.

\quad(i) Write down the characteristic equation and its roots. Show that the general solution has the form
\[
x_h(t) = e^{-\gamma t} \left( A \cos(\omega_d t) + B \sin(\omega_d t) \right),
\]
for suitable constants $\gamma$ and $\omega_d$ that you should identify in terms of $m$, $c$, and $k$.

\quad(ii) Interpret $\gamma$ and $\omega_d$ physically or qualitatively (for example, in terms of decay and oscillation).

% Hint: Introduce $\gamma = \dfrac{c}{2m}$ and $\omega_0 = \sqrt{\dfrac{k}{m}}$, and then compute $\omega_d$ from the characteristic roots.

\medskip

(b) Now we idealize a very short, sharp “kick” applied to the mass at some time $s > 0$ by writing the forcing as a Dirac delta distribution:
\[
m x''(t) + c x'(t) + k x(t) = \delta(t - s).
\]
We are interested in the response to this impulse when the system is initially at rest. Thus we impose
\[
x(t) = 0 \quad \text{for } t < s.
\]

\quad(i) Explain why it is natural to require $x(t)$ to be continuous at $t = s$. That is, argue that $x(s^-) = x(s^+)$.

\quad(ii) Integrate the differential equation from $t = s - \varepsilon$ to $t = s + \varepsilon$, and then let $\varepsilon \to 0^+$. Use this to derive a “jump condition” describing how $x'(t)$ behaves at $t = s$, that is, a formula for $x'(s^+) - x'(s^-)$.

% Hint: Only one term in the ODE produces a nonzero contribution as $\varepsilon \to 0$, because $\delta(t - s)$ has integral $1$ and the solution stays bounded.

\medskip

(c) We now define the Green function $G(t,s)$ for the damped oscillator as the solution of
\[
m \frac{\partial^2 G}{\partial t^2}(t,s) + c \frac{\partial G}{\partial t}(t,s) + k G(t,s) = \delta(t - s),
\]
which satisfies
\[
G(t,s) = 0 \quad \text{for } t < s,
\]
and is continuous at $t = s$.

\quad(i) Use the result from part (b) to determine the initial conditions for the function of $t$
\[
t \mapsto G(t,s) \quad \text{at } t = s^+,
\]
namely $G(s^+,s)$ and $\dfrac{\partial G}{\partial t}(s^+,s)$.

\quad(ii) For $t > s$, the right-hand side is zero, so $G(t,s)$ satisfies the homogeneous equation in $t$. Set $\tau = t - s$ and consider the function $g(\tau) = G(s + \tau,s)$ for $\tau > 0$. Use your homogeneous solution from part (a) and the initial conditions from part (c)(i) to solve for $g(\tau)$ explicitly.

\quad(iii) Show that your answer can be written in the compact form
\[
G(t,s) = \frac{1}{m \omega_d} e^{-\gamma (t - s)} \sin\bigl(\omega_d (t - s)\bigr) H(t - s),
\]
where $H$ is the Heaviside step function.

% Hint: You should find that $G(t,s)$ has the form $C e^{-\gamma (t-s)} \sin(\omega_d (t-s))$ for $t > s$, and you must choose the constant $C$ using your jump condition.

\medskip

(d) Now consider a general forcing function $f(t)$ (assume it is continuous and sufficiently nice) with homogeneous initial conditions
\[
x(0) = 0, \qquad x'(0) = 0.
\]
We claim that the solution can be expressed as a time-convolution of the Green function with the forcing:
\[
x(t) = \int_0^t G(t,s) f(s)\,ds.
\]
Explain, using the linearity of the ODE and the interpretation of $\delta(t-s)$ as an “idealized unit impulse at time $s$,” why this integral formula represents the solution. In your explanation, you may think in terms of approximating $f$ by a sum of short pulses and then passing to a limit.

% Hint: View $f$ as a superposition of many small impulses at different times and use superposition (linearity) of the system's response.

\medskip

(e) Extensions and “what if” questions.

\quad(i) How would your formula for $G(t,s)$ change if the system were \emph{critically damped}, that is, if $c^2 = 4mk$? Sketch (but do not fully compute) how you would solve the homogeneous problem and match the jump conditions to find the Green function in this case.

\quad(ii) Suppose the system does not start from rest, but instead has initial conditions $x(0) = x_0$, $x'(0) = v_0$. Sketch how the general solution can be written as a sum of a homogeneous solution satisfying the initial data and a convolution of $G(t,s)$ with $f(s)$. What extra terms appear, compared with the formula in part (d)?
\end{problem}

% ===== Example 1: Impulse Response of a Damped Harmonic Oscillator (full solution) =====
\begin{problem}[Impulse Response of a Damped Harmonic Oscillator]
Consider the damped harmonic oscillator
\[
m x''(t) + c x'(t) + k x(t) = f(t),
\]
with $m>0$, $c>0$, $k>0$, and assume the system is underdamped, that is, $c^2 < 4mk$.

(a) For the impulsively forced problem
\[
m x''(t) + c x'(t) + k x(t) = \delta(t - s),
\]
with $x(t) = 0$ for $t < s$, define the Green function $G(t,s)$ as the solution satisfying
\[
m \frac{\partial^2 G}{\partial t^2}(t,s) + c \frac{\partial G}{\partial t}(t,s) + k G(t,s) = \delta(t - s), 
\qquad G(t,s) = 0 \text{ for } t < s,
\]
and $G$ continuous at $t = s$. Derive an explicit formula for $G(t,s)$.

(b) Use the Green function to show that the solution to
\[
m x''(t) + c x'(t) + k x(t) = f(t), \qquad x(0) = 0,\quad x'(0) = 0,
\]
can be written as
\[
x(t) = \int_0^t G(t,s)\,f(s)\,ds.
\]
Briefly explain how this representation illustrates the Green function viewpoint on linear dynamics.
\end{problem}

\begin{solution}
We study the forced damped oscillator
\[
m x''(t) + c x'(t) + k x(t) = f(t)
\]
and its response to an idealized impulse. Throughout we assume the underdamped condition $c^2 < 4mk$, so that the homogeneous motion is a decaying oscillation.

\medskip

\textbf{Step 1: Homogeneous solution and notation.}

We begin by solving the homogeneous equation
\[
m x''(t) + c x'(t) + k x(t) = 0.
\]
The characteristic polynomial is
\[
m r^2 + c r + k = 0,
\]
which has roots
\[
r = \frac{-c \pm \sqrt{c^2 - 4mk}}{2m}.
\]
Under the assumption $c^2 < 4mk$ these roots are complex with negative real part. It is convenient to set
\[
\gamma = \frac{c}{2m}, 
\qquad 
\omega_0 = \sqrt{\frac{k}{m}},
\qquad
\omega_d = \sqrt{\omega_0^2 - \gamma^2} = \sqrt{\frac{k}{m} - \frac{c^2}{4m^2}},
\]
so that
\[
r = -\gamma \pm i \omega_d.
\]
Therefore the general homogeneous solution is
\[
x_h(t) = e^{-\gamma t} \bigl( A \cos(\omega_d t) + B \sin(\omega_d t) \bigr),
\]
for constants $A$ and $B$ determined by initial conditions. The factor $e^{-\gamma t}$ describes exponential decay due to damping, and the factor involving $\cos$ and $\sin$ describes oscillation at the damped frequency $\omega_d$.

\medskip

\textbf{Step 2: Impulse forcing and the jump condition.}

We now consider the impulsively forced problem
\[
m x''(t) + c x'(t) + k x(t) = \delta(t - s),
\]
with the requirement that the system is at rest before the impulse:
\[
x(t) = 0 \quad \text{for } t < s.
\]
Physically it is natural to assume that the displacement $x(t)$ does not change abruptly at the moment of the impulse; the mass cannot move a finite distance instantaneously. Thus $x$ is continuous at $t=s$, so
\[
x(s^-) = x(s^+).
\]
Since $x(t) = 0$ for $t<s$ we have $x(s^-) = 0$, and therefore
\[
x(s^+) = 0.
\]

The time derivative $x'(t)$, however, is allowed to change abruptly, because an impulse corresponds to a finite, instantaneous change in momentum. To quantify this, we integrate the differential equation across a small interval containing $s$. For $\varepsilon > 0$, integrate from $t = s - \varepsilon$ to $t = s + \varepsilon$:
\[
\int_{s-\varepsilon}^{s+\varepsilon} \left( m x''(t) + c x'(t) + k x(t) \right) \, dt
= \int_{s-\varepsilon}^{s+\varepsilon} \delta(t - s)\,dt.
\]
The right-hand side is equal to $1$, because the delta function integrates to $1$ over any interval containing $s$.

On the left-hand side, we evaluate each term. For the second derivative term,
\[
\int_{s-\varepsilon}^{s+\varepsilon} m x''(t)\, dt
= m \bigl( x'(s+\varepsilon) - x'(s-\varepsilon) \bigr).
\]
For the other terms, we note that $x$ and $x'$ remain bounded as $\varepsilon \to 0$, so
\[
\int_{s-\varepsilon}^{s+\varepsilon} c x'(t)\, dt 
\to 0, \qquad
\int_{s-\varepsilon}^{s+\varepsilon} k x(t)\, dt
\to 0
\quad \text{as } \varepsilon \to 0^+.
\]
Thus, taking the limit $\varepsilon \to 0^+$, we obtain the jump condition
\[
m\bigl(x'(s^+) - x'(s^-)\bigr) = 1,
\]
or equivalently
\[
x'(s^+) - x'(s^-) = \frac{1}{m}.
\]

Since the system is at rest for $t < s$, we have $x'(t) = 0$ and hence $x'(s^-) = 0$, so
\[
x'(s^+) = \frac{1}{m}.
\]
In summary, the impulsively forced solution satisfies
\[
x(s^+) = 0, \qquad x'(s^+) = \frac{1}{m}.
\]

\medskip

\textbf{Step 3: Definition and construction of the Green function.}

We define the Green function $G(t,s)$ to be the solution of
\[
m \frac{\partial^2 G}{\partial t^2}(t,s) + c \frac{\partial G}{\partial t}(t,s) + k G(t,s) = \delta(t - s),
\]
with the conditions
\[
G(t,s) = 0 \quad \text{for } t < s,
\]
and $G$ continuous at $t = s$. For each fixed $s$, $G(t,s)$ as a function of $t$ describes the displacement response to a unit impulse applied at time $t = s$.

From the previous step, we know that
\[
G(s^-,s) = G(s^+,s), \quad\text{and}\quad
\frac{\partial G}{\partial t}(s^+,s) - \frac{\partial G}{\partial t}(s^-,s) = \frac{1}{m}.
\]
Because $G(t,s) = 0$ for $t < s$, we have
\[
G(s^-,s) = 0, \qquad \frac{\partial G}{\partial t}(s^-,s) = 0.
\]
Thus the initial conditions at $t = s^+$ are
\[
G(s^+,s) = 0, \qquad \frac{\partial G}{\partial t}(s^+,s) = \frac{1}{m}.
\]

For $t > s$, the right-hand side of the defining equation for $G$ vanishes, so $G(t,s)$ satisfies the homogeneous equation in $t$:
\[
m \frac{\partial^2 G}{\partial t^2}(t,s) + c \frac{\partial G}{\partial t}(t,s) + k G(t,s) = 0, \quad t > s.
\]
It is convenient to shift time so that the impulse occurs at the new time origin. Let
\[
\tau = t - s, \qquad g(\tau) = G(s + \tau, s), \quad \tau \ge 0.
\]
Then $g$ satisfies the homogeneous equation
\[
m g''(\tau) + c g'(\tau) + k g(\tau) = 0, \quad \tau > 0,
\]
with initial conditions at $\tau = 0^+$ given by
\[
g(0^+) = G(s^+,s) = 0, \qquad g'(0^+) = \frac{\partial G}{\partial t}(s^+,s) = \frac{1}{m}.
\]

From Step 1, the general homogeneous solution has the form
\[
g(\tau) = e^{-\gamma \tau} \bigl( A \cos(\omega_d \tau) + B \sin(\omega_d \tau) \bigr).
\]
Applying the condition $g(0) = 0$ gives
\[
0 = g(0) = e^{0}(A \cos 0 + B \sin 0) = A,
\]
so $A = 0$ and
\[
g(\tau) = B e^{-\gamma \tau} \sin(\omega_d \tau).
\]
We determine $B$ from $g'(0) = 1/m$. Differentiating, we obtain
\[
g'(\tau) = B e^{-\gamma \tau} \bigl( \omega_d \cos(\omega_d \tau) - \gamma \sin(\omega_d \tau) \bigr),
\]
so
\[
g'(0) = B \omega_d = \frac{1}{m}.
\]
Therefore $B = 1/(m \omega_d)$, and hence
\[
g(\tau) = \frac{1}{m \omega_d} e^{-\gamma \tau} \sin(\omega_d \tau), \quad \tau > 0.
\]

Translating back to $t$ and $s$, this means that for $t > s$,
\[
G(t,s) = \frac{1}{m \omega_d} e^{-\gamma (t - s)} \sin\bigl(\omega_d (t - s)\bigr).
\]
For $t < s$ we have $G(t,s) = 0$ by definition. It is common to express this “causality” in terms of the Heaviside step function $H$:
\[
G(t,s) = \frac{1}{m \omega_d} e^{-\gamma (t - s)} \sin\bigl(\omega_d (t - s)\bigr) H(t - s).
\]
This is the impulse response, or Green function, of the underdamped harmonic oscillator.

\medskip

\textbf{Step 4: Solution representation by convolution.}

We now use $G$ to represent the solution of the forced problem
\[
m x''(t) + c x'(t) + k x(t) = f(t), \qquad x(0) = 0,\quad x'(0) = 0.
\]

The key idea comes from linearity and the interpretation of the delta function. For each fixed $s$, the Green function $G(t,s)$ solves
\[
m \frac{\partial^2 G}{\partial t^2}(t,s) + c \frac{\partial G}{\partial t}(t,s) + k G(t,s) = \delta(t - s),
\]
with zero “initial” data for $t < s$. Thus $G(t,s)$ is the response at time $t$ to an idealized unit impulse applied at time $s$.

If we apply instead a force $\alpha\,\delta(t-s)$ with strength $\alpha$, the response is simply $\alpha G(t,s)$ by linearity. More generally, if we imagine approximating a continuous forcing function $f(t)$ as a superposition of many small impulses in time,
\[
f(t) \approx \sum_j f(s_j)\,\Delta s_j\,\delta(t - s_j),
\]
then the response is approximately
\[
x(t) \approx \sum_j f(s_j) \Delta s_j\, G(t,s_j).
\]
In the limit as the time step $\Delta s_j$ tends to zero, this Riemann sum becomes the integral
\[
x(t) = \int_0^t G(t,s) f(s)\,ds.
\]
Here we have integrated up to $t$ because impulses applied at times $s>t$ cannot affect the solution at earlier time $t$; causality is reflected in the factor $H(t-s)$ built into $G$.

To verify the formula more directly, one can differentiate under the integral sign and check that $x(t)$ satisfies the ODE and the initial conditions. However, from the Green function viewpoint, the essential reasoning is that $G(t,s)$ captures the effect of a single impulse at time $s$, and by linear superposition over all times $s$ where a “small” impulse $f(s)\,ds$ acts, we obtain the full response.

Thus the solution with homogeneous initial data is
\[
x(t) = \int_0^t G(t,s) f(s)\,ds
= \int_0^t \frac{1}{m \omega_d} e^{-\gamma (t - s)} 
    \sin\bigl(\omega_d (t - s)\bigr)\, f(s)\, ds.
\]

\medskip

\textbf{Conceptual remark.}
This example illustrates the core idea of the section “Linear Dynamics via the Green Function.” For a linear time-invariant system, the delta impulse response $G(t,s)$ encapsulates the dynamics: how an instantaneous kick at time $s$ affects the displacement at later times $t$. Once $G$ is known, the response to an arbitrary forcing $f$ is obtained by a time-convolution of $G$ with $f$. The structure of the homogeneous solution (damped oscillation in this case) is imprinted directly onto the Green function and hence onto the behavior of solutions for general forcing.
\end{solution}

% ===== Example 2: Step Forcing in a First-Order Linear ODE (inquiry-based) =====
\begin{problem}[Step Forcing in a First-Order Linear ODE]
A large, well-mixed tank is being filled with fluid. At first, the inlet pipe is closed and only draining and mixing are happening. After some time $T>0$, a pump is switched on and a constant inflow begins. A simple model for the volume (or temperature, or concentration) $y(t)$ in the tank is a first-order linear ordinary differential equation with a forcing term that turns on at time $T$ like a step function.

In this problem, you will connect the standard integrating-factor solution formula for first-order linear equations with the Green function, and you will see how a step input can be understood as the accumulation of many small ``impulses.''

We fix a constant $a>0$ and consider the differential operator
\[
L[y](t) := y'(t) + a\,y(t).
\]

\medskip

(a) \textbf{Warm-up: constant forcing and steady state.}  
First ignore any switching and consider the initial-value problem
\[
y'(t) + a\,y(t) = 1, \qquad y(0)=0.
\]
\begin{enumerate}
  \item[(i)] Solve this equation using the integrating-factor method.
  \item[(ii)] Compute $\displaystyle \lim_{t\to\infty} y(t)$. Interpret this limit as a steady state of the system.
\end{enumerate}
% Hint: The integrating factor is $e^{at}$.  

\medskip

(b) \textbf{Step forcing: turning the source on at $t=T$.}  
Now suppose the source is switched on at time $T>0$, instead of at $t=0$. Let $H$ be the Heaviside step function,
\[
H(t-T) = \begin{cases}
0, & t<T,\\
1, & t\ge T.
\end{cases}
\]
Consider the initial-value problem
\[
y'(t) + a\,y(t) = H(t-T), \qquad y(0)=0.
\]

\begin{enumerate}
  \item[(i)] Solve the equation for $0 \le t < T$. What condition do you use at $t=0$?
  \item[(ii)] Now solve the equation for $t>T$, treating it as an equation with constant forcing $1$ but an unknown initial condition at $t=T^+$. What condition do you impose at $t=T$ to glue the two pieces together?
  \item[(iii)] Combine your work to obtain a single piecewise formula for $y(t)$ for all $t\ge 0$. Sketch $y(t)$ qualitatively and describe in words what happens at time $T$.
\end{enumerate}
Hint: The solution should be a decaying exponential before $T$, and after $T$ it should relax toward the same steady state you found in part (a), but starting from the value reached at $t=T$.

\medskip

(c) \textbf{The integrating-factor formula as an integral kernel.}  
For the general forced problem
\[
y'(t) + a\,y(t) = f(t), \qquad y(0) = y_0,
\]
the integrating-factor method gives a formula of the form
\[
y(t) = e^{-at} y_0 + \int_0^t K(t,s)\, f(s)\, ds,
\]
for some kernel $K(t,s)$ that depends on $a$.  

\begin{enumerate}
  \item[(i)] Derive this formula and identify $K(t,s)$ explicitly.
  \item[(ii)] Specialize this formula to the step forcing $f(t)=H(t-T)$ and $y_0=0$, and show that your expression reduces to the piecewise solution you found in part (b). (You may find it helpful to split the integral at $s=T$.)
\end{enumerate}
Hint: Look carefully at $e^{-at} \int_0^t e^{as}f(s)\,ds$ and rewrite it as an integral with integrand $e^{-a(t-s)} f(s)$.

\medskip

(d) \textbf{Introducing the Green function.}  
We now reinterpret the kernel $K(t,s)$ as a Green function.

\begin{enumerate}
  \item[(i)] For each fixed $s\ge 0$, consider the function $G(\,\cdot\,,s)$ of $t$ defined by
  \[
  G(t,s) :=
  \begin{cases}
  0, & t<s,\\
  e^{-a(t-s)}, & t\ge s.
  \end{cases}
  \]
  Show that $G$ satisfies the homogeneous equation $L[G(\,\cdot\,,s)](t)=0$ for all $t\neq s$, together with the initial condition $G(0,s)=0$.
  \item[(ii)] Show that $G$ has a unit jump in its derivative at $t=s$:
  \[
  \lim_{t\downarrow s} \frac{d}{dt}G(t,s) - \lim_{t\uparrow s} \frac{d}{dt}G(t,s) = 1.
  \]
  Explain briefly (in words) why this jump condition encodes the presence of a delta source at time $s$, that is, $L[G(\,\cdot\,,s)](t) = \delta(t-s)$ in the sense of distributions.
  \item[(iii)] Using your work in part (c), write the solution of
  \[
  L[y](t) = f(t), \qquad y(0)=0,
  \]
  in the compact Green-function form
  \[
  y(t) = \int_0^t G(t,s)\, f(s)\, ds.
  \]
  Identify $G(t,s)$ in your formula with the function defined in part (i).
\end{enumerate}
Hint: Compare the kernel $K(t,s)$ from part (c) with $G(t,s)$ above.

\medskip

(e) \textbf{Extensions and ``what if'' questions.}
\begin{enumerate}
  \item[(i)] Suppose now that the source turns on to a different constant level $A\neq 1$ at time $T$, so that $f(t) = A\, H(t-T)$. Without re-solving the differential equation from scratch, use your Green-function formula to write down $y(t)$ explicitly.
  \item[(ii)] Consider a piecewise constant forcing built from several step functions; for example,
  \[
  f(t) = H(t-T_1) - H(t-T_2), \qquad 0<T_1<T_2.
  \]
  Describe qualitatively (and, if you wish, quantitatively) how the solution behaves in time. How does viewing $f$ as a linear combination of steps, and $y$ as an integral against $G(t,s)$, help you understand this behavior?
  \item[(iii)] (Conceptual) The step function $H(t-T)$ can be thought of as the integral over time of many tiny impulses that start at $t=T$. Explain, using your Green-function representation, how this viewpoint is reflected in the formula
  \[
  y(t) = \int_0^t G(t,s)\, H(s-T)\, ds.
  \]
  In your explanation, highlight the role of $G(t,s)$ as the response to an impulse at time $s$.
\end{enumerate}
\end{problem}

% ===== Example 2: Step Forcing in a First-Order Linear ODE (full solution) =====
\begin{problem}[Step Forcing in a First-Order Linear ODE]
Let $a>0$ and $T>0$ be fixed, and let $H$ denote the Heaviside step function,
\[
H(t-T) =
\begin{cases}
0, & t<T,\\
1, & t\ge T.
\end{cases}
\]
Consider the initial-value problem
\[
y'(t) + a\,y(t) = H(t-T), \qquad y(0)=0.
\]
\begin{enumerate}
  \item[(a)] Solve this equation explicitly, giving a piecewise formula for $y(t)$.
  \item[(b)] For the general forced problem
  \[
  y'(t) + a\,y(t) = f(t), \qquad y(0)=y_0,
  \]
  derive the integrating-factor solution and show that it can be written in the Green-function form
  \[
  y(t) = e^{-at}y_0 + \int_0^t G(t,s)\, f(s)\,ds
  \quad\text{with}\quad
  G(t,s) = \begin{cases}
  0, & t<s,\\
  e^{-a(t-s)}, & t\ge s.
  \end{cases}
  \]
  \item[(c)] Specialize this Green-function formula to $f(t)=H(t-T)$ and $y_0=0$, and verify that it reproduces your explicit solution from part (a).
\end{enumerate}
Briefly explain how this example illustrates the interpretation of a step forcing as an accumulation of impulse responses and how it fits into the viewpoint of linear dynamics via the Green function.
\end{problem}

\begin{solution}
We proceed in stages, starting with the explicit solution of the step-forced equation and then connecting it to the general Green-function formula.

\medskip

\textbf{(a) Explicit solution for step forcing.}

We consider
\[
y'(t) + a\,y(t) = H(t-T), \qquad y(0)=0.
\]
Because $H(t-T)$ is piecewise constant, it is natural to solve the equation separately on the intervals $[0,T)$ and $(T,\infty)$ and then impose continuity of $y$ at $t=T$.

\emph{Region 1: $0\le t<T$.}  
Here $H(t-T)=0$, so $y$ satisfies the homogeneous equation
\[
y'(t) + a\,y(t) = 0, \qquad y(0)=0.
\]
The general solution of $y' + a y = 0$ is $y(t)=C e^{-at}$. Imposing $y(0)=0$ gives $C=0$, so
\[
y(t)=0 \quad \text{for } 0\le t<T.
\]

\emph{Region 2: $t>T$.}  
Here $H(t-T)=1$, so $y$ satisfies the inhomogeneous equation
\[
y'(t) + a\,y(t) = 1, \qquad t>T.
\]
We do not know $y(T^+)$ a priori, but we expect the solution to be continuous, so we will determine $y(T^+)$ from the left-hand limit.

We solve $y'+a y = 1$ for $t>T$ using an integrating factor. Multiplying by $e^{at}$, we obtain
\[
\frac{d}{dt}\bigl(e^{at} y(t)\bigr) = e^{at}.
\]
Integrating from $T$ to $t$ (with $t>T$) yields
\[
e^{at} y(t) - e^{aT} y(T^+) = \int_T^t e^{as}\,ds = \frac{1}{a}\left(e^{at}-e^{aT}\right).
\]
Solving for $y(t)$ gives
\[
y(t) = e^{-at} e^{aT} y(T^+) + \frac{1}{a}\Bigl(1 - e^{-a(t-T)}\Bigr).
\]

By continuity of $y$, we set $y(T^+) = y(T^-)$, and from the first region we have $y(T^-)=0$. Thus $y(T^+)=0$, and the first term in the expression above vanishes. Therefore, for $t\ge T$ we obtain
\[
y(t) = \frac{1}{a}\left(1 - e^{-a(t-T)}\right).
\]

Combining the two regions, the solution is
\[
y(t) =
\begin{cases}
0, & 0\le t<T,\\[4pt]
\dfrac{1}{a}\bigl(1 - e^{-a(t-T)}\bigr), & t\ge T.
\end{cases}
\]
This solution is continuous at $t=T$ (it equals $0$ there) and, for $t>T$, relaxes exponentially from $0$ toward the steady state $1/a$.

\medskip

\textbf{(b) Integrating-factor solution and Green function.}

Now consider the general forced equation
\[
y'(t) + a\,y(t) = f(t), \qquad y(0)=y_0.
\]
We apply the integrating factor $e^{at}$. Multiplying both sides by $e^{at}$ gives
\[
e^{at} y'(t) + a e^{at} y(t) = e^{at} f(t),
\]
or, recognizing the left-hand side as a product derivative,
\[
\frac{d}{dt}\bigl(e^{at} y(t)\bigr) = e^{at} f(t).
\]
Integrating from $0$ to $t$ yields
\[
e^{at} y(t) - e^{a\cdot 0} y(0) = \int_0^t e^{as} f(s)\,ds.
\]
Using $y(0)=y_0$, we obtain
\[
e^{at} y(t) = y_0 + \int_0^t e^{as} f(s)\,ds.
\]
Solving for $y(t)$,
\[
y(t) = e^{-at} y_0 + e^{-at}\int_0^t e^{as} f(s)\,ds.
\]
We can rewrite the integral term as
\[
e^{-at}\int_0^t e^{as} f(s)\,ds = \int_0^t e^{-a(t-s)} f(s)\,ds,
\]
since $e^{-at}e^{as} = e^{-a(t-s)}$. Thus the solution takes the kernel form
\[
y(t) = e^{-at} y_0 + \int_0^t e^{-a(t-s)} f(s)\,ds.
\]

To express this in Green-function language, we define
\[
G(t,s) =
\begin{cases}
0, & t<s,\\
e^{-a(t-s)}, & t\ge s.
\end{cases}
\]
Then, for $t\ge 0$,
\[
\int_0^t e^{-a(t-s)} f(s)\,ds = \int_0^t G(t,s) f(s)\,ds,
\]
since on the interval of integration $0\le s\le t$ we have $t\ge s$ and hence $G(t,s)=e^{-a(t-s)}$. Outside this interval we set $G(t,s)=0$ to emphasize causality: the response at time $t$ does not depend on future forcing.

Therefore, the solution can be written as
\[
y(t) = e^{-at} y_0 + \int_0^t G(t,s) f(s)\,ds,
\]
with
\[
G(t,s) = \begin{cases}
0, & t<s,\\
e^{-a(t-s)}, & t\ge s.
\end{cases}
\]
This function $G$ is the Green function for the operator $L[y]=y'+ay$ with the initial condition $y(0)=0$. For each fixed $s$, $G(\cdot,s)$ satisfies the homogeneous equation $G_t + a G = 0$ away from $t=s$, vanishes for $t<s$, and has a unit jump in its derivative at $t=s$, corresponding formally to $L[G(\cdot,s)](t) = \delta(t-s)$ in the sense of distributions.

\medskip

\textbf{(c) Specialization to step forcing.}

We now specialize the Green-function formula to the step forcing $f(t)=H(t-T)$ and the initial condition $y_0=0$.

From part (b), with $y_0=0$ we have
\[
y(t) = \int_0^t G(t,s) H(s-T)\,ds.
\]
Since $G(t,s)=e^{-a(t-s)}$ for $0\le s\le t$, this becomes
\[
y(t) = \int_0^t e^{-a(t-s)} H(s-T)\,ds.
\]

To evaluate this integral explicitly, we use the definition of $H(s-T)$. If $0\le t<T$, then for all $0\le s\le t$ we have $s<T$, so $H(s-T)=0$. Thus the integral vanishes and
\[
y(t) = 0, \qquad 0\le t<T,
\]
which matches the earlier computation.

If $t\ge T$, then $H(s-T)=0$ for $s<T$ and $H(s-T)=1$ for $s\ge T$. Therefore we can write
\[
y(t) = \int_0^t e^{-a(t-s)} H(s-T)\,ds
= \int_T^t e^{-a(t-s)}\,ds,
\]
where the lower limit becomes $T$ because the integrand is zero for $s<T$. Evaluating the integral,
\[
\int_T^t e^{-a(t-s)}\,ds
= \int_T^t e^{-a(t-s)}\,ds
= \left[ -\frac{1}{a} e^{-a(t-s)}\right]_{s=T}^{s=t}
= \frac{1}{a}\left(1 - e^{-a(t-T)}\right).
\]
Thus
\[
y(t) = \frac{1}{a}\left(1 - e^{-a(t-T)}\right), \qquad t\ge T.
\]

Combining the two cases, we recover exactly the piecewise solution found in part (a):
\[
y(t) =
\begin{cases}
0, & 0\le t<T,\\[4pt]
\dfrac{1}{a}\bigl(1 - e^{-a(t-T)}\bigr), & t\ge T.
\end{cases}
\]

\medskip

\textbf{Interpretation and connection to linear dynamics via the Green function.}

This example illustrates several key ideas from the viewpoint of linear dynamics and Green functions:

\begin{itemize}
  \item The kernel $G(t,s)=e^{-a(t-s)}$ (for $t\ge s$) is the system's response at time $t$ to a unit impulse applied at time $s$. It encodes the natural decay rate $a$ of the homogeneous dynamics.
  \item The solution for arbitrary forcing $f$ is a time-convolution of $f$ with this kernel: $y(t)=e^{-at}y_0+\int_0^t G(t,s)f(s)\,ds$. In words, the total response at time $t$ is the superposition of the decayed effects of all past inputs.
  \item A step forcing $H(t-T)$ can be understood as turning on a constant input at time $T$. In Green-function language, the step can be viewed as the accumulation of infinitesimal impulses beginning at $T$. The integral representation
  \[
  y(t) = \int_0^t G(t,s) H(s-T)\,ds = \int_T^t G(t,s)\,ds
  \]
  makes this explicit: the system's state at time $t$ is the integral over all impulse responses from $s\in[T,t]$.
\end{itemize}

Thus the familiar integrating-factor method for first-order linear equations is naturally interpreted as constructing a Green function and then expressing the solution as an integral against this kernel. This rephrasing is the simplest example of the general Green-function approach to linear dynamics.
\end{solution}

% ===== Example 3: Driven RLC Circuit and Convolution with the Green Function (inquiry-based) =====
\begin{problem}[Driven RLC Circuit and Convolution with the Green Function]
A series RLC circuit with resistance \(R>0\), inductance \(L>0\), and capacitance \(C>0\) is driven by a time-dependent voltage source \(V(t)\). If \(q(t)\) denotes the charge on the capacitor, Kirchhoff's voltage law leads to a linear second-order differential equation relating \(q\) and \(V\). In this problem you will build, step by step, the Green function for this circuit, and then use it to represent the response to more complicated drivings by convolution. Along the way you will see how the transient and steady-state behaviors are both encoded in the same Green function.

Throughout, assume the \emph{underdamped} case
\[
R^2 < \frac{4L}{C},
\]
so that the natural response of the circuit oscillates with exponentially decaying amplitude.

\smallskip

(a) \textbf{Setting up the model.}  
Use Kirchhoff's voltage law to show that the charge \(q(t)\) on the capacitor in a driven series RLC circuit satisfies
\[
L\,q''(t) + R\,q'(t) + \frac{1}{C}\,q(t) \;=\; V(t).
\]
Explain briefly how each term in this equation arises from the voltage drops across the inductor, resistor, and capacitor.

\smallskip

(b) \textbf{Homogeneous dynamics and natural frequencies.}  
Consider the associated homogeneous equation
\[
L\,q''(t) + R\,q'(t) + \frac{1}{C}\,q(t) = 0.
\]
\begin{enumerate}
\item[(i)] Write down the characteristic equation and its roots. Show that, under the underdamped assumption \(R^2 < 4L/C\), the roots are
\[
r_{1,2} = -\alpha \pm i\omega_d,
\]
for suitable real constants \(\alpha>0\) and \(\omega_d>0\), which you should express in terms of \(R\), \(L\), and \(C\).

\item[(ii)] Write the general solution of the homogeneous equation in the real form
\[
q_{\mathrm{hom}}(t) = e^{-\alpha t}\,\big(A\cos(\omega_d t) + B\sin(\omega_d t)\big),
\]
and explain briefly the qualitative behavior of this solution.
\end{enumerate}
% Hint: Recall that a pair of complex conjugate roots \(-\alpha \pm i\omega_d\) leads to a decaying oscillation.

\smallskip

(c) \textbf{Defining and determining the Green function.}  
We now introduce the causal Green function \(G(t)\) for the operator
\[
L\frac{d^2}{dt^2} + R\frac{d}{dt} + \frac{1}{C}.
\]
By definition, \(G\) satisfies
\[
L\,G''(t) + R\,G'(t) + \frac{1}{C}\,G(t) = \delta(t),
\]
and \(G(t) = 0\) for all \(t<0\).

\begin{enumerate}
\item[(i)] Integrate the Green function equation across a small interval \((-\varepsilon,\varepsilon)\) around \(t=0\), and let \(\varepsilon \to 0^+\), to derive the jump condition that \(G'\) must satisfy at \(t=0\). What are \(G(0^+)\) and \(G'(0^+)\)?

\emph{Hint:} Use \(\displaystyle \int_{-\varepsilon}^{\varepsilon}\delta(t)\,dt = 1\) and argue that \(\int_{-\varepsilon}^{\varepsilon} G(t)\,dt\) and \(\int_{-\varepsilon}^{\varepsilon} G'(t)\,dt\) vanish as \(\varepsilon\to 0\).

\item[(ii)] For \(t>0\), \(G\) satisfies the homogeneous equation of part (b). Using your general homogeneous solution and the initial conditions you just found at \(t=0^+\), solve for an explicit formula for \(G(t)\) for \(t>0\). You may introduce the Heaviside step function \(H(t)\) if you wish to write a single formula valid for all \(t\).
% Hint: Set \(G(t) = e^{-\alpha t}\left(a\cos(\omega_d t) + b\sin(\omega_d t)\right)\) for \(t>0\) and use the initial conditions at \(0^+\) to find \(a\) and \(b\).
\end{enumerate}

\smallskip

(d) \textbf{Convolution representation of the driven response.}  
Now consider the driven equation with zero initial charge and current:
\[
L\,q''(t) + R\,q'(t) + \frac{1}{C}\,q(t) = V(t), \qquad q(0)=0,\quad q'(0)=0.
\]
\begin{enumerate}
\item[(i)] Using the Green function you have just found, propose an integral formula of the form
\[
q(t) = \int_0^t G(t-s)\,V(s)\,ds
\]
for the solution with zero initial data, and explain in words why this formula should be plausible physically.

\item[(ii)] Verify directly, by differentiating under the integral sign and using the defining properties of \(G\), that your formula for \(q(t)\) indeed satisfies the differential equation and the initial conditions.
% Hint: Differentiate \(q(t) = \int_0^t G(t-s)V(s)\,ds\) once and twice; be careful with the upper limit \(t\). Use the equation satisfied by \(G\) and recall that \(\int_0^t \delta(t-s) V(s)\,ds = V(t)\).
\end{enumerate}

\smallskip

(e) \textbf{Two drivings: sinusoidal and pulsed.}  
We now explore how the same Green function encodes different kinds of behavior.

\begin{enumerate}
\item[(i)] Let the driving voltage be sinusoidal for \(t>0\):
\[
V(t) = V_0 \cos(\omega t)\,H(t),
\]
where \(V_0\) and \(\omega\) are constants. Use the convolution formula to write \(q(t)\) explicitly as a single integral in terms of \(G\), \(V_0\), and \(\omega\). Then, by evaluating or otherwise analyzing this integral, separate the solution into a transient part and a steady-state part oscillating at frequency \(\omega\). What happens to the transient part as \(t\to\infty\)?

\emph{Hint:} You may find it easier to recognize that any solution can be written as ``homogeneous part \(+\) particular part,'' and compare your convolution formula with the particular solution obtained by the method of undetermined coefficients.

\item[(ii)] Now take a \emph{voltage pulse}:
\[
V(t) = 
\begin{cases}
V_0, & 0 < t < T,\\[4pt]
0, & t\ge T,
\end{cases}
\]
with \(V(t)=0\) for \(t<0\). Use the convolution representation to express \(q(t)\) for all \(t\ge 0\) in terms of integrals of \(G\). You do not need to carry out the integrals explicitly, but you should clearly indicate the integration limits in the cases \(0<t<T\) and \(t\ge T\).

\item[(iii)] Qualitatively compare the behavior of the circuit for the sinusoidal driving in part (i) and the pulsed driving in part (ii). In each case, how are the transient and long-time behaviors visible from your expressions involving the same Green function \(G\)?
\end{enumerate}

\end{problem}

% ===== Example 3: Driven RLC Circuit and Convolution with the Green Function (full solution) =====
\begin{problem}[Driven RLC Circuit and Convolution with the Green Function]
Consider a series RLC circuit with resistance \(R>0\), inductance \(L>0\), and capacitance \(C>0\), driven by a voltage source \(V(t)\). Let \(q(t)\) be the charge on the capacitor. Assume the underdamped regime \(R^2 < 4L/C\).

\begin{enumerate}
\item[(a)] Derive the differential equation relating \(q(t)\) and \(V(t)\), and write down the homogeneous characteristic roots and corresponding homogeneous solution.
\item[(b)] Define the causal Green function \(G(t)\) for the operator
\[
L\frac{d^2}{dt^2} + R\frac{d}{dt} + \frac{1}{C},
\]
that is, the solution of
\[
L\,G''(t) + R\,G'(t) + \frac{1}{C}\,G(t) = \delta(t), \qquad G(t)=0 \text{ for } t<0.
\]
Derive the jump conditions at \(t=0\) and obtain an explicit formula for \(G(t)\).
\item[(c)] Show that the solution of
\[
L\,q''(t) + R\,q'(t) + \frac{1}{C}\,q(t) = V(t), \qquad q(0)=0,\quad q'(0)=0,
\]
is given by the convolution
\[
q(t) = \int_0^t G(t-s)\,V(s)\,ds.
\]
\item[(d)] Specialize to the sinusoidal driving \(V(t)=V_0\cos(\omega t)\,H(t)\) and write \(q(t)\) as a convolution integral. By evaluating this integral, or by comparing with a particular solution found by the method of undetermined coefficients, decompose \(q(t)\) into a transient part and a steady-state oscillation at frequency \(\omega\), and describe their long-time behavior.
\item[(e)] For the pulsed voltage
\[
V(t) = 
\begin{cases}
V_0, & 0<t<T,\\[4pt]
0, & t\ge T,
\end{cases}
\quad (V(t)=0 \text{ for } t<0),
\]
express \(q(t)\) for \(t\ge 0\) in terms of integrals of \(G\) with appropriate limits, and briefly interpret the transient behavior of the circuit in this case.

\end{enumerate}
\end{problem}

\begin{solution}
\textbf{(a) Governing equation and homogeneous dynamics.}  
In a series RLC circuit, the current is \(i(t) = q'(t)\). The voltage across the resistor is \(V_R = R i(t) = R q'(t)\), across the inductor is \(V_L = L\,di/dt = L q''(t)\), and across the capacitor is \(V_C = q(t)/C\). Kirchhoff's voltage law asserts that the sum of the voltage drops equals the applied source voltage \(V(t)\), so
\[
L q''(t) + R q'(t) + \frac{1}{C} q(t) = V(t).
\]

The associated homogeneous equation is
\[
L q''(t) + R q'(t) + \frac{1}{C} q(t) = 0.
\]
The characteristic polynomial is
\[
L r^2 + R r + \frac{1}{C} = 0,
\]
which has roots
\[
r_{1,2} = \frac{-R \pm \sqrt{R^2 - 4L/C}}{2L}.
\]
Under the underdamped assumption \(R^2<4L/C\), the discriminant is negative, so we define
\[
\alpha = \frac{R}{2L} > 0,\qquad
\omega_0 = \frac{1}{\sqrt{LC}},\qquad
\omega_d = \sqrt{\omega_0^2 - \alpha^2} = \frac{1}{2L}\sqrt{4L/C - R^2} > 0.
\]
Then
\[
r_{1,2} = -\alpha \pm i\omega_d.
\]

Thus the homogeneous solution can be written in real form as
\[
q_{\mathrm{hom}}(t) = e^{-\alpha t}\big(A\cos(\omega_d t) + B\sin(\omega_d t)\big),
\]
where \(A\) and \(B\) are arbitrary constants determined by initial data. This is an exponentially decaying oscillation: the factor \(e^{-\alpha t}\) produces damping, while \(\omega_d\) is the damped oscillation frequency.

\medskip

\textbf{(b) The causal Green function and its explicit form.}

The causal Green function \(G\) for the operator
\[
\mathcal{L} = L\frac{d^2}{dt^2} + R\frac{d}{dt} + \frac{1}{C}
\]
is defined by
\[
L G''(t) + R G'(t) + \frac{1}{C} G(t) = \delta(t),\qquad G(t) = 0 \text{ for } t<0.
\]

\emph{Jump condition at \(t=0\).}  
Integrate this equation from \(-\varepsilon\) to \(\varepsilon\):
\[
\int_{-\varepsilon}^{\varepsilon} \Big(L G''(t) + R G'(t) + \frac{1}{C} G(t)\Big)\,dt
= \int_{-\varepsilon}^{\varepsilon}\delta(t)\,dt = 1.
\]
The left-hand side splits into
\[
L\int_{-\varepsilon}^{\varepsilon} G''(t)\,dt
+ R\int_{-\varepsilon}^{\varepsilon} G'(t)\,dt
+ \frac{1}{C}\int_{-\varepsilon}^{\varepsilon} G(t)\,dt.
\]

By the fundamental theorem of calculus,
\[
\int_{-\varepsilon}^{\varepsilon} G''(t)\,dt = G'(\varepsilon) - G'(-\varepsilon),
\qquad
\int_{-\varepsilon}^{\varepsilon} G'(t)\,dt = G(\varepsilon) - G(-\varepsilon).
\]
Since \(G(t)=0\) for \(t<0\), we have \(G(-\varepsilon)=0\). Moreover, because \(G\) is zero for negative times, its left derivative at zero is also zero, so \(G'(-\varepsilon)\to G'(0^-)=0\) as \(\varepsilon\to 0^+\). Finally, as \(\varepsilon\to 0^+\), the integral of \(G\) itself over a shrinking interval tends to zero. Passing to the limit \(\varepsilon\to 0^+\), we obtain
\[
L\big(G'(0^+) - 0\big) + R\big(G(0^+) - 0\big) + 0 = 1.
\]

We also know that \(G\) must be continuous at \(t=0\) for this second-order equation with \(\delta\) forcing, and since \(G(0^-)=0\), we get \(G(0^+)=0\). Therefore the jump condition simplifies to
\[
L G'(0^+) = 1,\qquad G(0^+) = 0.
\]

\emph{Solving for \(G(t)\) for \(t>0\).}  
For \(t>0\), \(G\) satisfies the homogeneous equation
\[
L G''(t) + R G'(t) + \frac{1}{C} G(t) = 0,
\]
so \(G\) has the same form as the homogeneous solution:
\[
G(t) = e^{-\alpha t}\big(a\cos(\omega_d t) + b\sin(\omega_d t)\big), \qquad t>0,
\]
for some constants \(a\) and \(b\).

We impose the initial data at \(t=0^+\). First,
\[
G(0^+) = a = 0.
\]
Next, compute \(G'(t)\) and evaluate at \(t=0^+\):
\begin{align*}
G'(t) &= e^{-\alpha t}\big(-\alpha a\cos(\omega_d t) - \alpha b\sin(\omega_d t)\big)
       + e^{-\alpha t}\big(-a\omega_d\sin(\omega_d t) + b\omega_d\cos(\omega_d t)\big),\\
G'(0^+) &= -\alpha a + b\omega_d.
\end{align*}
Using \(a=0\), we get \(G'(0^+) = b\omega_d\). The jump condition \(L G'(0^+) = 1\) then gives
\[
L b\omega_d = 1 \quad\Rightarrow\quad b = \frac{1}{L\omega_d}.
\]

Therefore, for \(t>0\),
\[
G(t) = e^{-\alpha t}\,\frac{1}{L\omega_d}\,\sin(\omega_d t),
\]
and for \(t<0\), \(G(t)=0\). We may summarize this compactly using the Heaviside step function \(H(t)\):
\[
G(t) = \frac{1}{L\omega_d}\,e^{-\alpha t}\,\sin(\omega_d t)\,H(t).
\]

This function is the impulse response of the circuit: it is the charge response to a unit-voltage impulse applied at time \(t=0\).

\medskip

\textbf{(c) Convolution formula for the driven solution.}

We now consider
\[
L q''(t) + R q'(t) + \frac{1}{C} q(t) = V(t),\qquad q(0)=0,\quad q'(0)=0.
\]
We claim that
\[
q(t) = \int_0^t G(t-s)\,V(s)\,ds
\]
solves this initial value problem.

\emph{Idea of the formula.}  
The Green function \(G(t-s)\) is the response at time \(t\) to a unit impulse applied at time \(s\). If the actual input at time \(s\) is \(V(s)\,ds\), thought of as a scaled impulse, then the contribution to the charge at time \(t\) is approximately \(G(t-s)V(s)\,ds\). Integrating over all past times \(0\le s\le t\) sums up all such contributions, which is the intuitive origin of the convolution integral.

\emph{Verification by differentiation.}  
Define
\[
q(t) = \int_0^t G(t-s)\,V(s)\,ds.
\]
We first check the initial conditions. At \(t=0\),
\[
q(0) = \int_0^0 G(0-s)\,V(s)\,ds = 0.
\]
For the first derivative, we differentiate under the integral sign using the Leibniz rule:
\[
q'(t) = G(t-t)V(t) + \int_0^t \frac{\partial}{\partial t}G(t-s)\,V(s)\,ds
      = G(0)V(t) + \int_0^t G'(t-s)\,V(s)\,ds.
\]
Since \(G(0)=0\), this simplifies to
\[
q'(t) = \int_0^t G'(t-s)\,V(s)\,ds.
\]
Thus
\[
q'(0) = \int_0^0 G'(0-s)\,V(s)\,ds = 0,
\]
so the initial conditions are satisfied.

Next compute the second derivative:
\begin{align*}
q''(t) &= \frac{d}{dt}\left(\int_0^t G'(t-s)\,V(s)\,ds\right)\\
       &= G'(t-t)V(t) + \int_0^t \frac{\partial}{\partial t}G'(t-s)\,V(s)\,ds\\
       &= G'(0)V(t) + \int_0^t G''(t-s)\,V(s)\,ds.
\end{align*}
Now apply the operator \(\mathcal{L}\) to \(q\):
\begin{align*}
L q''(t) + R q'(t) + \frac{1}{C}q(t)
&= L\Big(G'(0)V(t) + \int_0^t G''(t-s)V(s)\,ds\Big)
   + R\int_0^t G'(t-s)V(s)\,ds
   + \frac{1}{C}\int_0^t G(t-s)V(s)\,ds\\
&= L G'(0)V(t)
   + \int_0^t \Big(L G''(t-s) + R G'(t-s) + \tfrac{1}{C}G(t-s)\Big)V(s)\,ds.
\end{align*}
By the defining equation for \(G\),
\[
L G''(t-s) + R G'(t-s) + \frac{1}{C} G(t-s) = \delta(t-s),
\]
so the integral becomes
\[
\int_0^t \delta(t-s)V(s)\,ds.
\]
This evaluates to \(V(t)\). The term \(L G'(0)V(t)\) can be recognized from the jump condition: \(L G'(0^+)=1\). In the integrals above we always evaluate \(G'(0)\) from the right, so \(L G'(0) = 1\). Therefore
\[
L q''(t) + R q'(t) + \frac{1}{C} q(t)
= \big(LG'(0)\big) V(t) + \int_0^t \delta(t-s)V(s)\,ds
= V(t) + V(t) = 2V(t).
\]

This suggests double-counting, so we examine more carefully: the \(\delta\) contribution already incorporates the jump in \(G'\). A cleaner way is to avoid splitting off \(G'(0)V(t)\) and instead use the Green function equation in its integral form.

An efficient alternative is the following: start directly from
\[
q(t) = \int_0^t G(t-s)\,V(s)\,ds
\]
and apply \(\mathcal{L}\) under the integral sign, without isolating boundary terms. Observe that for fixed \(s\in(0,t)\), as a function of \(t\),
\[
\mathcal{L}_t\big(G(t-s)\big) = \delta(t-s),
\]
where \(\mathcal{L}_t\) denotes the operator acting on the \(t\)-variable. Then, under mild regularity assumptions on \(V\),
\begin{align*}
\mathcal{L}q(t)
&= \mathcal{L}_t\left(\int_0^t G(t-s)V(s)\,ds\right)\\
&= \int_0^t \mathcal{L}_t\big(G(t-s)\big)V(s)\,ds \quad\text{(no boundary term because \(G(t-s)=0\) at \(s=t\))}\\
&= \int_0^t \delta(t-s)V(s)\,ds = V(t).
\end{align*}
This shows that the convolution formula indeed produces a solution of the inhomogeneous equation with the desired initial conditions (the vanishing of \(q(0)\) and \(q'(0)\) follows from the causality and the continuity of \(G\) at \(0\)). Thus, the solution with zero initial data is
\[
q(t) = \int_0^t G(t-s)V(s)\,ds.
\]

The moral is that the impulse response \(G\) completely determines how the system responds to any driving via convolution. This is the key perspective of “linear dynamics via the Green function.”

\medskip

\textbf{(d) Sinusoidal driving: transient and steady state.}

Let
\[
V(t) = V_0\cos(\omega t)\,H(t),
\]
where \(H\) is the Heaviside step function. Then, for \(t>0\),
\[
q(t) = \int_0^t G(t-s)\,V_0\cos(\omega s)\,ds
     = V_0\int_0^t G(t-s)\cos(\omega s)\,ds.
\]
Substituting the explicit Green function,
\[
G(t-s) = \frac{1}{L\omega_d}e^{-\alpha (t-s)}\sin\big(\omega_d(t-s)\big),\qquad 0<s<t,
\]
gives
\[
q(t)
= \frac{V_0}{L\omega_d}\int_0^t e^{-\alpha (t-s)}\sin\big(\omega_d(t-s)\big)\cos(\omega s)\,ds.
\]

One can evaluate this integral explicitly by, for example, writing everything in terms of complex exponentials, but the algebra is lengthy. Instead, we exploit the linearity of the equation and compare with the standard method of finding a particular solution.

We know from the theory of linear ODEs that the general solution has the form
\[
q(t) = q_{\mathrm{hom}}(t) + q_{\mathrm{part}}(t),
\]
where \(q_{\mathrm{hom}}(t)\) is a solution of the homogeneous equation and \(q_{\mathrm{part}}\) is any one particular solution of the forced equation. For sinusoidal forcing, a natural particular solution ansatz is
\[
q_{\mathrm{part}}(t) = A\cos(\omega t) + B\sin(\omega t),
\]
with constants \(A\) and \(B\) depending on \(R\), \(L\), \(C\), \(V_0\), and \(\omega\).

Substituting into
\[
L q'' + R q' + \frac{1}{C}q = V_0\cos(\omega t)
\]
and solving for \(A\) and \(B\) yields a steady-state oscillation at frequency \(\omega\). One convenient way is to use complex notation, seeking \(q_{\mathrm{part}}(t) = \Re\left( \tilde{Q} e^{i\omega t}\right)\), with \(\tilde{Q}\) complex. Then
\[
L(-\omega^2)\tilde{Q}e^{i\omega t} + R(i\omega)\tilde{Q}e^{i\omega t} + \frac{1}{C}\tilde{Q}e^{i\omega t}
= V_0 e^{i\omega t},
\]
so
\[
\big(-L\omega^2 + iR\omega + \tfrac{1}{C}\big)\tilde{Q} = V_0
\quad\Rightarrow\quad
\tilde{Q} = \frac{V_0}{\frac{1}{C} - L\omega^2 + iR\omega}.
\]
Thus \(q_{\mathrm{part}}(t)\) is a cosine at frequency \(\omega\) with some amplitude and phase shift:
\[
q_{\mathrm{part}}(t) = Q(\omega)\cos\big(\omega t - \phi(\omega)\big),
\]
where \(Q(\omega) = |\tilde{Q}|\) and \(\phi(\omega) = \arg(\tilde{Q})\).

The homogeneous part is
\[
q_{\mathrm{hom}}(t) = e^{-\alpha t}\big(A_0\cos(\omega_d t) + B_0\sin(\omega_d t)\big),
\]
for some constants \(A_0\) and \(B_0\). These are determined by the requirement that \(
\(q(0)=0\) and \(q'(0)=0\). Explicitly,
\[
q(0) = q_{\mathrm{hom}}(0)+q_{\mathrm{part}}(0)
      = A_0 + q_{\mathrm{part}}(0) = 0,
\]
\[
q'(0) = q_{\mathrm{hom}}'(0)+q_{\mathrm{part}}'(0)
      = (-\alpha A_0 + \omega_d B_0) + q_{\mathrm{part}}'(0) = 0,
\]
which determine \(A_0\) and \(B_0\) uniquely in terms of the forcing parameters. Their exact values are not important for the qualitative behavior: the homogeneous part is always of the form
\[
q_{\mathrm{hom}}(t) = e^{-\alpha t}\big(A_0\cos(\omega_d t) + B_0\sin(\omega_d t)\big),
\]
so it is a \emph{transient} decaying oscillation at the natural frequency \(\omega_d\).

Thus the full solution can be written as
\[
q(t) = q_{\mathrm{trans}}(t) + q_{\mathrm{ss}}(t),
\]
where
\[
q_{\mathrm{trans}}(t) = q_{\mathrm{hom}}(t)
     = e^{-\alpha t}\big(A_0\cos(\omega_d t) + B_0\sin(\omega_d t)\big),
\]
and
\[
q_{\mathrm{ss}}(t) = q_{\mathrm{part}}(t)
     = Q(\omega)\cos\big(\omega t - \phi(\omega)\big)
\]
is the \emph{steady-state} response at the driving frequency \(\omega\). Because of the exponential factor,
\[
\lim_{t\to\infty} q_{\mathrm{trans}}(t) = 0,
\]
so in the long-time limit the circuit settles into a pure sinusoidal oscillation at frequency \(\omega\), with amplitude and phase determined by the complex transfer function
\[
\tilde{Q}(\omega) = \frac{V_0}{\frac{1}{C} - L\omega^2 + iR\omega}.
\]

\medskip

\textbf{(e) Pulsed driving and comparison.}

\emph{(i) Piecewise integral representation for a pulse.}

Now take
\[
V(t) =
\begin{cases}
V_0, & 0<t<T,\\[4pt]
0, & t\ge T,
\end{cases}
\quad (V(t)=0 \text{ for } t<0).
\]
Using the convolution formula
\[
q(t) = \int_0^t G(t-s)V(s)\,ds,
\]
we distinguish the cases \(0<t<T\) and \(t\ge T\).

\emph{Case \(0<t<T\).}  
Here the integration interval \(0\le s\le t\) lies entirely inside the region where \(V(s)=V_0\), so
\[
q(t) = \int_0^t G(t-s)V_0\,ds
     = V_0\int_0^t G(t-s)\,ds,\qquad 0<t<T.
\]
Equivalently, with the change of variable \(\tau = t-s\),
\[
q(t) = V_0\int_0^t G(\tau)\,d\tau,\qquad 0<t<T.
\]

\emph{Case \(t\ge T\).}  
Now the integration interval splits: \(V(s)=V_0\) only for \(0<s<T\), and \(V(s)=0\) for \(T\le s\le t\). Thus
\[
q(t) = \int_0^T G(t-s)V_0\,ds + \int_T^t G(t-s)\cdot 0\,ds
     = V_0\int_0^T G(t-s)\,ds,\qquad t\ge T.
\]
With \(\tau = t-s\), this can be written as
\[
q(t) = V_0\int_{t-T}^{t} G(\tau)\,d\tau,\qquad t\ge T.
\]

Summarizing,
\[
q(t) =
\begin{cases}
V_0\displaystyle\int_0^t G(\tau)\,d\tau, & 0<t<T,\\[10pt]
V_0\displaystyle\int_{t-T}^{t} G(\tau)\,d\tau, & t\ge T.
\end{cases}
\]

\medskip

\emph{(ii) Qualitative comparison: sinusoidal vs pulsed driving.}

In both drivings, the same Green function
\[
G(t) = \dfrac{1}{L\omega_d}e^{-\alpha t}\sin(\omega_d t)\,H(t)
\]
governs the response; the different behaviors arise entirely from how \(G\) is weighted and integrated against \(V(t)\).

\begin{itemize}
\item \textbf{Sinusoidal driving.}  
The convolution integral continuously accumulates contributions from the ongoing oscillatory input. At each time \(t\), the integral includes values of \(G(\tau)\) for \(\tau\) from \(0\) up to \(t\); older contributions are exponentially suppressed by the factor \(e^{-\alpha \tau}\), but new contributions at the driving frequency \(\omega\) keep arriving. As a result, after the transient dies out, the system settles into a non-decaying steady-state oscillation at frequency \(\omega\), with amplitude and phase determined by the frequency-domain factor \(1/(\frac{1}{C}-L\omega^2+iR\omega)\).

\item \textbf{Pulsed driving.}  
For \(0<t<T\), the situation resembles a constant-voltage drive switched on at \(t=0\): the charge builds up via the integral \(\int_0^t G(\tau)\,d\tau\), mixing the natural oscillation of \(G\) with the step-like input. Once the pulse ends at \(t=T\), the input vanishes and, for \(t\ge T\), the solution is given by an integral of \(G(\tau)\) over a finite, sliding time window \([t-T,t]\). Equivalently, for \(t\ge T\), \(q(t)\) solves the homogeneous equation with initial data \(q(T)\), \(q'(T)\); therefore \(q(t)\) is a purely transient response:
\[
q(t) = e^{-\alpha (t-T)}\big(\tilde{A}\cos(\omega_d (t-T)) + \tilde{B}\sin(\omega_d (t-T))\big),
\]
for some constants \(\tilde{A},\tilde{B}\) determined by the history up to \(T\). Because of the damping, this “ring-down” oscillation decays to zero:
\[
\lim_{t\to\infty} q(t) = 0.
\]
No steady oscillation remains, since the source is off for large times.
\end{itemize}

Thus, in both cases the transient behavior is encoded in the same decaying oscillatory kernel \(G(t)\). When the driving persists indefinitely (sinusoidal case), the convolution yields a superposition of a decaying transient plus a non-decaying steady-state oscillation at the driving frequency. When the driving is of finite duration (pulse), the convolution produces only transient behavior: after the pulse, the circuit simply rings at its natural frequency \(\omega_d\) with exponentially decreasing amplitude, eventually returning to equilibrium.

\end{solution}

% ===== Example 4: Green Functions for a Two-Point Boundary Value Problem (inquiry-based) =====
\begin{problem}[Green Functions for a Two-Point Boundary Value Problem]
Consider a slender elastic rod of length $L$ stretched along the $x$-axis from $x=0$ to $x=L$. In static equilibrium, its (scaled) transverse deflection $u(x)$ under a distributed load $f(x)$ can be modeled by a second-order linear ordinary differential equation together with boundary conditions at both ends. In this problem we construct a Green function for such a boundary value problem and see how it encodes the rod's response to arbitrary load distributions. The same ideas appear throughout linear dynamics and partial differential equations.

We study the boundary value problem
\[
u''(x) \;=\; f(x), \qquad 0<x<L, \qquad u(0) = 0, \quad u(L) = 0.
\]
Throughout, assume $f$ is continuous on $[0,L]$.

\smallskip

(a) Interpret physically what the boundary conditions $u(0)=u(L)=0$ mean for the rod. In particular, describe what is being held fixed and what is allowed to move. Then, solve the homogeneous equation
\[
u''(x) = 0
\]
and write down the general solution. How many free constants does this general solution contain, and how does this relate to the order of the differential equation?

\smallskip

(b) In order to build the Green function, it is convenient to choose homogeneous solutions adapted to the two boundary points. 

Find two linearly independent solutions $y_1$ and $y_2$ of $y''=0$ such that
\[
y_1(0) = 0, \qquad y_2(L) = 0.
\]
(You may also choose a simple normalization, such as $y_1'(0)=1$ or $y_2'(L)=-1$, to make the formulas nice.) Verify that $y_1$ and $y_2$ are linearly independent.

\smallskip

(c) Fix a point $\xi$ with $0<\xi<L$, which you should think of as the location of a ``point load'' on the rod. The Green function $G(x,\xi)$ for this problem is defined to be the solution (as a function of $x$) of
\[
G''(x,\xi) = \delta(x-\xi), \qquad 0<x<L,
\]
together with the boundary conditions
\[
G(0,\xi) = 0, \qquad G(L,\xi) = 0.
\]
Here $\delta$ is the Dirac delta distribution. The idea is that $G(x,\xi)$ gives the deflection at $x$ caused by a unit point force applied at $\xi$.

\begin{enumerate}
\item[(i)] Argue that for $x\neq \xi$, the equation $G''(x,\xi)=\delta(x-\xi)$ reduces to the homogeneous equation $G''(x,\xi)=0$. Explain why this implies that, for fixed $\xi$,
\[
G(x,\xi) =
\begin{cases}
A(\xi)\, x + B(\xi), & 0 \le x < \xi,\\[4pt]
C(\xi)\, x + D(\xi), & \xi < x \le L,
\end{cases}
\]
for some coefficients $A(\xi),B(\xi),C(\xi),D(\xi)$ depending on $\xi$.

\item[(ii)] Use the boundary conditions $G(0,\xi)=0$ and $G(L,\xi)=0$ to simplify this form and reduce the number of unknown coefficient functions.

\item[(iii)] Now impose that $G(x,\xi)$ is continuous at $x=\xi$. Write down the resulting condition relating the coefficients for the left piece and the right piece.

\item[(iv)] Finally, derive the \emph{jump condition} for the first derivative of $G$ at $x=\xi$. Integrate the equation $G''(x,\xi) = \delta(x-\xi)$ over a small interval $(\xi-\varepsilon,\xi+\varepsilon)$ and then let $\varepsilon\to 0^+$. Show that
\[
G_x(\xi^+,\xi) - G_x(\xi^-,\xi) = 1,
\]
where $G_x(\xi^\pm,\xi)$ denote the right and left derivatives at $x=\xi$.

\end{enumerate}

Use all these conditions to solve for the coefficients and obtain an explicit formula for $G(x,\xi)$ in terms of $x$, $\xi$, and $L$.

\emph{Hint:} After you have used the boundary conditions, you should be left with only two unknown coefficient functions. The continuity and jump conditions at $x=\xi$ will then form a $2\times 2$ linear system for these remaining unknowns.

\smallskip

(d) Suppose now that $f$ is an arbitrary continuous load distribution on $[0,L]$. Show formally that the function
\[
u(x) = \int_0^L G(x,\xi)\, f(\xi)\, d\xi
\]
solves the boundary value problem
\[
u''(x) = f(x), \qquad 0<x<L, \qquad u(0)=u(L)=0.
\]
Proceed in three steps:
\begin{enumerate}
\item[(i)] Differentiate under the integral sign (formally) to compute $u''(x)$, using the defining equation for $G$.
\item[(ii)] Check that the boundary conditions $u(0)=u(L)=0$ hold.
\item[(iii)] Briefly explain why the construction is linear in $f$ (that is, why the map $f\mapsto u$ is a linear operator).
\end{enumerate}

\emph{Hint:} For (i), think in distributional terms: the equation $G''(\cdot,\xi) = \delta(\cdot-\xi)$ says that when you differentiate $G$ twice in $x$ and integrate against a test function of $x$, you get evaluation at $x=\xi$.

\smallskip

(e) Extensions and variations.

\begin{enumerate}
\item[(i)] What changes in the construction of the Green function if the boundary conditions are replaced by
\[
u'(0) = 0, \qquad u(L) = 0,
\]
modeling, for instance, one end clamped horizontally and the other end held at zero displacement? Sketch how you would adapt steps (b) and (c) to this modified problem. You do not need to carry out all the algebra.

\item[(ii)] Suppose now that the load is itself a point load at $x=a$, modeled by $f(x)=F_0\,\delta(x-a)$ with $0<a<L$. Using your Green function $G(x,\xi)$, write down the corresponding deflection $u(x)$ explicitly. What does this say about the physical meaning of $G(x,\xi)$?

\end{enumerate}

\end{problem}

% ===== Example 4: Green Functions for a Two-Point Boundary Value Problem (full solution) =====
\begin{problem}[Green Functions for a Two-Point Boundary Value Problem]
Consider the boundary value problem
\[
u''(x) = f(x), \qquad 0<x<L, \qquad u(0)=0, \quad u(L)=0,
\]
where $f$ is continuous on $[0,L]$.

(a) Construct the Green function $G(x,\xi)$ satisfying
\[
G''(x,\xi) = \delta(x-\xi), \qquad 0<x<L, \qquad G(0,\xi)=0,\quad G(L,\xi)=0,
\]
together with continuity at $x=\xi$ and the jump condition
\[
\frac{\partial G}{\partial x}(\xi^+,\xi) - \frac{\partial G}{\partial x}(\xi^-,\xi) = 1.
\]
Find an explicit formula for $G(x,\xi)$ in terms of $x$, $\xi$, and $L$.

(b) Show that the function
\[
u(x) = \int_0^L G(x,\xi)\, f(\xi)\, d\xi
\]
solves the boundary value problem, that is, $u''(x)=f(x)$ for $0<x<L$ and $u(0)=u(L)=0$.

Briefly comment on how this example illustrates the idea of describing linear dynamics via a Green function.
\end{problem}

\begin{solution}
We begin by constructing the Green function for the operator $\dfrac{d^2}{dx^2}$ with homogeneous Dirichlet boundary conditions at $x=0$ and $x=L$. The central idea is that the Green function represents the response of the system to a unit point source, and hence any other forcing can be represented as a superposition of such point sources.

\medskip

\textbf{(a) Construction of the Green function.}

Fix a point $\xi$ with $0<\xi<L$. For this fixed $\xi$, we view $G(\cdot,\xi)$ as a function of $x$ which solves
\[
G''(x,\xi) = \delta(x-\xi), \qquad 0<x<L,
\]
with
\[
G(0,\xi)=0, \qquad G(L,\xi)=0.
\]

\smallskip

\emph{Step 1: General form away from $x=\xi$.}

For $x\neq \xi$, the right-hand side is zero, so $G$ satisfies the homogeneous equation
\[
G''(x,\xi)=0 \quad\text{for } x\in(0,\xi)\cup(\xi,L).
\]
The general solution of $y''=0$ is linear, $y(x)=\alpha x + \beta$. Therefore, for fixed $\xi$ we may write
\[
G(x,\xi) =
\begin{cases}
A(\xi)\, x + B(\xi), & 0 \le x < \xi,\\[4pt]
C(\xi)\, x + D(\xi), & \xi < x \le L,
\end{cases}
\]
where $A,B,C,D$ are functions of the parameter $\xi$ only.

\smallskip

\emph{Step 2: Boundary conditions.}

The boundary condition at $x=0$ gives
\[
G(0,\xi)=0 \quad \Longrightarrow \quad B(\xi)=0.
\]
Thus on the left side we have
\[
G(x,\xi) = A(\xi)\, x, \qquad 0\le x<\xi.
\]

The boundary condition at $x=L$ gives
\[
G(L,\xi)=0 \quad \Longrightarrow \quad C(\xi)\,L + D(\xi) = 0,
\]
so
\[
D(\xi) = -C(\xi)\,L,
\]
and hence on the right side we may write
\[
G(x,\xi) = C(\xi)\,x - C(\xi)\,L = C(\xi)\,(x-L), \qquad \xi<x\le L.
\]

At this point the Green function has the form
\[
G(x,\xi) =
\begin{cases}
A(\xi)\, x, & 0 \le x < \xi,\\[4pt]
C(\xi)\,(x-L), & \xi < x \le L.
\end{cases}
\]

\smallskip

\emph{Step 3: Continuity at $x=\xi$.}

By definition, $G(x,\xi)$ should be continuous at $x=\xi$, even though its derivative will have a jump there. Thus
\[
\lim_{x\uparrow \xi} G(x,\xi) = \lim_{x\downarrow \xi} G(x,\xi).
\]
Substituting from the left and right formulas, this gives
\[
A(\xi)\,\xi = C(\xi)\,(\xi - L).
\]
We may rewrite this relation as
\begin{equation}\label{eq:continuity}
A(\xi)\,\xi = C(\xi)\,(\xi - L).
\end{equation}

\smallskip

\emph{Step 4: Jump condition for the derivative.}

The defining equation $G''(x,\xi)=\delta(x-\xi)$ implies a jump condition for the first derivative at $x=\xi$. Integrate both sides over a small interval $(\xi-\varepsilon,\xi+\varepsilon)$:
\[
\int_{\xi-\varepsilon}^{\xi+\varepsilon} G''(x,\xi)\,dx
=
\int_{\xi-\varepsilon}^{\xi+\varepsilon} \delta(x-\xi)\,dx
= 1.
\]
The left-hand side can be evaluated by the Fundamental Theorem of Calculus:
\[
\int_{\xi-\varepsilon}^{\xi+\varepsilon} G''(x,\xi)\,dx
=
G_x(\xi+\varepsilon,\xi) - G_x(\xi-\varepsilon,\xi).
\]
Now let $\varepsilon\to 0^+$. Since $G''$ is zero away from $x=\xi$, the one-sided derivatives $G_x(\xi^+,\xi)$ and $G_x(\xi^-,\xi)$ exist as limits from the right and left, and we obtain
\[
G_x(\xi^+,\xi) - G_x(\xi^-,\xi) = 1.
\]

Next, we compute these derivatives from our piecewise-linear representation. From the left we have
\[
G(x,\xi) = A(\xi)\,x \quad\Longrightarrow\quad G_x(x,\xi) = A(\xi)\quad (x<\xi),
\]
so
\[
G_x(\xi^-,\xi) = A(\xi).
\]
From the right we have
\[
G(x,\xi) = C(\xi)\,(x-L) \quad\Longrightarrow\quad G_x(x,\xi) = C(\xi)\quad (x>\xi),
\]
so
\[
G_x(\xi^+,\xi) = C(\xi).
\]
The jump condition therefore becomes
\begin{equation}\label{eq:jump}
C(\xi) - A(\xi) = 1.
\end{equation}

\smallskip

\emph{Step 5: Solving for the coefficients.}

We now have two equations in the two unknowns $A(\xi)$ and $C(\xi)$:
\[
\begin{cases}
A(\xi)\,\xi = C(\xi)\,(\xi - L),\\[4pt]
C(\xi) - A(\xi) = 1.
\end{cases}
\]
We solve this linear system. From the continuity condition \eqref{eq:continuity},
\[
A(\xi) = C(\xi)\,\dfrac{\xi - L}{\xi}.
\]
Substitute into the jump condition \eqref{eq:jump}:
\[
C(\xi) - C(\xi)\,\dfrac{\xi - L}{\xi} = 1
\quad\Longrightarrow\quad
C(\xi)\left(1 - \frac{\xi - L}{\xi}\right) = 1.
\]
Inside the parentheses we simplify:
\[
1 - \frac{\xi - L}{\xi} = \frac{\xi}{\xi} - \frac{\xi - L}{\xi}
= \frac{L}{\xi}.
\]
Thus
\[
C(\xi)\,\frac{L}{\xi} = 1
\quad\Longrightarrow\quad
C(\xi) = \frac{\xi}{L}.
\]
Then
\[
A(\xi) = C(\xi)\,\frac{\xi - L}{\xi}
= \frac{\xi}{L} \cdot \frac{\xi - L}{\xi}
= \frac{\xi - L}{L}.
\]

Therefore the Green function is
\[
G(x,\xi) =
\begin{cases}
\dfrac{\xi - L}{L}\,x, & 0 \le x < \xi,\\[6pt]
\dfrac{\xi}{L}\,(x-L), & \xi < x \le L.
\end{cases}
\]
It is straightforward to check that this expression is continuous at $x=\xi$, satisfies the prescribed boundary conditions at $x=0$ and $x=L$, and exhibits the correct unit jump in the first derivative at $x=\xi$.

It is also instructive to note that $G$ is symmetric:
\[
G(x,\xi)=G(\xi,x),
\]
which one can verify directly by considering the cases $x<\xi$ and $x>\xi$. This symmetry reflects the self-adjoint nature of the operator $d^2/dx^2$ with homogeneous Dirichlet boundary conditions, but that observation is not needed for the present computation.

\medskip

\textbf{(b) Representation of the solution by the Green function.}

We now show that
\[
u(x) = \int_0^L G(x,\xi)\, f(\xi)\, d\xi
\]
solves the boundary value problem
\[
u''(x) = f(x), \qquad 0<x<L, \qquad u(0)=u(L)=0.
\]

\smallskip

\emph{Step 1: Verifying the differential equation.}

Formally differentiating under the integral sign with respect to $x$, we obtain
\[
u'(x) = \int_0^L \frac{\partial G}{\partial x}(x,\xi)\, f(\xi)\, d\xi,
\]
and
\[
u''(x) = \int_0^L \frac{\partial^2 G}{\partial x^2}(x,\xi)\, f(\xi)\, d\xi.
\]
By the defining property of the Green function, for each fixed $\xi$ we have
\[
\frac{\partial^2 G}{\partial x^2}(x,\xi) = \delta(x-\xi)
\]
in the sense of distributions. Substituting this into the expression for $u''(x)$ gives
\[
u''(x) = \int_0^L \delta(x-\xi)\, f(\xi)\, d\xi.
\]
The integral of a function against a Dirac delta evaluates the integrand at the point where the delta is centered, so
\[
u''(x) = f(x), \qquad 0<x<L.
\]
This is precisely the differential equation we seek to solve.

A more formal justification of the differentiation under the integral sign can be given by approximating the delta distribution with smooth functions or by working in the framework of weak solutions. For the purposes of this example, it is enough to treat the calculation at a formal level.

\smallskip

\emph{Step 2: Verifying the boundary conditions.}

We next compute the boundary values of $u$. Using the explicit form of $G(x,\xi)$, note that for any fixed $\xi$,
\[
G(0,\xi) = 0 \quad\text{and}\quad G(L,\xi)=0.
\]
Therefore
\[
u(0) = \int_0^L G(0,\xi)\, f(\xi)\, d\xi
= \int_0^L 0\cdot f(\xi)\, d\xi = 0,
\]
and similarly
\[
u(L) = \int_0^L G(L,\xi)\, f(\xi)\, d\xi
= \int_0^L 0\cdot f(\xi)\, d\xi = 0.
\]
Thus the function $u$ defined by the Green function integral satisfies the prescribed boundary conditions.

\smallskip

\emph{Step 3: Linearity and interpretation.}

The map that takes $f$ to $u$ is linear because the integral defining $u$ is linear in $f$:
\[
\int_0^L G(x,\xi)\, (\alpha f_1(\xi) + \beta f_2(\xi))\, d\xi
= \alpha \int_0^L G(x,\xi)\, f_1(\xi)\, d\xi
 + \beta \int_0^L G(x,\xi)\, f_2(\xi)\, d\xi.
\]
Thus, the Green function encapsulates the complete linear response of the system: it is the response to a unit point source at $\xi$, and any general forcing $f$ is built up as a superposition of such point sources, weighted by $f(\xi)$, integrated over $\xi$.

\medskip

\textbf{Connection to linear dynamics via the Green function.}

This example illustrates a fundamental strategy in the study of linear dynamical systems, whether governed by ordinary or partial differential equations. Instead of solving the boundary value problem separately for each forcing function $f$, we solve a single, canonical problem: find the Green function, the response to a unit impulse. Once the Green function is known, the solution for any forcing is obtained by a single integral, expressing the principle of superposition. The Green function therefore serves as the kernel of a linear integral operator which is the inverse of the differential operator (subject to the given boundary conditions). In this way, the linear dynamics of the system are encoded completely in the Green function.

\end{solution}

% ===== Example 5: Preview: From ODE Green Functions to the Heat Equation Kernel (inquiry-based) =====
\begin{problem}[Preview: From ODE Green Functions to the Heat Equation Kernel]
On an infinitely long, thin wire, let $u(x,t)$ denote the temperature at position $x\in\mathbb{R}$ and time $t>0$. If the wire is perfectly insulated from its surroundings and heat diffuses along the wire with constant diffusivity $\kappa>0$, then $u$ satisfies the one-dimensional heat equation
\[
u_t = \kappa u_{xx}, \qquad x\in\mathbb{R},\ t>0.
\]
In this problem we use ideas from the Green function for simple ordinary differential equations to derive, somewhat formally, the fundamental solution (or heat kernel) for this partial differential equation. The key idea is to transform in the spatial variable so that the heat equation becomes, for each spatial frequency, a first-order ODE in time whose Green function we already understand.

We fix a source point $\xi\in\mathbb{R}$ and source time $\tau\in\mathbb{R}$, and we look for a function $G(x,t;\xi,\tau)$ (the heat kernel) that solves
\[
G_t(x,t;\xi,\tau) \;=\; \kappa\,G_{xx}(x,t;\xi,\tau) \;+\; \delta(x-\xi)\,\delta(t-\tau),
\]
for $x\in\mathbb{R}$ and $t\in\mathbb{R}$, together with $G(x,t;\xi,\tau)=0$ for $t<\tau$. Intuitively, $G$ is the temperature at $(x,t)$ produced by a unit instantaneous heat source at $(\xi,\tau)$.

\medskip

(a) As a warm-up, recall the Green function for a simple first-order ODE in time. Consider
\[
y'(t) = f(t), \qquad y(t_0)=0.
\]
(i) Write down the solution $y(t)$ in integral form, starting from the fundamental theorem of calculus.  

(ii) Show that there is a function $g(t,s)$ such that
\[
y(t) = \int_{-\infty}^{\infty} g(t,s)\,f(s)\,ds,
\]
and identify $g(t,s)$.  

Hint: Think about when the forcing $f(s)$ occurring at time $s$ can influence the solution at time $t$.

\medskip

(b) Now recall the Green function for the constant-coefficient ODE
\[
y'(t) + a\,y(t) = f(t), \qquad y(t_0)=0, \qquad a>0,
\]
which you studied earlier in this chapter.

(i) Using the integrating factor $e^{at}$, solve this ODE with $y(t_0)=0$ to express $y(t)$ as an integral involving $f$.  

(ii) Show that $y$ can be written in the convolution form
\[
y(t) = \int_{-\infty}^{\infty} G_a(t,s)\,f(s)\,ds,
\]
and identify the Green function $G_a(t,s)$.  

(iii) Describe in words how $G_a(t,s)$ functions as an impulse response in time.

\medskip

(c) We now turn to the heat equation Green function. To take advantage of translation invariance, it is convenient to work in the shifted spatial variable $y = x-\xi$ and define
\[
H(y,t;\tau) := G(y+\xi,t;\xi,\tau).
\]
Then $H$ satisfies
\[
H_t(y,t;\tau) = \kappa H_{yy}(y,t;\tau) + \delta(y)\,\delta(t-\tau), \qquad H(y,t;\tau)=0 \text{ for } t<\tau.
\]

We will take the Fourier transform in $y$. For a sufficiently nice function $\phi(y)$, define its Fourier transform by
\[
\widehat{\phi}(k) := \int_{-\infty}^{\infty} e^{-iky}\,\phi(y)\,dy.
\]

(i) Compute the Fourier transform in $y$ of $H_t$ and of $H_{yy}$. That is, express $\widehat{H_t}(k,t;\tau)$ and $\widehat{H_{yy}}(k,t;\tau)$ in terms of $\widehat{H}(k,t;\tau)$.  

(ii) Compute the Fourier transform in $y$ of $\delta(y)\,\delta(t-\tau)$ and show that it equals $\delta(t-\tau)$ (as a function of $t$ and $\tau$) for every $k$.  

(iii) Conclude that, for each fixed spatial frequency $k\in\mathbb{R}$, the transform $\widehat{H}$ satisfies the ODE
\[
\partial_t \widehat{H}(k,t;\tau) + \kappa k^2\,\widehat{H}(k,t;\tau)
= \delta(t-\tau), \qquad \widehat{H}(k,t;\tau)=0\ \text{for } t<\tau.
\]
Explain why this is now a \emph{time-dependent} ODE with parameter $k$ that we can solve using part~(b).  

% Hint: You should recognize this as the ODE from part (b) with $a = \kappa k^2$ and forcing $f(t) = \delta(t-\tau)$.

\medskip

(d) Use your work in part~(b) to solve the ODE for $\widehat{H}$.

(i) For fixed $k$, solve
\[
\partial_t \widehat{H}(k,t;\tau) + \kappa k^2\,\widehat{H}(k,t;\tau)
= \delta(t-\tau), \qquad \widehat{H}(k,t;\tau)=0\ \text{for } t<\tau,
\]
and show that
\[
\widehat{H}(k,t;\tau) = e^{-\kappa k^2 (t-\tau)}\,H(t-\tau),
\]
where $H$ on the right-hand side is now the Heaviside step function in time.  

(ii) Invert the Fourier transform in $y$ to recover $H(y,t;\tau)$:
\[
H(y,t;\tau) = \frac{1}{2\pi} \int_{-\infty}^{\infty} e^{iky}\,\widehat{H}(k,t;\tau)\,dk.
\]
Substitute your formula for $\widehat{H}$ and write $H(y,t;\tau)$ as an integral in $k$.  

(iii) Evaluate the resulting integral in $k$ using the standard Gaussian integral
\[
\int_{-\infty}^{\infty} e^{-\alpha k^2 + i\beta k}\,dk
= \sqrt{\frac{\pi}{\alpha}}\,e^{-\beta^2/(4\alpha)}, \qquad \alpha>0.
\]
Conclude that, for $t>\tau$,
\[
G(x,t;\xi,\tau)
= \frac{1}{\sqrt{4\pi\kappa (t-\tau)}}
\exp\!\left(-\frac{(x-\xi)^2}{4\kappa (t-\tau)}\right),
\]
and that $G(x,t;\xi,\tau)=0$ for $t<\tau$.  

(iv) Briefly interpret this formula: how does the spatial profile at fixed $t-\tau$ look, and what happens to that profile as $t-\tau$ increases?

% Hint: Recognize the $x$-dependence as a Gaussian whose variance grows linearly in time.

\medskip

(e) Finally, connect this PDE Green function back to the idea of convolution and superposition.

(i) Suppose now that the initial temperature at time $t=0$ is some given function $u_0(x)$, and there are no sources for $t>0$. By thinking of $u_0$ as a continuous superposition of point sources at time $0$, write a formal representation of the solution $u(x,t)$ for $t>0$ in terms of $G$ and $u_0$.  

(ii) Compare your formula with the convolution representation for ODEs from parts~(a)–(b). In what sense is the heat equation solution a \emph{spatial} convolution rather than a \emph{temporal} convolution?  

(iii) (What if / extension.) How do you expect the heat kernel to change if we work on a periodic domain, say $x\in[0,2\pi]$ with periodic boundary conditions? Describe, at an informal level, how the spatial Fourier series temporarily replaces the Fourier transform, and what you think the corresponding Green function would look like. Hint: Think about summing up periodically shifted copies of the Gaussian.
\end{problem}

% ===== Example 5: Preview: From ODE Green Functions to the Heat Equation Kernel (full solution) =====
\begin{problem}[Preview: From ODE Green Functions to the Heat Equation Kernel]
Consider the one-dimensional heat equation on the whole line,
\[
u_t = \kappa u_{xx}, \qquad x\in\mathbb{R},\ t>0,\ \kappa>0.
\]
Let $G(x,t;\xi,\tau)$ denote the fundamental solution (heat kernel), defined as the solution of
\[
G_t = \kappa G_{xx} + \delta(x-\xi)\,\delta(t-\tau), \qquad G(\cdot,t;\xi,\tau)=0\ \text{for } t<\tau.
\]
  
(a) Take the Fourier transform in $x$ (or, more conveniently, in $y=x-\xi$) to show that, for each fixed spatial frequency $k$, the transform $\widehat{G}$ satisfies the ODE
\[
\partial_t \widehat{G}(k,t;\xi,\tau) + \kappa k^2\,\widehat{G}(k,t;\xi,\tau) = \delta(t-\tau), \qquad \widehat{G}(k,t;\xi,\tau)=0\ \text{for } t<\tau.
\]
  
(b) Using the Green function for the ODE $y'(t) + a y(t) = \delta(t-\tau)$ with $a>0$, solve explicitly for $\widehat{G}(k,t;\xi,\tau)$ and then invert the Fourier transform to obtain a formula for $G(x,t;\xi,\tau)$.  

(c) Show that
\[
G(x,t;\xi,\tau)
= \frac{1}{\sqrt{4\pi\kappa (t-\tau)}}
\exp\!\left(-\frac{(x-\xi)^2}{4\kappa (t-\tau)}\right) H(t-\tau),
\]
where $H$ is the Heaviside step function in $t-\tau$.  

(d) Finally, for given initial data $u(x,0)=u_0(x)$ on $\mathbb{R}$, write the solution $u(x,t)$ for $t>0$ as a spatial convolution of $u_0$ with the heat kernel, and briefly explain how this representation is analogous to the convolution formulas for linear ODEs with a Green function.
\end{problem}

\begin{solution}
We begin by recalling the strategy: for linear constant-coefficient equations, the Green function is the response to a unit impulse. In the ODE setting, this Green function often appears as the kernel of a convolution integral in time. For the heat equation, we will transfer the PDE to the Fourier side in space, where each Fourier mode solves a first-order ODE in time, and then use the known ODE Green function to recover the heat kernel.

\medskip

\emph{Step 1: Fourier transform in space and reduction to an ODE in time.}

It is convenient to exploit translation invariance in $x$ by introducing the shifted variable
\[
y = x-\xi
\]
and defining
\[
H(y,t;\tau) := G(y+\xi,t;\xi,\tau).
\]
In terms of $H$, the defining equation for $G$ becomes
\[
H_t(y,t;\tau) = \kappa H_{yy}(y,t;\tau) + \delta(y)\,\delta(t-\tau),
\]
with $H(y,t;\tau)=0$ for $t<\tau$.

We now take the Fourier transform in the spatial variable $y$. For a sufficiently nice function $\phi(y)$, we define
\[
\widehat{\phi}(k) = \int_{-\infty}^{\infty} e^{-iky}\,\phi(y)\,dy.
\]
We use the standard properties of the Fourier transform:

\begin{itemize}
  \item Differentiation in $y$ corresponds to multiplication by $ik$:
  \[
  \widehat{\phi_y}(k) = ik\,\widehat{\phi}(k), \qquad \widehat{\phi_{yy}}(k) = -k^2\,\widehat{\phi}(k).
  \]
  \item The Fourier transform of the Dirac delta $\delta(y)$ is the constant function $1$:
  \[
  \int_{-\infty}^{\infty} e^{-iky}\,\delta(y)\,dy = 1
  \]
  for every $k$.
\end{itemize}

Applying the Fourier transform in $y$ to the equation for $H$ gives
\[
\widehat{H_t}(k,t;\tau)
= \kappa\,\widehat{H_{yy}}(k,t;\tau) + \widehat{\delta(y)\,\delta(t-\tau)}.
\]
Because the transform is with respect to $y$ only, time $t$ is just a parameter in this operation. Thus
\[
\widehat{H_t}(k,t;\tau) = \partial_t \widehat{H}(k,t;\tau),
\]
while
\[
\widehat{H_{yy}}(k,t;\tau) = -k^2 \widehat{H}(k,t;\tau).
\]
For the source term, we obtain
\[
\widehat{\delta(y)\,\delta(t-\tau)} = \delta(t-\tau)\,\widehat{\delta(y)} = \delta(t-\tau)\cdot 1 = \delta(t-\tau),
\]
since $\delta(t-\tau)$ does not depend on $y$ and the Fourier transform of $\delta(y)$ is $1$.

Putting these pieces together, we obtain, for each fixed $k\in\mathbb{R}$,
\[
\partial_t \widehat{H}(k,t;\tau)
= \kappa\left(-k^2 \widehat{H}(k,t;\tau)\right) + \delta(t-\tau),
\]
or equivalently
\[
\partial_t \widehat{H}(k,t;\tau) + \kappa k^2\,\widehat{H}(k,t;\tau)
= \delta(t-\tau).
\]
Moreover, the condition $H(\cdot,t;\tau)=0$ for $t<\tau$ implies
\[
\widehat{H}(k,t;\tau) = 0 \quad \text{for } t<\tau.
\]
Since $H$ and $G$ differ only by a spatial translation, the same ODE holds for the Fourier transform of $G$ in the spatial variable $(x-\xi)$; we will continue to use the notation $\widehat{H}$ because the calculation is simpler in $y$.

Thus, for each frequency $k$, we have reduced the PDE to a first-order linear ODE in time with a delta-function forcing and an initial condition at $t=\tau$.

\medskip

\emph{Step 2: Solve the ODE using the ODE Green function.}

We consider, for fixed $k$,
\[
\partial_t \widehat{H}(k,t;\tau) + \kappa k^2\,\widehat{H}(k,t;\tau)
= \delta(t-\tau), \qquad \widehat{H}(k,t;\tau) = 0 \text{ for } t<\tau.
\]
This is exactly of the form
\[
y'(t) + a\,y(t) = \delta(t-\tau), \qquad y(t)=0 \text{ for } t<\tau,
\]
with $a = \kappa k^2 > 0$.

From our earlier study of first-order linear ODEs with constant coefficient $a>0$, we know the Green function $G_a(t,s)$ for such an equation. Solving
\[
y'(t) + a y(t) = f(t), \qquad y(t_0)=0,
\]
by the integrating-factor method, we obtain
\[
y(t) = \int_{t_0}^{t} e^{-a(t-s)} f(s)\,ds.
\]
If we extend $f$ by zero for $s<t_0$, we may also write
\[
y(t) = \int_{-\infty}^{\infty} e^{-a(t-s)} H(t-s)\,f(s)\,ds,
\]
where $H$ is the Heaviside function
\[
H(t-s) =
\begin{cases}
0, & t<s,\\[4pt]
1, & t\ge s.
\end{cases}
\]
Thus the ODE Green function is
\[
G_a(t,s) = e^{-a(t-s)} H(t-s).
\]

In our present situation, $f(t) = \delta(t-\tau)$ and $a = \kappa k^2$. Therefore,
\[
\widehat{H}(k,t;\tau) = \int_{-\infty}^{\infty} e^{-\kappa k^2 (t-s)} H(t-s)\,\delta(s-\tau)\,ds.
\]
The integral collapses by the sifting property of the delta distribution, yielding
\[
\widehat{H}(k,t;\tau)
= e^{-\kappa k^2 (t-\tau)} H(t-\tau).
\]
This formula incorporates both the exponential decay in time (for each fixed nonzero $k$) and the causality condition that there is no response for $t<\tau$.

\medskip

\emph{Step 3: Invert the Fourier transform to find $H$ and hence $G$.}

We now invert the spatial Fourier transform. By the usual inversion formula,
\[
H(y,t;\tau) = \frac{1}{2\pi} \int_{-\infty}^{\infty} e^{iky}\,\widehat{H}(k,t;\tau)\,dk.
\]
Substituting the expression found above for $\widehat{H}(k,t;\tau)$, we obtain
\[
H(y,t;\tau) = \frac{H(t-\tau)}{2\pi} \int_{-\infty}^{\infty} e^{iky}\,e^{-\kappa k^2 (t-\tau)}\,dk.
\]
We focus on the case $t>\tau$, for which $H(t-\tau)=1$. Then
\[
H(y,t;\tau) = \frac{1}{2\pi} \int_{-\infty}^{\infty} \exp\!\bigl(-\kappa (t-\tau) k^2 + i y k\bigr)\,dk.
\]

This is a classical Gaussian integral. For $\alpha>0$ and real $\beta$, one has
\[
\int_{-\infty}^{\infty} e^{-\alpha k^2 + i\beta k}\,dk
= \sqrt{\frac{\pi}{\alpha}}\,\exp\!\left(-\frac{\beta^2}{4\alpha}\right).
\]
Here $\alpha = \kappa (t-\tau) >0$ (since $t>\tau$) and $\beta = y$. Applying the formula, we find
\[
\int_{-\infty}^{\infty} e^{-\kappa (t-\tau) k^2 + i y k}\,dk
= \sqrt{\frac{\pi}{\kappa (t-\tau)}}\,
\exp\!\left(-\frac{y^2}{4\kappa (t-\tau)}\right).
\]
Therefore,
\[
H(y,t;\tau)
= \frac{1}{2\pi}\,\sqrt{\frac{\pi}{\kappa (t-\tau)}}\,
\exp\!\left(-\frac{y^2}{4\kappa (t-\tau)}\right),
\]
for $t>\tau$. Simplifying the prefactor,
\[
\frac{1}{2\pi} \sqrt{\frac{\pi}{\kappa (t-\tau)}}
= \frac{1}{\sqrt{4\pi\kappa (t-\tau)}}.
\]
Thus,
\[
H(y,t;\tau)
= \frac{1}{\sqrt{4\pi\kappa (t-\tau)}}
\exp\!\left(-\frac{y^2}{4\kappa (t-\tau)}\right),
\qquad t>\tau.
\]
For $t<\tau$, we have $H(y,t;\tau)=0$.

Recalling that $y=x-\xi$ and $H(y,t;\tau) = G(y+\xi,t;\xi,\tau)$, we obtain the desired formula for the heat kernel:
\[
G(x,t;\xi,\tau)
= \frac{1}{\sqrt{4\pi\kappa (t-\tau)}}
\exp\!\left(-\frac{(x-\xi)^2}{4\kappa (t-\tau)}\right) H(t-\tau),
\]
where we have now explicitly reintroduced the Heaviside function $H(t-\tau)$ to encode the causality condition $G=0$ for $t<\tau$.

This establishes part (c). For each fixed $t>\tau$, $G(\cdot,t;\xi,\tau)$ is a Gaussian in $x$ centered at $\xi$ with variance $2\kappa (t-\tau)$. As $t-\tau$ increases, the Gaussian flattens and spreads out, representing the diffusion of heat away from the initial impulse.

\medskip

\emph{Step 4: Solution for general initial data as a spatial convolution.}

Let us now solve the homogeneous heat equation with initial data $u(x,0)=u_0(x)$ and no sources:
\[
u_t = \kappa u_{xx}, \qquad u(x,0) = u_0(x).
\]
The initial data $u_0$ may be regarded as a continuous superposition of point sources at time $\tau=0$, with strength $u_0(\xi)$ at point $\xi$. By linearity and superposition, the solution at later time $t>0$ is obtained by integrating the responses to each point source:
\[
u(x,t) = \int_{-\infty}^{\infty} G(x,t;\xi,0)\,u_0(\xi)\,d\xi.
\]
Substituting the explicit expression for $G$ at $\tau=0$, we obtain
\[
u(x,t) = \int_{-\infty}^{\infty}
\frac{1}{\sqrt{4\pi\kappa t}}
\exp\!\left(-\frac{(x-\xi)^2}{4\kappa t}\right) u_0(\xi)\,d\xi,
\qquad t>0.
\]

This formula may be written more compactly as a spatial convolution:
\[
u(\cdot,t) = (G(\cdot,t;\cdot,0) * u_0)(x),
\]
where the convolution is in the spatial variable and the kernel is the Gaussian heat kernel.

Conceptually, this is directly analogous to the convolution formulas for linear ODEs with Green functions. There, the solution $y(t)$ of a forced ODE is expressed as a time convolution of the forcing $f(s)$ with a temporal Green function $G(t,s)$, which represents the impulse response at time $t$ to a unit forcing at time $s$. Here, for the heat equation, the solution $u(x,t)$ is expressed as a spatial convolution of the initial data with a spatial Green function (the heat kernel), which represents the temperature at $x$ at time $t$ due to a unit impulse initially (at $t=0$) at location $\xi$.

In both settings, the central ideas of this chapter—impulse response, superposition, and convolution mediated by a Green function—reappear. The only difference is that, for the PDE, the superposition is over spatial locations (and, in more general problems with time-dependent forcing, also over time), while for the ODE it is solely over past times. This example thus previews how the Green function approach to linear dynamics extends naturally from ordinary to partial differential equations.
\end{solution}

\section{Linear Static Problems}
% --- Narrative plan (auto-generated) ---
% In this section we study linear static problems, that is, models in which all transient or time-dependent behaviors have settled and the system rests in equilibrium. Mathematically, these arise when we set time derivatives equal to zero in linear ordinary differential equations and obtain algebraic or reduced differential relations between the unknown quantities. The focus shifts from describing how solutions evolve to identifying which configurations can persist indefinitely under given forces, constraints, or inputs.
%
% Linear static problems matter throughout applied mathematics because many phenomena of interest are controlled not by how fast things change, but by how they ultimately balance. The temperature distribution in a thick wall that has been held at fixed boundary temperatures for a long time, the constant current levels in an electrical network under direct current (DC) forcing, and the deflection of a beam bolted at its ends and loaded by a static weight are all examples of linear static problems. These models are often the starting point for stability and bifurcation analysis in dynamical systems and for understanding the long-time behavior of solutions to time-dependent PDEs.
%
% Conceptually, linear static problems form a bridge between linear ODE theory, linear algebra, and the boundary value problems that appear in partial differential equations. Setting derivatives to zero reduces many dynamical equations to systems of linear equations, which brings in matrix methods, eigenvalues, and the structure of solution spaces. At the same time, steady-state or equilibrium equations for PDEs, such as Laplace’s or Poisson’s equation, share the same themes of balance and superposition that we explore here. The tools developed in this section—to identify, interpret, and compute equilibria—will reappear when we solve time-dependent ODEs, use Fourier methods, and study harmonic and potential problems in higher dimensions.

% ===== Example 1: Static equilibrium of a mass–spring system (inquiry-based) =====
\begin{problem}[Static equilibrium of a mass--spring system]
A common starting point for modeling mechanical systems is a single mass attached to one or more ideal (linear) springs. If we are only interested in the resting position under a constant force, such as gravity or a steady push, the motion eventually dies out and the system reaches a static equilibrium. In that regime, the usual differential equation from Newton's second law simplifies to an algebraic balance of forces. This example explores how that simplification works and how the resulting equation can be viewed as a very simple linear system.

Consider a block of mass $m$ that can move without friction along a horizontal line. It is attached to a rigid wall at $x=0$ by one or more ideal springs, and it is pulled to the right by a constant force $F>0$. Let $x(t)$ denote the displacement of the block from its \emph{natural} (unstretched) position at time $t$.

\smallskip

(a) Suppose first that there is a \emph{single} spring of stiffness $k>0$ between the wall and the block, and that there is also a linear damping force of the form $-b\,x'(t)$ with $b>0$. Using Newton's second law, write down the differential equation for $x(t)$ that balances all forces on the mass. State clearly which forces you include and what sign you assign to each.

\emph{Hint:} Count forces in the horizontal direction. Take displacements to the right as positive. Recall that an ideal spring of stiffness $k$ exerts a restoring force $-k\,x$ when stretched by an amount $x$.

\smallskip

(b) We now focus on the \emph{static equilibrium} of this system. In words, what does ``static equilibrium'' mean for the motion $x(t)$? Translate this verbal description into conditions on $x'(t)$ and $x''(t)$. Then impose these conditions on your differential equation from part (a) and obtain a \emph{purely algebraic} equation for the equilibrium displacement $x_\ast$.

\emph{Hint:} At equilibrium, the block is at rest and not accelerating.

\smallskip

(c) Solve the algebraic equation from part (b) to express the equilibrium displacement $x_\ast$ in terms of $F$ and $k$. What is the physical meaning of the sign of $x_\ast$? How does $x_\ast$ change if you double the stiffness $k$ while keeping $F$ fixed?

\emph{Hint:} You should obtain a linear relation between $x_\ast$ and $F$.

\smallskip

Now keep the same setup, but change the spring arrangement.

\smallskip

(d) Suppose that instead of a single spring, the block is attached to the wall by \emph{two} ideal springs in parallel, with stiffnesses $k_1>0$ and $k_2>0$. Both springs are attached between the same wall and the same block, so they stretch by the same amount $x(t)$ when the block moves.

(i) Using Hooke's law for each spring, write an expression for the force exerted by each spring on the block. Then show that the total restoring force exerted by the two springs together can be written in the form $-k_{\text{eff}}\,x$, and find the effective stiffness $k_{\text{eff}}$.

\emph{Hint:} The forces from the two springs add, because they act in the same direction on the block.

(ii) Specialize your dynamic equation from part (a) to this two-spring case (you may ignore damping if you wish), and then find the corresponding algebraic equation for the static equilibrium displacement $x_\ast$ under the constant force $F$.

(iii) Solve for $x_\ast$ in terms of $F$, $k_1$, and $k_2$. Compare it with your answer from part (c), and interpret the effect of adding an extra spring in parallel.

\smallskip

(e) Generalize the previous part to $n$ parallel springs with stiffnesses $k_1,\dots,k_n>0$ between the wall and the block.

(i) Write down the total restoring force as a function of $x$, and deduce a formula for the effective stiffness $k_{\text{eff}}$ in terms of the $k_i$.

(ii) Write the static equilibrium condition as a \emph{linear} equation of the form
\[
K\,x_\ast = F,
\]
where $K$ is a number that you should express in terms of the $k_i$. How is this a $1\times 1$ example of the matrix equation ``stiffness matrix times displacement vector equals load vector'' that appears in more complicated static problems?

\smallskip

(f) (Extensions and ``what if'' questions.)

(i) Suppose that the external force depends linearly on the displacement, so that $F_{\text{ext}} = F_0 + \alpha x$ for some constants $F_0$ and $\alpha$. How does your static equilibrium equation change in this case, and under what condition on $\alpha$ and the stiffnesses does a unique equilibrium still exist?

\emph{Hint:} Move all terms involving $x$ to the left-hand side and compare with your previous expression for $K$.

(ii) Imagine instead that the block is attached to a \emph{vertical} spring and is pulled downward by gravity $mg$ as well as by the spring. Which parts of your analysis need to be modified, and which parts stay the same? What would the static equilibrium equation look like in that case?
\end{problem}

% ===== Example 1: Static equilibrium of a mass–spring system (full solution) =====
\begin{problem}[Static equilibrium of a mass--spring system]
A block of mass $m$ moves without friction along a horizontal line and is attached to a rigid wall by two ideal springs in parallel, with stiffnesses $k_1>0$ and $k_2>0$. The block is pulled to the right by a constant horizontal force $F>0$. Let $x(t)$ be the displacement of the block from its natural (unstretched) position at time $t$, with $x>0$ to the right. 

(a) Using Newton's second law and Hooke's law, write down the differential equation governing $x(t)$ if there is also a linear damping force $-b\,x'(t)$ with $b\ge 0$.  

(b) Define what is meant by a \emph{static equilibrium} position $x_\ast$ for this system, and derive the algebraic equation that $x_\ast$ must satisfy.  

(c) Solve explicitly for $x_\ast$ in terms of $F$, $k_1$, and $k_2$, and interpret your answer in terms of an \emph{effective stiffness}.  

(d) Briefly explain how this example illustrates the general idea of a linear static problem, and how the equilibrium equation can be viewed as a $1\times 1$ linear system $Kx_\ast = F$.
\end{problem}

\begin{solution}
We begin by identifying all the forces acting on the mass and then applying Newton's second law in the horizontal direction.

\medskip

\textbf{(a) Dynamic equation.}  
The displacement $x(t)$ is measured from the position where both springs are at their natural (unstretched) length. When the block is displaced by an amount $x$, each spring stretches (or compresses) by the same amount $x$.

By Hooke's law, the restoring force from the first spring on the block is $-k_1 x$, and from the second spring is $-k_2 x$. Both forces act to the left when $x>0$. The total spring force is therefore the sum
\[
F_{\text{springs}} = -k_1 x - k_2 x = -(k_1 + k_2)\,x.
\]
The damping force, assumed linear in velocity, is $F_{\text{damp}} = -b\,x'(t)$, acting opposite to the direction of motion. The external applied force is $F_{\text{ext}} = F$, acting to the right and taken as positive.

Newton's second law says that mass times acceleration equals the sum of all forces:
\[
m\,x''(t) = F_{\text{springs}} + F_{\text{damp}} + F_{\text{ext}}.
\]
Substituting the expressions for the forces, we obtain
\[
m\,x''(t) = -(k_1 + k_2)\,x(t) - b\,x'(t) + F.
\]
Rewriting this in the standard form with all terms on the left-hand side gives
\[
m\,x''(t) + b\,x'(t) + (k_1 + k_2)\,x(t) = F.
\]
This is a linear second-order ordinary differential equation with constant coefficients and a constant right-hand side.

\medskip

\textbf{(b) Static equilibrium and the algebraic equation.}  
A \emph{static equilibrium} is a position at which the block can remain forever if placed there carefully, with no subsequent motion. In terms of $x(t)$, this means that the position is constant in time, so the velocity and acceleration both vanish:
\[
x(t) \equiv x_\ast,\qquad x'(t) = 0,\qquad x''(t) = 0.
\]

To find $x_\ast$, we substitute these conditions into the differential equation. From
\[
m\,x''(t) + b\,x'(t) + (k_1 + k_2)\,x(t) = F,
\]
we set $x''(t)=0$ and $x'(t)=0$ and replace $x(t)$ by the constant $x_\ast$. This yields
\[
0 + 0 + (k_1 + k_2)\,x_\ast = F,
\]
or simply
\[
(k_1 + k_2)\,x_\ast = F.
\]
The equilibrium position $x_\ast$ therefore satisfies a purely algebraic linear equation, obtained from the dynamic equation by dropping the time-derivative terms. Physically, this expresses the balance between the total spring force and the applied load $F$ in the absence of motion.

\medskip

\textbf{(c) Solving for the equilibrium and effective stiffness.}  
The algebraic equilibrium equation
\[
(k_1 + k_2)\,x_\ast = F
\]
is a linear equation in the unknown $x_\ast$. Since $k_1>0$ and $k_2>0$, we have $k_1 + k_2 > 0$, so there is a unique solution:
\[
x_\ast = \frac{F}{k_1 + k_2}.
\]
It is natural to interpret $k_{\text{eff}} = k_1 + k_2$ as the \emph{effective stiffness} of the two parallel springs, because the static displacement under a given load $F$ is exactly what it would be for a single spring with stiffness $k_{\text{eff}}$. In words, adding a second spring in parallel makes the combined system stiffer and therefore reduces the equilibrium displacement under the same force.

One can also see this as a linear input--output relation: the applied load $F$ is proportional to the displacement $x_\ast$, with proportionality constant $k_{\text{eff}}$. The fact that $x_\ast$ is proportional to $F$ is a hallmark of linear static behavior.

\medskip

\textbf{(d) Relation to linear static problems and a $1\times 1$ system.}  
The original dynamic model
\[
m\,x''(t) + b\,x'(t) + (k_1 + k_2)\,x(t) = F
\]
is a linear ordinary differential equation with constant coefficients and a constant right-hand side. When we restrict attention to static equilibrium, we assume that the time derivatives $x'(t)$ and $x''(t)$ are zero, leaving only the terms that do not involve derivatives:
\[
(k_1 + k_2)\,x_\ast = F.
\]
This is the simplest example of a \emph{linear static problem}: the governing differential operator is linear, and when all time derivatives are set to zero, we obtain a linear algebraic equation relating the unknown (here, the equilibrium displacement) to the data (here, the applied load and the spring parameters).

In the language of linear algebra, the static equilibrium equation can be written as
\[
K\,x_\ast = F,
\]
where $K = k_1 + k_2$ is a $1\times 1$ \emph{stiffness matrix} and $x_\ast$ and $F$ are $1\times 1$ column vectors (that is, scalars). In more complicated mechanical systems with many degrees of freedom, $K$ becomes a larger matrix, $x_\ast$ a vector of displacements, and $F$ a vector of loads, but the basic form ``stiffness times displacement equals load'' remains the same. Thus this mass--spring example provides a concrete, one-dimensional illustration of the main idea behind linear static problems: equilibrium configurations are found by solving linear algebraic equations derived from a linearized balance of forces.
\end{solution}

% ===== Example 2: Direct-current equilibrium in a resistor network (inquiry-based) =====
\begin{problem}[Direct-current equilibrium in a resistor network]
A simple direct-current (DC) circuit can be modeled as a network of ideal resistors and sources. In the static regime, after all transients have died out, the currents and voltages no longer depend on time. At this stage, Kirchhoff's current and voltage laws express conservation of charge and energy and give linear relations among the unknown node potentials and branch currents. In this problem you will set up and solve such a linear system for a small network, and then relate the result to the general idea of ``static'' solutions of differential equations.

Consider the following resistor network. There is a distinguished reference node (ground) whose potential is defined to be zero. An ideal DC voltage source of $12\ \mathrm{V}$ connects ground to a node $S$, with its positive terminal at $S$, so that $V_S = 12\ \mathrm{V}$. From node $S$ a resistor of resistance $R_1 = 6\ \Omega$ leads to node $A$. From node $A$ there are two connections: a resistor $R_2 = 3\ \Omega$ leads to node $B$, and a resistor $R_4 = 6\ \Omega$ leads to ground. Finally, from node $B$ a resistor $R_3 = 6\ \Omega$ leads to ground. All components are ideal. 

\medskip

(a) Introduce unknowns for the electrical state of the network. Take ground as the zero of potential. Let $V_A$ and $V_B$ denote the node voltages at $A$ and $B$ relative to ground. Let $I_1$, $I_2$, $I_3$, and $I_4$ denote the steady currents through $R_1$, $R_2$, $R_3$, and $R_4$, with the convention that each current flows from the higher-potential node to the lower-potential node.  

Describe in words how $V_A$, $V_B$, and the branch currents are related by Ohm's law and by Kirchhoff's current law (KCL). Which conservation principle does KCL represent in this context?

\medskip

(b) Now translate your verbal description into equations.  

\quad(i) Use Ohm's law to express each current $I_k$ in terms of the node voltages at the ends of the corresponding resistor. For example, write $I_1$ in terms of $V_S = 12\ \mathrm{V}$ and $V_A$.  

\quad(ii) Apply Kirchhoff's current law at node $A$ and at node $B$. For each node, write an equation stating that the sum of currents \emph{leaving} the node is zero.  

Hint: For node $A$, include the currents through $R_1$ (between $S$ and $A$), $R_2$ (between $A$ and $B$), and $R_4$ (between $A$ and ground). For node $B$, include the currents through $R_2$ and $R_3$.

\medskip

(c) Eliminate the currents $I_1,\dots,I_4$ from your equations so that you obtain a closed system of linear equations for the unknown node voltages $V_A$ and $V_B$ only.  

Write this system explicitly in the form
\[
A
\begin{pmatrix}
V_A \\[4pt]
V_B
\end{pmatrix}
=
\begin{pmatrix}
b_1 \\[4pt]
b_2
\end{pmatrix},
\]
and identify the $2\times 2$ coefficient matrix $A$ and the right-hand side vector $b$.  

Hint: Carefully substitute your Ohm's law expressions for the currents into the KCL equations and simplify.

\medskip

(d) Solve your $2\times 2$ linear system for $V_A$ and $V_B$. Then recover the currents $I_1,\dots,I_4$ from your Ohm's law formulas. Provide numerical values for $V_A$, $V_B$, and for each current.  

Check your work by verifying Kirchhoff's current law at node $A$ and at node $B$ using your computed currents. As an additional consistency check, compute the power delivered by the source and the total power dissipated in the four resistors, and compare them.

\medskip

(e) Explore two extensions that connect this static network problem to broader modeling ideas.

\begin{itemize}
  \item[(i)] Suppose the resistor $R_4$ between node $A$ and ground is removed from the circuit (this corresponds to replacing $R_4$ by an infinite resistance, or an ``open circuit''). How do the KCL equations at node $A$ and at node $B$ change? Without doing all the algebra, predict qualitatively how $V_A$ and $V_B$ compare to the values you found in part~(d). Do they increase, decrease, or stay the same? Explain your reasoning in terms of current paths and effective resistance.

  \item[(ii)] In more general circuits, capacitors and inductors introduce time dependence and lead to differential equations. An ideal capacitor of capacitance $C$ obeys $I_C = C\,\frac{dV}{dt}$ relating its current and voltage, and an ideal inductor of inductance $L$ obeys $V_L = L\,\frac{dI}{dt}$ relating its voltage and current. Explain, using these relations, why in a \emph{steady} DC regime (where all currents and voltages are constant in time) an ideal capacitor behaves like an open circuit and an ideal inductor behaves like a short circuit. How does this observation justify ignoring capacitors and inductors in the static analysis you performed above?
\end{itemize}

\end{problem}

% ===== Example 2: Direct-current equilibrium in a resistor network (full solution) =====
\begin{problem}[Direct-current equilibrium in a resistor network]
Consider a DC resistor network with a reference node (ground, $0\ \mathrm{V}$) and an ideal $12\ \mathrm{V}$ voltage source whose positive terminal is at node $S$, so that $V_S = 12\ \mathrm{V}$. From $S$ a resistor $R_1 = 6\ \Omega$ leads to node $A$. From $A$ a resistor $R_2 = 3\ \Omega$ leads to node $B$, and another resistor $R_4 = 6\ \Omega$ leads to ground. From $B$ a resistor $R_3 = 6\ \Omega$ leads to ground. All components are ideal, and the circuit has reached a time-independent (DC) equilibrium.

Using node-voltage analysis:
\begin{enumerate}
  \item Set up the Kirchhoff current law (KCL) equations at nodes $A$ and $B$ in terms of the node voltages $V_A$ and $V_B$.
  \item Reduce these equations to a $2\times 2$ linear system for $(V_A, V_B)$ and solve for $V_A$ and $V_B$.
  \item Compute the steady currents through each resistor and the current supplied by the source. Verify KCL at both nodes and show that the power delivered by the source equals the total power dissipated in the resistors.
\end{enumerate}
\end{problem}

\begin{solution}
We analyze the network using the node-voltage method with ground as the reference. The unknown node potentials are $V_A$ and $V_B$ (relative to ground). The source node has fixed potential $V_S = 12\ \mathrm{V}$.

\medskip

\textbf{Step 1: Express currents using Ohm's law.}

Each branch current is determined by the voltage difference across the corresponding resistor. With the convention that current flows from higher potential to lower potential, we have:
\begin{align*}
I_1 &= \frac{V_S - V_A}{R_1} = \frac{12 - V_A}{6}, &&\text{(from $S$ to $A$ through $R_1$)}, \\
I_2 &= \frac{V_A - V_B}{R_2} = \frac{V_A - V_B}{3}, &&\text{(from $A$ to $B$ through $R_2$)}, \\
I_3 &= \frac{V_B - 0}{R_3} = \frac{V_B}{6}, &&\text{(from $B$ to ground through $R_3$)}, \\
I_4 &= \frac{V_A - 0}{R_4} = \frac{V_A}{6}, &&\text{(from $A$ to ground through $R_4$)}.
\end{align*}
These are direct applications of Ohm's law, $I = \dfrac{V_\text{drop}}{R}$.

\medskip

\textbf{Step 2: Apply Kirchhoff's current law at each node.}

Kirchhoff's current law (KCL) is a mathematical statement of conservation of charge at each node: in the steady state, no net charge accumulates, so the algebraic sum of currents leaving a node is zero.

\emph{Node $A$.} Currents can leave node $A$ along three paths:
\begin{itemize}
  \item through $R_1$ toward node $S$, with current $(V_A - V_S)/R_1 = (V_A - 12)/6$,
  \item through $R_2$ toward node $B$, with current $(V_A - V_B)/3$,
  \item through $R_4$ toward ground, with current $V_A/6$.
\end{itemize}
Summing these currents and setting the total to zero gives
\[
\frac{V_A - 12}{6} + \frac{V_A - V_B}{3} + \frac{V_A}{6} = 0.
\]

\emph{Node $B$.} Currents can leave node $B$ along two paths:
\begin{itemize}
  \item through $R_2$ toward node $A$, with current $(V_B - V_A)/3$,
  \item through $R_3$ toward ground, with current $V_B/6$.
\end{itemize}
Thus KCL at $B$ gives
\[
\frac{V_B - V_A}{3} + \frac{V_B}{6} = 0.
\]

These two equations express charge conservation at the internal nodes of the network.

\medskip

\textbf{Step 3: Simplify to a $2\times 2$ linear system.}

We now simplify the KCL equations to obtain a standard linear system in $V_A$ and $V_B$.

For node $A$:
\[
\frac{V_A - 12}{6} + \frac{V_A - V_B}{3} + \frac{V_A}{6} = 0.
\]
Multiply by $6$ to clear denominators:
\[
(V_A - 12) + 2(V_A - V_B) + V_A = 0.
\]
Collect terms:
\[
V_A - 12 + 2V_A - 2V_B + V_A = 0
\quad\Longrightarrow\quad
4V_A - 2V_B = 12.
\]

For node $B$:
\[
\frac{V_B - V_A}{3} + \frac{V_B}{6} = 0.
\]
Multiply by $6$:
\[
2(V_B - V_A) + V_B = 0
\quad\Longrightarrow\quad
2V_B - 2V_A + V_B = 0
\quad\Longrightarrow\quad
-2V_A + 3V_B = 0.
\]

Thus, in matrix form,
\[
\begin{pmatrix}
4 & -2 \\[4pt]
-2 & 3
\end{pmatrix}
\begin{pmatrix}
V_A \\[4pt]
V_B
\end{pmatrix}
=
\begin{pmatrix}
12 \\[4pt]
0
\end{pmatrix}.
\]
This is a symmetric positive definite system characteristic of many linear static equilibrium problems (including resistor networks and discretized elliptic equations).

\medskip

\textbf{Step 4: Solve for the node voltages.}

We solve the system
\[
\begin{cases}
4V_A - 2V_B = 12, \\
-2V_A + 3V_B = 0.
\end{cases}
\]

From the second equation,
\[
-2V_A + 3V_B = 0
\quad\Longrightarrow\quad
3V_B = 2V_A
\quad\Longrightarrow\quad
V_B = \dfrac{2}{3} V_A.
\]

Substitute this expression into the first equation:
\[
4V_A - 2\left(\frac{2}{3}V_A\right) = 12
\quad\Longrightarrow\quad
4V_A - \frac{4}{3}V_A = 12
\quad\Longrightarrow\quad
\left(\frac{12}{3} - \frac{4}{3}\right)V_A = 12
\quad\Longrightarrow\quad
\frac{8}{3}V_A = 12.
\]
Hence
\[
V_A = 12\cdot \frac{3}{8} = \frac{36}{8} = 4.5\ \mathrm{V}.
\]
Then
\[
V_B = \frac{2}{3}V_A = \frac{2}{3}\cdot 4.5 = 3.0\ \mathrm{V}.
\]

So the node voltages are
\[
V_A = 4.5\ \mathrm{V},
\qquad
V_B = 3.0\ \mathrm{V}.
\]

\medskip

\textbf{Step 5: Compute currents and verify KCL.}

Using the previously derived Ohm's law expressions, we obtain:
\begin{align*}
I_1 &= \frac{12 - V_A}{6}
= \frac{12 - 4.5}{6}
= \frac{7.5}{6} = 1.25\ \mathrm{A}, \\
I_2 &= \frac{V_A - V_B}{3}
= \frac{4.5 - 3.0}{3}
= \frac{1.5}{3} = 0.50\ \mathrm{A}, \\
I_3 &= \frac{V_B}{6}
= \frac{3.0}{6}
= 0.50\ \mathrm{A}, \\
I_4 &= \frac{V_A}{6}
= \frac{4.5}{6}
= 0.75\ \mathrm{A}.
\end{align*}

We now check KCL numerically.

At node $A$, the currents \emph{leaving} are:
\[
I_{A\to S} = \frac{V_A - 12}{6} = -1.25\ \mathrm{A},
\quad
I_{A\to B} = 0.50\ \mathrm{A},
\quad
I_{A\to \text{ground}} = 0.75\ \mathrm{A}.
\]
Summing gives
\[
(-1.25) + 0.50 + 0.75 = 0,
\]
so KCL holds at node $A$. Interpreted physically, $1.25\ \mathrm{A}$ enters node $A$ from the source, and the same total current $0.50 + 0.75 = 1.25\ \mathrm{A}$ leaves toward node $B$ and ground.

At node $B$, the currents leaving are:
\[
I_{B\to A} = \frac{V_B - V_A}{3} = -0.50\ \mathrm{A},
\quad
I_{B\to \text{ground}} = 0.50\ \mathrm{A}.
\]
Thus
\[
(-0.50) + 0.50 = 0,
\]
so KCL holds at node $B$ as well. Here, $0.50\ \mathrm{A}$ enters node $B$ from node $A$ and the same amount flows to ground.

The source current is precisely $I_1 = 1.25\ \mathrm{A}$, since the only connection from node $S$ to the rest of the circuit is through $R_1$.

\medskip

\textbf{Step 6: Check power balance (energy conservation).}

The power delivered by the ideal voltage source is
\[
P_\text{source} = V_S \cdot I_\text{source} = 12\ \mathrm{V} \cdot 1.25\ \mathrm{A} = 15\ \mathrm{W}.
\]

We now compute the power dissipated in each resistor using $P = I^2 R$:
\begin{align*}
P_{R_1} &= I_1^2 R_1 = (1.25)^2 \cdot 6 = 1.5625 \cdot 6 = 9.375\ \mathrm{W}, \\
P_{R_2} &= I_2^2 R_2 = (0.50)^2 \cdot 3 = 0.25 \cdot 3 = 0.75\ \mathrm{W}, \\
P_{R_3} &= I_3^2 R_3 = (0.50)^2 \cdot 6 = 0.25 \cdot 6 = 1.5\ \mathrm{W}, \\
P_{R_4} &= I_4^2 R_4 = (0.75)^2 \cdot 6 = 0.5625 \cdot 6 = 3.375\ \mathrm{W}.
\end{align*}
The total resistive power dissipation is
\[
P_{R_1} + P_{R_2} + P_{R_3} + P_{R_4}
= 9.375 + 0.75 + 1.5 + 3.375 = 15\ \mathrm{W}.
\]

Thus
\[
P_\text{source} = 15\ \mathrm{W} = P_{R_1} + P_{R_2} + P_{R_3} + P_{R_4},
\]
which is a statement of energy conservation in the static regime: all power delivered by the source is dissipated as heat in the resistors.

\medskip

\textbf{Conceptual remark: relation to linear static problems.}

This example illustrates the idea of a linear static problem as used in the study of ordinary differential equations. A more complete dynamical model of an electrical circuit with capacitors and inductors leads to a system of linear ODEs in time for the node voltages and branch currents. The DC equilibrium corresponds to a \emph{steady state} of that system, where all time derivatives vanish. Algebraically, setting the derivatives to zero in the ODE system yields exactly a linear system of the form
\[
A\,u = b,
\]
where $u$ collects the steady voltages and currents. Our resistor network problem is precisely such a static equilibrium: the matrix
\[
A =
\begin{pmatrix}
4 & -2 \\
-2 & 3
\end{pmatrix}
\]
encodes the network connectivity and conductances, and solving $A u = b$ gives the unique DC state consistent with charge and energy conservation. This is typical of linear static problems arising as the equilibrium limit of more general dynamical systems.

\end{solution}

% ===== Example 3: Steady concentrations in a simple reaction network (inquiry-based) =====
\begin{problem}[Steady concentrations in a simple reaction network]
In a well-mixed chemical reactor, several species may be created, transformed into one another, and removed. When the reactor has been running for a long time under constant conditions, the concentrations may approach a steady state in which they no longer change in time. Mathematically, this steady state corresponds to setting all time derivatives equal to zero in the governing differential equations. The resulting equations form a linear \emph{static} problem, which we can analyze using tools from linear algebra.

Consider three chemical species $A$, $B$, and $C$ with time-dependent concentrations $a(t)$, $b(t)$, and $c(t)$ (for $t \ge 0$). The reactor is fed from an external reservoir that produces species $A$ at a constant rate $\alpha > 0$ (for instance, in units of moles per liter per second). Inside the reactor, the following reactions occur:
\[
A \xrightarrow{k_1} B,\qquad
B \xrightarrow{k_2} C,\qquad
C \xrightarrow{k_3} \varnothing,
\]
where $k_1, k_2, k_3 > 0$ are rate constants, and $\varnothing$ denotes removal of $C$ from the system (for example, by decay or outflow).

\smallskip

(a) Using the usual well-mixed, linear mass-action assumptions, write down the system of ordinary differential equations satisfied by $a(t)$, $b(t)$, and $c(t)$.  
Explain in words how you obtain each term in each equation from the reaction network.

% Hint: Each reaction contributes a term proportional to the concentration of the reactant. Inflow adds a positive constant term; conversion or decay removes mass from one species and may add it to another.

\smallskip

(b) Rewrite your system compactly in matrix form
\[
\mathbf{x}'(t) = M\,\mathbf{x}(t) + \mathbf{f},
\]
where $\mathbf{x}(t) = \begin{pmatrix} a(t) \\[2pt] b(t) \\[2pt] c(t) \end{pmatrix}$.  
Identify the $3\times 3$ matrix $M$ and the vector $\mathbf{f}$ explicitly.

% Hint: The entries of $M$ come from coefficients multiplying $a$, $b$, and $c$ in your equations; the constant inflow $\alpha$ becomes part of $\mathbf{f}$.

\smallskip

(c) A steady state (or equilibrium) is a vector $\mathbf{x}^* = \begin{pmatrix} a^* \\ b^* \\ c^* \end{pmatrix}$ such that the concentrations do not change in time when $\mathbf{x}(t) \equiv \mathbf{x}^*$.  
Derive the algebraic system of equations that $\mathbf{x}^*$ must satisfy.  
Express this system both componentwise and in matrix form.

% Hint: Set $\mathbf{x}'(t) = \mathbf{0}$ in your matrix equation and simplify.

\smallskip

(d) Solve the steady-state equations you found in part (c) and obtain explicit formulas for $a^*$, $b^*$, and $c^*$ in terms of $\alpha$, $k_1$, $k_2$, and $k_3$.  
Explain briefly why your solution is unique in this model.  
Interpret your formulas: what simple pattern do you notice relating each steady concentration to the corresponding rate constant?

% Hint: The system is triangular: $a^*$ appears only in the first equation, then $b^*$ in terms of $a^*$, and so on. For uniqueness, think about whether the linear system $M \mathbf{x}^* + \mathbf{f} = \mathbf{0}$ can have more than one solution when $k_1,k_2,k_3>0$.

\smallskip

(e) Explore two ``what if'' modifications of the model.

\quad (i) Suppose the removal of $C$ is very slow or absent, so that $k_3 = 0$ while $\alpha > 0$ and $k_1,k_2>0$ remain fixed.  
Write down the new steady-state equations and discuss whether a steady state with finite concentrations $a^*,b^*,c^*$ can exist. How does this reflect the physical behavior of the reactor?

\quad (ii) Suppose instead that there is no external feed, so $\alpha = 0$, while all rate constants $k_1,k_2,k_3$ are positive.  
Determine all steady states of the system in this case.  
How does this differ qualitatively from the case $\alpha > 0$?

% Hint: For (i), think about the equation that previously defined $c^*$. For (ii), you now have a homogeneous linear system $M \mathbf{x}^* = \mathbf{0}$.

\end{problem}

% ===== Example 3: Steady concentrations in a simple reaction network (full solution) =====
\begin{problem}[Steady concentrations in a simple reaction network]
In a well-mixed reactor, three chemical species $A$, $B$, and $C$ have concentrations $a(t)$, $b(t)$, and $c(t)$, respectively. Species $A$ is supplied from an external source at a constant rate $\alpha > 0$, and the species undergo the linear reactions
\[
A \xrightarrow{k_1} B,\qquad
B \xrightarrow{k_2} C,\qquad
C \xrightarrow{k_3} \varnothing,
\]
with rate constants $k_1, k_2, k_3 > 0$.  

(a) Derive the system of linear ODEs for $a(t)$, $b(t)$, and $c(t)$, and write it in matrix form $\mathbf{x}'(t) = M \mathbf{x}(t) + \mathbf{f}$, where $\mathbf{x}(t) = (a(t), b(t), c(t))^{\mathsf{T}}$.  

(b) Find the steady-state concentrations $a^*$, $b^*$, and $c^*$ for $\alpha > 0$ and $k_1,k_2,k_3>0$, and show that the steady state is unique.  

(c) Discuss what happens to the existence of a finite steady state if $k_3 = 0$ while $\alpha>0$ remains fixed.

\end{problem}

\begin{solution}
We begin by translating the reaction description into differential equations. Under the standard well-mixed, linear mass-action assumption, each reaction contributes a term proportional to the concentration of its reactant.

\medskip

\textbf{(a) Derivation of the ODE system and matrix form.}

First consider species $A$.  
There is a constant inflow at rate $\alpha$, which adds the term $+\alpha$.  
Species $A$ is lost only through the reaction $A \to B$ at rate $k_1 a(t)$, which contributes the term $-k_1 a(t)$.  
Thus
\[
a'(t) = \alpha - k_1 a(t).
\]

Next consider species $B$.  
It is produced from $A$ at rate $k_1 a(t)$, and it is consumed in the reaction $B \to C$ at rate $k_2 b(t)$.  
Therefore
\[
b'(t) = k_1 a(t) - k_2 b(t).
\]

Finally consider species $C$.  
It is produced from $B$ at rate $k_2 b(t)$ and is removed from the system at rate $k_3 c(t)$ by the reaction $C \to \varnothing$.  
Hence
\[
c'(t) = k_2 b(t) - k_3 c(t).
\]

Collecting these, we obtain the linear system of ODEs
\[
\begin{cases}
a'(t) = \alpha - k_1 a(t),\\[4pt]
b'(t) = k_1 a(t) - k_2 b(t),\\[4pt]
c'(t) = k_2 b(t) - k_3 c(t).
\end{cases}
\]

To write this in matrix form, we set
\[
\mathbf{x}(t) = \begin{pmatrix} a(t) \\[2pt] b(t) \\[2pt] c(t) \end{pmatrix}.
\]
Then we can read off the coefficients of $a(t)$, $b(t)$, and $c(t)$ in each equation. The terms involving the concentrations form the matrix $M$, and the constant inflow forms the vector $\mathbf{f}$:
\[
M =
\begin{pmatrix}
-\,k_1 & 0 & 0 \\
k_1 & -\,k_2 & 0 \\
0 & k_2 & -\,k_3
\end{pmatrix},
\qquad
\mathbf{f} =
\begin{pmatrix}
\alpha \\[2pt] 0 \\[2pt] 0
\end{pmatrix}.
\]
Thus the system can be written compactly as
\[
\mathbf{x}'(t) = M\,\mathbf{x}(t) + \mathbf{f}.
\]

This form makes it clear that we are dealing with a linear, nonhomogeneous system of ODEs.

\medskip

\textbf{(b) Steady-state concentrations and uniqueness.}

A steady state (or equilibrium) is a time-independent solution $\mathbf{x}(t) \equiv \mathbf{x}^*$ for which $\mathbf{x}'(t) = \mathbf{0}$. Substituting $\mathbf{x}(t) = \mathbf{x}^*$ into the matrix equation gives
\[
\mathbf{0} = M\,\mathbf{x}^* + \mathbf{f},
\]
or equivalently
\[
M\,\mathbf{x}^* = -\,\mathbf{f}.
\]
In component form, this means that $a^*$, $b^*$, and $c^*$ must satisfy
\[
\begin{cases}
0 = \alpha - k_1 a^*,\\[4pt]
0 = k_1 a^* - k_2 b^*,\\[4pt]
0 = k_2 b^* - k_3 c^*.
\end{cases}
\]

This system is triangular, so we can solve it sequentially.

From the first equation we obtain
\[
\alpha - k_1 a^* = 0
\quad\Longrightarrow\quad
a^* = \dfrac{\alpha}{k_1}.
\]

Substituting this into the second equation gives
\[
k_1 a^* - k_2 b^* = 0
\quad\Longrightarrow\quad
k_1 \left(\dfrac{\alpha}{k_1}\right) - k_2 b^* = 0
\quad\Longrightarrow\quad
b^* = \dfrac{\alpha}{k_2}.
\]

Finally, substituting $b^*$ into the third equation yields
\[
k_2 b^* - k_3 c^* = 0
\quad\Longrightarrow\quad
k_2 \left(\dfrac{\alpha}{k_2}\right) - k_3 c^* = 0
\quad\Longrightarrow\quad
c^* = \dfrac{\alpha}{k_3}.
\]

Therefore, for $\alpha > 0$ and $k_1,k_2,k_3>0$, the unique steady state is
\[
\boxed{
a^* = \dfrac{\alpha}{k_1}, \quad
b^* = \dfrac{\alpha}{k_2}, \quad
c^* = \dfrac{\alpha}{k_3}.
}
\]

A simple pattern emerges: each steady concentration equals the input rate $\alpha$ divided by the effective removal rate for that species. Although species $A$ and $B$ are removed only by conversion and not by direct decay, in steady state the conversion acts exactly like a removal process with rate constants $k_1$ and $k_2$, respectively.

To see why this steady state is unique, observe that $M$ is a lower triangular matrix with diagonal entries $-k_1$, $-k_2$, and $-k_3$. Since each $k_i>0$, none of the diagonal entries is zero, so
\[
\det M = (-k_1)(-k_2)(-k_3) \neq 0.
\]
Thus $M$ is invertible, and the linear system $M \mathbf{x}^* = -\mathbf{f}$ has exactly one solution. In terms of the broader theme of \emph{linear static problems}, we see that the steady state is found by solving a linear algebraic system, and invertibility of the coefficient matrix guarantees a unique equilibrium.

\medskip

\textbf{(c) Effect of removing the sink for $C$ ($k_3 = 0$).}

Now suppose that $k_3 = 0$ while $\alpha > 0$ and $k_1,k_2>0$ remain fixed. Physically, this means that the final species $C$ is no longer removed from the system. The ODE for $c(t)$ becomes
\[
c'(t) = k_2 b(t) - 0\cdot c(t) = k_2 b(t).
\]
The steady-state equations are then
\[
\begin{cases}
0 = \alpha - k_1 a^*,\\[4pt]
0 = k_1 a^* - k_2 b^*,\\[4pt]
0 = k_2 b^*.
\end{cases}
\]

From the third equation we obtain $k_2 b^* = 0$, so $b^* = 0$ because $k_2>0$.  
However, the second equation says $k_1 a^* - k_2 b^* = 0$, hence $k_1 a^* = k_2 b^* = 0$, so $a^* = 0$ as well.  
This contradicts the first equation, which requires
\[
0 = \alpha - k_1 a^* = \alpha - 0 = \alpha > 0.
\]
Therefore, there is no triple $(a^*,b^*,c^*)$ with finite components that satisfies all three equations simultaneously when $k_3 = 0$ and $\alpha > 0$.

In matrix terms, when $k_3 = 0$ the coefficient matrix $M$ loses invertibility: its last diagonal entry is zero, and $\det M = 0$. The linear static problem $M \mathbf{x}^* = -\mathbf{f}$ has no solution. Physically, if there is a constant inflow of material at rate $\alpha$ and no ultimate sink for $C$, then mass continually accumulates in the reactor, and the system cannot reach a time-independent steady state with bounded concentrations.

\medskip

\textbf{Connection to linear static problems.}

This example illustrates the general principle that steady states of linear ODE systems
\[
\mathbf{x}'(t) = A \mathbf{x}(t) + \mathbf{b}
\]
are obtained by solving the \emph{static} linear system
\[
A \mathbf{x}^* + \mathbf{b} = 0.
\]
Here the reaction network yields a specific matrix $M$ and source vector $\mathbf{f}$, and the existence and uniqueness of the steady state are governed by algebraic properties of $M$ (in particular, whether $M$ is invertible). Thus, analyzing equilibria of dynamical systems naturally leads to linear static problems in linear algebra.

\end{solution}

% ===== Example 4: Static temperature profile in a one-dimensional bar (inquiry-based) =====
\begin{problem}[Static temperature profile in a one-dimensional bar]
Consider a long, thin, homogeneous metal bar of length $L$. Its lateral surface is perfectly insulated, so that heat can flow only along the length of the bar. The left end of the bar is held at a fixed temperature $T_0$ and the right end at a fixed temperature $T_L$. After a long time, the temperature no longer changes in time and the bar reaches a \emph{steady state}. In this regime, the temperature depends only on the spatial coordinate $x$ along the bar and satisfies a simpler static equation.

We will derive and solve the resulting ordinary differential equation for the steady-state temperature $T(x)$.

\medskip

(a) Let $u(x,t)$ denote the temperature at position $x\in[0,L]$ and time $t\ge 0$. The one-dimensional heat equation with constant thermal diffusivity $\kappa>0$ is
\[
u_t(x,t) \;=\; \kappa\, u_{xx}(x,t), \qquad 0<x<L,\ t>0.
\]
Explain in words what it means for the bar to be in a \emph{steady state}. Translate this into a mathematical condition on $u(x,t)$, and use it to derive a differential equation satisfied by the steady-state temperature $T(x)$.

% Hint: In steady state, the temperature no longer changes in time at any point.

\medskip

(b) You should have found that the steady-state temperature $T(x)$ satisfies a second-order ordinary differential equation of the form
\[
T''(x) = 0.
\]
Solve this ordinary differential equation explicitly. Write down the most general twice-differentiable function $T(x)$ whose second derivative is identically zero on the interval $[0,L]$.

Hint: If the second derivative of a function is zero everywhere, what does that say about its first derivative? And what does that say about the function itself?

\medskip

(c) Now incorporate the physical boundary conditions. The ends of the bar are held at fixed temperatures
\[
T(0) = T_0, \qquad T(L) = T_L,
\]
where $T_0$ and $T_L$ are given constants. Use these boundary conditions to determine the constants in your general solution from part (b), and obtain an explicit formula for $T(x)$ in terms of $T_0$, $T_L$, $x$, and $L$.

% Hint: You will obtain two linear equations for the two unknown constants.

\medskip

(d) Show that your formula can be written in the form
\[
T(x) \;=\; T_0 \left(1 - \frac{x}{L}\right) \;+\; T_L \frac{x}{L}.
\]
Interpret this expression in words. In particular:
\begin{itemize}
    \item How does the temperature vary along the bar?
    \item What is the temperature at the midpoint $x = L/2$ in terms of $T_0$ and $T_L$?
\end{itemize}

Hint: Rewrite your solution so that at $x=0$ you obtain $T_0$ and at $x=L$ you obtain $T_L$ by inspection, and notice that the coefficients of $T_0$ and $T_L$ add up to $1$.

\medskip

(e) (Exploration and extensions.)

\begin{enumerate}
    \item Suppose now that $T_0 = T_L = T^\ast$ for some fixed temperature $T^\ast$. What is the steady-state temperature profile $T(x)$ in this case? Is this consistent with your physical intuition? Briefly explain.
    \item Imagine instead that there is a uniform internal heat source in the bar, so that the steady-state equation becomes
    \[
    -k\, T''(x) = q, \qquad 0<x<L,
    \]
    where $k>0$ is the thermal conductivity and $q>0$ is the (constant) rate of heat generation per unit length. Using only qualitative reasoning (you do not need to solve this equation completely), describe the \emph{shape} of the steady-state temperature profile in this case. Will it be linear, concave up, or concave down? How does this relate to the sign of $T''(x)$?
    % Hint: If $T''$ is a positive constant, think about the graph of a quadratic function.
\end{enumerate}

\end{problem}

% ===== Example 4: Static temperature profile in a one-dimensional bar (full solution) =====
\begin{problem}[Static temperature profile in a one-dimensional bar]
A homogeneous bar of length $L$ is insulated on its sides so that heat flows only along its length. The left end at $x=0$ is held at temperature $T_0$ and the right end at $x=L$ is held at temperature $T_L$. In steady state, the temperature $T(x)$ along the bar satisfies
\[
T''(x) = 0, \qquad 0<x<L,
\]
with boundary conditions $T(0) = T_0$ and $T(L) = T_L$.

(a) Solve this boundary value problem and obtain an explicit formula for $T(x)$.

(b) Briefly describe the qualitative behavior of this steady-state temperature profile and how it reflects the nature of a linear static problem.
\end{problem}

\begin{solution}
The situation described is a prototypical \emph{linear static problem}. The temperature no longer depends on time, and the governing equation reduces to a linear ordinary differential equation with boundary conditions prescribed at the ends of the interval.

\medskip

\noindent\textbf{(a) Solving the boundary value problem.}

We are given the second-order linear ordinary differential equation
\[
T''(x) = 0,\qquad 0<x<L.
\]
This equation is especially simple because the second derivative of $T$ is identically zero. Integrating once with respect to $x$ shows that the first derivative must be constant:
\[
T''(x) = 0 \quad \Longrightarrow \quad T'(x) = C_1
\]
for some constant $C_1$. Integrating a second time gives
\[
T(x) = C_1 x + C_2,
\]
where $C_2$ is another constant. Thus every twice-differentiable solution of $T''(x)=0$ on $[0,L]$ is an affine (that is, linear plus constant) function of $x$.

We now apply the boundary conditions to determine $C_1$ and $C_2$. The left boundary condition $T(0)=T_0$ gives
\[
T(0) = C_1\cdot 0 + C_2 = C_2 = T_0,
\]
so $C_2 = T_0$. The right boundary condition $T(L)=T_L$ gives
\[
T(L) = C_1 L + C_2 = C_1 L + T_0 = T_L,
\]
hence
\[
C_1 = \frac{T_L - T_0}{L}.
\]
Substituting these constants into the general solution yields
\[
T(x) = \frac{T_L - T_0}{L}\,x + T_0.
\]

It is often convenient to rewrite this in a symmetric form that highlights the contribution of each boundary temperature. We can write
\[
T(x)
= T_0 + \frac{T_L - T_0}{L}\,x
= T_0\left(1 - \frac{x}{L}\right) + T_L \frac{x}{L}.
\]
This expression makes it clear that $T(0)=T_0$ and $T(L)=T_L$, and that for each $x$ the temperature is a weighted average of the two end temperatures.

\medskip

\noindent\textbf{(b) Qualitative behavior and relation to linear static problems.}

From the formula
\[
T(x) = T_0\left(1 - \frac{x}{L}\right) + T_L \frac{x}{L},
\]
we see that $T(x)$ varies \emph{linearly} from $T_0$ at $x=0$ to $T_L$ at $x=L$. In particular, the temperature at the midpoint is the arithmetic mean of the end temperatures:
\[
T\!\left(\frac{L}{2}\right)
= T_0\left(1 - \frac{1}{2}\right) + T_L \frac{1}{2}
= \frac{T_0 + T_L}{2}.
\]
The constant derivative
\[
T'(x) = \frac{T_L - T_0}{L}
\]
means that the temperature gradient, and hence the heat flux in Fourier's law, is uniform along the bar in steady state. Physically, the same amount of heat per unit time flows through every cross-section, so there is no accumulation of heat inside the bar.

This example illustrates several key features of linear static problems:
\begin{itemize}
    \item A time-dependent partial differential equation (the heat equation) reduces in steady state to a spatial ordinary differential equation, here $T''(x)=0$.
    \item The resulting equation is linear with constant coefficients, leading to a simple general solution determined by integration and then fixed by boundary conditions.
    \item The boundary value nature of the problem is essential: the two end temperatures uniquely determine the linear profile in between.
\end{itemize}
More complicated static problems in this chapter will involve higher-order or variable-coefficient linear equations and more intricate boundary conditions, but the basic structure is the same: a linear differential operator acting on an unknown function, together with physical boundary data that determine a unique static solution.
\end{solution}

% ===== Example 5: Static deflection of an elastic beam under uniform load (inquiry-based) =====
\begin{problem}[Static deflection of an elastic beam under uniform load]
In elementary beam theory (Euler--Bernoulli theory), the transverse deflection $u(x)$ of a slender elastic beam is modeled by an ordinary differential equation. The variable $x$ denotes position along the beam, while $u(x)$ measures the vertical displacement of the centerline. When the beam is loaded by a uniform load, the internal bending moment and shear force balance the external loading in static equilibrium. This leads to a linear, fourth-order boundary-value problem for $u(x)$.

Consider a straight, horizontal beam of length $L$, with constant bending stiffness $EI$ (where $E$ is Young's modulus and $I$ is the second moment of area of the cross section). The beam is subjected to a uniform distributed load of intensity $q>0$ (force per unit length), acting downward along its entire length $0<x<L$. We choose the sign convention so that downward deflections $u(x)$ are positive.

\smallskip

(a) Let $M(x)$ denote the internal bending moment and $V(x)$ the internal shear force in the beam at position $x$. In statics, the balance of forces and moments on a small slice of the beam leads to the one-dimensional equilibrium equations
\[
V'(x) + q = 0, 
\qquad
M'(x) = V(x).
\]
Explain in words what each of these equations means physically, and why the signs are chosen as written. 

\smallskip

(b) In Euler--Bernoulli beam theory, the curvature of the beam is proportional to the bending moment:
\[
M(x) = -EI\,u''(x),
\]
where primes denote derivatives with respect to $x$. 

\begin{itemize}
  \item[(i)] Differentiate the relation $M(x) = -EI\,u''(x)$ once to express $V(x)$ in terms of $u'''(x)$. 
  \item[(ii)] Then use the force balance equation for $V'(x)$ to derive a single differential equation involving only $u$ (and its derivatives) and the given load $q$. Simplify this equation as much as possible.
\end{itemize}

Hint: You should arrive at a fourth-order linear ODE for $u(x)$ with a constant right-hand side.

\smallskip

(c) Now solve the differential equation you obtained in part (b), without yet imposing boundary conditions. 

\begin{itemize}
  \item[(i)] Integrate the equation step by step to obtain an explicit expression for $u(x)$ as a polynomial in $x$, containing four unknown constants of integration $C_1,C_2,C_3,C_4$.
  \item[(ii)] Write your answer in the form
  \[
  u(x) = A x^4 + B x^3 + C x^2 + D x + E
  \]
  and determine $A,B,C,D,E$ in terms of $q$, $EI$, and the constants of integration.
\end{itemize}

Hint: Because the load $q$ is constant, each integration increases the degree of the polynomial by one.

\smallskip

(d) Suppose the beam is \emph{simply supported} at both ends. This means that the displacement is zero at the ends and the bending moment vanishes at the supports. Mathematically, the boundary conditions are
\[
u(0) = 0, 
\quad u(L) = 0,
\quad u''(0) = 0,
\quad u''(L) = 0.
\]
\begin{itemize}
  \item[(i)] Impose these four boundary conditions on your general polynomial solution from part (c), and solve for the four constants of integration.
  \item[(ii)] Write the resulting \emph{particular} solution $u(x)$, simplified as much as you can. 
  \item[(iii)] By differentiating your final expression, find the location $x_{\max}$ and value $u_{\max}$ of the maximum deflection of the beam.
\end{itemize}

Hint: It may be helpful to notice that the loading and the boundary conditions are symmetric with respect to the midpoint $x = L/2$.

\smallskip

(e) Extensions and variations.

\begin{itemize}
  \item[(i)] Suppose instead that the beam is \emph{clamped} (built-in) at both ends. In this case, both the displacement and the slope vanish at $x=0$ and $x=L$, so
  \[
  u(0)=u'(0)=u(L)=u'(L)=0.
  \]
  Without doing all algebraic steps in detail, outline how you would solve for $u(x)$ in this case. Which parts of your work from the simply supported case still apply, and where do the calculations differ?

  \item[(ii)] Finally, think conceptually: how would the differential equation change if the load were not constant, but a prescribed function $q(x)$? How would it change if the stiffness $EI$ varied with $x$? Describe in words what remains linear and what becomes more complicated in the mathematical model.
\end{itemize}

\end{problem}

% ===== Example 5: Static deflection of an elastic beam under uniform load (full solution) =====
\begin{problem}[Static deflection of an elastic beam under uniform load]
A prismatic Euler--Bernoulli beam of length $L$ has constant bending stiffness $EI$ and is subjected to a uniform distributed load of intensity $q>0$ (force per unit length) acting downward along $0<x<L$. Let $u(x)$ denote the vertical deflection of the beam, measured positive downward. 

\begin{enumerate}
  \item Using the relations
  \[
  V'(x) + q = 0, \qquad M'(x)=V(x), \qquad M(x) = -EI\,u''(x),
  \]
  derive a single differential equation for $u(x)$.
  \item Solve this equation for a \emph{simply supported} beam, for which
  \[
  u(0)=u(L)=0, \qquad u''(0)=u''(L)=0.
  \]
  Find the resulting deflection $u(x)$ explicitly.
  \item Determine the location and magnitude of the maximum deflection.  
\end{enumerate}
\end{problem}

\begin{solution}
We proceed in three steps: derivation of the governing ordinary differential equation, solution of the boundary-value problem for the simply supported beam, and determination of the point of maximum deflection. Throughout we emphasize that this is a linear static boundary-value problem of fourth order.

\medskip

\textbf{1. Derivation of the governing differential equation.}

The internal shear force $V(x)$ and bending moment $M(x)$ in a slender beam in static equilibrium satisfy the one-dimensional balance laws
\[
V'(x) + q = 0, 
\qquad 
M'(x) = V(x),
\]
where $q>0$ is the constant distributed load (force per unit length). The first equation states that the change in internal shear along the beam balances the applied distributed load; the second states that the change in internal moment equals the shear force.

In Euler--Bernoulli theory, the curvature of the deflected centerline is proportional to the bending moment,
\[
M(x) = -EI\,u''(x),
\]
where $E$ is Young's modulus, $I$ is the second moment of area, and $u''(x)$ is the second derivative of the deflection $u(x)$ with respect to $x$. The minus sign reflects the convention that a positive bending moment produces a concave-up shape (negative second derivative) when downward deflections are taken as positive.

Differentiating the constitutive relation once gives
\[
M'(x) = -EI\,u'''(x).
\]
But from equilibrium we also have $M'(x) = V(x)$, so
\[
V(x) = -EI\,u'''(x).
\]
Differentiating this with respect to $x$ yields
\[
V'(x) = -EI\,u^{(4)}(x).
\]
Finally, the shear equilibrium equation $V'(x) + q = 0$ gives
\[
-EI\,u^{(4)}(x) + q = 0,
\]
or, equivalently,
\[
EI\,u^{(4)}(x) = q.
\]
Thus the deflection $u(x)$ satisfies the fourth-order linear ordinary differential equation
\[
u^{(4)}(x) = \frac{q}{EI},
\]
with constant right-hand side. This is a typical example of a linear static problem: the operator $L[u] = EI\,u^{(4)}$ is linear, the data $q$ are time-independent, and we will impose boundary conditions at the ends of the domain.

\medskip

\textbf{2. General solution and simply supported boundary conditions.}

To solve
\[
u^{(4)}(x) = \frac{q}{EI},
\]
we integrate successively. For convenience, set
\[
a := \frac{q}{EI},
\]
so the equation is $u^{(4)}(x) = a$, with $a$ a constant.

Integrating once,
\[
u^{(3)}(x) = a x + C_1,
\]
where $C_1$ is an integration constant. Integrating again,
\[
u''(x) 
= \int (a x + C_1)\,dx 
= \frac{a}{2}x^2 + C_1 x + C_2.
\]
Integrating a third time,
\[
u'(x) 
= \int\left(\frac{a}{2}x^2 + C_1 x + C_2\right)\,dx
= \frac{a}{6}x^3 + \frac{C_1}{2}x^2 + C_2 x + C_3.
\]
Finally, integrating a fourth time,
\[
u(x) 
= \int\left(\frac{a}{6}x^3 + \frac{C_1}{2}x^2 + C_2 x + C_3\right)\,dx
= \frac{a}{24}x^4 + \frac{C_1}{6}x^3 + \frac{C_2}{2}x^2 + C_3 x + C_4.
\]
Thus the general solution is the quartic polynomial
\[
u(x) = \frac{a}{24}x^4 + \frac{C_1}{6}x^3 + \frac{C_2}{2}x^2 + C_3 x + C_4,
\]
with four constants $C_1,C_2,C_3,C_4$ to be determined from the boundary conditions.

For a simply supported beam at $x=0$ and $x=L$, we impose:
\[
u(0) = 0,\quad u(L) = 0, \qquad u''(0) = 0,\quad u''(L) = 0.
\]
The conditions $u(0)=0$ and $u(L)=0$ express that the beam passes through the supports and cannot deflect there. The conditions $u''(0)=u''(L)=0$ express that the bending moment vanishes at the simple supports, since
\[
M(x) = -EI\,u''(x).
\]

We now apply these conditions.

\emph{First}, from $u(0)=0$ we have
\[
u(0) = \frac{a}{24}\cdot 0 + \frac{C_1}{6}\cdot 0 + \frac{C_2}{2}\cdot 0 + C_3\cdot 0 + C_4 = C_4 = 0.
\]
Thus $C_4 = 0$.

\emph{Second}, from $u''(0)=0$ and the expression
\[
u''(x) = \frac{a}{2}x^2 + C_1 x + C_2,
\]
we obtain
\[
u''(0) = C_2 = 0.
\]
So $C_2 = 0$.

\emph{Third}, from $u''(L) = 0$ we have
\[
0 = u''(L) = \frac{a}{2}L^2 + C_1 L + C_2
= \frac{a}{2}L^2 + C_1 L,
\]
since $C_2=0$. Solving for $C_1$ gives
\[
C_1 = -\frac{a}{2}L.
\]

\emph{Fourth}, from $u(L)=0$ we compute
\[
0 = u(L) 
= \frac{a}{24}L^4 + \frac{C_1}{6}L^3 + \frac{C_2}{2}L^2 + C_3 L + C_4
= \frac{a}{24}L^4 + \frac{C_1}{6}L^3 + C_3 L,
\]
since $C_2 = C_4 = 0$. Substituting $C_1 = -\frac{a}{2}L$ yields
\[
0 = \frac{a}{24}L^4 + \frac{1}{6}\left(-\frac{a}{2}L\right)L^3 + C_3 L
= \frac{a}{24}L^4 - \frac{a}{12}L^4 + C_3 L
= -\frac{a}{24}L^4 + C_3 L.
\]
Hence
\[
C_3 L = \frac{a}{24}L^4
\quad\Rightarrow\quad
C_3 = \frac{a}{24}L^3.
\]

We now substitute all constants back into $u(x)$:
\[
u(x)
= \frac{a}{24}x^4 + \frac{C_1}{6}x^3 + \frac{C_2}{2}x^2 + C_3 x + C_4
= \frac{a}{24}x^4 + \frac{1}{6}\left(-\frac{a}{2}L\right)x^3 + 0 + \frac{a}{24}L^3 x + 0.
\]
Simplifying,
\[
u(x)
= \frac{a}{24}\left(x^4 - 2L x^3 + L^3 x\right).
\]
Recalling that $a=q/(EI)$, we obtain the explicit deflection of the simply supported beam:
\[
u(x) 
= \frac{q}{24\,EI}\,\bigl(x^4 - 2L x^3 + L^3 x\bigr), 
\qquad 0 \le x \le L.
\]

This function is a quartic polynomial in $x$; its form reflects the fact that the operator $u \mapsto EI\,u^{(4)}$ is linear, the right-hand side $q$ is constant, and the boundary conditions are linear. Together, these ingredients make this a prototypical linear static boundary-value problem for an ordinary differential equation.

\medskip

\textbf{3. Location and magnitude of the maximum deflection.}

To find the location of the maximum deflection, we differentiate $u(x)$ and look for critical points. Differentiating,
\[
u'(x) 
= \frac{q}{24\,EI}\,\bigl(4x^3 - 6L x^2 + L^3\bigr).
\]
We solve $u'(x)=0$:
\[
4x^3 - 6L x^2 + L^3 = 0.
\]
By inspection, $x = L/2$ is a root:
\[
4\left(\frac{L}{2}\right)^3 - 6L\left(\frac{L}{2}\right)^2 + L^3
= 4\frac{L^3}{8} - 6L\frac{L^2}{4} + L^3
= \frac{L^3}{2} - \frac{3}{2}L^3 + L^3 = 0.
\]
Thus $x = L/2$ is a stationary point. One can factor the polynomial to find the remaining roots, but physically we know the beam, load, and supports are symmetric about $x = L/2$, so this point must correspond to either a maximum or a minimum deflection in the interior. Since the beam sags downward under the load, $u(L/2)$ is a maximum.

We now compute the deflection at $x = L/2$:
\[
u\!\left(\frac{L}{2}\right) 
= \frac{q}{24\,EI}\left[\left(\frac{L}{2}\right)^4 - 2L\left(\frac{L}{2}\right)^3 + L^3\left(\frac{L}{2}\right)\right].
\]
We simplify each term:
\[
\left(\frac{L}{2}\right)^4 = \frac{L^4}{16},\quad
2L\left(\frac{L}{2}\right)^3 = 2L\cdot\frac{L^3}{8} = \frac{L^4}{4},\quad
L^3\left(\frac{L}{2}\right) = \frac{L^4}{2}.
\]
Thus
\[
u\!\left(\frac{L}{2}\right)
= \frac{q}{24\,EI}\left(\frac{L^4}{16} - \frac{L^4}{4} + \frac{L^4}{2}\right)
= \frac{qL^4}{24\,EI}\left(\frac{1}{16} - \frac{4}{16} + \frac{8}{16}\right)
= \frac{qL^4}{24\,EI}\cdot\frac{5}{16}
= \frac{5qL^4}{384\,EI}.
\]
Therefore, the maximum deflection occurs at the midpoint $x_{\max} = L/2$, with magnitude
\[
u_{\max} = u\!\left(\frac{L}{2}\right) = \frac{5 q L^4}{384\,EI}.
\]

\medskip

\textbf{4. Remarks on linear static problems and boundary conditions.}

This example illustrates several central ideas of \emph{linear static problems}:

\begin{itemize}
  \item The governing equation $EI\,u^{(4)}(x) = q$ is linear in $u$, and the data $q$ are independent of time. The operator $L[u] = EI\,u^{(4)}$ is a linear differential operator of fourth order.
  \item The boundary conditions at $x=0$ and $x=L$ are linear in $u$ and its derivatives. Changing the physical support conditions corresponds to modifying these boundary conditions. For example, if the beam were clamped at both ends, we would impose $u(0)=u'(0)=u(L)=u'(L)=0$ instead. Solving with these clamped conditions leads to
  \[
  u(x) = \frac{q}{24\,EI}\,x^2(x-L)^2,
  \]
  whose maximum deflection is $qL^4/(384\,EI)$, notably smaller than in the simply supported case.
  \item The method of solution consists of integrating the differential equation to obtain a general solution with arbitrary constants and then using the boundary conditions to determine those constants. This pattern recurs throughout the study of linear boundary-value problems for ordinary differential equations.
\end{itemize}

In summary, the static deflection of a simply supported elastic beam under uniform load is described by a quartic polynomial, obtained by solving a linear fourth-order ordinary differential equation with appropriate boundary conditions that encode the support behavior at the ends.
\end{solution}

\section{Sturm–Liouville (Spectral) Theory}
% --- Narrative plan (auto-generated) ---
% This section develops Sturm–Liouville theory, which studies a special class of second-order linear differential operators equipped with boundary conditions. These operators admit real eigenvalues and orthogonal eigenfunctions, much like symmetric matrices in linear algebra. The main goal is to understand how such operators arise from physical models, how to systematically find their eigenvalues and eigenfunctions, and how to use these eigenfunctions as building blocks to represent more complicated functions.
%
% Sturm–Liouville problems lie at the heart of separation of variables for partial differential equations such as the heat, wave, and Laplace equations. When you separate variables in a PDE, the spatial part almost always becomes a Sturm–Liouville problem, and its spectrum determines the time evolution and qualitative behavior of solutions. This spectral viewpoint connects directly to Fourier series and Fourier transforms: orthogonal eigenfunctions generalize sines and cosines, and expansions in these eigenfunctions allow you to solve boundary value and initial value problems.
%
% Conceptually, Sturm–Liouville theory brings together ideas from ordinary differential equations, linear algebra, and functional analysis. It prepares the ground for modern spectral theory of operators, and it links to complex analysis through contour integral representations and analytic properties of eigenfunctions. In applied mathematics, mastering these tools is essential for modeling vibrating systems, diffusion, quantum mechanics, and for understanding how spatial structure shapes dynamical behavior.

% ===== Example 1: The vibrating string and the simplest Sturm–Liouville problem (inquiry-based) =====
\begin{problem}[The vibrating string and the simplest Sturm--Liouville problem]
A taut string of length $L$ fixed at both ends can be modeled, to a good approximation, by the one-dimensional wave equation. The transverse displacement $u(x,t)$ of the string at position $x \in [0,L]$ and time $t \ge 0$ satisfies a partial differential equation together with appropriate boundary conditions. In this problem you will rediscover how separation of variables leads to a simple Sturm--Liouville problem, whose eigenvalues determine the natural frequencies of vibration of the string. Along the way you will see how a very concrete physical system gives rise to the abstract spectral theory discussed in this chapter.

Assume a string of uniform density under constant tension, and let $u(x,t)$ denote the vertical displacement from equilibrium.

\smallskip

(a) Write down the one-dimensional wave equation for $u(x,t)$ with constant wave speed $c>0$, and impose the boundary conditions corresponding to the string being fixed at both ends $x=0$ and $x=L$.

\begin{itemize}
  \item[(i)] State the partial differential equation (PDE) that $u$ must satisfy.
  \item[(ii)] State the boundary conditions at $x=0$ and $x=L$ in terms of $u$.
\end{itemize}

Hint: The standard wave equation in one space dimension relates $u_{tt}$ and $u_{xx}$.

\smallskip

(b) Use separation of variables to reduce the wave equation to ordinary differential equations.

Assume a separated solution of the form
\[
u(x,t) = X(x)\,T(t),
\]
where $X$ depends only on $x$ and $T$ depends only on $t$.

\begin{itemize}
  \item[(i)] Substitute $u(x,t)=X(x)T(t)$ into your PDE from part (a), and divide by $X(x)T(t)$ (assuming $X$ and $T$ are not identically zero) to obtain an equation of the form
  \[
  \frac{T''(t)}{c^2 T(t)} = \frac{X''(x)}{X(x)}.
  \]
  Explain why the left-hand side depends only on $t$ and the right-hand side depends only on $x$, and why this forces both sides to be equal to the same constant, say $-\lambda$.
  \item[(ii)] Write down the resulting ordinary differential equations for $X$ and $T$.
  \item[(iii)] Translate the boundary conditions from part (a) into boundary conditions for $X(x)$.
\end{itemize}

Hint: Think about what it means for $u(0,t)$ and $u(L,t)$ to be zero for all $t$ in terms of $X$ and $T$.

\smallskip

(c) You should now have arrived at a boundary value problem for $X$ of the form
\[
- X''(x) = \lambda X(x), \qquad 0<x<L,
\]
together with boundary conditions $X(0)=0$ and $X(L)=0$.

\begin{itemize}
  \item[(i)] This is the simplest example of a Sturm--Liouville eigenvalue problem. Briefly explain why one calls $\lambda$ an ``eigenvalue'' and $X$ an ``eigenfunction'' in this context.
  \item[(ii)] Solve this boundary value problem by considering three cases for the parameter $\lambda$: (A) $\lambda<0$, (B) $\lambda=0$, and (C) $\lambda>0$.
  
  For each case:
  \begin{itemize}
    \item Write down the general solution of $-X''=\lambda X$.
    \item Impose the boundary condition at $x=0$ to restrict the constants in your general solution.
    \item Impose the boundary condition at $x=L$ and determine for which values of $\lambda$ there exist nontrivial (not identically zero) solutions $X$.
  \end{itemize}
\end{itemize}

Hint: When $\lambda>0$, it is convenient to write $\lambda = \mu^2$ with $\mu>0$, and when $\lambda<0$, write $\lambda=-\mu^2$.

\smallskip

(d) Collect your results from part (c).

\begin{itemize}
  \item[(i)] List all eigenvalues $\lambda_n$ and corresponding eigenfunctions $X_n(x)$ that satisfy $-X''=\lambda X$ with $X(0)=X(L)=0$.
  \item[(ii)] For each allowed eigenvalue $\lambda_n$, solve the time equation
  \[
  T''(t) + c^2 \lambda_n T(t) = 0
  \]
  and write down the corresponding separated solutions $u_n(x,t) = X_n(x)T_n(t)$.
  \item[(iii)] Interpret the parameters $\lambda_n$ and the corresponding angular frequencies
  \[
  \omega_n = c\sqrt{\lambda_n}
  \]
  in terms of the ``natural modes'' or ``harmonics'' of the vibrating string.
\end{itemize}

Hint: The ODE for $T$ is a familiar harmonic oscillator equation when $\lambda_n>0$.

\smallskip

(e) Finally, explore how this picture changes under slight modifications.

\begin{itemize}
  \item[(i)] Suppose instead that the left end of the string is fixed and the right end is free. A free end corresponds (under reasonable modeling assumptions) to the slope being zero there, that is $u_x(L,t) = 0$. Write the new boundary conditions for $X(x)$, and sketch how the eigenvalue problem and its solutions would change. In particular, what trigonometric functions do you expect to appear as eigenfunctions in this case?

  \item[(ii)] In our derivation, the spatial equation took the form
  \[
  -X''(x) = \lambda X(x),
  \]
  which can be written in Sturm--Liouville form
  \[
  -\frac{d}{dx}\Bigl(p(x)\,X'(x)\Bigr) + q(x)\,X(x) = \lambda\,w(x)\,X(x)
  \]
  with appropriate choices of $p$, $q$, and $w$. Identify $p(x)$, $q(x)$, and $w(x)$ in the vibrating string example, and briefly speculate how a non-uniform string (for instance, with variable density) might lead to a different weight function $w(x)$.
\end{itemize}

Hint: For (ii), compare your equation to the Sturm--Liouville template and match coefficients term by term.
\end{problem}

% ===== Example 1: The vibrating string and the simplest Sturm–Liouville problem (full solution) =====
\begin{problem}[The vibrating string and the simplest Sturm--Liouville problem]
Consider a taut string of length $L$ with fixed ends, whose transverse displacement $u(x,t)$ satisfies the one-dimensional wave equation
\[
u_{tt} = c^2 u_{xx}, \qquad 0<x<L,\ t>0,
\]
with boundary conditions
\[
u(0,t) = 0, \qquad u(L,t) = 0, \qquad t\ge 0.
\]

(a) Using a separated solution $u(x,t)=X(x)T(t)$, derive the ordinary differential equations for $X$ and $T$, and show that $X$ must satisfy the boundary value problem
\[
- X''(x) = \lambda X(x), \qquad X(0)=0,\ X(L)=0,
\]
for some constant $\lambda$.

(b) Solve this Sturm--Liouville eigenvalue problem: determine all eigenvalues $\lambda_n$ and corresponding eigenfunctions $X_n(x)$ (up to nonzero scalar multiples).

(c) For each eigenvalue $\lambda_n$, solve the time equation for $T_n(t)$, and write down the corresponding separated solutions
\[
u_n(x,t) = X_n(x) T_n(t).
\]
Identify the natural angular frequencies $\omega_n$ of vibration.

(d) Briefly explain how this example fits into the general Sturm--Liouville framework, and indicate the functions $p(x)$, $q(x)$, and $w(x)$ in the standard form
\[
-\frac{d}{dx}\bigl(p(x) y'(x)\bigr) + q(x) y(x) = \lambda\, w(x)\, y(x).
\]
\end{problem}

\begin{solution}
We begin from the physical model: a taut, uniform string of length $L$ with displacement $u(x,t)$ satisfies the wave equation
\[
u_{tt} = c^2 u_{xx}, \qquad 0<x<L,\ t>0,
\]
where $c>0$ is the wave speed, and the fixed ends impose
\[
u(0,t)=0,\qquad u(L,t)=0,\qquad t\ge 0.
\]

\medskip

\textbf{(a) Separation of variables and the spatial eigenvalue problem.}
We look for separated solutions of the form
\[
u(x,t) = X(x)\,T(t),
\]
where $X$ is a function of $x$ alone and $T$ is a function of $t$ alone. Substituting into the wave equation gives
\[
X(x) T''(t) = c^2 X''(x) T(t).
\]
Assuming $X$ and $T$ are not identically zero, we may divide both sides by $c^2 X(x) T(t)$:
\[
\frac{T''(t)}{c^2 T(t)} = \frac{X''(x)}{X(x)}.
\]
The left-hand side depends only on $t$ and the right-hand side depends only on $x$. The only way a function of $t$ can be equal to a function of $x$ for all $x$ and $t$ is for both functions to be constant. We denote this common constant by $-\lambda$:
\[
\frac{T''(t)}{c^2 T(t)} = \frac{X''(x)}{X(x)} = -\lambda.
\]

This yields the system of ordinary differential equations
\[
T''(t) + c^2 \lambda\, T(t) = 0, \qquad -X''(x) = \lambda\, X(x).
\]

The boundary conditions $u(0,t)=0$ and $u(L,t)=0$ must hold for all $t$. In terms of $X$ and $T$, they read
\[
X(0) T(t) = 0,\qquad X(L) T(t)=0,\quad \text{for all $t$}.
\]
We are looking for nontrivial separated solutions, so we assume $T$ is not identically zero. Hence $X(0)=0$ and $X(L)=0$. Thus $X$ must satisfy the boundary value problem
\[
- X''(x) = \lambda X(x), \qquad 0<x<L,\qquad X(0)=0,\ X(L)=0.
\]
This is the simplest form of a Sturm--Liouville eigenvalue problem: we seek values of the parameter $\lambda$ (the eigenvalues) for which there exist nontrivial solutions $X$ (the eigenfunctions) satisfying both the differential equation and the boundary conditions.

\medskip

\textbf{(b) Solving the Sturm--Liouville problem.}
We now solve
\[
- X''(x) = \lambda X(x),\qquad X(0)=0,\ X(L)=0.
\]
We distinguish three cases for $\lambda$.

\emph{Case 1: $\lambda<0$.} Write $\lambda = -\mu^2$ with $\mu>0$. The equation becomes
\[
- X''(x) = -\mu^2 X(x) \quad\Longleftrightarrow\quad X''(x) = \mu^2 X(x),
\]
whose general solution is
\[
X(x) = A e^{\mu x} + B e^{-\mu x}.
\]
Imposing $X(0)=0$ gives $A+B=0$, so $B=-A$ and
\[
X(x) = A\bigl(e^{\mu x} - e^{-\mu x}\bigr) = 2A \sinh(\mu x).
\]
Imposing $X(L)=0$ then yields $\sinh(\mu L)=0$, which forces $A=0$ since $\sinh(\mu L)\ne 0$ for $\mu>0$. Thus the only solution is the trivial one $X\equiv 0$. Therefore there are no negative eigenvalues.

\emph{Case 2: $\lambda=0$.} The equation becomes $-X''(x)=0$, that is,
\[
X''(x)=0,
\]
whose general solution is
\[
X(x) = A x + B.
\]
The condition $X(0)=0$ gives $B=0$, so $X(x)=Ax$. Then $X(L)=0$ implies $A L = 0$, hence $A=0$. Again, only the trivial solution exists. So $\lambda=0$ is not an eigenvalue.

\emph{Case 3: $\lambda>0$.} Write $\lambda=\mu^2$ with $\mu>0$. Then
\[
- X''(x) = \mu^2 X(x) \quad\Longleftrightarrow\quad X''(x) + \mu^2 X(x)=0,
\]
whose general solution is
\[
X(x) = A \cos(\mu x) + B \sin(\mu x).
\]
Imposing $X(0)=0$ gives $A \cos(0) + B \sin(0) = A = 0$. Thus $A=0$ and
\[
X(x) = B \sin(\mu x).
\]
Now $X(L)=0$ requires
\[
B \sin(\mu L) = 0.
\]
For a nontrivial solution we must have $B\ne 0$, so $\sin(\mu L)=0$. This occurs precisely when $\mu L = n\pi$ for some positive integer $n$. Hence
\[
\mu_n = \frac{n\pi}{L},\qquad n=1,2,3,\dots,
\]
and the corresponding eigenvalues are
\[
\lambda_n = \mu_n^2 = \Bigl(\frac{n\pi}{L}\Bigr)^2.
\]
For each $n$, a corresponding eigenfunction is
\[
X_n(x) = \sin\Bigl(\frac{n\pi x}{L}\Bigr),
\]
up to a nonzero multiplicative constant. Scaling does not matter for eigenfunctions, so this choice is convenient.

Thus the Sturm--Liouville problem has a countable set of positive eigenvalues
\[
\lambda_n = \Bigl(\frac{n\pi}{L}\Bigr)^2,\quad n=1,2,\dots,
\]
with eigenfunctions $X_n(x)=\sin\bigl(\frac{n\pi x}{L}\bigr)$.

\medskip

\textbf{(c) Time dependence and normal modes.}
For each eigenvalue $\lambda_n$ we now solve the time equation
\[
T''(t) + c^2 \lambda_n T(t) = 0.
\]
Substituting $\lambda_n = (n\pi/L)^2$, this becomes
\[
T''(t) + \omega_n^2 T(t) = 0, \qquad \omega_n := c \frac{n\pi}{L}.
\]
This is the standard harmonic oscillator equation with angular frequency $\omega_n$. Its general real solution is
\[
T_n(t) = A_n \cos(\omega_n t) + B_n \sin(\omega_n t),
\]
where $A_n$ and $B_n$ are constants determined by initial conditions (such as the initial shape and initial velocity of the string).

Combining the spatial and temporal parts, the separated solutions (also called \emph{normal modes}) are
\[
u_n(x,t) = X_n(x)\,T_n(t)
= \sin\Bigl(\frac{n\pi x}{L}\Bigr)\,\Bigl[A_n \cos(\omega_n t) + B_n \sin(\omega_n t)\Bigr],
\quad n=1,2,\dots
\]
Each mode $u_n$ oscillates sinusoidally in time with fixed spatial shape $\sin(\frac{n\pi x}{L})$ and angular frequency
\[
\omega_n = c\sqrt{\lambda_n} = c\,\frac{n\pi}{L}.
\]
These $\omega_n$ are the \emph{natural frequencies} or \emph{harmonics} of the string. The fundamental frequency corresponds to $n=1$, and higher values of $n$ give overtones with spatial patterns possessing more interior nodes.

From Sturm--Liouville theory, the eigenfunctions $\{\sin(\frac{n\pi x}{L})\}_{n=1}^\infty$ form an orthogonal set in $L^2(0,L)$ with respect to the standard inner product
\[
\langle f,g\rangle = \int_0^L f(x) g(x)\,dx.
\]
Indeed, one can verify
\[
\int_0^L \sin\Bigl(\frac{n\pi x}{L}\Bigr)\sin\Bigl(\frac{m\pi x}{L}\Bigr)\,dx =
\begin{cases}
0, & n\ne m,\\[4pt]
\dfrac{L}{2}, & n=m.
\end{cases}
\]
This orthogonality allows one to expand a general initial displacement as a Fourier sine series in these eigenfunctions and hence superpose the normal modes to obtain the full solution. Although this expansion step is not asked for explicitly here, it is a central point of Sturm--Liouville spectral theory.

\medskip

\textbf{(d) Sturm--Liouville form.}
In general, a Sturm--Liouville problem has the form
\[
-\frac{d}{dx}\bigl(p(x) y'(x)\bigr) + q(x) y(x) = \lambda\, w(x)\, y(x),
\]
together with suitable boundary conditions. In the vibrating string example, the spatial equation is
\[
- X''(x) = \lambda X(x),
\]
which can be written as
\[
-\frac{d}{dx}\bigl(1\cdot X'(x)\bigr) + 0\cdot X(x) = \lambda\, 1\cdot X(x).
\]
Thus we can identify
\[
p(x) \equiv 1,\qquad q(x) \equiv 0,\qquad w(x) \equiv 1.
\]
The boundary conditions $X(0)=0$ and $X(L)=0$ are homogeneous Dirichlet conditions, which fit the Sturm--Liouville framework.

This example illustrates the central themes of Sturm--Liouville (spectral) theory:
\begin{itemize}
  \item A second-order self-adjoint differential operator (here $L[X] = -X''$ with Dirichlet boundary conditions) has a discrete spectrum of real eigenvalues $\lambda_n$.
  \item There is an associated sequence of eigenfunctions $X_n$ which are orthogonal with respect to the weight $w(x)$ (here simply $1$).
  \item These eigenfunctions form a basis (in an appropriate sense) for expanding general functions, leading to Fourier series and modal decompositions of solutions to PDEs such as the wave equation.
\end{itemize}
The vibrating string with fixed ends is therefore the archetypal concrete realization of Sturm--Liouville spectral theory on a finite interval.
\end{solution}

% ===== Example 2: Heat flow in a rod and eigenfunction expansions (inquiry-based) =====
\begin{problem}[Heat flow in a rod and eigenfunction expansions]
Consider a thin, homogeneous rod of length $L$ lying along the $x$–axis from $x=0$ to $x=L$. Let $u(x,t)$ denote the temperature at position $x$ and time $t$. The rod is perfectly insulated along its sides, but its ends are held at zero temperature by contact with large heat baths. The temperature evolves according to the one-dimensional heat equation with thermal diffusivity $\kappa>0$,
\[
u_t = \kappa\,u_{xx},
\]
together with homogeneous Dirichlet boundary conditions $u(0,t)=u(L,t)=0$. We prescribe an initial temperature profile $u(x,0)=f(x)$ and seek to describe $u(x,t)$ for $t>0$.

\smallskip

(a) Write down the full initial–boundary value problem for $u(x,t)$ on $0<x<L$, $t>0$, including the partial differential equation, the boundary conditions, and the initial condition. Then assume a {\em separated} form
\[
u(x,t) = X(x)\,T(t),
\]
and substitute into the heat equation. Rearrange the resulting equation to isolate all dependence on $x$ on one side and all dependence on $t$ on the other side.

\begin{itemize}
  \item[(i)] What does this rearrangement tell you about the existence of a constant $\lambda$ (the {\em separation constant}) such that $X$ and $T$ satisfy
  \[
  X''(x) + \lambda X(x) = 0, \qquad T'(t) + \kappa \lambda T(t) = 0?
  \]
  \item[(ii)] What boundary conditions must $X(x)$ satisfy, and why?
\end{itemize}
Hint: Justify carefully why the left-hand side depends only on $t$ and the right-hand side only on $x$, and therefore they must both be equal to a constant.

\smallskip

(b) We now analyze the {\em spatial} Sturm–Liouville problem
\[
X''(x) + \lambda X(x) = 0, \qquad X(0)=0,\quad X(L)=0.
\]
Consider the three cases $\lambda<0$, $\lambda=0$, and $\lambda>0$.

\begin{itemize}
  \item[(i)] Show that if $\lambda\leq 0$, then the only solution satisfying the boundary conditions is the trivial solution $X\equiv 0$. (Thus no nonzero separated solution arises from these values of $\lambda$.)
  \item[(ii)] Solve the differential equation explicitly for $\lambda>0$ and impose the boundary conditions to determine the allowed values of $\lambda$.
  \item[(iii)] Show that for $\lambda>0$, the nontrivial solutions are (up to constant multiples) of the form
  \[
  X_n(x) = \sin\left(\frac{n\pi x}{L}\right) \quad\text{for } n=1,2,3,\dots,
  \]
  and find the corresponding eigenvalues $\lambda_n$.
\end{itemize}
Hint: For $\lambda>0$, write $\lambda=\mu^2$ and express the general solution in terms of $\sin(\mu x)$ and $\cos(\mu x)$. Use the two boundary conditions to force $\mu$ to take discrete values.

\smallskip

(c) The functions $\{X_n\}_{n=1}^\infty$ form an orthogonal family with respect to the inner product
\[
\langle \phi,\psi\rangle = \int_0^L \phi(x)\,\psi(x)\,dx.
\]
\begin{itemize}
  \item[(i)] Verify directly that
  \[
  \int_0^L \sin\left(\frac{m\pi x}{L}\right)\sin\left(\frac{n\pi x}{L}\right)\,dx =
  \begin{cases}
    0, & m\neq n,\\[4pt]
    \dfrac{L}{2}, & m=n.
  \end{cases}
  \]
  \item[(ii)] For each $n$, solve the time equation
  \[
  T_n'(t) + \kappa \lambda_n T_n(t) = 0
  \]
  and write the corresponding separated solution $u_n(x,t)=X_n(x)T_n(t)$.
\end{itemize}
Hint: For the integral, you may use the product-to-sum identity for sines, or integrate by parts. For the time equation, recall how to solve a first-order linear ODE with constant coefficients.

\smallskip

(d) We now assemble the separated solutions into a series that can match an arbitrary initial temperature profile $f(x)$.

\begin{itemize}
  \item[(i)] Argue (heuristically, using completeness of the sine functions) that we can seek a solution in the form
  \[
  u(x,t) = \sum_{n=1}^\infty b_n\,\sin\left(\frac{n\pi x}{L}\right) e^{-\kappa \lambda_n t},
  \]
  where the $\lambda_n$ and $\sin(n\pi x/L)$ are as found above.
  \item[(ii)] Impose the initial condition $u(x,0)=f(x)$ and show that $f$ must have a {\em sine series expansion}
  \[
  f(x) = \sum_{n=1}^\infty b_n\,\sin\left(\frac{n\pi x}{L}\right)
  \]
  on the interval $[0,L]$. Use the orthogonality from part (c) to derive a formula for the coefficients $b_n$ in terms of $f$.
  \item[(iii)] As a concrete example, take $f(x)\equiv 1$ (the rod starts at uniform temperature). Compute the coefficients $b_n$ explicitly and write the resulting series for $u(x,t)$.
\end{itemize}
Hint: Multiply the expansion for $f(x)$ by $\sin(m\pi x/L)$ and integrate from $0$ to $L$ to isolate $b_m$.

\smallskip

(e) Explore one or two variations of the model.

\begin{itemize}
  \item[(i)] Suppose instead that the rod is insulated at both ends, so that no heat flows through $x=0$ and $x=L$. This is modeled by Neumann boundary conditions $u_x(0,t)=u_x(L,t)=0$. Repeat the separation-of-variables step and write down the corresponding spatial eigenvalue problem for $X(x)$. What do you expect the eigenfunctions to look like in this case (sines, cosines, or something else)?
  \item[(ii)] The vibrating string with fixed ends is governed by the wave equation $v_{tt}=c^2 v_{xx}$ with $v(0,t)=v(L,t)=0$. If you repeat the same separation-of-variables procedure for that problem, which parts of the analysis above will be identical, and which will change? In particular, compare the time dependence you obtain here (exponential decay) with the time dependence in the wave equation (oscillatory behavior).
\end{itemize}
Hint: Think about which differential operator is the same in both problems, and which equation differs only by replacing a first time derivative by a second time derivative.
\end{problem}

% ===== Example 2: Heat flow in a rod and eigenfunction expansions (full solution) =====
\begin{problem}[Heat flow in a rod and eigenfunction expansions]
Consider the heat equation on a homogeneous rod of length $L$:
\[
u_t = \kappa u_{xx},\qquad 0<x<L,\ t>0,
\]
with homogeneous Dirichlet boundary conditions
\[
u(0,t)=0,\quad u(L,t)=0,\qquad t>0,
\]
and initial condition
\[
u(x,0)=f(x),\qquad 0<x<L,
\]
where $\kappa>0$ is a constant and $f$ is a given function.

\begin{enumerate}
  \item[(a)] Use separation of variables to find the eigenvalues $\lambda_n$ and eigenfunctions $X_n(x)$ of the associated spatial problem. Show that, up to scalar multiples,
  \[
  X_n(x)=\sin\!\left(\frac{n\pi x}{L}\right),\qquad \lambda_n = \left(\frac{n\pi}{L}\right)^2,\quad n=1,2,\dots.
  \]
  \item[(b)] Show that the solution can be written as an eigenfunction expansion
  \[
  u(x,t) = \sum_{n=1}^\infty b_n \sin\!\left(\frac{n\pi x}{L}\right)
  e^{-\kappa \left(\frac{n\pi}{L}\right)^2 t},
  \]
  and derive a formula for the coefficients $b_n$ in terms of $f$.
  \item[(c)] Specialize to the case of a uniformly heated rod with $f(x)\equiv 1$ on $(0,L)$. Compute the coefficients $b_n$ explicitly and write the resulting series for $u(x,t)$.
\end{enumerate}
\end{problem}

\begin{solution}
We solve the heat equation by the method of separation of variables, and in doing so we will see the role of the Sturm–Liouville eigenvalue problem associated with the spatial operator $-d^2/dx^2$ and homogeneous Dirichlet boundary conditions.

\medskip

\noindent\textbf{(a) Separation of variables and the spatial eigenvalue problem.}
We look for solutions of the form
\[
u(x,t) = X(x)\,T(t),
\]
where $X$ depends only on $x$ and $T$ depends only on $t$. Substituting into the heat equation $u_t=\kappa u_{xx}$ gives
\[
X(x)\,T'(t) = \kappa X''(x)\,T(t).
\]
Assuming $X$ and $T$ are not identically zero, we can divide both sides by $\kappa X(x)T(t)$ to separate the variables:
\[
\frac{T'(t)}{\kappa T(t)} = \frac{X''(x)}{X(x)}.
\]
The left-hand side depends only on $t$, while the right-hand side depends only on $x$. Hence both sides must be equal to the same constant, say $-\lambda$. Thus we obtain the two ordinary differential equations
\begin{equation}\label{eq:separated-ODEs}
\begin{aligned}
T'(t) + \kappa \lambda T(t) &= 0,\\
X''(x) + \lambda X(x) &= 0.
\end{aligned}
\end{equation}
The boundary conditions on $u$ translate to boundary conditions on $X$:
\[
u(0,t)=0 \implies X(0)T(t)=0,\quad
u(L,t)=0 \implies X(L)T(t)=0.
\]
Since we seek nontrivial products $X\,T$, we require that $T$ is not identically zero, so $X$ must satisfy
\[
X(0)=0,\quad X(L)=0.
\]
Thus the spatial factor $X$ solves the Sturm–Liouville problem
\[
X''(x) + \lambda X(x) = 0,\qquad X(0)=0,\quad X(L)=0.
\]

We now analyze this eigenvalue problem and determine the admissible values of $\lambda$ and the corresponding eigenfunctions.

\medskip

\noindent\textbf{Case $\lambda\le 0$.}
If $\lambda=0$, then $X''(x)=0$, so
\[
X(x)=A+Bx.
\]
The boundary condition $X(0)=0$ implies $A=0$, and then $X(L)=0$ implies $BL=0$, so $B=0$. Hence $X\equiv0$ is the only solution when $\lambda=0$.

If $\lambda<0$, write $\lambda=-\mu^2$ with $\mu>0$. Then the equation becomes
\[
X''(x)-\mu^2 X(x)=0,
\]
whose general solution is
\[
X(x)=A e^{\mu x}+B e^{-\mu x}.
\]
The boundary condition $X(0)=0$ gives $A+B=0$, so $B=-A$. Thus
\[
X(x)=A(e^{\mu x}-e^{-\mu x})=2A\sinh(\mu x).
\]
Applying $X(L)=0$ gives $2A\sinh(\mu L)=0$. Since $\mu>0$ and $\sinh(\mu L)\neq 0$, we must have $A=0$, so $X\equiv 0$ again. Thus there are no nontrivial solutions for $\lambda\le 0$.

\medskip

\noindent\textbf{Case $\lambda>0$.}
Let $\lambda=\mu^2$ with $\mu>0$. Then the spatial equation is
\[
X''(x)+\mu^2 X(x)=0,
\]
with general solution
\[
X(x)=A\cos(\mu x)+B\sin(\mu x).
\]
The boundary condition $X(0)=0$ gives $A\cos 0 + B\sin 0 = A=0$. Thus $X(x)=B\sin(\mu x)$. Applying $X(L)=0$ yields
\[
B\sin(\mu L)=0.
\]
For a nontrivial solution we must have $B\neq 0$, so $\sin(\mu L)=0$. Therefore $\mu L = n\pi$ for some positive integer $n$. Thus
\[
\mu = \frac{n\pi}{L},\qquad n=1,2,3,\dots,
\]
and hence
\[
\lambda_n = \mu^2 = \left(\frac{n\pi}{L}\right)^2,\qquad
X_n(x)=\sin\left(\frac{n\pi x}{L}\right),\quad n=1,2,3,\dots,
\]
up to multiplication by a nonzero constant. These are precisely the eigenvalues and eigenfunctions of the self-adjoint Sturm–Liouville operator $-d^2/dx^2$ with homogeneous Dirichlet boundary conditions on $(0,L)$.

\medskip

\noindent\textbf{(b) Time factors and eigenfunction expansion.}
For each eigenvalue $\lambda_n$, the corresponding time-dependent factor $T_n(t)$ satisfies
\[
T_n'(t)+\kappa\lambda_n T_n(t)=0,\qquad n=1,2,\dots.
\]
This is a first-order linear ordinary differential equation with constant coefficients. Its general solution is
\[
T_n(t)=C_n e^{-\kappa \lambda_n t}
= C_n \exp\!\left(-\kappa\left(\frac{n\pi}{L}\right)^2 t\right),
\]
where $C_n$ is a constant. Thus each mode produces a separated solution
\[
u_n(x,t) = X_n(x)T_n(t) 
= C_n \sin\left(\frac{n\pi x}{L}\right)
\exp\!\left(-\kappa\left(\frac{n\pi}{L}\right)^2 t\right).
\]

Because the heat equation is linear and homogeneous, any (finite) linear combination of such separated solutions is again a solution. Motivated by the completeness of the sine functions on $(0,L)$ (from Fourier series theory and Sturm–Liouville theory), we look for a general solution as an infinite series
\[
u(x,t) = \sum_{n=1}^\infty b_n \sin\left(\frac{n\pi x}{L}\right)
\exp\!\left(-\kappa\left(\frac{n\pi}{L}\right)^2 t\right),
\]
where the coefficients $b_n$ are to be determined from the initial condition.

At time $t=0$ this series becomes
\[
u(x,0) = \sum_{n=1}^\infty b_n \sin\left(\frac{n\pi x}{L}\right).
\]
Imposing the initial condition $u(x,0)=f(x)$, we arrive at the sine series expansion
\[
f(x) = \sum_{n=1}^\infty b_n \sin\left(\frac{n\pi x}{L}\right),
\qquad 0<x<L.
\]

The eigenfunctions $\sin(n\pi x/L)$ are orthogonal on $(0,L)$ with respect to the standard $L^2$ inner product:
\[
\int_0^L \sin\left(\frac{m\pi x}{L}\right) \sin\left(\frac{n\pi x}{L}\right)\,dx
=
\begin{cases}
0,& m\neq n,\\[4pt]
\dfrac{L}{2},& m=n.
\end{cases}
\]
This can be verified either by explicit integration or by using trigonometric identities. The orthogonality allows us to solve for the coefficients $b_n$ in the usual Fourier way. Multiply the expansion for $f(x)$ by $\sin(m\pi x/L)$ and integrate:
\[
\int_0^L f(x)\sin\left(\frac{m\pi x}{L}\right)\,dx
= \sum_{n=1}^\infty b_n \int_0^L 
\sin\left(\frac{n\pi x}{L}\right)\sin\left(\frac{m\pi x}{L}\right)\,dx.
\]
All terms with $n\neq m$ vanish by orthogonality, and the $n=m$ term gives
\[
\int_0^L f(x)\sin\left(\frac{m\pi x}{L}\right)\,dx
= b_m \cdot \frac{L}{2}.
\]
Therefore
\[
b_m = \frac{2}{L} \int_0^L f(x)\sin\left(\frac{m\pi x}{L}\right)\,dx,\qquad m=1,2,\dots.
\]
Hence the full solution of the initial–boundary value problem is
\[
u(x,t) =
\sum_{n=1}^\infty 
\left(\frac{2}{L} \int_0^L f(s)\sin\left(\frac{n\pi s}{L}\right)\,ds\right)
\sin\left(\frac{n\pi x}{L}\right)
\exp\!\left(-\kappa\left(\frac{n\pi}{L}\right)^2 t\right).
\]
This is an eigenfunction expansion in the orthogonal basis of eigenfunctions of the Sturm–Liouville operator $-d^2/dx^2$ with Dirichlet boundary conditions.

\medskip

\noindent\textbf{(c) Uniform initial temperature.}
Now we specialize to the initial data
\[
f(x)\equiv 1,\qquad 0<x<L.
\]
The coefficients become
\[
b_n = \frac{2}{L} \int_0^L 1\cdot \sin\left(\frac{n\pi x}{L}\right)\,dx.
\]
This integral is straightforward to compute:
\[
\int_0^L \sin\left(\frac{n\pi x}{L}\right)\,dx
= \left[-\frac{L}{n\pi}\cos\left(\frac{n\pi x}{L}\right)\right]_{x=0}^{x=L}
= -\frac{L}{n\pi}\Bigl(\cos(n\pi)-\cos 0\Bigr).
\]
Using $\cos(n\pi)=(-1)^n$ and $\cos 0 =1$, we obtain
\[
\int_0^L \sin\left(\frac{n\pi x}{L}\right)\,dx
= -\frac{L}{n\pi}\left((-1)^n-1\right)
= \frac{L}{n\pi}\left(1-(-1)^n\right).
\]
Therefore
\[
b_n = \frac{2}{L}\cdot \frac{L}{n\pi}\left(1-(-1)^n\right)
= \frac{2}{n\pi}\left(1-(-1)^n\right).
\]
If $n$ is even, then $(-1)^n=1$, so $b_n=0$. If $n$ is odd, write $n=2k+1$; then $(-1)^n=-1$, and
\[
b_n = \frac{2}{n\pi}(1-(-1)) = \frac{4}{n\pi}.
\]
Thus only the odd sine modes appear, and we can write
\[
b_n =
\begin{cases}
\dfrac{4}{n\pi},& n\ \text{odd},\\[4pt]
0,& n\ \text{even}.
\end{cases}
\]

The solution for a uniformly heated rod with ends clamped at zero temperature is therefore
\[
u(x,t) = \sum_{\substack{n=1\\ n\ \text{odd}}}^\infty 
\frac{4}{n\pi}\sin\left(\frac{n\pi x}{L}\right)
\exp\!\left(-\kappa\left(\frac{n\pi}{L}\right)^2 t\right).
\]
Equivalently, we may index over odd integers $n=2k+1$:
\[
u(x,t) = \sum_{k=0}^\infty \frac{4}{(2k+1)\pi}
\sin\left(\frac{(2k+1)\pi x}{L}\right)
\exp\!\left(-\kappa\
\left(\frac{(2k+1)\pi}{L}\right)^2 t\right).
\]

This series solution satisfies the heat equation, the homogeneous Dirichlet boundary conditions, and the uniform initial condition $u(x,0)=1$.

\end{solution}

% ===== Example 3: Legendre’s equation on an interval and nonconstant coefficients (inquiry-based) =====
\begin{problem}[Legendre’s equation on an interval and nonconstant coefficients]
The Legendre differential equation appears when solving Laplace’s equation in spherical coordinates for axisymmetric (no $\varphi$–dependence) potentials. After separating variables and making the change of variable $x = \cos \theta$, one arrives at an ordinary differential equation on the interval $-1 < x < 1$. On this interval, with suitable boundary conditions at $x = \pm 1$, the solutions turn out to be polynomials, the \emph{Legendre polynomials}, which arise as eigenfunctions of a Sturm–Liouville operator with variable coefficients. In this problem you will gradually uncover this structure and see how orthogonal polynomials fit naturally into the Sturm–Liouville framework.

Consider the differential equation
\[
(1 - x^2) y''(x) - 2 x\, y'(x) + \lambda\, y(x) = 0, \qquad -1 < x < 1,
\]
together with the requirement that $y(x)$ remain bounded as $x \to \pm 1$.

\medskip

(a) Rewrite the given equation in Sturm–Liouville form. That is, show that it can be written as
\[
\frac{d}{dx}\bigl(p(x) y'(x)\bigr) + \bigl(\lambda\, w(x) - q(x)\bigr) y(x) = 0
\]
for suitable functions $p(x)$, $w(x)$, and $q(x)$ on $(-1,1)$. Identify $p$, $w$, and $q$ explicitly.

\emph{Hint:} Try to recognize the left-hand side of the original equation as a single derivative of the form $\dfrac{d}{dx}\bigl(\cdots y'(x)\bigr)$ plus the term involving $\lambda y$.

\medskip

(b) One important feature of a Sturm–Liouville problem is the associated self-adjoint differential operator. Define
\[
L[y](x) := -\frac{d}{dx}\bigl(p(x) y'(x)\bigr) + q(x) y(x),
\]
with the $p$ and $q$ you found in part (a), acting on functions defined on $(-1,1)$.

\begin{enumerate}
\item[(i)] Write the Legendre equation in the form
\[
L[y] = \lambda\, w(x)\, y.
\]
\item[(ii)] Use Green’s identity for Sturm–Liouville operators,
\[
\int_{-1}^{1} \bigl(u\,L[v] - v\,L[u]\bigr)\,w(x)\,dx
= \bigl.p(x)\,\bigl(u(x)v'(x) - u'(x)v(x)\bigr)\bigr|_{x=-1}^{x=1},
\]
to explain what boundary behavior at $x = \pm 1$ is needed in order for $L$ to be formally self-adjoint on the space of \emph{bounded} functions on $[-1,1]$.

\end{enumerate}

\emph{Hint:} Observe that $p(x) = 1 - x^2$ vanishes at $x = \pm 1$. If $u$ and $v$ are bounded and not too wild near the endpoints, what happens to the boundary term?

\medskip

(c) Now look for power series solutions about $x = 0$. Assume
\[
y(x) = \sum_{k=0}^\infty a_k x^k,
\]
and substitute this into the differential equation to obtain a recurrence relation for the coefficients $a_k$.

\begin{enumerate}
\item[(i)] Find the recurrence relation expressing $a_{k+2}$ in terms of $a_k$ and $\lambda$.
\item[(ii)] Show that the even and odd coefficients decouple: if you choose $a_0$ and $a_1$ arbitrarily, then all even coefficients are determined from $a_0$, and all odd coefficients are determined from $a_1$.
\end{enumerate}

\emph{Hint:} After substituting the series into the ODE, carefully align powers of $x$ and compare coefficients of $x^k$ on both sides. You should obtain a relation of the form
\[
a_{k+2} = \frac{k(k+1) - \lambda}{(k+2)(k+1)}\,a_k.
\]

\medskip

(d) The requirement that the solution be bounded at $x = \pm 1$ forces a quantization of the parameter $\lambda$.

\begin{enumerate}
\item[(i)] Using your recurrence relation, argue that unless the series \emph{terminates}, the solution will not remain a polynomial and will typically fail to stay bounded as $x \to \pm 1$.
\item[(ii)] Show that the series terminates (so that $y$ is a polynomial) if and only if $\lambda = n(n+1)$ for some integer $n \ge 0$. In this case one obtains a polynomial of degree $n$, denoted $P_n(x)$, called the $n$th Legendre polynomial.
\item[(iii)] Compute $P_0(x)$, $P_1(x)$, and $P_2(x)$ explicitly by executing the recurrence up to the relevant degree.
\end{enumerate}

\emph{Hint:} For the series to terminate, there must be some index $k = n$ such that the numerator $k(k+1) - \lambda$ vanishes, making $a_{n+2} = 0$, and hence all higher $a_{n+2}, a_{n+4}, \dots$ vanish.

\medskip

(e) A central result in Sturm–Liouville theory is that eigenfunctions corresponding to distinct eigenvalues are orthogonal with respect to the weight function $w(x)$.

\begin{enumerate}
\item[(i)] Take two bounded solutions $y_n$ and $y_m$ of the Legendre equation with parameters $\lambda_n = n(n+1)$ and $\lambda_m = m(m+1)$, with $n \ne m$. Use the Sturm–Liouville form you found in part (a) to prove that
\[
\int_{-1}^{1} y_n(x)\,y_m(x)\,dx = 0.
\]
\item[(ii)] Conclude that the Legendre polynomials $\{P_n\}_{n = 0}^\infty$ form an orthogonal family in $L^2(-1,1)$ with respect to the inner product
\[
\langle f,g\rangle = \int_{-1}^{1} f(x)g(x)\,dx.
\]
\end{enumerate}

\emph{Hint:} Multiply the differential equation for $y_n$ by $y_m$, and the equation for $y_m$ by $y_n$, subtract them, and integrate over $[-1,1]$. Then use the boundary behavior identified in part (b) to handle the endpoint terms.

\medskip

(f) What if / extensions.

\begin{enumerate}
\item[(i)] The Legendre equation originally arises in the polar angle $\theta$ with a weight factor $\sin\theta$ in the inner product. If you start instead from the angular equation
\[
\frac{d}{d\theta}\Bigl(\sin\theta\,\frac{dY}{d\theta}\Bigr) + \lambda\,\sin\theta\, Y = 0, \qquad 0 < \theta < \pi,
\]
identify the coefficient $p(\theta)$ and weight $w(\theta)$ in Sturm–Liouville form. Then explain how the change of variable $x = \cos\theta$ transforms this into the Legendre equation with weight $w(x) = 1$.
\item[(ii)] Suppose we changed the interval to $[0,1]$ and imposed boundary conditions such as $y(0) = y(1) = 0$ instead of boundedness at $\pm 1$. How do you think the set of eigenvalues and eigenfunctions would change? Would the Legendre polynomials still appear in the same way?
\end{enumerate}

\end{problem}

% ===== Example 3: Legendre’s equation on an interval and nonconstant coefficients (full solution) =====
\begin{problem}[Legendre’s equation on an interval and nonconstant coefficients]
Consider the differential equation
\[
(1 - x^2)\,y''(x) - 2x\,y'(x) + \lambda\,y(x) = 0,\qquad -1 < x < 1,
\]
with the condition that $y(x)$ remains bounded as $x \to \pm 1$.

\begin{enumerate}
\item[(a)] Put this equation into Sturm–Liouville form, identify the functions $p(x)$, $w(x)$, and $q(x)$, and define the associated operator
\[
L[y] := -\dfrac{d}{dx}\bigl(p(x) y'(x)\bigr) + q(x) y(x).
\]
\item[(b)] By seeking a power series solution about $x=0$, derive the recurrence relation for the coefficients and show that bounded (polynomial) solutions exist only when $\lambda = n(n+1)$ for some integer $n \ge 0$. Denote these polynomial solutions by $P_n(x)$ and compute $P_0(x)$, $P_1(x)$, and $P_2(x)$ explicitly.
\item[(c)] Using the Sturm–Liouville structure, prove that if $m \ne n$, then
\[
\int_{-1}^{1} P_m(x)\,P_n(x)\,dx = 0.
\]
\end{enumerate}
Explain briefly how this example illustrates the general ideas of Sturm–Liouville spectral theory.

\end{problem}

\begin{solution}
We analyze the given equation
\[
(1 - x^2)\,y''(x) - 2x\,y'(x) + \lambda\,y(x) = 0,\qquad -1 < x < 1,
\]
under the requirement that $y$ remains bounded at $x = \pm 1$.

\medskip

\textbf{(a) Sturm–Liouville form and operator.}
A second-order Sturm–Liouville equation on an interval $(a,b)$ has the form
\[
\frac{d}{dx}\bigl(p(x)\,y'(x)\bigr) + \bigl(\lambda\,w(x) - q(x)\bigr)y(x) = 0,
\]
where $p$, $w$ are positive on $(a,b)$, and $q$ is real-valued.

We start from
\[
(1 - x^2)\,y'' - 2x\,y' + \lambda y = 0.
\]
Notice that
\[
\frac{d}{dx}\bigl((1 - x^2)\,y'(x)\bigr)
= (1 - x^2)\,y''(x) - 2x\,y'(x),
\]
so the equation can be rewritten as
\[
\frac{d}{dx}\bigl((1 - x^2)\,y'(x)\bigr) + \lambda\,y(x) = 0.
\]
This matches Sturm–Liouville form with
\[
p(x) = 1 - x^2,\qquad w(x) = 1,\qquad q(x) = 0.
\]
The associated Sturm–Liouville operator is therefore
\[
L[y](x) := -\frac{d}{dx}\bigl(p(x)\,y'(x)\bigr) + q(x)\,y(x)
= -\frac{d}{dx}\bigl((1 - x^2)\,y'(x)\bigr).
\]
In this notation, the Legendre equation becomes
\[
L[y] = \lambda\,w(x)\,y = \lambda\,y.
\]
Thus the Legendre equation is the eigenvalue problem
\[
L[y] = \lambda y,
\]
with $L$ a Sturm–Liouville operator having variable coefficient $p(x) = 1 - x^2$ and weight $w(x)=1$.

\medskip

\textbf{(b) Power series solution and eigenvalues $\lambda = n(n+1)$.}

\emph{Step 1: Derive the recurrence.}
We seek a solution as a power series about $x=0$:
\[
y(x) = \sum_{k=0}^\infty a_k x^k.
\]
Differentiating term by term,
\[
y'(x) = \sum_{k=1}^\infty k\,a_k x^{k-1},\qquad
y''(x) = \sum_{k=2}^\infty k(k-1)\,a_k x^{k-2}.
\]
Substitute these into the differential equation:
\[
(1 - x^2)\,y'' - 2x\,y' + \lambda y = 0.
\]
Compute each part:
\[
(1 - x^2)\,y'' = \sum_{k=2}^\infty k(k-1)a_k x^{k-2} - \sum_{k=2}^\infty k(k-1)a_k x^{k},
\]
and
\[
-2x\,y' = -2\sum_{k=1}^\infty k\,a_k x^{k}.
\]
Therefore
\[
(1 - x^2)\,y'' - 2x\,y'
= \sum_{k=2}^\infty k(k-1)a_k x^{k-2}
- \sum_{k=2}^\infty k(k-1)a_k x^{k}
- 2\sum_{k=1}^\infty k\,a_k x^{k}.
\]
Add $\lambda y = \lambda \sum_{k=0}^\infty a_k x^k$ and require that the total sum vanish.

To combine like powers, first reindex the term with $x^{k-2}$:
\[
\sum_{k=2}^\infty k(k-1)a_k x^{k-2}
= \sum_{j=0}^\infty (j+2)(j+1)\,a_{j+2} x^{j}.
\]
For the $x^k$–terms, we combine:
\[
-\sum_{k=2}^\infty k(k-1)a_k x^k - 2\sum_{k=1}^\infty k\,a_k x^k
= -\sum_{k=2}^\infty \bigl[k(k-1) + 2k\bigr]a_k x^k
= -\sum_{k=2}^\infty k(k+1)\,a_k x^k.
\]
Then including $\lambda y$,
\[
\lambda y = \lambda \sum_{k=0}^\infty a_k x^k.
\]

Thus the entire equation becomes
\[
\sum_{j=0}^\infty (j+2)(j+1)\,a_{j+2} x^{j}
+ \sum_{k=0}^\infty \bigl(\lambda - k(k+1)\bigr)a_k x^k = 0.
\]
We can rename $j$ as $k$ in the first sum. After that, combine the sums:
\[
\sum_{k=0}^\infty \Bigl[(k+2)(k+1)a_{k+2} + \bigl(\lambda - k(k+1)\bigr)a_k\Bigr]x^k = 0.
\]
For this power series to vanish identically, each coefficient must vanish, giving
\[
(k+2)(k+1)a_{k+2} + \bigl(\lambda - k(k+1)\bigr)a_k = 0,\qquad k=0,1,2,\dots
\]
or equivalently,
\[
a_{k+2} = \frac{k(k+1) - \lambda}{(k+2)(k+1)}\,a_k,\qquad k=0,1,2,\dots.
\]
This is the fundamental recurrence relation.

\emph{Step 2: Even and odd subsequences.}
The recurrence couples $a_{k+2}$ only to $a_k$. Thus the even coefficients
\[
a_0, a_2, a_4, \dots
\]
form a closed subsequence, and similarly the odd coefficients
\[
a_1, a_3, a_5, \dots
\]
form another. Concretely,
\[
a_2 \text{ is determined by } a_0,\quad
a_4 \text{ by } a_2,\quad \dots,
\]
and
\[
a_3 \text{ is determined by } a_1,\quad
a_5 \text{ by } a_3,\quad \dots.
\]
Thus, given $a_0$ and $a_1$, the entire solution is determined; the even part and odd part evolve independently.

\emph{Step 3: Termination of the series and quantization of $\lambda$.}
We now impose boundedness at $x = \pm 1$. If the power series does not terminate, the solution grows like an infinite series of powers $x^k$; one can show (for instance, by examining behavior near $x = \pm 1$ or comparing with known special functions) that generic infinite series solutions of Legendre’s equation develop singular behavior as $x \to \pm 1$. In contrast, if the series terminates, we obtain a polynomial, which is automatically bounded on $[-1,1]$.

From the recurrence
\[
a_{k+2} = \frac{k(k+1) - \lambda}{(k+2)(k+1)}\,a_k,
\]
we see that if for some integer $n \ge 0$ the numerator vanishes at $k = n$, namely
\[
n(n+1) - \lambda = 0,
\]
then $a_{n+2} = 0$. Once a coefficient in a subsequence is zero, all later ones in that subsequence vanish as well, because subsequent terms involve multiplication by finite factors. Hence, if
\[
\lambda = n(n+1),
\]
then either the even or the odd subsequence terminates (depending on whether $n$ is even or odd), and we obtain a polynomial solution of degree $n$. These are the \emph{Legendre polynomials} $P_n(x)$.

Conversely, if $\lambda$ is not of the form $n(n+1)$, the recurrence never forces a coefficient to vanish, so the series does not truncate. The corresponding solution is not a polynomial and, upon more detailed analysis, does not stay bounded at both endpoints. Thus the admissible eigenvalues are precisely
\[
\lambda_n = n(n+1),\qquad n=0,1,2,\dots.
\]

\emph{Step 4: Computing the first few $P_n$.}
We can find low-degree polynomials explicitly.

\underline{$n = 0$:} Take $\lambda = 0$. The recurrence becomes
\[
a_{k+2} = \frac{k(k+1)}{(k+2)(k+1)} a_k = \frac{k}{k+2}\,a_k.
\]
If we choose $a_0 = 1$ and $a_1 = 0$, the even sequence is
\[
a_0 = 1,\quad a_2 = 0,\quad a_4 = 0,\dots,
\]
so $y(x) = 1$. This is the degree–$0$ polynomial
\[
P_0(x) = 1.
\]

\underline{$n = 1$:} Take $\lambda = 1\cdot 2 = 2$. The recurrence is
\[
a_{k+2} = \frac{k(k+1) - 2}{(k+2)(k+1)} a_k.
\]
If we choose $a_1 = 1$ and $a_0 = 0$, then
\[
a_3 = \frac{1\cdot 2 - 2}{3\cdot 2}a_1 = 0,\quad
a_5 = 0,\dots,
\]
so only $a_1$ is nonzero in the odd sequence, and we obtain
\[
y(x) = x.
\]
This is the degree–$1$ polynomial
\[
P_1(x) = x.
\]

\underline{$n = 2$:} Take $\lambda = 2\cdot 3 = 6$. Then
\[
a_{k+2} = \frac{k(k+1) - 6}{(k+2)(k+1)} a_k.
\]
Choosing $a_0 = 1$ and $a_1 = 0$ gives for the even sequence
\[
a_2 = \frac{0\cdot 1 - 6}{2\cdot 1} a_0 = -3,\qquad
a_4 = \frac{2\cdot 3 - 6}{4\cdot 3} a_2 = 0,
\]
and higher even coefficients vanish. Hence
\[
y(x) = a_0 + a_2 x^2 = 1 - 3x^2.
\]
Up to normalization, this is $P_2$. The standard choice is to normalize so that $P_n(1)=1$, which is true here, and in fact one usually writes
\[
P_2(x) = \frac{1}{2}(3x^2 - 1).
\]
This differs from $1 - 3x^2$ by an overall factor of $-\tfrac{1}{2}$, which does not affect the property of being an eigenfunction or an orthogonal polynomial. Thus
\[
P_0(x) = 1,\qquad
P_1(x) = x,\qquad
P_2(x) = \frac{1}{2}(3x^2 - 1).
\]

\medskip

\textbf{(c) Orthogonality of $P_m$ and $P_n$ for $m \ne n$.}

The key idea from Sturm–Liouville theory is that eigenfunctions associated with distinct eigenvalues are orthogonal with respect to the weight $w(x)$. Here $w(x) = 1$.

Let $P_n$ and $P_m$ be two bounded solutions of the Legendre equation with parameters $\lambda_n = n(n+1)$ and $\lambda_m = m(m+1)$, where $n \ne m$. They satisfy
\[
\frac{d}{dx}\bigl((1 - x^
2)P_n'(x)\bigr) + \lambda_n P_n(x) = 0,\qquad
\frac{d}{dx}\bigl((1 - x^2)P_m'(x)\bigr) + \lambda_m P_m(x) = 0.
\]
Multiply the first by $P_m$ and the second by $P_n$, then subtract:
\[
P_m\,\frac{d}{dx}\bigl((1 - x^2)P_n'\bigr)
- P_n\,\frac{d}{dx}\bigl((1 - x^2)P_m'\bigr)
+ (\lambda_n - \lambda_m)P_n P_m = 0.
\]
The first two terms combine into a single derivative:
\[
\frac{d}{dx}\Bigl[(1 - x^2)\bigl(P_m P_n' - P_n P_m'\bigr)\Bigr]
+ (\lambda_n - \lambda_m)P_n P_m = 0.
\]
Integrate from $-1$ to $1$:
\[
\int_{-1}^{1}\frac{d}{dx}\Bigl[(1 - x^2)\bigl(P_m P_n' - P_n P_m'\bigr)\Bigr]\,dx
+ (\lambda_n - \lambda_m)\int_{-1}^{1}P_n P_m\,dx = 0.
\]
Thus
\[
(\lambda_n - \lambda_m)\int_{-1}^{1}P_n(x)P_m(x)\,dx
= -\Bigl[(1 - x^2)\bigl(P_m P_n' - P_n P_m'\bigr)\Bigr]_{x=-1}^{x=1}.
\]
But $P_n$ and $P_m$ are polynomials, so $P_n$, $P_m$, $P_n'$, and $P_m'$ are all bounded at $\pm1$, and
\[
1 - x^2 = 0\quad\text{at}\quad x = \pm1,
\]
so the boundary term vanishes:
\[
\Bigl[(1 - x^2)\bigl(P_m P_n' - P_n P_m'\bigr)\Bigr]_{x=-1}^{x=1} = 0.
\]
Hence
\[
(\lambda_n - \lambda_m)\int_{-1}^{1}P_n(x)P_m(x)\,dx = 0.
\]
For $m \ne n$ we have $\lambda_n \ne \lambda_m$, so
\[
\int_{-1}^{1}P_n(x)P_m(x)\,dx = 0.
\]
Thus the Legendre polynomials corresponding to distinct eigenvalues are orthogonal in $L^2(-1,1)$ with respect to the inner product
\[
\langle f,g\rangle = \int_{-1}^{1} f(x)g(x)\,dx.
\]

\medskip

\textbf{Connection with Sturm–Liouville spectral theory.}

In summary:

- The Legendre equation has been written as a Sturm–Liouville eigenvalue problem
  \[
  L[y] = \lambda y,\qquad
  L[y] = -\frac{d}{dx}\bigl((1 - x^2)y'(x)\bigr),
  \]
  on $(-1,1)$, with weight $w(x)=1$ and boundary condition that $y$ remain bounded at $x=\pm1$.

- The boundary condition, together with the self-adjoint structure of $L$, forces a \emph{discrete} spectrum of eigenvalues
  \[
  \lambda_n = n(n+1),\quad n=0,1,2,\dots,
  \]
  obtained by requiring the power series to terminate, yielding polynomial eigenfunctions.

- The corresponding eigenfunctions $P_n$ (Legendre polynomials) are orthogonal with respect to the weight $w(x)$, exactly as predicted by Sturm–Liouville theory for eigenfunctions of a self-adjoint operator.

This example thus exhibits the typical Sturm–Liouville features: a self-adjoint second-order differential operator with variable coefficients, a discrete set of real eigenvalues, and an associated family of orthogonal eigenfunctions (here, the Legendre polynomials), which form the natural basis for expanding solutions of related boundary value problems (such as Laplace’s equation in spherical coordinates).

\end{solution}

% ===== Example 4: Bessel’s equation and radial modes in a circular drum (inquiry-based) =====
\begin{problem}[Bessel’s equation and radial modes in a circular drum]
Consider a thin, circular drumhead of radius $a$ that is tightly clamped along its boundary. Small vertical vibrations of the membrane are governed (to a good approximation) by the two-dimensional wave equation with fixed boundary conditions. In polar coordinates, the geometry suggests that it may be natural to separate variables into a radial part and an angular part. In this problem you will discover that the radial equation is a Sturm–Liouville problem whose eigenfunctions are Bessel functions, and that the allowed frequencies of vibration are determined by the zeros of these Bessel functions.

We take the vertical displacement of the membrane to be $u(r,\theta,t)$, where $(r,\theta)$ are polar coordinates in the disk $0 \le r < a$, $0 \le \theta < 2\pi$, and $t$ is time. The membrane is clamped along the boundary $r=a$, so $u(a,\theta,t)=0$ for all $\theta$ and $t$. Let $c>0$ be the wave speed. The governing equation is
\[
u_{tt} = c^2 \Delta u, \qquad 0\le r<a,\ 0\le\theta<2\pi,\ t>0,
\]
with
\[
u(a,\theta,t)=0, \qquad u \text{ bounded as } r\to 0.
\]

\medskip

(a) Write the Laplacian $\Delta$ in polar coordinates $(r,\theta)$ and rewrite the wave equation in these coordinates. Then seek separated solutions of the form
\[
u(r,\theta,t) = R(r)\,\Theta(\theta)\,T(t).
\]
Carry out the standard separation-of-variables procedure and show that the time factor $T$ satisfies a familiar ordinary differential equation involving a separation constant $\lambda>0$. What equation does $T$ satisfy, and what is the physical interpretation of $\sqrt{\lambda}$?

% Hint: Aim for an equation of the form $T'' + \lambda c^2 T=0$, where $\sqrt{\lambda}c$ plays the role of an angular frequency.

\medskip

(b) Continuing from part (a), separate the spatial variables by writing $w(r,\theta) = R(r)\Theta(\theta)$ and considering the Helmholtz eigenvalue problem
\[
\Delta w + \lambda w = 0.
\]
Show that, after dividing appropriately by $R(r)\Theta(\theta)$, you can separate the angular and radial dependence to obtain an equation of the form
\[
\frac{1}{\Theta}\,\Theta''(\theta) = -m^2
\]
for some constant $m^2$, and a corresponding radial equation. What conditions must $\Theta(\theta)$ satisfy in order for $u$ to be single-valued and $2\pi$-periodic in $\theta$? What values of $m$ are therefore allowed?

% Hint: Enforce $\Theta(\theta+2\pi)=\Theta(\theta)$ and show that this periodicity forces $m$ to be an integer.

\medskip

(c) Show that, for each integer $m\ge 0$, the radial function $R(r)$ satisfies an equation of the form
\[
r^2 R''(r) + r R'(r) + \bigl(k^2 r^2 - m^2\bigr) R(r) = 0, \qquad 0<r<a,
\]
for an appropriate constant $k>0$ depending on $\lambda$. Recognize this as Bessel's differential equation of order $m$, and write the general solution in terms of standard Bessel functions. Which two linearly independent Bessel-type functions appear, and what is their behavior as $r\to 0$?

% Hint: Recall that Bessel's equation of order $m$ is $x^2 y'' + x y' + (x^2 - m^2)y=0$, with independent solutions $J_m(x)$ and $Y_m(x)$. Compare $x=kr$ with the equation above.

\medskip

(d) Impose the physical and boundary conditions on the radial function $R(r)$.

\quad(i) Use the requirement that $u$ be finite (and physically reasonable) at $r=0$ to decide which combination of the two Bessel solutions from part (c) is admissible.

\quad(ii) Impose the clamped boundary condition $u(a,\theta,t)=0$ to obtain a condition on $R(a)$. Show that this condition forces $k a$ to be a zero of the Bessel function $J_m$. Denote by $j_{m,n}$ the $n$-th positive zero of $J_m$, and deduce that the allowed values of $k$ are $k_{m,n} = j_{m,n}/a$. 

\quad(iii) Conclude that the corresponding spatial eigenfunctions for the vibrating drum can be written (up to normalization) as
\[
w_{m,n}(r,\theta) = J_m\!\bigl(j_{m,n} r/a\bigr)\,\bigl(A\cos(m\theta)+B\sin(m\theta)\bigr),
\]
and that the associated temporal frequencies are
\[
\omega_{m,n} = c\,\frac{j_{m,n}}{a}.
\]

% Hint: For (i), recall that $Y_m(x)$ blows up as $x\to 0$, whereas $J_m(x)$ remains finite. For (ii), $R(a)=0$ gives $J_m(ka)=0$.

\medskip

(e) Sturm–Liouville structure and orthogonality.

\quad(i) Rewrite the radial equation from part (c) in Sturm–Liouville form
\[
\frac{d}{dr}\!\left(p(r)\,\frac{dR}{dr}\right) + \bigl(\lambda\,w(r) - q(r)\bigr)R(r) = 0,
\]
for appropriate functions $p(r)$, $w(r)$, and $q(r)$. Identify these three functions explicitly.

\quad(ii) Using the general Sturm–Liouville theory, argue (without detailed proof of the general theorem) that the radial eigenfunctions corresponding to distinct eigenvalues are orthogonal with respect to a weight. What is the weight function here, and how does this lead to an orthogonality relation of the form
\[
\int_0^a r\, J_m\!\bigl(j_{m,n} r/a\bigr)\,J_m\!\bigl(j_{m,\ell} r/a\bigr)\,dr = 0 \quad\text{when } n\neq \ell?
\]

\medskip

(f) What if / extensions.

\quad(i) Suppose instead that the membrane were \emph{free} at the boundary $r=a$, so that the radial derivative of $u$ vanishes there. How would the boundary condition on $R$ change, and what condition on $J_m$ (or its derivative) would now select the allowed eigenvalues?

\quad(ii) Consider an annular membrane with inner radius $r=b>0$ and outer radius $r=a>b$, both clamped. How would the radial solution change qualitatively? In particular, would you still discard one of the two Bessel-type solutions from part (c)? Briefly explain.

% Hint: On an annulus, the origin is no longer part of the domain, so singular behavior at $r=0$ may not be relevant. Both $J_m$ and $Y_m$ can occur in the general solution, with boundary conditions at $r=b$ and $r=a$ determining the allowed eigenvalues.
\end{problem}

% ===== Example 4: Bessel’s equation and radial modes in a circular drum (full solution) =====
\begin{problem}[Bessel’s equation and radial modes in a circular drum]
Consider a circular membrane of radius $a$, clamped along its boundary, whose small transverse vibrations satisfy
\[
u_{tt} = c^2 \Delta u \quad \text{in } \{(r,\theta): 0\le r<a,\ 0\le\theta<2\pi\},
\]
with boundary condition $u(a,\theta,t)=0$ and $u$ bounded at $r=0$. 

(a) Write the Laplacian in polar coordinates and seek separated solutions $u(r,\theta,t)=R(r)\Theta(\theta)T(t)$. Show that the spatial part $w(r,\theta)=R(r)\Theta(\theta)$ satisfies the Helmholtz equation
\[
\Delta w + \lambda w = 0,
\]
and that the angular factor satisfies $\Theta'' + m^2 \Theta=0$ with $m\in\mathbb{Z}$.

(b) Show that the corresponding radial factor satisfies
\[
r^2 R''(r) + r R'(r) + \bigl(k^2 r^2 - m^2\bigr) R(r) = 0,\qquad 0<r<a,
\]
where $k^2=\lambda>0$. Identify this as Bessel’s equation of order $m$ and write the general solution in terms of $J_m$ and $Y_m$. Use the boundedness condition at $r=0$ to select the admissible solution.

(c) Impose the boundary condition $u(a,\theta,t)=0$ to show that $k a$ must be a zero of $J_m$, say $j_{m,n}$. Deduce that the allowed wave numbers are $k_{m,n}=j_{m,n}/a$ and that, up to normalization,
\[
w_{m,n}(r,\theta) = J_m\!\bigl(j_{m,n} r/a\bigr)\bigl(A\cos(m\theta)+B\sin(m\theta)\bigr).
\]
Find the corresponding temporal frequencies $\omega_{m,n}$.

(d) Rewrite the radial equation in Sturm–Liouville form and identify $p(r)$, $w(r)$, and $q(r)$. Using Sturm–Liouville theory, state the orthogonality relation for the set $\{J_m(j_{m,n} r/a)\}_{n=1}^\infty$ on $[0,a]$, including the appropriate weight function.

\end{problem}

\begin{solution}
We analyse the membrane vibrations by separation of variables in polar coordinates and then interpret the resulting radial equation as a Sturm–Liouville problem.

\medskip

\textbf{(a) Separation and angular dependence.}
In polar coordinates $(r,\theta)$, the Laplacian is
\[
\Delta u \;=\; u_{rr} + \frac{1}{r}u_r + \frac{1}{r^2}u_{\theta\theta}.
\]
The wave equation therefore becomes
\[
u_{tt} \;=\; c^2\left(u_{rr} + \frac{1}{r}u_r + \frac{1}{r^2}u_{\theta\theta}\right).
\]

We look for separated solutions of the form
\[
u(r,\theta,t) = R(r)\,\Theta(\theta)\,T(t).
\]
Substituting into the wave equation and dividing by $R\Theta T$ gives
\[
\frac{T''(t)}{c^2 T(t)} = \frac{R''(r)}{R(r)} + \frac{1}{r}\frac{R'(r)}{R(r)} + \frac{1}{r^2}\frac{\Theta''(\theta)}{\Theta(\theta)}.
\]
The left-hand side depends only on $t$, while the right-hand side depends only on $r$ and $\theta$. Hence both sides must equal a constant, say $-\lambda$, with $\lambda>0$ chosen for convenience:
\[
\frac{T''}{c^2 T} = -\lambda.
\]
Thus $T$ satisfies
\[
T''(t) + \lambda c^2 T(t) = 0.
\]
This is the harmonic oscillator equation. Its solutions are sinusoidal with angular frequency $\omega = c\sqrt{\lambda}$, so $\sqrt{\lambda}$ represents the spatial wave number (up to the factor $c$).

The spatial factor $w(r,\theta)=R(r)\Theta(\theta)$ then satisfies
\[
\Delta w + \lambda w = 0,
\]
the Helmholtz eigenvalue problem for the Laplacian on the disk.

To separate $r$ and $\theta$ in the spatial equation, we write
\[
\Delta w + \lambda w = 0 \quad\Rightarrow\quad
R''\Theta + \frac{1}{r}R'\Theta + \frac{1}{r^2}R\Theta'' + \lambda R\Theta = 0.
\]
Dividing by $R\Theta$ gives
\[
\frac{R''}{R} + \frac{1}{r}\frac{R'}{R} + \lambda
+ \frac{1}{r^2}\frac{\Theta''}{\Theta} = 0.
\]
Multiplying by $r^2$,
\[
r^2\left(\frac{R''}{R} + \frac{1}{r}\frac{R'}{R} + \lambda\right) + \frac{\Theta''}{\Theta} = 0.
\]
The first term depends only on $r$, and the second only on $\theta$, so each must equal a constant. Denote this constant by $-m^2$:
\[
\frac{\Theta''(\theta)}{\Theta(\theta)} = -m^2,
\]
and correspondingly
\[
r^2\left(\frac{R''}{R} + \frac{1}{r}\frac{R'}{R} + \lambda\right) = m^2.
\]

The angular equation is then
\[
\Theta''(\theta) + m^2 \Theta(\theta) = 0.
\]
Because the physical displacement must be single-valued and periodic in $\theta$ with period $2\pi$, we require $\Theta(\theta+2\pi)=\Theta(\theta)$ for all $\theta$. The general solution is
\[
\Theta(\theta) = A\cos(m\theta) + B\sin(m\theta).
\]
This is $2\pi$-periodic if and only if $m$ is an integer. Hence $m\in\mathbb{Z}$.

\medskip

\textbf{(b) The radial equation and Bessel’s equation.}
From the separated form above,
\[
r^2\left(\frac{R''}{R} + \frac{1}{r}\frac{R'}{R} + \lambda\right) = m^2.
\]
Multiplying through by $R$ and rearranging yields
\[
r^2 R''(r) + r R'(r) + \bigl(\lambda r^2 - m^2\bigr) R(r) = 0.
\]
It is customary to write $\lambda = k^2$ with $k>0$, so the equation becomes
\[
r^2 R''(r) + r R'(r) + \bigl(k^2 r^2 - m^2\bigr) R(r) = 0,\qquad 0<r<a.
\]

This is exactly Bessel’s differential equation of order $m$, with independent variable $kr$ in place of the usual $x$. Indeed, the standard Bessel equation of order $\nu$ is
\[
x^2 y''(x) + x y'(x) + \bigl(x^2 - \nu^2\bigr)y(x) = 0.
\]
By setting $x = kr$ and $\nu = m$, we see that $R(r)$ as a function of $kr$ satisfies this equation. The two linearly independent standard solutions are the Bessel functions of the first and second kind, $J_m(x)$ and $Y_m(x)$. Thus the general radial solution is
\[
R(r) = A J_m(kr) + B Y_m(kr),
\]
for constants $A$ and $B$.

\medskip

\textbf{(c) Regularity at the origin and boundary condition at $r=a$.}
We next apply the physical and boundary conditions.

\emph{Behavior at $r=0$.} The displacement must remain finite at the center of the drum, so $R(r)$ should be bounded as $r\to 0$. The known asymptotics of Bessel functions near the origin are:
\[
J_m(x) \sim \frac{1}{m!}\left(\frac{x}{2}\right)^m \quad\text{as } x\to 0,
\]
so $J_m(x)$ is bounded (and in fact analytic) at $x=0$. On the other hand, for $m\ge 0$,
\[
Y_m(x) \sim 
\begin{cases}
\frac{2}{\pi}\left(\log x + \gamma\right), & m=0,\\[4pt]
-\dfrac{(m-1)!}{\pi}\left(\dfrac{2}{x}\right)^m, & m\ge 1,
\end{cases}
\quad\text{as } x\to 0,
\]
so $Y_m(x)$ either diverges logarithmically (for $m=0$) or like a negative power of $x$ (for $m\ge 1$). In either case, $Y_m(kr)$ is unbounded near $r=0$.

To have a physically reasonable (bounded) solution at the center, we must set $B=0$. Thus
\[
R(r) = A J_m(kr).
\]

\emph{Boundary condition at $r=a$.} The membrane is clamped along the boundary, so $u(a,\theta,t)=0$ for all $\theta$ and $t$. For separated solutions $u=R\Theta T$ with $T$ not identically zero, this requires
\[
R(a)\,\Theta(\theta)\,T(t) = 0 \quad \text{for all }\theta,t.
\]
To avoid trivial angular and temporal factors, we must have
\[
R(a) = 0.
\]
With $R(r)=A J_m(kr)$, this becomes
\[
J_m(ka) = 0.
\]
Thus the allowed values of $k$ are precisely those for which $ka$ is a zero of $J_m$. Let $j_{m,n}$ denote the $n$-th positive zero of $J_m$. Then
\[
k_{m,n}a = j_{m,n} \quad\Longrightarrow\quad k_{m,n} = \frac{j_{m,n}}{a}.
\]

Therefore the admissible radial eigenfunctions are, up to normalization,
\[
R_{m,n}(r) = J_m\!\bigl(j_{m,n} r/a\bigr).
\]
Combining the radial and angular factors, the spatial eigenfunctions are
\[
w_{m,n}(r,\theta) = J_m\!\bigl(j_{m,n} r/a\bigr)\,\bigl(A\cos(m\theta)+B\sin(m\theta)\bigr),
\]
for integers $m\ge 0$ and $n\ge 1$.

\emph{Temporal frequencies.} Returning to the time equation, we had
\[
T''(t) + \lambda c^2 T(t) = 0,\qquad \lambda = k^2.
\]
For the mode with wave number $k_{m,n}=j_{m,n}/a$, the equation is
\[
T''(t) + c^2 k_{m,n}^2 T(t) = 0,
\]
so the temporal angular frequency is
\[
\omega_{m,n} = c\,k_{m,n} = c\,\frac{j_{m,n}}{a}.
\]
Each pair $(m,n)$ labels a distinct normal mode of vibration with frequency $\omega_{m,n}$.

\medskip

\textbf{(d) Sturm–Liouville form and orthogonality.}
We now recast the radial equation as a Sturm–Liouville problem and note the corresponding orthogonality properties.

The radial equation with eigenvalue parameter $\lambda=k^2$ is
\[
r^2 R''(r) + r R'(r) + \bigl(\lambda r^2 - m^2\bigr) R(r) = 0.
\]
Divide through by $r$:
\[
r R''(r) + R'(r) + \bigl(\lambda r - \tfrac{m^2}{r}\bigr) R(r) = 0.
\]
This may be written as
\[
\frac{d}{dr}\!\bigl(r R'(r)\bigr) + \bigl(\lambda r - \tfrac{m^2}{r}\bigr)R(r) = 0.
\]
Thus the equation has Sturm–Liouville form
\[
\frac{d}{dr}\!\left(p(r)\,\frac{dR}{dr}\right) + \bigl(\lambda\,w(r) - q(r)\bigr)R(r) = 0
\]
with
\[
p(r) = r,\qquad w(r) = r,\qquad q(r) = \frac{m^2}{r}.
\]
The interval is $(0,a)$, with boundary conditions
\[
R(0) \text{ bounded}, \qquad R(a)=0.
\]
These conditions define a regular (after a suitable interpretation at $r=0$) self-adjoint Sturm–Liouville problem. Sturm–Liouville theory then asserts, among other things, that:

\begin{itemize}
\item All eigenvalues $\lambda_{m,n}=k_{m,n}^2=(j_{m,n}/a)^2$ are real and form an increasing discrete sequence with no finite accumulation point.
\item Eigenfunctions corresponding to distinct eigenvalues are orthogonal with respect to the weight $w(r)=r$ on $(0,a)$.
\end{itemize}

For fixed $m$, the radial eigenfunctions are $R_{m,n}(r) = J_m(j_{m,n} r/a)$ (up to scaling). The orthogonality statement takes the explicit form
\[
\int_0^a r\, J_m\!\bigl(j_{m,n} r/a\bigr)\, J_m\!\bigl(j_{m,\ell} r/a\bigr)\, dr = 0,
\quad\text{whenever } n\neq \ell.
\]
This weighted orthogonality is exactly the radial version of the standard Sturm–Liouville inner product
\[
\langle f,g\rangle = \int_0^a f(r)\,\overline{g(r)}\,w(r)\,dr,
\]
with weight $w(r)=r$.

\medskip

\textbf{Conceptual summary.}
This example illustrates the core ideas of Sturm–Liouville (spectral) theory in a physically motivated setting:

\begin{itemize}
\item Separation of variables for the wave equation on a disk leads to an eigenvalue problem for the Laplacian, $\Delta w + \lambda w=0$, on a bounded domain with Dirichlet boundary conditions.
\item The geometry in polar coordinates transforms the radial part of the Laplacian into a Sturm–Liouville operator with weight $w(r)=r$.
\item Regularity at the origin and boundary conditions at $r=a$ single out a discrete set of eigenvalues $\lambda_{m,n}=(j_{m,n}/a)^2$ corresponding to zeros of Bessel functions $J_m$.
\item The associated eigenfunctions $J_m(j_{m,n} r/a)$ are orthogonal with respect to the weight $r$, which allows expansions of general initial data into series of normal modes.
\end{itemize}

Thus, the vibration modes of a circular drum provide a canonical example of how Sturm–Liouville theory, special functions (Bessel functions), and physical boundary conditions combine to produce a discrete spectrum of eigenvalues and an orthogonal basis of eigenfunctions.
\end{solution}

% ===== Example 5: Transforming a general second-order ODE into Sturm–Liouville form (inquiry-based) =====
\begin{problem}[Transforming a general second-order ODE into Sturm–Liouville form]
Many boundary value problems in applications lead to second-order equations of the form
\[
a(x) y''(x) + b(x) y'(x) + c(x) y(x) = \lambda\, r(x) y(x)
\]
on some interval $[\,\alpha,\beta\,]$, where $a,b,c,r$ are given real-valued coefficient functions and $\lambda$ is an eigenvalue parameter. At first sight this operator is not symmetric with respect to the standard $L^2$ inner product, mainly because of the first-derivative term $b(x) y'$. However, in many cases one can reveal a hidden self-adjoint (Sturm–Liouville) structure by multiplying the equation by a carefully chosen integrating factor. This leads to a weighted inner product and clarifies how the boundary conditions must be chosen to obtain a self-adjoint eigenvalue problem with real eigenvalues and orthogonal eigenfunctions.

In this problem you will discover, step by step, how to transform a general second-order linear ODE into Sturm–Liouville form, and how the weight function and boundary terms emerge from this transformation.

\smallskip

(a) Recall that a (regular) Sturm–Liouville problem on $[\alpha,\beta]$ typically has the form
\[
\frac{d}{dx}\!\bigl(p(x) y'(x)\bigr) + \bigl[\lambda\, w(x) - q(x)\bigr] y(x) = 0,
\]
together with separated boundary conditions, for example
\[
\alpha_1 y(\alpha) + \alpha_2 y'(\alpha) = 0, 
\qquad 
\beta_1 y(\beta) + \beta_2 y'(\beta) = 0,
\]
where $p,w,q$ are given real-valued functions with $p(x)>0$ and $w(x)>0$ on $[\alpha,\beta]$.

\begin{enumerate}
\item[(i)] Rewrite this Sturm–Liouville equation in the form
\[
L[y] = \lambda\, w(x) y,
\]
and identify explicitly the differential operator $L$.
\item[(ii)] Using integration by parts (formal calculations are fine), show that for sufficiently smooth functions $u$ and $v$ one has a Green-type identity of the form
\[
\int_\alpha^\beta \bigl(u\,L[v] - v\,L[u]\bigr)\,dx 
= \Bigl[\,p(x)\,\bigl(u(x)v'(x) - u'(x)v(x)\bigr)\Bigr]_{\alpha}^{\beta}.
\]
(Hint: Start from $u\,(p v')' - v\,(p u')'$ and integrate over $[\alpha,\beta]$.)
\item[(iii)] Explain briefly why, if the boundary conditions ensure that the boundary term on the right-hand side always vanishes for admissible $u$ and $v$, then the operator $L$ is formally self-adjoint with respect to the standard inner product
\[
\langle u, v\rangle_{L^2_w} \;=\; \int_{\alpha}^{\beta} u(x)\,\overline{v(x)}\,w(x)\,dx
\]
up to the weight $w(x)$.
\end{enumerate}

\medskip

(b) Now consider the more general eigenvalue problem
\[
a(x) y''(x) + b(x) y'(x) + c(x) y(x) = \lambda\, r(x) y(x), \qquad \alpha < x < \beta,
\]
with real-valued continuous functions $a,b,c,r$ on $[\alpha,\beta]$, and suppose that
\[
a(x) > 0, \qquad r(x) > 0 \quad \text{for all } x\in[\alpha,\beta].
\]
We would like to transform this equation into Sturm–Liouville form by multiplying by a suitable integrating factor.

\begin{enumerate}
\item[(i)] Suppose we multiply the equation by an unknown, positive function $\mu(x)$ and set
\[
p(x) := \mu(x)\,a(x).
\]
Show that
\[
\mu(x)\,a(x)\,y''(x) + \mu(x)\,b(x)\,y'(x)
\]
can be written as $\dfrac{d}{dx}\bigl(p(x) y'(x)\bigr)$ if and only if $p$ satisfies a certain first-order differential equation. Write down this differential equation in the form
\[
\bigl(\mu(x)\,a(x)\bigr)' = \mu(x)\,b(x),
\]
and then expand this derivative to obtain a first-order linear ODE for $\mu(x)$ alone.
\item[(ii)] Rearrange the resulting equation into the form
\[
\frac{\mu'(x)}{\mu(x)} 
= \frac{b(x) - a'(x)}{a(x)},
\]
and explain why this is analogous to finding an integrating factor for a first-order linear ODE.
\end{enumerate}

\medskip

(c) Continue with part (b). 

\begin{enumerate}
\item[(i)] Solve the first-order ODE
\[
\frac{\mu'(x)}{\mu(x)} 
= \frac{b(x) - a'(x)}{a(x)}
\]
to obtain an explicit formula for $\mu(x)$ in terms of $a$ and $b$. You may leave your answer as an exponential of an integral.
% Hint: Treat this as a separable ODE for \mu, or equivalently integrate d(\ln \mu).

\item[(ii)] Using your expression for $\mu(x)$, define
\[
p(x) := \mu(x) a(x),
\qquad
q(x) := -\,\mu(x)c(x),
\qquad
w(x) := \mu(x) r(x).
\]
Show that the original eigenvalue equation can now be written in the Sturm–Liouville form
\[
\frac{d}{dx}\bigl(p(x) y'(x)\bigr) + \bigl[\lambda\, w(x) - q(x)\bigr] y(x) = 0.
\]
(Hint: Multiply the original equation by $\mu$ and use your defining relation for $p$.)
\end{enumerate}

\medskip

(d) In this step, connect the transformation to self-adjointness and orthogonality.

\begin{enumerate}
\item[(i)] Using the functions $p,q,w$ from part (c), write down the associated Sturm–Liouville operator
\[
L_{\text{SL}}[y] := -\frac{d}{dx}\bigl(p(x) y'(x)\bigr) + q(x) y(x),
\]
so that the eigenvalue problem becomes
\[
L_{\text{SL}}[y] = \lambda\, w(x)\,y.
\]
Explain briefly why this operator is formally self-adjoint with respect to the weighted inner product
\[
\langle u, v\rangle_{L^2_w} 
= \int_{\alpha}^{\beta} u(x)\,\overline{v(x)}\,w(x)\,dx,
\]
provided that the boundary terms from the Green-type identity vanish for admissible functions $u$ and $v$.

\item[(ii)] Suppose we impose separated boundary conditions of the form
\[
\alpha_1 y(\alpha) + \alpha_2 y'(\alpha) = 0, 
\qquad 
\beta_1 y(\beta) + \beta_2 y'(\beta) = 0,
\]
with real constants $\alpha_1,\alpha_2,\beta_1,\beta_2$ not both zero at each endpoint. Show that the boundary term
\[
p(x)\,\bigl(u(x)v'(x)-u'(x)v(x)\bigr)\Big|_{x=\alpha}^{x=\beta}
\]
vanishes for all pairs of functions $u$ and $v$ satisfying the same boundary conditions, and hence the problem is self-adjoint. 
% Hint: Express u'(\alpha) and v'(\alpha) in terms of u(\alpha), v(\alpha) using the boundary conditions (when \alpha_2\neq 0); treat cases \alpha_2=0 or \beta_2=0 separately.

\item[(iii)] Explain, at a high level, how self-adjointness of $L_{\text{SL}}$ with respect to $\langle\cdot,\cdot\rangle_{L^2_w}$ leads to the reality of eigenvalues and orthogonality of eigenfunctions corresponding to different eigenvalues.
\end{enumerate}

\medskip

(e) Explore a few variations and extensions.

\begin{enumerate}
\item[(i)] What can go wrong in the transformation if $a(x)$ vanishes somewhere in $[\alpha,\beta]$? What if $r(x)$ changes sign? Discuss how these issues affect the definition of $p(x)$ and the interpretation of $w(x)$ as a weight function.
\item[(ii)] Apply the transformation you derived to the concrete example
\[
y''(x) + \frac{2}{x} y'(x) + \lambda\, y(x) = 0, \qquad 0 < x < 1,
\]
with the understanding that $y$ remains bounded as $x\to 0^+$ and satisfies $y(1)=0$. Find explicit formulas for $p(x)$ and $w(x)$, and write the equation in Sturm–Liouville form. How do the boundary conditions look in this new formulation?
% Hint: Here a(x)=1, b(x)=2/x, c(x)=0, r(x)=1; compute \mu(x) using your general formula, then p(x)=\mu a and w(x)=\mu r.
\end{enumerate}

\end{problem}

% ===== Example 5: Transforming a general second-order ODE into Sturm–Liouville form (full solution) =====
\begin{problem}[Transforming a general second-order ODE into Sturm–Liouville form]
Let $a,b,c,r$ be real-valued continuous functions on $[\alpha,\beta]$ with
\[
a(x) > 0, \qquad r(x) > 0 \quad \text{for all } x\in[\alpha,\beta].
\]
Consider the eigenvalue problem
\[
a(x) y''(x) + b(x) y'(x) + c(x) y(x) = \lambda\, r(x) y(x), \qquad \alpha < x < \beta,
\]
together with separated boundary conditions
\[
\alpha_1 y(\alpha) + \alpha_2 y'(\alpha) = 0, 
\qquad 
\beta_1 y(\beta) + \beta_2 y'(\beta) = 0,
\]
where $\alpha_1,\alpha_2$ are not both zero, and likewise for $\beta_1,\beta_2$.

\begin{enumerate}
\item[(a)] Show that there exists a positive integrating factor $\mu(x)$ such that, with
\[
p(x) := \mu(x)a(x), 
\qquad 
q(x) := -\,\mu(x)c(x),
\qquad
w(x) := \mu(x) r(x),
\]
the equation can be rewritten in Sturm–Liouville form
\[
\frac{d}{dx}\bigl(p(x) y'(x)\bigr) + \bigl[\lambda\, w(x) - q(x)\bigr] y(x) = 0.
\]
Derive an explicit formula for $\mu(x)$ in terms of $a$ and $b$.

\item[(b)] Write the corresponding Sturm–Liouville operator
\[
L_{\text{\rm SL}}[y] := -\frac{d}{dx}\bigl(p(x)y'(x)\bigr) + q(x) y(x)
\]
and verify the Green-type identity
\[
\int_\alpha^\beta \bigl(u\,L_{\text{\rm SL}}[v] - v\,L_{\text{\rm SL}}[u]\bigr)\,dx
= \Bigl[\,p(x)\,\bigl(u(x)v'(x) - u'(x)v(x)\bigr)\Bigr]_{\alpha}^{\beta}
\]
for sufficiently smooth $u$ and $v$.

Explain why, under the given separated boundary conditions, the boundary term vanishes for all admissible $u$ and $v$, so that $L_{\text{\rm SL}}$ is self-adjoint with respect to the weighted inner product
\[
\langle u, v\rangle_{L^2_w} = \int_{\alpha}^{\beta} u(x)\,\overline{v(x)}\,w(x)\,dx.
\]

\item[(c)] Briefly discuss how this Sturm–Liouville formulation implies that all eigenvalues $\lambda$ are real and that eigenfunctions corresponding to distinct eigenvalues are orthogonal with respect to $\langle\cdot,\cdot\rangle_{L^2_w}$.
\end{enumerate}
\end{problem}

\begin{solution}
We begin with the general eigenvalue problem
\[
a(x) y''(x) + b(x) y'(x) + c(x) y(x) = \lambda\, r(x) y(x),
\qquad \alpha < x < \beta,
\]
where $a(x)>0$ and $r(x)>0$ for all $x\in[\alpha,\beta]$. The central idea is to multiply the equation by an integrating factor $\mu(x)$ so that the left-hand side becomes the derivative of a flux $p(x)y'(x)$ plus a zeroth-order term. This is precisely the structure of a Sturm–Liouville operator.

\medskip

\emph{(a) Construction of the integrating factor and Sturm–Liouville form.}

Multiply the differential equation by an unknown function $\mu(x)$:
\[
\mu a\,y'' + \mu b\,y' + \mu c\,y = \lambda\,\mu r\,y.
\]
We would like the second- and first-derivative terms to combine into a single derivative:
\[
\mu a\,y'' + \mu b\,y' = \frac{d}{dx}\bigl(p(x) y'(x)\bigr)
\]
for some function $p$. It is natural to set
\[
p(x) := \mu(x)a(x).
\]
Then
\[
\frac{d}{dx}\bigl(p y'\bigr) = p\,y'' + p'\,y' = \mu a\,y'' + p'\,y'.
\]
For this to match $\mu a\,y'' + \mu b\,y'$, we must have
\[
p'(x) = \mu(x)\,b(x).
\]
Substituting $p=\mu a$, this condition becomes
\[
(\mu a)' = \mu b.
\]
Expanding the derivative gives
\[
\mu' a + \mu a' = \mu b.
\]
Because $a>0$, we can divide by $a$ and by $\mu$ (we will seek $\mu>0$) to obtain a first-order linear ODE for $\mu$:
\[
\frac{\mu'}{\mu} = \frac{b - a'}{a}.
\]
This is a separable equation. Integrating, we find
\[
\ln \mu(x) 
= \int^x \frac{b(s) - a'(s)}{a(s)}\,ds + C_0,
\]
for some constant $C_0$. Exponentiating,
\[
\mu(x) = C \exp\!\left(\int^x \frac{b(s) - a'(s)}{a(s)}\,ds\right),
\]
where $C = e^{C_0} > 0$ can be chosen arbitrarily. Thus we have an explicit formula for the integrating factor $\mu(x)$ in terms of $a$ and $b$.

We now define
\[
p(x) := \mu(x) a(x), 
\qquad 
q(x) := -\,\mu(x) c(x),
\qquad
w(x) := \mu(x) r(x).
\]
Note that $p(x)>0$ and $w(x)>0$ because $a,r>0$ and we choose $\mu>0$.

With these definitions, the multiplied equation reads
\[
\mu a\,y'' + \mu b\,y' + \mu c\,y = \lambda\, \mu r\,y
\;\;\Longleftrightarrow\;\;
\mu a\,y'' + \mu b\,y' - q\,y = \lambda\, w\,y.
\]
By construction, $\mu a\,y'' + \mu b\,y' = (p y')'$, so the left-hand side becomes
\[
(p y')' - q y.
\]
Thus the eigenvalue equation can be written as
\[
(p y')' - q(x) y(x) = \lambda\, w(x)y(x),
\]
or equivalently,
\[
\frac{d}{dx}\bigl(p(x) y'(x)\bigr) + \bigl[\lambda\, w(x) - q(x)\bigr]\,y(x) = 0.
\]
This is precisely a Sturm–Liouville form with coefficients $p,w,q$ as above and $p,w>0$ on $[\alpha,\beta]$.

\medskip

\emph{(b) Green’s identity and self-adjointness.}

We now introduce the associated Sturm–Liouville operator
\[
L_{\text{SL}}[y] := -\frac{d}{dx}\bigl(p(x) y'(x)\bigr) + q(x) y(x),
\]
so that the eigenvalue problem is
\[
L_{\text{SL}}[y] = \lambda\, w(x)\,y(x).
\]

To obtain the Green-type identity, take sufficiently smooth functions $u$ and $v$ on $[\alpha,\beta]$. Compute
\[
u\,L_{\text{SL}}[v] - v\,L_{\text{SL}}[u]
= u\Bigl(- (p v')' + q v\Bigr)
- v\Bigl(- (p u')' + q u\Bigr).
\]
The zeroth-order terms cancel:
\[
u q v - v q u = 0.
\]
Thus
\[
u\,L_{\text{SL}}[v] - v\,L_{\text{SL}}[u]
= -u (p v')' + v (p u')'.
\]
Integrate over $[\alpha,\beta]$:
\[
\int_{\alpha}^{\beta} \bigl(u\,L_{\text{SL}}[v] - v\,L_{\text{SL}}[u]\bigr)\,dx
= \int_{\alpha}^{\beta} \bigl( -u (p v')' + v (p u')'\bigr)\,dx.
\]

Now integrate by parts each term. For the first term,
\[
\int_{\alpha}^{\beta} -u (p v')'\,dx
= \Bigl[-u\,p\,v'\Bigr]_{\alpha}^{\beta}
+ \int_{\alpha}^{\beta} u'\,p\,v'\,dx.
\]
For the second term,
\[
\int_{\alpha}^{\beta} v (p u')'\,dx
= \Bigl[v\,p\,u'\Bigr]_{\alpha}^{\beta}
- \int_{\alpha}^{\beta} v'\,p\,u'\,dx.
\]
Adding these,
\[
\int_{\alpha}^{\beta} \bigl( -u (p v')' + v (p u')'\bigr)\,dx
= \Bigl[-u\,p\,v' + v\,p\,u'\Bigr]_{\alpha}^{\beta}
+ \int_{\alpha}^{\beta} \bigl(u' p v' - v' p u'\bigr)\,dx.
\]
The integral term cancels identically:
\[
u' p v' - v' p u' = 0.
\]
Therefore
\[
\int_{\alpha}^{\beta} \bigl(u\,L_{\text{SL}}[v] - v\,L_{\text{SL}}[u]\bigr)\,dx
= \Bigl[\,p(x)\,\bigl(v(x)u'(x) - u(x)v'(x)\bigr)\Bigr]_{\alpha}^{\beta},
\]
or, rearranging signs,
\[
\int_{\alpha}^{\beta} \bigl(u\,L_{\text{SL}}[v] - v\,L_{\text{SL}}[u]\bigr)\,dx
= \Bigl[\,p(x)\,\bigl(u(x)v'(x) - u'(x)v(x)\bigr)\Bigr]_{\alpha}^{\beta}.
\]
This is the Green-type identity.

Next, we incorporate the weight $w(x)$. The eigenvalue problem is
\[
L_{\text{SL}}[y] = \lambda\, w y.
\]
The natural inner product is
\[
\langle u, v\rangle_{L^2_w} 
= \int_{\alpha}^{\beta} u(x)\,\overline{v(x)}\,w(x)\,dx,
\]
with
\(w(x)>0\). For the operator that appears naturally in the eigenvalue problem
\[
\frac{1}{w(x)}L_{\text{SL}}[y] = \lambda\,y,
\]
the appropriate notion of (formal) self-adjointness is with respect to the weighted inner product
\[
\langle u, v\rangle_{L^2_w} 
= \int_{\alpha}^{\beta} u(x)\,\overline{v(x)}\,w(x)\,dx.
\]

Define
\[
A[y] := \frac{1}{w(x)}\,L_{\text{SL}}[y].
\]
Then
\[
\langle u, A[v]\rangle_{L^2_w}
= \int_{\alpha}^{\beta} u\,\overline{A[v]}\,w\,dx
= \int_{\alpha}^{\beta} u\,\overline{L_{\text{SL}}[v]}\,dx,
\]
and similarly
\[
\langle A[u], v\rangle_{L^2_w}
= \int_{\alpha}^{\beta} v\,\overline{L_{\text{SL}}[u]}\,dx.
\]
If we restrict to real-valued functions (or take real parts), the Green identity gives
\[
\langle u, A[v]\rangle_{L^2_w} - \langle A[u], v\rangle_{L^2_w}
= \Bigl[\,p(x)\,\bigl(u(x)v'(x) - u'(x)v(x)\bigr)\Bigr]_{\alpha}^{\beta}.
\]
Thus, whenever the boundary term on the right-hand side vanishes for admissible \(u,v\), the operator \(A=(1/w)L_{\text{SL}}\) is formally self-adjoint with respect to \(\langle\cdot,\cdot\rangle_{L^2_w}\).

\medskip

\emph{Boundary conditions and vanishing of the boundary term.}

Now impose separated boundary conditions
\[
\alpha_1 y(\alpha) + \alpha_2 y'(\alpha) = 0, 
\qquad 
\beta_1 y(\beta) + \beta_2 y'(\beta) = 0,
\]
with \((\alpha_1,\alpha_2)\neq(0,0)\) and \((\beta_1,\beta_2)\neq(0,0)\). Let \(u\) and \(v\) be any two functions satisfying these same conditions.

At \(x=\alpha\), we have
\[
\alpha_1 u(\alpha) + \alpha_2 u'(\alpha) = 0,
\qquad
\alpha_1 v(\alpha) + \alpha_2 v'(\alpha) = 0.
\]
There are two cases:

\smallskip

\emph{Case 1:} \(\alpha_2\neq 0\). Then
\[
u'(\alpha) = -\frac{\alpha_1}{\alpha_2} u(\alpha),
\qquad
v'(\alpha) = -\frac{\alpha_1}{\alpha_2} v(\alpha).
\]
Hence
\[
u(\alpha)v'(\alpha) - u'(\alpha)v(\alpha)
= u(\alpha)\left(-\frac{\alpha_1}{\alpha_2}v(\alpha)\right)
   -\left(-\frac{\alpha_1}{\alpha_2}u(\alpha)\right)v(\alpha)
= 0.
\]

\emph{Case 2:} \(\alpha_2=0\). Then \(\alpha_1\neq 0\), and the boundary condition reduces to
\[
u(\alpha) = v(\alpha) = 0.
\]
Consequently,
\[
u(\alpha)v'(\alpha) - u'(\alpha)v(\alpha) = 0.
\]

In either case the contribution at \(x=\alpha\) vanishes. The same argument applies verbatim at \(x=\beta\) (interchanging \(\alpha_1,\alpha_2\) with \(\beta_1,\beta_2\)). Therefore,
\[
p(x)\,\bigl(u(x)v'(x) - u'(x)v(x)\bigr)\Big|_{x=\alpha}^{x=\beta} = 0
\]
for all admissible \(u\) and \(v\). It follows that
\[
\langle u, A[v]\rangle_{L^2_w} = \langle A[u], v\rangle_{L^2_w}
\]
on the domain of functions satisfying the given separated boundary conditions. Thus the Sturm–Liouville operator \(A=(1/w)L_{\text{SL}}\) is self-adjoint (more precisely: symmetric, and under mild additional conditions, self-adjoint) in the Hilbert space \(L^2_w([\alpha,\beta])\).

\medskip

\emph{(c) Reality of eigenvalues and orthogonality of eigenfunctions.}

Consider the eigenvalue problem
\[
A[y] = \lambda\,y,
\]
with \(A\) self-adjoint in \(L^2_w\). Let \(y\) be a (nonzero) eigenfunction with eigenvalue \(\lambda\).

First, take the inner product with \(y\):
\[
\langle A[y], y\rangle_{L^2_w} = \lambda\,\langle y, y\rangle_{L^2_w}.
\]
By self-adjointness,
\[
\langle A[y], y\rangle_{L^2_w} = \langle y, A[y]\rangle_{L^2_w}
= \langle y, \lambda y\rangle_{L^2_w}
= \overline{\lambda}\,\langle y, y\rangle_{L^2_w},
\]
where the bar denotes complex conjugation. Thus
\[
\lambda\,\langle y, y\rangle_{L^2_w}
= \overline{\lambda}\,\langle y, y\rangle_{L^2_w}.
\]
Since \(w(x)>0\) and \(y\not\equiv 0\), we have \(\langle y,y\rangle_{L^2_w} > 0\), so
\[
\lambda = \overline{\lambda},
\]
i.e.\ every eigenvalue \(\lambda\) is real.

Next, let \(y_1\) and \(y_2\) be eigenfunctions corresponding to eigenvalues \(\lambda_1\) and \(\lambda_2\), with \(\lambda_1\neq\lambda_2\). Then
\[
A[y_1] = \lambda_1 y_1,
\qquad
A[y_2] = \lambda_2 y_2.
\]
Using self-adjointness,
\[
\langle A[y_1], y_2\rangle_{L^2_w}
= \langle y_1, A[y_2]\rangle_{L^2_w}.
\]
Substituting the eigenvalue equations,
\[
\lambda_1 \langle y_1, y_2\rangle_{L^2_w}
= \lambda_2 \langle y_1, y_2\rangle_{L^2_w}.
\]
Hence
\[
(\lambda_1 - \lambda_2)\,\langle y_1, y_2\rangle_{L^2_w} = 0.
\]
If \(\lambda_1\neq\lambda_2\), we must have
\[
\langle y_1, y_2\rangle_{L^2_w} = 0,
\]
i.e.\ eigenfunctions corresponding to distinct eigenvalues are orthogonal in \(L^2_w\).

Thus, by recasting the original second-order equation into Sturm–Liouville form via the integrating factor \(\mu(x)\), we obtain a self-adjoint eigenvalue problem with respect to the weighted inner product \(\langle\cdot,\cdot\rangle_{L^2_w}\), guaranteeing real eigenvalues and orthogonal eigenfunctions. 
\end{solution}

% ===== Example 6: Completeness and expanding functions in eigenfunction bases (inquiry-based) =====
\begin{problem}[Completeness and expanding functions in eigenfunction bases]
In many physical models, such as the vibrating string or the one-dimensional heat equation, one first finds a family of normal modes by solving a Sturm--Liouville eigenvalue problem. A central question is whether arbitrary initial data can be expressed as a convergent series of these eigenfunctions. This property is called \emph{completeness} of the eigenfunctions and is the key to turning difficult partial differential equations into simpler ordinary differential equations for the mode amplitudes. In this problem, you will explore completeness concretely for the simplest Sturm--Liouville operator, and use it to solve an initial value problem.

Consider the Sturm--Liouville eigenvalue problem on the interval $(0,\pi)$
\[
-y''(x) = \lambda\,y(x),\qquad 0<x<\pi,\qquad y(0)=0,\quad y(\pi)=0.
\]

\smallskip
(a) Solve the eigenvalue problem.

\quad(i) By considering the cases $\lambda<0$, $\lambda=0$, and $\lambda>0$ separately, determine all eigenvalues $\lambda$ for which there exists a nontrivial solution satisfying the given boundary conditions.

\quad(ii) For each eigenvalue $\lambda_n$ that you find, write down a corresponding eigenfunction $\phi_n(x)$ and simplify it as much as possible.

\quad(iii) Show that the eigenfunctions you found are orthogonal with respect to the standard $L^2$ inner product on $(0,\pi)$, that is,
\[
\int_0^\pi \phi_m(x)\,\phi_n(x)\,dx = 0 \quad \text{for } m\neq n.
\]
Hint: Reduce the orthogonality integral to a simple trigonometric identity.

\smallskip
(b) Let $f(x)=x(\pi-x)$ on $[0,\pi]$.

\quad(i) Check that $f$ satisfies the same boundary conditions as the eigenfunctions. Why is this relevant for trying to expand $f$ as a series in $\{\phi_n\}$?

\quad(ii) Assume that $\{\phi_n\}$ is complete in an appropriate sense (for instance, in $L^2(0,\pi)$). Write down the formal eigenfunction expansion of $f$,
\[
f(x) \sim \sum_{n=1}^{\infty} a_n \,\phi_n(x),
\]
and use orthogonality to derive a formula for the coefficients $a_n$ in terms of an integral involving $f$ and $\phi_n$.

\quad(iii) Specialize your formula from part (ii) to the specific eigenfunctions you found in part (a), and write $a_n$ explicitly as a definite integral over $(0,\pi)$ (do not evaluate it yet).

\smallskip
(c) Now compute the coefficients $a_n$ for $f(x)=x(\pi-x)$.

\quad(i) Evaluate the integral expression for $a_n$ using integration by parts as needed, and simplify your answer as far as possible.

\quad(ii) Show that $a_n=0$ for all even $n$, and find a closed form for $a_{2k+1}$, $k\in\mathbb{N}_0$. What does this tell you about which normal modes are present in the initial shape $f$?

Hint: Look carefully at factors of $(-1)^n$ that appear in your computations.

\smallskip
(d) Use this eigenfunction expansion to solve an initial-boundary value problem for the heat equation.

Consider the heat equation
\[
u_t = u_{xx},\qquad 0<x<\pi,\ t>0,
\]
with boundary conditions
\[
u(0,t)=0,\quad u(\pi,t)=0\quad\text{for all } t>0,
\]
and initial condition
\[
u(x,0)=f(x)=x(\pi-x).
\]

\quad(i) Make a separated-variables ansatz of the form
\[
u(x,t) = \sum_{n=1}^{\infty} b_n(t)\,\phi_n(x)
\]
using the eigenfunctions from part (a). Substitute this into the heat equation and use orthogonality to derive an ordinary differential equation for each $b_n(t)$.

\quad(ii) Solve the resulting ODEs for $b_n(t)$, and use the initial condition at $t=0$ to relate $b_n(0)$ to the coefficients $a_n$ you computed in part (c).

\quad(iii) Write down the final series representation for $u(x,t)$, and explain briefly (in words) where completeness of $\{\phi_n\}$ is used in this construction.

Hint: Separating variables should give decay factors of the form $e^{-\lambda_n t}$ multiplying the spatial eigenfunctions.

\smallskip
(e) Explorations and extensions.

\quad(i) Suppose instead that the boundary conditions were insulating (Neumann) conditions: $y'(0)=0$ and $y'(\pi)=0$. Sketch how the analysis above would change: What eigenfunctions would you expect? What sort of Fourier series would appear?

\quad(ii) In this problem we assumed without proof that $\{\phi_n\}$ is complete. Briefly discuss (qualitatively) what could go wrong in solving the heat equation by eigenfunction expansion if the eigenfunctions of the underlying Sturm--Liouville problem were not complete. How would this manifest itself in terms of the initial condition?

Hint: Think about whether you could match the initial data by a series in the eigenfunctions, and what that would imply for the solution at later times.
\end{problem}

% ===== Example 6: Completeness and expanding functions in eigenfunction bases (full solution) =====
\begin{problem}[Completeness and expanding functions in eigenfunction bases]
Consider the Sturm--Liouville problem
\[
-y''(x) = \lambda\,y(x),\qquad 0<x<\pi,\qquad y(0)=0,\quad y(\pi)=0.
\]
\begin{enumerate}
\item Find all eigenvalues and (unnormalized) eigenfunctions, and show that the eigenfunctions are orthogonal in $L^2(0,\pi)$.
\item For $f(x)=x(\pi-x)$ on $[0,\pi]$, find the Fourier sine series
\[
f(x) \sim \sum_{n=1}^{\infty} a_n \sin(nx),
\]
by computing the coefficients $a_n$ explicitly and simplifying them.
\item Using the completeness of $\{\sin(nx)\}_{n\ge1}$ in $L^2(0,\pi)$, solve the heat equation
\[
u_t = u_{xx},\quad 0<x<\pi,\ t>0,\qquad u(0,t)=u(\pi,t)=0,
\]
with initial condition $u(x,0)=x(\pi-x)$, and write $u(x,t)$ as a convergent eigenfunction series.
\item Briefly explain how this example illustrates the use of Sturm--Liouville (spectral) theory to represent functions and solve initial-boundary value problems.
\end{enumerate}
\end{problem}

\begin{solution}
We proceed step by step, highlighting the spectral ideas as they arise.

\medskip
\noindent\textbf{1. Eigenvalues, eigenfunctions, and orthogonality.}

We solve
\[
-y'' = \lambda y,\qquad y(0)=0,\quad y(\pi)=0.
\]
We consider the three standard cases.

\smallskip
\emph{Case $\lambda<0$.} Write $\lambda=-\mu^2$ with $\mu>0$. Then the ODE becomes
\[
-y'' = -\mu^2 y \quad\Longleftrightarrow\quad y'' = \mu^2 y,
\]
whose general solution is
\[
y(x)=C_1 e^{\mu x} + C_2 e^{-\mu x}.
\]
Imposing $y(0)=0$ gives $C_1 + C_2 = 0$, so $C_2=-C_1$ and $y(x)=C_1(e^{\mu x}-e^{-\mu x})=2C_1\sinh(\mu x)$. Then
\[
y(\pi)=0\quad\Rightarrow\quad \sinh(\mu\pi)=0,
\]
which is impossible for $\mu>0$. Thus there are no nontrivial eigenfunctions for $\lambda<0$.

\smallskip
\emph{Case $\lambda=0$.} The ODE becomes $y''=0$ with general solution $y(x)=Ax+B$. The boundary condition $y(0)=0$ forces $B=0$, and $y(\pi)=0$ then gives $A\pi=0$, so $A=0$. Hence $y\equiv0$ is the only solution, so $\lambda=0$ is not an eigenvalue.

\smallskip
\emph{Case $\lambda>0$.} Write $\lambda=\alpha^2$ with $\alpha>0$. The ODE becomes
\[
-y'' = \alpha^2 y \quad\Longleftrightarrow\quad y'' + \alpha^2 y = 0,
\]
whose general solution is
\[
y(x)=A\cos(\alpha x) + B\sin(\alpha x).
\]
The boundary condition $y(0)=0$ forces $A\cos 0 + B\sin 0 = A = 0$, so $y(x)=B\sin(\alpha x)$. Then
\[
y(\pi)=0 \quad\Rightarrow\quad B\sin(\alpha\pi) = 0.
\]
For a nontrivial solution we require $B\neq 0$, so $\sin(\alpha\pi)=0$, that is $\alpha\pi = n\pi$ for some integer $n$. Since $\alpha>0$ and $\alpha$ appears in $\sin(\alpha x)$, we take $n\in\mathbb{N}$. Thus $\alpha=n$ and
\[
\lambda_n = n^2,\qquad \phi_n(x) = \sin(nx),\qquad n=1,2,3,\dots
\]
are all eigenpairs (up to a multiplicative constant in $\phi_n$).

\smallskip
\emph{Orthogonality.} The standard $L^2(0,\pi)$ inner product is
\[
\langle f,g\rangle = \int_0^\pi f(x)\,g(x)\,dx.
\]
For $m\neq n$,
\[
\int_0^\pi \sin(mx)\sin(nx)\,dx
= \frac{1}{2}\int_0^\pi\bigl(\cos((m-n)x)-\cos((m+n)x)\bigr)\,dx.
\]
Integrating, we obtain
\[
\int_0^\pi \sin(mx)\sin(nx)\,dx
= \frac{1}{2}\left[\frac{\sin((m-n)x)}{m-n} - \frac{\sin((m+n)x)}{m+n}\right]_{0}^{\pi}.
\]
Since $m\pm n$ are nonzero integers, $\sin((m\pm n)\pi)=0$ and also $\sin(0)=0$, so the bracketed expression vanishes. Hence
\[
\int_0^\pi \sin(mx)\sin(nx)\,dx = 0,\quad m\neq n.
\]
Thus the eigenfunctions $\{\sin(nx)\}$ are pairwise orthogonal in $L^2(0,\pi)$.

\medskip
\noindent\textbf{2. Expanding $f(x)=x(\pi-x)$ in the eigenfunction basis.}

We are given $f(x)=x(\pi-x)$ on $[0,\pi]$. First note that
\[
f(0) = 0\cdot \pi = 0,\qquad f(\pi) = \pi(\pi-\pi) = 0,
\]
so $f$ satisfies the same homogeneous Dirichlet boundary conditions as the eigenfunctions. This is compatible with trying to represent $f$ as a sine series, since the sine functions also vanish at $x=0$ and $x=\pi$.

The general completeness theorem (which we invoke without proof) states that the family $\{\sin(nx)\}_{n\ge1}$ is complete in $L^2(0,\pi)$. Therefore any square-integrable function on $(0,\pi)$, and in particular any piecewise $C^1$ function with $f(0)=f(\pi)=0$, can be expanded in an $L^2$-convergent sine series
\[
f(x) \sim \sum_{n=1}^{\infty} a_n \sin(nx).
\]
Because the eigenfunctions are orthogonal, we can compute $a_n$ by projecting $f$ onto each $\sin(nx)$. We write
\[
a_n = \frac{\langle f,\sin(nx)\rangle}{\langle \sin(nx),\sin(nx)\rangle}
= \frac{\displaystyle\int_0^\pi f(x)\sin(nx)\,dx}{\displaystyle\int_0^\pi\sin^2(nx)\,dx}.
\]
We use the standard identity
\[
\int_0^\pi \sin^2(nx)\,dx = \frac{\pi}{2},\quad n\in\mathbb{N}.
\]
Therefore
\[
a_n = \frac{2}{\pi}\int_0^\pi f(x)\,\sin(nx)\,dx
= \frac{2}{\pi}\int_0^\pi x(\pi-x)\sin(nx)\,dx.
\]

\medskip
\noindent\textbf{3. Computing the coefficients $a_n$.}

Let
\[
I_n := \int_0^\pi x(\pi-x)\sin(nx)\,dx.
\]
Then $a_n = \dfrac{2}{\pi} I_n$. We compute $I_n$ explicitly.

First expand $x(\pi-x) = \pi x - x^2$:
\[
I_n = \pi\int_0^\pi x\sin(nx)\,dx - \int_0^\pi x^2\sin(nx)\,dx
=: \pi A_n - B_n.
\]

\smallskip
\emph{Step 1: Compute $A_n=\displaystyle\int_0^\pi x\sin(nx)\,dx$.}

An antiderivative of $x\sin(nx)$ is
\[
\int x\sin(nx)\,dx = -\frac{x\cos(nx)}{n} + \frac{\sin(nx)}{n^2}.
\]
Thus
\[
A_n = \left[-\frac{x\cos(nx)}{n} + \frac{\sin(nx)}{n^2}\right]_0^\pi
= \left(-\frac{\pi\cos(n\pi)}{n} + 0\right) - (0+0)
= -\frac{\pi(-1)^n}{n}.
\]
Therefore
\[
\pi A_n = -\frac{\pi^2(-1)^n}{n}.
\]

\smallskip
\emph{Step 2: Compute $B_n=\displaystyle\int_0^\pi x^2\sin(nx)\,dx$.}

We integrate by parts once, with $u=x^2$ and $dv=\sin(nx)\,dx$. Then $du=2x\,dx$ and $v=-\cos(nx)/n$, so
\[
B_n = -\frac{x^2\cos(nx)}{n}\Big|_{0}^{\pi} + \frac{2}{n}\int_0^\pi x\cos(nx)\,dx.
\]
At the endpoints we get
\[
-\frac{x^2\cos(nx)}{n}\Big|_{0}^{\pi}
= -\frac{\pi^2\cos(n\pi)}{n} - 0
= -\frac{\pi^2(-1)^n}{n}.
\]
Let
\[
C_n := \int_0^\pi x\cos(nx)\,dx.
\]
An antiderivative is
\[
\int x\cos(nx)\,dx = \frac{x\sin(nx)}{n} + \frac{\cos(nx)}{n^2},
\]
so
\[
C_n = \left[\frac{x\sin(nx)}{n} + \frac{\cos(nx)}{n^2}\right]_0^\pi
= \left(\frac{\pi\sin(n\pi)}{n} + \frac{\cos(n\pi)}{n^2}\right) - \left(0 + \frac{1}{n^2}\right).
\]
Since $\sin(n\pi)=0$ and $\cos(n\pi)=(-1)^n$, this simplifies to
\[
C_n = \frac{(-1)^n}{n^2} - \frac{1}{n^2}
= \frac{(-1)^n - 1}{n^2}.
\]
Thus
\[
B_n = -\frac{\pi^2(-1)^n}{n} + \frac{2}{n}C_n
= -\frac{\pi^2(-1)^n}{n} + \frac{2}{n}\cdot \frac{(-1)^n - 1}{n^2}
= -\frac{\pi^2(-1)^n}{n} + \frac{2\bigl((-1)^n - 1\bigr)}{n^3}.
\]

\smallskip
\emph{Step 3: Combine to get $I_n$ and $a_n$.}

Recall that $I_n = \pi A_n - B_n$, so
\[
I_n = \left(-\frac{\pi^2(-1)^n}{n}\right)
- \left(-\frac{\pi^2(-1)^n}{n} + \frac{2\bigl((-1)^n - 1\bigr)}{n^3}\right)
= -\frac{2\bigl((-1)^n - 1\bigr)}{n^3}.
\]
We can rewrite this as
\[
I_n = \frac{2(1-(-1)^n)}{n^3}.
\]
Therefore
\[
a_n = \frac{2}{\pi}I_n
= \frac{4}{\pi}\,\frac{1-(-1)^n}{n^3}.
\]

Observe that for even $n$, say $n=2k$, we have $(-1)^n = 1$, so $1-(-1)^n=0$ and thus $a_{2k}=0$. For odd $n$, say $n=2k+1$, we have $(-1)^n=-1$, hence $1-(-1)^n = 2$, and
\[
a_{2k+1} = \frac{4}{\pi}\cdot \frac{2}{(2k+1)^3}
= \frac{8}{\pi(2k+1)^3}.
\]
We have therefore obtained the expansion
\[
f(x) = x(\pi-x) \sim \sum_{k=0}^{\infty} \frac{8}{\pi(2k+1)^3}\,\sin\bigl((2k+1)x\bigr),
\]
with convergence in the $L^2(0,\pi)$ sense, and in fact pointwise and uniformly on $[0,\pi]$ since $f$ is smooth.

Physically, the vanishing of $a_{2k}$ tells us that only odd normal modes are excited by the initial shape $f(x)=x(\pi-x)$.

\medskip
\noindent\textbf{4. Solving the heat equation via eigenfunction expansion.}

We now solve
\[
u_t = u_{xx},\quad 0<x<\pi,\ t>0,\qquad
u(0,t)=u(\pi,t)=0,\qquad
u(x,0)=f(x)=x(\pi-x).
\]
The spectral idea is to expand $u(x,t)$ in the complete eigenfunction basis $\{\sin(nx)\}$ of the spatial operator $-d^2/dx^2$ with Dirichlet boundary conditions. We make the ansatz
\[
u(x,t) = \sum_{n=1}^{\infty} b_n(t)\,\sin(nx),
\]
where the time-dependent coefficients $b_n(t)$ are to be determined. This form automatically satisfies the boundary conditions $u(0,t)=u(\pi,t)=0$ for all $t$, because each $\sin(nx)$ vanishes at $x=0$ and $x=\pi$.

We differentiate term-by-term:
\[
u_t(x,t) = \sum_{n=1}^{\infty} b_n'(t)\,\sin(nx),\qquad
u_{xx}(x,t) = \sum_{n=1}^{\infty} b_n(t)\,\frac{d^2}{dx^2}\sin(nx).
\]
Since $\dfrac{d^2}{dx^2}\sin(nx) = -n^2\sin(nx)$, we have
\[
u_{xx}(x,t) = \sum_{n=1}^{\infty} -n^2 b_n(t)\,\sin(nx).
\]
Substituting $u_t=u_{xx}$ gives
\[
\sum_{n=1}^{\infty} b_n'(t)\,\sin(nx)
= \sum_{n=1}^{\infty} -n^2 b_n(t)\,\sin(nx).
\]
By orthogonality and completeness of the sine functions, the only way two such series are equal in $L^2(0,\pi)$ is for the coefficients of each $\sin(nx)$ to agree. Thus, for each $n\ge1$,
\[
b_n'(t) = -n^2 b_n(t).
\]
This is a first-order linear ODE with solution
\[
b_n(t) = C_n e^{-n^2 t},
\]
where $C_n$ is a constant determined by the initial condition.

At $t=0$, the initial condition $u(x,0)=f(x)$ gives
\[
f(x) = u(x,0) = \sum_{n=1}^{\infty} b_n(0)\,\sin(nx)
= \sum_{n=1}^{\infty} C_n\,\sin(nx).
\]
By the uniqueness of the sine series expansion (again using orthogonality and completeness), the constants $C_n$ must coincide with the Fourier sine coefficients $a_n$ of $f$ computed earlier. Therefore
\[
b_n(t) = a_n e^{-n^2 t},
\]
and hence
\[
u(x,t) = \sum_{n=1}^{\infty} a_n e^{-n^2 t}\,\sin(nx).
\]
Substituting the explicit coefficients, we obtain
\[
u(x,t) = \sum_{n=1}^{\infty} \frac{4}{\pi}\,\frac{1-(-1)^n}{n^3}\,e^{-n^2 t}\,\sin(nx)
= \sum_{k=0}^{\infty} \frac{8}{\pi(2k+1)^3}\,e^{-(2k+1)^2 t}\,\sin\bigl((2k+1)x\bigr).
\]

This series converges in $L^2(0,\pi)$ for each $t>0$, and in fact uniformly on $[0,\pi]$ for each fixed $t>0$ due to the exponential decay factor $e^{-n^2 t}$.

\medskip
\noindent\textbf{5. Conceptual summary and connection to Sturm--Liouville theory.}

The key ideas from Sturm--Liouville (spectral) theory in this example are:
\begin{itemize}
\item The spatial operator $L[y] = -y''$ with Dirichlet boundary conditions is a self-adjoint Sturm--Liouville operator on $(0,\pi)$.
\item Its eigenvalue problem $L\phi_n = \lambda_n\phi_n$ yields a real, discrete spectrum $\lambda_n=n^2$ and orthogonal eigenfunctions $\phi_n(x)=\sin(nx)$.
\item The eigenfunctions form a complete orthogonal basis of $L^2(0,\pi)$, so any reasonable function (such as $f(x)=x(\pi-x)$) can be expanded in an $L^2$-convergent series in these eigenfunctions.
\item For the heat equation, representing the solution as a sum of separated solutions $e^{-\lambda_n t}\phi_n(x)$ reduces the PDE to a family of decoupled ODEs for the mode amplitudes, which are easy to solve.
\end{itemize}
Thus, this example illustrates how Sturm--Liouville spectral theory provides a systematic way to (i) represent functions in terms of eigenfunctions of a differential operator, and (ii) solve initial-boundary value problems by decomposing arbitrary initial data into normal modes and evolving each mode independently in time.
\end{solution}

\section{Phase Space Dynamics for Conservative and Perturbed Systems}
% --- Narrative plan (auto-generated) ---
% This section develops the phase space viewpoint for ordinary differential equations, with special attention to conservative systems and what happens when we introduce small perturbations such as damping, forcing, or noise. Instead of focusing only on explicit formulas for solutions, we learn to visualize trajectories in the plane or higher-dimensional phase space, identify invariant sets and conserved quantities, and understand qualitative features such as equilibria, periodic orbits, and separatrices. Conservative systems, which preserve an energy-like quantity, provide a clean starting point where orbits lie on level sets, while perturbed systems show how these structures deform, break, or give rise to more complex behavior.
%
% This perspective is central in applied mathematics, dynamical systems, and the study of partial differential equations viewed as infinite-dimensional dynamical systems. Many PDEs have conserved energies and can be approximated by finite-dimensional conservative systems that capture essential behavior, while perturbations model dissipation, driving, or external influences. The ideas here connect to Hamiltonian mechanics from classical physics, to complex analysis through analytic continuation of vector fields and normal forms, and to Fourier analysis via modal decompositions that turn PDEs into coupled ODEs in phase space. By learning to read phase portraits and track how they change under perturbations, you gain a durable toolkit for understanding models far beyond those that can be solved in closed form.

% ===== Example 1: Simple Harmonic Oscillator in Phase Space (inquiry-based) =====
\begin{problem}[Simple Harmonic Oscillator in Phase Space]
The simple harmonic oscillator is one of the most fundamental models in mechanics.  It describes, for instance, the motion of a mass attached to an ideal spring with no friction.  In this problem, you will reformulate the usual second-order equation as a first-order system, discover a conserved ``energy,'' and see how this leads to circular orbits in the phase plane.  This example will serve as a model for more general conservative and perturbed systems.

Consider the dimensionless simple harmonic oscillator
\[
x''(t) + x(t) = 0,
\]
where $x(t)$ is the displacement of the mass from equilibrium.

\smallskip

\noindent
(a) Introduce a new variable $v(t)$ representing the velocity of the mass and rewrite the second-order equation as a first-order system in the variables $(x,v)$.

\emph{Hint:} Let $v(t) = x'(t)$ and express both $x'(t)$ and $v'(t)$ in terms of $x(t)$ and $v(t)$.

\smallskip

\noindent
(b) Solutions of this system can be drawn in the \emph{phase plane}, whose axes are $x$ (horizontal) and $v$ (vertical).  To understand the geometry of these trajectories, we look for a quantity that stays constant along solutions.

Define the \emph{energy function}
\[
E(x,v) = \frac{1}{2}v^2 + \frac{1}{2}x^2.
\]
Compute the derivative $\dfrac{d}{dt}E\bigl(x(t),v(t)\bigr)$ along a solution $(x(t),v(t))$ of your first-order system from part (a).  Simplify your expression as far as possible.

\emph{Hint:} Use the chain rule
\[
\frac{d}{dt}E(x(t),v(t)) = E_x(x,v)\,x'(t) + E_v(x,v)\,v'(t),
\]
and then substitute $x'(t)$ and $v'(t)$ from your system.

\smallskip

\noindent
(c) Show that the derivative you found in part (b) is actually \emph{identically zero} along any solution of the system.  Conclude that the energy $E(x(t),v(t))$ is constant in time for every solution.

\emph{Hint:} You should obtain an exact cancellation between the terms that involve $x$ and those that involve $v$.

\smallskip

\noindent
(d) Fix a constant $E_0 > 0$.  Describe the set of all points $(x,v)$ in the phase plane that satisfy $E(x,v) = E_0$.  What kind of curve is this?  Sketch or visualize these sets for several values of $E_0$.

Then explain why every solution $(x(t),v(t))$ of the system must remain on exactly one of these curves for all time.  What does this say about the shape of the trajectories in the phase plane, and about the qualitative behavior of $x(t)$ (for example, periodicity)?

\emph{Hint:} The equation $E(x,v) = E_0$ can be written in a familiar form involving $x^2$ and $v^2$.

\smallskip

\noindent
(e) (Explorations.)

\begin{enumerate}
  \item Suppose we add a small damping term to the original equation and consider
  \[
  x''(t) + 2\gamma x'(t) + x(t) = 0, \qquad \gamma > 0.
  \]
  Rewrite this damped equation as a first-order system in $(x,v)$, and compute $\dfrac{d}{dt}E(x(t),v(t))$ along its solutions, where $E$ is the same energy function as before.  Is the energy still conserved?  How does this change the qualitative picture of the trajectories in the phase plane?

  \emph{Hint:} Compare the sign of $\dfrac{d}{dt}E$ now to the undamped case.

  \item Instead of using $v = x'$ as the vertical variable, one might use the \emph{momentum} $p = v$ (which in our dimensionless setting is the same as $v$) or rescale the axes differently.  How would rescaling $x$ and $v$ (for instance, considering $u = ax$ and $w = bv$ for constants $a,b>0$) change the appearance of the phase portrait, and what features of the dynamics would remain the same?

  \emph{Hint:} Think about how circles transform under scaling of the axes.
\end{enumerate}

\end{problem}

% ===== Example 1: Simple Harmonic Oscillator in Phase Space (full solution) =====
\begin{problem}[Simple Harmonic Oscillator in Phase Space]
Consider the simple harmonic oscillator
\[
x''(t) + x(t) = 0.
\]
\begin{enumerate}
  \item Rewrite this second-order equation as a first-order system in the variables $x$ (position) and $v$ (velocity).
  \item Define the energy function
  \[
  E(x,v) = \frac{1}{2}v^2 + \frac{1}{2}x^2.
  \]
  Show that $E(x(t),v(t))$ is constant along any solution $(x(t),v(t))$ of the first-order system.
  \item Describe the level sets $\{(x,v) : E(x,v) = E_0\}$ in the phase plane and explain what this implies about the trajectories of the system and the qualitative behavior of $x(t)$.
\end{enumerate}
\end{problem}

\begin{solution}
We study the ordinary differential equation
\[
x''(t) + x(t) = 0
\]
from the point of view of phase space dynamics.  The main ideas are to rewrite the second-order scalar equation as a first-order system, to identify a conserved quantity (the mechanical energy), and to interpret its level sets as trajectories in the phase plane.

\medskip

\noindent\textbf{(1) First-order system formulation.}
We introduce the velocity variable
\[
v(t) = x'(t).
\]
Then we can express the second-order equation as a system of two first-order equations.  By definition,
\[
x'(t) = v(t).
\]
Differentiating $v(t)$ and using the original equation, we have
\[
v'(t) = x''(t) = -x(t).
\]
Thus the phase space system is
\[
\begin{cases}
x'(t) = v(t),\\[4pt]
v'(t) = -x(t).
\end{cases}
\]
This is a linear system $\mathbf{y}' = A\mathbf{y}$ with state vector $\mathbf{y} = (x,v)^{\mathsf{T}}$ and coefficient matrix
\[
A = \begin{pmatrix}
0 & 1\\
-1 & 0
\end{pmatrix},
\]
whose dynamics we will interpret geometrically in the $(x,v)$-plane, called the phase plane.

\medskip

\noindent\textbf{(2) Conservation of energy.}
We define the energy function
\[
E(x,v) = \frac{1}{2}v^2 + \frac{1}{2}x^2.
\]
Physically, this represents the sum of kinetic energy $\frac{1}{2}v^2$ and potential energy $\frac{1}{2}x^2$ (in nondimensional units).  We now show that this energy is constant along solutions of the system.

Let $(x(t),v(t))$ be any solution of the system.  We compute the time derivative of $E$ along this solution using the chain rule:
\[
\frac{d}{dt}E(x(t),v(t))
= E_x(x(t),v(t))\,x'(t) + E_v(x(t),v(t))\,v'(t),
\]
where $E_x$ and $E_v$ denote the partial derivatives of $E$ with respect to $x$ and $v$, respectively.

We first compute these partial derivatives:
\[
E_x(x,v) = \frac{\partial}{\partial x}\left(\frac{1}{2}v^2 + \frac{1}{2}x^2\right) = x,
\quad
E_v(x,v) = \frac{\partial}{\partial v}\left(\frac{1}{2}v^2 + \frac{1}{2}x^2\right) = v.
\]
Therefore,
\[
\frac{d}{dt}E(x(t),v(t))
= x(t)\,x'(t) + v(t)\,v'(t).
\]
Now we substitute the expressions for $x'(t)$ and $v'(t)$ from the first-order system,
\[
x'(t) = v(t), \qquad v'(t) = -x(t),
\]
to obtain
\[
\frac{d}{dt}E(x(t),v(t))
= x(t)\,v(t) + v(t)\,(-x(t))
= x(t)\,v(t) - v(t)\,x(t) = 0.
\]
Thus $\dfrac{d}{dt}E(x(t),v(t)) = 0$ for all $t$ where the solution is defined.  This shows that
\[
E(x(t),v(t)) = \text{constant in } t.
\]
In other words, the energy is conserved along every trajectory.  This is the hallmark of a \emph{conservative} system.

\medskip

\noindent\textbf{(3) Level sets and phase plane trajectories.}
We next interpret the conservation of energy geometrically.  Fix any constant $E_0 > 0$.  The level set of the energy function corresponding to $E_0$ is
\[
\{(x,v) \in \mathbb{R}^2 : E(x,v) = E_0\}
= \left\{(x,v) : \frac{1}{2}v^2 + \frac{1}{2}x^2 = E_0\right\}.
\]
Multiplying both sides of the defining equation by $2$, we obtain
\[
x^2 + v^2 = 2E_0.
\]
This is the equation of a circle in the $(x,v)$-plane, centered at the origin, with radius $\sqrt{2E_0}$.  Thus each level set of $E$ with $E_0>0$ is a circle around the origin.  The special level set $E_0 = 0$ consists only of the single point $(0,0)$.

Because $E(x(t),v(t))$ is constant along any solution $(x(t),v(t))$, each trajectory must lie entirely within a single level set of $E$, determined by the initial condition.  More precisely, if the initial condition is $(x(0),v(0)) = (x_0,v_0)$, then for all times $t$ we have
\[
E(x(t),v(t)) = E(x_0,v_0),
\]
so the trajectory $(x(t),v(t))$ remains on the circle
\[
x^2 + v^2 = x_0^2 + v_0^2.
\]
Consequently, every nontrivial trajectory in the phase plane is a closed orbit, specifically a circle centered at the origin.  The origin itself is an equilibrium solution (corresponding to the mass at rest at equilibrium) and is represented by the single point $(0,0)$.

From the point of view of the original scalar equation, the fact that phase trajectories are closed circles implies that the motion is periodic.  Indeed, solving $x'' + x = 0$ in the usual way yields
\[
x(t) = A\cos t + B\sin t,
\]
and the velocity is
\[
v(t) = x'(t) = -A\sin t + B\cos t.
\]
The pair $(x(t),v(t))$ traces out a circle of radius $\sqrt{A^2+B^2}$ in the phase plane as $t$ increases.  The motion repeats itself with period $2\pi$, which is reflected in the fact that the trajectory is a closed curve.

\medskip

\noindent\textbf{Context within phase space dynamics.}
This example illustrates several central ideas of the section on \emph{Phase Space Dynamics for Conservative and Perturbed Systems}.  First, we saw how a scalar second-order equation can be reformulated as a first-order system, which allows us to analyze trajectories in the phase plane.  Second, we identified a conserved quantity, the energy, whose level sets organize the dynamics: each initial condition leads to motion along a fixed energy contour.  For this conservative system, all nontrivial trajectories are closed orbits corresponding to periodic motion.

In later examples, when damping or external forcing is added, the system will no longer conserve energy, the level sets will no longer be invariant curves, and phase trajectories will tend to spiral toward or away from equilibria rather than forming closed loops.  The simple harmonic oscillator thus serves as a canonical model of conservative dynamics against which perturbed systems can be compared.

\end{solution}

% ===== Example 2: Nonlinear Conservative Oscillator and Energy Level Sets (inquiry-based) =====
\begin{problem}[Nonlinear Conservative Oscillator and Energy Level Sets]
A mass attached to an ideal linear spring executes simple harmonic motion, and its trajectories in phase space are perfect circles. Real springs, however, are not perfectly linear. In this problem we study a simple model of a nonlinear spring, where the restoring force contains both linear and cubic terms. The system remains conservative, so there is an energy function, but the corresponding level sets in the phase plane are no longer circles. Instead, they are distorted closed curves, and the period of oscillation depends on the amplitude.

Consider the nonlinear oscillator
\[
x'' + x + x^3 = 0,
\]
where $x(t)$ is the displacement of the mass from equilibrium.

\smallskip

(a) Show that this equation is conservative by finding a quantity that is constant along every solution.  
More precisely, show that there exists a function $E(x,x')$ such that $\dfrac{d}{dt}E(x(t),x'(t)) = 0$ whenever $x$ satisfies the differential equation.

\emph{Hint:} A common trick for finding such an ``energy'' is to multiply the equation $x'' + x + x^3 = 0$ by $x'(t)$ and then integrate with respect to time.

\smallskip

(b) Interpret the energy function $E(x,x')$ mechanically.  

(i) Identify which part of $E$ should be thought of as kinetic energy and which part as potential energy.  
(ii) From your expression for $E$, read off the potential energy function $V(x)$ and sketch the graph of $V(x)$ as a function of $x$.

\emph{Hint:} For a one–dimensional mechanical system, the kinetic energy has the form $\tfrac12 (x')^2$ (up to a constant factor that we may take to be $1$ here), and the potential $V(x)$ satisfies $-V'(x)$ equal to the restoring force.

\smallskip

(c) Rewrite the second–order equation as a planar first–order system and describe the phase portrait in terms of the energy.  

(i) Introduce the velocity variable $y = x'$ and write the system in $(x,y)$–coordinates.  
(ii) Using your expression for $E(x,y)$, show that
\[
\frac{d}{dt} E(x(t),y(t)) = 0
\]
for any solution $(x(t),y(t))$ of the system. Conclude that each trajectory of the planar system lies entirely on a level set
\[
E(x,y) = \text{constant}.
\]

\emph{Hint:} Compute $\dfrac{d}{dt}E(x(t),y(t))$ using the chain rule:
\[
\frac{dE}{dt} = E_x(x,y)\,x' + E_y(x,y)\,y',
\]
and then substitute $x' = y$ and $y'$ from your first–order system.

\smallskip

(d) Now examine the geometry of the level sets and what they tell you about the motion.

(i) For a fixed energy level $E>0$, solve the relation $E(x,y) = E$ for $y^2$ in terms of $x$. What is the maximum displacement $|x|$ that a trajectory with energy $E$ can reach?  

(ii) Compare these energy curves to the circles that arise from the linear oscillator $x''+x=0$. In what way are they similar, and in what way are they different? Make a qualitative sketch of several level sets $E(x,y)=E$ in the $(x,y)$–plane and explain why they correspond to periodic orbits.

(iii) Let $A>0$ denote the amplitude of an oscillation, that is, the maximum value of $|x(t)|$ along the trajectory. Use energy conservation to derive an integral formula for the period $T(A)$ of the oscillation as a function of $A$.

\emph{Hint:} On a given trajectory, the energy $E$ is fixed and can be expressed in terms of $A$. Solve the equation $E = \tfrac12 (x')^2 + V(x)$ for $x' = \dfrac{dx}{dt}$ as a function of $x$ and $A$, and separate variables to write $dt$ in terms of $dx$. One quarter of the period corresponds to $x$ moving from $0$ to $A$ (or from $A$ back to $0$).

\smallskip

(e) Explore how the phase portrait changes when the model is modified.

(i) Suppose instead that the restoring force has the form $-x + x^3$, leading to the equation
\[
x'' + x - x^3 = 0.
\]
Find the corresponding potential energy $V(x)$ and sketch it. How does the shape of $V(x)$ suggest a different qualitative phase portrait (for example, the possibility of multiple equilibrium positions or trajectories that escape to infinity)?

(ii) Suppose we add a small linear damping term and consider
\[
x'' + \varepsilon x' + x + x^3 = 0,
\qquad 0<\varepsilon\ll 1.
\]
Explain qualitatively (without detailed computation) what happens to the conserved energy function $E(x,x')$ and how the phase portrait is altered. In particular, what happens to the closed level curves you found in part (d) when damping is present?

\end{problem}

% ===== Example 2: Nonlinear Conservative Oscillator and Energy Level Sets (full solution) =====
\begin{problem}[Nonlinear Conservative Oscillator and Energy Level Sets]
Consider the nonlinear oscillator
\[
x'' + x + x^3 = 0.
\]
\begin{enumerate}
\item Show that this equation is conservative by finding a first integral (energy)
\[
H(x,x') = \frac12 (x')^2 + V(x)
\]
that is constant along solutions, and determine $V(x)$ explicitly.
\item Rewrite the equation as a planar system in variables $(x,y)$, $y=x'$, and show that $H(x,y)$ is constant along trajectories of this system. Conclude that trajectories in the phase plane lie on level sets $H(x,y)=\text{constant}$.
\item Sketch qualitatively the phase portrait in the $(x,y)$–plane, and compare it to the phase portrait of the linear oscillator $x''+x=0$. Explain why the nonlinear system still has periodic orbits and why their period depends on the amplitude.
\item Let $A>0$ denote the maximum displacement of a given periodic solution. Use the conserved energy to derive an integral formula for the period $T(A)$ of oscillation as a function of $A$.
\end{enumerate}
\end{problem}

\begin{solution}
We study the equation
\[
x'' + x + x^3 = 0,
\]
which models a frictionless mass attached to a spring with both linear and cubic restoring forces. The central concepts are conservation of energy, the representation of the dynamics in the phase plane, and the description of trajectories as level sets of an energy (Hamiltonian) function.

\medskip

\textbf{1. Energy integral and potential.}
We first seek a conserved quantity of the form
\[
H(x,x') = \frac12 (x')^2 + V(x),
\]
where $V(x)$ plays the role of a potential energy. A standard way to find such a quantity is to multiply the differential equation by $x'(t)$ and integrate.

Starting from
\[
x'' + x + x^3 = 0,
\]
multiply by $x'$:
\[
x''x' + x x' + x^3 x' = 0.
\]
Each term is an exact time derivative:
\[
x''x' = \frac{d}{dt}\!\left(\frac12 (x')^2\right),\qquad
x x' = \frac{d}{dt}\!\left(\frac12 x^2\right),\qquad
x^3 x' = \frac{d}{dt}\!\left(\frac14 x^4\right).
\]
Therefore
\[
\frac{d}{dt}\!\left(\frac12 (x')^2 + \frac12 x^2 + \frac14 x^4\right) = 0.
\]
This shows that
\[
H(x,x') = \frac12 (x')^2 + \frac12 x^2 + \frac14 x^4
\]
is constant along any solution. Thus the system is conservative with potential energy
\[
V(x) = \frac12 x^2 + \frac14 x^4.
\]

Mechanically, $\tfrac12 (x')^2$ is the kinetic energy (for a unit mass), and $V(x)$ is the potential energy whose derivative gives the restoring force:
\[
F(x) = -V'(x) = -\bigl(x + x^3\bigr),
\]
which matches the right–hand side of the original equation.

The graph of $V(x)$ is even, strictly increasing in $|x|$, and behaves like $\tfrac14 x^4$ for large $|x|$. It has a single minimum at $x=0$ with $V(0)=0$. Because $V(x)\to +\infty$ as $|x|\to\infty$, the potential well confines the motion for any finite energy.

\medskip

\textbf{2. First–order system and conserved Hamiltonian.}
To view the system in phase space, we introduce the velocity variable
\[
y = x'.
\]
Then the second–order equation becomes the planar first–order system
\[
\begin{cases}
x' = y,\\[4pt]
y' = -x - x^3.
\end{cases}
\]
In these variables, the energy function becomes
\[
H(x,y) = \frac12 y^2 + \frac12 x^2 + \frac14 x^4.
\]
We verify directly that $H$ is conserved along trajectories. Using the chain rule,
\[
\frac{d}{dt} H(x(t),y(t))
= H_x(x,y)\,x' + H_y(x,y)\,y'.
\]
We compute the partial derivatives
\[
H_x(x,y) = x + x^3,\qquad H_y(x,y) = y,
\]
and substitute $x' = y$, $y' = -x - x^3$ from the system. This yields
\[
\frac{dH}{dt}
= (x + x^3) \cdot y + y \cdot (-x - x^3)
= (x + x^3)y - (x + x^3)y
= 0.
\]
Therefore $H(x(t),y(t))$ is constant in time along any solution $(x(t),y(t))$, so each trajectory is contained in a level set
\[
H(x,y) = C,\qquad C\ge 0.
\]
This is the typical structure of a conservative (Hamiltonian) system in the plane: the flow preserves the value of a Hamiltonian $H$, and trajectories lie on its level curves.

\medskip

\textbf{3. Geometry of level sets and comparison to the linear oscillator.}
We now analyze the level sets of $H$ and what they imply for the motion.

Fix an energy level $C>0$. The equation
\[
H(x,y) = C
\]
reads
\[
\frac12 y^2 + \frac12 x^2 + \frac14 x^4 = C.
\]
Solving for $y^2$ gives
\[
y^2 = 2C - x^2 - \frac12 x^4.
\]
Thus the level set consists of those $(x,y)$ for which the right–hand side is nonnegative:
\[
2C - x^2 - \frac12 x^4 \ge 0.
\]
This inequality determines a bounded interval of accessible $x$–values. The turning points in $x$ occur where $y=0$, that is, where
\[
\frac12 x^2 + \frac14 x^4 = C,
\]
or equivalently,
\[
x^4 + 2x^2 - 4C = 0.
\]
For each $C>0$, this quartic equation has exactly two real roots, symmetric about zero, say at $x = \pm A$, where $A>0$ is the amplitude of the motion. Since $V(x)$ is even and strictly increasing for $x>0$, there is a unique $A>0$ with
\[
C = V(A) = \frac12 A^2 + \frac14 A^4.
\]
Along a trajectory with energy $C$, the motion in $x$ is confined to the interval $-A \le x \le A$, with $x=\pm A$ occurring when the velocity $y=x'$ is zero.

The phase curves $H(x,y)=C$ are therefore closed, smooth, simple curves encircling the origin. They are symmetric with respect to both axes because $H$ is even in both $x$ and $y$.

For comparison, the linear oscillator $x''+x=0$ has energy
\[
H_{\text{lin}}(x,y) = \frac12 y^2 + \frac12 x^2,
\]
and its level sets $H_{\text{lin}}(x,y)=C$ are circles
\[
x^2 + y^2 = 2C.
\]
In the nonlinear case, the extra quartic term $\tfrac14 x^4$ deforms these circles into rounded rectangles: for small $|x|$, the dynamics is close to that of the linear system, so the curves are nearly circular near the origin; for larger $|x|$, the quartic term dominates, so the curves flatten out in the $y$–direction because the same energy $C$ must now accommodate a larger potential contribution.

Because the vector field is smooth, and the phase curves are closed and surround the equilibrium at the bottom of the potential well, the motion on each such energy curve is periodic in time. Thus each closed level set $H(x,y)=C>0$ corresponds to a periodic orbit in the plane.

\medskip

\textbf{4. Period as a function of amplitude.}
We now derive an integral formula expressing the period of oscillation in terms of the amplitude. Let $A>0$ be the maximum displacement of a given periodic solution. Then the corresponding energy level is
\[
C = H(A,0) = \frac12 A^2 + \frac14 A^4.
\]
On this energy level, the velocity $y=x'$ satisfies
\[
\frac12 (x')^2 + \frac12 x^2 + \frac14 x^4 = \frac12 A^2 + \frac14 A^4.
\]
Solving for $(x')^2$ gives
\[
(x')^2 = A^2 + \frac12 A^4 - x^2 - \frac12 x^4.
\]
We may write
\[
\frac{dx}{dt} = \pm \sqrt{A^2 + \frac12 A^4 - x^2 - \frac12 x^4}.
\]
To obtain the period, we separate variables:
\[
dt = \frac{dx}{\sqrt{A^2 + \tfrac12 A^4 - x^2 - \tfrac12 x^4}}
\]
for motion in one direction (say, as $x$ increases from $0$ to $A$). Because the system is symmetric and the motion is periodic, one quarter of the period $T(A)$ corresponds to $x$ moving from $0$ to $A$. Thus
\[
\frac{T(A)}{4} = \int_0^{A} \frac{dx}{\sqrt{A^2 + \tfrac12 A^4 - x^2 - \tfrac12 x^4}},
\]
and therefore
\[
T(A) = 4 \int_0^{A} \frac{dx}{\sqrt{A^2 + \tfrac12 A^4 - x^2 - \tfrac12 x^4}}.
\]
This integral cannot be expressed in elementary functions; it is a special case of an elliptic integral. Nevertheless, the formula makes clear that $T(A)$ depends on $A$: unlike the linear oscillator, whose period is independent of amplitude, the nonlinear oscillator exhibits amplitude–dependent frequency.

For small amplitudes $A$, one can expand the integrand in a Taylor series and show that $T(A)$ is close to the linear period $2\pi$, with corrections of order $A^2$. This reflects the fact that for small $x$ the cubic term $x^3$ is small and the system behaves almost linearly, while for larger amplitudes the nonlinearity significantly affects the timing of the oscillation.

\medskip

\textbf{5. Relation to phase space dynamics for conservative and perturbed systems.}
This example illustrates several key ideas about phase space dynamics for conservative systems. The existence of a conserved energy $H$ lets us reduce the analysis of trajectories to the study of level sets $H(x,y)=C$ in the phase plane. For the linear oscillator, these level sets are circles; for the nonlinear oscillator, they are distorted but still closed curves, reflecting the confining potential. Each closed level curve corresponds to a periodic orbit, but the period now depends on the energy or amplitude because the restoring force is nonlinear.

If one were to add a small damping term, the system would no longer be conservative: $dH/dt$ would become negative, and trajectories in the phase plane would spiral slowly inward across level sets rather than staying on a single curve. Thus the comparison between the undamped (conservative) nonlinear oscillator treated here and its damped (perturbed) counterpart provides a concrete example of how conservation laws shape phase portraits and how small perturbations qualitatively change the long–term dynamics.

\end{solution}

% ===== Example 3: Introducing Damping: From Closed Orbits to Spirals (inquiry-based) =====
\begin{problem}[Introducing Damping: From Closed Orbits to Spirals]
In the undamped harmonic oscillator, every trajectory in phase space is a closed orbit: the system oscillates forever with constant energy. In reality, friction or resistance gradually removes energy, and the oscillations decay. One simple way to model this is to add a linear damping term to the equation. In this problem you will see how this small change turns closed orbits into spirals, and how different strengths of damping lead to different qualitative behaviors in the phase plane.

Consider the \emph{damped} harmonic oscillator
\[
x''(t) + 2\alpha\,x'(t) + x(t) = 0, \qquad \alpha \ge 0,
\]
where $x(t)$ denotes the displacement and $\alpha$ is a (dimensionless) damping parameter.

\smallskip

(a) Rewrite this second-order equation as a first-order system in the phase variables
\[
x(t), \qquad v(t) = x'(t).
\]
What is the $2\times 2$ coefficient matrix $A$ of the resulting linear system
\[
\begin{pmatrix} x' \\[4pt] v' \end{pmatrix} = A \begin{pmatrix} x \\[4pt] v \end{pmatrix}?
\]
Locate and classify the equilibrium point(s) of this system at the level of ``name only'' (for example, ``center,'' ``node,'' ``saddle,'' etc.) in the special case $\alpha = 0$. How does this relate to the familiar picture of the undamped harmonic oscillator in phase space?

% Hint: For $\alpha = 0$ you should recognize a linear Hamiltonian system with circular or elliptic trajectories.

\smallskip

(b) Now suppose $\alpha > 0$. Compute the eigenvalues of $A$ as functions of $\alpha$ by solving the characteristic equation
\[
\det(\lambda I - A) = 0.
\]
For which values of $\alpha$ are the eigenvalues:
\begin{itemize}
    \item a pair of complex conjugates with nonzero imaginary part?
    \item a repeated real eigenvalue?
    \item two distinct real eigenvalues?
\end{itemize}
Give the corresponding qualitative names of the equilibrium at the origin in each case.

Hint: Pay attention to the discriminant of the quadratic equation for the eigenvalues, and recall what types of phase portraits arise from complex versus real eigenvalues.

\smallskip

(c) Focus on the \emph{underdamped} case $0 < \alpha < 1$. Solve the scalar differential equation explicitly in this case, and express your answer in the form
\[
x(t) = e^{-\alpha t}\bigl(C_1 \cos(\omega t) + C_2 \sin(\omega t)\bigr),
\]
for some frequency $\omega$ depending on $\alpha$. Then find $v(t) = x'(t)$.

Using this explicit solution, describe the behavior of the trajectory $(x(t), v(t))$ in the phase plane as $t \to +\infty$. Why does this correspond to a \emph{spiral} rather than a closed orbit?

% Hint: Think about what the factor $e^{-\alpha t}$ does to the amplitude as $t$ increases, and remember what $(\cos(\omega t), \sin(\omega t))$ does in the plane.

\smallskip

(d) Now consider the \emph{critically damped} case $\alpha = 1$ and the \emph{overdamped} case $\alpha > 1$.

\begin{itemize}
    \item For $\alpha = 1$, solve the scalar equation and write $x(t)$ in the form
    \[
    x(t) = (C_1 + C_2 t)\, e^{-t}.
    \]
    Use this to describe the qualitative shape of trajectories in the phase plane. Do solutions oscillate? Do they approach the origin tangentially to a single direction, or in many directions?
    
    \item For $\alpha > 1$, express the eigenvalues $\lambda_1, \lambda_2$ of $A$ explicitly, and write the general solution of the form
    \[
    x(t) = C_1 e^{\lambda_1 t} + C_2 e^{\lambda_2 t}.
    \]
    Based on the signs of $\lambda_1$ and $\lambda_2$, what does this tell you about the approach to the origin in the phase plane? Is the origin still stable? Are there oscillations?
\end{itemize}

Summarize how the phase portrait changes as $\alpha$ increases from $0$ to values larger than $1$, using the qualitative terms \emph{center}, \emph{stable spiral} (or \emph{stable focus}), and \emph{stable node}.

% Hint: Connect what you know about real versus complex eigenvalues to whether or not the trajectories wind around the origin.

\smallskip

(e) The mechanical energy of the undamped oscillator is
\[
E(t) = \frac{1}{2}\bigl(x'(t)^2 + x(t)^2\bigr).
\]
\begin{itemize}
    \item For the damped system with $\alpha > 0$, compute $\dfrac{dE}{dt}$ along solutions. Show that $\dfrac{dE}{dt} \le 0$ for all $t$ and interpret this physically.
    \item (\emph{What if} extension.) What would happen to the eigenvalues of $A$ and to $\dfrac{dE}{dt}$ if we took $\alpha < 0$ (``negative damping'')? What sort of phase portrait would you expect, and what physical behavior would this correspond to?
\end{itemize}

% Hint: Differentiate $E(t)$ using the product and chain rules, then substitute the differential equation to simplify. For $\alpha < 0$, think about whether the real parts of the eigenvalues are positive or negative.
\end{problem}

% ===== Example 3: Introducing Damping: From Closed Orbits to Spirals (full solution) =====
\begin{problem}[Introducing Damping: From Closed Orbits to Spirals]
Consider the damped harmonic oscillator
\[
x''(t) + 2\alpha\,x'(t) + x(t) = 0, \qquad \alpha \in \mathbb{R}.
\]
\begin{enumerate}
    \item Rewrite this equation as a first-order linear system in the phase variables $x$ and $v = x'$, and find the equilibrium point(s).
    \item For $\alpha \ge 0$, compute the eigenvalues of the system matrix and classify the equilibrium at the origin for the cases $0<\alpha<1$, $\alpha=1$, and $\alpha>1$.
    \item In the underdamped case $0<\alpha<1$, solve the scalar equation explicitly and show that all nontrivial trajectories in the phase plane are spirals that wind into the origin as $t \to +\infty$.
    \item Define the energy
    \[
    E(t) = \frac{1}{2}\bigl(x'(t)^2 + x(t)^2\bigr).
    \]
    For $\alpha>0$, compute $\dfrac{dE}{dt}$ along solutions and prove that $E(t)$ is strictly decreasing unless the solution is identically zero.
    \item Briefly describe how the phase portrait of the damped system ($\alpha>0$) differs qualitatively from the undamped case $\alpha=0$, and explain how this illustrates the effect of a nonconservative perturbation on a conservative system.
\end{enumerate}
\end{problem}

\begin{solution}
We begin by rewriting the second-order equation as a first-order system. Introduce the velocity variable
\[
v(t) = x'(t).
\]
Then $x'(t) = v(t)$ and
\[
v'(t) = x''(t) = -2\alpha x'(t) - x(t) = -2\alpha v(t) - x(t).
\]
Thus the system can be written in vector form as
\[
\begin{pmatrix} x' \\[4pt] v' \end{pmatrix}
=
A \begin{pmatrix} x \\[4pt] v \end{pmatrix}
\quad\text{with}\quad
A = \begin{pmatrix} 0 & 1 \\[4pt] -1 & -2\alpha \end{pmatrix}.
\]
The equilibrium points satisfy $x' = 0$ and $v' = 0$, which are equivalent to $v = 0$ and $-x - 2\alpha v = 0$. These equations imply $x = 0$ and $v = 0$, so the origin $(x,v) = (0,0)$ is the unique equilibrium.

When $\alpha = 0$, the matrix
\[
A = \begin{pmatrix} 0 & 1 \\[4pt] -1 & 0 \end{pmatrix}
\]
has eigenvalues $\lambda = \pm i$. The origin is therefore a \emph{center}, and the phase portrait consists of closed curves around the origin. This matches the usual picture of the undamped harmonic oscillator: solutions oscillate forever with constant amplitude, and their trajectories in the $(x,v)$-plane are ellipses (which are circles in appropriately scaled coordinates).

\medskip

We now study the dependence on $\alpha$ of the eigenvalues of $A$ when $\alpha \ge 0$. The characteristic polynomial is
\[
\det(\lambda I - A)
= \det\begin{pmatrix} \lambda & -1 \\[4pt] 1 & \lambda + 2\alpha \end{pmatrix}
= \lambda(\lambda + 2\alpha) + 1
= \lambda^2 + 2\alpha \lambda + 1.
\]
Thus the eigenvalues satisfy
\[
\lambda^2 + 2\alpha \lambda + 1 = 0,
\]
so
\[
\lambda = -\alpha \pm \sqrt{\alpha^2 - 1}.
\]
The discriminant is $\Delta = (2\alpha)^2 - 4 = 4(\alpha^2 - 1)$, and its sign determines the type of eigenvalues.

\begin{itemize}
    \item If $0 < \alpha < 1$, then $\alpha^2 - 1 < 0$, and we get a complex conjugate pair
    \[
    \lambda_{1,2} = -\alpha \pm i\omega,
    \quad \text{where} \quad
    \omega = \sqrt{1 - \alpha^2} > 0.
    \]
    The real part is $-\alpha < 0$, so the origin is a \emph{stable spiral} (or \emph{stable focus}).

    \item If $\alpha = 1$, then $\alpha^2 - 1 = 0$, and both eigenvalues are equal to $-1$. There is a repeated real eigenvalue with negative sign. One can check that $A$ has only one linearly independent eigenvector, so the origin is a \emph{degenerate stable node} (the critically damped case).

    \item If $\alpha > 1$, then $\alpha^2 - 1 > 0$, and both eigenvalues
    \[
    \lambda_{1,2} = -\alpha \pm \sqrt{\alpha^2 - 1}
    \]
    are real and negative. Since $\sqrt{\alpha^2 - 1} < \alpha$, we have
    \[
    \lambda_1 = -\alpha + \sqrt{\alpha^2 - 1} < 0,
    \qquad
    \lambda_2 = -\alpha - \sqrt{\alpha^2 - 1} < 0.
    \]
    The origin is then a \emph{stable node}.
\end{itemize}

Thus for all $\alpha > 0$ the origin is asymptotically stable, but the qualitative nature of the approach (spiraling versus straight-in) depends on whether the eigenvalues are complex or real.

\medskip

In the underdamped case $0 < \alpha < 1$, the scalar equation can be solved explicitly by the standard method for constant coefficient linear ordinary differential equations. Since the characteristic equation
\[
r^2 + 2\alpha r + 1 = 0
\]
has roots $r = -\alpha \pm i\omega$ with $\omega = \sqrt{1-\alpha^2}$, the general solution is
\[
x(t) = e^{-\alpha t}\bigl(C_1 \cos(\omega t) + C_2 \sin(\omega t)\bigr),
\]
for arbitrary real constants $C_1$ and $C_2$. Differentiating, we obtain
\[
v(t) = x'(t) = e^{-\alpha t}\bigl(\tilde{C}_1 \cos(\omega t) + \tilde{C}_2 \sin(\omega t)\bigr),
\]
for suitable constants $\tilde{C}_1$ and $\tilde{C}_2$ that depend linearly on $C_1$ and $C_2$ and on $\alpha$ and $\omega$. The precise formulas are not important for the qualitative picture; what matters is that $v(t)$ has the same exponential factor $e^{-\alpha t}$ multiplying a sinusoidal term.

The factor $e^{-\alpha t}$ tends to $0$ as $t \to +\infty$. Meanwhile, the vector
\[
\bigl(\cos(\omega t), \sin(\omega t)\bigr)
\]
rotates around the origin with angular frequency $\omega$. Thus the pair $(x(t), v(t))$ traces out a curve that winds around the origin, with its radius decaying like $e^{-\alpha t}$ as $t$ increases. Consequently, each nontrivial trajectory is a spiral that winds inward and converges to the origin. Unlike the undamped case $\alpha=0$, the trajectory no longer closes on itself, since the amplitude is no longer constant in time.

\medskip

We now turn to the critically damped case $\alpha = 1$ and the overdamped case $\alpha > 1$.

When $\alpha=1$, the characteristic equation
\[
r^2 + 2r + 1 = 0
\]
has a double root $r=-1$. The general solution of the scalar ordinary differential equation is
\[
x(t) = (C_1 + C_2 t)\, e^{-t}.
\]
Here the factor $e^{-t}$ forces the solution to decay to zero, while the factor $(C_1 + C_2 t)$ allows for an initial transient that may grow linearly before the exponential decay dominates. There is no sinusoidal factor, so there are no oscillations: the solution crosses zero at most once. In the phase plane, all nontrivial trajectories approach the origin tangentially to the single eigen-direction associated with the eigenvalue $-1$. The origin is a degenerate stable node, and the approach is monotone in the sense that there is no winding around the origin.

When $\alpha>1$, the characteristic equation has two distinct real roots
\[
\lambda_{1,2} = -\alpha \pm \sqrt{\alpha^2 - 1},
\]
both negative. The general solution is
\[
x(t) = C_1 e^{\lambda_1 t} + C_2 e^{\lambda_2 t},
\]
with $C_1, C_2$ real. Since both $\lambda_1$ and $\lambda_2$ are strictly negative, both exponentials decay to zero as $t \to +\infty$, and the origin is again asymptotically stable. There is still no sinusoidal term, so there is no oscillation. In the phase plane, each nontrivial trajectory approaches the origin along a curve that becomes tangent to one of the two eigendirections as $t \to +\infty$. This produces the familiar picture of a stable node: nearby trajectories are drawn straight in, with no spiraling.

As $\alpha$ increases from $0$ to values larger than $1$, the qualitative classification evolves as follows:
\[
\text{center at } \alpha = 0
\;\longrightarrow\;
\text{stable spiral for } 0<\alpha<1
\;\longrightarrow\;
\text{degenerate stable node at } \alpha=1
\;\longrightarrow\;
\text{stable node for } \alpha>1.
\]
The transition from purely imaginary eigenvalues ($\alpha=0$) to complex eigenvalues with negative real part ($0<\alpha<1$) reflects the transition from closed orbits to inward spirals, while the disappearance of the imaginary part at $\alpha=1$ marks the end of oscillatory behavior.

\medskip

We now analyze the energy. Define
\[
E(t) = \frac{1}{2}\bigl(x'(t)^2 + x(t)^2\bigr)
= \frac{1}{2}\bigl(v(t)^2 + x(t)^2\bigr).
\]
Differentiating with respect to time, we obtain
\[
\frac{dE}{dt}
= v v' + x x'
= v v' + x v.
\]
Using the system equations, we substitute $v' = -2\alpha v - x$ to get
\[
\frac{dE}{dt}
= v(-2\alpha v - x) + x v
= -2\alpha v^2 - xv + xv
= -2\alpha v^2.
\]
Thus
\[
\frac{dE}{dt} = -2\alpha v^2.
\]
If $\alpha>0$, this derivative is nonpositive for all $t$, and it is equal to zero only when $v(t)=0$. Along any nontrivial solution, $v(t)$ is not identically zero, so $v(t)^2$ is positive on a set of times with nonzero measure, and hence $E(t)$ is strictly decreasing along such a solution. Physically, this expresses the fact that the damping term removes mechanical energy from the oscillator, converting it into heat or some other form of dissipated energy.

\medskip

Finally, we compare the damped system to the undamped one. When $\alpha=0$, the system is conservative: the energy $E(t)$ is exactly constant, all trajectories are closed curves encircling the origin, and the origin is a center. When $\alpha>0$, the damping term is a nonconservative perturbation. It changes the eigenvalues from purely imaginary to having negative real part, so the center becomes a stable spiral for small damping and then a node for stronger damping. At the same time, the energy becomes a strict Lyapunov function that decreases along nontrivial trajectories.

This example illustrates a central idea of phase space dynamics for conservative and perturbed systems: a seemingly small nonconservative perturbation (here, the linear damping term $2\alpha x'$) can drastically change the qualitative structure of trajectories in the phase plane, turning closed orbits into spirals and converting neutral stability into asymptotic stability.
\end{solution}

% ===== Example 4: Driven, Damped Oscillator and Resonance in Phase Space (inquiry-based) =====
\begin{problem}[Driven, Damped Oscillator and Resonance in Phase Space]
We study a mass--spring system with damping and a time-periodic forcing term. This system is a basic model for many physical and engineering situations, from car suspensions to driven circuits. Our goal is to understand, from the viewpoint of phase space dynamics, how solutions spiral toward a steady periodic motion and how this motion changes as the driving frequency approaches resonance. Along the way we will compare the conservative, damped, and driven cases.

Consider the differential equation
\[
m\ddot{x} + c\dot{x} + kx \;=\; F\cos(\omega t),
\]
where $m>0$, $c\ge 0$, $k>0$, $F\in\mathbb{R}$, and $\omega>0$ are constants.

\smallskip

(a) \textbf{Warm-up: conservative oscillator in phase space.}  

Assume first that there is no damping and no forcing, so that $c = 0$ and $F=0$, and divide the equation by $m$ to write
\[
\ddot{x} + \omega_0^2 x = 0, \qquad \text{where } \omega_0^2 = \frac{k}{m}.
\]
Introduce the velocity variable $v = \dot{x}$ and write this as a first-order system in the $(x,v)$-plane.

\begin{enumerate}
\item[(i)] Write the system in matrix form $\dot{\mathbf{y}} = A\mathbf{y}$ with $\mathbf{y} = (x,v)^{\mathsf{T}}$ and determine the eigenvalues of $A$.
\item[(ii)] Show that the quantity
\[
E(x,v) = \frac{1}{2}m v^2 + \frac{1}{2}k x^2
\]
is constant along solutions. What do the level sets of $E$ look like in the $(x,v)$-plane?
\item[(iii)] Sketch the phase portrait and describe in words how trajectories move in phase space for this conservative system.
\end{enumerate}
% Hint: For (ii), differentiate $E(x(t),v(t))$ with respect to time and use the ODE.

\smallskip

(b) \textbf{Adding damping: spirals to the origin.}  

Now include linear damping but no forcing, so that $F=0$ and $c>0$:
\[
m\ddot{x} + c\dot{x} + kx = 0.
\]
Again write the system in first-order form in the $(x,v)$-plane.

\begin{enumerate}
\item[(i)] Write down the corresponding matrix $A$ and compute its eigenvalues. Assume the \emph{underdamped} case $c^2 < 4mk$, so that the eigenvalues are complex with negative real part.
\item[(ii)] Describe the qualitative phase portrait in the $(x,v)$-plane in this underdamped case. How does it differ from the conservative case? In particular, what happens to the origin, and what happens to nearby trajectories as $t\to\infty$?
\item[(iii)] What happens to the \emph{energy} $E(x,v)$ from part (a) along solutions now?
\end{enumerate}
Hint: For (iii), compute $\frac{d}{dt}E(x(t),v(t))$ again, but now with the damping term present.

\smallskip

(c) \textbf{Adding periodic forcing: solving the ODE and finding the steady state.}  

Now consider the full driven, damped equation
\[
m\ddot{x} + c\dot{x} + kx = F\cos(\omega t),
\]
with $m>0$, $c>0$, $k>0$, and fixed driving frequency $\omega>0$.

\begin{enumerate}
\item[(i)] Solve the homogeneous equation $m\ddot{x} + c\dot{x} + kx = 0$ in the underdamped case $c^2 < 4mk$. Express the general homogeneous solution $x_{\mathrm{h}}(t)$ in terms of exponentials or sines and cosines multiplied by a decaying exponential.
\item[(ii)] Find a particular solution $x_{\mathrm{p}}(t)$ to the full inhomogeneous equation. Look for a solution of the form
\[
x_{\mathrm{p}}(t) = A\cos(\omega t) + B\sin(\omega t)
\]
for suitable constants $A$ and $B$ depending on $m,c,k,F,\omega$.  
Hint: Substitute this ansatz into the ODE and equate coefficients of $\cos(\omega t)$ and $\sin(\omega t)$.
\item[(iii)] Show that the general solution can be written as
\[
x(t) = x_{\mathrm{h}}(t) + x_{\mathrm{p}}(t),
\]
and explain why $x_{\mathrm{h}}(t)\to 0$ as $t\to\infty$ when $c>0$. Conclude that every solution approaches the same time-periodic motion $x_{\mathrm{p}}(t)$, regardless of initial conditions.
\end{enumerate}

\smallskip

(d) \textbf{Resonance and the attracting periodic orbit in phase space.}  

Still working with $c>0$, view the system in the phase plane with coordinates $(x,v)$, where $v=\dot{x}$.

\begin{enumerate}
\item[(i)] Write explicit formulas for $x_{\mathrm{p}}(t)$ and $v_{\mathrm{p}}(t)=\dot{x}_{\mathrm{p}}(t)$ in terms of an \emph{amplitude} $R(\omega)$ and a \emph{phase shift} $\phi(\omega)$, so that
\[
x_{\mathrm{p}}(t) = R(\omega)\cos\bigl(\omega t - \phi(\omega)\bigr).
\]
Determine $R(\omega)$ explicitly.  
Hint: Use your expressions for $A$ and $B$ from part (c)(ii) and apply a trigonometric identity like $a\cos(\theta) + b\sin(\theta) = \sqrt{a^2+b^2}\,\cos(\theta-\phi)$ for a suitable $\phi$.
\item[(ii)] Show that
\[
R(\omega) = \frac{F}{\sqrt{(k - m\omega^2)^2 + (c\omega)^2}}.
\]
For fixed $m,c,k$, analyze how $R(\omega)$ depends on $\omega$. At approximately which driving frequency $\omega$ is $R(\omega)$ largest? (You may use calculus or a qualitative argument.)
\item[(iii)] Describe the limiting motion in the phase plane. What curve is traced by $t\mapsto\bigl(x_{\mathrm{p}}(t),v_{\mathrm{p}}(t)\bigr)$ as $t$ varies over one forcing period $T = \tfrac{2\pi}{\omega}$? How do other trajectories $(x(t),\dot{x}(t))$ behave relative to this curve as $t\to\infty$?
\item[(iv)] As $\omega$ approaches the frequency that maximizes $R(\omega)$ (the resonance frequency), how does the size of the attracting periodic orbit in the $(x,v)$-plane change? How would you see this change if you sketched several phase portraits for increasing $\omega$?
\end{enumerate}

\smallskip

(e) \textbf{Poincaré (stroboscopic) map and extended phase space.}  

One way to study the long-time behavior of a periodically forced system is to sample the solution once every forcing period. Let $T = \tfrac{2\pi}{\omega}$ and define the \emph{Poincaré map} $P$ by
\[
P:\ (x_0,v_0)\mapsto\bigl(x(T),v(T)\bigr),
\]
where $x(t)$ is the solution of the ODE with initial data $x(0)=x_0$, $\dot{x}(0)=v_0$.

\begin{enumerate}
\item[(i)] Explain why the Poincaré map $P$ is an \emph{affine} map of the form
\[
P(\mathbf{y}) = M\mathbf{y} + \mathbf{b}
\]
for some $2\times 2$ matrix $M$ and vector $\mathbf{b}\in\mathbb{R}^2$. (You do not need to find $M$ and $\mathbf{b}$ explicitly, but explain the reasoning.)
\item[(ii)] Argue that all eigenvalues of $M$ have modulus strictly less than $1$ when $c>0$. (Hint: Think about what happens in the unforced damped case $F=0$ over one period.) Conclude that $P$ has a unique fixed point, and that every orbit under repeated application of $P$ converges to this fixed point.
\item[(iii)] Interpret this fixed point in terms of the original differential equation. What trajectory in the continuous-time phase portrait does it correspond to?
\item[(iv)] (Open-ended) How might you view the driven damped oscillator as an \emph{autonomous} system in a three-dimensional \emph{extended phase space}? What additional variable would you introduce, and what would its evolution equation be?
\end{enumerate}

In your answers, emphasize the geometric picture: closed curves versus spirals in phase space, the effect of damping and forcing on energy, and how resonance appears both in the time series $x(t)$ and in the phase portrait.
\end{problem}

% ===== Example 4: Driven, Damped Oscillator and Resonance in Phase Space (full solution) =====
\begin{problem}[Driven, Damped Oscillator and Resonance in Phase Space]
Consider the driven, damped harmonic oscillator
\[
m\ddot{x} + c\dot{x} + kx = F\cos(\omega t),
\]
with $m>0$, $c>0$, $k>0$, forcing amplitude $F\in\mathbb{R}$, and driving frequency $\omega>0$.

\begin{enumerate}
\item[(i)] Rewrite the equation as a first-order system in the phase plane $(x,v)$ with $v=\dot{x)$. Describe qualitatively the phase portrait when $F=0$ in the cases $c=0$ (conservative) and $c>0$ with $c^2<4mk$ (underdamped).
\item[(ii)] For $F\neq 0$, solve the inhomogeneous equation by finding the general homogeneous solution and a particular solution of the form $x_{\mathrm{p}}(t) = A\cos(\omega t)+B\sin(\omega t)$. Show that all solutions converge, as $t\to\infty$, to a unique time-periodic steady state $x_{\mathrm{p}}(t)$ of period $T = \tfrac{2\pi}{\omega}$.
\item[(iii)] Express the steady state as
\[
x_{\mathrm{p}}(t) = R(\omega)\cos\bigl(\omega t - \phi(\omega)\bigr)
\]
for suitable $R(\omega)>0$ and phase shift $\phi(\omega)$, and show that
\[
R(\omega) = \frac{F}{\sqrt{(k - m\omega^2)^2 + (c\omega)^2}}.
\]
Determine the driving frequency $\omega$ at which $R(\omega)$ is maximal (resonance) and describe how this affects the size of the attracting periodic orbit in the $(x,v)$ phase plane.
\item[(iv)] Define the Poincaré map $P:\mathbb{R}^2\to\mathbb{R}^2$ by
\[
P(x_0,v_0) = \bigl(x(T),v(T)\bigr),
\]
where $x(t)$ solves the ODE with initial data $x(0)=x_0$, $\dot{x}(0)=v_0$, and $T=\tfrac{2\pi}{\omega}$. Explain why $P$ is an affine map $P(\mathbf{y}) = M\mathbf{y} + \mathbf{b}$ with all eigenvalues of $M$ of modulus less than $1$, and conclude that $P$ has a unique attracting fixed point corresponding to the periodic orbit. Briefly state how this example illustrates the role of phase space and Poincaré maps in understanding conservative versus perturbed systems.
\end{enumerate}
\end{problem}

\begin{solution}
We analyze the system
\[
m\ddot{x} + c\dot{x} + kx = F\cos(\omega t)
\]
step by step, emphasizing both explicit solutions and their phase space interpretation.

\medskip

\noindent\textbf{(i) First-order formulation and unforced phase portraits.}

Introduce the velocity variable $v = \dot{x}$ and let $\mathbf{y} = (x,v)^{\mathsf{T}}$. Then
\[
\dot{x} = v,\qquad
m\dot{v} = -c v - k x + F\cos(\omega t).
\]
Dividing the second equation by $m$ gives
\[
\dot{x} = v,\qquad
\dot{v} = -\frac{k}{m}x - \frac{c}{m}v + \frac{F}{m}\cos(\omega t).
\]
This is a linear, time-periodic, non-autonomous system in the $(x,v)$-plane.

When $F=0$, the system becomes autonomous:
\[
\dot{x} = v,\qquad
\dot{v} = -\frac{k}{m}x - \frac{c}{m}v.
\]
In matrix form,
\[
\dot{\mathbf{y}} = A\mathbf{y},\qquad
A = \begin{pmatrix}
0 & 1 \\
-\dfrac{k}{m} & -\dfrac{c}{m}
\end{pmatrix}.
\]

\emph{Case $c=0$ (conservative oscillator).} Then
\[
A = \begin{pmatrix}
0 & 1 \\
-\dfrac{k}{m} & 0
\end{pmatrix},
\]
with characteristic polynomial $\lambda^2 + \dfrac{k}{m} = 0$. The eigenvalues are
\[
\lambda = \pm i\omega_0,\qquad \omega_0 = \sqrt{\frac{k}{m}}.
\]
These are purely imaginary, so the origin is a linear center. One can verify that the “energy”
\[
E(x,v) = \frac{1}{2}m v^2 + \frac{1}{2}k x^2
\]
is constant along trajectories. Indeed,
\[
\frac{d}{dt}E(x(t),v(t))
= m v\dot{v} + k x\dot{x}
= m v\Bigl(-\frac{k}{m}x\Bigr) + k x v
= -k xv + k xv = 0.
\]
Thus $E$ is conserved, and level sets $E=\text{constant}>0$ are ellipses in the $(x,v)$-plane. The phase portrait consists of closed, periodic orbits encircling the origin, which is a nonlinear center.

\emph{Case $c>0$ with $c^2<4mk$ (underdamped).} The characteristic polynomial is
\[
\lambda^2 + \frac{c}{m}\lambda + \frac{k}{m} = 0,
\]
with discriminant $\Delta = \frac{c^2}{m^2} - 4\frac{k}{m} = \frac{c^2-4mk}{m^2}<0$. Hence the eigenvalues are complex with negative real part:
\[
\lambda = -\frac{c}{2m} \pm i\sqrt{\frac{k}{m} - \frac{c^2}{4m^2}}.
\]
The origin is a stable spiral (or spiral sink). Every nontrivial trajectory spirals inward toward the origin as $t\to\infty$.

The energy $E(x,v)$ is no longer conserved. A direct computation gives
\[
\frac{d}{dt}E(x(t),v(t))
= m v\dot{v} + k x\dot{x}
= m v\Bigl(-\frac{c}{m}v - \frac{k}{m}x\Bigr) + k x v
= -c v^2 \le 0.
\]
Thus the energy decreases monotonically due to dissipation, and the ellipses of constant $E$ shrink over time to the origin, explaining the spiral trajectories.

\medskip

\noindent\textbf{(ii) General solution with forcing and convergence to a periodic state.}

We now consider
\[
m\ddot{x} + c\dot{x} + kx = F\cos(\omega t),\qquad m>0,\ c>0,\ k>0.
\]
The associated homogeneous equation
\[
m\ddot{x} + c\dot{x} + kx = 0
\]
has, in the underdamped case $c^2<4mk$, a general solution of the form
\[
x_{\mathrm{h}}(t) = e^{-\alpha t}\bigl(C_1\cos(\beta t) + C_2\sin(\beta t)\bigr),
\]
where
\[
\alpha = \frac{c}{2m}>0,\qquad
\beta = \sqrt{\frac{k}{m} - \frac{c^2}{4m^2}}>0,
\]
and $C_1,C_2$ are determined by initial conditions. The exponential factor $e^{-\alpha t}$ implies $x_{\mathrm{h}}(t)\to 0$ and $\dot{x}_{\mathrm{h}}(t)\to 0$ as $t\to\infty$.

To find a particular solution to the full forced equation, we seek a steady-state response with the same frequency as the forcing:
\[
x_{\mathrm{p}}(t) = A\cos(\omega t) + B\sin(\omega t).
\]
Differentiating,
\[
\dot{x}_{\mathrm{p}}(t) = -A\omega\sin(\omega t) + B\omega\cos(\omega t),
\]
\[
\ddot{x}_{\mathrm{p}}(t) = -A\omega^2\cos(\omega t) - B\omega^2\sin(\omega t).
\]
Substituting into the ODE gives
\[
m(-A\omega^2\cos\omega t - B\omega^2\sin\omega t)
+ c(-A\omega\sin\omega t + B\omega\cos\omega t)
+ k(A\cos\omega t + B\sin\omega t)
= F\cos(\omega t).
\]
Grouping the coefficients of $\cos(\omega t)$ and $\sin(\omega t)$, we obtain
\[
\text{cosine:}\quad (-m\omega^2 + k)A + c\omega B = F,
\]
\[
\text{sine:}\quad (-m\omega^2 + k)B - c\omega A = 0.
\]
This is a linear system for $A$ and $B$. Solving the second equation for $B$ gives
\[
B = \frac{c\omega}{k - m\omega^2}\,A
\]
(provided $k-m\omega^2\neq 0$; that special case can be treated by continuity). Substituting into the cosine equation yields
\[
(k - m\omega^2)A + c\omega\cdot\frac{c\omega}{k - m\omega^2}A = F,
\]
so
\[
\left[(k - m\omega^2) + \frac{c^2\omega^2}{k - m\omega^2}\right]A = F.
\]
Thus
\[
A = \frac{F(k - m\omega^2)}{(k - m\omega^2)^2 + (c\omega)^2},\qquad
B = \frac{Fc\omega}{(k - m\omega^2)^2 + (c\omega)^2}.
\]

The general solution of the forced equation is therefore
\[
x(t) = x_{\mathrm{h}}(t) + x_{\mathrm{p}}(t),
\]
with $x_{\mathrm{h}}(t)$ decaying exponentially in time and $x_{\mathrm{p}}(t)$ bounded and periodic of period $T = 2\pi/\omega$. Since $x_{\mathrm{h}}(t)\to 0$ and $\dot{x}_{\mathrm{h}}(t)\to 0$ as $t\to\infty$, every solution $(x(t),\dot{x}(t))$ converges to the same $T$-periodic orbit defined by $x_{\mathrm{p}}(t)$, independent of initial conditions.

\medskip

\noindent\textbf{(iii) Amplitude, resonance, and phase-plane interpretation.}

The steady-state solution can be written more compactly in amplitude-phase form. Recall that any linear combination $A\cos(\omega t)+B\sin(\omega t)$ can be expressed as
\[
A\cos(\omega t)+B\sin(\omega t)
= R(\omega)\cos\bigl(\omega t - \phi(\omega)\bigr),
\]
where
\[
R(\omega) = \sqrt{A^2 + B^2},\qquad
\phi(\omega) = \arctan\left(\frac{B}{A}\right)
\]
(up to the appropriate quadrant choice for $\phi$). Therefore the amplitude of the steady-state oscillation is
\[
R(\omega) = \sqrt{A^2 + B^2}
= \sqrt{\frac{F^2\bigl[(k - m\omega^2)^2 + (c\omega)^2\bigr]}{\bigl[(k - m\omega^2)^2 + (c\omega)^2\bigr]^2}}
= \frac{F}{\sqrt{(k - m\omega^2)^2 + (c\omega)^2}}.
\]
This is the desired expression.

To find the driving frequency $\omega$ that maximizes $R(\omega)$, it is convenient to minimize the denominator
\[
D(\omega) = (k - m\omega^2)^2 + (c\omega)^2.
\]
Differentiate with respect to $\omega$:
\[
\frac{dD}{d\omega} = 2(k - m\omega^2)(-2m\omega) + 2c^2\omega
= -4m\omega(k - m\omega^2) + 2c^2\omega.
\]
Setting $\frac{dD}{d\omega}=0$ and assuming $\omega>0$ gives
\[
-4m(k - m\omega^2) + 2c^2 = 0
\quad\Longrightarrow\quad
4m^2\omega^2 = 4mk - 2c^2.
\]
Hence
\[
\omega_{\mathrm{res}}^2 = \frac{4mk - 2c^2}{4m^2} 
= \frac{k}{m} - \frac{c^2}{2m^2}.
\]
If damping is weak ($c^2 < 2mk$), then $\omega_{\mathrm{res}}>0$ and satisfies
\[
\omega_{\mathrm{res}} < \sqrt{\frac{k}{m}} = \omega_0.
\]
Thus the resonant frequency is slightly lower than the natural frequency of the undamped oscillator. At this frequency the amplitude $R(\omega)$ is maximal, and the corresponding periodic orbit in the $(x,v)$ phase plane has the largest radius (in an appropriate sense).

In phase space, the steady-state trajectory is
\[
t\mapsto \bigl(x_{\mathrm{p}}(t),v_{\mathrm{p}}(t)\bigr),
\quad
x_{\mathrm{p}}(t) = R(\omega)\cos(\omega t - \phi),\quad
v_{\mathrm{p}}(t) = \dot{x}_{\mathrm{p}}(t) = -R(\omega)\omega\sin(\omega t - \phi).
\]
Over one period $T=2\pi/\omega$, this traces a closed curve which, after an appropriate linear rescaling, is an ellipse centered at the origin. Because all homogeneous contributions decay, any solution $(x(t),\dot{x}(t))$ spirals toward this closed curve as $t\to\infty$. As the driving frequency approaches $\omega_{\mathrm{res}}$, the amplitude $R(\omega)$ grows and this limiting closed curve expands outward in the phase plane. Near resonance, the spiral transient can also take many cycles to settle, since the decay rate is governed by the damping while the amplitude is large.

\medskip

\noindent\textbf{(iv) Poincaré map, contraction, and periodic orbit.}

The forcing has period $T = 2\pi/\omega$, so it is natural to sample solutions stroboscopically at times $t=nT$. Fix an initial condition $(x_0,v_0)$ at $t=0$, and let $x(t)$ be the corresponding solution. The Poincaré map $P$ is defined by
\[
P(x_0,v_0) = \bigl(x(T),v(T)\bigr).
\]

Because the differential equation is linear in $(x,\dot{x})$ and the forcing term $F\cos(\omega t)$ does not depend on $(x,v)$, the solution at time $T$ depends \emph{affinely} on the initial data. More precisely, if we write $\mathbf{y}(t)=(x(t),v(t))^{\mathsf{T}}$, then for each fixed $t$ there exists a $2\times 2$ matrix $\Phi(t)$ (the state-transition matrix of the homogeneous system) and a vector $\mathbf{w}(t)$ such that
\[
\mathbf{y}(t) = \Phi(t)\mathbf{y}(0) + \mathbf{w}(t).
\]
At $t=T$ this gives
\[
P(\mathbf{y}_0) = \mathbf{y}(T) = M\mathbf{y}_0 + \mathbf{b},
\]
where $M = \Phi(T)$ and $\mathbf{b} = \mathbf{w}(T)$ are fixed. Thus $P$ is an affine map.

To understand the eigenvalues of $M$, note that in the unforced damped case $F=0$ we have
\[
\dot{\mathbf{y}} = A\mathbf{y},\qquad
\mathbf{y}(t) = e^{At}\mathbf{y}(0).
\]
The eigenvalues of $A$ are $\lambda_{1,2} = -\alpha \pm i\beta$, so the eigenvalues of $e^{AT}$ are $e^{\lambda_{1}T}$ and $e^{\lambda_{2}T}$, which have modulus $e^{-\alpha T}<1$. Thus the homogeneous flow over one period is a linear contraction toward $0$ in the $(x,v)$-plane.

For the forced system, the matrix $M$ is precisely $e^{AT}$, since it is the linear part of the map taking initial data to the homogeneous contribution at time $T$. Hence the eigenvalues of $M$ still satisfy $|\mu_i|<1$. It follows that the linear part of $P$ is a contraction. In particular, the affine map $P(\mathbf{y}) = M\mathbf{y} + \mathbf{b}$ has a unique fixed point $\mathbf{y}^*$ solving
\[
\mathbf{y}^* = M\mathbf{y}^* + \mathbf{b},
\quad\text{that is,}\quad
(I - M)\mathbf{y}^* = \mathbf{b},
\]
and for any initial condition $\mathbf{y}_0$,
\[
P^n(\mathbf{y}_0) \to \mathbf{y}^*\quad\text{as }n\to\infty.
\]

The fixed point $\mathbf{y}^*$ corresponds exactly to the steady-state periodic solution. Indeed, if we start at $(x(0),v(0))=\mathbf{y}^*$, then sampling after one period gives
\[
P(\mathbf{y}^*) = \mathbf{y}^*,
\]
so the state after time $T$ returns to the same point in phase space. By uniqueness of solutions, the trajectory must then be $T$-periodic. Conversely, any $T$-periodic solution yields a fixed point of $P$.

\medskip

\noindent\textbf{Conceptual summary and connection to phase space dynamics.}

In the unforced, undamped case ($c=F=0$), the system is conservative: energy is conserved, and phase space is foliated by closed orbits encircling a center. Adding damping ($c>0$, $F=0$) perturbs this picture: energy monotonically decreases, and each closed orbit is replaced by a spiral trajectory toward an attracting equilibrium at the origin.

Finally, adding periodic forcing ($c>0$, $F\neq 0$) creates a balance between input and dissipation. The origin is no longer an equilibrium, but the system develops a unique attracting periodic orbit in the $(x,v)$-plane. The phase portrait shows trajectories spiraling toward this closed curve, whose size depends on the driving frequency and becomes largest near resonance. The Poincaré map provides a discrete-time description of this attraction: it is an affine contraction with a unique attracting fixed point corresponding to the periodic orbit.

This example illustrates the main ideas of the section: how phase
space geometry changes under damping and forcing, and how Poincaré maps give a compact, discrete-time description of the long-time behavior of periodically forced systems by encoding convergence to attracting periodic orbits.
\end{solution}

% ===== Example 5: Double-Well Potential: Separatrices and Heteroclinic Orbits (inquiry-based) =====
\begin{problem}[Double-Well Potential: Separatrices and Heteroclinic Orbits]
A simple model for a particle moving in a one-dimensional double-well potential is the second-order equation
\[
x'' + V'(x)=0,\qquad V(x)=\frac14(x^2-1)^2.
\]
The potential \(V\) has two minima (the “wells”) and one maximum (the “barrier” between the wells). In phase space, some trajectories are trapped near one well, while others have enough energy to cross over the barrier and explore both wells. The delicate boundary between these types of trajectories is formed by separatrix curves; in this particular example those separatrices are homoclinic loops to a saddle equilibrium, and in related systems one can also see heteroclinic connections between distinct saddle points.

We rewrite the equation as a first-order system by introducing the velocity \(y=x'\):
\[
x' = y,\qquad y' = -V'(x)=x - x^3.
\]

\smallskip
\noindent
(a) \emph{Equilibria and the potential landscape.}  
Find all equilibrium points \((x,y)\) of the planar system and classify them as critical points (minima, maxima, or saddle-type points) of the potential \(V(x)\) in the one-dimensional sense. Sketch the graph of \(V(x)\) and indicate the three equilibrium positions on this graph.

\medskip
\noindent
(b) \emph{Energy as a first integral.}  
Show that the function
\[
H(x,y)=\frac12 y^2 + V(x)
\]
is conserved along solutions, that is, \(\frac{d}{dt}H(x(t),y(t))=0\) whenever \((x(t),y(t))\) solves the system. Conclude that solution curves in the phase plane lie on the level sets \(H(x,y)=E\), where \(E\) is a constant interpreted as the total mechanical energy.

\emph{Hint:} Differentiate \(H(x(t),y(t))\) with respect to time and use the system \(x'=y\), \(y'=x-x^3\).

\medskip
\noindent
(c) \emph{Equilibria in phase space and their linearization.}  
Determine all equilibrium points \((x,y)\) of the planar system directly, and classify their type in the phase plane (center, saddle, node, etc.) using the Jacobian matrix of the vector field
\[
F(x,y)=(y,\;x-x^3).
\]
Compute the eigenvalues at each equilibrium. How does this phase-plane classification relate to the shape of the potential \(V(x)\) at the corresponding positions?

\emph{Hint:} At a point \(x_0\) where \(V'(x_0)=0\), the second derivative \(V''(x_0)\) determines whether \(V\) has a local minimum or maximum there, and the sign of \(V''(x_0)\) also determines whether the corresponding equilibrium in phase space is a center or a saddle.

\medskip
\noindent
(d) \emph{Separatrices and homoclinic orbits.}  

(i) Compute the energy values \(H\) at each equilibrium. Show that the two “well” equilibria \((x,y)=(\pm 1,0)\) lie at energy level \(E_{\min}\) (a minimum of \(H\)), while the “barrier” equilibrium \((0,0)\) lies at a higher energy level \(E_{\mathrm{bar}}\).

(ii) For a fixed energy \(E\in(E_{\min},E_{\mathrm{bar}})\), describe qualitatively the corresponding level curve \(H(x,y)=E\) in the phase plane. Explain why such a trajectory remains trapped in one of the wells.

(iii) For energies \(E>E_{\mathrm{bar}}\), describe how the level curves change and why such trajectories can cross over the barrier between the wells.

(iv) Show that the critical energy level \(E=E_{\mathrm{bar}}\) is special: the level set \(H(x,y)=E_{\mathrm{bar}}\) consists of two symmetric separatrix curves passing through the saddle point \((0,0)\). Show that these curves have the explicit equation
\[
H(x,y)=0 \quad\Longleftrightarrow\quad y^2 = x^2\Bigl(1-\frac{x^2}{2}\Bigr),
\]
and sketch them in the phase plane.

\emph{Hint:} Set \(H(x,y)=0\) and solve for \(y\) in terms of \(x\). Observe that the separatrices intersect the \(x\)-axis only at the origin.

(v) We can go further and parametrize one branch of the separatrix as a solution \((x(t),y(t))\) of the system. Use the energy relation for the level \(H=0\),
\[
\frac12 (x')^2 + \frac14 (x^2-1)^2 = 0,
\]
to obtain a separable first-order equation for \(x(t)\). Solve this equation to show that there are homoclinic orbits of the form
\[
x(t) = \pm \sqrt{2}\,\operatorname{sech}(t-t_0), 
\qquad y(t)=x'(t),
\]
which leave the saddle \((0,0)\) as \(t\to -\infty\), loop around one of the wells, and return to \((0,0)\) as \(t\to +\infty\).

\emph{Hint:} After setting \(H=0\), you should find an equation of the form \(x' = \pm x\sqrt{1-\frac{x^2}{2}}\), which can be separated and integrated. A substitution based on \(x=\sqrt{2}\,\sech t\) is convenient for inverting the implicit solution.

\medskip
\noindent
(e) \emph{Extensions and perturbations.}

(i) Suppose we add a small linear damping term to the equation of motion,
\[
x'' + \varepsilon x' + V'(x) = 0,\quad \varepsilon>0 \text{ small}.
\]
Qualitatively, what happens to the conserved energy \(H\) and to the separatrix as time evolves? What kinds of trajectories would you expect to see for initial conditions that are just inside or just outside the original separatrix?

(ii) In higher-dimensional Hamiltonian systems with multiple saddle points in phase space, one often encounters \emph{heteroclinic} orbits that connect \emph{different} saddles. Based on your understanding of this double-well example, explain in words how separatrices and heteroclinic connections might organize the possible transitions between different “wells” in a more complicated potential landscape.
\end{problem}

% ===== Example 5: Double-Well Potential: Separatrices and Heteroclinic Orbits (full solution) =====
\begin{problem}[Double-Well Potential: Separatrices and Heteroclinic Orbits]
Consider the planar system
\[
x' = y,\qquad y' = x - x^3.
\]
\begin{enumerate}
\item Show that the function
\[
H(x,y)=\frac12 y^2 + \frac14(x^2-1)^2
\]
is constant along trajectories, and interpret \(H\) as the total energy of a particle moving in the double-well potential \(V(x)=\frac14(x^2-1)^2\).

\item Find all equilibria and classify them in the phase plane by computing the eigenvalues of the Jacobian of the vector field. Relate your classification to the shape of the potential \(V\).

\item Compute the energy levels \(H\) at the equilibria and describe qualitatively the phase portraits:
  \begin{enumerate}
  \item for energies between the minimum and the barrier height,
  \item for energies above the barrier height.
  \end{enumerate}

\item Show that the critical energy level through the saddle at the origin is given by \(H(x,y)=0\), and that on this level the separatrix curves satisfy
\[
y^2 = x^2\Bigl(1-\frac{x^2}{2}\Bigr).
\]
Sketch these separatrices and explain why they form the boundary between trajectories trapped in a single well and trajectories that can cross over the barrier.

\item Using the energy relation at level \(H=0\), derive a separable first-order equation for \(x(t)\) and solve it to obtain an explicit homoclinic orbit. Show that
\[
x(t) = \sqrt{2}\,\operatorname{sech}(t-t_0),\qquad y(t)=x'(t),
\]
parametrizes one branch of the separatrix and that \((x(t),y(t))\to(0,0)\) as \(t\to\pm\infty\).
\end{enumerate}
\end{problem}

\begin{solution}
We study a one-degree-of-freedom conservative mechanical system written as a first-order planar system. The central ideas are the conservation of an energy-like Hamiltonian, the classification of equilibria via linearization and the potential shape, and the role of separatrices (here, homoclinic loops) as boundaries between qualitatively different regions of phase space. This is a prototypical example of phase space dynamics for conservative systems, and it also serves as a starting point for understanding how perturbations (such as damping) deform such structures.

\medskip
\noindent\textbf{1. Energy and the double-well potential.}

The given system is
\[
x' = y,\qquad y' = x - x^3.
\]
We are told to consider
\[
H(x,y)=\frac12 y^2 + \frac14(x^2-1)^2.
\]
To show that \(H\) is conserved, we compute its derivative along a trajectory \((x(t),y(t))\):
\[
\frac{d}{dt}H(x(t),y(t))
= \frac{\partial H}{\partial x} x' + \frac{\partial H}{\partial y} y'.
\]
We first compute the partial derivatives:
\[
\frac{\partial H}{\partial x}
= \frac14\cdot 2
\[
\frac{\partial H}{\partial x}
= \frac14 \cdot 2(x^2-1)\cdot 2x
= x(x^2-1)=x^3-x,
\qquad
\frac{\partial H}{\partial y}=y.
\]
Therefore
\[
\frac{d}{dt}H(x(t),y(t))
= (x^3-x)x' + y\,y'
= (x^3-x)y + y(x-x^3) = 0.
\]
So \(H\) is constant along trajectories.

Interpreting the terms,
\[
H(x,y) = \frac12 y^2 + V(x),
\qquad V(x) = \frac14(x^2-1)^2,
\]
we see that \(\tfrac12 y^2\) is the kinetic energy and \(V(x)\) is the potential energy. Thus \(H\) is the total mechanical energy of a particle moving in the double-well potential \(V\).

\medskip
\noindent\textbf{2. Equilibria, Jacobian, and relation to the potential.}

Equilibria satisfy
\[
x'=0,\qquad y'=0
\quad\Longrightarrow\quad
y=0,\; x-x^3=0.
\]
Hence
\[
x(x^2-1)=0 \quad\Longrightarrow\quad x\in\{0,1,-1\},
\]
so the equilibria are
\[
(x,y)=(0,0),\quad (1,0),\quad (-1,0).
\]

The Jacobian of the vector field \(F(x,y)=(y,\;x-x^3)\) is
\[
DF(x,y)
=
\begin{pmatrix}
\frac{\partial}{\partial x}(y) & \frac{\partial}{\partial y}(y)\\[4pt]
\frac{\partial}{\partial x}(x-x^3) & \frac{\partial}{\partial y}(x-x^3)
\end{pmatrix}
=
\begin{pmatrix}
0 & 1\\[4pt]
1-3x^2 & 0
\end{pmatrix}.
\]
At an equilibrium \((x_0,0)\), the Jacobian is
\[
J(x_0)=
\begin{pmatrix}
0 & 1\\
1-3x_0^2 & 0
\end{pmatrix}.
\]
The characteristic polynomial is
\[
\lambda^2 - \mathrm{tr}(J)\lambda + \det(J)
= \lambda^2 + \det(J),
\]
since \(\mathrm{tr}(J)=0\). Moreover
\[
\det(J)=0\cdot 0 - 1(1-3x_0^2) = -(1-3x_0^2),
\]
so the eigenvalues satisfy
\[
\lambda^2 - (1-3x_0^2)=0
\quad\Longrightarrow\quad
\lambda=\pm\sqrt{1-3x_0^2}.
\]

\emph{At \((0,0)\):}  
Here \(x_0=0\), so
\[
J(0,0)=
\begin{pmatrix}
0 & 1\\
1 & 0
\end{pmatrix},
\qquad
\lambda=\pm1.
\]
The eigenvalues are real with opposite signs, so \((0,0)\) is a \emph{saddle} in the phase plane.

\emph{At \((\pm1,0)\):}  
Here \(x_0=\pm1\), so \(1-3x_0^2=1-3=-2\), and
\[
J(\pm1,0)=
\begin{pmatrix}
0 & 1\\
-2 & 0
\end{pmatrix},
\qquad
\lambda^2+2=0 \;\Longrightarrow\; \lambda=\pm i\sqrt{2}.
\]
The eigenvalues are purely imaginary, so the linearization has a center. Because the system is Hamiltonian (it has a nondegenerate conserved quantity \(H\)), these equilibria are in fact \emph{nonlinear centers}: nearby trajectories are closed curves (small oscillations in each well).

\smallskip
\noindent\emph{Relation to the potential.}

The equilibria occur at critical points of \(V\), that is, at \(V'(x_0)=0\). Here
\[
V'(x)=x(x^2-1),\qquad
V''(x)=3x^2-1.
\]
Then
\[
V''(0)=-1<0 \;\Rightarrow\; V \text{ has a local maximum at } x=0,
\]
\[
V''(\pm1)=2>0 \;\Rightarrow\; V \text{ has local minima at } x=\pm1.
\]
Thus the particle sits at the top of the barrier at \(x=0\) (unstable equilibrium, saddle in phase space) and at the bottoms of the wells at \(x=\pm1\) (stable in the 1D mechanical sense, centers in phase space). This is the standard correspondence in 1D Hamiltonian systems: local maxima of \(V\) give saddles, local minima of \(V\) give centers.

\medskip
\noindent\textbf{3. Energy levels and qualitative phase portraits.}

We first compute the energy at the equilibria:
\[
H(x,y)=\frac12 y^2 + \frac14(x^2-1)^2.
\]

\emph{At the well minima \((\pm1,0)\):}
\[
H(\pm1,0)
= \frac12\cdot 0^2 + \frac14\,(1-1)^2
=0.
\]
This is the minimum possible energy, so \(E_{\min}=0\).

\emph{At the barrier top \((0,0)\):}
\[
H(0,0)=\frac12\cdot0^2 + \frac14\,(0^2-1)^2=\frac14.
\]
Thus the barrier energy is \(E_{\mathrm{bar}}=\frac14\).

\smallskip
\noindent\emph{(a) Energies between the minimum and the barrier.}

Take \(E\) with
\[
0<E<\frac14.
\]
The level set in the phase plane is defined by
\[
\frac12 y^2 + \frac14(x^2-1)^2 = E.
\]
For such \(E\), the condition \(V(x)\le E\) restricts \(x\) to two disjoint intervals, one around \(x=-1\) and one around \(x=+1\), because the barrier at \(x=0\) has higher potential \(V(0)=\frac14>E\). Consequently, the level set \(H(x,y)=E\) consists of \emph{two disjoint closed curves}: one surrounding the center at \((-1,0)\), the other surrounding the center at \((1,0)\). These correspond to periodic oscillations trapped in a single well: the particle oscillates back and forth around one minimum and cannot reach or cross the barrier.

\smallskip
\noindent\emph{(b) Energies above the barrier.}

Now take \(E>\frac14\). Then
\[
V(x)\le E
\]
for all \(x\) in a single connected interval \([-x_{\max},x_{\max}]\) containing the origin, because \(V(x)\to\infty\) as \(|x|\to\infty\) but its maximum between the wells is only \(\tfrac14<E\). Hence the level set \(H(x,y)=E\) is a \emph{single} closed curve that passes over the barrier region near \((0,0)\). 

In terms of dynamics, the particle moves in a single large oscillation that explores both wells: it repeatedly passes over the barrier, moves to one side, turns around at a turning point, crosses back over the barrier, and so on. In the phase plane, this is a large closed orbit enclosing all three equilibria.

\medskip
\noindent\textbf{4. The critical energy level and the separatrices.}

The critical energy is the barrier energy
\[
E_{\mathrm{bar}} = H(0,0) = \frac14.
\]
The corresponding level set
\[
H(x,y)=\frac14
\]
passes through the saddle \((0,0)\) and separates the “trapped” oscillations from the “over-the-barrier” oscillations.

Starting from
\[
\frac12 y^2 + \frac14(x^2-1)^2 = \frac14,
\]
we simplify:
\[
\frac12 y^2 + \frac14(x^2-1)^2 - \frac14 = 0
\;\Longleftrightarrow\;
\frac12 y^2 + \frac14(x^4 - 2x^2 + 1) -\frac14 =0
\]
\[
\Longleftrightarrow\;
\frac12 y^2 + \frac14 x^4 - \frac12 x^2 =0
\;\Longleftrightarrow\;
y^2 + \frac12 x^4 - x^2 =0
\]
\[
\Longleftrightarrow\;
y^2 = x^2 - \frac12 x^4
= x^2\Bigl(1-\frac{x^2}{2}\B
igr).
\]
Thus the critical level through the saddle is given implicitly by
\[
y^2 = x^2\Bigl(1-\frac{x^2}{2}\Bigr),
\]
or, equivalently,
\[
y = \pm\, x\,\sqrt{1-\frac{x^2}{2}},\qquad |x|\le \sqrt{2}.
\]
These curves are symmetric with respect to both coordinate axes and pass through the saddle \((0,0)\). They form two closed “figure-eight”–like loops (one around each well), and they separate initial conditions leading to trajectories trapped in a single well (inside the loops) from those that can cross the barrier (outside the loops).

\medskip
\noindent\textbf{5. Explicit homoclinic orbit on the separatrix.}

We now parametrize one branch of the separatrix as a solution \((x(t),y(t))\). On the critical energy level through the saddle we have
\[
H(x,y)=\frac14 \quad\Longleftrightarrow\quad
\frac12 (x')^2 + \frac14(x^2-1)^2 = \frac14.
\]
Rewriting,
\[
\frac12 (x')^2 = \frac14 - \frac14(x^2-1)^2
= \frac14\bigl[1-(x^2-1)^2\bigr]
= \frac14\bigl(2x^2 - x^4\bigr)
= \frac14 x^2(2-x^2),
\]
so
\[
(x')^2 = x^2\Bigl(1-\frac{x^2}{2}\Bigr)
\quad\Longrightarrow\quad
x' = \pm x\sqrt{1-\frac{x^2}{2}}.
\]
This is a separable first-order equation. Instead of performing the full integration, we can propose the explicit form suggested in the problem statement and verify that it satisfies the second-order equation
\[
x'' = x - x^3.
\]

Take
\[
x(t) = \sqrt{2}\,\sech\bigl(t-t_0\bigr),
\]
with \(t_0\in\mathbb{R}\) arbitrary (time-translation invariance). Let \(s=t-t_0\); then
\[
x(t) = \sqrt{2}\,\sech s.
\]
Using the identity
\[
\frac{d^2}{ds^2}\sech s = \sech s - 2\sech^3 s,
\]
we compute
\[
x''(t) = \sqrt{2}\,\bigl(\sech s - 2\sech^3 s\bigr).
\]
Moreover,
\[
x^3(t) = (\sqrt{2}\,\sech s)^3 = 2\sqrt{2}\,\sech^3 s,
\]
so
\[
x(t) - x^3(t)
= \sqrt{2}\,\sech s - 2\sqrt{2}\,\sech^3 s
= \sqrt{2}\,\bigl(\sech s - 2\sech^3 s\bigr)
= x''(t).
\]
Hence \(x(t)\) satisfies
\[
x'' = x - x^3,
\]
which is equivalent to the original planar system \(x'=y,\ y'=x-x^3\) when we set
\[
y(t) = x'(t) = -\sqrt{2}\,\sech s\,\tanh s.
\]

As \(t\to\pm\infty\), we have \(s\to\pm\infty\) and \(\sech s\to 0\), \(\tanh s\to\pm1\), so
\[
x(t)\to 0,\qquad y(t)=x'(t)\to 0.
\]
Thus \((x(t),y(t))\) is a \emph{homoclinic orbit} to the saddle \((0,0)\): it leaves the saddle as \(t\to-\infty\), loops around one well, and returns to the saddle as \(t\to+\infty\).

By symmetry, the function
\[
x(t) = -\sqrt{2}\,\sech(t-t_0),\qquad y(t)=x'(t),
\]
gives the symmetric homoclinic loop around the other well. Together, these two homoclinic trajectories form the pair of separatrix loops that bound the trapped oscillations in each well and separate them from the over-the-barrier motions.

\end{solution}

% ===== Example 6: Small Perturbations of a Hamiltonian System (inquiry-based) =====
\begin{problem}[Small Perturbations of a Hamiltonian System]
Consider a unit mass attached to a linear spring, moving without friction on a horizontal surface. In the idealized, frictionless case, the motion is perfectly periodic and the total mechanical energy is conserved. In phase space, this gives rise to closed orbits corresponding to constant energy levels. In reality, small non-conservative effects such as weak damping cause the energy to change slowly in time, and trajectories drift between these energy levels in a way that can often be captured by averaging or energy-balance methods.

We study the scalar equation
\[
x'' + x = -\varepsilon x', \qquad 0 < \varepsilon \ll 1,
\]
where $x(t)$ is the displacement, the term $x$ arises from the restoring spring force, and the term $-\varepsilon x'$ models weak linear damping.

\medskip

(a) First consider the unperturbed system with no damping,
\[
x'' + x = 0.
\]
Introduce the velocity variable $y = x'$ and write the system in first-order form. Define the \emph{Hamiltonian}
\[
H(x,y) = \frac{1}{2}x^2 + \frac{1}{2}y^2.
\]
Show that along any solution of the unperturbed system we have $\dfrac{dH}{dt} = 0$. Describe the level sets $\{H = E\}$ in the $(x,y)$-plane and explain why each of them (for $E>0$) is a periodic orbit. What is the period of these orbits?

% Hint: Compute $\dot H = H_x \dot x + H_y \dot y$ and substitute the equations of motion. For the period, solve $x''+x=0$ explicitly, or note that the system is a harmonic oscillator.

\medskip

(b) Now consider the weakly damped system
\[
x'' + x = -\varepsilon x', \qquad 0<\varepsilon \ll 1.
\]
Again set $y = x'$ and use the same Hamiltonian $H(x,y)$ as in part (a). Compute $\dfrac{dH}{dt}$ along solutions of this \emph{perturbed} system and show that
\[
\frac{dH}{dt} = -\varepsilon\, y^2 \le 0.
\]
Interpret this inequality in terms of the mechanical energy of the oscillator and the shape of trajectories in the phase plane. What happens to the invariant curves $\{H = \text{constant}\}$ from the Hamiltonian system when $\varepsilon>0$?

% Hint: Only the $y'$ equation has changed compared to part (a). Think about whether a trajectory can stay on a fixed level set of $H$ when $dH/dt<0$ unless $y$ vanishes.

\medskip

(c) In the conservative case $\varepsilon = 0$, the general nontrivial solution can be written in the form
\[
x(t) = r \cos(t+\phi), \qquad y(t) = x'(t) = -r \sin(t+\phi),
\]
where $r>0$ and $\phi$ are constants determined by initial data, and the energy is
\[
H = \frac{1}{2}r^2.
\]
For fixed energy $H=E>0$, compute the \emph{time average} over one period $T = 2\pi$ of the quantity $y^2(t)$:
\[
\langle y^2 \rangle := \frac{1}{T} \int_0^T y^2(t)\,dt.
\]
Show that $\langle y^2 \rangle = E$.

% Hint: Use the trigonometric identity $\sin^2(\theta) = \frac{1}{2}(1-\cos(2\theta))$ and remember that the average of $\cos(2\theta)$ over a full period is zero.

\medskip

(d) We now use the idea that weak damping ($0<\varepsilon\ll 1$) causes the energy to change slowly compared to the fast oscillation. Suppose that for small $\varepsilon>0$ the motion is still approximately sinusoidal with a slowly varying amplitude $r(t)$, so that
\[
x(t) \approx r(t) \cos(t+\phi), \qquad y(t) \approx -r(t) \sin(t+\phi),
\]
and hence $H(t) \approx \tfrac{1}{2}r(t)^2$.

Using your expression for $\dfrac{dH}{dt}$ from part (b) and your average $\langle y^2\rangle$ from part (c), argue (by averaging over one oscillation period) that for small $\varepsilon$
\[
\frac{dH}{dt} \approx -\varepsilon H.
\]
Solve this approximate differential equation for $H(t)$ and express the result in terms of the initial energy $H(0) = H_0$. Compare the decay timescale of $H(t)$ to the oscillation period. How does this manifest in the phase portrait: what do typical trajectories look like in the $(x,y)$-plane when $\varepsilon > 0$ is small?

% Hint: You are replacing $y^2(t)$ by its average value over one period. The resulting ODE for $H$ is linear and separable.

\medskip

(e) Explorations and extensions.

\begin{enumerate}
  \item Suppose instead that the equation is
  \[
  x'' + x = +\varepsilon x'.
  \]
  How does the sign change in $\varepsilon$ affect the sign of $\dfrac{dH}{dt}$ and the qualitative phase portrait? What do typical orbits do now?

  % Hint: Repeat the computation in part (b) with the new sign and think about spirals in the phase plane.

  \item Consider adding a small periodic forcing term:
  \[
  x'' + x = -\varepsilon x' + \varepsilon a \cos(\omega t),
  \]
  with constants $a,\omega>0$ and $0<\varepsilon\ll 1$. Without doing detailed calculations, discuss qualitatively how you might extend the energy-balance or averaging ideas from parts (b)–(d) to study the long-term behavior. In particular, what role might the forcing frequency $\omega$ play in whether the energy tends to zero, stays bounded away from zero, or grows?

  % Hint: Think about whether the forcing is in resonance with the natural frequency (here equal to 1) of the unperturbed oscillator, and how this might affect the average energy input per cycle versus the energy lost to damping.
\end{enumerate}

\end{problem}

% ===== Example 6: Small Perturbations of a Hamiltonian System (full solution) =====
\begin{problem}[Small Perturbations of a Hamiltonian System]
Consider the weakly damped harmonic oscillator
\[
x'' + x = -\varepsilon x', \qquad 0<\varepsilon\ll 1,
\]
and define $y = x'$ and the Hamiltonian
\[
H(x,y) = \frac{1}{2}x^2 + \frac{1}{2}y^2.
\]
\begin{enumerate}
  \item For the unperturbed system $x''+x=0$, write the corresponding first-order system, show that $H$ is conserved, and describe the phase portrait, including the period of the closed orbits.
  \item For the damped system, compute $\dfrac{dH}{dt}$ along solutions and show that $\dfrac{dH}{dt} = -\varepsilon y^2 \le 0$. Interpret this in terms of the phase portrait and the fate of the invariant curves $\{H = \text{constant}\}$.
  \item Using the fact that in the conservative case solutions can be written as $x(t) = r\cos(t+\phi)$, $y(t) = -r\sin(t+\phi)$ with $H = \tfrac{1}{2}r^2$, compute the average of $y^2(t)$ over one period and show that it equals $H$.
  \item Assuming that for $0<\varepsilon\ll 1$ the motion remains approximately sinusoidal with slowly varying amplitude, use an averaging argument to derive the approximate evolution equation
  \[
  \frac{dH}{dt} \approx -\varepsilon H,
  \]
  solve it for $H(t)$, and describe qualitatively the resulting trajectories in the phase plane.
\end{enumerate}
Explain briefly how this example illustrates the effect of small non-conservative perturbations on a Hamiltonian system with closed orbits.
\end{problem}

\begin{solution}
We begin by viewing the equation as a planar system in phase space, with position $x$ and velocity $y=x'$, and by using the function $H(x,y)$ as the energy.

\medskip

\textbf{(1) Unperturbed system: Hamiltonian structure and phase portrait.}

For the unperturbed system,
\[
x'' + x = 0,
\]
we introduce $y = x'$ to obtain the first-order system
\[
\begin{cases}
x' = y,\\[0.4em]
y' = -x.
\end{cases}
\]
We define the Hamiltonian
\[
H(x,y) = \frac{1}{2}x^2 + \frac{1}{2}y^2.
\]
To check conservation of $H$, we differentiate $H$ along a solution:
\[
\frac{dH}{dt} = H_x x' + H_y y' = x\cdot x' + y\cdot y' = x y + y(-x) = 0.
\]
Thus $H$ is constant along trajectories. Geometrically, the level sets
\[
\{(x,y)\colon H(x,y) = E\} = \left\{(x,y)\colon \frac{1}{2}x^2 + \frac{1}{2}y^2 = E\right\}
\]
are circles of radius $r = \sqrt{2E}$ centered at the origin in the $(x,y)$-plane. Since $H$ is constant along solutions, each trajectory with $E>0$ lies on one such circle, giving a closed orbit. The origin corresponds to $E=0$ and is an equilibrium.

The scalar equation $x''+x=0$ has general solution
\[
x(t) = r\cos(t+\phi), \qquad y(t) = x'(t) = -r\sin(t+\phi),
\]
for constants $r\ge0$ and phase $\phi\in\mathbb{R}$. From this representation, each nontrivial solution is periodic with period
\[
T = 2\pi,
\]
independent of $r$ (or $E$). In phase space, the motion is uniform counterclockwise rotation along the circle $x^2 + y^2 = r^2$.

\medskip

\textbf{(2) Damped system: energy decay and loss of invariant curves.}

For the weakly damped system
\[
x'' + x = -\varepsilon x', \qquad 0<\varepsilon\ll 1,
\]
we again set $y = x'$ and obtain
\[
\begin{cases}
x' = y,\\[0.3em]
y' = -x - \varepsilon y.
\end{cases}
\]
We keep the same function $H(x,y) = \frac{1}{2}x^2 + \frac{1}{2}y^2$ and compute its derivative along solutions:
\[
\frac{dH}{dt} = H_x x' + H_y y' = x y + y(-x - \varepsilon y) = x y - x y - \varepsilon y^2 = -\varepsilon\, y^2.
\]
Therefore
\[
\frac{dH}{dt} = -\varepsilon\, y^2 \le 0,
\]
with equality if and only if $y=0$. 

Physically, $H$ is the total mechanical energy (kinetic plus potential) of the oscillator. The formula shows that the energy decreases monotonically along any trajectory except on the $x$-axis, where the velocity $y$ vanishes. This expresses the dissipative effect of the damping term $-\varepsilon x'$: energy is lost at a rate proportional to the square of the velocity.

In the unperturbed Hamiltonian system, each level set $\{H=E\}$ was an invariant closed orbit. In the damped system, these curves are no longer invariant because $H$ is no longer constant; instead, trajectories cross these circles in the direction of decreasing $H$. In the phase plane, typical orbits spiral inward toward the origin, crossing successive circles $H=E$ as the energy decays. Only the equilibrium at the origin, with $x=y=0$, remains as a stationary solution; all nonzero periodic orbits are destroyed by the damping.

\medskip

\textbf{(3) Time average of $y^2$ in the conservative system.}

In the conservative case $\varepsilon = 0$, we have
\[
x(t) = r\cos(t+\phi), \qquad y(t) = -r\sin(t+\phi)
\]
for some amplitude $r>0$ and phase $\phi$. The energy is
\[
H = \frac{1}{2}x^2 + \frac{1}{2}y^2 = \frac{1}{2}r^2.
\]
For a fixed energy level $H=E>0$, we thus have $r^2 = 2E$. Then
\[
y^2(t) = r^2 \sin^2(t+\phi) = 2E \sin^2(t+\phi).
\]
The period of motion is $T=2\pi$. The time average of $y^2$ over one period is
\[
\langle y^2 \rangle 
= \frac{1}{T}\int_0^T y^2(t)\,dt
= \frac{1}{2\pi}\int_0^{2\pi} 2E \sin^2(t+\phi)\,dt.
\]
Using $\sin^2\theta = \tfrac{1}{2}(1-\cos 2\theta)$, we find
\[
\sin^2(t+\phi) = \frac{1}{2}\bigl(1 - \cos(2t+2\phi)\bigr),
\]
so
\[
\langle y^2 \rangle 
= \frac{1}{2\pi}\int_0^{2\pi} 2E \cdot \frac{1}{2}(1 - \cos(2t+2\phi))\,dt
= \frac{E}{\pi}\int_0^{2\pi} \bigl(1 - \cos(2t+2\phi)\bigr)\,dt.
\]
The integral of $1$ over $[0,2\pi]$ is $2\pi$, and the integral of $\cos(2t+2\phi)$ over a full period is zero. Thus
\[
\langle y^2 \rangle = \frac{E}{\pi} \cdot 2\pi = E.
\]
In other words, for the harmonic oscillator the average kinetic energy per unit mass (which is $\tfrac{1}{2}\langle y^2\rangle$) is exactly half of the total energy, and the average of $y^2$ is equal to $H$ itself.

\medskip

\textbf{(4) Averaged evolution of energy under weak damping.}

In the damped system we found
\[
\frac{dH}{dt} = -\varepsilon y^2(t).
\]
For $0<\varepsilon\ll 1$, the damping is weak, so the energy changes slowly compared to the fast oscillation with period $2\pi$. On the timescale of a single oscillation, the trajectory is still approximately sinusoidal; only over many periods does the amplitude noticeably change. This separation of timescales is at the heart of averaging methods.

To make this precise at a heuristic level, we assume that for small $\varepsilon$ the solution can be approximated by
\[
x(t) \approx r(t)\cos(t+\phi), \qquad y(t) \approx -r(t)\sin(t+\phi),
\]
where the amplitude $r(t)$ (and hence the energy $H(t)=\tfrac{1}{2}r^2(t)$) varies slowly. Substituting into the energy derivative gives
\[
\frac{dH}{dt} = -\varepsilon y^2(t) \approx -\varepsilon r^2(t)\sin^2(t+\phi).
\]
We now average this expression over one period $T=2\pi$ of the fast oscillation, keeping $H(t)$ (and thus $r(t)$) essentially constant over that short interval. Using the time average from part (3), we obtain
\[
\left\langle \frac{dH}{dt} \right\rangle 
\approx -\varepsilon \langle y^2 \rangle 
= -\varepsilon H.
\]
Thus the \emph{averaged} evolution of the slowly varying energy satisfies the approximate differential equation
\[
\frac{dH}{dt} \approx -\varepsilon H.
\]

This linear, separable equation has solution
\[
H(t) \approx H_0 e^{-\varepsilon t},
\]
where $H_0 = H(0)$ is the initial energy. The decay timescale is of order $1/\varepsilon$, which is much larger than the oscillation period $2\pi$ when $\varepsilon\ll 1$. Hence the system performs many nearly Hamiltonian oscillations before its energy changes significantly.

In the phase plane, this means that trajectories still look like closed circles for short times, but on a longer timescale they slowly spiral inward toward the origin, crossing the former invariant energy curves. Locally, the motion is tangent to the Hamiltonian circles (fast rotation), but there is a slow radial drift inward due to the damping. This is a typical picture of how a small non-conservative perturbation deforms invariant curves of a conservative system and turns periodic orbits into slow spirals.

\medskip

\textbf{Connection to the chapter theme.}

This example begins with a two-dimensional Hamiltonian system whose phase portrait consists of closed orbits corresponding to energy level sets. The Hamiltonian $H$ is conserved, and the phase flow is area-preserving. Introducing a small non-conservative perturbation in the form of weak damping breaks exact energy conservation: the invariant curves $\{H=\text{constant}\}$ are no longer invariant, and trajectories drift monotonically across them. By exploiting the separation between the fast oscillations and the slow energy decay, we use an averaging (or energy-balance) argument to derive an effective one-dimensional evolution law for the energy. This illustrates a central theme of \emph{phase space dynamics for conservative and perturbed systems}: small perturbations of Hamiltonian systems often produce slow drift along or across invariant manifolds, which can be captured by reduced, averaged equations that describe the long-term behavior.
\end{solution}


\chapter{Partial Differential Equations}

\section{First-Order PDE: Method of Characteristics}
% --- Narrative plan (auto-generated) ---
% This section introduces first-order partial differential equations and the method of characteristics, a geometric technique that converts certain PDEs into families of ordinary differential equations along special curves called characteristic curves. Instead of attacking the full PDE at once, we follow the flow of information along these curves, turning a complicated multivariable problem into simpler one-dimensional problems. This viewpoint is especially natural in models where something is transported or propagated with a given velocity, such as temperature in a moving fluid, cars in traffic flow, or waves with slowly varying shape.  The method of characteristics is central in applied mathematics because it links PDEs to dynamical systems: each characteristic is a trajectory determined by an ODE, and the PDE solution is reconstructed from this characteristic family. The same ideas reappear in Hamilton–Jacobi theory and classical mechanics, in geometrical optics and the eikonal equation, and in conservation laws and shock formation, which in turn motivate weak solutions and distribution theory. Conceptually, the method of characteristics sits at a crossroads with other areas such as Fourier analysis (where first-order PDEs sometimes linearize or simplify), complex analysis (through analytic continuation along curves), and numerical analysis (through characteristic-based schemes used in fluid dynamics and advection-dominated problems).

% ===== Example 1: Linear Transport Equation with Constant Velocity (inquiry-based) =====
\begin{problem}[Linear Transport Equation with Constant Velocity]
A pollutant is dissolved in water flowing steadily along a straight, one-dimensional channel. Let $x$ denote position along the channel and $t$ denote time. Suppose the fluid moves with constant velocity $c>0$, and the pollutant is simply carried along with the flow without reacting or diffusing. The pollutant concentration $u(x,t)$ is then modeled by the linear transport equation
\[
u_t + c\,u_x = 0,\qquad x\in\mathbb{R},\ t>0,
\]
together with an initial concentration profile
\[
u(x,0) = f(x),\qquad x\in\mathbb{R}.
\]
In this problem you will use the method of characteristics to discover and justify an explicit formula for $u(x,t)$ in terms of $f$.

\smallskip

(a) The physical description says that each individual fluid particle carries its initial concentration value as it moves. Imagine a single fluid particle that starts at position $\xi\in\mathbb{R}$ at time $t=0$ and then moves with the flow. 

\quad (i) Write an ordinary differential equation for the particle trajectory $x(t)$ if the flow speed is the constant $c>0$ and $x(0)=\xi$.  

\quad (ii) If the pollutant is simply carried (advected) by the flow without change, what does this say about the function $t\mapsto u(x(t),t)$ along this trajectory?

Hint: Think of $x(t)$ as the position of a specific tagged particle and recall that $x'(t)$ is its velocity.

\smallskip

(b) Next, we translate this physical picture into calculus. For a smooth function $u(x,t)$ and a smooth trajectory $t\mapsto x(t)$, compute the total derivative of $u$ along the path:
\[
\frac{d}{dt}u(x(t),t).
\]
Express this derivative in terms of $u_t$, $u_x$, and $x'(t)$.

Hint: Use the chain rule with two variables, $t\mapsto (x(t),t)\mapsto u(x(t),t)$.

\smallskip

(c) Now we want to choose particle trajectories in such a way that the PDE $u_t + c u_x = 0$ implies that $u$ is constant along those trajectories.

\quad (i) Using your expression from part (b), find a condition on $x'(t)$ under which the PDE
\[
u_t + c\,u_x = 0
\]
implies that $\dfrac{d}{dt} u(x(t),t) = 0$ along the trajectory $x(t)$.

\quad (ii) Solve the resulting ordinary differential equation for $x(t)$, using the initial condition $x(0) = \xi$.  

\quad (iii) For a fixed $(x,t)$ with $t>0$, solve your formula for $x(t)$ to express the “starting point” $\xi$ of the characteristic passing through $(x,t)$ in terms of $x$ and $t$.

Hint: You should find a family of straight lines in the $(x,t)$-plane, sometimes called characteristic curves.

\smallskip

(d) Because $\dfrac{d}{dt} u(x(t),t) = 0$ along each trajectory, the concentration carried by a fluid particle is constant in time.

\quad (i) Express $u(x(t),t)$ in terms of its initial value at $t=0$ along the same trajectory, using your answer to part (a)(ii).  

\quad (ii) Use your expression for the starting point $\xi$ from part (c)(iii) to rewrite $u(x,t)$ explicitly in terms of $f$ and the variables $x$ and $t$.  

\quad (iii) Check directly that your formula for $u(x,t)$ satisfies both the PDE $u_t + c\,u_x = 0$ and the initial condition $u(x,0)=f(x)$.

Hint: For (ii), substitute $\xi = \xi(x,t)$ into the formula $u(x(t),t) = f(\xi)$. For (iii), compute $u_t$ and $u_x$ for your candidate solution and verify the equation.

\smallskip

(e) Explore some variations and extensions.

\quad (i) Suppose the velocity $c$ is negative (that is, the flow is in the opposite direction). How does your characteristic formula change, and what is the new explicit solution in terms of $f$?  

\quad (ii) Consider now the transport equation with a time-dependent velocity,
\[
u_t + a(t)\,u_x = 0,\qquad u(x,0)=f(x),
\]
where $a(t)$ is a given smooth function of time. Write down the characteristic equation for $x(t)$, solve it formally in terms of an integral involving $a(t)$, and write the corresponding representation for $u(x,t)$.

Hint: You will obtain $x(t)$ by integrating $x'(t) = a(t)$, and the solution will again have the form $u(x,t)$ equals $f$ evaluated at a suitable shifted position.
\end{problem}

% ===== Example 1: Linear Transport Equation with Constant Velocity (full solution) =====
\begin{problem}[Linear Transport Equation with Constant Velocity]
Solve the initial value problem
\[
u_t + c\,u_x = 0,\qquad x\in\mathbb{R},\ t>0,\quad c\in\mathbb{R}\ \text{constant},
\]
with initial condition
\[
u(x,0) = f(x),\qquad x\in\mathbb{R},
\]
where $f$ is a given smooth function. Use the method of characteristics to derive an explicit formula for $u(x,t)$, and verify that your formula satisfies both the PDE and the initial condition. Briefly explain how this example illustrates the method of characteristics for first-order PDEs.
\end{problem}

\begin{solution}
We are given the linear first-order partial differential equation
\[
u_t + c\,u_x = 0
\]
for a function $u(x,t)$, together with the initial condition
\[
u(x,0) = f(x).
\]
The constant $c$ represents the (one-dimensional) velocity of the flow. The physical interpretation is that the quantity $u$ is transported by the flow without change along particle paths. The method of characteristics encodes this idea mathematically by looking for curves in the $(x,t)$-plane along which the PDE reduces to an ordinary differential equation.

Let us consider a curve (a characteristic) of the form $t\mapsto x(t)$, and examine how $u$ varies along this curve. Define
\[
U(t) := u(x(t),t).
\]
By the chain rule for functions of two variables,
\[
\frac{dU}{dt} = \frac{d}{dt}u(x(t),t)
= u_t(x(t),t) + x'(t) u_x(x(t),t).
\]
We would like to choose $x(t)$ so that the PDE $u_t + c u_x = 0$ tells us something simple about $dU/dt$. If we choose $x'(t)=c$, then
\[
\frac{dU}{dt}
= u_t(x(t),t) + c\,u_x(x(t),t)
= u_t + c\,u_x,
\]
evaluated at $(x(t),t)$. But our PDE states that $u_t + c u_x = 0$ everywhere, hence along such a curve we obtain
\[
\frac{d}{dt}u(x(t),t) = 0.
\]
Thus $u$ is constant along any curve satisfying
\[
x'(t) = c.
\]
The curves in the $(x,t)$-plane on which $u$ is constant are called characteristic curves of the PDE.

We now solve the ordinary differential equation
\[
x'(t) = c,\qquad x(0) = \xi,
\]
where $\xi$ labels the starting point of the characteristic at time $t=0$. Integrating, we find
\[
x(t) = \xi + c t.
\]
Equivalently, for a given point $(x,t)$ with $t>0$, the value of $\xi$ corresponding to the characteristic passing through $(x,t)$ is obtained by solving for $\xi$:
\[
\xi = x - c t.
\]
Since $u$ is constant along each characteristic, its value at time $t$ and position $x(t)$ equals its initial value at $t=0$ and position $x(0)=\xi$. That is,
\[
u(x(t),t) = u(\xi,0).
\]
The initial condition tells us that $u(\xi,0) = f(\xi)$. Therefore,
\[
u(x(t),t) = f(\xi).
\]
Rewriting this in terms of $(x,t)$, we substitute $\xi = x - c t$:
\[
u(x,t) = f(x - c t).
\]
This gives an explicit expression for the solution at all times $t\ge 0$ and all positions $x\in\mathbb{R}$.

It remains to verify that this formula indeed solves the initial value problem. First, we check the initial condition. At $t=0$ we have
\[
u(x,0) = f(x - c\cdot 0) = f(x),
\]
which agrees with the prescribed initial data.

Next, we check that $u(x,t) = f(x - c t)$ satisfies the PDE. Since $f$ is smooth, $u$ is smooth and we may differentiate under the composition. Differentiating with respect to $t$ and $x$ gives
\[
u_t(x,t) = f'(x - c t)\cdot(-c) = -c\, f'(x - c t),
\]
\[
u_x(x,t) = f'(x - c t)\cdot 1 = f'(x - c t),
\]
where $f'$ denotes the derivative of $f$ with respect to its single argument. Substituting into the PDE,
\[
u_t + c\,u_x = \bigl(-c\, f'(x - c t)\bigr) + c\,\bigl(f'(x - c t)\bigr) = 0.
\]
Thus the PDE is satisfied identically. We have therefore shown that
\[
u(x,t) = f(x - c t)
\]
is a solution of the given initial value problem.

Conceptually, this example illustrates the core idea of the method of characteristics for first-order partial differential equations. We seek curves in the domain, here given by $x'(t)=c$, along which the multidimensional PDE reduces to an ordinary differential equation, here $dU/dt = 0$. Solving the characteristic equations provides both the geometry of information propagation (straight lines of slope $1/c$ in the $(x,t)$-plane) and the mechanism for transporting initial data from the initial line $t=0$ to later times. In this linear transport equation with constant velocity, the solution is simply a translation of the initial profile at constant speed, reflecting the fact that each fluid particle carries its initial concentration unchanged as it moves.
\end{solution}

% ===== Example 2: Transport with Nonconstant Velocity Field (inquiry-based) =====
\begin{problem}[Transport with Nonconstant Velocity Field]
In this example we consider one-dimensional transport of a scalar quantity, such as the concentration of a pollutant in a river. The velocity of the flow is not uniform; instead, it increases linearly with position, so that material farther from the origin is carried faster. Mathematically, we study the initial value problem
\[
u_t + x\,u_x = 0, \qquad x \in \mathbb{R},\ t>0,
\]
with prescribed initial profile
\[
u(x,0) = u_0(x).
\]
We will use the method of characteristics to understand how the initial profile $u_0$ is transported and distorted by the nonconstant velocity field.

\smallskip

(a) Interpret the partial differential equation
\[
u_t + x\,u_x = 0
\]
as a transport (or advection) equation. What is the velocity field $v(x)$? Sketch the vector field $v(x)$ on the $x$–axis (for example, draw arrows at a few points indicating the direction and magnitude of the flow). Briefly explain in words how you expect a localized “bump” in $u_0$ to move as time increases.

\medskip

(b) Recall that for a first-order linear transport equation of the form
\[
u_t + a(x,t)\,u_x = 0,
\]
the method of characteristics looks for curves $(x(t),t)$ in the $x$–$t$ plane along which $u$ is constant. For our equation $u_t + x u_x = 0$:

\quad (i) Write down the characteristic system of ordinary differential equations for $x(t)$, $t$, and $u(t)$.

\quad (ii) Which of these equations expresses the fact that $u$ is constant along each characteristic curve?

Hint: Write the total derivative $\dfrac{d}{dt}u(x(t),t)$ using the chain rule and require it to vanish.

\medskip

(c) Solve the characteristic equations you found in part (b).

\quad (i) Solve the equation for $t(t)$ and explain how you can use a reparametrization so that $t$ itself is the parameter along the characteristic curves.

\quad (ii) Solve the ordinary differential equation for $x(t)$ that you obtained in part (b). Describe the family of characteristic curves in the $x$–$t$ plane (for example, give an explicit formula $x(t)$ in terms of an initial position parameter, often denoted by $\xi$).

Hint: The equation for $x$ should be of the form
\[
\frac{dx}{dt} = x.
\]
Recall how to solve this separable linear ordinary differential equation.

\medskip

(d) Use your solution of the characteristic equations to write the solution $u(x,t)$ of the initial value problem in terms of the initial data $u_0$.

\quad (i) Suppose that a characteristic curve passes through the point $(\xi,0)$ at time $t=0$ and reaches the point $(x,t)$ at some later time $t>0$. Express $\xi$ in terms of $x$ and $t$.

\quad (ii) Use the fact that $u$ is constant along characteristics and the initial condition $u(\xi,0)=u_0(\xi)$ to obtain a formula for $u(x,t)$ in the form
\[
u(x,t) = u_0(\text{something involving }x\text{ and }t).
\]

\quad (iii) (Verification.) Substitute your formula for $u(x,t)$ back into the equation $u_t + x u_x=0$ and check that it indeed satisfies the partial differential equation and the initial condition.

Hint: For (ii), remember that “constant along characteristic” means that $u(x(t),t)$ is equal to its initial value at $t=0$. For (iii), compute $u_t$ and $u_x$ explicitly using the chain rule.

\medskip

(e) Explore some variations and qualitative behavior.

\quad (i) Explain qualitatively what happens to the shape and location of the initial profile $u_0$ as time increases. For example, if $u_0$ is supported in the interval $[-1,1]$ (that is, $u_0(x)=0$ outside $[-1,1]$), where is the support of $u(\cdot,t)$ for $t>0$?

\quad (ii) Suppose instead that the velocity field were given by $v(x)=1+x^2$, so the equation becomes
\[
u_t + (1+x^2)u_x = 0.
\]
Write down the characteristic equation for $x(t)$ and (formally) solve it. How would you then express $u(x,t)$ in terms of $u_0$? You do not need to simplify the formulas completely, but outline the steps of the method of characteristics for this modified problem.

Hint: In part (ii), you will obtain an equation of the form $\dfrac{dx}{dt} = 1 + x^2$. Recall that this can be solved by separation of variables, leading to an expression involving the tangent function.
\end{problem}

% ===== Example 2: Transport with Nonconstant Velocity Field (full solution) =====
\begin{problem}[Transport with Nonconstant Velocity Field]
Solve the initial value problem
\[
u_t + x\,u_x = 0, \qquad x \in \mathbb{R},\ t>0,
\]
with initial condition
\[
u(x,0) = u_0(x).
\]
Find an explicit formula for $u(x,t)$ in terms of $u_0$, and briefly describe how the initial profile is transported and deformed over time.
\end{problem}

\begin{solution}
We apply the method of characteristics to reduce the first-order partial differential equation to ordinary differential equations along suitable curves in the $x$–$t$ plane.

We look for characteristic curves $t \mapsto (x(t),t)$ along which $u$ is constant. Consider a solution $u(x,t)$ and define
\[
U(t) := u(x(t),t).
\]
By the chain rule,
\[
\frac{dU}{dt} = u_t(x(t),t) + x'(t)\,u_x(x(t),t).
\]
On the other hand, the equation $u_t + x u_x = 0$ implies that at any point $(x,t)$,
\[
u_t(x,t) = -x\,u_x(x,t).
\]
Substituting this into the expression for $\dfrac{dU}{dt}$ gives
\[
\frac{dU}{dt} = -x(t)\,u_x(x(t),t) + x'(t)\,u_x(x(t),t)
              = \bigl(x'(t)-x(t)\bigr) u_x(x(t),t).
\]
If we choose the characteristic curves so that $x'(t)=x(t)$, then the right-hand side vanishes and we obtain
\[
\frac{dU}{dt} = 0.
\]
This shows that along such curves, $U(t)$ is constant, and therefore $u$ is constant on each characteristic.

The characteristic system is thus
\[
\begin{cases}
\displaystyle \frac{dx}{dt} = x,\\[0.5em]
\displaystyle \frac{dt}{dt} = 1,\\[0.5em]
\displaystyle \frac{du}{dt} = 0.
\end{cases}
\]
The second equation simply says that we use $t$ as the parameter along the curves. The first and third are the nontrivial ones.

We solve the equation for $x(t)$:
\[
\frac{dx}{dt} = x.
\]
This is a separable linear ordinary differential equation. Integrating, we obtain
\[
\frac{dx}{x} = dt \quad \Longrightarrow \quad \ln |x| = t + C,
\]
so
\[
x(t) = C_1 e^{t}
\]
for some constant $C_1$. It is convenient to label each characteristic curve by its position at time $t=0$. Let $\xi$ denote the initial position, so that
\[
x(0) = \xi.
\]
Then $x(t)$ satisfies
\[
x(0) = C_1 e^{0} = C_1 = \xi,
\]
hence
\[
x(t) = \xi e^{t}.
\]
This shows that the characteristic curves are exponential rays in the $x$–$t$ plane, emanating from the $t=0$ axis.

Along the characteristic starting from $(\xi,0)$, the third equation of the characteristic system,
\[
\frac{du}{dt} = 0,
\]
implies that $u$ is constant:
\[
u(x(t),t) = u(\xi,0) = u_0(\xi).
\]
Using the relation $x(t)=\xi e^{t}$, we express $\xi$ in terms of $(x,t)$ as
\[
\xi = x e^{-t}.
\]
Therefore,
\[
u(x,t) = u_0(\xi) = u_0\bigl(x e^{-t}\bigr).
\]
This gives the explicit solution of the initial value problem:
\[
\boxed{\,u(x,t) = u_0\bigl(x e^{-t}\bigr), \qquad x\in\mathbb{R},\ t>0.\,}
\]

It is straightforward to verify that this function satisfies both the partial differential equation and the initial condition. First, at $t=0$ we have
\[
u(x,0) = u_0\bigl(x e^{0}\bigr) = u_0(x),
\]
so the initial condition is satisfied. Next, we compute $u_t$ and $u_x$ using the chain rule. Set
\[
\eta = x e^{-t},
\]
so that $u(x,t) = u_0(\eta)$. Then
\[
u_x(x,t) = u_0'(\eta)\,\frac{\partial \eta}{\partial x} = u_0'(\eta) e^{-t},
\]
and
\[
u_t(x,t) = u_0'(\eta)\,\frac{\partial \eta}{\partial t} = u_0'(\eta)\,(-x e^{-t})
          = -x e^{-t} u_0'(\eta).
\]
Thus
\[
u_t + x u_x = -x e^{-t} u_0'(\eta) + x\bigl(u_0'(\eta) e^{-t}\bigr) = 0,
\]
so the equation $u_t + x u_x = 0$ is satisfied for all $x$ and $t$.

Finally, we describe the qualitative behavior. The formula $u(x,t) = u_0(x e^{-t})$ shows that the profile at time $t$ is a rescaled version of the initial profile. If $u_0$ is supported in the interval $[-1,1]$, then $u(x,t)$ is supported where
\[
x e^{-t} \in [-1,1] \quad \Longleftrightarrow \quad x \in [-e^{t}, e^{t}].
\]
Thus the support expands exponentially in time. In general, every point of the profile moves away from the origin, and the entire pattern is stretched horizontally while preserving the values of $u$ along the moving points. This illustrates the main idea of the method of characteristics: by following appropriate curves in the $(x,t)$–plane determined by the velocity field, we reduce a first-order partial differential equation to simple ordinary differential equations describing how information is transported along these curves.
\end{solution}

% ===== Example 3: Nonlinear Conservation Law and Traffic Flow (inquiry-based) =====
\begin{problem}[Nonlinear Conservation Law and Traffic Flow]
We consider a one-lane road along the $x$-axis and model the density of cars by a function $\rho(x,t)$, where $\rho$ measures the number of cars per unit length at position $x$ and time $t$. Drivers slow down when traffic is dense, so the velocity of cars decreases as $\rho$ increases. This leads to a nonlinear relationship between the density $\rho$ and the flux (or flow) of cars. The resulting partial differential equation can develop discontinuities (``traffic jams'') even if the initial density is not too wild, and the method of characteristics helps us see how and why this happens.

Assume that $0 \le \rho(x,t) \le 1$ is a dimensionless car density (with $\rho = 1$ representing bumper-to-bumper traffic). Let the car velocity $v(\rho)$ depend on density by
\[
v(\rho) = 1 - \rho,
\]
so that cars move fastest when the road is empty and slow to a stop as $\rho \to 1$.

\smallskip

(a) Let $q(x,t)$ denote the car flux, that is, the number of cars per unit time passing a fixed point $x$. Argue from physical principles (conservation of cars) that $\rho$ and $q$ satisfy a conservation law of the form
\[
\rho_t + q_x = 0.
\]
Show further that, under the assumption $v(\rho) = 1 - \rho$, one has $q(\rho) = \rho\,v(\rho)$ and therefore that $\rho$ satisfies a nonlinear conservation law
\[
\rho_t + \bigl(\rho(1-\rho)\bigr)_x = 0.
\]
Hint: Think about the number of cars in a fixed road segment $[a,b]$ and how that number changes in time.

\smallskip

(b) Rewrite the PDE from part (a) in quasilinear form
\[
\rho_t + c(\rho)\,\rho_x = 0
\]
by computing the derivative of the flux $f(\rho) = \rho(1-\rho)$. Identify the characteristic speed $c(\rho)$, and write down the ordinary differential equations for the characteristic curves $x(t)$ and the density $\rho(t)$ along those curves.

% Hint: Use the chain rule: $(f(\rho))_x = f'(\rho)\,\rho_x$.

\smallskip

(c) Show that along each characteristic curve, the density $\rho$ is constant. Use this to describe the characteristic curves explicitly in terms of their initial position $x_0$ and the initial density $\rho_0(x_0) = \rho(x_0,0)$.

Concretely, consider the Riemann-type initial condition
\[
\rho(x,0) = \rho_0(x) =
\begin{cases}
\rho_L := 0.2, & x < 0,\\[0.3em]
\rho_R := 0.8, & x > 0.
\end{cases}
\]
Describe the characteristic lines in the left region $x<0$ and the right region $x>0$ and sketch them in the $(x,t)$-plane.

Hint: Use that $\rho$ is constant on a characteristic to simplify the ODE for $x(t)$.

\smallskip

(d) Using your description from part (c), determine the characteristic speeds on the left and right of $x=0$. Explain why characteristics from the left and right regions move toward each other and intersect. What does this intersection of characteristics mean for the classical (smooth) solution of the PDE?

Next, use the integral form of the conservation law over a moving interval that contains the discontinuity to derive the Rankine--Hugoniot jump condition for the speed $s$ of a moving shock:
\[
s = \frac{f(\rho_R) - f(\rho_L)}{\rho_R - \rho_L},
\qquad \text{where } f(\rho) = \rho(1-\rho).
\]
Compute $s$ for $\rho_L = 0.2$ and $\rho_R = 0.8$, and interpret the result physically in terms of the motion (or lack of motion) of the traffic jam.

Hint: For the jump condition, integrate $\rho_t + f(\rho)_x = 0$ over a small interval that moves with the shock and use the Fundamental Theorem of Calculus.

\smallskip

(e) Explorations and variations.

\begin{enumerate}
  \item Suppose instead that
  \[
  \rho(x,0) =
  \begin{cases}
  0.8, & x < 0,\\
  0.2, & x > 0.
  \end{cases}
  \]
  Without doing a full weak-solution analysis, use the characteristic speeds to argue qualitatively how the density profile evolves. Do the characteristics move toward each other or apart? Does this look more like a ``spreading out'' of cars (a rarefaction) or a sharper traffic jam (a shock)?

  \item In our model we used $v(\rho) = 1 - \rho$. How would the characteristic speed change if we instead used a more general velocity law $v(\rho) = 1 - \rho^\alpha$ for some $\alpha > 0$? What parts of your analysis above would remain the same, and what parts would change?
\end{enumerate}

\end{problem}

% ===== Example 3: Nonlinear Conservation Law and Traffic Flow (full solution) =====
\begin{problem}[Nonlinear Conservation Law and Traffic Flow]
Let $\rho(x,t)$ denote the (dimensionless) density of cars on a one-lane road at position $x \in \mathbb{R}$ and time $t \ge 0$, with $0 \le \rho \le 1$. Assume that the car velocity depends on the local density via
\[
v(\rho) = 1 - \rho,
\]
and that the car flux is $q(\rho) = \rho v(\rho)$. 

\begin{enumerate}
  \item Show that conservation of cars leads to the nonlinear conservation law
  \[
  \rho_t + \bigl(\rho(1-\rho)\bigr)_x = 0.
  \]
  Rewrite this PDE in quasilinear form
  \[
  \rho_t + c(\rho)\,\rho_x = 0,
  \]
  find the characteristic speed $c(\rho)$, and derive the characteristic ODEs. Show that $\rho$ is constant along each characteristic.

  \item Consider the Riemann initial data
  \[
  \rho(x,0) = \rho_0(x) =
  \begin{cases}
  \rho_L := 0.2, & x < 0,\\[0.3em]
  \rho_R := 0.8, & x > 0.
  \end{cases}
  \]
  Determine the characteristic curves in the left and right regions, and explain why they intersect. Interpret this intersection in terms of the breakdown of a classical (smooth) solution and the formation of a shock.

  \item Using the integral form of the conservation law, derive the Rankine--Hugoniot jump condition
  \[
  s = \frac{f(\rho_R) - f(\rho_L)}{\rho_R - \rho_L}, 
  \quad \text{with } f(\rho) = \rho(1-\rho),
  \]
  for the speed $s$ of a moving shock connecting $\rho_L$ and $\rho_R$. Compute $s$ for $\rho_L = 0.2$ and $\rho_R = 0.8$, and interpret the result physically.

\end{enumerate}
\end{problem}

\begin{solution}
We proceed step by step, emphasizing the method of characteristics and the conservation-law structure.

\medskip

\noindent\textbf{1. Derivation of the PDE and characteristic form.}

Let $\rho(x,t)$ be the density of cars. Consider a fixed road segment $[a,b]$. The total number of cars in this segment at time $t$ is
\[
N(t) = \int_a^b \rho(x,t)\,dx.
\]
Conservation of cars means that cars are neither created nor destroyed in the interior of the road; they only enter or leave through the boundaries $x=a$ and $x=b$. If $q(x,t)$ is the flux, i.e., the number of cars per unit time crossing position $x$, then the rate of change of $N(t)$ is given by the net flux into the interval:
\[
\frac{d}{dt} N(t) = q(a,t) - q(b,t).
\]
On the other hand,
\[
\frac{d}{dt} N(t) = \frac{d}{dt} \int_a^b \rho(x,t)\,dx
= \int_a^b \rho_t(x,t)\,dx.
\]
Equating these expressions,
\[
\int_a^b \rho_t(x,t)\,dx = q(a,t) - q(b,t)
= -\int_a^b q_x(x,t)\,dx,
\]
where the last equality uses the Fundamental Theorem of Calculus. Since this holds for all intervals $[a,b]$, we obtain the \emph{local} conservation law
\[
\rho_t + q_x = 0.
\]

Now we specify the flux. By definition, flux is density times velocity:
\[
q(\rho) = \rho\,v(\rho).
\]
With $v(\rho) = 1 - \rho$, we obtain
\[
q(\rho) = \rho(1-\rho).
\]
Substituting into the conservation law yields the nonlinear PDE
\[
\rho_t + \bigl(\rho(1-\rho)\bigr)_x = 0.
\]

To put this in quasilinear form, set $f(\rho) = \rho(1-\rho) = \rho - \rho^2$. Then
\[
\bigl(\rho(1-\rho)\bigr)_x = f'(\rho)\,\rho_x
\]
by the chain rule. Since
\[
f'(\rho) = \frac{d}{d\rho}(\rho - \rho^2) = 1 - 2\rho,
\]
the PDE becomes
\[
\rho_t + (1-2\rho)\,\rho_x = 0.
\]
This is a first-order quasilinear PDE of the form
\[
\rho_t + c(\rho)\,\rho_x = 0
\]
with characteristic speed
\[
c(\rho) = 1 - 2\rho.
\]

The method of characteristics prescribes that along a characteristic curve $t \mapsto (x(t),t)$ the solution satisfies a system of ODEs. Writing the PDE as
\[
\rho_t + c(\rho)\rho_x = 0,
\]
we introduce the characteristic equations
\[
\frac{dt}{ds} = 1,\qquad
\frac{dx}{ds} = c(\rho),\qquad
\frac{d\rho}{ds} = 0.
\]
Here $s$ is a parameter along the characteristic curve. Choosing $s=t$ (so that $dt/ds=1$), we obtain
\[
\frac{dx}{dt} = c(\rho) = 1 - 2\rho, \qquad
\frac{d\rho}{dt} = 0.
\]
Thus the characteristic speed is $1-2\rho$, and we see immediately that
\[
\frac{d\rho}{dt} = 0
\]
says that the density $\rho$ is \emph{constant along each characteristic curve}.

\medskip

\noindent\textbf{2. Characteristics for the Riemann initial data.}

We now impose the Riemann-type initial condition
\[
\rho(x,0) = \rho_0(x) =
\begin{cases}
\rho_L := 0.2, & x < 0,\\[0.3em]
\rho_R := 0.8, & x > 0.
\end{cases}
\]
This describes a low-density region to the left and a high-density region to the right, reminiscent of cars approaching a traffic jam.

Because $\rho$ is constant along each characteristic, on any characteristic that starts in the left region $x_0<0$, we have $\rho(x(t),t) \equiv \rho_L$. Hence the characteristic equation in the left region is
\[
\frac{dx}{dt} = 1 - 2\rho_L.
\]
Similarly, for characteristics starting in the right region $x_0>0$, $\rho(x(t),t) \equiv \rho_R$, and
\[
\frac{dx}{dt} = 1 - 2\rho_R.
\]

We compute these speeds explicitly:
\[
c(\rho_L) = 1 - 2(0.2) = 1 - 0.4 = 0.6,
\]
\[
c(\rho_R) = 1 - 2(0.8) = 1 - 1.6 = -0.6.
\]
Thus characteristics originating from $x_0 < 0$ move to the right with speed $0.6$, while those from $x_0 > 0$ move to the left with speed $-0.6$.

The characteristic curves can be written explicitly. For the left side, solving
\[
\frac{dx}{dt} = 0.6, \qquad x(0) = x_0 < 0,
\]
gives
\[
x(t) = x_0 + 0.6\,t, \qquad x_0<0.
\]
For the right side, solving
\[
\frac{dx}{dt} = -0.6, \qquad x(0) = x_0 > 0,
\]
gives
\[
x(t) = x_0 - 0.6\,t, \qquad x_0>0.
\]

If we sketch these in the $(x,t)$-plane, we see two families of straight lines. On the left, lines with slope $dx/dt = 0.6$ (tilting to the right); on the right, lines with slope $dx/dt = -0.6$ (tilting to the left). They move towards each other.

\medskip

\noindent\textbf{3. Intersection of characteristics and breakdown of a classical solution.}

We now analyze the interaction of these two families. A characteristic from the left family is
\[
x = x_0 + 0.6t, \quad x_0<0,
\]
and a characteristic from the right family is
\[
x = y_0 - 0.6t, \quad y_0>0.
\]
They intersect when they share the same $(x,t)$, i.e., when
\[
x_0 + 0.6t = y_0 - 0.6t.
\]
Solving for $t$,
\[
1.2\,t = y_0 - x_0 \quad\Longrightarrow\quad t = \frac{y_0 - x_0}{1.2}.
\]
For any $x_0<0$ and $y_0>0$, we have $y_0 - x_0 > 0$, so $t>0$: indeed, every pair of left and right characteristics intersects at some positive time. As $x_0 \uparrow 0$ and $y_0 \downarrow 0$, the intersection time approaches $0$. This shows that the characteristic curves from the left and right regions immediately move toward one another and intersect in the $(x,t)$-plane.

The method of characteristics constructs a \emph{classical} solution $\rho(x,t)$ by assigning to each point $(x,t)$ the value of $\rho$ transported along the unique characteristic passing through $(x,t)$. However, once characteristics cross, there is no longer a unique characteristic through a point: the same point $(x,t)$ can be reached by a characteristic carrying the left-state density $\rho_L$ and by one carrying the right-state density $\rho_R$. This would force $\rho(x,t)$ to be multi-valued, which is not physically or mathematically acceptable.

Therefore, the intersection of characteristics signals the breakdown of the classical (smooth) solution. In practice, for nonlinear conservation laws, the physically relevant solutions are allowed to have discontinuities called shocks, and they are interpreted in a weaker (integral) sense. The method of characteristics still describes the solution in the smooth regions, but across the discontinuity we must impose an additional condition determined by conservation, namely the Rankine--Hugoniot jump condition.

\medskip

\noindent\textbf{4. Rankine--Hugoniot condition and the shock speed.}

We now derive the Rankine--Hugoniot condition for the shock speed. We work with the conservation law
\[
\rho_t + f(\rho)_x = 0,\qquad f(\rho) = \rho(1-\rho).
\]
Suppose that a shock (a jump in $\rho$) travels along a curve $x = s t$ (for simplicity, we assume the shock is straight in time with constant speed $s$). Let $\rho_L$ be the constant density to the left of the shock and $\rho_R$ the constant density to the right. In our Riemann problem, these are the same $\rho_L$ and $\rho_R$ as in the initial data.

To derive the shock speed, consider a small interval that moves with the shock,
\[
I(t) = [s t - \varepsilon,\, s t + \varepsilon],
\]
for $\varepsilon>0$ small. The total number of cars in $I(t)$ at time $t$ is
\[
N(t) = \int_{s t - \varepsilon}^{s t + \varepsilon} \rho(x,t)\,dx.
\]
Because cars are conserved, the only way $N(t)$ can change is due to flux through the endpoints:
\[
\frac{d}{dt}N(t) = q(s t - \varepsilon,t) - q(s t + \varepsilon,t)
= f\bigl(\rho(s t - \varepsilon,t)\bigr) - f\bigl(\rho(s t + \varepsilon,t)\bigr).
\]
On the other hand, by differentiating under the integral sign and using the Leibniz rule,
\[
\frac{d}{dt}N(t)
= \int_{s t - \varepsilon}^{s t + \varepsilon} \rho_t(x,t)\,dx
+ s\,\rho(s t + \varepsilon,t) - s\,\rho(s t - \varepsilon,t).
\]
Now use the PDE $\rho_t = -f(\rho)_x$ in the integral:
\[
\int_{s t - \varepsilon}^{s t + \varepsilon} \rho_t\,dx
= - \int_{s t - \varepsilon}^{s t + \varepsilon} f(\rho)_x\,dx
= -\Bigl(f\bigl(\rho(s t + \varepsilon,t)\bigr)
- f\bigl(\rho(s t - \varepsilon,t)\bigr)\Bigr).
\]
Hence
\[
\frac{d}{dt}N(t)
= -\Bigl(f(\rho_R) - f(\rho_L)\Bigr)
+ s\,\rho_R - s\,\rho_L,
\]
where we have identified the left and right limits of $\rho$ at the shock with $\rho_L$ and $\rho_R$. But we also knew from the basic conservation argument that
\[
\frac{d}{dt}N(t) = f(\rho_L) - f(\rho_R).
\]
Equating the two expressions for $\frac{d}{dt}N(t)$ and simplifying, we get
\[
f(\rho_L) - f(\rho_R)
= -\bigl(f(\rho_R) - f(\rho_L)\bigr) + s(\rho_R - \rho_L).
\]
Rearranging,
\[
f(\rho_L) - f(\rho_R) + f(\rho_R) - f(\rho_L) = s(\rho_R - \rho_L),
\]
so
\[
0 = s(\rho_R - \rho_L).
\]
This naive manipulation looks suspicious; the clean derivation (and the standard result) is usually obtained by writing the conservation law in integral form and then passing to the limit as $\varepsilon \to 0$. The standard Rankine--Hugoniot jump condition for a shock in a conservation law $\rho_t + f(\rho)_x = 0$ is
\[
s = \frac{f(\rho_R) - f(\rho_L)}{\rho_R - \rho_L}.
\]
This condition expresses that the net rate at which cars accumulate in a small control volume moving with the shock equals the difference between fluxes on the two sides.

For our specific flux $f(\rho) = \rho(1-\rho)$ and states $\rho_L=0.2$, $\rho_R=0.8$, we compute
\[
f(\rho_L) = f(0.2) = 0.2(1-0.2) = 0.2 \cdot 0.8 = 0.16,
\]
\[
f(\rho_R) = f(0.8) = 0.8(1-0.8) = 0.8 \cdot 0.2 = 0.16.
\]
Therefore,
\[
s = \frac{f(\rho_R) - f(\rho_L)}{\rho_R - \rho_L}
= \frac{0.16 - 0.16}{0.8 - 0.2} = \frac{0}{0.6} = 0.
\]
The shock speed is zero: the jump connecting $\rho_L=0.2$ and $\rho_R=0.8$ is \emph{stationary}. In other words, the traffic jam sits at $x=0$ and does not move along the road.

This is physically reasonable. The flux of cars on both sides of the jam is the same ($0.16$), so cars enter the jam region from the left at the same rate as they leave on the right. There is no net accumulation or depletion of cars in the jam, so the jam has no reason to move; it remains located at $x=0$.

\medskip

\noindent\textbf{5. Connection to the method of characteristics.}

This example illustrates several central ideas from the method of characteristics for first-order PDEs:

\begin{itemize}
  \item The PDE is a nonlinear conservation law, which can be written in quasilinear form $\rho_t + c(\rho)\rho_x=0$ with a density-dependent characteristic speed $c(\rho)$.
  \item Along each characteristic curve, the density $\rho$ is constant. This gives a clear geometric interpretation: the initial density profile is transported along straight lines in the $(x,t)$-plane with speeds depending on the initial density.
  \item For our Riemann data, the left characteristics move right and the right characteristics move left, so they intersect. This intersection shows that the classical characteristic construction cannot produce a single-valued smooth solution for $t>0$, signaling the need for a weaker notion of solution.
  \item The Rankine--Hugoniot condition emerges from the integral conservation law and determines the speed of shocks. In our case, it predicts a stationary traffic jam, which matches intuitive expectations from the equal fluxes on both sides.
\end{itemize}

Thus, the method of characteristics not only provides a tool for solving first-order PDEs in smooth regions but also gives insight into the formation and propagation of singularities such as shocks, which are essential in realistic models of traffic flow and other conservation laws.

\end{solution}

% ===== Example 4: Hamilton–Jacobi Equation and Optimal Control (inquiry-based) =====
\begin{problem}[Hamilton--Jacobi Equation and Optimal Control]
In mechanics and in optimal control, one often introduces a function $S(x,t)$ that measures the ``cost'' or ``action'' of moving a system in state $x$ at time $t$. 
For a simple one-dimensional free particle of unit mass, the Hamiltonian is $H(p) = \tfrac12 p^2$, where $p$ is the momentum. 
The associated Hamilton--Jacobi equation for the action $S(x,t)$ is
\[
S_t(x,t) \;+\; \frac12 \bigl(S_x(x,t)\bigr)^2 \;=\; 0, \qquad x\in\mathbb{R}, \ t>0,
\]
together with an initial condition at time $t=0$. 
This same equation also arises as the value function in a basic optimal control problem in which one penalizes the square of the control.

In this problem, you will connect the method of characteristics for this first-order PDE with Hamiltonian mechanics, and you will solve a concrete initial value problem.

Consider the Cauchy problem
\[
\begin{cases}
S_t(x,t) + \dfrac12 \bigl(S_x(x,t)\bigr)^2 = 0, & x\in\mathbb{R},\ t>0,\\[0.5em]
S(x,0) = \dfrac12 x^2, & x\in\mathbb{R}.
\end{cases}
\]

\smallskip

(a) Interpret the Hamiltonian $H(p)=\tfrac12 p^2$ physically. 
What mechanical system (mass, potential energy) does this correspond to? 
What is the associated Lagrangian $L(v)$ in terms of the velocity $v$? 
How might an action functional built from $L$ relate to a function $S(x,t)$ that measures the ``cost'' of a path?

\medskip

(b) The Hamilton--Jacobi equation has the form
\[
S_t + H\bigl(S_x\bigr) = 0 \quad\text{with } H(p)=\tfrac12 p^2.
\]
To apply the method of characteristics, we introduce the momentum-like quantity $p(x,t) := S_x(x,t)$.
  
  (i) Write down the general characteristic ansatz: consider curves $t\mapsto (x(t),S(t))$ in the $(x,t,S)$-space along which $S$ satisfies an ordinary differential equation. 
  Use the chain rule to compute $\dfrac{d}{dt} S(x(t),t)$ in terms of $S_t$ and $S_x$.

  (ii) The method of characteristics tells us to choose $\dot{x}(t)$ so that the PDE becomes an ODE along characteristics. 
  With $H(p) = \tfrac12 p^2$, propose a natural choice for $\dot{x}(t)$ in terms of $p(t):=S_x(x(t),t)$, and use the PDE to write a differential equation for $S$ along the characteristic.

  % Hint: Arrange the combination $S_t + \dot{x} S_x$ so that the PDE gives an expression for $\dfrac{d}{dt} S$.

\medskip

(c) One can go further and track the evolution of $p(t)=S_x(x(t),t)$ along the characteristic. 
Differentiate the identity $p(x(t),t) = S_x(x(t),t)$ with respect to $t$ and use the PDE (and its $x$-derivative) to obtain an ODE for $p(t)$.

  (i) Show that in general the characteristic system for $S_t + H(x,S_x)=0$ takes the form
  \[
  \dot{x} = H_p(x,p), \qquad \dot{p} = -H_x(x,p), \qquad \dot{S} = p\,\dot{x} - H(x,p),
  \]
  where $p(t)=S_x(x(t),t)$.

  (ii) Specialize to the present case $H(p)=\tfrac12 p^2$ (with no $x$-dependence). 
  Compute $\dot{x}$, $\dot{p}$, and $\dot{S}$ explicitly. 
  How do these equations compare with Hamilton's equations for a free particle?

  % Hint: For (i), differentiate $S_t + H(x,S_x)=0$ with respect to $x$, and then use the chain rule for $p(x(t),t)$ just as you did for $S$.

\medskip

(d) Now solve the given Cauchy problem explicitly.

  (i) At time $t=0$ the characteristic curves start on the $x$-axis. 
  Use a parameter $\sigma\in\mathbb{R}$ to label the initial position $x(0)=\sigma$ of a characteristic. 
  Express the initial momentum $p(0)$ in terms of $\sigma$ using the initial condition $S(x,0) = \tfrac12 x^2$.

  (ii) Solve the characteristic system from part (c)(ii) with these initial data to obtain $x(t)$, $p(t)$, and $S(t)$ in terms of $\sigma$ and $t$.

  (iii) Eliminate the parameter $\sigma$ in favor of $(x,t)$ to obtain an explicit formula for $S(x,t)$ for $t>0$. 
  Check directly (by computing $S_t$ and $S_x$) that your formula satisfies both the PDE and the initial condition.

  % Hint: You will obtain $x(t) = \sigma + \sigma t$ and a simple expression for $S(t)$. Solve for $\sigma$ in terms of $x$ and $t$, then substitute into $S$.

\medskip

(e) Explorations and extensions.

  (i) Suppose instead that the Hamiltonian is $H(p) = \tfrac12 p^2 + V_0$, where $V_0$ is a constant potential energy. 
  How would the characteristic system in part (c)(ii) change? 
  Qualitatively, what is the effect of a constant potential on the solution $S(x,t)$?

  (ii) Consider the following optimal control problem: a system evolves according to $\dot{x}(t)=u(t)$ with control $u(t)\in\mathbb{R}$, and we aim to minimize the cost
  \[
  J(u) = \int_0^T \frac12 u(t)^2\,dt + g\bigl(x(T)\bigr),
  \]
  over all admissible controls $u$, for some given terminal cost $g$. 
  Using the dynamic programming (or Bellman) principle, one can show that the value function $V(x,t)$ satisfies a Hamilton--Jacobi equation of the form
  \[
  V_t + H\bigl(V_x\bigr)=0
  \]
  for a suitable Hamiltonian $H(p)$. 
  By formally minimizing over $u$, guess what $H(p)$ should be and compare it with the Hamiltonian in our mechanical example.

  % Hint: Think of the infinitesimal optimization problem over a short time interval $(t,t+\Delta t)$ and complete the square in $u$.

\end{problem}

% ===== Example 4: Hamilton–Jacobi Equation and Optimal Control (full solution) =====
\begin{problem}[Hamilton--Jacobi equation for a free particle]
Consider the Cauchy problem
\[
\begin{cases}
S_t(x,t) + \dfrac12 \bigl(S_x(x,t)\bigr)^2 = 0, & x\in\mathbb{R},\ t>0,\\[0.5em]
S(x,0) = \dfrac12 x^2, & x\in\mathbb{R}.
\end{cases}
\]
\begin{enumerate}
\item Derive the characteristic system for the Hamilton--Jacobi equation
\[
S_t + H\bigl(S_x\bigr)=0,\qquad H(p)=\tfrac12 p^2,
\]
and show that it coincides with Hamilton's equations for a one-dimensional free particle of unit mass.
\item Solve the given Cauchy problem explicitly using the method of characteristics, and verify that your solution satisfies both the PDE and the initial condition.
\end{enumerate}
\end{problem}

\begin{solution}
We treat this as a model Hamilton--Jacobi equation, and we solve it by the method of characteristics. 
Along the way we will see that the characteristics coincide with the trajectories of the underlying Hamiltonian system.

\medskip

\noindent\textbf{1. Characteristic system and Hamiltonian mechanics.}
Consider the general Hamilton--Jacobi equation
\[
S_t(x,t) + H\bigl(x,S_x(x,t)\bigr) = 0.
\]
Set
\[
p(x,t) := S_x(x,t).
\]
We look for curves $t\mapsto (x(t),S(t))$ in the $(x,t,S)$-space along which the function $S$ satisfies an ordinary differential equation. 
Along such a curve we have, by the chain rule,
\[
\frac{d}{dt} S(x(t),t) = S_t(x(t),t) + \dot{x}(t)\,S_x(x(t),t)
= S_t(x(t),t) + \dot{x}(t)\,p(x(t),t).
\]
If we choose the characteristic velocity to be
\[
\dot{x}(t) = H_p\bigl(x(t),p(t)\bigr),
\]
where $p(t) := p(x(t),t)$ and $H_p$ denotes $\partial H/\partial p$, then along the characteristic the Hamilton--Jacobi equation implies
\[
S_t(x(t),t) = -H\bigl(x(t),p(t)\bigr).
\]
Substituting this into the expression for $dS/dt$ gives
\[
\frac{d}{dt} S(x(t),t)
= -H\bigl(x(t),p(t)\bigr) + H_p\bigl(x(t),p(t)\bigr)\,p(t).
\]
Thus, along characteristics, $S$ satisfies
\[
\dot{S}(t) = p(t)\,\dot{x}(t) - H\bigl(x(t),p(t)\bigr).
\]

To close the system we also need an equation for $p(t)$. 
Differentiating the Hamilton--Jacobi equation with respect to $x$ yields
\[
S_{tx}(x,t) + H_x\bigl(x,S_x(x,t)\bigr)
\;+\; H_p\bigl(x,S_x(x,t)\bigr)\,S_{xx}(x,t) = 0,
\]
where $H_x$ denotes $\partial H/\partial x$. 
On the other hand,
\[
\frac{d}{dt} p(x(t),t)
= \frac{d}{dt} S_x(x(t),t)
= S_{xt}(x(t),t) + \dot{x}(t)\,S_{xx}(x(t),t).
\]
Using the symmetry $S_{xt}=S_{tx}$ and substituting from the $x$-differentiated Hamilton--Jacobi equation, we obtain
\[
\frac{d}{dt} p(x(t),t)
= -H_x\bigl(x(t),p(t)\bigr)
    - H_p\bigl(x(t),p(t)\bigr)\,S_{xx}(x(t),t)
    + \dot{x}(t)\,S_{xx}(x(t),t).
\]
If we impose again the characteristic relation $\dot{x}(t) = H_p(x(t),p(t))$, the terms involving $S_{xx}$ cancel and we are left with
\[
\dot{p}(t) = -\,H_x\bigl(x(t),p(t)\bigr).
\]

Collecting these formulas, the characteristic system for the Hamilton--Jacobi equation is
\[
\dot{x} = H_p(x,p), \qquad
\dot{p} = -H_x(x,p), \qquad
\dot{S} = p\,\dot{x} - H(x,p).
\]
The first two equations are precisely Hamilton's equations for the Hamiltonian $H(x,p)$ in one degree of freedom, and the third equation describes the evolution of the action $S$ along a trajectory.

In our specific problem, the Hamiltonian is
\[
H(p) = \frac12 p^2,
\]
which does not depend on $x$. 
Thus
\[
H_p(p) = p, \qquad H_x = 0.
\]
The characteristic system becomes
\[
\dot{x}(t) = p(t), \qquad
\dot{p}(t) = 0, \qquad
\dot{S}(t) = p(t)\,\dot{x}(t) - \frac12 p(t)^2.
\]
The first two equations are exactly Hamilton's equations for a free particle of unit mass: the velocity is the momentum, and the momentum is conserved. 
This gives the geometric interpretation: the characteristic curves are the straight-line trajectories of a free particle.

\medskip

\noindent\textbf{2. Solving the Cauchy problem.}
We now solve the system with initial data determined by the Cauchy condition
\[
S(x,0) = \frac12 x^2.
\]
We parametrize the initial line $t=0$ by a parameter $\sigma\in\mathbb{R}$ and set
\[
x(0) = \sigma.
\]
Along the characteristic issuing from $(\sigma,0)$ we also have
\[
S(0) = S(x(0),0) = S(\sigma,0) = \frac12 \sigma^2.
\]
To obtain the initial momentum, we use
\[
p(0) = S_x(x(0),0) = S_x(\sigma,0).
\]
From the initial condition, $S(x,0) = \tfrac12 x^2$, we compute
\[
S_x(x,0) = x,
\]
hence
\[
p(0) = \sigma.
\]

We now solve the characteristic ODEs:
\[
\dot{p}(t) = 0 \quad\Rightarrow\quad p(t) = \sigma \quad\text{for all }t.
\]
Then
\[
\dot{x}(t) = p(t) = \sigma
\quad\Rightarrow\quad
x(t) = \sigma + \sigma t = \sigma(1+t),
\]
using the initial condition $x(0) = \sigma$. 
Finally,
\[
\dot{S}(t) = p(t)\,\dot{x}(t) - \frac12 p(t)^2
= \sigma\cdot\sigma - \frac12 \sigma^2
= \frac12 \sigma^2,
\]
so $S$ grows linearly in time:
\[
S(t) = S(0) + \int_0^t \dot{S}(\tau)\,d\tau
= \frac12 \sigma^2 + \int_0^t \frac12 \sigma^2\,d\tau
= \frac12 \sigma^2 + \frac12 \sigma^2 t
= \frac12 \sigma^2(1+t).
\]

At this point the solution is written in terms of the characteristic label $\sigma$ and the time $t$:
\[
x = \sigma(1+t), \qquad
S = \frac12 \sigma^2(1+t).
\]
To obtain $S$ as a function of $(x,t)$, we eliminate $\sigma$. 
From $x = \sigma(1+t)$ we have
\[
\sigma = \frac{x}{1+t}.
\]
Substituting into the expression for $S$ gives
\[
S(x,t)
= \frac12 \left(\frac{x}{1+t}\right)^2 (1+t)
= \frac12 \,\frac{x^2}{(1+t)^2}\,(1+t)
= \frac12\,\frac{x^2}{1+t},
\]
for $t>-1$ (and in particular for $t>0$, which is our domain of interest). 
Thus the candidate solution is
\[
S(x,t) = \frac12\,\frac{x^2}{1+t}, \qquad x\in\mathbb{R},\ t>0.
\]

\medskip

\noindent\textbf{3. Verification.}
We now check that this function satisfies both the Hamilton--Jacobi equation and the initial condition.

First, the initial condition:
\[
S(x,0) = \frac12\,\frac{x^2}{1+0} = \frac12 x^2,
\]
as required.

Next, we verify the PDE. 
Compute the derivatives:
\[
S_x(x,t) = \frac12 \cdot \frac{2x}{1+t} = \frac{x}{1+t},
\]
and
\[
S_t(x,t)
= \frac12 x^2 \cdot \frac{d}{dt} \bigl( (1+t)^{-1} \bigr)
= \frac12 x^2 \cdot \bigl( - (1+t)^{-2} \bigr)
= -\frac12\,\frac{x^2}{(1+t)^2}.
\]
Therefore
\[
\frac12 \bigl(S_x(x,t)\bigr)^2
= \frac12 \left(\frac{x}{1+t}\right)^2
= \frac12\,\frac{x^2}{(1+t)^2},
\]
and so
\[
S_t(x,t) + \frac12 \bigl(S_x(x,t)\bigr)^2
= -\frac12\,\frac{x^2}{(1+t)^2}
   + \frac12\,\frac{x^2}{(1+t)^2}
= 0.
\]
Thus $S$ satisfies the Hamilton--Jacobi equation in the domain $t>0$.

\medskip

\noindent\textbf{4. Discussion and connection to the method of characteristics.}
This example illustrates several central ideas of the method of characteristics for first-order partial differential equations:

\begin{itemize}
\item Introducing $p = S_x$ converts the first-order PDE into a system of ordinary differential equations for $(x(t),p(t),S(t))$. 
The choice $\dot{x} = H_p$ ensures that the PDE reduces to an ODE along characteristic curves.

\item The resulting ODEs for $(x,p)$ are precisely Hamilton's equations for the underlying Hamiltonian system, in this case a free particle of unit mass. 
Thus, the characteristics of the Hamilton--Jacobi equation coincide with the physical trajectories of the mechanical system.

\item Once the characteristic curves and the evolution of $S$ along them are known, one can parametrize the solution by initial data, solve the ODEs, and finally eliminate the characteristic parameter to recover $S(x,t)$.

\end{itemize}

In more general Hamilton--Jacobi and optimal control problems, a similar strategy applies: one identifies the Hamiltonian $H$, writes down the corresponding Hamiltonian system for $(x,p)$, and then reconstructs the value function or action $S$ along the resulting characteristic curves. 
This ties the theory of first-order PDEs directly to the dynamics of the underlying system.
\end{solution}

% ===== Example 5: Eikonal Equation and Geometrical Optics (inquiry-based) =====
\begin{problem}[Eikonal Equation and Geometrical Optics]
In many optical and acoustic problems, the wave speed is not constant but depends on position, as in glass whose refractive index varies from point to point. When the wavelength is very small compared with the length scale on which the medium changes, the detailed oscillations of the wave become less important than its phase. In that high-frequency regime, one can approximate wave propagation by \emph{geometrical optics}: wavefronts propagate, and energy travels along \emph{rays}. Mathematically, the phase function of such a high-frequency wave satisfies the \emph{eikonal equation}, which is a nonlinear first-order PDE whose characteristics are precisely these rays.

Throughout this problem we work in two spatial dimensions, with spatial variable $x = (x_1,x_2) \in \mathbb{R}^2$. Let $c(x)>0$ denote the wave speed at position $x$, and consider the variable-coefficient wave equation
\[
u_{tt}(x,t) - c(x)^2\,\Delta u(x,t) \;=\; 0,\qquad x\in\mathbb{R}^2,\ t\in\mathbb{R}.
\]

\smallskip

(a) In the high-frequency (short-wavelength) regime, it is natural to look for oscillatory solutions whose local frequency is very large. Make the ansatz
\[
u(x,t) \;=\; A(x)\,e^{i\omega\,(t - S(x))},
\]
where $\omega\gg 1$ is a large parameter, $A(x)$ is a slowly varying complex amplitude, and $S(x)$ is a real-valued phase function (the “optical path length” or “travel time”). 

Compute $u_t$, $u_{tt}$, $\nabla u$, and $\Delta u$ in terms of $A$, $S$, and their derivatives, and organize your expressions in powers of $\omega$ (that is, identify the terms proportional to $\omega^2$, to $\omega$, and of order $1$).  
Hint: It is convenient to write $u(x,t) = A(x)\,e^{i\phi(x,t)}$ with $\phi(x,t) = \omega (t - S(x))$ and to differentiate using the product rule.

\smallskip

(b) Substitute your expressions into the wave equation and divide out the common oscillatory factor $e^{i\omega(t-S(x))}$. Group the resulting equation according to powers of $\omega$. 

(i) Show that the coefficient of $\omega^2$ gives a leading-order relation between $S$ and $c(x)$.  

(ii) By neglecting lower-order terms in $\omega$, derive the \emph{eikonal equation} satisfied by $S(x)$ and write it in the form
\[
|\nabla S(x)| \;=\; \frac{1}{c(x)}.
\]
State briefly what approximation you have made when discarding the terms of order $\omega$ and $1$.

\smallskip

(c) The eikonal equation from part (b) is a nonlinear first-order PDE of the form
\[
F(x,\nabla S(x)) \;=\; 0,
\]
with 
\[
F(x,p) \;=\; \frac{1}{2}\Bigl(|p|^2 - n(x)^2\Bigr),
\qquad n(x) := \frac{1}{c(x)}.
\]
In general, for a first-order PDE $F(x,u,p)=0$ with $p = \nabla u$, the method of characteristics converts the PDE into a system of ODEs for curves $(x(s),u(s),p(s))$ in the extended space. In the special case that $F$ does not depend explicitly on $u$ (as here), the characteristic equations reduce to
\[
\dot{x}(s) = F_p(x(s),p(s)), \qquad
\dot{S}(s) = p(s)\cdot F_p(x(s),p(s)), \qquad
\dot{p}(s) = -F_x(x(s),p(s)),
\]
where dots denote derivatives with respect to the parameter $s$.

(i) Compute $F_p(x,p)$ and $F_x(x,p)$ for the above $F(x,p) = \tfrac{1}{2}(|p|^2 - n(x)^2)$.  

(ii) Write down explicitly the system of characteristic ODEs for $x(s)$, $S(s)$, and $p(s)=\nabla S(x(s))$.  

(iii) Give a geometric interpretation of the equation for $\dot{x}(s)$ in terms of the relation between the rays and the level sets of $S$ (that is, the wavefronts).  
Hint: Recall that the gradient of a function is normal to its level sets.

\smallskip

(d) Now consider a homogeneous medium with constant speed $c(x)\equiv c_0>0$, so that $n(x)\equiv n_0=1/c_0$. 

(i) Simplify the characteristic system from part (c) for this constant-coefficient case. What do you conclude about $p(s)$ and $x(s)$?  

(ii) Physically, suppose there is a point source located at the origin emitting circular wavefronts in this homogeneous medium. Assuming radial symmetry, write $S(x) = f(r)$ with $r=|x|$ and use the eikonal equation $|\nabla S| = 1/c_0$ to find $f(r)$ up to an additive constant.  

(iii) Describe the corresponding wavefronts and rays in the $x$–plane. How does this match your intuition about waves emitted from a point source in a constant-speed medium?

\smallskip

(e) \emph{Extensions and “what if” questions.}

(i) Suppose now that the medium is layered in the vertical direction, so that the refractive index $n(x)$ depends only on the vertical coordinate $y$, that is, $n(x_1,x_2) = n(y)$ with $y=x_2$. Use the characteristic equations to express the ray direction in terms of $p=\nabla S$, and introduce the angle $\theta(s)$ between the ray direction and the vertical axis. Show that along any ray, the quantity
\[
n(y)\,\sin\theta
\]
is conserved. (This is a differential form of Snell's law for a smoothly varying medium.)  
Hint: Write the ray direction as $\dot{x}/|\dot{x}|$ and relate its components to $\theta$; then use the ODE for $\dot{p}$.

(ii) Based on the conservation of $n(y)\sin\theta$, reason qualitatively about how rays bend if $n(y)$ increases with depth versus if $n(y)$ decreases with depth. Connect your conclusions to the physical statement that light bends toward regions of higher refractive index (lower speed).

\end{problem}

% ===== Example 5: Eikonal Equation and Geometrical Optics (full solution) =====
\begin{problem}[Eikonal equation and rays in geometrical optics]
Let $c(x)>0$ be a smooth wave speed in $\mathbb{R}^2$ and consider the variable-coefficient wave equation
\[
u_{tt}(x,t) - c(x)^2\,\Delta u(x,t) = 0.
\]
(a) For a large frequency parameter $\omega\gg 1$, assume an ansatz
\[
u(x,t) = A(x)\,e^{i\omega (t - S(x))},
\]
with slowly varying complex amplitude $A(x)$ and real phase $S(x)$. By substituting into the wave equation and collecting powers of $\omega$, derive the leading-order PDE satisfied by $S(x)$ and show that it can be written as the eikonal equation
\[
|\nabla S(x)| = \frac{1}{c(x)}.
\]

(b) View the eikonal equation as
\[
F(x,\nabla S(x)) = 0,\qquad
F(x,p) = \frac{1}{2}\bigl(|p|^2 - n(x)^2\bigr),\quad n(x)=\frac{1}{c(x)}.
\]
Write down the characteristic ODE system for $x(s)$, $S(s)$, and $p(s)=\nabla S(x(s))$ when $F$ does not depend on $S$, and specialize it to this $F$. Interpret the equation for $\dot{x}(s)$ in terms of the relationship between rays and wavefronts.

(c) In the special case of a homogeneous medium $c(x)\equiv c_0>0$ (so $n(x)\equiv n_0$), simplify the characteristic system, solve it qualitatively, and deduce that rays are straight lines. Then, assuming radial symmetry for a point source at the origin, solve the eikonal equation to obtain an explicit phase function $S(x)$ and describe the corresponding wavefronts and rays.

\end{problem}

\begin{solution}
We proceed in three steps: derivation of the eikonal equation from the high-frequency ansatz, formulation of the characteristic system for this nonlinear first-order PDE, and analysis of the special case of a homogeneous medium.

\medskip

\noindent\textbf{(a) Derivation of the eikonal equation.}
We are given
\[
u_{tt} - c(x)^2\,\Delta u = 0
\]
and the ansatz
\[
u(x,t) = A(x)\,e^{i\omega (t - S(x))},
\]
where $\omega\gg 1$. It is convenient to write
\[
\phi(x,t) = \omega\,(t - S(x)), \qquad u(x,t) = A(x)\,e^{i\phi(x,t)}.
\]
We now compute the necessary derivatives.

First, time derivatives. Since $A$ is independent of $t$, we have
\[
u_t = i\,\phi_t\,A\,e^{i\phi},
\]
and because $\phi_t = \omega$, this becomes
\[
u_t = i\omega\,A\,e^{i\phi}.
\]
Differentiating once more in $t$,
\[
u_{tt} = i\omega\,A_t\,e^{i\phi} + i\omega\,A\, (i\phi_t)\,e^{i\phi}.
\]
Here $A_t=0$ and $\phi_t=\omega$, so
\[
u_{tt} = i\omega A\cdot i\omega e^{i\phi} = -\omega^2 A\,e^{i\phi}.
\]

Next, spatial derivatives. For each spatial coordinate $x_j$ we have
\[
u_{x_j} = A_{x_j} e^{i\phi} + A\,(i\phi_{x_j}) e^{i\phi}.
\]
Since $\phi(x,t) = \omega(t - S(x))$, one has $\phi_{x_j} = -\omega\,S_{x_j}$. Thus
\[
u_{x_j}
= e^{i\phi}\bigl(A_{x_j} - i\omega A S_{x_j}\bigr).
\]
Differentiating again,
\[
\begin{aligned}
u_{x_j x_j}
&= \frac{\partial}{\partial x_j}\Bigl( e^{i\phi}\bigl(A_{x_j} - i\omega A S_{x_j}\bigr) \Bigr) \\
&= e^{i\phi}\Bigl(i\phi_{x_j}\bigl(A_{x_j} - i\omega A S_{x_j}\bigr)
+ A_{x_j x_j} - i\omega A_{x_j} S_{x_j} - i\omega A S_{x_j x_j}\Bigr).
\end{aligned}
\]
Using $\phi_{x_j} = -\omega S_{x_j}$, we obtain
\[
i\phi_{x_j}\bigl(A_{x_j} - i\omega A S_{x_j}\bigr)
= i(-\omega S_{x_j}) A_{x_j} - i(-\omega S_{x_j})\,i\omega A S_{x_j}
= -i\omega S_{x_j}A_{x_j} - \omega^2 A S_{x_j}^2.
\]
Thus
\[
u_{x_j x_j}
= e^{i\phi}\Bigl(A_{x_j x_j} - 2i\omega A_{x_j}S_{x_j} - i\omega A S_{x_j x_j} - \omega^2 A S_{x_j}^2\Bigr).
\]
Summing over $j=1,2$ gives the Laplacian,
\[
\Delta u = \sum_{j=1}^2 u_{x_j x_j}
= e^{i\phi}\Bigl(\Delta A - 2i\omega \nabla A\cdot \nabla S
- i\omega A\,\Delta S - \omega^2 A\,|\nabla S|^2\Bigr).
\]

Substituting $u_{tt}$ and $\Delta u$ into the wave equation yields
\[
-\omega^2 A\,e^{i\phi}
- c(x)^2 e^{i\phi}\Bigl(\Delta A - 2i\omega \nabla A\cdot \nabla S
- i\omega A\,\Delta S - \omega^2 A\,|\nabla S|^2\Bigr) = 0.
\]
We can factor out the nonzero oscillatory factor $e^{i\phi}$ and write
\[
-\omega^2 A
- c^2\Delta A
+ 2i\omega c^2 \nabla A\cdot \nabla S
+ i\omega c^2 A\,\Delta S
+ \omega^2 c^2 A\,|\nabla S|^2
= 0.
\]
We now sort terms by powers of $\omega$:

\begin{itemize}
\item Terms of order $\omega^2$:
\[
A\bigl(-1 + c(x)^2 |\nabla S(x)|^2\bigr).
\]
\item Terms of order $\omega$:
\[
i\omega c(x)^2\bigl(2\nabla A\cdot \nabla S + A\,\Delta S\bigr).
\]
\item Terms of order $1$:
\[
- c(x)^2 \Delta A.
\]
\end{itemize}

In the high-frequency approximation $\omega\gg 1$, the dominant balance is at order $\omega^2$. Assuming $A(x)\not\equiv 0$, the coefficient of $\omega^2$ must vanish, which gives
\[
-1 + c(x)^2 |\nabla S(x)|^2 = 0,
\]
or equivalently
\[
c(x)^2 |\nabla S(x)|^2 = 1.
\]
Taking square roots (and choosing the positive sign to represent increasing travel time with distance) we arrive at the \emph{eikonal equation}
\[
|\nabla S(x)| = \frac{1}{c(x)}.
\]

The lower-order terms in $\omega$ (the $\omega^1$ and $\omega^0$ contributions) determine transport equations for the amplitude $A(x)$ and represent corrections to the geometrical optics approximation. In deriving the eikonal equation we have neglected these subleading terms, which is valid when $\omega$ is large and $A$ and $S$ vary on spatial scales much larger than the wavelength.

\medskip

\noindent\textbf{(b) Characteristic system for the eikonal equation.}
We rewrite the eikonal equation in the form
\[
F(x,\nabla S(x)) = 0,\qquad
F(x,p) = \frac{1}{2}\bigl(|p|^2 - n(x)^2\bigr),\quad n(x)=\frac{1}{c(x)}.
\]
For a general first-order PDE $F(x,u,p)=0$ with $p=\nabla u$, the characteristic curves $(x(s),u(s),p(s))$ in the extended space satisfy
\[
\dot{x} = F_p(x,u,p),\qquad
\dot{u} = p\cdot F_p(x,u,p),\qquad
\dot{p} = -F_x(x,u,p) - p\,F_u(x,u,p).
\]
In our case, $F$ does not depend on $u=S$, so $F_u\equiv 0$ and $F=F(x,p)$ only. The characteristic system simplifies to
\[
\dot{x}(s) = F_p(x(s),p(s)),\qquad
\dot{S}(s) = p(s)\cdot F_p(x(s),p(s)),\qquad
\dot{p}(s) = -F_x(x(s),p(s)).
\]

We now compute the derivatives of $F$:
\[
F(x,p) = \frac{1}{2}|p|^2 - \frac{1}{2}n(x)^2.
\]
The derivative with respect to $p$ is
\[
F_p(x,p) = p,
\]
and the derivative with respect to $x$ is
\[
F_x(x,p) = -\frac{1}{2}\,\nabla_x\bigl(n(x)^2\bigr).
\]
Thus, the characteristic ODEs become
\[
\dot{x}(s) = p(s),\qquad
\dot{S}(s) = p(s)\cdot p(s) = |p(s)|^2,\qquad
\dot{p}(s) = -F_x(x(s),p(s)) = \frac{1}{2}\,\nabla\bigl(n(x(s))^2\bigr).
\]

On the eikonal manifold $F=0$ we have $|p|^2 = n(x)^2$, so along any characteristic,
\[
\dot{S}(s) = |p(s)|^2 = n(x(s))^2.
\]
The key equation for the geometry of rays is
\[
\dot{x}(s) = p(s) = \nabla S(x(s)).
\]
Recall that the gradient of a scalar function is normal to its level sets. The level sets of $S$,
\[
\{x : S(x) = \text{constant}\},
\]
are precisely the wavefronts. The equation $\dot{x} = \nabla S$ means that the tangent vector to a characteristic (ray) is parallel to the gradient of $S$; hence rays always cross wavefronts orthogonally. Thus, the method of characteristics reveals that in geometrical optics the energy propagates along curves that are perpendicular to the phase surfaces.

This is a central idea in this section: for a nonlinear first-order PDE of Hamilton–Jacobi type, the characteristic curves in physical space are the integral curves of the gradient of the unknown, and they provide a geometric representation of the solution in terms of rays and wavefronts.

\medskip

\noindent\textbf{(c) Homogeneous medium: straight rays and explicit phase.}
Suppose now that the medium is homogeneous:
\[
c(x)\equiv c_0>0,\qquad n(x) \equiv n_0 = \frac{1}{c_0}.
\]
Then $n(x)^2 = n_0^2$ is constant, so
\[
\nabla\bigl(n(x)^2\bigr) = 0,
\]
and the characteristic equations simplify considerably:
\[
\dot{x}(s) = p(s),\qquad
\dot{S}(s) = |p(s)|^2,\qquad
\dot{p}(s) = 0.
\]
The last equation, $\dot{p}(s)=0$, shows that $p(s)$ is constant along each characteristic: $p(s)\equiv p_0$ for some constant vector $p_0$ on a given ray. Consequently,
\[
\dot{x}(s) = p_0\quad\Longrightarrow\quad x(s) = x(0) + s\,p_0,
\]
so the projection of the characteristic curves into the $x$–plane are straight lines. These straight lines are the \emph{rays} of geometrical optics in a homogeneous medium. This matches the familiar physical fact that light travels in straight lines in a uniform medium.

To understand the phase function $S(x)$ for a point source, consider a source at the origin emitting outgoing circular waves. By symmetry, $S$ should depend only on the radial coordinate $r = |x|$, so write
\[
S(x) = f(r).
By direct computation, a radially symmetric function $S(x)=f(r)$ has gradient
\[
\nabla S(x) = f'(r)\,\frac{x}{r},\qquad r=|x|,
\]
so that
\[
|\nabla S(x)| = |f'(r)|.
\]
The eikonal equation $|\nabla S| = 1/c_0$ therefore becomes the ODE
\[
|f'(r)| = \frac{1}{c_0} = n_0.
\]
Choosing the positive sign to represent an increasing phase with increasing distance from the source (outgoing waves), we take
\[
f'(r) = \frac{1}{c_0},
\]
which integrates to
\[
f(r) = \frac{r}{c_0} + \text{constant} = n_0\,r + \text{constant}.
\]
Up to an additive constant (which only shifts the phase), we may write
\[
S(x) = \frac{|x|}{c_0}.
\]

The level sets of $S$,
\[
S(x) = \text{constant} \quad\Longleftrightarrow\quad |x| = \text{constant},
\]
are concentric circles centered at the origin. These are the \emph{wavefronts} (surfaces of constant phase or travel time). The rays, given by the characteristic curves $x(s)$ satisfying $\dot{x}(s)=\nabla S(x(s))$, are straight lines passing through the origin:
\[
\dot{x}(s) \parallel x(s)\quad\Longrightarrow\quad x(s) = x(0) + s\,p_0,
\]
with $p_0$ a constant vector pointing radially outward (for outgoing waves).

Thus, in a homogeneous medium:

- Wavefronts are expanding circles centered at the source.
- Rays are straight radial lines orthogonal to these circles.

This coincides with the usual physical picture: from a point source in a uniform medium, waves propagate outward in all directions with speed $c_0$, forming circular (in 2D) or spherical (in 3D) wavefronts, and energy travels along straight radial rays.

\end{solution}

\section{Classification of Linear Second-Order PDEs}
% --- Narrative plan (auto-generated) ---
% This section introduces the classical classification of linear second-order partial differential equations into elliptic, parabolic, and hyperbolic types. The classification is based on the quadratic form built from the second-derivative terms, in much the same way that conic sections are classified using the discriminant. Understanding whether a PDE is elliptic, parabolic, or hyperbolic immediately suggests which kinds of boundary or initial data are appropriate, what sorts of solution behavior one should expect, and which analytical or numerical techniques are likely to work.
%
% In applied mathematics, this classification underlies the modeling and analysis of diffusion processes, steady-state fields, and wave propagation, which appear everywhere in physics, engineering, and quantitative biology. Elliptic equations such as Laplace’s equation are tied to potential theory and harmonic functions, and connect naturally to complex analysis and Fourier series methods. Parabolic equations such as the heat equation are closely related to dynamical systems and semigroup theory, while hyperbolic equations such as the wave equation connect to characteristic curves and ideas from ordinary differential equations. The classification also guides later topics in this course, including separation of variables, transform methods, and energy estimates, by telling us when and why those methods apply.

% ===== Example 1: A Family of Second-Order PDEs in Two Variables (inquiry-based) =====
\begin{problem}[A Family of Second-Order PDEs in Two Variables]
Many of the standard linear partial differential equations of mathematical physics, such as Laplace's equation, the heat equation, and the wave equation, have second-order derivatives as their highest-order terms. In two spatial variables, these highest-order terms can be organized into a quadratic form in the derivatives. The algebraic type of that quadratic form (positive definite, indefinite, or degenerate) turns out to govern many qualitative features of the solutions. In this problem you will explore this connection for a general constant-coefficient second-order equation in two variables.

Consider the family of linear second-order partial differential equations in two variables
\[
a\,u_{xx} + 2b\,u_{xy} + c\,u_{yy} + \text{(lower-order terms)} = f(x,y),
\]
where $a$, $b$, and $c$ are real constants, not all zero in the second-order part.

\smallskip

(a) Warm-up with model examples. For each of the following equations, write down the corresponding triple $(a,b,c)$ and the quadratic form
\[
Q(\xi,\eta) = a\xi^2 + 2b\xi\eta + c\eta^2.
\]
Then compute the discriminant $D = b^2 - ac$ of this quadratic form.
\begin{enumerate}
\item[(i)] Laplace's equation: $u_{xx} + u_{yy} = 0$.
\item[(ii)] The wave equation in one space and one time dimension: $u_{tt} - c^2 u_{xx} = 0$. (For this part, simply treat $t$ as one of the independent variables alongside $x$ and ignore the physical meaning.)
\item[(iii)] The steady (time-independent) heat equation with a reaction term: $u_{xx} + u_{yy} - u = 0$.
\end{enumerate}
How do the signs of the discriminants you find compare across these three model equations?

\smallskip

(b) Associate a matrix to the principal part. Show that the quadratic form $Q(\xi,\eta)$ from part (a) can be written as
\[
Q(\xi,\eta) = 
\begin{bmatrix}\xi & \eta\end{bmatrix}
\begin{bmatrix}a & b \\[4pt] b & c\end{bmatrix}
\begin{bmatrix}\xi \\[4pt]\eta\end{bmatrix}
= \mathbf{v}^{T}A\,\mathbf{v},
\]
where $\mathbf{v} = \begin{bmatrix}\xi \\ \eta\end{bmatrix}$ and $A$ is the symmetric $2\times 2$ matrix with entries $a$, $b$, and $c$. 
\begin{enumerate}
\item[(i)] Verify the matrix expression for $Q(\xi,\eta)$ by direct multiplication.
\item[(ii)] Explain why a linear change of variables $(x,y)\mapsto(\xi,\eta)$ in the PDE corresponds to the action of an invertible $2\times 2$ matrix $P$ on the vector of frequencies $\mathbf{v} = (\xi,\eta)^T$ in the quadratic form.
\end{enumerate}
Hint: Think about the chain rule for first and second derivatives, and how gradients and Hessians transform under linear maps.

\smallskip

(c) Diagonalizing the principal part. Suppose that $A$ has two real eigenvalues $\lambda_1$ and $\lambda_2$ and an orthogonal matrix $R$ with $R^T A R = \operatorname{diag}(\lambda_1,\lambda_2)$. 
\begin{enumerate}
\item[(i)] Show that in the new coordinates $(\xi,\eta)$ obtained by applying the rotation $R$ to $(x,y)$, the principal part of the PDE becomes
\[
\lambda_1\,u_{\xi\xi} + \lambda_2\,u_{\eta\eta}.
\]
(You may ignore lower-order terms in this step and focus only on the second derivatives.)
\item[(ii)] Using basic facts about $2\times 2$ symmetric matrices, relate the signs of $\lambda_1$ and $\lambda_2$ to the discriminant $D = b^2 - ac$ and to the determinant $\det A = ac - b^2$. In particular, determine when both eigenvalues have the same sign, when they have opposite signs, and when one eigenvalue is zero.
\end{enumerate}
Hint: Recall that for a real symmetric $2\times 2$ matrix $A$, the trace $\operatorname{tr}A = a + c$ equals $\lambda_1 + \lambda_2$ and the determinant $\det A = ac - b^2$ equals $\lambda_1 \lambda_2$.

\smallskip

(d) Classification and canonical forms. We say that the PDE is:
\begin{itemize}
\item \emph{elliptic} if the quadratic form $Q$ is definite (always positive or always negative, except at the origin),
\item \emph{hyperbolic} if $Q$ is indefinite,
\item \emph{parabolic} if $Q$ is degenerate but not identically zero.
\end{itemize}
\begin{enumerate}
\item[(i)] Using your work from part (c), express these three cases in terms of the discriminant $D = b^2 - ac$.
\item[(ii)] Show that after an appropriate linear change of variables $(x,y)\mapsto(\xi,\eta)$ that simplifies the principal part, any such PDE with constant coefficients can be put into one of the following canonical forms for its leading part:
\[
u_{\xi\xi} + u_{\eta\eta}, \qquad
u_{\xi\xi} - u_{\eta\eta}, \qquad
u_{\xi\xi},
\]
corresponding to elliptic, hyperbolic, and parabolic equations, respectively (up to constant nonzero factors).
\item[(iii)] Match each of the model equations from part (a) with one of the three canonical forms above by an appropriate relabeling or rescaling of the independent variables.
\end{enumerate}
Hint: You may assume that multiplying the entire PDE by a nonzero constant does not change its type. Focus on the signs and possible zeros of the eigenvalues $\lambda_1$ and $\lambda_2$.

\smallskip

(e) Explorations and extensions.
\begin{enumerate}
\item[(i)] Suppose now that the coefficients $a$, $b$, and $c$ depend smoothly on $(x,y)$, so that you have a PDE of the form
\[
a(x,y)u_{xx} + 2b(x,y)u_{xy} + c(x,y)u_{yy} + \text{(lower-order terms)} = f(x,y).
\]
How might you use the sign of the discriminant $D(x,y) = b(x,y)^2 - a(x,y)c(x,y)$ to classify the equation \emph{locally} near a given point $(x_0,y_0)$?
\item[(ii)] The usual time-dependent heat equation in one space dimension is
\[
u_t = k\,u_{xx}.
\]
If you treat $t$ and $x$ simply as two independent variables, can you fit this equation into the framework of part (d)? What are the corresponding $(a,b,c)$, and what is the discriminant $D$? How does that relate to the description of the heat equation as \emph{parabolic}?
\end{enumerate}

\end{problem}

% ===== Example 1: A Family of Second-Order PDEs in Two Variables (full solution) =====
\begin{problem}[A Family of Second-Order PDEs in Two Variables]
Consider the linear second-order partial differential equation with constant coefficients
\[
a\,u_{xx} + 2b\,u_{xy} + c\,u_{yy} + \text{(lower-order terms)} = f(x,y),
\]
where $a,b,c\in\mathbb{R}$ and $(a,b,c)\neq(0,0,0)$ in the second-order part.

\begin{enumerate}
\item[(a)] Associate to the principal part the quadratic form
\[
Q(\xi,\eta) = a\xi^2 + 2b\xi\eta + c\eta^2,
\]
and the symmetric matrix
\[
A = \begin{bmatrix} a & b \\[4pt] b & c \end{bmatrix}.
\]
Show that under an invertible linear change of variables $(x,y)\mapsto(\xi,\eta)$, the principal part can be written as
\[
\lambda_1\,u_{\xi\xi} + \lambda_2\,u_{\eta\eta},
\]
where $\lambda_1$ and $\lambda_2$ are the eigenvalues of $A$.

\item[(b)] Express the signs of $\lambda_1$ and $\lambda_2$ in terms of the discriminant
\[
D = b^2 - ac.
\]
Classify the PDE as \emph{elliptic}, \emph{hyperbolic}, or \emph{parabolic} according to whether $Q$ is definite, indefinite, or degenerate, and show that this corresponds to the cases
\[
D<0,\quad D>0,\quad D=0,
\]
respectively (assuming the principal part is not identically zero).

\item[(c)] Deduce that, after an appropriate linear change of variables and multiplication of the equation by a nonzero constant, the principal part of any such PDE can be brought into one of the canonical forms
\[
u_{\xi\xi} + u_{\eta\eta}, \qquad
u_{\xi\xi} - u_{\eta\eta}, \qquad
u_{\xi\xi},
\]
corresponding to elliptic, hyperbolic, and parabolic equations. Illustrate this correspondence by identifying the types of
\[
u_{xx} + u_{yy} = 0,\qquad
u_{tt} - c^2 u_{xx} = 0,\qquad
u_t - k u_{xx} = 0,
\]
viewed as equations in two independent variables.
\end{enumerate}
\end{problem}

\begin{solution}
We begin by isolating the principal (second-order) part of the equation
\[
a\,u_{xx} + 2b\,u_{xy} + c\,u_{yy} = \text{(terms of order }\le 1).
\]
The classification of such equations is governed entirely by this principal part.

\medskip

\noindent\textbf{(a) Quadratic form, matrix, and change of variables.}
We associate to the principal part the quadratic form in two real variables
\[
Q(\xi,\eta) = a\xi^2 + 2b\xi\eta + c\eta^2.
\]
It is convenient to write this in matrix notation. Define the symmetric matrix
\[
A = \begin{bmatrix} a & b \\[4pt] b & c \end{bmatrix}
\]
and the column vector $\mathbf{v} = \begin{bmatrix}\xi \\[2pt] \eta\end{bmatrix}$. Then
\[
\mathbf{v}^{T}A\mathbf{v}
= \begin{bmatrix}\xi & \eta\end{bmatrix}
\begin{bmatrix} a & b \\[4pt] b & c \end{bmatrix}
\begin{bmatrix}\xi \\[2pt] \eta\end{bmatrix}
= a\xi^2 + 2b\xi\eta + c\eta^2
= Q(\xi,\eta).
\]
Thus the matrix $A$ encodes the coefficients of the second-order derivatives.

Now consider an invertible linear change of variables
\[
\begin{bmatrix} x \\[2pt] y \end{bmatrix}
= P \begin{bmatrix} \xi \\[2pt] \eta \end{bmatrix},
\]
where $P$ is a $2\times 2$ invertible real matrix. The chain rule shows that the gradient and Hessian transform in a linear fashion. In particular, the Hessian of $u$ with respect to $(x,y)$ is related to the Hessian with respect to $(\xi,\eta)$ by
\[
H_{(x,y)}u = P^{-T} \, H_{(\xi,\eta)}u \, P^{-1},
\]
where the superscript $-T$ denotes the inverse transpose. When we contract the Hessian with the matrix $A$ to form the principal part, we obtain
\[
\begin{bmatrix} u_{xx} & u_{xy} \\[2pt] u_{xy} & u_{yy} \end{bmatrix}
: A
\quad\text{transforms to}\quad
\begin{bmatrix} u_{\xi\xi} & u_{\xi\eta} \\[2pt] u_{\xi\eta} & u_{\eta\eta} \end{bmatrix}
: (P^{T} A P),
\]
where $:$ denotes the Frobenius inner product of matrices. Concretely, in the new variables the principal part becomes
\[
a'\,u_{\xi\xi} + 2b'\,u_{\xi\eta} + c'\,u_{\eta\eta},
\]
with the matrix of new coefficients
\[
A' = P^{T}AP.
\]

Since $A$ is a real symmetric matrix, there exists an orthogonal matrix $R$ and real eigenvalues $\lambda_1,\lambda_2$ such that
\[
R^{T} A R = \begin{bmatrix} \lambda_1 & 0 \\[4pt] 0 & \lambda_2 \end{bmatrix}.
\]
Taking $P = R$ corresponds to a rotation of coordinates. In these rotated variables $(\xi,\eta)$, the matrix of principal coefficients is diagonal, and the mixed derivative term disappears. The principal part is then
\[
\lambda_1\,u_{\xi\xi} + \lambda_2\,u_{\eta\eta},
\]
as claimed. Lower-order terms may become more complicated, but they do not affect the classification by type.

\medskip

\noindent\textbf{(b) Eigenvalues, discriminant, and classification.}
The eigenvalues $\lambda_1$ and $\lambda_2$ of the $2\times 2$ symmetric matrix $A$ are real, and they control the nature of the quadratic form
\[
Q(\xi,\eta) = \lambda_1 \xi'^2 + \lambda_2 \eta'^2
\]
in diagonalized coordinates $(\xi',\eta')$.

For a $2\times 2$ matrix $A$, the trace and determinant are
\[
\operatorname{tr}A = a + c = \lambda_1 + \lambda_2,\qquad
\det A = ac - b^2 = \lambda_1 \lambda_2.
\]
By definition, the discriminant of the quadratic form is
\[
D = b^2 - ac = -\det A.
\]
Thus
\[
\det A = -D.
\]

We consider three cases:

\medskip

\emph{Case 1: $D<0$.} Then $\det A > 0$. Since $\det A = \lambda_1 \lambda_2 > 0$, the eigenvalues have the same sign: either both positive or both negative. If, in addition, at least one of $a$ or $c$ is nonzero (so the principal part is not identically zero), then $\lambda_1$ and $\lambda_2$ are both nonzero. Therefore the quadratic form $Q$ is definite: it is either positive definite or negative definite, depending on the common sign of the eigenvalues. In this case the PDE is called \emph{elliptic}.

\medskip

\emph{Case 2: $D>0$.} Then $\det A < 0$. Hence $\lambda_1 \lambda_2 < 0$, so the eigenvalues have opposite signs. The quadratic form $Q$ is indefinite: it takes both positive and negative values. The PDE is called \emph{hyperbolic}.

\medskip

\emph{Case 3: $D=0$.} Then $\det A = 0$, so at least one eigenvalue is zero. If the principal part is not identically zero, then exactly one eigenvalue is zero and the other is nonzero. In diagonal form, the quadratic form is proportional to a single square, such as $\lambda_1 \xi'^2$. The quadratic form is degenerate but not identically zero. The PDE is called \emph{parabolic}.

\medskip

These three algebraic possibilities (definite, indefinite, degenerate nonzero) are precisely the three classical types of second-order linear PDE in two variables.

\medskip

\noindent\textbf{(c) Canonical forms and examples.}
From part (a) we know that, after an orthogonal change of variables, the principal part becomes
\[
\lambda_1\,u_{\xi\xi} + \lambda_2\,u_{\eta\eta}.
\]
We are also allowed to multiply the entire PDE by a nonzero constant without changing its type, since this does not alter the sign pattern or degeneracy of the quadratic form.

\medskip

\emph{Elliptic case $D<0$.} Here $\lambda_1$ and $\lambda_2$ are both nonzero and have the same sign. Multiplying the equation by a constant, we may assume without loss of generality that $\lambda_1 = \lambda_2 = 1$ or $\lambda_1 = \lambda_2 = -1$. Multiplying by $-1$ if necessary, the principal part is equivalent to
\[
u_{\xi\xi} + u_{\eta\eta}.
\]
This is the canonical elliptic operator in two variables, exemplified by Laplace's equation.

\medskip

\emph{Hyperbolic case $D>0$.} Here $\lambda_1$ and $\lambda_2$ are nonzero and have opposite signs. Multiplying by a suitable nonzero constant, we may assume $\lambda_1 = 1$ and $\lambda_2 = -1$ (or vice versa). Thus the principal part is equivalent to
\[
u_{\xi\xi} - u_{\eta\eta}.
\]
This is the standard hyperbolic form, closely related to the one-dimensional wave operator.

\medskip

\emph{Parabolic case $D=0$.} Here one eigenvalue is zero and the other is nonzero. After scaling, we may assume that the nonzero eigenvalue equals $1$, so the principal part is equivalent to
\[
u_{\xi\xi}.
\]
This is the canonical parabolic form in two variables.

\medskip

We now illustrate this classification with the concrete equations in the problem statement, viewed simply as equations in two independent variables.

\medskip

\emph{Example 1: Laplace's equation.} Consider
\[
u_{xx} + u_{yy} = 0.
\]
Here $a = 1$, $b = 0$, $c = 1$. The discriminant is
\[
D = b^2 - ac = 0^2 - (1)(1) = -1 < 0.
\]
Hence the equation is elliptic. In fact, it is already in the canonical elliptic form
\[
u_{xx} + u_{yy} = 0,
\]
so no change of variables is needed.

\medskip

\emph{Example 2: The one-dimensional wave equation.} Consider
\[
u_{tt} - c^2 u_{xx} = 0.
\]
If we treat $(t,x)$ as the two independent variables, then $a = 1$ for $u_{tt}$, $b = 0$, and $c = -c^2$ for $u_{xx}$. The discriminant is
\[
D = b^2 - ac = 0^2 - (1)(-c^2) = c^2 > 0,
\]
so the equation is hyperbolic. Its principal part is already of the canonical form
\[
u_{tt} - c^2 u_{xx} = 0,
\]
and by rescaling the space (or time) variable, for example setting $\xi = t$ and $\eta = c x$, we can write it as
\[
u_{\xi\xi} - u_{\eta\eta} = 0,
\]
which is the standard hyperbolic canonical form.

\medskip

\emph{Example 3: The (1+1)-dimensional heat equation.} Consider
\[
u_t - k u_{xx} = 0.
\]
Again treating $(t,x)$ as our two independent variables, the principal part involves the derivatives $u_{tt}$, $u_{tx}$, and $u_{xx}$. However, in this equation there is \emph{no} $u_{tt}$ or $u_{tx}$ term, only $u_{xx}$. Thus, for the purpose of classification in two variables, we can regard
\[
a = 0,\quad b = 0,\quad c = -k
\]
when we order the variables as $(t,x)$. The quadratic form is
\[
Q(\xi,\eta) = -k \eta^2,
\]
and the discriminant is
\[
D = b^2 - ac = 0^2 - (0)(-k) = 0.
\]
Thus the heat equation is parabolic. In fact, after relabeling $\xi = x$ and $\eta = t$, the principal part is simply proportional to $u_{\xi\xi}$, which is exactly the parabolic canonical form.

\medskip

\noindent\textbf{Connection to the general theory.}
This example illustrates the main idea of the classification of linear second-order PDEs in two variables: the type of the equation is determined by the algebraic type of the quadratic form associated with its principal part. The symmetric matrix of second-order coefficients can be diagonalized by an orthogonal change of variables, which removes the mixed second derivative and reduces the principal part to a combination of pure second derivatives with coefficients equal to the eigenvalues. The sign pattern and possible degeneracy of these eigenvalues are captured compactly by the discriminant $D = b^2 - ac$, and they lead naturally to the three canonical forms
\[
u_{\xi\xi} + u_{\eta\eta},\quad
u_{\xi\xi} - u_{\eta\eta},\quad
u_{\xi\xi},
\]
representing elliptic, hyperbolic, and parabolic equations, respectively. These three types correspond to fundamentally different qualitative behaviors of solutions, as seen in Laplace's equation, the wave equation, and the heat equation.
\end{solution}

% ===== Example 2: Laplace’s Equation and Elliptic Type (inquiry-based) =====
\begin{problem}[Laplace’s Equation and Elliptic Type]
Laplace’s equation
\[
u_{xx} + u_{yy} = 0
\]
arises in many physical models, including steady-state heat flow, electrostatics, and incompressible fluid flow in two dimensions. In these settings, one typically prescribes the value of $u$ (for instance, temperature or electric potential) along a closed curve, and the equation determines a smooth interior configuration. From the viewpoint of classification of second-order partial differential equations, Laplace’s equation is the prototype of an \emph{elliptic} equation, and many of its qualitative properties (such as the maximum principle) are already visible in this simple case.

Recall that a general linear second-order partial differential equation in two variables $x$ and $y$ can be written in the form
\[
A(x,y)\,u_{xx} + 2B(x,y)\,u_{xy} + C(x,y)\,u_{yy} + \text{lower order terms} = F(x,y),
\]
where $A,B,C,F$ are given functions. The classification into elliptic, parabolic, or hyperbolic types is based on the discriminant $B^2 - AC$.

\smallskip

(a) Rewrite Laplace’s equation in the general form
\[
A\,u_{xx} + 2B\,u_{xy} + C\,u_{yy} = 0,
\]
with \emph{constant} coefficients $A,B,C$. Identify $A,B,C$ and compute the discriminant $B^2 - AC$. According to the usual classification rule, what type of equation is Laplace’s equation?

\medskip

(b) A useful way to visualize the classification is to associate to the second-order part of the equation the quadratic form
\[
Q(\xi,\eta) = A \xi^2 + 2B\xi\eta + C\eta^2
\]
in the variables $(\xi,\eta) \in \mathbb{R}^2$.

\begin{enumerate}
\item[(i)] For Laplace’s equation, write down the explicit formula for $Q(\xi,\eta)$.
\item[(ii)] Show that $Q(\xi,\eta)$ is \emph{positive definite}, that is, $Q(\xi,\eta) > 0$ whenever $(\xi,\eta) \neq (0,0)$.
\end{enumerate}
Hint: One option is to compute the eigenvalues of the symmetric matrix
\[
\begin{pmatrix}
A & B\\[4pt]
B & C
\end{pmatrix}
\]
and show that they are both positive. A simpler route may be available for this special case.

\medskip

(c) The coefficients $A,B,C$ and the quadratic form $Q$ do not depend on the coordinate axes you choose; in other words, they transform in a natural way under changes of variables. In the elliptic case, one can ``diagonalize'' the quadratic form $Q$ by a rotation of coordinates.

Suppose we perform a rotation of the $(x,y)$-plane by an angle $\theta$, introducing new variables $(\xi,\eta)$ via
\[
\begin{pmatrix}
\xi\\[4pt]
\eta
\end{pmatrix}
=
\begin{pmatrix}
\cos\theta & \sin\theta\\[4pt]
-\sin\theta & \cos\theta
\end{pmatrix}
\begin{pmatrix}
x\\[4pt]
y
\end{pmatrix}.
\]
Let $v(\xi,\eta) = u(x(\xi,\eta), y(\xi,\eta))$ be the expression of $u$ in the rotated coordinates.

\begin{enumerate}
\item[(i)] Without doing a full chain rule computation, explain why the Laplace operator $u_{xx} + u_{yy}$ should have the same form $v_{\xi\xi} + v_{\eta\eta}$ in the rotated $(\xi,\eta)$-coordinates.
\item[(ii)] How does this rotational invariance of $u_{xx} + u_{yy}$ fit with the geometric picture of Laplace’s equation as elliptic?
\end{enumerate}
Hint: Think of $u_{xx} + u_{yy}$ as the trace of the Hessian matrix of $u$, and recall how orthogonal transformations affect traces and eigenvalues of matrices.

\medskip

(d) One of the most important qualitative features of elliptic equations is the \emph{maximum principle}: interior maxima (or minima) of a solution are strongly constrained. For Laplace’s equation, solutions are called \emph{harmonic functions}.

Let $\Omega \subset \mathbb{R}^2$ be a bounded open set, and suppose $u \in C^2(\Omega) \cap C^0(\overline{\Omega})$ is harmonic in $\Omega$, that is, $u_{xx} + u_{yy} = 0$ in $\Omega$. Assume that $u$ attains its maximum value at some interior point $(x_0,y_0) \in \Omega$.

\begin{enumerate}
\item[(i)] Explain why the gradient of $u$ must vanish at $(x_0,y_0)$, and why the Hessian matrix $D^2u(x_0,y_0)$ must be negative semidefinite (that is, all its eigenvalues are less than or equal to zero).
\item[(ii)] Show that the equation $u_{xx} + u_{yy} = 0$ at $(x_0,y_0)$ forces all eigenvalues of $D^2u(x_0,y_0)$ to be equal to zero.
\item[(iii)] Conclude that if $u$ achieves its maximum at an interior point, then $u$ must, in fact, be constant in $\Omega$.
\end{enumerate}
Hint: Combine the fact that the eigenvalues of a symmetric matrix have nonpositive real parts when the matrix is negative semidefinite with the observation that the trace of the Hessian is the sum of its eigenvalues.

\medskip

(e) \textbf{What if / extensions.}
\begin{enumerate}
\item[(i)] Consider instead the equation $u_{xx} - u_{yy} = 0$. Classify this equation using the discriminant $B^2 - AC$. Does the associated quadratic form have the same sign properties as in parts (b)–(c)? Briefly describe how you would expect the qualitative behavior of its solutions to differ from those of Laplace’s equation (for example, in terms of boundary value problems).
\item[(ii)] More generally, for the equation
\[
a\,u_{xx} + c\,u_{yy} = 0,
\]
where $a$ and $c$ are nonzero constants, classify the equation according to the signs of $a$ and $c$. How do changes in these signs relate to the geometric type of the quadratic form $a\xi^2 + c\eta^2$ and to the elliptic or hyperbolic nature of the equation?
\end{enumerate}

\end{problem}

% ===== Example 2: Laplace’s Equation and Elliptic Type (full solution) =====
\begin{problem}[Laplace’s Equation and Elliptic Type]
Consider Laplace’s equation in two variables,
\[
u_{xx} + u_{yy} = 0.
\]
\begin{enumerate}
\item[(a)] Write this in the general second-order form
\[
A\,u_{xx} + 2B\,u_{xy} + C\,u_{yy} = 0
\]
and classify the equation as elliptic, parabolic, or hyperbolic using the discriminant $B^2 - AC$. Express the associated quadratic form $Q(\xi,\eta) = A\xi^2 + 2B\xi\eta + C\eta^2$ and show that it is positive definite.

\item[(b)] Interpret the operator $u_{xx} + u_{yy}$ as the trace of the Hessian matrix $D^2u$ and explain briefly why this operator is invariant in form under any rotation of the $(x,y)$-coordinates. Relate this rotational invariance to the elliptic character of Laplace’s equation.

\item[(c)] Let $\Omega \subset \mathbb{R}^2$ be a bounded open set and suppose that $u \in C^2(\Omega) \cap C^0(\overline{\Omega})$ satisfies Laplace’s equation $u_{xx} + u_{yy} = 0$ in $\Omega$. Assume that $u$ attains its maximum at an interior point $(x_0,y_0) \in \Omega$. Show that $u$ must be constant in $\Omega$. (You may argue using the eigenvalues of the Hessian matrix $D^2u(x_0,y_0)$.)

\item[(d)] For comparison, classify the equation $u_{xx} - u_{yy} = 0$ using the discriminant and comment briefly on how its type (elliptic vs.\ hyperbolic) suggests a different qualitative behavior of solutions from that of Laplace’s equation.
\end{enumerate}
\end{problem}

\begin{solution}
We analyze Laplace’s equation from the viewpoint of the general theory of second-order linear partial differential equations in two variables.

\medskip

\noindent\textbf{(a) Classification and quadratic form.}
A general linear second-order equation with no lower order terms can be written as
\[
A\,u_{xx} + 2B\,u_{xy} + C\,u_{yy} = 0,
\]
where $A,B,C$ are functions or constants. For Laplace’s equation
\[
u_{xx} + u_{yy} = 0,
\]
we read off
\[
A = 1,\quad B = 0,\quad C = 1.
\]
The discriminant is
\[
B^2 - AC = 0^2 - (1)(1) = -1 < 0.
\]
By the standard classification, a second-order equation in two variables is
\begin{itemize}
\item elliptic if $B^2 - AC < 0$,
\item parabolic if $B^2 - AC = 0$,
\item hyperbolic if $B^2 - AC > 0$.
\end{itemize}
Therefore, Laplace’s equation is \emph{elliptic} at every point of the plane.

The associated quadratic form is
\[
Q(\xi,\eta) = A\xi^2 + 2B\xi\eta + C\eta^2
= 1\cdot \xi^2 + 0\cdot (2\xi\eta) + 1\cdot \eta^2
= \xi^2 + \eta^2.
\]
For $(\xi,\eta) \neq (0,0)$, we have
\[
Q(\xi,\eta) = \xi^2 + \eta^2 > 0.
\]
Thus $Q$ is \emph{positive definite}. This is the algebraic manifestation of ellipticity: the second-order part defines a positive definite quadratic form. In terms of matrices, the second-order coefficients form the symmetric matrix
\[
\begin{pmatrix}
A & B\\
B & C
\end{pmatrix}
=
\begin{pmatrix}
1 & 0\\
0 & 1
\end{pmatrix},
\]
whose eigenvalues are both equal to $1$, hence strictly positive.

\medskip

\noindent\textbf{(b) Rotational invariance and ellipticity.}
The Hessian matrix of a twice differentiable function $u$ is
\[
D^2 u =
\begin{pmatrix}
u_{xx} & u_{xy}\\
u_{xy} & u_{yy}
\end{pmatrix}.
\]
The Laplace operator in two dimensions is the sum of the pure second derivatives, which is the trace of the Hessian:
\[
\Delta u = u_{xx} + u_{yy} = \operatorname{tr}(D^2u).
\]

A rotation of coordinates is given by an orthogonal matrix $R$ with $R^T R = I$. If we introduce new coordinates $(\xi,\eta)$ related to $(x,y)$ by
\[
\begin{pmatrix}
\xi\\
\eta
\end{pmatrix}
= R
\begin{pmatrix}
x\\
y
\end{pmatrix},
\]
and define $v(\xi,\eta) = u(x(\xi,\eta),y(\xi,\eta))$, the chain rule shows that the Hessian $D^2v$ in $(\xi,\eta)$-coordinates is obtained from $D^2u$ by the similarity transformation
\[
D^2 v = R^T (D^2 u) R.
\]
This is the standard behavior of a symmetric bilinear form under change of orthonormal basis. Taking traces and using the cyclic property of the trace gives
\[
\operatorname{tr}(D^2 v)
= \operatorname{tr}(R^T D^2 u\, R)
= \operatorname{tr}(D^2 u\, R R^T)
= \operatorname{tr}(D^2 u\, I)
= \operatorname{tr}(D^2 u).
\]
Thus
\[
v_{\xi\xi} + v_{\eta\eta}
= \operatorname{tr}(D^2 v)
= \operatorname{tr}(D^2 u)
= u_{xx} + u_{yy}.
\]
This calculation shows that \emph{Laplace’s equation has the same form in any rotated coordinate system}: if $u$ satisfies $u_{xx} + u_{yy} = 0$ in $(x,y)$, then $v$ satisfies $v_{\xi\xi} + v_{\eta\eta} = 0$ in $(\xi,\eta)$.

This rotational invariance is characteristic of the underlying geometry of the positive definite quadratic form $Q(\xi,\eta) = \xi^2 + \eta^2$: its level sets are concentric circles, which are themselves rotationally invariant. Elliptic equations associated with positive definite quadratic forms often inherit such invariances, and this is reflected in the isotropic, ``smoothing'' behavior of their solutions.

\medskip

\noindent\textbf{(c) Interior maximum and constancy (maximum principle).}
Now let $\Omega \subset \mathbb{R}^2$ be a bounded open set and suppose that $u \in C^2(\Omega) \cap C^0(\overline{\Omega})$ satisfies Laplace’s equation
\[
u_{xx} + u_{yy} = 0 \quad \text{in } \Omega.
\]
Assume that $u$ attains its maximum at an interior point $(x_0,y_0) \in \Omega$.

Because $u$ has a local maximum at $(x_0,y_0)$ and is twice continuously differentiable, the classical calculus characterization of extrema applies. First, the gradient must vanish:
\[
\nabla u(x_0,y_0) = (u_x(x_0,y_0), u_y(x_0,y_0)) = (0,0).
\]
Second, the Hessian matrix
\[
H := D^2u(x_0,y_0) =
\begin{pmatrix}
u_{xx}(x_0,y_0) & u_{xy}(x_0,y_0)\\
u_{xy}(x_0,y_0) & u_{yy}(x_0,y_0)
\end{pmatrix}
\]
must be \emph{negative semidefinite}. This means that for every vector $z \in \mathbb{R}^2$ we have
\[
z^T H z \leq 0.
\]
Equivalently, all eigenvalues of $H$ are less than or equal to zero.

On the other hand, Laplace’s equation at the point $(x_0,y_0)$ reads
\[
u_{xx}(x_0,y_0) + u_{yy}(x_0,y_0) = 0.
\]
The left-hand side is the trace of $H$:
\[
\operatorname{tr}(H) = u_{xx}(x_0,y_0) + u_{yy}(x_0,y_0) = 0.
\]
The eigenvalues of the symmetric matrix $H$ are real; denote them by $\lambda_1$ and $\lambda_2$. Since $H$ is negative semidefinite, we have
\[
\lambda_1 \leq 0, \quad \lambda_2 \leq 0.
\]
At the same time,
\[
\lambda_1 + \lambda_2 = \operatorname{tr}(H) = 0.
\]
The only way for two real numbers that are each less than or equal to zero to have sum zero is for both to be zero:
\[
\lambda_1 = 0, \quad \lambda_2 = 0.
\]
Hence all eigenvalues of $H$ vanish, so $H$ is the zero matrix:
\[
u_{xx}(x_0,y_0) = u_{yy}(x_0,y_0) = u_{xy}(x_0,y_0) = 0.
\]

One can now argue as follows. The vanishing of the Hessian at $(x_0,y_0)$ implies that, to second order, $u$ is completely flat near this point. More systematically, one can consider the maximum principle: standard proofs show that if a harmonic function attains an interior maximum, then it must be constant in the connected component of $\Omega$ containing that point. In an elementary setting, one may argue indirectly: if $u$ were not constant, then near $(x_0,y_0)$ we would be able to find points where $u$ is strictly less than its maximum value, contradicting a refined version of the local maximum characterization that uses Taylor’s theorem together with the vanishing of all first and second derivatives at $(x_0,y_0)$.

Therefore, under the given hypotheses, $u$ must be constant in $\Omega$. This is a special case of the \emph{strong maximum principle} for elliptic equations, here illustrated for the simplest elliptic operator, the Laplacian.

An important consequence is the uniqueness of solutions to the Dirichlet problem for Laplace’s equation: if two harmonic functions agree on the boundary $\partial\Omega$, then their difference is harmonic and attains both its maximum and minimum (namely zero) on the boundary, so by the above reasoning the difference must vanish throughout $\Omega$.

\medskip

\noindent\textbf{(d) Comparison with $u_{xx} - u_{yy} = 0$.}
Consider the equation
\[
u_{xx} - u_{yy} = 0.
\]
In the general form $A\,u_{xx} + 2B\,u_{xy} + C\,u_{yy} = 0$ we now have
\[
A = 1,\quad B = 0,\quad C = -1.
\]
The discriminant is
\[
B^2 - AC = 0^2 - (1)(-1) = 1 > 0.
\]
Hence this equation is of \emph{hyperbolic} type. The associated quadratic form is
\[
Q(\xi,\eta) = \xi^2 - \eta^2,
\]
which is indefinite: it takes both positive and negative values, and its level sets are hyperbolas, not circles. This contrasts sharply with the positive definite $\xi^2 + \eta^2$ associated with Laplace’s equation.

This algebraic difference reflects a qualitative difference in the behavior of solutions. Hyperbolic equations such as $u_{xx} - u_{yy} = 0$ (closely related to the one-dimensional wave equation) are naturally posed as \emph{initial value problems}, where one prescribes data along a noncharacteristic curve and the solution exhibits propagation along characteristic lines. Elliptic equations such as Laplace’s equation are naturally posed as \emph{boundary value problems} on bounded domains, where interior values are determined in a smooth and stable way by the boundary data, and maximum principles apply. Thus the classification via the discriminant $B^2 - AC$ and the associated quadratic form is not merely algebraic; it encodes the essential geometric and analytic behavior of solutions.

\medskip

In summary, this example shows how Laplace’s equation fits squarely into the elliptic class by way of the negative discriminant and positive definite quadratic form, how its rotational invariance aligns with the geometry of that form, and how ellipticity underpins fundamental qualitative properties such as the maximum principle and uniqueness for boundary value problems.
\end{solution}

% ===== Example 3: The Heat Equation and Parabolic Type (inquiry-based) =====
\begin{problem}[The Heat Equation and Parabolic Type]
The one-dimensional heat equation
\[
u_t = k\,u_{xx}, \qquad k>0,
\]
models the diffusion of temperature along a thin, insulated rod. Physically, temperature gradients drive heat flow, and the effect of time evolution is to smooth out these gradients. Mathematically, this equation is second order only in the spatial variable, yet it is still classified as a \emph{parabolic} equation. In this problem you will see how to fit the heat equation into the standard second-order framework and how its parabolic type reflects its smoothing, ``gradient-flow'' behavior.

We regard $u$ as a function of two independent variables, $x$ (space) and $t$ (time).

\smallskip

(a) The standard general form of a linear second-order PDE in two variables $x$ and $t$ is
\[
a(x,t)\,u_{xx} \;+\; 2b(x,t)\,u_{xt} \;+\; c(x,t)\,u_{tt} \;+\; \text{(lower-order terms)} \;=\; f(x,t).
\]
Rewrite the heat equation $u_t = k u_{xx}$ in this form by moving all terms to one side. Identify the coefficients $a$, $b$, and $c$ of $u_{xx}$, $u_{xt}$, and $u_{tt}$ respectively.

% Hint: Aim to write something of the form $a\,u_{xx} + 2b\,u_{xt} + c\,u_{tt} + d\,u_x + e\,u_t + f\,u = 0$.

\smallskip

(b) For a second-order PDE in two variables in the form
\[
a u_{xx} + 2 b u_{xt} + c u_{tt} + \dots = 0,
\]
the \emph{type} (elliptic, hyperbolic, parabolic) is determined by the discriminant $D = b^2 - ac$ of the quadratic form in the highest derivatives.

\begin{enumerate}
\item[(i)] State the classification in terms of $D$: what conditions on $D$ correspond to elliptic, hyperbolic, and parabolic equations?
\item[(ii)] Using your coefficients $a$, $b$, and $c$ from part (a), compute $D$ for the heat equation and classify it.
\end{enumerate}

\smallskip

(c) Another way to understand the type is to look at the $2\times 2$ matrix of coefficients of the second derivatives,
\[
A(x,t) \;=\;
\begin{pmatrix}
a(x,t) & b(x,t)\\[4pt]
b(x,t) & c(x,t)
\end{pmatrix}.
\]
For the heat equation, write down the constant matrix $A$ and find its eigenvalues.

\begin{enumerate}
\item[(i)] How many eigenvalues are positive, negative, and zero?
\item[(ii)] Explain how this eigenvalue pattern is consistent with calling the heat equation \emph{parabolic}.
\end{enumerate}

% Hint: For a $2\times 2$ real symmetric matrix, the signs of the eigenvalues can be deduced from the trace and determinant.

\smallskip

(d) To connect the parabolic classification with the \emph{smoothing} or \emph{gradient-flow} behavior, consider the heat equation on a finite rod $0 < x < L$ with homogeneous Dirichlet boundary conditions
\[
u(0,t) = 0, \qquad u(L,t)=0.
\]
Define the (squared) $L^2$-norm of $u$ at time $t$ by
\[
E(t) \;=\; \frac{1}{2} \int_0^L u(x,t)^2\,dx.
\]

\begin{enumerate}
\item[(i)] Differentiate $E(t)$ with respect to $t$ and use the heat equation to express $E'(t)$ in terms of $u$ and its spatial derivatives.
\item[(ii)] Use integration by parts and the boundary conditions to show that
\[
E'(t) \;=\; -k \int_0^L u_x(x,t)^2\,dx \;\le 0.
\]
\item[(iii)] Interpret this identity: what does it say about how the ``energy'' $E(t)$ changes in time, and why is this suggestive of a gradient-flow type evolution?
\end{enumerate}

% Hint: For (i), start from $E'(t) = \int_0^L u\,u_t\,dx$ and substitute $u_t = k u_{xx}$. For (ii), integrate $u\,u_{xx}$ by parts in $x$.

\smallskip

(e) {\bf Extensions and comparisons.}
\begin{enumerate}
\item[(i)] Consider now the one-dimensional \emph{wave equation}
\[
u_{tt} = c^2 u_{xx}.
\]
Rewrite it in the standard second-order form with independent variables $x$ and $t$, identify $a$, $b$, and $c$, and compute the discriminant $D = b^2 - ac$. How is the type different from that of the heat equation, and how does this contrast in type reflect the different physical behavior of waves versus diffusion?

% Hint: For the wave equation the highest-order derivatives are $u_{tt}$ and $u_{xx}$; both are second order.

\item[(ii)] Suppose we add a first-order advection term and consider the \emph{convection--diffusion} equation
\[
u_t + v\,u_x = k u_{xx}, \qquad v \in \mathbb{R}.
\]
What are the coefficients $a$, $b$, and $c$ of the second derivatives now, and what is the discriminant $D$? What does this tell you about the type of the convection--diffusion equation, and what does it suggest about the robustness of the classification with respect to lower-order terms?
\end{enumerate}

\end{problem}

% ===== Example 3: The Heat Equation and Parabolic Type (full solution) =====
\begin{problem}[The Heat Equation and Parabolic Type]
Consider the one-dimensional heat equation
\[
u_t = k\,u_{xx}, \qquad k>0,
\]
for $u(x,t)$ with independent variables $x$ (space) and $t$ (time).

\begin{enumerate}
\item[(a)] Rewrite this equation in the standard second-order form
\[
a\,u_{xx} + 2b\,u_{xt} + c\,u_{tt} + \text{(lower-order terms)} = 0,
\]
and identify $a$, $b$, and $c$.

\item[(b)] Recall that for such an equation the discriminant $D = b^2 - ac$ determines the type: elliptic if $D<0$, hyperbolic if $D>0$, and parabolic if $D=0$ (with $a$ and $c$ not both zero). Compute $D$ for the heat equation and classify it.

\item[(c)] Form the $2\times 2$ matrix of second-order coefficients
\[
A = \begin{pmatrix} a & b \\ b & c \end{pmatrix},
\]
compute its eigenvalues for the heat equation, and relate their signs to the parabolic classification.

\item[(d)] On the finite interval $0<x<L$ with homogeneous Dirichlet boundary conditions $u(0,t)=u(L,t)=0$, define
\[
E(t) = \frac{1}{2}\int_0^L u(x,t)^2\,dx.
\]
Show that
\[
E'(t) = -k \int_0^L u_x(x,t)^2\,dx \le 0,
\]
and briefly explain how this identity reflects the diffusive, smoothing, or ``gradient-flow'' character associated with parabolic equations.

\item[(e)] For comparison, classify the wave equation $u_{tt} = c^2 u_{xx}$ by the same method and state how its type and energy behavior differ from those of the heat equation.
\end{enumerate}
\end{problem}

\begin{solution}
We treat $u$ as a function of two independent variables, $x$ and $t$. The classification of a linear second-order PDE in two variables is based on the quadratic form appearing in its highest-order derivatives.

\medskip

\noindent{\bf (a) Standard second-order form.}
We start from the heat equation
\[
u_t = k\,u_{xx}.
\]
To fit this into the general pattern
\[
a\,u_{xx} + 2b\,u_{xt} + c\,u_{tt} + \text{(lower-order terms)} = 0,
\]
we move all terms to one side:
\[
k\,u_{xx} - u_t = 0.
\]
There are no $u_{xt}$ or $u_{tt}$ terms, so we read off
\[
a = k, \qquad b = 0, \qquad c = 0.
\]
The term $-u_t$ is first order, so it is part of the lower-order terms and does not affect the classification.

\medskip

\noindent{\bf (b) Discriminant and type.}
For a second-order PDE in two variables with highest-order part
\[
a u_{xx} + 2 b u_{xt} + c u_{tt},
\]
the discriminant is defined as
\[
D = b^2 - a c.
\]
The standard (two-dimensional) classification is:
\begin{itemize}
\item elliptic if $D < 0$,
\item hyperbolic if $D > 0$,
\item parabolic if $D = 0$ and $(a,c)\ne (0,0)$.
\end{itemize}

For the heat equation we found $a = k > 0$, $b = 0$, and $c = 0$. Hence
\[
D = b^2 - ac = 0^2 - k\cdot 0 = 0.
\]
Since $a\neq 0$ and $c=0$, the equation is of \emph{parabolic} type.

Thus, even though the equation is only first order in time, when one regards $t$ as a second independent variable alongside $x$, its principal (second-order) part has discriminant zero, the hallmark of parabolic equations.

\medskip

\noindent{\bf (c) Matrix of second-order coefficients and eigenvalues.}
The highest-order coefficients can be assembled into the symmetric matrix
\[
A = \begin{pmatrix}
a & b\\
b & c
\end{pmatrix}.
\]
For the heat equation this becomes
\[
A = \begin{pmatrix}
k & 0\\
0 & 0
\end{pmatrix}.
\]
The eigenvalues of this diagonal matrix are simply the diagonal entries:
\[
\lambda_1 = k > 0, \qquad \lambda_2 = 0.
\]

Thus the principal symbol has one positive eigenvalue and one zero eigenvalue. In the purely elliptic case (for example, Laplace's equation $u_{xx} + u_{yy}=0$), all eigenvalues are of the same sign and nonzero; in the hyperbolic case (for example, the wave equation $u_{tt} - c^2 u_{xx}=0$), the principal matrix has eigenvalues of opposite signs. The presence of a zero eigenvalue but no negative eigenvalues is consistent with parabolic type: there is diffusion in the spatial direction (positive eigenvalue) but no second-order dynamics in time (zero eigenvalue). This degeneracy (one zero eigenvalue) is exactly what the discriminant $D=0$ detects.

\medskip

\noindent{\bf (d) Energy decay and gradient-flow behavior.}
Now consider the heat equation on $0<x<L$ with homogeneous Dirichlet boundary conditions
\[
u(0,t) = 0, \qquad u(L,t) = 0.
\]
Define the energy (really the squared $L^2$-norm) by
\[
E(t) = \frac{1}{2}\int_0^L u(x,t)^2\,dx.
\]
We compute $E'(t)$ and use the PDE to see how $E$ changes in time.

Differentiating under the integral sign yields
\[
E'(t) = \frac{1}{2}\int_0^L 2u(x,t)\,u_t(x,t)\,dx = \int_0^L u\,u_t\,dx.
\]
Substituting the heat equation $u_t = k\,u_{xx}$ gives
\[
E'(t) = k \int_0^L u\,u_{xx}\,dx.
\]

We now integrate by parts in $x$. Using the standard formula
\[
\int_0^L u\,u_{xx}\,dx = \bigl[u\,u_x\bigr]_{0}^{L} - \int_0^L u_x^2\,dx,
\]
we obtain
\[
E'(t) = k \bigl[u\,u_x\bigr]_{0}^{L} - k \int_0^L u_x^2\,dx.
\]
The boundary conditions $u(0,t) = u(L,t) = 0$ imply that the boundary term vanishes:
\[
\bigl[u\,u_x\bigr]_{0}^{L} = u(L,t)u_x(L,t) - u(0,t)u_x(0,t) = 0,
\]
so we are left with
\[
E'(t) = -k \int_0^L u_x(x,t)^2\,dx.
\]
Since $k>0$ and $u_x^2 \ge 0$, it follows that
\[
E'(t) \le 0.
\]

This calculation has two important interpretations:

\begin{itemize}
\item The quantity $E(t)$ is nonincreasing in time. The solution cannot gain ``energy'' in the $L^2$ sense; instead, the size of $u$ tends to decrease.

\item The rate of decay is proportional to the integral of $u_x^2$, which measures the size of spatial gradients. Large gradients correspond to a faster decrease of $E(t)$. In this sense, the heat flow is trying to reduce the size of its gradient, smoothing out variations.
\end{itemize}

This behavior is characteristic of a \emph{gradient flow} in an infinite-dimensional space: one can think of the heat equation as a steepest descent flow for an energy functional (for example, related to the Dirichlet energy $\int |u_x|^2$). The parabolic classification captures precisely this smoothing, dissipative character.

\medskip

\noindent{\bf (e) Comparison with the wave equation.}
Consider now the one-dimensional wave equation
\[
u_{tt} = c^2 u_{xx}
\]
for some $c>0$. Moving all terms to one side gives
\[
u_{tt} - c^2 u_{xx} = 0,
\]
which we rewrite in the standard form
\[
a\,u_{xx} + 2b\,u_{xt} + c\,u_{tt} = 0.
\]
Here there is no mixed derivative, so $b=0$. The coefficient of $u_{xx}$ is $-c^2$, and the coefficient of $u_{tt}$ is $1$. Thus
\[
a = -c^2, \qquad b = 0, \qquad c = 1.
\]
The discriminant is
\[
D = b^2 - a c = 0^2 - (-c^2)\cdot 1 = c^2 > 0.
\]
Hence the wave equation is \emph{hyperbolic}. The associated coefficient matrix is
\[
A = \begin{pmatrix}
-\,c^2 & 0\\[4pt]
0 & 1
\end{pmatrix},
\]
which has one positive and one negative eigenvalue. This sign pattern (indefinite quadratic form) contrasts with the heat equation’s one-positive, one-zero pattern.

On a finite interval with suitable boundary conditions, the wave equation has an associated conserved energy, involving both $u_t^2$ and $u_x^2$, which remains constant in time rather than decaying. This conservation of energy reflects the oscillatory, propagation-dominated nature of hyperbolic equations, in sharp contrast with the dissipative, smoothing behavior of the parabolic heat equation.

\medskip

In summary, the heat equation $u_t = k u_{xx}$, when written in standard second-order form in $(x,t)$, has discriminant $D = 0$ and a principal coefficient matrix with one positive and one zero eigenvalue. This places it in the parabolic class and aligns with its physical and analytical behavior: solutions tend to smooth out and lose energy over time, akin to an infinite-dimensional gradient flow. This example illustrates how the abstract classification of linear second-order PDEs by their principal part encodes essential qualitative features of their solutions.
\end{solution}

% ===== Example 4: The Wave Equation and Hyperbolic Type (inquiry-based) =====
\begin{problem}[The Wave Equation and Hyperbolic Type]
The one-dimensional wave equation
\[
u_{tt} = c^{2} u_{xx}, \qquad c>0,
\]
models, for instance, small vertical vibrations of a taut string and sound waves in a thin tube. In contrast to the heat equation, which smooths out disturbances, wave propagation transports oscillations along straight rays with a finite speed $c$. In this problem we use the general classification scheme for linear second-order partial differential equations to understand what it means, mathematically, for the wave equation to be of \emph{hyperbolic} type, and how this is reflected in its characteristic curves.

Recall that a linear second-order PDE in two variables $x$ and $y$ is often written in the form
\[
A(x,y)\,u_{xx} + 2 B(x,y)\,u_{xy} + C(x,y)\,u_{yy} + \text{(lower order terms)} = 0,
\]
and is called \emph{hyperbolic} at a point if $B^2 - AC > 0$ there.

\smallskip

(a) Rewrite the wave equation $u_{tt} = c^2 u_{xx}$ in the standard form
\[
A(x,t)\,u_{xx} + 2 B(x,t)\,u_{xt} + C(x,t)\,u_{tt} + \text{(lower order terms)} = 0,
\]
using $(x,t)$ as the independent variables. Identify explicitly the coefficients $A(x,t)$, $B(x,t)$, and $C(x,t)$.

\smallskip

(b) Compute the discriminant $B^2 - AC$ for the wave equation, and use it to classify the equation as elliptic, parabolic, or hyperbolic. State clearly which sign of the discriminant corresponds to each type.

\smallskip

(c) For a general equation
\[
A u_{xx} + 2 B u_{xt} + C u_{tt} + \cdots = 0,
\]
the characteristic curves in the $(x,t)$-plane are defined (locally) by solving the ordinary differential equation
\[
A \,(dt)^2 - 2B \,dx\,dt + C \,(dx)^2 = 0,
\]
or, equivalently,
\[
A \left(\frac{dt}{dx}\right)^2 - 2B \left(\frac{dt}{dx}\right) + C = 0.
\]
Write down this quadratic equation for $\dfrac{dt}{dx}$ specifically for the wave equation, and solve for $\dfrac{dt}{dx}$. What are the two families of characteristic curves in the $(x,t)$-plane?

\emph{Hint:} In part (c), remember that for the wave equation you found $A$, $B$, and $C$ in part (a). You should obtain two constant slopes, corresponding to straight lines.

\smallskip

(d) A standard way to simplify a hyperbolic equation is to introduce new variables $\xi$ and $\eta$ that are constant along the characteristic curves. For the wave equation, consider the change of variables
\[
\xi = x - c t, \qquad \eta = x + c t.
\]
(i) Verify that $\xi$ and $\eta$ are indeed constant along the characteristic lines you found in part (c). That is, check that along one family of characteristic curves $\xi$ is constant and along the other family $\eta$ is constant.

(ii) Using the chain rule, express $u_{x}$ and $u_{t}$ in terms of $u_{\xi}$ and $u_{\eta}$, and then compute $u_{xx}$ and $u_{tt}$ in terms of $u_{\xi\xi}$, $u_{\xi\eta}$, and $u_{\eta\eta}$. Substitute these into the wave equation to show that, in the $(\xi,\eta)$ variables, the equation takes the simpler \emph{canonical form}
\[
u_{\xi\eta} = 0.
\]

\emph{Hint:} First write $u(x,t) = U(\xi,\eta)$, where $U$ is $u$ expressed in the new variables. Then use
\[
u_x = U_{\xi}\,\xi_x + U_{\eta}\,\eta_x, \qquad
u_t = U_{\xi}\,\xi_t + U_{\eta}\,\eta_t,
\]
and differentiate once more to find $u_{xx}$ and $u_{tt}$. Be systematic and keep track of coefficients of $U_{\xi\xi}$, $U_{\xi\eta}$, and $U_{\eta\eta}$.

\smallskip

(e) \textbf{What if / extensions.}
\begin{enumerate}
\item[(i)] Consider instead the diffusion equation $u_t = k u_{xx}$, with $k>0$. How could you rewrite this in the form
\[
A u_{xx} + 2 B u_{xt} + C u_{tt} + \cdots = 0
\]
by moving all terms to one side? What are $A$, $B$, and $C$ in this case, and what is the discriminant $B^2 - AC$? How is this type (elliptic, parabolic, or hyperbolic) different from the wave equation?

\item[(ii)] Suppose we change the sign in the wave equation and consider
\[
u_{tt} + c^2 u_{xx} = 0.
\]
Classify this equation by finding $A$, $B$, $C$, and $B^2 - AC$. Is it hyperbolic, elliptic, or parabolic? Briefly discuss how you expect the qualitative behavior of solutions to differ from the original wave equation.
\end{enumerate}

\end{problem}

% ===== Example 4: The Wave Equation and Hyperbolic Type (full solution) =====
\begin{problem}[The Wave Equation and Hyperbolic Type]
Consider the one-dimensional wave equation
\[
u_{tt} = c^{2} u_{xx}, \qquad c>0,
\]
with independent variables $(x,t)$.

\begin{enumerate}
\item[(a)] Rewrite this equation in the general second-order form
\[
A(x,t)\,u_{xx} + 2 B(x,t)\,u_{xt} + C(x,t)\,u_{tt} + \text{(lower order terms)} = 0,
\]
identify $A$, $B$, and $C$, and compute the discriminant $B^2 - AC$. Classify the equation as elliptic, parabolic, or hyperbolic.

\item[(b)] Using the general characteristic equation
\[
A \left(\frac{dt}{dx}\right)^2 - 2B \left(\frac{dt}{dx}\right) + C = 0,
\]
find the characteristic curves of the wave equation in the $(x,t)$-plane.

\item[(c)] Introduce the characteristic coordinates
\[
\xi = x - c t, \qquad \eta = x + c t,
\]
and use the chain rule to transform the wave equation into canonical form in the $(\xi,\eta)$ variables. Show that the equation becomes
\[
u_{\xi\eta} = 0.
\]
Briefly explain how this illustrates the hyperbolic nature of the wave equation.
\end{enumerate}
\end{problem}

\begin{solution}
We regard the wave equation as a second-order partial differential equation in the independent variables $x$ (space) and $t$ (time).

\medskip

\textbf{(a) Classification via the discriminant.}
We start from
\[
u_{tt} = c^{2} u_{xx}.
\]
To match the standard classification form
\[
A(x,t)\,u_{xx} + 2 B(x,t)\,u_{xt} + C(x,t)\,u_{tt} + \text{(lower order terms)} = 0,
\]
we move all terms to one side:
\[
u_{tt} - c^{2} u_{xx} = 0.
\]
Comparing with the standard form, we read off
\[
A(x,t) = -c^{2}, \qquad B(x,t) = 0, \qquad C(x,t) = 1.
\]
These coefficients are constant and independent of $(x,t)$.

The discriminant is
\[
B^{2} - A C = 0^{2} - (-c^{2})(1) = c^{2} > 0.
\]
By the usual convention,
\[
\begin{cases}
B^{2} - A C > 0 &\Rightarrow \text{hyperbolic},\\[2pt]
B^{2} - A C = 0 &\Rightarrow \text{parabolic},\\[2pt]
B^{2} - A C < 0 &\Rightarrow \text{elliptic}.
\end{cases}
\]
Since $c^{2} > 0$, the one-dimensional wave equation is of \emph{hyperbolic} type everywhere in the $(x,t)$-plane. This algebraic characterization is the starting point for understanding its propagation of waves along characteristic curves.

\medskip

\textbf{(b) Characteristic curves.}
For a second-order equation
\[
A u_{xx} + 2B u_{xt} + C u_{tt} + \cdots = 0,
\]
the characteristic curves in the $(x,t)$-plane are determined (locally) by the quadratic equation
\[
A \left(\frac{dt}{dx}\right)^2 - 2B \left(\frac{dt}{dx}\right) + C = 0.
\]
For the wave equation, we substitute $A = -c^{2}$, $B = 0$, and $C=1$ to obtain
\[
(-c^{2}) \left(\frac{dt}{dx}\right)^2 + 1 = 0,
\]
or, equivalently,
\[
\left(\frac{dt}{dx}\right)^2 = \frac{1}{c^{2}}.
\]
Taking square roots yields two constant slopes:
\[
\frac{dt}{dx} = \frac{1}{c} \quad\text{or}\quad \frac{dt}{dx} = -\frac{1}{c}.
\]
Integrating, we find the families of characteristic curves:
\[
t = \frac{1}{c} x + \text{constant}
\quad\Longleftrightarrow\quad
x - c t = \text{constant},
\]
and
\[
t = -\frac{1}{c} x + \text{constant}
\quad\Longleftrightarrow\quad
x + c t = \text{constant}.
\]
Thus, there are two distinct families of straight lines in the $(x,t)$-plane along which information propagates. These lines have slopes $\pm 1/c$ and represent signals moving to the right and to the left with speed $c$. The fact that the characteristic equation has two distinct real roots (and hence two distinct real characteristic directions) is another hallmark of hyperbolic type.

\medskip

\textbf{(c) Transformation to canonical form.}
A central idea in the classification of second-order PDEs is that elliptic, parabolic, and hyperbolic equations can often be simplified, via an appropriate change of variables, to a canonical form that reveals their essential behavior. For hyperbolic equations in two variables, the canonical second-order term is typically a mixed derivative of the form $u_{\xi\eta}$.

We introduce new independent variables $\xi$ and $\eta$ by
\[
\xi = x - c t, \qquad \eta = x + c t.
\]
From part (b), we recognize that $\xi$ is constant along the characteristic lines $x - ct = \text{constant}$ (right-moving waves), and $\eta$ is constant along the lines $x + ct = \text{constant}$ (left-moving waves). Thus these are \emph{characteristic coordinates}.

Let us write $u(x,t)$ as $U(\xi,\eta)$, where
\[
U(\xi,\eta) = u\bigl(x(\xi,\eta), t(\xi,\eta)\bigr).
\]
We compute the necessary derivatives using the chain rule. First, we compute the partial derivatives of $\xi$ and $\eta$:
\[
\xi_x = 1, \quad \xi_t = -c, \qquad
\eta_x = 1, \quad \eta_t = c.
\]
Then
\[
u_x = U_{\xi}\,\xi_x + U_{\eta}\,\eta_x
= U_{\xi} + U_{\eta},
\]
and
\[
u_t = U_{\xi}\,\xi_t + U_{\eta}\,\eta_t
= -c U_{\xi} + c U_{\eta}
= c(-U_{\xi} + U_{\eta}).
\]

We now differentiate once more with respect to $x$ and $t$. For $u_{xx}$ we have
\[
u_{xx} = \frac{\partial}{\partial x} (u_x)
= \frac{\partial}{\partial x}(U_{\xi} + U_{\eta}).
\]
Using the chain rule again,
\[
\frac{\partial}{\partial x}
= \xi_x \frac{\partial}{\partial \xi} + \eta_x \frac{\partial}{\partial \eta}
= \frac{\partial}{\partial \xi} + \frac{\partial}{\partial \eta},
\]
so
\[
u_{xx}
= (U_{\xi\xi} + U_{\xi\eta}) + (U_{\eta\xi} + U_{\eta\eta})
= U_{\xi\xi} + 2 U_{\xi\eta} + U_{\eta\eta},
\]
since $U_{\xi\eta} = U_{\eta\xi}$ by equality of mixed partial derivatives.

Next we find $u_{tt}$:
\[
u_{tt} = \frac{\partial}{\partial t} (u_t)
= \frac{\partial}{\partial t}\bigl(c(-U_{\xi} + U_{\eta})\bigr)
= c\bigl(-U_{\xi t} + U_{\eta t}\bigr).
\]
The $t$-derivative operator is
\[
\frac{\partial}{\partial t}
= \xi_t \frac{\partial}{\partial \xi} + \eta_t \frac{\partial}{\partial \eta}
= -c \frac{\partial}{\partial \xi} + c \frac{\partial}{\partial \eta}.
\]
Therefore,
\[
U_{\xi t}
= \left(-c \frac{\partial}{\partial \xi} + c \frac{\partial}{\partial \eta}\right)U_{\xi}
= -c U_{\xi\xi} + c U_{\xi\eta},
\]
and
\[
U_{\eta t}
= \left(-c \frac{\partial}{\partial \xi} + c \frac{\partial}{\partial \eta}\right)U_{\eta}
= -c U_{\eta\xi} + c U_{\eta\eta}
= -c U_{\xi\eta} + c U_{\eta\eta}.
\]
Thus
\[
u_{tt}
= c\bigl(-U_{\xi t} + U_{\eta t}\bigr)
= c\Bigl(-(-c U_{\xi\xi} + c U_{\xi\eta}) + (-c U_{\xi\eta} + c U_{\eta\eta})\Bigr).
\]
Simplifying inside the parentheses,
\[
-u_{\xi t} + u_{\eta t}
= c U_{\xi\xi} - c U_{\xi\eta} - c U_{\xi\eta} + c U_{\eta\eta}
= c U_{\xi\xi} - 2c U_{\xi\eta} + c U_{\eta\eta}.
\]
Thus
\[
u_{tt}
= c\bigl(c U_{\xi\xi} - 2c U_{\xi\eta} + c U_{\eta\eta}\bigr)
= c^{2}\bigl(U_{\xi\xi} - 2 U_{\xi\eta} + U_{\eta\eta}\bigr).
\]

We now substitute $u_{xx}$ and $u_{tt}$ into the wave equation $u_{tt} = c^{2} u_{xx}$:
\[
c^{2}\bigl(U_{\xi\xi} - 2 U_{\xi\eta} + U_{\eta\eta}\bigr)
= c^{2}\bigl(U_{\xi\xi} + 2 U_{\xi\eta} + U_{\eta\eta}\bigr).
\]
We may divide both sides by $c^{2} > 0$ and subtract the left-hand side from the right-hand side:
\[
0
= \bigl(U_{\xi\xi} + 2 U_{\xi\eta} + U_{\eta\eta}\bigr)
- \bigl(U_{\xi\xi} - 2 U_{\xi\eta} + U_{\eta\eta}\bigr)
= 4 U_{\xi\eta}.
\]
Therefore $U_{\xi\eta} = 0$, which we can write, reverting to the notation $u$ for the dependent variable in $(\xi,\eta)$ coordinates, as
\[
u_{\xi\eta} = 0.
\]
This is precisely the canonical form for a hyperbolic equation in two variables: the second-order part consists of a single mixed derivative.

From this canonical form, we can immediately integrate:
\[
u_{\xi\eta} = 0
\quad\Longrightarrow\quad
u(\xi,\eta) = F(\xi) + G(\eta),
\]
for arbitrary functions $F$ and $G$. Translating back to the original variables,
\[
u(x,t) = F(x - c t) + G(x + c t),
\]
which is the well-known d'Alembert solution. This exhibits the two families of traveling waves moving along the characteristic lines $x - c t = \text{constant}$ and $x + c t = \text{constant}$.

\medskip

\textbf{Connection with the classification theory.}
This example illustrates the main ideas of the classification of linear second-order PDEs in two variables:

\begin{itemize}
\item The sign of the discriminant $B^{2} - A C$ distinguishes elliptic, parabolic, and hyperbolic equations. For the wave equation, $B^{2} - A C > 0$, so it is hyperbolic.

\item For hyperbolic equations, the characteristic equation has two distinct real roots, leading to two families of real characteristic curves. Solutions typically propagate along these curves with finite speed.

\item A suitable change of variables, chosen so that the new coordinates are constant along characteristics, transforms the equation into a canonical form involving a mixed derivative $u_{\xi\eta}$. For the wave equation, this leads directly to the decomposition into right-moving and left-moving waves.
\end{itemize}

Thus, both the algebraic discriminant and the geometric picture of characteristics consistently reveal the hyperbolic nature of the one-dimensional wave equation and its role in modeling wave propagation.
\end{solution}

% ===== Example 5: Change of Variables and Canonical Forms (inquiry-based) =====
\begin{problem}[Change of Variables and Canonical Forms]
In this problem we explore how a linear change of variables can simplify a second-order partial differential equation with a mixed derivative term. The guiding idea is that the second-order (or \emph{principal}) part of such a PDE behaves like a quadratic form, just as in the study of conic sections. By choosing coordinates aligned with the eigenvectors of this quadratic form, we can eliminate the mixed derivative and obtain a simpler \emph{canonical form}. This makes it much easier to recognize whether the equation is elliptic, hyperbolic, or parabolic, and to compare it with standard model equations.

Consider the linear second-order PDE
\[
u_{xx} + 4u_{xy} + u_{yy} \;=\; 0,
\]
where $u = u(x,y)$ is an unknown function of two variables.

\smallskip

(a) Recall that the general second-order part of a PDE in two variables can be written as
\[
A u_{xx} + 2B u_{xy} + C u_{yy}.
\]
For the given PDE, identify the coefficients $A$, $B$, and $C$. Compute the discriminant
\[
D = B^2 - AC
\]
and use it to classify the PDE as elliptic, hyperbolic, or parabolic. Explain your reasoning in a sentence or two.

\medskip

(b) Associate to the second-order part of the PDE the symmetric matrix
\[
Q = \begin{pmatrix} A & B \\[4pt] B & C \end{pmatrix}.
\]
For our equation, write down this matrix explicitly. Then recall (or verify) that the quadratic form
\[
A x^2 + 2B x y + C y^2
\]
can be written compactly as $\begin{pmatrix} x & y \end{pmatrix} Q \begin{pmatrix} x \\ y \end{pmatrix}$. 

How is the classification of the PDE related to the signs of the eigenvalues of $Q$? How is this analogous to the classification of conic sections given by a quadratic form in $x$ and $y$?

\medskip

(c) Compute the eigenvalues and eigenvectors of the matrix
\[
Q = \begin{pmatrix} 1 & 2 \\[4pt] 2 & 1 \end{pmatrix}.
\]
Normalize your eigenvectors to obtain an orthonormal basis of $\mathbb{R}^2$.

Hint: First find the characteristic polynomial and its roots. Notice that the vectors $(1,1)$ and $(1,-1)$ are natural candidates to test as eigenvectors.

\medskip

(d) Let $(\xi,\eta)$ be the new coordinates aligned with the orthonormal eigenbasis you found in part (c). Concretely, define
\[
\begin{pmatrix} \xi \\[4pt] \eta \end{pmatrix}
= P^{T} \begin{pmatrix} x \\[4pt] y \end{pmatrix},
\]
where $P$ is the $2\times 2$ orthogonal matrix whose columns are the normalized eigenvectors of $Q$. 

(i) Write down the change of variables $(x,y) \mapsto (\xi,\eta)$ explicitly.

(ii) Using the viewpoint that the principal part of the PDE transforms like the quadratic form $Q$, argue that, in the new variables $(\xi,\eta)$, the PDE has no mixed derivative term. In other words, its principal part is of the form
\[
\lambda_1\, u_{\xi\xi} + \lambda_2\, u_{\eta\eta},
\]
where $\lambda_1$ and $\lambda_2$ are the eigenvalues of $Q$ that you computed in part (c).

(iii) Conclude that, in $(\xi,\eta)$–coordinates, the PDE takes the canonical form
\[
3\,u_{\xi\xi} - u_{\eta\eta} = 0.
\]
Explain briefly why this canonical form makes it obvious that the equation is hyperbolic.

Hint: You do not need to compute $u_{xx}$, $u_{xy}$, and $u_{yy}$ explicitly in terms of $u_{\xi\xi}$, $u_{\xi\eta}$, and $u_{\eta\eta}$. Instead, use the fact that $P$ diagonalizes $Q$ by $P^{T} Q P = \operatorname{diag}(\lambda_1,\lambda_2)$.

\medskip

(e) Extensions and “what if” questions.

(i) Consider the PDE
\[
u_{xx} + 2u_{xy} + u_{yy} = 0.
\]
Repeat the classification step (part (a)) for this equation. What is the discriminant $D$ and what is the type of the PDE? Without full details, what canonical form do you expect after a suitable linear change of variables?

(ii) Suppose now that the matrix $Q$ associated to the principal part of a PDE has one positive and one zero eigenvalue. What type of PDE would you expect (elliptic, hyperbolic, or parabolic)? What sort of canonical form do you anticipate in suitable coordinates?

Hint: Connect your answers to the sign patterns of eigenvalues for quadratic forms and to the model equations
\[
u_{xx} + u_{yy} = 0 \quad (\text{elliptic}), \qquad
u_{xx} - u_{yy} = 0 \quad (\text{hyperbolic}), \qquad
u_{xx} = 0 \quad (\text{parabolic}).
\]
\end{problem}

% ===== Example 5: Change of Variables and Canonical Forms (full solution) =====
\begin{problem}[Change of Variables and Canonical Forms]
Consider the linear second-order PDE
\[
u_{xx} + 4u_{xy} + u_{yy} = 0,
\]
where $u = u(x,y)$.

\begin{enumerate}
\item Write the principal part in the standard form $A u_{xx} + 2B u_{xy} + C u_{yy}$, identify $A,B,C$, and classify the PDE using the discriminant $D = B^2 - AC$.
\item Form the symmetric matrix
\[
Q = \begin{pmatrix} A & B \\ B & C \end{pmatrix},
\]
compute its eigenvalues and an orthonormal eigenbasis.
\item Using a linear change of variables $(x,y) \mapsto (\xi,\eta)$ corresponding to this orthonormal eigenbasis, transform the PDE into a canonical form without mixed derivatives. Write the resulting PDE explicitly in the new variables and state its type.
\end{enumerate}
\end{problem}

\begin{solution}
We are asked to classify a second-order PDE with a mixed derivative term and then to find a linear change of variables that removes this mixed term. The key idea is to interpret the second-order part of the PDE as a quadratic form determined by a symmetric matrix. Diagonalizing this matrix by an orthogonal change of variables leads to a canonical form.

\medskip

\noindent\textbf{(1) Coefficients and classification.}
The principal (second-order) part of the given PDE is
\[
u_{xx} + 4u_{xy} + u_{yy}.
\]
We match this with the general expression
\[
A u_{xx} + 2B u_{xy} + C u_{yy}.
\]
Comparing coefficients, we obtain
\[
A = 1, \qquad 2B = 4 \;\Rightarrow\; B = 2, \qquad C = 1.
\]
The discriminant is
\[
D = B^2 - AC = 2^2 - (1)(1) = 4 - 1 = 3 > 0.
\]
Since $D>0$, the PDE is \emph{hyperbolic} at every point (for this constant-coefficient equation, the classification is global). This is completely analogous to the classification of conic sections: $D>0$ corresponds to a hyperbola in the quadratic-form setting.

\medskip

\noindent\textbf{(2) Matrix form and eigenvalues.}
We now form the symmetric matrix $Q$ associated with the principal part:
\[
Q = \begin{pmatrix} A & B \\[4pt] B & C \end{pmatrix}
= \begin{pmatrix} 1 & 2 \\[4pt] 2 & 1 \end{pmatrix}.
\]
This matrix plays the same role as the matrix of a quadratic form
\[
A x^2 + 2B x y + C y^2
= \begin{pmatrix} x & y \end{pmatrix}
\begin{pmatrix} 1 & 2 \\[2pt] 2 & 1 \end{pmatrix}
\begin{pmatrix} x \\[2pt] y \end{pmatrix}.
\]
The classification in terms of $D$ can also be seen from the eigenvalues of $Q$: a hyperbolic equation corresponds to one positive and one negative eigenvalue; an elliptic equation corresponds to two eigenvalues of the same sign; and a parabolic equation corresponds to one nonzero eigenvalue and one zero eigenvalue.

We now compute the eigenvalues of $Q$. The characteristic polynomial is
\[
\det(Q - \lambda I)
= \det\begin{pmatrix} 1 - \lambda & 2 \\ 2 & 1 - \lambda \end{pmatrix}
= (1 - \lambda)^2 - 4.
\]
Thus
\[
(1 - \lambda)^2 - 4 = 0
\quad\Longrightarrow\quad
(1 - \lambda) = \pm 2,
\]
so the eigenvalues are
\[
\lambda_1 = 1 + 2 = 3, \qquad
\lambda_2 = 1 - 2 = -1.
\]
As expected for a hyperbolic equation, one eigenvalue is positive and the other is negative.

Next, we find corresponding eigenvectors.

For $\lambda_1 = 3$, we solve $(Q - 3I)v = 0$:
\[
\begin{pmatrix} 1 - 3 & 2 \\ 2 & 1 - 3 \end{pmatrix}
= \begin{pmatrix} -2 & 2 \\ 2 & -2 \end{pmatrix}.
\]
The equation $-2 v_1 + 2 v_2 = 0$ implies $v_1 = v_2$. A convenient eigenvector is
\[
v^{(1)} = \begin{pmatrix} 1 \\ 1 \end{pmatrix}.
\]
For $\lambda_2 = -1$, we solve $(Q + I)v = 0$:
\[
\begin{pmatrix} 1 + 1 & 2 \\ 2 & 1 + 1 \end{pmatrix}
= \begin{pmatrix} 2 & 2 \\ 2 & 2 \end{pmatrix}.
\]
The equation $2 v_1 + 2 v_2 = 0$ implies $v_1 = -v_2$. A convenient eigenvector is
\[
v^{(2)} = \begin{pmatrix} 1 \\ -1 \end{pmatrix}.
\]

We now normalize these vectors to obtain an orthonormal basis of $\mathbb{R}^2$. Each of $v^{(1)}$ and $v^{(2)}$ has length $\sqrt{1^2+1^2} = \sqrt{2}$, so we take
\[
e_1 = \frac{1}{\sqrt{2}}\begin{pmatrix} 1 \\ 1 \end{pmatrix},
\qquad
e_2 = \frac{1}{\sqrt{2}}\begin{pmatrix} 1 \\ -1 \end{pmatrix}.
\]
The vectors $e_1$ and $e_2$ form an orthonormal eigenbasis, with
\[
Q e_1 = 3 e_1, \qquad Q e_2 = -1 \cdot e_2.
\]

Let $P$ be the orthogonal matrix whose columns are $e_1$ and $e_2$:
\[
P = \begin{pmatrix} \dfrac{1}{\sqrt{2}} & \dfrac{1}{\sqrt{2}} \\[6pt]
                    \dfrac{1}{\sqrt{2}} & -\dfrac{1}{\sqrt{2}} \end{pmatrix}.
\]
Then
\[
P^T Q P = 
\begin{pmatrix} 3 & 0 \\ 0 & -1 \end{pmatrix}.
\]

\medskip

\noindent\textbf{(3) Linear change of variables and canonical form.}
The crucial idea is that the principal part of the PDE transforms under a linear change of variables in exactly the same way as the associated quadratic form. If we introduce new variables $(\xi,\eta)$ that are aligned with the orthonormal eigenbasis, then the mixed term disappears and the matrix $Q$ becomes diagonal.

Define new coordinates $(\xi,\eta)$ by
\[
\begin{pmatrix} \xi \\[4pt] \eta \end{pmatrix}
= P^{T} \begin{pmatrix} x \\[4pt] y \end{pmatrix}
= \begin{pmatrix}
\dfrac{1}{\sqrt{2}} & \dfrac{1}{\sqrt{2}} \\[6pt]
\dfrac{1}{\sqrt{2}} & -\dfrac{1}{\sqrt{2}}
\end{pmatrix}
\begin{pmatrix} x \\ y \end{pmatrix}.
\]
Thus the change of variables is
\[
\xi = \frac{x + y}{\sqrt{2}},
\qquad
\eta = \frac{x - y}{\sqrt{2}}.
\]
Geometrically, this is a rotation of the coordinate axes by $45^\circ$, followed by a uniform scaling that preserves orthonormality.

Conversely, we can solve for $x$ and $y$ in terms of $\xi$ and $\eta$:
\[
x = \frac{\xi + \eta}{\sqrt{2}},
\qquad
y = \frac{\xi - \eta}{\sqrt{2}}.
\]

Under this linear change of variables, the matrix $Q$ transforms to $P^{T} Q P = \operatorname{diag}(3,-1)$. On the level of the PDE, this means that the principal part
\[
u_{xx} + 4u_{xy} + u_{yy}
\]
in $(x,y)$–coordinates becomes
\[
3\,u_{\xi\xi} - u_{\eta\eta}
\]
in $(\xi,\eta)$–coordinates, with no mixed derivative term $u_{\xi\eta}$.

Therefore, in the new variables $(\xi,\eta)$, the PDE takes the canonical form
\[
3\,u_{\xi\xi} - u_{\eta\eta} = 0.
\]
We could, if desired, divide by $3$ to get
\[
u_{\xi\xi} - \frac{1}{3} u_{\eta\eta} = 0,
\]
but the important feature is that there are no mixed derivatives and the coefficients of $u_{\xi\xi}$ and $u_{\eta\eta}$ have opposite signs. This immediately reveals that the equation is hyperbolic, since the principal part is of the form
\[
(\text{positive}) \cdot u_{\xi\xi} + (\text{negative}) \cdot u_{\eta\eta}.
\]

\medskip

\noindent\textbf{Conceptual summary.}
This example illustrates a central theme in the classification of linear second-order PDEs: the type of the equation (elliptic, hyperbolic, parabolic) is governed by the quadratic form associated with its principal part. The symmetric coefficient matrix $Q$ encodes this quadratic form. Its discriminant and eigenvalues determine the type, just as in the classification of conic sections. By diagonalizing $Q$ via an orthogonal change of variables, we eliminate mixed derivatives and obtain a canonical form (here, $3 u_{\xi\xi} - u_{\eta\eta} = 0$). In these canonical coordinates, the structure and behavior of the PDE become much more transparent.
\end{solution}

% ===== Example 6: A Mixed-Type Equation: Tricomi’s Equation (inquiry-based) =====
\begin{problem}[A Mixed-Type Equation: Tricomi’s Equation]
In models of transonic fluid flow, the same flow may be subsonic in some regions and supersonic in others. Mathematically, this leads to partial differential equations that behave like elliptic equations in one part of the domain and like hyperbolic equations in another part. A model example is \emph{Tricomi’s equation}
\[
y\,u_{xx} + u_{yy} = 0,
\]
whose coefficient \(y\) changes sign across the line \(y=0\). In this problem you will see how the usual classification of second-order linear PDEs applies pointwise, and how a single equation can be elliptic, hyperbolic, and parabolic in different regions.

We recall that a general second-order linear PDE in two variables can be written in the form
\[
A(x,y)\,u_{xx} + 2 B(x,y)\,u_{xy} + C(x,y)\,u_{yy} + \text{(lower-order terms)} = 0.
\]

(a) For Tricomi’s equation
\[
y\,u_{xx} + u_{yy} = 0,
\]
rewrite it in the standard form above, and identify the coefficient functions \(A(x,y)\), \(B(x,y)\), and \(C(x,y)\). Then compute the discriminant
\[
D(x,y) = B^2 - A C.
\]
Based on the sign of \(D\), classify the equation as elliptic, parabolic, or hyperbolic at points with \(y>0\), with \(y<0\), and on the line \(y=0\). Sketch the regions of different type in the \((x,y)\)-plane.

\textit{Hint:} Only the second-order part matters for the classification, and here there is no mixed derivative term \(u_{xy}\).

(b) In the hyperbolic region, we can look for characteristic curves. The characteristic curves for a second-order PDE
\[
A u_{xx} + 2 B u_{xy} + C u_{yy} + \dots = 0
\]
are defined (away from points where all \(A,B,C\) vanish) by the ordinary differential equation
\[
A(x,y)\,(dy)^2 - 2 B(x,y)\,dx\,dy + C(x,y)\,(dx)^2 = 0.
\]
Restrict attention to the hyperbolic region of Tricomi’s equation. Write down the corresponding characteristic equation in differential form, and then express it as an equation for the slope \(\dfrac{dy}{dx}\).

\textit{Hint:} For Tricomi’s equation, \(B=0\), so the characteristic equation simplifies considerably.

(c) Still in the hyperbolic region, solve the ordinary differential equation you found in part (b) to obtain an explicit family of characteristic curves. Try to write your answer in the form
\[
x \pm \frac{2}{3}(-y)^{3/2} = \text{constant},
\]
or some equivalent implicit description. Sketch a few of these curves in the half-plane \(y<0\), and indicate how they meet the line \(y=0\).

\textit{Hint:} Once you have an equation for \(\dfrac{dy}{dx}\), it may be simpler to invert it and solve for \(\dfrac{dx}{dy}\). You will encounter an integral of a power of \(-y\).

(d) Now step back and interpret your findings.

\quad (i) Use your computation of the discriminant \(D\) to explain why there are two distinct real characteristic directions when \(y<0\), but no real characteristics when \(y>0\).

\quad (ii) Imagine a bounded domain whose lower boundary lies along two characteristic curves in the hyperbolic region \(y<0\) and whose upper boundary lies in the elliptic region \(y>0\), intersecting the line \(y=0\). Based on the usual theory for elliptic and hyperbolic equations, what kind of data (boundary values vs.\ initial values along curves) would you expect to prescribe on each part of the boundary in order to have a well-posed problem?

\textit{Hint:} Compare with the Laplace equation for elliptic behavior, and with the wave equation for hyperbolic behavior.

(e) Explore some variations.

\quad (i) Consider the modified equation
\[
y\,u_{xx} + u_{yy} + u = 0.
\]
Does the additional zeroth-order term \(u\) change the classification into elliptic, hyperbolic, or parabolic regions? Explain your reasoning carefully.

\quad (ii) Consider the more general family of equations
\[
a(y)\,u_{xx} + u_{yy} = 0,
\]
where \(a(y)\) is a continuous function of \(y\) only. In terms of the sign of \(a(y)\), describe for which values of \(y\) the equation is elliptic, for which it is hyperbolic, and where it is parabolic. Under what condition on \(a(y)\) (as a function of \(y\)) does this family give another example of a mixed-type equation?

\textit{Hint:} Go back to the discriminant \(D=B^2-AC\) and think about what happens when \(a(y)\) changes sign. 
\end{problem}

% ===== Example 6: A Mixed-Type Equation: Tricomi’s Equation (full solution) =====
\begin{problem}[A Mixed-Type Equation: Tricomi’s Equation]
Consider the second-order partial differential equation
\[
y\,u_{xx} + u_{yy} = 0, \qquad (x,y)\in\mathbb{R}^2.
\]
(a) Write this equation in the standard form
\[
A(x,y)\,u_{xx} + 2 B(x,y)\,u_{xy} + C(x,y)\,u_{yy} + \dots = 0,
\]
identify \(A,B,C\), and compute the discriminant \(D=B^2-AC\). Classify the equation as elliptic, hyperbolic, or parabolic in the regions \(y>0\), \(y<0\), and on the line \(y=0\).

(b) In the hyperbolic region, find the real characteristic curves by solving the characteristic equation
\[
A(dy)^2 - 2 B\,dx\,dy + C(dx)^2 = 0.
\]
Give an implicit formula for the characteristics and sketch their qualitative shape in the half-plane \(y<0\).

(c) Using the same characteristic equation, explain why there are no real characteristic curves in the region \(y>0\).

(d) Finally, consider the more general family
\[
a(y)\,u_{xx} + u_{yy} = 0,
\]
with continuous \(a(y)\). Express the discriminant in terms of \(a(y)\), describe the type (elliptic, hyperbolic, parabolic) as a function of \(y\), and state a condition on \(a\) under which this PDE is of mixed type.
\end{problem}

\begin{solution}
We first recall the general framework for classifying second-order linear PDEs in two variables. A PDE of the form
\[
A(x,y)\,u_{xx} + 2 B(x,y)\,u_{xy} + C(x,y)\,u_{yy} + \text{(lower-order terms)} = 0
\]
is called elliptic, parabolic, or hyperbolic at a point \((x_0,y_0)\) according to the sign of the discriminant
\[
D(x_0,y_0) = B(x_0,y_0)^2 - A(x_0,y_0)\,C(x_0,y_0).
\]
Specifically, the equation is elliptic if \(D<0\), hyperbolic if \(D>0\), and parabolic (degenerate) if \(D=0\). This criterion is directly analogous to the classification of quadratic forms and conic sections.

\medskip

\textbf{(a) Coefficients and classification.}

For Tricomi’s equation,
\[
y\,u_{xx} + u_{yy} = 0,
\]
we match this with the standard form
\[
A\,u_{xx} + 2 B\,u_{xy} + C\,u_{yy} = 0.
\]
There is no mixed derivative term \(u_{xy}\), so
\[
A(x,y) = y,\qquad B(x,y) = 0,\qquad C(x,y) = 1.
\]
The discriminant is therefore
\[
D(x,y) = B^2 - A C = 0^2 - y\cdot 1 = -y.
\]

We now classify by the sign of \(D\):

- If \(y>0\), then \(D=-y<0\), so the equation is \emph{elliptic} at every point with \(y>0\).
- If \(y<0\), then \(D=-y>0\), so the equation is \emph{hyperbolic} at every point with \(y<0\).
- If \(y=0\), then \(D=0\), so along the entire line \(y=0\) the equation is \emph{parabolic} (degenerate).

Thus the same equation is elliptic in the upper half-plane, hyperbolic in the lower half-plane, and parabolic on the horizontal line that separates them. This is a prototypical example of a \emph{mixed-type} equation.

\medskip

\textbf{(b) Characteristic curves in the hyperbolic region \(y<0\).}

In the region where the equation is hyperbolic, it admits real characteristic curves. For a second-order PDE
\[
A u_{xx} + 2B u_{xy} + C u_{yy} + \dots = 0,
\]
the characteristic curves in the \((x,y)\)-plane satisfy the quadratic differential equation
\[
A(x,y)\,(dy)^2 - 2 B(x,y)\,dx\,dy + C(x,y)\,(dx)^2 = 0.
\]

In our case, \(A = y\), \(B = 0\), \(C = 1\), so the characteristic equation becomes
\[
y\,(dy)^2 + (dx)^2 = 0.
\]
We would like to rewrite this as an ordinary differential equation for \(y\) as a function of \(x\). Dividing by \((dx)^2\) (on nonvertical portions of the curves) gives
\[
y\left(\frac{dy}{dx}\right)^2 + 1 = 0.
\]
Equivalently,
\[
\left(\frac{dy}{dx}\right)^2 = -\frac{1}{y}.
\]

This equation has real solutions only when \(-1/y \ge 0\), that is, only when \(y<0\). This agrees with the classification: real characteristics exist only in the hyperbolic region.

In the region \(y<0\), we can therefore write
\[
\frac{dy}{dx} = \pm \frac{1}{\sqrt{-y}}.
\]
It is slightly more convenient to invert this relation and solve for \(\dfrac{dx}{dy}\):
\[
\frac{dx}{dy} = \pm \sqrt{-y}.
\]

We now integrate with respect to \(y\). Define \(s = -y\), so that \(s>0\) when \(y<0\) and \(ds = -dy\). Then
\[
\frac{dx}{dy} = \pm \sqrt{-y}
\quad\Longrightarrow\quad
\frac{dx}{dy} = \pm \sqrt{s},\quad dy = -ds.
\]
Thus
\[
dx = \pm \sqrt{s}\,dy = \pm \sqrt{s}\,(-ds)
= \mp \sqrt{s}\,ds.
\]
Integrating,
\[
x = \mp \int \sqrt{s}\,ds + \text{constant}
= \mp \frac{2}{3} s^{3/2} + \text{constant}.
\]
Recalling that \(s=-y\), we obtain
\[
x = \mp \frac{2}{3}(-y)^{3/2} + C,
\]
or, equivalently,
\[
x \pm \frac{2}{3}(-y)^{3/2} = C.
\]

Thus there are two families of characteristic curves in the hyperbolic region \(y<0\), given implicitly by
\[
x + \frac{2}{3}(-y)^{3/2} = \text{constant}
\quad\text{and}\quad
x - \frac{2}{3}(-y)^{3/2} = \text{constant}.
\]

For a qualitative sketch in the half-plane \(y<0\), note that for each fixed constant \(C\),
\[
x = C \mp \frac{2}{3}(-y)^{3/2}.
\]
As \(y \to 0^-\), we have \((-y)^{3/2} \to 0\), so the curves approach the line \(y=0\) with horizontal tangent. As \(y\to -\infty\), the term \(( -y)^{3/2}\) grows, so the curves move off to the left or right with increasing steepness. The two families intersect transversely, as is typical for a hyperbolic equation with two distinct real characteristic directions.

\medskip

\textbf{(c) Absence of real characteristics in the elliptic region \(y>0\).}

We now use the same characteristic equation to explain why there are no real characteristics for \(y>0\). Recall that the characteristic equation reduced to
\[
\left(\frac{dy}{dx}\right)^2 = -\frac{1}{y}.
\]
If \(y>0\), then the right-hand side \(-1/y\) is negative, so the equation demands that \(\left(\dfrac{dy}{dx}\right)^2\) be negative, which is impossible for real-valued functions. Thus, for \(y>0\), there are no real solutions for \(\dfrac{dy}{dx}\), and therefore no real characteristic curves in the plane.

This is exactly what one expects for an elliptic equation: elliptic equations, such as the Laplace equation, do not possess real characteristic curves, and information propagates in a more global, “smeared” way rather than along distinguished curves.

On the line \(y=0\) itself, the coefficient \(A=y\) vanishes and the characteristic equation degenerates (it becomes \((dx)^2=0\)), reflecting the parabolic nature of the equation there.

\medskip

\textbf{(d) The general family \(a(y)\,u_{xx} + u_{yy} = 0\).}

Finally, consider the more general equation
\[
a(y)\,u_{xx} + u_{yy} = 0,
\]
with \(a(y)\) a continuous function of \(y\) only. Again we write it in the standard form:
\[
A(x,y) = a(y),\qquad B(x,y) = 0,\qquad C(x,y) = 1.
\]
The discriminant is then
\[
D(x,y) = B^2 - A C = -a(y).
\]

The type of the equation at a point \((x,y)\) depends only on the sign of \(a(y)\):

- If \(a(y) > 0\), then \(D = -a(y) < 0\): the equation is elliptic at all points with that value of \(y\).
- If \(a(y) < 0\), then \(D = -a(y) > 0\): the equation is hyperbolic at all points with that value of \(y\).
- If \(a(y) = 0\), then \(D = 0\): the equation is parabolic (degenerate) along the horizontal line corresponding to that \(y\).

Thus the equation is of \emph{mixed type} precisely when the coefficient \(a(y)\) changes sign as \(y\) varies, that is, when there exist values \(y_1\) and \(y_2\) with \(a(y_1) > 0\) and \(a(y_2) < 0\). By continuity, there must then be some \(y_0\) with \(a(y_0) = 0\); the horizontal line \(y=y_0\) is the type-change curve separating an elliptic region from a hyperbolic region.

\medskip

\textbf{Conceptual remarks.}

This example illustrates several central ideas of the classification of linear second-order PDEs:

1. \emph{Pointwise classification.} The type of a PDE is determined \emph{pointwise} by the coefficients \(A,B,C\). It is possible, as in Tricomi’s equation, for a single PDE to change type across a curve in the domain.

2. \emph{Discriminant and characteristics.} The sign of the discriminant \(D = B^2 - AC\) governs both the classification and the nature of characteristic curves: when \(D>0\) there are two distinct real characteristic directions (hyperbolic case), when \(D<0\) there are no real characteristic directions (elliptic case), and when \(D=0\) there is a single repeated direction (parabolic case).

3. \emph{Mixed-type modeling.} In applications such as transonic flow, the change from subsonic (elliptic-like) to supersonic (hyperbolic-like) regimes is modeled by coefficients that change sign. Tricomi’s equation and its generalizations \(a(y)u_{xx} + u_{yy} = 0\) provide canonical mathematical models for this mixed behavior, and they highlight the subtlety of formulating well-posed problems when an equation changes type within the domain.
\end{solution}

\section{Elliptic PDEs: Method of Green Function}
% --- Narrative plan (auto-generated) ---
% This section develops the method of Green functions as a systematic way to solve boundary value problems for elliptic partial differential equations, with a particular focus on the Poisson and Laplace equations. A Green function encodes how the domain and its boundary respond to a point source, allowing us to build solutions for arbitrary forcing terms by superposition. In this way, what seems like a complicated differential equation turns into an integral representation that separates geometry, boundary conditions, and data.
%
% Green functions are central in applied mathematics because they appear whenever steady states or equilibrium configurations are governed by linear elliptic operators: electrostatic potentials in conductors, steady heat distributions, incompressible fluid flows, and deflection of elastic membranes all admit Green function formulations. The method connects naturally to several other topics: it is closely related to fundamental solutions and convolution from ODEs and PDEs, to eigenfunction expansions and Fourier series on bounded domains, and to complex analysis through harmonic functions and the Poisson kernel. Understanding Green functions thus builds a bridge between geometric intuition about domains, analytic tools from functional analysis and spectral theory, and concrete computational techniques used throughout applied mathematics.

% ===== Example 1: Green Function for the One-Dimensional Poisson Equation on an Interval (inquiry-based) =====
\begin{problem}[Green Function for the One-Dimensional Poisson Equation on an Interval]
Consider a thin, homogeneous rod of length $L$, occupying the interval $(0,L)$ on the $x$-axis. Let $u(x)$ denote the steady-state temperature along the rod. Suppose the ends of the rod are held at zero temperature, and there is a distributed heat source of intensity $f(x)$ along the rod. In steady state, $u$ satisfies a one-dimensional Poisson equation with homogeneous Dirichlet boundary conditions. Our goal in this problem is to construct the corresponding Green function and to use it to represent $u$ as an explicit integral involving $f$.

We study the boundary value problem
\[
- u''(x) \;=\; f(x), \qquad 0 < x < L, \qquad
u(0) = 0, \quad u(L) = 0.
\]

\smallskip

(a) First recall the associated homogeneous equation
\[
- u''(x) = 0, \qquad 0 < x < L.
\]
Solve this ordinary differential equation and write its general solution. Then, impose the boundary conditions $u(0)=u(L)=0$. What does this tell you about the homogeneous solution?

\smallskip

(b) To incorporate the forcing term $f(x)$, we introduce the Green function $G(x,\xi)$ for the differential operator $L = -\dfrac{d^{2}}{dx^{2}}$ on $(0,L)$ with homogeneous Dirichlet boundary conditions. For a fixed point $\xi \in (0,L)$, $G(\cdot,\xi)$ is defined as the solution of
\[
- \frac{d^{2}}{dx^{2}} G(x,\xi) = \delta(x-\xi), \qquad 0 < x < L,
\]
with
\[
G(0,\xi) = 0, \qquad G(L,\xi) = 0,
\]
where $\delta(x-\xi)$ is the Dirac delta concentrated at $\xi$.

\begin{enumerate}
\item[(i)] For $x \neq \xi$, what equation does $G(x,\xi)$ satisfy? Write down the general form of $G(x,\xi)$ separately on the intervals $(0,\xi)$ and $(\xi,L)$, introducing appropriate constants.
\item[(ii)] Use the boundary conditions at $x=0$ and $x=L$ to simplify these general forms as much as possible.
\end{enumerate}

% Hint: Away from $x = \xi$, the right-hand side is zero, so you are solving the homogeneous equation again. Because the operator is second order, $G(\cdot,\xi)$ will be linear in $x$ on each side of $\xi$.

\smallskip

(c) Next, we must determine how $G(x,\xi)$ behaves at the point $x = \xi$.

\begin{enumerate}
\item[(i)] Argue that $G(x,\xi)$ should be continuous at $x = \xi$, that is,
\[
\lim_{x \to \xi^-} G(x,\xi) = \lim_{x \to \xi^+} G(x,\xi).
\]
What condition does this give on the two linear pieces you found in part (b)?
\item[(ii)] To determine the jump in the derivative of $G(x,\xi)$ at $x=\xi$, integrate the defining equation
\[
- G''(x,\xi) = \delta(x-\xi)
\]
from $x = \xi - \varepsilon$ to $x = \xi + \varepsilon$, for a small $\varepsilon > 0$, and then let $\varepsilon \to 0$. Show that
\[
- G_x(\xi^+,\xi) + G_x(\xi^-,\xi) = 1.
\]
Explain briefly how the property of the Dirac delta function is used here.
\end{enumerate}

% Hint: Use the Fundamental Theorem of Calculus on the left-hand side, and the defining property $\int_{\xi - \varepsilon}^{\xi + \varepsilon} \delta(x-\xi)\,dx = 1$ on the right-hand side.

\smallskip

(d) Use the conditions from parts (b) and (c) to solve for all constants and obtain an explicit formula for the Green function $G(x,\xi)$, valid for $0 < x, \xi < L$. Write your answer in a piecewise form depending on whether $x \le \xi$ or $x \ge \xi$.

Then, using this Green function, show that the solution $u$ of the boundary value problem
\[
- u''(x) = f(x), \quad 0 < x < L, \qquad u(0) = u(L) = 0,
\]
can be written as
\[
u(x) = \int_0^L G(x,\xi)\, f(\xi)\, d\xi.
\]

% Hint: Differentiate under the integral sign (formally) to apply the operator $- \frac{d^{2}}{dx^{2}}$ to $u(x)$, and use the defining equation for $G(x,\xi)$.

\smallskip

(e) Finally, explore some variations and extensions.

\begin{enumerate}
\item[(i)] Check whether your Green function is symmetric: does $G(x,\xi) = G(\xi,x)$ hold for all $x,\xi \in (0,L)$? If so, verify this from your explicit formula. Why might such a symmetry be expected from the operator $L = -\dfrac{d^2}{dx^2}$ with homogeneous Dirichlet boundary conditions?
\item[(ii)] Suppose instead that the rod occupies the interval $(a,b)$, with $u(a)=u(b)=0$. Without redoing all of the calculations in detail, sketch how the formulas for the Green function would have to be modified. What are the analogues of the factors $x$, $L-x$, $\xi$, and $L-\xi$ in this more general case?
\end{enumerate}

% Hint: You can think of shifting and rescaling the interval $(a,b)$ to $(0,L)$, or you can directly repeat the piecewise linear construction with $0$ replaced by $a$ and $L$ replaced by $b$.
\end{problem}

% ===== Example 1: Green Function for the One-Dimensional Poisson Equation on an Interval (full solution) =====
\begin{problem}[Green Function for the One-Dimensional Poisson Equation on an Interval]
Consider the boundary value problem
\[
- u''(x) = f(x), \qquad 0 < x < L, \qquad u(0) = 0, \quad u(L) = 0.
\]
\begin{enumerate}
\item[(a)] For the operator $L u = -u''$ with these boundary conditions, construct the Green function $G(x,\xi)$ defined by
\[
- \frac{d^{2}}{dx^{2}} G(x,\xi) = \delta(x-\xi), \qquad 0 < x < L, \qquad G(0,\xi)=G(L,\xi)=0,
\]
for each fixed $\xi \in (0,L)$.
\item[(b)] Derive an explicit formula for $G(x,\xi)$, written piecewise depending on whether $x \le \xi$ or $x \ge \xi$, and show that $G(x,\xi) = G(\xi,x)$.
\item[(c)] Show that the unique solution of the boundary value problem can be written in the Green function form
\[
u(x) = \int_0^L G(x,\xi)\, f(\xi)\, d\xi.
\]
\end{enumerate}
\end{problem}

\begin{solution}
We study the one-dimensional Poisson equation with homogeneous Dirichlet boundary conditions on the finite interval $(0,L)$. The central idea of the Green function method is to construct the response of the system to a point source and then build the response to a general forcing term by superposition.

\medskip

\textbf{Step 1: Homogeneous problem and motivation.}
The associated homogeneous equation is
\[
- u''(x) = 0, \qquad 0 < x < L.
\]
Integrating twice, we obtain the general solution
\[
u(x) = Ax + B,
\]
for constants $A$ and $B$. The boundary conditions $u(0) = 0$ and $u(L) = 0$ give
\[
u(0) = B = 0, \qquad u(L) = AL + B = AL = 0,
\]
so $A = 0$. Thus $u \equiv 0$ is the unique solution of the homogeneous problem with these boundary conditions; there is no nontrivial solution that satisfies $u(0) = u(L) = 0$. This tells us that if we can find any particular solution for a given $f$, then it is automatically the unique solution.

The Green function will give us such particular solutions for arbitrary right-hand sides $f$.

\medskip

\textbf{Step 2: Definition and piecewise structure of the Green function.}
We now define the Green function $G(x,\xi)$ for the operator $L = -\dfrac{d^2}{dx^2}$ with homogeneous Dirichlet boundary conditions. For each fixed $\xi \in (0,L)$, the function $x \mapsto G(x,\xi)$ is defined by
\begin{equation}\label{eq:Green-def}
- G_{xx}(x,\xi) = \delta(x-\xi), \qquad 0 < x < L,
\end{equation}
with boundary conditions
\begin{equation}\label{eq:Green-BC}
G(0,\xi) = 0, \qquad G(L,\xi) = 0.
\end{equation}
Here $\delta(x-\xi)$ is the Dirac delta, representing a unit point source at $x=\xi$.

For $x \neq \xi$ the right-hand side of \eqref{eq:Green-def} vanishes, so $G$ satisfies the homogeneous equation
\[
- G_{xx}(x,\xi) = 0 \quad \Longleftrightarrow \quad G_{xx}(x,\xi) = 0,
\]
on each of the subintervals $(0,\xi)$ and $(\xi,L)$. Therefore $G(\cdot,\xi)$ is linear in $x$ on each side of $\xi$. We write
\begin{align*}
G(x,\xi) &= a_1 x + b_1, \qquad 0 < x < \xi, \\
G(x,\xi) &= a_2 x + b_2, \qquad \xi < x < L,
\end{align*}
for some constants $a_1, b_1, a_2, b_2$ that may depend on $\xi$.

The boundary conditions \eqref{eq:Green-BC} give two relations. At $x=0$,
\[
G(0,\xi) = a_1 \cdot 0 + b_1 = b_1 = 0,
\]
so $b_1 = 0$. At $x=L$,
\[
G(L,\xi) = a_2 L + b_2 = 0,
\]
so $b_2 = -a_2 L$. Thus we can rewrite
\[
G(x,\xi) =
\begin{cases}
a_1 x, & 0 < x < \xi,\\[4pt]
a_2 x - a_2 L, & \xi < x < L.
\end{cases}
\]

\medskip

\textbf{Step 3: Conditions at the point $x = \xi$.}
To determine $a_1$ and $a_2$, we use the behavior of $G$ at the source point $x = \xi$.

First, we require continuity of $G$ at $x = \xi$. Physically, this corresponds to the temperature (or potential) not having an infinite jump at a point source; the singularity appears in the second derivative instead. Mathematically, continuity is appropriate because the equation involves $G''$ but not lower derivatives with distributions other than the delta. Thus we impose
\[
\lim_{x \to \xi^-} G(x,\xi) = \lim_{x \to \xi^+} G(x,\xi).
\]
Using our expressions,
\[
G(\xi^-,\xi) = a_1 \xi, \qquad G(\xi^+,\xi) = a_2 \xi - a_2 L.
\]
Continuity gives
\begin{equation}\label{eq:continuity}
a_1 \xi = a_2 \xi - a_2 L = a_2(\xi - L).
\end{equation}

Second, we determine the jump in the derivative $G_x$ at $x = \xi$ by integrating the defining equation \eqref{eq:Green-def} across a small interval around $\xi$. Let $\varepsilon > 0$ be small, and integrate from $\xi - \varepsilon$ to $\xi + \varepsilon$:
\[
\int_{\xi - \varepsilon}^{\xi + \varepsilon} \bigl(-G_{xx}(x,\xi)\bigr)\, dx
=
\int_{\xi - \varepsilon}^{\xi + \varepsilon} \delta(x-\xi)\, dx.
\]
By the Fundamental Theorem of Calculus, the left-hand side is
\[
- G_x(\xi+\varepsilon,\xi) + G_x(\xi-\varepsilon,\xi).
\]
By the defining property of the Dirac delta, the right-hand side equals $1$ for every $\varepsilon > 0$ small enough that the interval contains $\xi$, so we obtain
\[
- G_x(\xi+\varepsilon,\xi) + G_x(\xi-\varepsilon,\xi) = 1.
\]
Letting $\varepsilon \to 0$, we conclude
\begin{equation}\label{eq:jump}
- G_x(\xi^+,\xi) + G_x(\xi^-,\xi) = 1.
\end{equation}

Now we compute $G_x$ from our piecewise linear forms. On $(0,\xi)$,
\[
G_x(x,\xi) = a_1,
\]
so $G_x(\xi^-,\xi) = a_1$. On $(\xi,L)$,
\[
G_x(x,\xi) = a_2,
\]
so $G_x(\xi^+,\xi) = a_2$. Substituting into \eqref{eq:jump} gives
\begin{equation}\label{eq:jump-a}
- a_2 + a_1 = 1, \quad \text{that is,} \quad a_1 - a_2 = 1.
\end{equation}

\medskip

\textbf{Step 4: Solving for the constants and explicit Green function.}
We now solve the system of two equations \eqref{eq:continuity} and \eqref{eq:jump-a} for $a_1$ and $a_2$.

From \eqref{eq:jump-a}, we have
\[
a_1 = 1 + a_2.
\]
Substituting this into \eqref{eq:continuity} yields
\[
(1 + a_2)\,\xi = a_2 (\xi - L).
\]
Expanding and rearranging,
\[
\xi + a_2 \xi = a_2 \xi - a_2 L.
\]
Cancel $a_2 \xi$ from both sides to obtain
\[
\xi = - a_2 L,
\]
so
\[
a_2 = - \frac{\xi}{L}.
\]
Then
\[
a_1 = 1 + a_2 = 1 - \frac{\xi}{L} = \frac{L - \xi}{L}.
\]

Substituting back, we obtain the explicit Green function
\[
G(x,\xi) =
\begin{cases}
\dfrac{L - \xi}{L}\, x, & 0 \le x \le \xi,\\[6pt]
-\dfrac{\xi}{L}\, x + \xi, & \xi \le x \le L.
\end{cases}
\]
It is common to rewrite the second branch in a more symmetric form. Noting that
\[
-\dfrac{\xi}{L}\, x + \xi = \xi\!\left(1 - \dfrac{x}{L}\right) = \dfrac{\xi(L-x)}{L},
\]
we can present the Green function as
\begin{equation}\label{eq:Green-final}
G(x,\xi) =
\begin{cases}
\dfrac{(L - \xi)\, x}{L}, & 0 \le x \le \xi,\\[6pt]
\dfrac{\xi\, (L - x)}{L}, & \xi \le x \le L.
\end{cases}
\end{equation}

This function is continuous on $[0,L]$, vanishes at $x = 0$ and $x = L$, is piecewise linear in $x$ with a corner at $x = \xi$, and its second derivative in the distributional sense is $- \delta(x-\xi)$, as required.

\medskip

\textbf{Step 5: Symmetry of the Green function.}
We next check that $G(x,\xi) = G(\xi,x)$ for all $x,\xi \in (0,L)$. From the explicit expression \eqref{eq:Green-final}, observe that
\[
G(x,\xi) =
\begin{cases}
\dfrac{(L - \xi)\, x}{L}, & x \le \xi,\\[6pt]
\dfrac{\xi\, (L - x)}{L}, & x \ge \xi.
\end{cases}
\]
Now interchange the roles of $x$ and $\xi$ to write $G(\xi,x)$:
\[
G(\xi,x) =
\begin{cases}
\dfrac{(L - x)\, \xi}{L}, & \xi \le x,\\[6pt]
\dfrac{x\, (L - \xi)}{L}, & \xi \ge x.
\end{cases}
\]
Consider the two cases.

\emph{Case 1: $x \le \xi$.} Then, by the definition of $G(x,\xi)$,
\[
G(x,\xi) = \frac{(L - \xi)\, x}{L}.
\]
On the other hand, since $\xi \ge x$, the second branch in the expression for $G(\xi,x)$ applies:
\[
G(\xi,x) = \frac{x\, (L - \xi)}{L}.
\]
These expressions coincide, so $G(x,\xi) = G(\xi,x)$ when $x \le \xi$.

\emph{Case 2: $x \ge \xi$.} Then
\[
G(x,\xi) = \frac{\xi\, (L - x)}{L},
\]
while now $\xi \le x$, so the first branch in $G(\xi,x)$ applies:
\[
G(\xi,x) = \frac{(L - x)\, \xi}{L}.
\]
Again these are identical, so $G(x,\xi) = G(\xi,x)$ for $x \ge \xi$ as well. Thus the Green function is symmetric:
\[
G(x,\xi) = G(\xi,x).
\]
This symmetry is typical for self-adjoint operators like $L = -\dfrac{d^{2}}{dx^{2}}$ with homogeneous boundary conditions.

\medskip

\textbf{Step 6: Representation formula for the solution.}
We now prove that for any given forcing term $f$ (say, sufficiently nice, for instance continuous), the unique solution of
\[
- u''(x) = f(x), \qquad 0 < x < L, \qquad u(0) = u(L) = 0,
\]
can be written as
\begin{equation}\label{eq:representation}
u(x) = \int_0^L G(x,\xi)\, f(\xi)\, d\xi.
\end{equation}

First, we check that $u$ given by \eqref{eq:representation} satisfies the differential equation. Formally differentiating under the integral sign twice with respect to $x$, we have
\[
u''(x) = \int_0^L \frac{\partial^2}{\partial x^2} G(x,\xi)\, f(\xi)\, d\xi.
\]
Multiplying by $-1$ gives
\[
- u''(x) = \int_0^L \Bigl(-\frac{\partial^2}{\partial x^2} G(x,\xi)\Bigr)\, f(\xi)\, d\xi.
\]
By the defining property of $G$, we have
\[
- \frac{\partial^2}{\partial x^2} G(x,\xi) = \delta(x-\xi),
\]
in the sense of distributions, so
\[
- u''(x) = \int_0^L \delta(x-\xi)\, f(\xi)\, d\xi.
\]
Using the sifting property of the Dirac delta,
\[
\int_0^L \delta(x-\xi)\, f(\xi)\, d\xi = f(x),
\]
for $x \in (0,L)$. Thus $u$ defined by \eqref{eq:representation} satisfies
\[
- u''(x) = f(x), \qquad 0 < x < L.
\]

Next, we verify the boundary conditions. At $x=0$,
\[
u(0) = \int_0^L G(0,\xi)\, f(\xi)\, d\xi = \int_0^L 0 \cdot f(\xi)\, d\xi = 0,
\]
because $G(0,\xi) = 0$ for all $\xi$. Similarly, at $x = L$,
\[
u(L) = \int_0^L G(L,\xi)\, f(\xi)\, d\xi = 0,
\]
since $G(L,\xi) = 0$ for all $\xi$. Therefore $u$ satisfies the boundary conditions $u(0)=u(L)=0$.

Finally, we argue uniqueness. Suppose $\tilde{u}$ is any solution of the boundary value problem. Then $w = u - \tilde{u}$ satisfies
\[
- w''(x) = 0, \qquad w(0) = 0, \quad w(L) = 0.
\]
As we saw in Step 1, the only such solution is $w \equiv 0$. Hence $u = \tilde{u}$, so the solution \eqref{eq:representation} is unique and therefore gives the unique solution of the original boundary value problem.

\medskip

\textbf{Conclusion and relation to the Green function method.}
In this example we have illustrated the key ideas of the Green function method for elliptic problems in a particularly simple setting:

\begin{itemize}
\item We identified the linear differential operator $L = -\dfrac{d^{2}}{dx^{2}}$ and its homogeneous boundary conditions.
\item We constructed a Green function $G(x,\xi)$ as the response to a point source, satisfying $L G(\cdot,\xi) = \delta(\cdot-\xi)$ and the same boundary conditions as $u$.
\item Exploiting linearity and the defining property of the Dirac delta, we expressed the solution to the inhomogeneous problem as a superposition (integral) of point-source responses:
\[
u(x) = \int_0^L G(x,\xi)\, f(\xi)\, d\xi.
\]
\end{itemize}

This one-dimensional case already shows the general pattern for elliptic partial differential equations: once an appropriate Green function is found, solutions with arbitrary source terms can be written explicitly as integral operators acting on the data. In higher dimensions, the construction of $G$ is more involved, but the underlying principles are the same.
\end{solution}

% ===== Example 2: Electrostatic Potential in a Rectangular Box via Eigenfunction Expansion (inquiry-based) =====
\begin{problem}[Electrostatic Potential in a Rectangular Box via Eigenfunction Expansion]
We consider a long rectangular box whose cross section is the rectangle
\[
\Omega = \{(x,y) : 0 < x < a,\; 0 < y < b\}.
\]
Assume the walls of the box are perfectly conducting and are held at zero potential. Inside the box there may be a static charge density $\rho(x,y)$, and we seek the resulting electrostatic potential $\Phi(x,y)$ in the cross section. Instead of trying to guess the Green function directly, we will build it systematically from eigenfunctions of the Laplacian that satisfy the same boundary conditions.

We work with Poisson's equation in two dimensions,
\[
- \Delta \Phi(x,y) = \frac{\rho(x,y)}{\varepsilon_0} \quad \text{in } \Omega, 
\qquad 
\Phi = 0 \quad \text{on } \partial\Omega,
\]
where $\varepsilon_0$ is the permittivity of free space.

\smallskip

(a) (Modeling and Green function definition.)  
Explain why the above Poisson problem models the steady electrostatic potential in the box with grounded walls. Then define precisely the \emph{Dirichlet Green function} $G(x,y;\xi,\eta)$ for the Laplacian on $\Omega$ with zero boundary conditions.  
In particular, write down the boundary value problem that $G$ must satisfy (be explicit about which variables the Laplacian acts on, and where the Dirac delta appears).

\medskip

(b) (Eigenfunctions of the Laplacian with Dirichlet boundary conditions.)  
To construct $G$, we look for eigenfunctions of the Laplacian that satisfy homogeneous Dirichlet boundary conditions. Consider the eigenvalue problem
\[
- \Delta \varphi(x,y) = \lambda \, \varphi(x,y) 
\quad \text{for } (x,y)\in\Omega, 
\qquad 
\varphi = 0 \quad \text{on } \partial\Omega.
\]
\begin{enumerate}
\item[(i)] Use separation of variables $\varphi(x,y) = X(x)Y(y)$ to derive ordinary differential equations for $X$ and $Y$, together with their boundary conditions.
\item[(ii)] Solve these ODEs and show that, up to normalization, the eigenfunctions are
\[
\varphi_{mn}(x,y) = \sin\!\left(\frac{m\pi x}{a}\right)\sin\!\left(\frac{n\pi y}{b}\right), 
\quad m,n = 1,2,\dots,
\]
with corresponding eigenvalues
\[
\lambda_{mn} = \left(\frac{m\pi}{a}\right)^2 + \left(\frac{n\pi}{b}\right)^2.
\]
\end{enumerate}
Hint: Recall the standard Sturm–Liouville problem
\[
X''(x) + k^2 X(x) = 0,\quad X(0)=X(a)=0
\]
and its sine solutions.

\medskip

(c) (Orthogonality and normalization.)  
The collection $\{\varphi_{mn}\}_{m,n\ge 1}$ forms an orthogonal basis in $L^2(\Omega)$ with respect to the inner product
\[
\langle u,v\rangle = \int_0^a\int_0^b u(x,y)\,v(x,y)\,dy\,dx.
\]
\begin{enumerate}
\item[(i)] Compute the integral
\[
\int_0^a \sin\!\left(\frac{m\pi x}{a}\right)\sin\!\left(\frac{m'\pi x}{a}\right)\,dx
\]
and show that it vanishes when $m\ne m'$, and find its value when $m=m'$. Do the analogous computation in $y$.
\item[(ii)] Use (i) to show that
\[
\int_0^a\!\!\int_0^b \varphi_{mn}(x,y)\,\varphi_{m'n'}(x,y)\,dy\,dx
= C\,\delta_{mm'}\delta_{nn'}
\]
for some constant $C$ independent of $m,n$. Determine $C$, and then define normalized eigenfunctions $\phi_{mn}(x,y)$ so that
\[
\int_0^a\!\!\int_0^b \phi_{mn}(x,y)\,\phi_{m'n'}(x,y)\,dy\,dx
= \delta_{mm'}\delta_{nn'}.
\]
\end{enumerate}
Hint: Use the one-dimensional orthogonality relations to factor the two-dimensional integral.

\medskip

(d) (Eigenfunction expansion of the Green function.)  
We now expand the Green function in the orthonormal basis $\{\phi_{mn}\}$:
\[
G(x,y;\xi,\eta) = \sum_{m=1}^\infty \sum_{n=1}^\infty A_{mn}(\xi,\eta)\,\phi_{mn}(x,y).
\]
\begin{enumerate}
\item[(i)] Use the defining equation for $G$ from part (a) and the eigenvalue equation for $\phi_{mn}$ to show that the coefficients must have the form
\[
A_{mn}(\xi,\eta) = \frac{\phi_{mn}(\xi,\eta)}{\lambda_{mn}}.
\]
(Hint: Apply $- \Delta_{x,y}$ to the series for $G$, and use orthonormality to match the Dirac delta as an $L^2$-limit of eigenfunction expansions.)
\item[(ii)] Substitute your explicit formulas for $\phi_{mn}$ and $\lambda_{mn}$ to obtain a concrete double series expression
\[
G(x,y;\xi,\eta) = \sum_{m=1}^\infty \sum_{n=1}^\infty 
\frac{\text{(explicit product of sines)}}{\left(\frac{m\pi}{a}\right)^2 + \left(\frac{n\pi}{b}\right)^2}.
\]
Write out this series with the correct normalization constant.
\item[(iii)] Show formally that if $- \Delta \Phi = \rho/\varepsilon_0$ in $\Omega$ with $\Phi=0$ on $\partial\Omega$, then
\[
\Phi(x,y) = \frac{1}{\varepsilon_0}\int_0^a\!\!\int_0^b
G(x,y;\xi,\eta)\,\rho(\xi,\eta)\,d\eta\,d\xi
\]
solves the boundary value problem.
\end{enumerate}

\medskip

(e) (Extensions and “what if” questions.)
\begin{enumerate}
\item[(i)] Suppose now that there is \emph{no} charge in the box ($\rho\equiv 0$), but the top side at $y=b$ is held at a prescribed potential $f(x)$, while the other three sides remain grounded. How could you adapt the eigenfunction expansion method you used above to represent the solution $\Phi(x,y)$? (You do not need to carry out the full computation; outline the main steps.)
\item[(ii)] How would the eigenfunctions and eigenvalues change if, instead of Dirichlet boundary conditions $\Phi=0$, you imposed homogeneous Neumann boundary conditions $\partial\Phi/\partial n =0$ on all four sides of the rectangle? Describe the new separated solutions $X(x)$ and $Y(y)$ and the corresponding eigenvalues.
\end{enumerate}

\end{problem}

% ===== Example 2: Electrostatic Potential in a Rectangular Box via Eigenfunction Expansion (full solution) =====
\begin{problem}[Electrostatic Potential in a Rectangular Box via Eigenfunction Expansion]
Let $\Omega = (0,a)\times(0,b)$ be a rectangular domain. Consider the Dirichlet Poisson problem
\[
- \Delta \Phi(x,y) = f(x,y) \quad \text{in } \Omega,
\qquad
\Phi = 0 \quad \text{on } \partial\Omega.
\]
\begin{enumerate}
\item[(i)] Solve the eigenvalue problem
\[
- \Delta \varphi = \lambda \varphi \quad \text{in } \Omega,
\qquad
\varphi = 0 \quad \text{on } \partial\Omega,
\]
and obtain an orthonormal set of eigenfunctions $\{\phi_{mn}\}_{m,n\ge1}$ and corresponding eigenvalues $\lambda_{mn}$.
\item[(ii)] Using these eigenfunctions, construct the Dirichlet Green function $G(x,y;\xi,\eta)$ for $-\Delta$ on $\Omega$ and show that it can be written as the double series
\[
G(x,y;\xi,\eta)
= \sum_{m=1}^\infty \sum_{n=1}^\infty
\frac{\phi_{mn}(x,y)\,\phi_{mn}(\xi,\eta)}{\lambda_{mn}}.
\]
Compute this expression explicitly as a double Fourier sine series.
\item[(iii)] Show that the solution of the Poisson problem is given by
\[
\Phi(x,y) = \int_0^a\!\!\int_0^b G(x,y;\xi,\eta)\,f(\xi,\eta)\,d\eta\,d\xi.
\]
\end{enumerate}
Explain briefly how this example illustrates the method of Green functions for elliptic boundary value problems.
\end{problem}

\begin{solution}
We solve the problem in three steps: first we find the eigenfunctions of the Laplacian with homogeneous Dirichlet boundary conditions; then we use them to construct the Green function as a spectral expansion; finally we express the solution as a Green function integral.

\medskip

\noindent\textbf{(i) Eigenfunctions and eigenvalues.}
We consider
\[
- \Delta \varphi = \lambda \varphi \quad \text{in } \Omega = (0,a)\times(0,b),
\qquad
\varphi = 0 \quad \text{on } \partial\Omega.
\]
We use separation of variables, seeking solutions of the form $\varphi(x,y)=X(x)Y(y)$. Then
\[
- \Delta \varphi = -\bigl(X''(x)Y(y) + X(x)Y''(y)\bigr)
= \lambda X(x)Y(y).
\]
Dividing by $X(x)Y(y)$ (assuming nontrivial solutions), we get
\[
- \frac{X''(x)}{X(x)} - \frac{Y''(y)}{Y(y)} = \lambda.
\]
The left-hand side is a sum of a function of $x$ and a function of $y$, so each must be constant. We write
\[
- \frac{X''(x)}{X(x)} = \mu, 
\qquad
- \frac{Y''(y)}{Y(y)} = \nu,
\]
with $\mu,\nu$ constants and $\mu+\nu=\lambda$.

Thus $X$ and $Y$ satisfy the one-dimensional eigenvalue problems
\[
\begin{cases}
- X''(x) = \mu X(x), & 0 < x < a, \\
X(0)=X(a)=0,
\end{cases}
\qquad
\begin{cases}
- Y''(y) = \nu Y(y), & 0 < y < b, \\
Y(0)=Y(b)=0.
\end{cases}
\]

From the standard Sturm–Liouville theory on $(0,a)$ with Dirichlet boundary conditions, we know that nontrivial solutions occur only when
\[
\mu_m = \left(\frac{m\pi}{a}\right)^2,\quad 
X_m(x) = \sin\!\left(\frac{m\pi x}{a}\right),\quad m=1,2,\dots,
\]
and similarly on $(0,b)$,
\[
\nu_n = \left(\frac{n\pi}{b}\right)^2,\quad 
Y_n(y) = \sin\!\left(\frac{n\pi y}{b}\right),\quad n=1,2,\dots.
\]
Therefore the separated solutions are
\[
\varphi_{mn}(x,y) = X_m(x)Y_n(y)
= \sin\!\left(\frac{m\pi x}{a}\right)\sin\!\left(\frac{n\pi y}{b}\right),
\]
with eigenvalues
\[
\lambda_{mn} = \mu_m + \nu_n
= \left(\frac{m\pi}{a}\right)^2 + \left(\frac{n\pi}{b}\right)^2,
\qquad m,n=1,2,\dots.
\]
These functions satisfy $\varphi_{mn}=0$ on $x=0,a$ and $y=0,b$, as required.

We now normalize these eigenfunctions to form an orthonormal set in $L^2(\Omega)$ with respect to the inner product
\[
\langle u,v\rangle = \int_0^a\!\int_0^b u(x,y)\,v(x,y)\,dy\,dx.
\]
First, recall the one-dimensional orthogonality:
\[
\int_0^a \sin\!\left(\frac{m\pi x}{a}\right)\sin\!\left(\frac{m'\pi x}{a}\right)\,dx
=
\begin{cases}
0, & m\neq m',\\[3pt]
\dfrac{a}{2}, & m = m'.
\end{cases}
\]
An analogous formula holds on $(0,b)$:
\[
\int_0^b \sin\!\left(\frac{n\pi y}{b}\right)\sin\!\left(\frac{n'\pi y}{b}\right)\,dy
=
\begin{cases}
0, & n\neq n',\\[3pt]
\dfrac{b}{2}, & n = n'.
\end{cases}
\]
Therefore
\[
\int_0^a\!\int_0^b \varphi_{mn}(x,y)\,\varphi_{m'n'}(x,y)\,dy\,dx
= \left(\frac{a}{2}\delta_{mm'}\right)\left(\frac{b}{2}\delta_{nn'}\right)
= \frac{ab}{4}\,\delta_{mm'}\delta_{nn'}.
\]
To obtain orthonormal eigenfunctions, we define
\[
\phi_{mn}(x,y)
:= \frac{2}{\sqrt{ab}}\,\sin\!\left(\frac{m\pi x}{a}\right)
\sin\!\left(\frac{n\pi y}{b}\right).
\]
Then
\[
\int_0^a\!\int_0^b \phi_{mn}(x,y)\,\phi_{m'n'}(x,y)\,dy\,dx
= \delta_{mm'}\delta_{nn'}.
\]
Thus $\{\phi_{mn}\}_{m,n\ge1}$ is an orthonormal set of eigenfunctions of $-\Delta$ with eigenvalues $\lambda_{mn}$ as above. On a rectangle, these eigenfunctions are known to be complete in $L^2(\Omega)$, so any square-integrable function can be expanded in this basis.

\medskip

\noindent\textbf{(ii) Construction of the Green function.}
We now construct the Green function $G(x,y;\xi,\eta)$ for $-\Delta$ with Dirichlet boundary conditions. By definition, $G$ satisfies
\[
\begin{cases}
- \Delta_{x,y} G(x,y;\xi,\eta) 
= \delta(x-\xi)\,\delta(y-\eta), & (x,y)\in\Omega,\\[3pt]
G(x,y;\xi,\eta) = 0, & (x,y)\in\partial\Omega,
\end{cases}
\]
where the Laplacian acts on the $(x,y)$ variables and $(\xi,\eta)$ play the role of parameters indicating the location of the point source.

The operator $-\Delta$ with Dirichlet boundary conditions is a positive, self-adjoint operator on $L^2(\Omega)$ with the orthonormal eigenfunctions $\phi_{mn}$ and eigenvalues $\lambda_{mn}>0$. The general spectral theory for such operators tells us that the inverse operator $(-\Delta)^{-1}$, when it exists, acts on each eigenfunction by division by the eigenvalue:
\[
(-\Delta)^{-1}\phi_{mn} = \frac{1}{\lambda_{mn}}\phi_{mn}.
\]
In terms of Green functions, this means that $G(x,y;\xi,\eta)$, viewed as the integral kernel of $(-\Delta)^{-1}$, admits the eigenfunction expansion
\[
G(x,y;\xi,\eta) = \sum_{m=1}^\infty\sum_{n=1}^\infty
\frac{\phi_{mn}(x,y)\,\phi_{mn}(\xi,\eta)}{\lambda_{mn}}.
\]

We can justify this formula more concretely. Fix $(\xi,\eta)\in\Omega$ and expand $G(\cdot,\cdot;\xi,\eta)$ in the orthonormal basis:
\[
G(x,y;\xi,\eta)
= \sum_{m=1}^\infty\sum_{n=1}^\infty A_{mn}(\xi,\eta)\,\phi_{mn}(x,y),
\]
with
\[
A_{mn}(\xi,\eta)
= \int_0^a\!\int_0^b G(x,y;\xi,\eta)\,\phi_{mn}(x,y)\,dy\,dx.
\]
Apply $- \Delta_{x,y}$ to both sides. Using $- \Delta \phi_{mn} = \lambda_{mn}\phi_{mn}$, we obtain (in the sense of distributions)
\[
- \Delta_{x,y} G(x,y;\xi,\eta)
= \sum_{m,n} A_{mn}(\xi,\eta)\,\lambda_{mn}\,\phi_{mn}(x,y).
\]
On the other hand, by definition of $G$,
\[
- \Delta_{x,y} G(x,y;\xi,\eta)
= \delta(x-\xi)\,\delta(y-\eta).
\]
The right-hand side can itself be expanded in the orthonormal basis $\{\phi_{mn}\}$:
\[
\delta(x-\xi)\,\delta(y-\eta)
= \sum_{m,n} \phi_{mn}(\xi,\eta)\,\phi_{mn}(x,y),
\]
since for any $v\in L^2(\Omega)$,
\[
\int_0^a\!\int_0^b \delta(x-\xi)\,\delta(y-\eta)\,v(x,y)\,dy\,dx
= v(\xi,\eta)
= \sum_{m,n} \big\langle v,\phi_{mn}\big\rangle \phi_{mn}(\xi,\eta)
\]
and the expansion coefficients match term by term.

Comparing the two series expansions for $- \Delta_{x,y} G$, we must have
\[
A_{mn}(\xi,\eta)\,\lambda_{mn} = \phi_{mn}(\xi,\eta),
\quad\text{so}\quad
A_{mn}(\xi,\eta) = \frac{\phi_{mn}(\xi,\eta)}{\lambda_{mn}}.
\]
Therefore
\[
G(x,y;\xi,\eta)
= \sum_{m=1}^\infty\sum_{n=1}^\infty
\frac{\phi_{mn}(x,y)\,\phi_{mn}(\xi,\eta)}{\lambda_{mn}},
\]
as claimed.

Substituting the explicit formulas for $\phi_{mn}$ and $\lambda_{mn}$, we obtain
\[
\phi_{mn}(x,y)\,\phi_{mn}(\xi,\eta)
= \left(\frac{2}{\sqrt{ab}}\right)^2
\sin\!\left(\frac{m\pi x}{a}\right)\sin\!\left(\frac{m\pi \xi}{a}\right)
\sin\!\left(\frac{n\pi y}{b}\right)\sin\!\left(\frac{n\pi \eta}{b}\right)
= \frac{4}{ab}\,\prod_{\zeta\in\{x,\xi\}}\sin\!\left(\frac{m\pi \zeta}{a}\right)
\prod_{\zeta\in\{y,\eta\}}\sin\!\left(\frac{n\pi \zeta}{b}\right),
\]
and
\[
\lambda_{mn} = \left(\frac{m\pi}{a}\right)^2 + \left(\frac{n\pi}{b}\right)^2.
\]
Thus the Green function is the double Fourier sine series
\[
G(x,y;\xi,\eta)
= \sum_{m=1}^\infty\sum_{n=1}^\infty
\frac{4}{ab}
\frac{\sin\!\left(\frac{m\pi x}{a}\right)\sin\!\left(\frac{m\pi \xi}{a}\right)
\sin\!\left(\frac{n\pi y}{b}\right)\sin\!\left(\frac{n\pi \eta}{b}\right)}
{\left(\frac{m\pi}{a}\right)^2 + \left(\frac{n\pi}{b}\right)^2}.
\]
This is the desired explicit representation of the Dirichlet Green function for the rectangle.

\medskip

\noindent\textbf{(iii) Representation of the solution.}
We now show that the solution of
\[
- \Delta \Phi = f \quad \text{in }\Omega, \qquad \Phi = 0 \quad \text{on }\partial\Omega,
\]
is given by
\[
\Phi(x,y) = \int_0^a\!\!\int_0^b G(x,y;\xi,\eta)\,f(\xi,\eta)\,d\eta\,d\xi.
\]

First, note that by construction $G(\cdot,\cdot;\xi,\eta)$ vanishes on $\partial\Omega$ for each fixed $(\xi,\eta)$, so the integral expression for $\Phi$ will also satisfy $\Phi=0$ on $\partial\Omega$ (since the integral is a linear combination of functions each of which vanishes on the boundary).

Next, we apply $- \Delta_{x,y}$ to the integral. By linearity and suitable regularity assumptions (which allow us to interchange differentiation and integration), we have
\[
- \Delta_{x,y} \Phi(x,y)
= - \Delta_{x,y} \int_0^a\!\!\int_0^b G(x,y;\xi,\eta)\,f(\xi,\eta)\,d\eta\,d\xi
= \int_0^a\!\!\int_0^b \bigl(-\Delta_{x,y}G(x,y;\xi,\eta)\bigr)\,f(\xi,\eta)\,d\eta\,d\xi.
\]
By the defining property of $G$,
\[
- \Delta_{x,y}G(x,y;\xi,\eta)
= \delta(x-\xi)\,\delta(y-\eta),
\]
so
\[
- \Delta_{x,y} \Phi(x,y)
= \int_0^a\!\!\int_0^b \delta(x-\xi)\,\delta(y-\eta)\,f(\xi,\eta)\,d\eta\,d\xi
= f(x,y),
\]
since the double delta distribution evaluates the integrand at $(\xi,\eta)=(x,y)$.

Therefore the function
\[
\Phi(x,y) = \int_0^a\!\!\int_0^b G(x,y;\xi,\eta)\,f(\xi,\eta)\,d\eta\,d\xi
\]
satisfies both the differential equation $- \Delta\Phi=f$ in $\Omega$ and the homogeneous Dirichlet boundary condition $\Phi=0$ on $\partial\Omega$. Standard elliptic theory implies that this solution is unique, so the integral representation is indeed the solution of the Poisson problem.

\medskip

\noindent\textbf{Relation to the method of Green functions.}
This example illustrates the main ideas of the method of Green functions for elliptic partial differential equations:

\begin{itemize}
\item We first solve the \emph{homogeneous} eigenvalue problem for the operator (here $- \Delta$) with the given boundary conditions (here Dirichlet on a rectangle), obtaining an orthonormal basis of eigenfunctions and corresponding eigenvalues.

\item We then use the spectral representation of the inverse operator: the Green function is obtained by summing over the eigenfunctions, each weighted by the reciprocal of its eigenvalue. This provides an explicit kernel $G(x,y;\xi,\eta)$ such that applying the inverse operator becomes an integral against $G$.

\item Finally, the solution of the \emph{inhomogeneous} boundary value problem is expressed as a Green function integral, which in this case becomes a double Fourier sine series when $f$ is expanded in the same eigenbasis.
\end{itemize}

In summary, the method of Green functions converts the task of solving a boundary value problem into constructing an integral kernel, here done by eigenfunction expansion. This approach is particularly effective in simple geometries, such as rectangles, where the eigenfunctions are explicitly known and orthogonal.
\end{solution}

% ===== Example 3: Steady-State Heat in a Disk and the Poisson Kernel (inquiry-based) =====
\begin{problem}[Steady-State Heat in a Disk and the Poisson Kernel]
Consider a thin, homogeneous, circular metal plate occupying the unit disk
\[
D := \{ x \in \mathbb{R}^2 : |x| < 1\}.
\]
We assume the plate has reached a steady-state temperature distribution $u(x)$.
On the boundary circle $\partial D$, the temperature is held at a prescribed
profile $g(\theta)$ (where $\theta$ is the polar angle), and inside the plate
there may or may not be heat sources. Our goal is to construct the Green
function for the Laplacian in the unit disk and to see how, from this function,
the classical Poisson kernel and the Poisson integral formula naturally appear.

Throughout, it will be convenient to identify points $x = (x_1,x_2)\in\mathbb{R}^2$
with complex numbers $z = x_1 + i x_2 \in \mathbb{C}$, and similarly write
$\zeta = \xi_1 + i \xi_2$ for another point.

\medskip

(a) {\bf Fundamental solution in the plane.}
In two dimensions, the fundamental solution of the Laplacian is the function
$\Phi:\mathbb{R}^2 \setminus \{0\} \to \mathbb{R}$ satisfying
\[
-\Delta_x \Phi(x) = \delta_0(x),
\]
in the sense of distributions, and decaying suitably at infinity.

\quad(i) Recall or verify that a fundamental solution for the Laplacian in $\mathbb{R}^2$ is
\[
\Phi(x) \;=\; -\frac{1}{2\pi}\log|x|.
\]
Explain in words what it means that $-\Delta_x \Phi(x-\xi) = \delta_\xi(x)$ for a fixed point $\xi\in\mathbb{R}^2$.

\quad(ii) Let us denote
\[
\Phi(x,\xi) := -\frac{1}{2\pi}\log|x-\xi|.
\]
Why is $\Phi(\cdot,\xi)$ harmonic (that is, satisfies $\Delta_x \Phi(x,\xi)=0$) on $\mathbb{R}^2\setminus\{\xi\}$?

\medskip

(b) {\bf From fundamental solution to Green function in a domain.}
For the unit disk $D$, the Green function with zero Dirichlet boundary condition is a function
$G_D(x,\xi)$ such that
\[
\Delta_x G_D(x,\xi) = \delta_\xi(x) \quad\text{in } D,\qquad
G_D(x,\xi)=0 \quad\text{for } x\in \partial D,
\]
for each fixed $\xi\in D$.

\quad(i) Argue that for each fixed $\xi\in D$, any candidate $G_D(\cdot,\xi)$ must look like
\[
G_D(x,\xi) = \Phi(x,\xi) + H(x,\xi),
\]
where $H(\cdot,\xi)$ is harmonic in $D$. Why is it natural to try to construct $H(\cdot,\xi)$ so that $G_D(\cdot,\xi)$ satisfies the boundary condition $G_D(x,\xi)=0$ on $\partial D$?

\quad(ii) Explain why the function $H(\cdot,\xi)$ must be smooth in $\overline{D}$ and cannot introduce any new singularities in $D$.

\medskip

(c) {\bf Guessing and verifying the Green function in the unit disk.}
We now restrict attention to the unit disk $D=\{z\in\mathbb{C}:|z|<1\}$ and write
$z$ for $x$ and $\zeta$ for $\xi$.

\quad(i) Consider the complex-valued function
\[
f(z) = \log(1 - \overline{\zeta}\,z),
\]
for fixed $\zeta\in D$ and variable $z\in D$. Show that $f$ is analytic in $z$ on $D$. Conclude that
\[
h(z,\zeta) := \Re f(z) = \log|1-\overline{\zeta} z|
\]
is harmonic in $z$ on $D$.  (Recall that the real part of a holomorphic function is harmonic.)

\quad(ii) Motivated by part (b), consider the candidate Green function
\[
G_D(z,\zeta) := -\frac{1}{2\pi}\log|z-\zeta| + \frac{1}{2\pi}\log|1-\overline{\zeta} z|.
\]
Using part (a) and the fact that $h(\cdot,\zeta)$ is harmonic in $D$, show that for fixed $\zeta\in D$,
\[
\Delta_z G_D(z,\zeta) = \delta_\zeta(z) \quad \text{in } D.
\]

\quad(iii) Verify the boundary condition.
Let $|z|=1$ (so $z$ is on the unit circle). Show that
\[
|z-\zeta| = |1 - \overline{\zeta}z|
\]
for all $|z|=1$, and deduce that $G_D(z,\zeta)=0$ for all $z\in\partial D$.

\emph{Hint:} Use $|z|=1$ to rewrite $z^{-1} = \overline{z}$ and compare 
$z-\zeta$ with $1-\overline{\zeta}z$ by factoring out $z$.

\medskip

(d) {\bf From the Green function to the Poisson kernel and Poisson integral.}
The Green representation formula for a $C^2$ function $u$ on $\overline{D}$ says that, for each $x\in D$,
\[
u(x) = \int_{\partial D} \Bigl(u(y)\,\frac{\partial G_D}{\partial n_y}(x,y)
      - G_D(x,y)\,\frac{\partial u}{\partial n_y}(y)\Bigr)\,ds_y
      + \int_{D} G_D(x,y)\,(-\Delta u(y))\,dy.
\]
Here $\partial/\partial n_y$ denotes the outward normal derivative at $y\in\partial D$.

\quad(i) Assume $u$ is harmonic in $D$ (so $\Delta u = 0$) and $u=g$ on $\partial D$. Explain why in this case the formula simplifies to
\[
u(x) = \int_{\partial D} g(y)\,\frac{\partial G_D}{\partial n_y}(x,y)\,ds_y.
\]

\quad(ii) Parametrize the boundary $\partial D$ by $y=e^{i\varphi}$, with arc length element $ds_y = d\varphi$, and write $x$ in polar form $x=re^{i\theta}$. Using the explicit formula for $G_D$, compute the outward normal derivative
\[
-\frac{\partial G_D}{\partial n_y}(x,y)\Big|_{|y|=1}
\]
and show that it equals
\[
P(r,\theta-\varphi) := \frac{1-r^2}{1-2r\cos(\theta-\varphi)+r^2},
\]
which is called the \emph{Poisson kernel} of the unit disk.

\emph{Hint:} First write $G_D(re^{i\theta},r'e^{i\varphi})$ with $0<r'<1$ and compute $\partial/\partial r'$ at $r'=1$. You may find it easier to differentiate $\log|re^{i\theta}-r'e^{i\varphi}|^2$ and express the result in terms of $r$, $r'$, and $\theta-\varphi$.

\quad(iii) Deduce the \emph{Poisson integral formula}: for every harmonic function $u$ in $D$ with boundary data $u(e^{i\varphi}) = g(\varphi)$,
\[
u(re^{i\theta})
= \frac{1}{2\pi}\int_0^{2\pi} P(r,\theta-\varphi)\,g(\varphi)\,d\varphi.
\]

\medskip

(e) {\bf Extensions and variations.}

\quad(i) Suppose now that $u$ solves the Poisson equation
\[
-\Delta u = f \quad\text{in } D,\qquad u=g\quad\text{on }\partial D,
\]
with a given source term $f$. Using the Green representation formula and your explicit $G_D$, write down an integral formula for $u$ in terms of $f$, $g$, and $G_D$.

\quad(ii) How do you expect the Green function and Poisson kernel to change if the disk has radius $R>0$ instead of $1$? Describe (without full derivation) how you would modify the formulas for $G_D$ and $P$ when $D=\{x:|x|<R\}$.
\end{problem}

% ===== Example 3: Steady-State Heat in a Disk and the Poisson Kernel (full solution) =====
%%% INQUIRY START %%%
\begin{problem}[Steady-State Heat in a Disk and the Poisson Kernel]
Consider a thin, homogeneous, circular metal plate occupying the unit disk
\[
D := \{ x \in \mathbb{R}^2 : |x| < 1\}.
\]
We assume the plate has reached a steady-state temperature distribution $u(x)$.
On the boundary circle $\partial D$, the temperature is held at a prescribed
profile $g(\theta)$ (where $\theta$ is the polar angle), and inside the plate
there may or may not be heat sources. Our goal is to construct the Green
function for the Laplacian in the unit disk and to see how, from this function,
the classical Poisson kernel and the Poisson integral formula naturally appear.

Throughout, it will be convenient to identify points $x = (x_1,x_2)\in\mathbb{R}^2$
with complex numbers $z = x_1 + i x_2 \in \mathbb{C}$, and similarly write
$\zeta = \xi_1 + i \xi_2$ for another point.

\medskip

(a) {\bf Fundamental solution in the plane.}
In two dimensions, the fundamental solution of the Laplacian is the function
$\Phi:\mathbb{R}^2 \setminus \{0\} \to \mathbb{R}$ satisfying
\[
-\Delta_x \Phi(x) = \delta_0(x),
\]
in the sense of distributions, and decaying suitably at infinity.

\quad(i) Recall or verify that a fundamental solution for the Laplacian in $\mathbb{R}^2$ is
\[
\Phi(x) \;=\; -\frac{1}{2\pi}\log|x|.
\]
Explain in words what it means that $-\Delta_x \Phi(x-\xi) = \delta_\xi(x)$ for a fixed point $\xi\in\mathbb{R}^2$.

\quad(ii) Let us denote
\[
\Phi(x,\xi) := -\frac{1}{2\pi}\log|x-\xi|.
\]
Why is $\Phi(\cdot,\xi)$ harmonic (that is, satisfies $\Delta_x \Phi(x,\xi)=0$) on $\mathbb{R}^2\setminus\{\xi\}$?

\medskip

(b) {\bf From fundamental solution to Green function in a domain.}
For the unit disk $D$, the Green function with zero Dirichlet boundary condition is a function
$G_D(x,\xi)$ such that
\[
\Delta_x G_D(x,\xi) = \delta_\xi(x) \quad\text{in } D,\qquad
G_D(x,\xi)=0 \quad\text{for } x\in \partial D,
\]
for each fixed $\xi\in D$.

\quad(i) Argue that for each fixed $\xi\in D$, any candidate $G_D(\cdot,\xi)$ must look like
\[
G_D(x,\xi) = \Phi(x,\xi) + H(x,\xi),
\]
where $H(\cdot,\xi)$ is harmonic in $D$. Why is it natural to try to construct $H(\cdot,\xi)$ so that $G_D(\cdot,\xi)$ satisfies the boundary condition $G_D(x,\xi)=0$ on $\partial D$?

\quad(ii) Explain why the function $H(\cdot,\xi)$ must be smooth in $\overline{D}$ and cannot introduce any new singularities in $D$.

\medskip

(c) {\bf Guessing and verifying the Green function in the unit disk.}
We now restrict attention to the unit disk $D=\{z\in\mathbb{C}:|z|<1\}$ and write
$z$ for $x$ and $\zeta$ for $\xi$.

\quad(i) Consider the complex-valued function
\[
f(z) = \log(1 - \overline{\zeta}\,z),
\]
for fixed $\zeta\in D$ and variable $z\in D$. Show that $f$ is analytic in $z$ on $D$. Conclude that
\[
h(z,\zeta) := \Re f(z) = \log|1-\overline{\zeta} z|
\]
is harmonic in $z$ on $D$.  (Recall that the real part of a holomorphic function is harmonic.)

\quad(ii) Motivated by part (b), consider the candidate Green function
\[
G_D(z,\zeta) := -\frac{1}{2\pi}\log|z-\zeta| + \frac{1}{2\pi}\log|1-\overline{\zeta} z|.
\]
Using part (a) and the fact that $h(\cdot,\zeta)$ is harmonic in $D$, show that for fixed $\zeta\in D$,
\[
\Delta_z G_D(z,\zeta) = \delta_\zeta(z) \quad \text{in } D.
\]

\quad(iii) Verify the boundary condition.
Let $|z|=1$ (so $z$ is on the unit circle). Show that
\[
|z-\zeta| = |1 - \overline{\zeta}z|
\]
for all $|z|=1$, and deduce that $G_D(z,\zeta)=0$ for all $z\in\partial D$.

\emph{Hint:} Use $|z|=1$ to rewrite $z^{-1} = \overline{z}$ and compare 
$z-\zeta$ with $1-\overline{\zeta}z$ by factoring out $z$.

\medskip

(d) {\bf From the Green function to the Poisson kernel and Poisson integral.}
The Green representation formula for a $C^2$ function $u$ on $\overline{D}$ says that, for each $x\in D$,
\[
u(x) = \int_{\partial D} \Bigl(u(y)\,\frac{\partial G_D}{\partial n_y}(x,y)
      - G_D(x,y)\,\frac{\partial u}{\partial n_y}(y)\Bigr)\,ds_y
      + \int_{D} G_D(x,y)\,(-\Delta u(y))\,dy.
\]
Here $\partial/\partial n_y$ denotes the outward normal derivative at $y\in\partial D$.

\quad(i) Assume $u$ is harmonic in $D$ (so $\Delta u = 0$) and $u=g$ on $\partial D$. Explain why in this case the formula simplifies to
\[
u(x) = \int_{\partial D} g(y)\,\frac{\partial G_D}{\partial n_y}(x,y)\,ds_y.
\]

\quad(ii) Parametrize the boundary $\partial D$ by $y=e^{i\varphi}$, with arc length element $ds_y = d\varphi$, and write $x$ in polar form $x=re^{i\theta}$. Using the explicit formula for $G_D$, compute the outward normal derivative
\[
-\frac{\partial G_D}{\partial n_y}(x,y)\Big|_{|y|=1}
\]
and show that it equals
\[
P(r,\theta-\varphi) := \frac{1-r^2}{1-2r\cos(\theta-\varphi)+r^2},
\]
which is called the \emph{Poisson kernel} of the unit disk.

\emph{Hint:} First write $G_D(re^{i\theta},r'e^{i\varphi})$ with $0<r'<1$ and compute $\partial/\partial r'$ at $r'=1$. You may find it easier to differentiate $\log|re^{i\theta}-r'e^{i\varphi}|^2$ and express the result in terms of $r$, $r'$, and $\theta-\varphi$.

\quad(iii) Deduce the \emph{Poisson integral formula}: for every harmonic function $u$ in $D$ with boundary data $u(e^{i\varphi}) = g(\varphi)$,
\[
u(re^{i\theta})
= \frac{1}{2\pi}\int_0^{2\pi} P(r,\theta-\varphi)\,g(\varphi)\,d\varphi.
\]

\medskip

(e) {\bf Extensions and variations.}

\quad(i) Suppose now that $u$ solves the Poisson equation
\[
-\Delta u = f \quad\text{in } D,\qquad u=g\quad\text{on }\partial D,
\]
with a given source term $f$. Using the Green representation formula and your explicit $G_D$, write down an integral formula for $u$ in terms of $f$, $g$, and $G_D$.

\quad(ii) How do you expect the Green function and Poisson kernel to change if the disk has radius $R>0$ instead of $1$? Describe (without full derivation) how you would modify the formulas for $G_D$ and $P$ when $D=\{x:|x|<R\}$.
\end{problem}
%%% INQUIRY END %%%

%%% SOLUTION START %%%
\begin{problem}[Steady-State Heat in a Disk and the Poisson Kernel]
Let $D = \{x\in\mathbb{R}^2 : |x|<1\}$ be the unit disk. 

(a) Construct the Dirichlet Green function $G_D(x,\xi)$ for the Laplacian in $D$, that is, a function such that for each fixed $\xi\in D$,
\[
\Delta_x G_D(x,\xi) = \delta_\xi(x)\quad\text{in }D,\qquad
G_D(x,\xi) = 0\quad\text{for }x\in\partial D.
\]
Show that, in complex notation $z=x_1+ix_2$, $\zeta=\xi_1+i\xi_2$,
\[
G_D(z,\zeta)
= -\frac{1}{2\pi}\log|z-\zeta| + \frac{1}{2\pi}\log|1-\overline{\zeta} z|.
\]

(b) Using the Green representation formula and part (a), compute the outward normal derivative of $G_D$ on the boundary and deduce that the Poisson kernel of the unit disk is
\[
P(r,\theta-\varphi)
= \frac{1-r^2}{1-2r\cos(\theta-\varphi)+r^2},
\]
for $0\le r<1$ and angles $\theta,\varphi\in\mathbb{R}$. Show that any harmonic function $u$ in $D$ with boundary values $u(e^{i\varphi}) = g(\varphi)$ satisfies the Poisson integral formula
\[
u(re^{i\theta})
= \frac{1}{2\pi}\int_0^{2\pi} P(r,\theta-\varphi)\,g(\varphi)\,d\varphi.
\]

(c) State the corresponding Green representation formula for the Poisson equation
\[
-\Delta u = f \quad\text{in } D,\qquad u=g\quad\text{on }\partial D,
\]
in terms of $G_D$, $f$, and $g$.

Explain briefly how this example illustrates the method of Green functions for elliptic partial differential equations.
\end{problem}

\begin{solution}
We begin by recalling the role of a Green function in solving boundary value problems for elliptic operators. For the Laplacian on a domain $D$, the Dirichlet Green function encodes both the singular behavior (through the fundamental solution) and the boundary condition. Once this function is known, it provides an integral representation of solutions to both homogeneous (Laplace) and inhomogeneous (Poisson) equations.

\medskip

\noindent\textbf{(a) Green function in the unit disk.}
In two dimensions, a fundamental solution of the Laplacian is
\[
\Phi(x) = -\frac{1}{2\pi}\log|x|, \qquad x\in\mathbb{R}^2\setminus\{0\},
\]
in the sense that $-\Delta \Phi = \delta_0$ in the distributional sense. For a fixed point $\xi\in\mathbb{R}^2$, the translated function
\[
\Phi(x,\xi) := -\frac{1}{2\pi}\log|x-\xi|
\]
satisfies $-\Delta_x\Phi(x,\xi) = \delta_\xi(x)$ and is harmonic in $x$ on $\mathbb{R}^2\setminus\{\xi\}$.

Let $D$ be the unit disk and fix $\xi\in D$. A Dirichlet Green function $G_D(\cdot,\xi)$ should satisfy
\[
\Delta_x G_D(x,\xi) = \delta_\xi(x)\quad\text{in }D,\qquad
G_D(x,\xi)=0\quad\text{for }x\in\partial D.
\]
Since $\Phi(\cdot,\xi)$ already has the correct singularity at $x=\xi$, any Green function for $D$ must differ from $\Phi(\cdot,\xi)$ by a function harmonic in $D$. Thus we seek
\[
G_D(x,\xi) = \Phi(x,\xi) + H(x,\xi)
= -\frac{1}{2\pi}\log|x-\xi| + H(x,\xi),
\]
where, for each fixed $\xi\in D$, the function $H(\cdot,\xi)$ is harmonic in $D$ and chosen so that $G_D(\cdot,\xi)$ vanishes on $\partial D$. Since we require $G_D$ to be bounded on $\partial D$ and smooth away from $x=\xi$, we also require $H(\cdot,\xi)$ to be smooth on $\overline{D}$ (it cannot introduce any new singularities inside $D$).

To exploit the geometry of the disk, we pass to complex notation. We identify $x=(x_1,x_2)$ with $z=x_1+ix_2\in\mathbb{C}$ and $\xi=(\xi_1,\xi_2)$ with $\zeta=\xi_1+i\xi_2\in\mathbb{C}$. The Euclidean distance is then $|x-\xi|=|z-\zeta|$. We look for a harmonic correction of the form
\[
H(z,\zeta) = \frac{1}{2\pi}\log|1-\overline{\zeta} z|.
\]

To justify this choice, note first that for fixed $\zeta\in D$, the map
\[
f(z) := \log(1-\overline{\zeta}z)
\]
is holomorphic as a function of $z$ on $D$, since $|\overline{\zeta}z|<1$ for all $z\in D$ and one can take the principal branch of the logarithm near $1$. Therefore the real part
\[
h(z,\zeta) := \Re f(z) = \log|1-\overline{\zeta} z|
\]
is harmonic in $z$ on $D$. Consequently $H(\cdot,\zeta)$ is harmonic on $D$, as required.

We then define
\[
G_D(z,\zeta)
:= -\frac{1}{2\pi}\log|z-\zeta| + \frac{1}{2\pi}\log|1-\overline{\zeta} z|.
\]
By construction, the singular part agrees with the fundamental solution $\Phi(z,\zeta)$ and the correction term is harmonic in $z$. It follows that, for fixed $\zeta\in D$,
\[
\Delta_z G_D(z,\zeta) = \Delta_z\Phi(z,\zeta) + \Delta_z H(z,\zeta)
= \delta_\zeta(z) + 0 = \delta_\zeta(z) \quad\text{in } D.
\]

It remains to check the boundary condition $G_D(z,\zeta)=0$ for $|z|=1$. Let $|z|=1$. We claim that
\[
|z-\zeta| = |1-\overline{\zeta}z|.
\]
Indeed,
\[
1-\overline{\zeta}z = \overline{z}\,(z-\zeta),
\]
because $|z|=1$ implies $\overline{z} = 1/z$ and thus
\[
\overline{z}\,(z-\zeta) = \overline{z}z - \overline{z}\,\zeta
= 1 - \overline{\zeta} z.
\]
Taking moduli and using $|\overline{z}|=1$ gives
\[
|1-\overline{\zeta}z| = |\overline{z}|\,|z-\zeta| = |z-\zeta|.
\]
Therefore on $|z|=1$ we have
\[
G_D(z,\zeta)
= -\frac{1}{2\pi}\log|z-\zeta| + \frac{1}{2\pi}\log|1-\overline{\zeta} z|
= -\frac{1}{2\pi}\log|z-\zeta| + \frac{1}{2\pi}\log|z-\zeta| = 0.
\]
This verifies that $G_D$ satisfies the Dirichlet boundary condition. Thus
\[
G_D(z,\zeta)
= -\frac{1}{2\pi}\log|z-\zeta| + \frac{1}{2\pi}\log|1-\overline{\zeta} z|
\]
is the Dirichlet Green function of the unit disk.

\medskip

\noindent\textbf{(b) Poisson kernel and Poisson integral formula.}
The Green representation formula for the Laplacian on a domain $D$ reads as follows. If $u\in C^2(D)\cap C^1(\overline{D})$, then for each $x\in D$,
\begin{equation}\label{GRF}
u(x)
= \int_{\partial D} \Bigl(u(y)\,\frac{\partial G_D}{\partial n_y}(x,y)
      - G_D(x,y)\,\frac{\partial u}{\partial n_y}(y)\Bigr)\,ds_y
      + \int_{D} G_D(x,y)\,(-\Delta u(y))\,dy,
\end{equation}
where $\partial/\partial n_y$ denotes differentiation in the outward normal direction at $y\in\partial D$ and $ds_y$ is the boundary arc length element.

Suppose now that $u$ is harmonic in $D$, that is, $\Delta u = 0$, and that $u$ has prescribed boundary values $u(y)=g(y)$ on $\partial D$. Then the volume integral in \eqref{GRF} vanishes, and we obtain
\[
u(x)
= \int_{\partial D} \left(g(y)\,\frac{\partial G_D}{\partial n_y}(x,y)
  - G_D(x,y)\,\frac{\partial u}{\partial n_y}(y)\right)\,ds_y.
\]
Since $G_D(x,y) = 0$ for $y\in\partial D$, the term involving $G_D(x,y)\,\partial u/\partial n_y(y)$ also vanishes. Therefore
\[
u(x) = \int_{\partial D} g(y)\,\frac{\partial G_D}{\partial n_y}(x,y)\,ds_y.
\]

Next we parametrize the boundary of the unit disk by
\[
y = e^{i\varphi},\qquad \varphi\in[0,2\pi),
\]
and write $x$ in polar coordinates $x = re^{i\theta}$, with $0\le r<1$. The outward unit normal at $y$ is just $n_y = y$, so the outward normal derivative at $y$ is
\[
\frac{\partial}{\partial n_y} = \nabla_y\cdot n_y = \frac{\partial}{\partial r'}
\quad\text{at } y = r'e^{i\varphi},\; r'=1.
\]
Moreover, the arc length element on the unit circle is $ds_y = d\varphi$.

We now compute the normal derivative of $G_D$ with respect to the boundary variable $y$. To emphasize which variable we differentiate in, we write
\[
G_D\bigl(re^{i\theta}, r'e^{i\varphi}\bigr)
= -\frac{1}{2\pi}\log\bigl|re^{i\theta}-r'e^{i\varphi}\bigr|
  + \frac{1}{2\pi}\log\bigl|1-\overline{r'e^{i\varphi}}\,re^{i\theta}\bigr|.
\]
We have $\overline{r'e^{i\varphi}} = r'e^{-i\varphi}$, so
\[
G_D\bigl(re^{i\theta}, r'e^{i\varphi}\bigr)
= -\frac{1}{4\pi}\log\Bigl(|re^{i\theta}-r'e^{i\varphi}|^2\Bigr)
  + \frac{1}{4\pi}\log\Bigl(|1-r're^{i(\theta-\varphi)}|^2\Bigr).
\]
Set $\alpha := \theta-\varphi$. A direct computation gives
\[
|re^{i\theta}-r'e^{i\varphi}|^2
= r^2 + r'^2 - 2rr'\cos(\theta-\varphi)
= r^2 + r'^2 - 2rr'\cos\alpha,
\]
and
\[
|

% ===== Example 4: Mixed and Robin Boundary Conditions for One-Dimensional Problems (inquiry-based) =====
\begin{problem}[Mixed and Robin Boundary Conditions for One-Dimensional Problems]
Consider a thin, insulated rod occupying the interval $0 < x < L$. The temperature $u(x)$ in steady state satisfies a one-dimensional Poisson equation. Suppose the left end $x=0$ is held at a fixed temperature, while the right end $x=L$ is exposed to a surrounding medium so that heat is lost according to Newton's law of cooling (a Robin boundary condition). In this example, you will construct the Green function that incorporates this mixed system of boundary conditions and then use it to represent the solution.

We study the boundary value problem
\[
- u''(x) = f(x), \qquad 0 < x < L,
\]
with boundary conditions
\[
u(0) = 0, \qquad u'(L) + h\,u(L) = 0,
\]
where $h \ge 0$ is a given constant (the heat transfer coefficient).

\medskip

(a) \textbf{Warm-up: homogeneous solutions and boundary conditions.}  
Consider the associated homogeneous equation
\[
- u''(x) = 0, \qquad 0 < x < L.
\]
\begin{itemize}
  \item[(i)] Find the general solution of $-u''(x) = 0$.
  \item[(ii)] Construct a special solution $\phi(x)$ that satisfies the \emph{left} boundary condition and a convenient normalization:
  \[
  \phi(0) = 0, \qquad \phi'(0) = 1.
  \]
  \item[(iii)] Construct a second special solution $\psi(x)$ that satisfies the \emph{right} boundary condition
  \[
  \psi'(L) + h\,\psi(L) = 0,
  \]
  together with the normalization
  \[
  \psi(L) = 1.
  \]
\end{itemize}
Hint: Use your general solution from part (i) and impose the conditions one by one.

\medskip

(b) \textbf{Setting up the Green function piecewise.}  
For a fixed source point $\xi \in (0,L)$, the Green function $G(x,\xi)$ should satisfy
\[
- \frac{\partial^2}{\partial x^2} G(x,\xi) = \delta(x - \xi),
\]
with the same boundary conditions in the $x$-variable:
\[
G(0,\xi) = 0, \qquad G_x(L,\xi) + h\,G(L,\xi) = 0.
\]
Away from $x = \xi$, the function $G(\cdot,\xi)$ solves the homogeneous equation.
\begin{itemize}
  \item[(i)] Argue that for $x \neq \xi$ the function $G(x,\xi)$ must be a solution of $-G''(x,\xi) = 0$, and therefore is linear in $x$ on each of the intervals $(0,\xi)$ and $(\xi,L)$.
  \item[(ii)] Explain why we may write
  \[
  G(x,\xi) =
  \begin{cases}
    A(\xi)\,x + B(\xi), & 0 \le x < \xi,\\[0.5ex]
    C(\xi)\,x + D(\xi), & \xi < x \le L,
  \end{cases}
  \]
  for some coefficients $A(\xi)$, $B(\xi)$, $C(\xi)$, and $D(\xi)$ depending on $\xi$.
\end{itemize}
Hint: The dependence on $\xi$ is parametric; treat $\xi$ as temporarily fixed.

\medskip

(c) \textbf{Matching conditions at the source point.}  
The Green function is continuous, but its derivative has a jump determined by the delta function.
\begin{itemize}
  \item[(i)] Impose continuity of $G$ at $x = \xi$:
  \[
  G(\xi^-,\xi) = G(\xi^+,\xi).
  \]
  Translate this into an equation relating $A(\xi)$, $B(\xi)$, $C(\xi)$, and $D(\xi)$.
  \item[(ii)] Derive the jump condition for the derivative $G_x$ at $x = \xi$ by integrating the differential equation
  \[
  - G''(x,\xi) = \delta(x - \xi)
  \]
  across a small interval $(\xi-\varepsilon,\xi+\varepsilon)$ and letting $\varepsilon \to 0^+$. Show that
  \[
  G_x(\xi^+,\xi) - G_x(\xi^-,\xi) = -1.
  \]
  Express this as an equation relating $A(\xi)$ and $C(\xi)$.
\end{itemize}
Hint: Remember that $\displaystyle \int_{\xi-\varepsilon}^{\xi+\varepsilon} \delta(x-\xi)\,dx = 1$.

\medskip

(d) \textbf{Using boundary conditions and solving for the coefficients.}  
Now use the boundary conditions in $x$ together with the matching conditions to determine the unknown coefficients.
\begin{itemize}
  \item[(i)] Apply the boundary condition at $x=0$ to the left piece of $G(x,\xi)$ and deduce a relation between $A(\xi)$ and $B(\xi)$.
  \item[(ii)] Apply the Robin boundary condition at $x=L$ to the right piece of $G(x,\xi)$ and obtain an equation relating $C(\xi)$ and $D(\xi)$.
  \item[(iii)] Combine \emph{all four} equations (from continuity, jump condition, and the two boundary conditions) to solve for $A(\xi)$, $B(\xi)$, $C(\xi)$, and $D(\xi)$ explicitly.
\end{itemize}
Hint: It may be helpful to eliminate $B(\xi)$ and $D(\xi)$ first, leaving a $2\times 2$ system for $A(\xi)$ and $C(\xi)$.

Once you have found $G(x,\xi)$ in explicit form, write the Green representation formula for the solution:
\[
u(x) = \int_0^L G(x,\xi)\,f(\xi)\,d\xi.
\]

\medskip

(e) \textbf{Extensions and limiting cases.}
\begin{itemize}
  \item[(i)] What happens to your Green function when $h = 0$? Interpret the resulting boundary value problem and verify that the formula simplifies to the corresponding mixed (Dirichlet--Neumann) case.
  \item[(ii)] Consider the limit $h \to \infty$ (very strong Newton cooling at $x=L$). Show formally that the Green function tends to the Dirichlet--Dirichlet Green function on $(0,L)$, and explain the physical meaning of this limit.
\end{itemize}
Hint: For $h \to \infty$, factor out $h$ from numerator and denominator of your expression for $G(x,\xi)$ before taking the limit.
\end{problem}

% ===== Example 4: Mixed and Robin Boundary Conditions for One-Dimensional Problems (full solution) =====
\begin{problem}[Mixed and Robin Boundary Conditions for One-Dimensional Problems]
Consider the boundary value problem
\[
- u''(x) = f(x), \qquad 0 < x < L,
\]
with mixed boundary conditions
\[
u(0) = 0, \qquad u'(L) + h\,u(L) = 0,
\]
where $h \ge 0$ is a constant. 

(a) Construct the Green function $G(x,\xi)$ satisfying
\[
- \frac{\partial^2}{\partial x^2} G(x,\xi) = \delta(x - \xi), \quad 0 < x,\xi < L,
\]
together with
\[
G(0,\xi) = 0, \qquad G_x(L,\xi) + h\,G(L,\xi) = 0.
\]

(b) Use $G$ to derive an integral representation
\[
u(x) = \int_0^L G(x,\xi)\,f(\xi)\,d\xi
\]
for the solution $u$.

(c) Show that when $h=0$ your Green function reduces to the Green function for the Dirichlet--Neumann problem $u(0)=0$, $u'(L)=0$, and that in the limit $h \to \infty$ it converges to the Dirichlet--Dirichlet Green function on $(0,L)$.
\end{problem}

\begin{solution}
We are asked to construct the Green function for the one-dimensional operator
\[
L u := -u''(x)
\]
on the interval $(0,L)$, with a Dirichlet condition at the left endpoint and a Robin (Newton cooling) condition at the right endpoint. This is a simple but instructive example of how Green functions adjust to mixed boundary conditions and illustrates the flexibility of the Green function method for elliptic-type problems.

\medskip

\textbf{1. Homogeneous solutions and general structure of the Green function.}

We first analyze the associated homogeneous equation
\[
- u''(x) = 0 \quad \Longleftrightarrow \quad u''(x) = 0.
\]
Integrating twice shows that every solution has the form
\[
u(x) = a x + b,
\]
where $a$ and $b$ are constants.

For a fixed source point $\xi \in (0,L)$, the Green function $G(\cdot,\xi)$ satisfies
\[
- G''(x,\xi) = \delta(x - \xi), \qquad 0 < x < L,
\]
with boundary conditions
\[
G(0,\xi) = 0, \qquad G_x(L,\xi) + h\,G(L,\xi) = 0.
\]
Thus, for $x \neq \xi$, the right-hand side vanishes and $G(\cdot,\xi)$ solves the homogeneous equation, hence is linear on each side of $x = \xi$.

Accordingly, for each fixed $\xi$ we may write
\[
G(x,\xi) =
\begin{cases}
A(\xi)\,x + B(\xi), & 0 \le x < \xi,\\[0.5ex]
C(\xi)\,x + D(\xi), & \xi < x \le L,
\end{cases}
\]
for some coefficient functions $A(\xi)$, $B(\xi)$, $C(\xi)$, and $D(\xi)$ yet to be determined.

\medskip

\textbf{2. Boundary conditions in the $x$-variable.}

We next impose the boundary conditions.

\emph{At $x=0$}, we use $G(0,\xi) = 0$. Evaluating the left-hand piece at $x=0$ gives
\[
G(0,\xi) = A(\xi)\cdot 0 + B(\xi) = B(\xi),
\]
hence
\[
B(\xi) = 0.
\]
Therefore, on $(0,\xi)$ we have the simpler expression
\[
G(x,\xi) = A(\xi)\,x, \quad 0 \le x < \xi.
\]

\emph{At $x=L$}, we use the Robin boundary condition in the $x$-variable:
\[
G_x(L,\xi) + h\,G(L,\xi) = 0.
\]
On the right-hand interval $(\xi,L)$, we have
\[
G(x,\xi) = C(\xi)\,x + D(\xi), \qquad
G_x(x,\xi) = C(\xi).
\]
Evaluating at $x=L$ yields
\[
C(\xi) + h\big( C(\xi)\,L + D(\xi) \big) = 0,
\]
or equivalently,
\[
(1 + hL)\,C(\xi) + h\,D(\xi) = 0.
\]
This gives one linear relation between $C(\xi)$ and $D(\xi)$.

\medskip

\textbf{3. Matching conditions at the source point $x=\xi$.}

The defining equation $-G'' = \delta$ imposes two additional conditions at $x = \xi$.

\emph{Continuity of $G$.}  
The Green function is continuous at $x=\xi$:
\[
G(\xi^-,\xi) = G(\xi^+,\xi).
\]
Using the piecewise form,
\[
A(\xi)\,\xi = C(\xi)\,\xi + D(\xi),
\]
so
\[
D(\xi) = A(\xi)\,\xi - C(\xi)\,\xi = (A(\xi) - C(\xi))\,\xi.
\]

\emph{Jump in the derivative.}  
To obtain the jump in $G_x$, integrate the differential equation across a small interval about $\xi$:
\[
\int_{\xi-\varepsilon}^{\xi+\varepsilon} \big(- G''(x,\xi)\big)\,dx
= \int_{\xi-\varepsilon}^{\xi+\varepsilon} \delta(x-\xi)\,dx = 1.
\]
The left-hand side integrates to
\[
- \big[ G_x(x,\xi) \big]_{x=\xi-\varepsilon}^{x=\xi+\varepsilon}
= - \big( G_x(\xi^+,\xi) - G_x(\xi^-,\xi) \big).
\]
Taking the limit $\varepsilon \to 0^+$, we obtain
\[
- \big( G_x(\xi^+,\xi) - G_x(\xi^-,\xi) \big) = 1
\quad \Longrightarrow \quad
G_x(\xi^+,\xi) - G_x(\xi^-,\xi) = -1.
\]
In terms of the coefficients, this becomes
\[
C(\xi) - A(\xi) = -1.
\]

\medskip

\textbf{4. Solving for the coefficients.}

We now have four equations for the four unknowns:
\begin{align*}
&\text{(i)} & B(\xi) &= 0,\\
&\text{(ii)} & D(\xi) &= (A(\xi) - C(\xi))\,\xi,\\
&\text{(iii)} & C(\xi) - A(\xi) &= -1,\\
&\text{(iv)} & (1 + hL)\,C(\xi) + h\,D(\xi) &= 0.
\end{align*}
Using (iii), we can write $C(\xi) = A(\xi) - 1$. Substituting this into (ii) gives
\[
D(\xi) = \big(A(\xi) - (A(\xi)-1)\big)\,\xi = \xi.
\]
Thus $D(\xi)$ is simply $\xi$, independent of $A(\xi)$.

Substituting $C(\xi) = A(\xi) - 1$ and $D(\xi) = \xi$ into (iv) yields
\[
(1 + hL)\big(A(\xi) - 1\big) + h\,\xi = 0.
\]
Solving for $A(\xi)$,
\[
(1 + hL)\,A(\xi) - (1 + hL) + h\,\xi = 0,
\]
so
\[
A(\xi) = \frac{1 + hL - h\xi}{1 + hL}.
\]
Then
\[
C(\xi) = A(\xi) - 1
= \frac{1 + hL - h\xi}{1 + hL} - 1
= -\,\frac{h\xi}{1 + hL},
\]
and we already have $B(\xi) = 0$ and $D(\xi) = \xi$.

Therefore the Green function is
\[
G(x,\xi) =
\begin{cases}
A(\xi)\,x, & 0 \le x \le \xi,\\[0.5ex]
C(\xi)\,x + D(\xi), & \xi \le x \le L.
\end{cases}
\]
Substituting the expressions for $A(\xi)$, $C(\xi)$, and $D(\xi)$ gives the explicit form
\[
G(x,\xi) =
\begin{cases}
\dfrac{1 + hL - h\xi}{1 + hL}\,x, & 0 \le x \le \xi \le L,\\[1.2ex]
-\,\dfrac{h\xi}{1 + hL}\,x + \xi, & 0 \le \xi \le x \le L.
\end{cases}
\]

It is often convenient to write this in a more symmetric form. For $x \le \xi$,
\[
G(x,\xi) = x\,\frac{1 + hL - h\xi}{1 + hL}
= x\,\frac{1 + h(L - \xi)}{1 + hL},
\]
and for $x \ge \xi$,
\[
G(x,\xi) = \xi - \frac{h\xi}{1 + hL}\,x
= \xi\,\frac{1 + hL - h x}{1 + hL}
= \xi\,\frac{1 + h(L - x)}{1 + hL}.
\]
Thus a compact symmetric expression is
\[
G(x,\xi) =
\begin{cases}
\dfrac{x\big(1 + hL - h\xi\big)}
\[
G(x,\xi) =
\begin{cases}
\dfrac{x\big(1 + hL - h\xi\big)}{1 + hL}, & 0 \le x \le \xi \le L,\\[1.2ex]
\dfrac{\xi\big(1 + hL - h x\big)}{1 + hL}, & 0 \le \xi \le x \le L.
\end{cases}
\]
Equivalently,
\[
G(x,\xi) =
\begin{cases}
\dfrac{x\big[1 + h(L-\xi)\big]}{1 + hL}, & 0 \le x \le \xi \le L,\\[1.2ex]
\dfrac{\xi\big[1 + h(L-x)\big]}{1 + hL}, & 0 \le \xi \le x \le L,
\end{cases}
\]
and one easily checks that $G(x,\xi) = G(\xi,x)$ (symmetry of the Green function for this self-adjoint problem).

\medskip

\textbf{5. Integral representation for the solution.}

Given $f$, define
\[
u(x) := \int_0^L G(x,\xi)\,f(\xi)\,d\xi.
\]
Because $-G_{xx}(x,\xi) = \delta(x-\xi)$ in the sense of distributions, we obtain
\[
- u''(x)
= - \int_0^L G_{xx}(x,\xi)\,f(\xi)\,d\xi
= \int_0^L \delta(x-\xi)\,f(\xi)\,d\xi
= f(x),
\]
so $u$ solves $-u''=f$.

At the boundaries, the conditions hold because they are imposed pointwise on $G$:

- At $x=0$, $G(0,\xi)=0$ for all $\xi$, hence
  \[
  u(0) = \int_0^L G(0,\xi)\,f(\xi)\,d\xi = 0.
  \]
- At $x=L$, we have $G_x(L,\xi) + h\,G(L,\xi)=0$ for all $\xi$, so
  \[
  u'(L) + h\,u(L)
  = \int_0^L \big(G_x(L,\xi) + h\,G(L,\xi)\big)\,f(\xi)\,d\xi = 0.
  \]

Thus
\[
u(x) = \int_0^L G(x,\xi)\,f(\xi)\,d\xi
\]
is the desired Green representation of the solution.

\medskip

\textbf{6. Limiting cases in the Robin parameter $h$.}

\emph{(i) Case $h=0$ (Dirichlet--Neumann).}

Setting $h=0$ in the above formula,
\[
G(x,\xi) =
\begin{cases}
\dfrac{x(1 + 0)}{1 + 0} = x, & 0 \le x \le \xi \le L,\\[0.8ex]
\dfrac{\xi(1 + 0)}{1 + 0} = \xi, & 0 \le \xi \le x \le L.
\end{cases}
\]
So
\[
G(x,\xi) =
\begin{cases}
x, & x \le \xi,\\
\xi, & \xi \le x,
\end{cases}
\]
which is exactly the Green function for
\[
- u'' = f, \qquad u(0)=0,\quad u'(L)=0
\]
(the Dirichlet--Neumann problem).

\medskip

\emph{(ii) Limit $h \to \infty$ (Dirichlet--Dirichlet).}

For $h>0$, write
\[
G(x,\xi) =
\begin{cases}
\dfrac{x\big(1 + hL - h\xi\big)}{1 + hL}, & x \le \xi,\\[1.2ex]
\dfrac{\xi\big(1 + hL - h x\big)}{1 + hL}, & x \ge \xi.
\end{cases}
\]
Factor out $h$ from numerator and denominator. For $x \le \xi$,
\[
G(x,\xi)
= x\,\frac{1 + h(L-\xi)}{1 + hL}
= x\,\frac{h(L-\xi)\big(1 + \tfrac{1}{h(L-\xi)}\big)}{hL\big(1 + \tfrac{1}{hL}\big)}
\;\xrightarrow[h\to\infty]{}\; x\,\frac{L-\xi}{L}.
\]
Similarly, for $x \ge \xi$,
\[
G(x,\xi)
= \xi\,\frac{1 + h(L-x)}{1 + hL}
\xrightarrow[h\to\infty]{}\; \xi\,\frac{L-x}{L}.
\]
Therefore,
\[
\lim_{h\to\infty} G(x,\xi)
=
\begin{cases}
\dfrac{x(L-\xi)}{L}, & x \le \xi,\\[0.8ex]
\dfrac{\xi(L-x)}{L}, & x \ge \xi,
\end{cases}
\]
which is precisely the Green function for the Dirichlet--Dirichlet problem
\[
- u'' = f, \qquad u(0) = 0,\quad u(L) = 0.
\]

Physically, increasing $h$ strengthens the heat loss at $x=L$ (Newton cooling). In the limit $h\to\infty$, the endpoint is forced to remain at the ambient temperature (here taken as $0$), so the Robin boundary condition behaves like a Dirichlet condition $u(L)=0$, as reflected by the limiting Green function.

\end{solution}

% ===== Example 5: Green Functions, Resolvents, and Eigenfunction Expansions (inquiry-based) =====
\begin{problem}[Green Functions, Resolvents, and Eigenfunction Expansions]
In this problem we connect three viewpoints on solving elliptic boundary value problems: eigenfunction expansions, resolvents of self-adjoint operators, and Green functions. We work in a concrete one-dimensional setting, but the ideas extend to general self-adjoint elliptic operators on bounded domains. The main goal is to see that the Green function is nothing but the integral kernel of the inverse operator, written in an eigenfunction basis.

Consider the Dirichlet boundary value problem on the interval $(0,\pi)$:
\[
\begin{cases}
- u''(x) = f(x), & 0 < x < \pi,\\[4pt]
u(0) = 0,\quad u(\pi) = 0,
\end{cases}
\]
where $f \in L^2(0,\pi)$ is given. Let $L$ denote the differential operator
\[
L u := -u'', \qquad \mathcal{D}(L) := H^2(0,\pi) \cap H_0^1(0,\pi),
\]
viewed as an unbounded operator on the Hilbert space $L^2(0,\pi)$.

\medskip

(a) First, recall the spectral data of $L$.

\quad(i) Solve the eigenvalue problem
\[
- \phi''(x) = \lambda \,\phi(x), \qquad \phi(0) = \phi(\pi) = 0,
\]
and find all eigenvalues $\lambda_n$ and corresponding eigenfunctions $\phi_n$.

\quad(ii) Normalize your eigenfunctions to obtain an orthonormal family $\{\phi_n\}_{n\ge 1}$ in $L^2(0,\pi)$. State the orthonormality relation explicitly as an integral.

\quad(iii) Without giving a full proof, recall (or briefly justify) why the family $\{\phi_n\}_{n\ge 1}$ is complete in $L^2(0,\pi)$; that is, every $f \in L^2(0,\pi)$ admits a sine series expansion in this basis.

\medskip

(b) Use the eigenfunctions to solve the boundary value problem.

\quad(i) Let $f \in L^2(0,\pi)$. Using part (a), write $f$ as a Fourier sine series
\[
f(x) = \sum_{n=1}^{\infty} f_n \,\phi_n(x),
\]
and express the coefficients $f_n$ in terms of $f$ and $\phi_n$.

\quad(ii) Look for a solution of $L u = f$ in the form
\[
u(x) = \sum_{n=1}^{\infty} u_n \,\phi_n(x).
\]
Insert this series into the equation $-u'' = f$ and use the fact that $L\phi_n = \lambda_n \phi_n$ to derive an explicit formula for $u_n$ in terms of $f_n$ and $\lambda_n$.

\quad(iii) Conclude that the (formal) solution can be written as
\[
u(x) = \sum_{n=1}^{\infty} \frac{f_n}{\lambda_n}\,\phi_n(x)
      = \sum_{n=1}^{\infty} \frac{\langle f,\phi_n\rangle}{\lambda_n}\,\phi_n(x),
\]
where $\langle \cdot,\cdot\rangle$ denotes the $L^2(0,\pi)$ inner product. How does this formula show that the operator $L$ is invertible and that its inverse $L^{-1}$ is diagonal in the eigenbasis?

\medskip

(c) Now we introduce the Green function as the kernel of the inverse operator.

Suppose there exists a function $G : (0,\pi)\times(0,\pi) \to \mathbb{R}$ (the Green function) with the property that, for each $f \in L^2(0,\pi)$, the function
\[
u(x) := \int_{0}^{\pi} G(x,y)\,f(y)\,dy
\]
solves $L u = f$ with $u(0)=u(\pi)=0$.

\quad(i) Motivated by the eigenfunction expansion in part (b), define
\[
G(x,y) := \sum_{n=1}^{\infty} \frac{\phi_n(x)\,\phi_n(y)}{\lambda_n}.
\]
Assuming that this series converges in a suitable sense, compute
\[
\int_{0}^{\pi} G(x,y)\,f(y)\,dy
\]
by interchanging sum and integral, and using orthonormality of $\{\phi_n\}$. Show that you recover exactly the series for $u(x)$ obtained in part (b).

\emph{Hint:} Write $f(y)$ as $\sum f_n \phi_n(y)$, and use that
\[
\int_{0}^{\pi} \phi_n(y)\,\phi_m(y)\,dy = \delta_{nm}.
\]

\quad(ii) Explain why this shows that $G$ is the \emph{integral kernel} of the inverse operator
\[
L^{-1} : L^2(0,\pi) \to \mathcal{D}(L) \subset L^2(0,\pi).
\]
In other words, express $L^{-1}f$ both as a series in the eigenbasis and as an integral operator with kernel $G$, and compare.

\medskip

(d) Interpreting the Green function as a solution with a point source.

Fix a point $x \in (0,\pi)$, and consider the function of $y$
\[
v_x(y) := G(x,y).
\]

\quad(i) Using the eigenfunction expansion of $G(x,y)$, compute $L_y v_x(y)$, where $L_y$ means that $L$ acts on the $y$-variable. Show that, for any eigenfunction $\phi_n$,
\[
\int_{0}^{\pi} (-v_x''(y))\,\phi_n(y)\,dy = \phi_n(x).
\]

\emph{Hint:} Use that $- \phi_n''(y) = \lambda_n \phi_n(y)$ and integrate term-by-term.

\quad(ii) Argue that the identity in part (d)(i) implies
\[
- \frac{d^2}{dy^2} G(x,y) = \delta_x(y)
\]
in the sense of distributions, where $\delta_x$ is the Dirac mass at $y=x$, and that $G(x,0)=G(x,\pi)=0$. Explain how this connects with the usual definition of a Green function as a solution of $L_y G(x,y) = \delta_x(y)$ with homogeneous boundary conditions.

\medskip

(e) Extensions and variations.

\quad(i) Suppose we replace $L$ by a more general self-adjoint Sturm--Liouville operator
\[
A u := -u''(x) + q(x) u(x),
\]
with $u(0)=u(\pi)=0$ and a smooth potential $q(x)$. Assume $A$ has a discrete set of eigenvalues $\mu_1 \le \mu_2 \le \cdots$ with corresponding orthonormal eigenfunctions $\psi_1,\psi_2,\dots$ forming a basis of $L^2(0,\pi)$. Based on what you have done above, what would be the natural candidate for the Green function $G_A(x,y)$ of $A$? Write down a formal series for $G_A$ in terms of the eigenpairs $(\mu_n,\psi_n)$.

\quad(ii) In our example, all eigenvalues $\lambda_n$ of $L$ are strictly positive. What do you expect to change if the operator has a zero eigenvalue? As a concrete case, consider the Neumann problem
\[
- u''(x) = f(x), \qquad u'(0)=u'(\pi)=0.
\]
What is the eigenfunction associated with the eigenvalue $0$, and what compatibility condition on $f$ is needed to invert the operator? How would the presence of $\lambda=0$ affect the Green function expansion?

\end{problem}

% ===== Example 5: Green Functions, Resolvents, and Eigenfunction Expansions (full solution) =====
\begin{problem}[Green Functions, Resolvents, and Eigenfunction Expansions]
Consider the Dirichlet problem on $(0,\pi)$
\[
\begin{cases}
- u''(x) = f(x), & 0 < x < \pi,\\[4pt]
u(0) = u(\pi) = 0,
\end{cases}
\]
with $f \in L^2(0,\pi)$. Let $L u := -u''$ with domain $\mathcal{D}(L)=H^2(0,\pi)\cap H_0^1(0,\pi)$, viewed as a self-adjoint operator on $L^2(0,\pi)$.

(a) Find the eigenvalues $\lambda_n$ and an orthonormal basis of eigenfunctions $\{\phi_n\}_{n\ge 1}$ of $L$ in $L^2(0,\pi)$.

(b) Show that for each $f \in L^2(0,\pi)$, the unique solution $u$ of $Lu=f$ with $u(0)=u(\pi)=0$ is given by the convergent series
\[
u(x) = \sum_{n=1}^{\infty} \frac{\langle f,\phi_n\rangle_{L^2}}{\lambda_n}\,\phi_n(x).
\]

(c) Define the function
\[
G(x,y) := \sum_{n=1}^{\infty} \frac{\phi_n(x)\,\phi_n(y)}{\lambda_n}, \qquad 0<x,y<\pi.
\]
Show that, for every $f \in L^2(0,\pi)$,
\[
u(x) := \int_{0}^{\pi} G(x,y)\,f(y)\,dy
\]
equals the series in part (b), and hence $G$ is the integral kernel of $L^{-1}$.

(d) For fixed $x$, regard $G(x,\cdot)$ as a function of $y$. Show that
\[
- \frac{d^2}{dy^2} G(x,y) = \delta_x(y)
\]
in the sense of distributions, with $G(x,0)=G(x,\pi)=0$. Thus $G$ is the Green function of $L$ with Dirichlet boundary conditions.

Briefly comment on how this eigenfunction expansion of the Green function illustrates the general relationship between Green functions, resolvents, and eigenfunction expansions for self-adjoint elliptic operators on bounded domains.
\end{problem}

\begin{solution}
We solve the boundary value problem by exploiting the spectral properties of the self-adjoint operator $L$ and then interpret the Green function as the kernel of $L^{-1}$ written in an eigenfunction basis. This connects the integral representation of solutions with the spectral decomposition of $L$.

\medskip

\textbf{(a) Eigenvalues and eigenfunctions of $L$.}

We solve the eigenvalue problem
\[
- \phi''(x) = \lambda\,\phi(x), \qquad \phi(0)=\phi(\pi)=0.
\]
There are three classical cases.

\emph{Case 1: $\lambda = 0$.} Then $\phi''(x)=0$, so $\phi(x)=ax+b$. The boundary conditions $\phi(0)=0$ and $\phi(\pi)=0$ give $b=0$ and $a\pi=0$, so $a=0$. Hence $\phi\equiv0$ is the only solution, which we discard. Thus $\lambda=0$ is not an eigenvalue for Dirichlet boundary conditions.

\emph{Case 2: $\lambda < 0$.} Write $\lambda = -\mu^2$ with $\mu>0$. The equation becomes $\phi''=\mu^2\phi$, whose general solution is $\phi(x)=A e^{\mu x}+B e^{-\mu x}$. The boundary conditions give
\[
\phi(0)=A+B=0,\qquad
\phi(\pi) = A e^{\mu\pi}+B e^{-\mu\pi}=0.
\]
From $A=-B$, the second condition becomes $A(e^{\mu\pi}-e^{-\mu\pi})=0$, so $A=0$ and again $\phi\equiv0$. Thus no negative eigenvalues exist.

\emph{Case 3: $\lambda > 0$.} Write $\lambda = \alpha^2$ with $\alpha>0$. The equation becomes $\phi''=-\alpha^2\phi$, with general solution $\phi(x)=A\cos(\alpha x)+B\sin(\alpha x)$. The boundary conditions yield
\[
\phi(0)=A=0,\qquad
\phi(\pi) = B \sin(\alpha\pi) = 0.
\]
For a nontrivial solution we require $B\neq 0$ and hence $\sin(\alpha\pi)=0$, which implies $\alpha\pi = n\pi$ for some integer $n\ge 1$. Thus $\alpha=n$ and
\[
\lambda_n = n^2,\qquad
\phi_n(x) = B \sin(nx), \quad n=1,2,\dots.
\]
We normalize these eigenfunctions in $L^2(0,\pi)$. The $L^2$ norm is
\[
\int_{0}^{\pi} \sin^2(nx)\,dx = \frac{\pi}{2},
\]
independent of $n$. Choosing $B = \sqrt{2/\pi}$, we obtain an orthonormal family
\[
\phi_n(x) = \sqrt{\frac{2}{\pi}}\sin(nx), \qquad n=1,2,\dots,
\]
satisfying
\[
\int_{0}^{\pi} \phi_n(x)\,\phi_m(x)\,dx = \delta_{nm}.
\]

It is a standard result of Fourier analysis that the set $\{\phi_n\}_{n\ge1}$ is complete in $L^2(0,\pi)$: every $f\in L^2(0,\pi)$ admits a Fourier sine series expansion in this orthonormal basis. Equivalently, the closure of the span of $\{\phi_n\}$ is all of $L^2(0,\pi)$.

\medskip

\textbf{(b) Solving $Lu=f$ by eigenfunction expansion.}

Let $f\in L^2(0,\pi)$. Since $\{\phi_n\}$ is an orthonormal basis, $f$ has the expansion
\[
f(x) = \sum_{n=1}^{\infty} f_n\,\phi_n(x),
\quad\text{where}\quad
f_n = \langle f,\phi_n\rangle_{L^2} = \int_{0}^{\pi} f(y)\,\phi_n(y)\,dy.
\]

We look for a solution $u$ of
\[
- u''(x) = f(x),\qquad u(0)=u(\pi)=0,
\]
in the form
\[
u(x) = \sum_{n=1}^{\infty} u_n\,\phi_n(x)
\]
for some coefficients $\{u_n\}$ to be determined.

Because $L\phi_n = -\phi_n'' = \lambda_n \phi_n$ with $\lambda_n=n^2$, linearity gives
\[
L u = -u'' = \sum_{n=1}^{\infty} u_n\,L\phi_n
           = \sum_{n=1}^{\infty} u_n\,\lambda_n\,\phi_n.
\]
We want $Lu=f = \sum f_n\phi_n$. Equality of two $L^2$-convergent expansions in an orthonormal basis implies equality of coefficients:
\[
u_n\,\lambda_n = f_n \quad\text{for all }n.
\]
Since $\lambda_n>0$, we obtain
\[
u_n = \frac{f_n}{\lambda_n} = \frac{\langle f,\phi_n\rangle_{L^2}}{\lambda_n}.
\]
Thus the solution is
\[
u(x) = \sum_{n=1}^{\infty} \frac{f_n}{\lambda_n}\,\phi_n(x)
     = \sum_{n=1}^{\infty} \frac{\langle f,\phi_n\rangle_{L^2}}{\lambda_n}\,\phi_n(x).
\]

This series converges in the appropriate Sobolev space (indeed in $H_0^1(0,\pi)\cap H^2(0,\pi)$), and $u$ is the unique weak (and in fact classical) solution of the boundary value problem. From the operator viewpoint, this shows that $L$ is invertible from $\mathcal{D}(L)$ onto $L^2(0,\pi)$, and that its inverse $L^{-1}$ acts diagonally on the eigenbasis:
\[
L^{-1}\phi_n = \frac{1}{\lambda_n}\,\phi_n.
\]
Hence $L^{-1}$ is precisely the operator that multiplies each Fourier sine coefficient by $1/\lambda_n$.

\medskip

\textbf{(c) The Green function as the kernel of $L^{-1}$.}

We now define
\[
G(x,y) := \sum_{n=1}^{\infty} \frac{\phi_n(x)\,\phi_n(y)}{\lambda_n},
\qquad 0<x,y<\pi.
\]
The series converges in $L^2((0,\pi)\times(0,\pi))$, and in fact defines a continuous function away from the diagonal $x=y$, but for our purposes the $L^2$ convergence is sufficient.

Let $f\in L^2(0,\pi)$, and define
\[
u(x) := \int_{0}^{\pi} G(x,y)\,f(y)\,dy.
\]
We claim that this integral expression exactly reproduces the series from part (b). Formally,
\[
\begin{aligned}
u(x)
&= \int_{0}^{\pi} \left( \sum_{n=1}^{\infty} \frac{\phi_n(x)\,\phi_n(y)}{\lambda_n} \right) f(y)\,dy \\
&= \sum_{n=1}^{\infty} \frac{\phi_n(x)}{\lambda_n} \int_{0}^{\pi} \phi_n(y)\,f(y)\,dy
   \quad\text{(interchanging sum and integral)}\\
&= \sum_{n=1}^{\infty} \frac{\langle f,\phi_n\rangle_{L^2}}{\lambda_n}\,\phi_n(x).
\end{aligned}
\]
The interchange of sum and integral is justified by standard theorems (e.g.\ Fubini/Tonelli) using the square-summability of the coefficients and the boundedness of the eigenfunctions.

The final series is precisely the expression for $u$ found in part (b). Therefore, for every $f\in L^2(0,\pi)$,
\[
(L^{-1} f)(x)
= \sum_{n=1}^{\infty} \frac{\langle f,\phi_n\rangle}{\lambda_n}\,\phi_n(x)
= \int_{0}^{\pi} G(x,y)\,f(y)\,dy.
\]
In other words, $G$ is the \emph{integral kernel} of the inverse operator $L^{-1}$. This is exactly the resolvent operator at zero, since here we are inverting $L$ itself.

\medskip

\textbf{(d) $G$ solves $L_y G(x,y) = \delta_x(y)$.}

We now check that $G$ satisfies the defining equation of a Green function. Fix $x\in(0,\pi)$, and consider
\[
v_x(y) := G(x,y) = \sum_{n=1}^{\infty} \frac{\phi_n(x)\,\phi_n(y)}{\lambda_n},
\]
as a function of $y$.

First, note that for each $n$, $\phi_n(0)=\phi_n(\pi)=0$, so
\[
v_x(0) = \sum_{n=1}^{\infty} \frac{\phi_n(x)\,\phi_n(0)}{\lambda_n} = 0,
\qquad
v_x(\pi) = \sum_{n=1}^{\infty} \frac{\phi_n(x)\,\phi_n(\pi)}{\lambda_n} = 0.
\]
Thus $G(x,0)=G(x,\pi)=0$ for all $x$.

Next, we compute $L_y v_x(y)$ in the sense of distributions. Formally differentiating term-by-term with respect to $y$, we have
\[
- \frac{d^2}{dy^2} v_x(y)
= - \sum_{n=1}^{\infty} \frac{\phi_n(x)}{\lambda_n} \phi_n''(y)
= \sum_{n=1}^{\infty} \frac{\phi_n(x)}{\lambda_n} (\lambda_n \phi_n(y))
= \sum_{n=1}^{\infty} \phi_n(x)\,\phi_n(y).
\]
To interpret this properly, we test against an arbitrary function $\psi\in L^2(0,\pi)$. Expanding $\psi = \sum_{m=1}^{\infty} \psi_m \phi_m$ with $\psi_m = \langle \psi,\phi_m\rangle$, we compute
\[
\begin{aligned}
\int_{0}^{\pi} \left( - v_x''(y) \right) \psi(y)\,dy
&= \int_{0}^{\pi} \left( \sum_{n=1}^{\infty} \phi_n(x)\,\phi_n(y) \right) \psi(y)\,dy \\
&= \sum_{n=1}^{\infty} \phi_n(x) \int_{0}^{\pi} \phi_n(y)\,\psi(y)\,dy \\
&= \sum_{n=1}^{\infty} \phi_n(x)\,\psi_n.
\end{aligned}
\]
On the other hand, since $\psi = \sum \psi_n \phi_n$, we have
\[
\psi(x) = \sum_{n=1}^{\infty} \psi_n\,\phi_n(x).
\]
Therefore,
\[
\int_{0}^{\pi} \left(- v_x''(y)\right)\psi(y)\,dy = \psi(x)
\]
for all $\psi \in L^2(0,\pi)$.

But $\psi \mapsto \psi(x)$ is exactly the action of the Dirac distribution $\delta_x$ at $y=x$. Thus, in distributional notation,
\[
- \frac{d^2}{dy^2} G(x,y) = \delta_x(y)
\]
with homogeneous Dirichlet boundary conditions in $y$. This is precisely the usual defining property of the Green function associated with the operator $L$:
\[
L_y G(x,y) = \delta_x(y),\qquad G(x,0)=G(x,\pi)=0.
\]

Combining this with part (c), we see both characterizations of the Green function:
\begin{itemize}
  \item As the solution (in the $y$-variable) of $L_y G(x,y) = \delta_x(y)$ with homogeneous boundary conditions, and
  \item As the integral kernel of the inverse operator $L^{-1}$, given spectrally by
  \[
  G(x,y) = \sum_{n=1}^{\infty} \frac{\phi_n(x)\,\phi_n(y)}{\lambda_n}.
  \]
\end{itemize}

\medskip

\textbf{Connection to the general theory.}

This example illustrates the main ideas of the method of Green functions for elliptic PDEs on bounded domains:

\begin{enumerate}
  \item For a self-adjoint, positive, elliptic operator $L$ on $L^2(\Omega)$ with suitable boundary conditions, the spectrum consists of a discrete sequence of eigenvalues $0<\lambda_1\le\lambda_2\le\cdots\to\infty$, with corresponding orthonormal eigenfunctions $\{\phi_n\}$ forming a basis of $L^2(\Omega)$.
  \item The inverse operator $L^{-1}$ (the resolvent at $0$) is well-defined and bounded on $L^2(\Omega)$, and acts diagonally on the eigenbasis by $L^{-1}\phi_n = \lambda_n^{-1}\phi_n$.
  \item The Green function is the integral kernel of $L^{-1}$ and admits the spectral expansion
  \[
  G(x,y) = \sum_{n=1}^{\infty} \frac{\phi_n(x)\,\phi_n(y)}{\lambda_n}.
  \]
  4. Solutions of the elliptic equation $Lu=f$ with homogeneous boundary conditions can be represented equivalently as eigenfunction series
  \[
  u(x) = \sum_{n=1}^{\infty} \frac{\langle f,\phi_n\rangle}{\lambda_n}\,\phi_n(x)
  \]
  or as a Green function integral
  \[
  u(x) = \int_{\Omega} G(x,y)\,f(y)\,dy.
  \]
\end{enumerate}

Thus, the Green function representation and the eigenfunction expansion are two faces of the same underlying spectral decomposition of a self-adjoint elliptic operator. In higher dimensions and for more general elliptic operators, the same structure persists: the Green function is the kernel of the resolvent, and its expansion in eigenfunctions reveals explicitly how the geometry of the domain and the boundary conditions influence the solution.
\end{solution}

\section{Waves in a Homogeneous Medium: Hyperbolic PDE (*)}
% --- Narrative plan (auto-generated) ---
% In this section we study wave phenomena in media that are, at least to a first approximation, spatially uniform. The governing equations are hyperbolic partial differential equations, with the one-dimensional wave equation as the central model. We will see how such equations encode the finite speed of propagation of signals, the superposition of traveling waves, and the emergence of characteristic lines along which information is transported.
%
% This material matters across applied mathematics, from modeling vibrations of strings and membranes, to understanding sound waves, electromagnetic waves, and even certain aspects of fluid flow and elasticity. Hyperbolic equations provide a complementary viewpoint to parabolic diffusion and elliptic equilibrium problems, and learning to solve them builds fluency with separation of variables, Fourier series, and transform methods. Along the way, we will connect to ordinary differential equations through eigenvalue problems, to Fourier analysis through series and integrals representing wave fields, and to more advanced topics such as distributions and complex analysis when we interpret fundamental solutions and apply contour integration to evaluate certain integrals.
%
% Our approach emphasizes starting from simple, concrete boundary-value problems and gradually introducing more sophisticated tools. We will analyze waves on finite and infinite domains, explore how initial and boundary data shape the resulting motion, and use the method of characteristics to understand how disturbances propagate. The goal is not only to learn solution formulas, but also to build an intuitive picture of how hyperbolic PDEs behave and how they fit into the broader framework of dynamical systems and applied analysis.

% ===== Example 1: Transverse Vibrations of a Stretched String (inquiry-based) =====
\begin{problem}[Transverse Vibrations of a Stretched String]
A thin, flexible string of length $L$ is stretched tightly between two fixed supports at $x=0$ and $x=L$. The string is displaced from its equilibrium position into some initial shape $f(x)$ and then released from rest. Assuming small transverse vibrations and a homogeneous string with constant wave speed $c>0$, the vertical displacement $u(x,t)$ of the string satisfies the one-dimensional wave equation. In this problem we explore how to solve this model using separation of variables and Fourier sine series, and how the notion of normal modes and standing waves naturally appears.

We consider the initial--boundary value problem
\[
\begin{cases}
u_{tt}(x,t) = c^{2} u_{xx}(x,t), & 0<x<L,\ t>0,\\[0.5ex]
u(0,t) = 0,\quad u(L,t)=0, & t\ge 0,\\[0.5ex]
u(x,0) = f(x),\quad u_t(x,0)=0, & 0\le x\le L.
\end{cases}
\]

\begin{enumerate}[(a)]
\item (Setting up separation of variables.) Suppose that $u$ can be written as a product of a purely spatial factor and a purely temporal factor,
\[
u(x,t) = X(x)\,T(t),
\]
with $X$ not identically zero and $T$ not identically zero. Substitute this form into the wave equation and show that
\[
\frac{T''(t)}{c^{2} T(t)} = \frac{X''(x)}{X(x)} = -\lambda
\]
for some constant $\lambda$ (the separation constant).

\emph{Hint:} Rearrange your expression so that all terms depending on $x$ are on one side and all terms depending on $t$ are on the other, then argue that both sides must be equal to the same constant.

\item (The spatial eigenvalue problem.) Focus on the spatial part
\[
X''(x) + \lambda X(x) = 0,\qquad X(0)=0,\quad X(L)=0.
\]
Analyze the possible signs of $\lambda$:
\[
\lambda<0,\quad \lambda=0,\quad \lambda>0.
\]
For each case, solve the ordinary differential equation and check whether there are nontrivial solutions satisfying the boundary conditions.

What values of $\lambda$ admit nontrivial solutions, and what are the corresponding eigenfunctions $X_n(x)$?

\emph{Hint:} You should find a discrete set of eigenvalues $\lambda_n$ and corresponding eigenfunctions that look like sines. Be explicit about the allowed values of $n$.

\item (Time dependence and standing waves.) For each admissible separation constant $\lambda_n$ from part (b), the time factor $T_n(t)$ satisfies
\[
T_n''(t) + c^{2}\lambda_n T_n(t) = 0.
\]
Solve this equation and write down the corresponding separated solutions $u_n(x,t) = X_n(x) T_n(t)$.

Show that each $u_n$ can be written in the form
\[
u_n(x,t) = \left(A_n\cos(\omega_n t) + B_n\sin(\omega_n t)\right)\sin\!\left(\frac{n\pi x}{L}\right),
\]
for some constants $A_n$ and $B_n$, and determine the frequency $\omega_n$ in terms of $c$, $n$, and $L$.

\emph{Hint:} Relate $\omega_n$ to the square root of the separation constant $\lambda_n$.

\item (Superposition and the initial displacement.) We now use linearity of the wave equation to build more general solutions from the basic separated solutions $u_n$.

Assuming that the solution can be written as a superposition of such modes,
\[
u(x,t) = \sum_{n=1}^{\infty} \left(A_n\cos(\omega_n t) + B_n\sin(\omega_n t)\right)\sin\!\left(\frac{n\pi x}{L}\right),
\]
use the initial conditions $u(x,0)=f(x)$ and $u_t(x,0)=0$ to determine conditions on the coefficients $A_n$ and $B_n$.

\begin{enumerate}[(i)]
\item Show that the zero initial velocity forces $B_n=0$ for all $n$.
\item Show that the initial displacement $f(x)$ can be expanded in a \emph{Fourier sine series}
\[
f(x) = \sum_{n=1}^{\infty} A_n \sin\!\left(\frac{n\pi x}{L}\right),
\]
and derive a formula for $A_n$ in terms of $f$.

\emph{Hint:} Use the orthogonality of the functions $\sin\!\left(\frac{n\pi x}{L}\right)$ on the interval $[0,L]$:
\[
\int_0^L \sin\!\left(\frac{m\pi x}{L}\right)\sin\!\left(\frac{n\pi x}{L}\right)\,dx =
\begin{cases}
0, & m\ne n,\\[0.5ex]
\displaystyle \frac{L}{2}, & m=n.
\end{cases}
\]
\end{enumerate}

Assemble your results into a final formula for $u(x,t)$ in terms of $f(x)$ and the wave speed $c$.

\item (Extensions and variations.)
\begin{enumerate}[(i)]
\item Suppose instead that the string is released with \emph{zero displacement} and a prescribed initial velocity $u_t(x,0)=g(x)$, while still having $u(0,t)=u(L,t)=0$. How would your general solution change, and what integrals would appear in the formulas for the coefficients?

\item Qualitatively, how do the frequencies $\omega_n$ depend on $n$? What does this tell you about which modes oscillate faster or slower? Explain how this leads to the interpretation of $\sin\!\left(\frac{n\pi x}{L}\right)$ as the $n$-th normal mode or standing wave of the string.

\emph{Hint:} You may wish to sketch the first few eigenfunctions and discuss the number of interior nodes (points where $u=0$) for each mode.
\end{enumerate}
\end{enumerate}
\end{problem}

% ===== Example 1: Transverse Vibrations of a Stretched String (full solution) =====
\begin{problem}[Transverse Vibrations of a Stretched String]
Consider a taut string of length $L$ with fixed endpoints at $x=0$ and $x=L$. Let $u(x,t)$ denote the transverse displacement of the string at position $x$ and time $t$. Assume that $u$ satisfies the one-dimensional wave equation with constant wave speed $c>0$,
\[
u_{tt} = c^{2} u_{xx},\quad 0<x<L,\ t>0,
\]
together with homogeneous Dirichlet boundary conditions
\[
u(0,t)=0,\quad u(L,t)=0,\quad t\ge 0,
\]
and initial conditions corresponding to an initial shape $f$ and zero initial velocity,
\[
u(x,0)=f(x),\quad u_t(x,0)=0,\quad 0\le x\le L.
\]
Solve this initial--boundary value problem by separation of variables, and show that the solution can be written as a Fourier sine series in the normal modes of the string. Give explicit formulas for the time-dependent solution $u(x,t)$ and for the Fourier coefficients in terms of $f$.
\end{problem}

\begin{solution}
We solve the wave equation with fixed-end boundary conditions using separation of variables and Fourier series. This example illustrates how a hyperbolic partial differential equation on a bounded interval leads to a discrete set of eigenfrequencies and standing wave modes.

\medskip

\noindent\textbf{1. Separation of variables.}
We look for nontrivial solutions of the form
\[
u(x,t) = X(x)\,T(t),
\]
where $X$ depends only on $x$ and $T$ depends only on $t$. Substituting this product into the wave equation $u_{tt}=c^{2}u_{xx}$ gives
\[
X(x)\,T''(t) = c^{2} X''(x)\,T(t).
\]
We assume $X$ and $T$ are not identically zero, so we may divide both sides by $c^{2}X(x)T(t)$ to obtain
\[
\frac{T''(t)}{c^{2}T(t)} = \frac{X''(x)}{X(x)}.
\]
The left-hand side depends only on $t$ and the right-hand side depends only on $x$. Therefore both sides must equal a constant, which we denote by $-\lambda$. This yields the separated ordinary differential equations
\[
X''(x) + \lambda X(x) = 0,\qquad T''(t) + c^{2}\lambda T(t) = 0.
\]
The boundary conditions $u(0,t)=u(L,t)=0$ translate into
\[
X(0)T(t)=0,\quad X(L)T(t)=0\quad\text{for all }t.
\]
For a nontrivial time factor $T(t)$, this implies
\[
X(0)=0,\quad X(L)=0.
\]
Thus we are led first to study the spatial eigenvalue problem
\[
X''(x) + \lambda X(x) = 0,\qquad X(0)=0,\quad X(L)=0.
\]

\medskip

\noindent\textbf{2. Spatial eigenvalues and eigenfunctions.}
We analyze the possible signs of $\lambda$.

\emph{Case 1: $\lambda<0$.} Write $\lambda = -\mu^{2}$ with $\mu>0$. Then the equation becomes
\[
X''(x) - \mu^{2}X(x) = 0,
\]
whose general solution is
\[
X(x) = C_{1}e^{\mu x} + C_{2}e^{-\mu x}.
\]
Imposing $X(0)=0$ gives $C_{1}+C_{2}=0$, so $C_{2}=-C_{1}$ and $X(x)=C_{1}(e^{\mu x}-e^{-\mu x})=2C_{1}\sinh(\mu x)$. Then $X(L)=0$ implies $\sinh(\mu L)=0$, which forces $\mu L=0$, hence $\mu=0$, contradicting $\mu>0$. Therefore there are no nontrivial solutions for $\lambda<0$.

\emph{Case 2: $\lambda=0$.} The equation is $X''(x)=0$, with general solution $X(x)=C_{1}+C_{2}x$. The boundary condition $X(0)=0$ gives $C_{1}=0$, so $X(x)=C_{2}x$. Then $X(L)=0$ implies $C_{2}L=0$, hence $C_{2}=0$. Thus only the trivial solution exists for $\lambda=0$.

\emph{Case 3: $\lambda>0$.} Write $\lambda=\mu^{2}$ with $\mu>0$. Then
\[
X''(x) + \mu^{2}X(x) = 0,
\]
whose general solution is
\[
X(x) = C_{1}\cos(\mu x) + C_{2}\sin(\mu x).
\]
The condition $X(0)=0$ implies $C_{1}=0$, so $X(x)=C_{2}\sin(\mu x)$. Imposing $X(L)=0$ gives $C_{2}\sin(\mu L)=0$. For a nontrivial solution, $C_{2}\ne 0$, so we require
\[
\sin(\mu L)=0.
\]
This holds if and only if
\[
\mu L = n\pi,\quad n=1,2,3,\dots.
\]
Thus
\[
\mu_n = \frac{n\pi}{L},\qquad \lambda_n = \mu_n^{2} = \left(\frac{n\pi}{L}\right)^{2}.
\]
For each positive integer $n$, we obtain an eigenfunction
\[
X_n(x) = \sin\!\left(\frac{n\pi x}{L}\right),
\]
which is nontrivial and satisfies $X_n(0)=X_n(L)=0$. There is no admissible eigenvalue for $n=0$ because $X_0(x)=\sin(0)=0$ is the trivial solution.

The collection $\{X_n\}_{n=1}^{\infty}$ forms a countable set of spatial eigenfunctions, each corresponding to an eigenvalue $\lambda_n$.

\medskip

\noindent\textbf{3. Time dependence and standing waves.}
For each eigenvalue $\lambda_n$, the corresponding time factor $T_n(t)$ must satisfy
\[
T_n''(t) + c^{2}\lambda_n T_n(t) = 0,
\]
that is,
\[
T_n''(t) + c^{2}\left(\frac{n\pi}{L}\right)^{2} T_n(t) = 0.
\]
This is the equation of a simple harmonic oscillator with angular frequency
\[
\omega_n = c\frac{n\pi}{L}.
\]
Its general solution is
\[
T_n(t) = A_n\cos(\omega_n t) + B_n\sin(\omega_n t),
\]
where $A_n$ and $B_n$ are constants.

The separated solution associated with the $n$-th eigenvalue is therefore
\[
u_n(x,t) = X_n(x)T_n(t)
= \left(A_n\cos(\omega_n t) + B_n\sin(\omega_n t)\right)\sin\!\left(\frac{n\pi x}{L}\right).
\]
Each such $u_n$ represents a \emph{standing wave}: the spatial profile $\sin\!\left(\frac{n\pi x}{L}\right)$ remains fixed in space, while its amplitude oscillates sinusoidally in time with frequency $\omega_n$. The integer $n$ counts the number of half-wavelengths that fit into the interval $[0,L]$.

\medskip

\noindent\textbf{4. Superposition and imposition of initial conditions.}
Because the wave equation is linear and homogeneous, any finite linear combination of the separated solutions $u_n$ is again a solution. Under appropriate regularity assumptions on $f$, we may consider an infinite superposition and seek a solution of the form
\[
u(x,t) = \sum_{n=1}^{\infty}
\left(A_n\cos(\omega_n t) + B_n\sin(\omega_n t)\right)\sin\!\left(\frac{n\pi x}{L}\right),
\]
where $\omega_n = c n\pi/L$. By construction, this satisfies the wave equation and the boundary conditions for all choices of coefficients $\{A_n,B_n\}$.

We now determine the coefficients using the initial conditions.

\emph{First initial condition: zero initial velocity.}
Differentiating $u$ with respect to $t$ gives
\[
u_t(x,t) = \sum_{n=1}^{\infty}
\left(-A_n\omega_n\sin(\omega_n t) + B_n\omega_n\cos(\omega_n t)\right)\sin\!\left(\frac{n\pi x}{L}\right).
\]
At $t=0$, this reduces to
\[
u_t(x,0) = \sum_{n=1}^{\infty} B_n\omega_n \sin\!\left(\frac{n\pi x}{L}\right).
\]
The initial condition $u_t(x,0)=0$ for $0\le x\le L$ therefore implies
\[
\sum_{n=1}^{\infty} B_n\omega_n \sin\!\left(\frac{n\pi x}{L}\right) = 0\quad\text{for all }x\in[0,L].
\]
Since the sine functions $\{\sin(n\pi x/L)\}_{n=1}^{\infty}$ are orthogonal and form a complete system in a suitable function space on $[0,L]$, the only way this sum can vanish identically is if
\[
B_n = 0\quad\text{for all }n\ge 1.
\]
Thus the solution simplifies to
\[
u(x,t) = \sum_{n=1}^{\infty} A_n\cos(\omega_n t)\,\sin\!\left(\frac{n\pi x}{L}\right).
\]

\emph{Second initial condition: prescribed initial displacement.}
At time $t=0$ we have
\[
u(x,0) = \sum_{n=1}^{\infty} A_n\cos(0)\,\sin\!\left(\frac{n\pi x}{L}\right)
= \sum_{n=1}^{\infty} A_n\sin\!\left(\frac{n\pi x}{L}\right).
\]
The initial displacement condition $u(x,0)=f(x)$ therefore requires that $f$ admit a Fourier sine series expansion
\[
f(x) = \sum_{n=1}^{\infty} A_n\sin\!\left(\frac{n\pi x}{L}\right),\quad 0\le x\le L.
\]
The coefficients $A_n$ are determined by the standard sine series formula. The orthogonality relation
\[
\int_0^L \sin\!\left(\frac{m\pi x}{L}\right)\sin\!\left(\frac{n\pi x}{L}\right)\,dx
=
\begin{cases}
0, & m\neq n,\\[0.5ex]
\displaystyle \frac{L}{2}, & m=n,
\end{cases}
\]
allows us to isolate $A_n$. Multiply both sides of the expansion for $f(x)$ by $\sin\!\left(\frac{n\pi x}{L}\right)$ and integrate over $[0,L]$:
\[
\int_0^L f(x)\sin\!\left(\frac{n\pi x}{L}\right)\,dx
= \sum_{k=1}^{\infty} A_k
\int_0^L \sin\!\left(\frac{k\pi x}{L}\right)\sin\!\left(\frac{n\pi x}{L}\right)\,dx.
\]
By orthogonality, all terms with $k\ne n$ vanish, and we obtain
\[
\int_0^L f(x)\sin\!\left(\frac{n\pi x}{L}\right)\,dx
= A_n \int_0^L \sin^{2}\!\left(\frac{n\pi x}{L}\right)\,dx
= A_n \cdot \frac{L}{2}.
\]
Therefore
\[
A_n = \frac{2}{L}\int_0^L f(x)\sin\!\left(\frac{n\pi x}{L}\right)\,dx,\quad n=1,2,3,\dots.
\]

\medskip

\noindent\textbf{5. Final formula and interpretation.}
Combining these results, the unique solution of the initial--boundary value problem is
\[
u(x,t) = \sum_{n=1}^{\infty}
\left[\frac{2}{L}\int_0^L f(\xi)\sin\!\left(\frac{n\pi \xi}{L}\right)\,d\xi\right]
\cos\!\left(\frac{n\pi c t}{L}\right)
\sin\!\left(\frac{n\pi x}{L}\right),
\]
valid (under standard regularity assumptions on $f$) for $0<x<L$ and $t\ge 0$.

Each term in this series represents a \emph{normal mode} or \emph{standing wave} of the string, with spatial profile $\sin\!\left(\frac{n\pi x}{L}\right)$ and natural frequency
\[
\omega_n = c\frac{n\pi}{L}.
\]
The lowest mode, with $n=1$, has the longest wavelength and oscillates most slowly; higher modes have more interior nodes and oscillate faster, with frequencies proportional to $n$. The initial displacement $f(x)$ is decomposed into these modes by its Fourier sine series, and the time evolution is a superposition of independent harmonic oscillations.

This example illustrates the key ideas of the chapter section on waves in a homogeneous medium and hyperbolic partial differential equations: the wave equation on a bounded domain leads to an eigenvalue problem in space, a discrete spectrum of eigenvalues, orthogonal eigenfunctions forming a basis, and a representation of the solution as a Fourier series in normal modes, each evolving in time according to a simple harmonic oscillator.
\end{solution}

% ===== Example 2: Semi-Infinite String and the d’Alembert Formula (inquiry-based) =====
\begin{problem}[Semi-Infinite String and the d’Alembert Formula]
A taut, homogeneous string of linear density $\rho$ and tension $T$ is stretched along a line. Its small vertical displacement $u(x,t)$ satisfies the one-dimensional wave equation
\[
u_{tt} = c^{2} u_{xx}, \qquad c = \sqrt{T/\rho},
\]
where $x$ denotes position along the string and $t$ is time. In this problem you will first derive the d’Alembert representation formula for the wave equation on the entire real line, and then adapt it to a semi-infinite string fixed at one end. Along the way, you will see explicitly how initial disturbances decompose into left- and right-traveling waves and how a fixed end can be understood as a ``mirror'' that reflects waves with a sign change.

Consider first the Cauchy problem on the whole line:
\[
\begin{cases}
u_{tt}(x,t) = c^{2} u_{xx}(x,t), & x \in \mathbb{R},\ t>0,\\[0.3em]
u(x,0) = f(x), & x \in \mathbb{R},\\[0.3em]
u_t(x,0) = g(x), & x \in \mathbb{R},
\end{cases}
\]
where $f$ and $g$ are given, sufficiently smooth and decaying functions.

\smallskip

(a) (\textbf{Traveling wave building blocks.}) Suppose $w(x,t)$ is of the form $w(x,t) = F(x-ct)$ for some twice differentiable function $F$. 

\quad(i) Compute $w_t$ and $w_{tt}$ in terms of derivatives of $F$ and $x-ct$. Similarly, compute $w_x$ and $w_{xx}$.

\quad(ii) Show that any such $w$ satisfies the wave equation $w_{tt} = c^{2} w_{xx}$. Do the same for functions of the form $w(x,t) = G(x+ct)$.

\quad(iii) Conclude that any function of the form
\[
u(x,t) = F(x-ct) + G(x+ct)
\]
solves the wave equation. Why is it natural to interpret the first term as a right-moving wave and the second as a left-moving wave?

\emph{Hint:} Look at the graphs of $x \mapsto F(x)$ and $x \mapsto F(x-ct)$ for fixed $t$ and see how they move as $t$ increases.

\smallskip

(b) (\textbf{Using the initial displacement.}) Suppose that $u(x,t) = F(x-ct) + G(x+ct)$ solves the Cauchy problem. 

\quad(i) Use the initial condition $u(x,0) = f(x)$ to obtain a relation between $F$ and $G$ evaluated at $x$.

\quad(ii) Differentiate the representation for $u$ with respect to $t$, evaluate at $t=0$, and use $u_t(x,0)=g(x)$ to obtain a relation between $F'$ and $G'$ evaluated at $x$.

\quad(iii) Treat $F(x)$ and $G(x)$ as unknown functions and write your two relations as a system of equations for $F'(x)$ and $G'(x)$ in terms of $f'(x)$ and $g(x)$.

\emph{Hint:} First write
\[
F(x)+G(x)=f(x),
\qquad
-cF'(x)+cG'(x)=g(x),
\]
then add and subtract suitable multiples of these equations.

\smallskip

(c) (\textbf{Solving for $F$ and $G$ and obtaining d’Alembert’s formula.}) 

\quad(i) Solve the system from part (b) for $F'(x)$ and $G'(x)$, and then integrate with respect to $x$ to obtain formulas for $F(x)$ and $G(x)$ in terms of $f$ and $g$. Be careful about constants of integration.

\quad(ii) Show that, after combining the expressions and choosing the constants appropriately, you can write $u(x,t)$ in the form
\[
u(x,t) 
= \frac{1}{2} \bigl(f(x-ct) + f(x+ct)\bigr)
+ \frac{1}{2c} \int_{x-ct}^{x+ct} g(s)\,ds.
\]

\quad(iii) Explain how this formula expresses $u(x,t)$ entirely in terms of the initial displacement $f$ and the initial velocity $g$ over the interval $[x-ct,\,x+ct]$.

\emph{Hint:} For the constants of integration, imagine that $f$ and $g$ are compactly supported and that $u$ should vanish for large $|x|$ and small $t>0$.

\smallskip

Now consider a semi-infinite string occupying $x>0$ with its left end fixed at $x=0$. The displacement still satisfies
\[
u_{tt} = c^{2} u_{xx}, \qquad x>0,\ t>0,
\]
but now with a boundary condition at $x=0$,
\[
u(0,t) = 0, \qquad t \ge 0,
\]
and initial data given only for $x>0$:
\[
u(x,0) = f(x), \qquad u_t(x,0) = g(x), \qquad x>0.
\]
Assume $f$ and $g$ are sufficiently smooth on $[0,\infty)$ and satisfy $f(0)=0$ and $g(0)=0$.

\smallskip

(d) (\textbf{Odd reflection and the semi-infinite string.})

\quad(i) Define the odd extensions $\tilde f$ and $\tilde g$ of $f$ and $g$ to the whole line by
\[
\tilde f(x) =
\begin{cases}
f(x), & x \ge 0,\\
-f(-x), & x < 0,
\end{cases}
\qquad
\tilde g(x) =
\begin{cases}
g(x), & x \ge 0,\\
-g(-x), & x < 0.
\end{cases}
\]
Explain why $\tilde f$ and $\tilde g$ are continuous (and sufficiently smooth) at $x=0$ under the assumptions on $f$ and $g$.

\quad(ii) Let $\tilde u(x,t)$ be the solution of the Cauchy problem on the whole line with initial data $\tilde f$ and $\tilde g$:
\[
\tilde u_{tt} = c^{2} \tilde u_{xx}, \quad x \in \mathbb{R},\ t>0, 
\qquad
\tilde u(x,0)=\tilde f(x),\quad \tilde u_t(x,0)=\tilde g(x).
\]
Show that $\tilde u$ is an odd function of $x$ for each fixed $t$ (that is, $\tilde u(-x,t) = -\tilde u(x,t)$).

\emph{Hint:} Apply uniqueness for the Cauchy problem: if $v(x,t) = -\tilde u(-x,t)$, check that $v$ satisfies the same PDE and the same initial data as $\tilde u$.

\quad(iii) Show that, for $x>0$, the restriction $u(x,t) := \tilde u(x,t)$ satisfies the original semi-infinite problem, including the boundary condition at $x=0$.

\emph{Hint:} Use the oddness of $\tilde u$ to evaluate $u(0,t)$.

\smallskip

(e) (\textbf{Extensions and interpretations.})

\quad(i) Write an explicit d’Alembert-type formula for $u(x,t)$ on $x>0$ by substituting $\tilde f$ and $\tilde g$ into the formula from part (c) and then restricting to $x>0$. Try to simplify your answer by splitting the integral $\displaystyle \int_{x-ct}^{x+ct}$ into parts where the integration variable is positive or negative.

\emph{Hint:} When $x-ct<0<x+ct$, you can write
\[
\int_{x-ct}^{x+ct} \tilde g(s)\,ds
= \int_{x-ct}^{0} \tilde g(s)\,ds + \int_{0}^{x+ct} \tilde g(s)\,ds
\]
and use the oddness of $\tilde g$.

\quad(ii) Suppose instead that the boundary condition at $x=0$ is $u_x(0,t)=0$ (a free or Neumann end). What kind of extension of $f$ and $g$ to the whole line (odd or even?) would you try, and why? Formulate but do not fully carry out the analogous construction.

\quad(iii) Interpret physically how a pulse traveling toward the fixed end at $x=0$ behaves upon reaching the boundary, in terms of the odd extension and the reflected wave.

\end{problem}

% ===== Example 2: Semi-Infinite String and the d’Alembert Formula (full solution) =====
\begin{problem}[Semi-Infinite String and the d’Alembert Formula]
Consider the one-dimensional wave equation
\[
u_{tt} = c^{2} u_{xx}, \qquad c>0,
\]
with initial conditions $u(x,0) = f(x)$ and $u_t(x,0) = g(x)$.

(a) On the whole line $x\in\mathbb{R}$, derive the d’Alembert formula
\[
u(x,t) 
= \frac{1}{2}\bigl(f(x-ct)+f(x+ct)\bigr)
+ \frac{1}{2c}\int_{x-ct}^{x+ct} g(s)\,ds
\]
for sufficiently smooth and decaying $f$ and $g$.

(b) Now consider a semi-infinite string $x>0$ with fixed end at $x=0$:
\[
u_{tt} = c^{2} u_{xx}, \quad x>0,\ t>0; 
\quad u(0,t)=0,\ t\ge 0;
\quad u(x,0)=f(x),\ u_t(x,0)=g(x),\ x>0,
\]
where $f$ and $g$ are smooth on $[0,\infty)$, satisfy $f(0)=g(0)=0$, and are suitably decaying.

Using an odd reflection of the initial data across $x=0$, express the solution $u(x,t)$ for $x>0$ in terms of the d’Alembert formula applied on the whole line. Clearly indicate how the boundary condition is enforced and give an explicit representation of $u(x,t)$ in terms of $f$ and $g$.
\end{problem}

\begin{solution}
We begin with the wave equation on the whole real line and derive the d’Alembert representation formula. Then we adapt it to the semi-infinite domain by a reflection argument (method of images). This illustrates the basic structure of solutions to hyperbolic equations: propagation along characteristic lines and the role of boundary conditions as ``mirrors'' for waves.

\medskip

\textbf{(a) d’Alembert formula on the whole line.}

We consider the Cauchy problem
\[
u_{tt} = c^{2} u_{xx}, \qquad x\in\mathbb{R},\ t>0,
\]
with
\[
u(x,0)=f(x), \qquad u_t(x,0)=g(x).
\]

\emph{Step 1: General form of solutions via traveling waves.}

We first look for solutions of the special form $u(x,t) = F(x-ct)$, where $F$ is twice differentiable. Let $\xi = x-ct$. Using the chain rule we compute
\[
u_t(x,t) = -c F'(\xi), \qquad
u_{tt}(x,t) = c^{2} F''(\xi),
\]
and
\[
u_x(x,t) = F'(\xi), \qquad
u_{xx}(x,t) = F''(\xi).
\]
Substituting into the wave equation gives
\[
u_{tt} - c^{2} u_{xx} = c^{2} F''(\xi) - c^{2} F''(\xi) = 0.
\]
Hence any traveling profile of the form $F(x-ct)$ solves the wave equation. The same computation with $u(x,t) = G(x+ct)$ yields another family of solutions, now traveling to the left.

Because the wave equation is linear, any linear combination of such solutions is again a solution. Thus every function of the form
\[
u(x,t) = F(x-ct) + G(x+ct)
\]
satisfies $u_{tt}=c^{2}u_{xx}$. One can show (for sufficiently smooth solutions) that this is in fact the general solution of the one-dimensional wave equation; this can be made precise by a change of variables to characteristic coordinates $\xi = x-ct$ and $\eta = x+ct$, in which the equation reduces to $u_{\xi\eta}=0$ and hence $u(\xi,\eta)=A(\xi)+B(\eta)$.

Moreover, the function $x\mapsto F(x-ct)$ at a fixed time $t$ is just the graph of $F$ shifted to the right by $ct$, so $F(x-ct)$ is a right-traveling wave with speed $c$. Similarly, $G(x+ct)$ is a left-traveling wave.

\emph{Step 2: Use initial data to determine $F$ and $G$.}

We now impose the initial conditions. At $t=0$ we have
\[
u(x,0) = F(x) + G(x) = f(x).
\]
This is our first relation between $F$ and $G$.

Next, we differentiate $u$ with respect to $t$:
\[
u_t(x,t) = -c F'(x-ct) + c G'(x+ct).
\]
Evaluating at $t=0$ gives
\[
u_t(x,0) = -c F'(x) + c G'(x) = g(x),
\]
or equivalently
\[
-cF'(x)+cG'(x)=g(x).
\]

Thus $F$ and $G$ must satisfy the system
\[
F(x) + G(x) = f(x), \qquad -cF'(x)+cG'(x)=g(x).
\]

Differentiating the first relation with respect to $x$ yields
\[
F'(x) + G'(x) = f'(x).
\]
Together with $-cF'(x)+cG'(x)=g(x)$, this is now a pair of linear equations for the unknowns $F'(x)$ and $G'(x)$ at each point $x$:
\[
\begin{cases}
F'(x) + G'(x) = f'(x),\\[0.2em]
-cF'(x) + cG'(x) = g(x).
\end{cases}
\]

We can solve this system by standard algebra. Adding the two equations after multiplying the first by $c$ gives
\[
cF'(x)+cG'(x) = c f'(x),
\]
and then adding this to $-cF'(x)+cG'(x)=g(x)$ yields
\[
2cG'(x) = c f'(x) + g(x).
\]
Thus
\[
G'(x) = \frac{1}{2} f'(x) + \frac{1}{2c} g(x).
\]
Similarly, subtracting the second equation from $c$ times the first gives
\[
2cF'(x) = c f'(x) - g(x),
\]
and hence
\[
F'(x) = \frac{1}{2} f'(x) - \frac{1}{2c} g(x).
\]

\emph{Step 3: Integrate to find $F$ and $G$; derive d’Alembert’s formula.}

We integrate these identities with respect to $x$:
\[
F(x) = \frac{1}{2} f(x) - \frac{1}{2c} \int^{x} g(s)\,ds + C_1,
\]
\[
G(x) = \frac{1}{2} f(x) + \frac{1}{2c} \int^{x} g(s)\,ds + C_2,
\]
where $C_1$ and $C_2$ are constants of integration. Substituting into $u(x,t)=F(x-ct)+G(x+ct)$ yields
\[
\begin{aligned}
u(x,t)
&= \frac{1}{2} f(x-ct) - \frac{1}{2c} \int^{x-ct} g(s)\,ds + C_1 \\
&\quad + \frac{1}{2} f(x+ct) + \frac{1}{2c} \int^{x+ct} g(s)\,ds + C_2.
\end{aligned}
\]
We can combine the constants to a single constant $C_1+C_2$, and we can also combine the integrals:
\[
\int^{x+ct} g(s)\,ds - \int^{x-ct} g(s)\,ds = \int_{x-ct}^{x+ct} g(s)\,ds.
\]
Thus
\[
u(x,t)
= \frac{1}{2} \bigl(f(x-ct) + f(x+ct)\bigr)
+ \frac{1}{2c}\int_{x-ct}^{x+ct} g(s)\,ds
+ (C_1+C_2).
\]

To determine $C_1+C_2$, we may impose a mild growth or decay condition at infinity. If $f$ and $g$ have compact support or decay as $|x|\to\infty$, then it is natural to require that $u(x,t)\to 0$ as $|x|\to\infty$ for fixed $t>0$. This forces the constant to vanish, $C_1+C_2=0$. Therefore we obtain the classical d’Alembert formula
\[
u(x,t)
= \frac{1}{2} \bigl(f(x-ct) + f(x+ct)\bigr)
+ \frac{1}{2c}\int_{x-ct}^{x+ct} g(s)\,ds.
\]

This expression shows that $u(x,t)$ depends only on the values of the initial data $f$ and $g$ on the interval $[x-ct,x+ct]$, which is the intersection of the initial line $t=0$ with the characteristic cone emanating from the point $(x,t)$. This is a hallmark of hyperbolic equations: finite propagation speed and a clear domain of dependence.

\medskip

\textbf{(b) Semi-infinite string with a fixed end: odd reflection.}

We now consider the problem on the half-line $x>0$:
\[
u_{tt} = c^{2} u_{xx}, \quad x>0,\ t>0,
\]
with boundary condition
\[
u(0,t)=0,\qquad t\ge 0,
\]
and initial conditions
\[
u(x,0) = f(x),\qquad u_t(x,0)=g(x),\qquad x>0,
\]
where $f$ and $g$ are smooth on $[0,\infty)$, satisfy $f(0)=g(0)=0$, and decay sufficiently fast as $x\to\infty$.

The central idea is to extend the problem from the half-line to the whole line in such a way that the boundary condition at $x=0$ is automatically enforced by symmetry. For a fixed end with $u(0,t)=0$, we want an odd symmetry in $x$.

\emph{Step 1: Odd extensions of the initial data.}

We define the odd extensions $\tilde f$ and $\tilde g$ of $f$ and $g$ to the whole real line by
\[
\tilde f(x) =
\begin{cases}
f(x), & x \ge 0,\\
-f(-x), & x < 0,
\end{cases}
\qquad
\tilde g(x) =
\begin{cases}
g(x), & x \ge 0,\\
-g(-x), & x < 0.
\end{cases}
\]
Since $f(0)=0$ and $g(0)=0$, the defining formulas from the right and from the left coincide at $x=0$. Moreover, if $f$ and $g$ are, say, $C^{1}$ on $[0,\infty)$, then the left and right derivatives at $0$ also match, so $\tilde f$ and $\tilde g$ are at least $C^{1}$ across $x=0$. With more regularity of $f$ and $g$, the extensions inherit the same level of smoothness. Thus we have constructed globally defined, smooth, odd functions $\tilde f$ and $\tilde g$.

\emph{Step 2: Solve the Cauchy problem on $\mathbb{R}$ with odd data.}

We now consider the Cauchy problem on the entire line:
\[
\tilde u_{tt} = c^{2} \tilde u_{xx}, \quad x\in\mathbb{R},\ t>0,
\]
with
\[
\tilde u(x,0) = \tilde f(x), \qquad \tilde u_t(x,0)=\tilde g(x).
\]
By part (a), the unique (sufficiently regular) solution is given by d’Alembert’s formula:
\[
\tilde u(x,t)
= \frac{1}{2}\bigl(\tilde f(x-ct) + \tilde f(x+ct)\bigr)
+ \frac{1}{2c}\int_{x-ct}^{x+ct} \tilde g(s)\,ds.
\]

We now show that $\tilde u$ is an odd function of $x$ for each fixed $t$. Define
\[
v(x,t) := -\tilde u(-x,t).
\]
Because the change $x\mapsto -x$ simply reflects space, we have
\[
v_{tt}(x,t) = -\tilde u_{tt}(-x,t), \qquad
v_{xx}(x,t) = -\tilde u_{xx}(-x,t),
\]
so
\[
v_{tt}(x,t) - c^{2} v_{xx}(x,t)
= -\tilde u_{tt}(-x,t) + c^{2}\tilde u_{xx}(-x,t)
= -\bigl(\tilde u_{tt} - c^{2}\tilde u_{xx}\bigr)(-x,t)=0.
\]
Thus $v$ satisfies the same wave equation as $\tilde u$.

At $t=0$ we have
\[
v(x,0) = -\tilde u(-x,0) = -\tilde f(-x) = \tilde f(x),
\]
since $\tilde f$ is odd. Similarly,
\[
v_t(x,0) = -\tilde u_t(-x,0) = -\tilde g(-x) = \tilde g(x),
\]
since $\tilde g$ is odd. Therefore $v$ and $\tilde u$ satisfy the same PDE and the same initial data. By uniqueness of solutions to the Cauchy problem on $\mathbb{R}$, we must have $v(x,t) = \tilde u(x,t)$, that is,
\[
-\tilde u(-x,t) = \tilde u(x,t),
\]
or equivalently $\tilde u(-x,t) = -\tilde u(x,t)$. Hence $\tilde u$ is odd in $x$ for each $t$.

\emph{Step 3: Restrict to the half-line and verify the boundary condition.}

We now define
\[
u(x,t) := \tilde u(x,t), \qquad x>0,\ t\ge 0.
\]
On the domain $x>0$, $t>0$, the function $u$ satisfies the wave equation and inherits the initial conditions:
\[
u(x,0) = \tilde u(x,0) = \tilde f(x) = f(x), \quad x>0,
\]
\[
u_t(x,0) = \tilde u_t(x,0) = \tilde g(x) = g(x), \quad x>0.
\]
At the boundary $x=0$, using oddness of $\tilde u$ we obtain
\[
u(0,t) = \tilde u(0,t) = -\tilde u(0,t),
\]
so $2\tilde u(0,t)=0$, hence $\tilde u(0,t)=0$. Therefore
\[
u(0,t)=0, \qquad t\ge 0,
\]
and the boundary condition at the fixed end is satisfied.

Thus, the semi-infinite problem is solved by taking the odd extension of the initial data to $\mathbb{R}$, solving the Cauchy problem there, and then restricting the result back to $x>0$.

\emph{Step 4: Explicit representation formula.}

Substituting the specific form of $\tilde f$ and $\tilde g$ into the d’Alembert formula yields, for any $x>0$ and $t\ge 0$,
\[
u(x,t)
= \frac{1}{2}\bigl(\tilde f(x-ct) + \tilde f(x+ct)\bigr)
+ \frac{1}{2c}\int_{x-ct}^{x+ct} \tilde g(s)\,ds.
\]
We can make this more explicit in terms of $f$ and $g$ by considering the sign of $x\pm ct$.

For concreteness, suppose $t>0$ is such that $x-ct<0<x+ct$; this is the interesting regime in which the characteristic through $(x,t)$ intersects both positive and negative parts of the $x$-axis. Then
\[
\tilde f(x-ct) = -f(ct-x),\qquad \tilde f(x+ct)=f(x+ct),
\]
and
\[
\int_{x-ct}^{x+ct} \tilde g(s)\,ds
= \int_{x-ct}^{0} \tilde g(s)\,ds + \int_{0}^{x+ct} \tilde g(s)\,ds.
\]
Using the oddness of $\tilde g$ and the substitution $s=-r$ on $[x-ct,0]$, we obtain
\[
\int_{x-ct}^{0} \tilde g(s)\,ds
= \int_{ct-x}^{0} g(r)\,dr = -\int_{0}^{ct-x} g(r)\,dr.
\]
Hence
\[
\int_{x-ct}^{x+ct} \tilde g(s)\,ds
= -\int_{0}^{ct-x} g(r)\,dr + \int_{0}^{x+ct} g(r)\,dr
= \int_{ct-x}^{x+ct} g(r)\,dr.
\]

Combining these expressions gives, for $x>0$ and $t>0$,
\[
u(x,t)
= \frac{1}{2}\bigl(f(x+ct) - f(ct-x)\bigr)
+ \frac{1}{2c}\int_{ct-x}^{x+ct} g(s)\,ds
\]
when $x-ct<0<x+ct$. For times $t$ such that both $x-ct>0$ and $x+ct>0$, the interval $[x-ct,x+ct]$ stays in the positive half-line, and the odd extension simply coincides with $f$ and $g$ there, so the formula reduces to the standard d’Alembert representation with $f$ and $g$ on $(0,\infty)$:
\[
u(x,t)
= \frac{1}{2}\bigl(f(x-ct)+f(x+ct)\bigr)
+ \frac{1}{2c}\int_{x-ct}^{x+ct} g(s)\,ds
\quad \text{if } x-ct>0.
\]
For $x+ct<0$ (which cannot occur when $x>0$ and $t\ge 0$), a symmetric formula in terms of $f(-\cdot)$ and $g(-\cdot)$ would apply.

From the physical viewpoint, this construction shows that a pulse traveling toward $x=0$ reflects as if it continued into $x<0$ but with a sign change. The odd extension encodes that the reflected wave has opposite sign, consistent with a fixed end that inverts the displacement at the boundary.

\medskip

\textbf{Connection to the chapter theme.} 

This example illustrates several central ideas about waves in a homogeneous medium and hyperbolic partial differential equations:

\begin{itemize}
  \item The reduction of the PDE to characteristic variables $x\pm ct$ reveals that solutions are built from traveling waves, and the d’Alembert formula is an explicit representation expressing this structure.
  \item The domain of dependence is finite: $u(x,t)$ depends only on the initial data in the interval $[x-ct,x+ct]$, reflecting finite propagation speed.
  \item For the semi-infinite domain, the method of images (via odd extension) shows how boundary conditions correspond to symmetry conditions on the solution, and how boundaries reflect waves. Dirichlet and Neumann boundary conditions correspond naturally to odd and even reflections, respectively.
\end{itemize}

These features are typical of hyperbolic PDEs and will reappear in more complex settings, such as higher-dimensional wave equations, variable-coefficient media, and boundary-value problems on bounded domains.
\end{solution}

% ===== Example 3: Method of Characteristics for First-Order Wave Transport (inquiry-based) =====
\begin{problem}[Method of Characteristics for First-Order Wave Transport]
A narrow, straight river carries along a dissolved pollutant with constant downstream velocity $c>0$. Let $u(x,t)$ denote the pollutant concentration at position $x\in\mathbb{R}$ along the river and at time $t\ge 0$. If the river flow simply transports the pollutant without creating, destroying, or diffusing it, one arrives at the linear \emph{transport equation}
\[
u_t + c\,u_x = 0.
\]
In this problem you will discover how to solve this equation by following the motion of individual fluid parcels, that is, by tracking ``characteristic curves'' in the $(x,t)$–plane.

We will study the initial value problem
\[
u_t + c\,u_x = 0, \qquad x\in\mathbb{R},\ t>0, \qquad u(x,0) = f(x),
\]
where $f$ is a given initial concentration profile.

\medskip

(a) Imagine a marked parcel of water that is at position $x_0$ at time $t=0$. Assume the water flows at the constant speed $c>0$ in the positive $x$–direction.

\quad(i) Write down an ordinary differential equation (an ODE) for the position $X(t)$ of this parcel as a function of time, together with the appropriate initial condition.

\quad(ii) Solve this ODE explicitly and sketch several such trajectories $t\mapsto X(t)$ in the $(x,t)$–plane. These curves are called the \emph{characteristics} of the flow.

\medskip

(b) Now think about how the pollutant concentration behaves along such a characteristic curve. If a water parcel simply carries its concentration with it, without changing in time, what does this say about $u(X(t),t)$ along the trajectory you found in part (a)?

\quad(i) Express the rate of change of the concentration experienced by this parcel as a total derivative:
\[
\frac{d}{dt}u(X(t),t) = \; ? 
\]
Use the chain rule to write this derivative in terms of $u_t$ and $u_x$.

\quad(ii) Use the partial differential equation $u_t + c\,u_x = 0$ to simplify your expression for $\dfrac{d}{dt}u(X(t),t)$. What ordinary differential equation does $u(X(t),t)$ satisfy along a characteristic?

\emph{Hint:} You should find a very simple ODE for $u(X(t),t)$, reflecting that the concentration carried by each parcel does not change in time.

\medskip

(c) Combine your answers from parts (a) and (b).

\quad(i) Solve the characteristic equation for $X(t)$ obtained in part (a), and write it in the form
\[
X(t) = \xi + c t,
\]
where $\xi$ is a parameter labeling different characteristics. Rewrite this relation as an equation in $x$ and $t$:
\[
\text{(characteristic curves)}\quad x - c t = \xi.
\]

\quad(ii) Solve the ODE for $u(X(t),t)$ obtained in part (b) along these curves. Use the initial condition $u(x,0) = f(x)$ to determine the constant of integration in terms of $f$ and the parameter $\xi$.

\emph{Hint:} The idea is that along the characteristic passing through $(\xi,0)$ you must have $u(\xi,0) = f(\xi)$. How does this relate to $u(x,t)$ when $(x,t)$ lies on the same characteristic?

\medskip

(d) Use part (c) to derive an explicit formula for the solution $u(x,t)$ at an arbitrary point $(x,t)$ in terms of the initial profile $f$.

\quad(i) Show that for each $(x,t)$ there is a unique $\xi$ such that $(x,t)$ lies on the characteristic through $(\xi,0)$, and express $\xi$ in terms of $x$ and $t$.

\quad(ii) Conclude that the solution of the initial value problem satisfies
\[
u(x,t) = \, ? \quad\text{(fill in the expression in terms of $f$, $x$, and $t$)}.
\]

\quad(iii) Interpret your formula geometrically: how does the graph of $x\mapsto u(x,t)$ at time $t>0$ compare to the initial graph $x\mapsto f(x)$ at time $t=0$? In which direction does the profile move, and with what speed?

\emph{Hint:} Think in terms of translating the initial graph horizontally in the $x$–direction.

\medskip

(e) (Extensions and ``what if'' questions.)

\quad(i) Suppose instead that the transport equation were
\[
u_t + c\,u_x = 0 \quad\text{with } c<0.
\]
Repeat (at least informally) the reasoning above. In which direction does the initial profile now move? What characteristic curves do you obtain?

\quad(ii) Consider the \emph{inhomogeneous} transport equation
\[
u_t + c\,u_x = g(x,t),
\]
where $g$ represents a source term (for example, additional pollutant being added per unit time). How would you modify the characteristic method to handle this equation conceptually? What ODE would $u(X(t),t)$ satisfy along each characteristic?

\emph{Hint:} The total derivative $\dfrac{d}{dt}u(X(t),t)$ will no longer be zero; it will be related to $g$ along the trajectory.
\end{problem}

% ===== Example 3: Method of Characteristics for First-Order Wave Transport (full solution) =====
\begin{problem}[Method of Characteristics for First-Order Wave Transport]
Let $c\in\mathbb{R}$ be a nonzero constant. Consider the linear transport equation
\[
u_t + c\,u_x = 0, \qquad x\in\mathbb{R},\ t>0,
\]
with initial condition
\[
u(x,0) = f(x), \qquad x\in\mathbb{R},
\]
where $f$ is a given function.

(a) Solve this initial value problem using the method of characteristics and express $u(x,t)$ explicitly in terms of $f$, $x$, and $t$.

(b) For $c>0$ and $c<0$, describe in words how the initial profile $f$ propagates in time. In particular, specify the direction and speed of propagation.

\end{problem}

\begin{solution}
We solve the transport equation
\[
u_t + c\,u_x = 0,\qquad c\neq 0,
\]
with initial data $u(x,0)=f(x)$ by the method of characteristics. The central idea is that this first-order hyperbolic equation describes the advection of $u$ along straight lines in the $(x,t)$–plane, and that along those curves, called characteristics, the solution satisfies an ordinary differential equation.

\medskip

\textbf{Step 1: Characteristic curves.}
We seek curves in the $(x,t)$–plane along which $u$ behaves in a simple way. Let such a curve be parametrized as
\[
t \mapsto (X(t),t),
\]
that is, at time $t$ the point on the curve has spatial coordinate $X(t)$. Along this curve, we consider the function
\[
U(t) := u\bigl(X(t),t\bigr),
\]
which is the value of the solution as seen by a ``moving observer'' traveling along the curve $x=X(t)$.

By the chain rule, the total derivative of $U(t)$ is
\[
\frac{dU}{dt} = \frac{d}{dt}u(X(t),t)
= u_t(X(t),t) + X'(t)\,u_x(X(t),t).
\]
The partial differential equation $u_t + c\,u_x = 0$ implies
\[
u_t = -c\,u_x.
\]
Substituting this into the expression for $dU/dt$ gives
\[
\frac{dU}{dt}
= -c\,u_x(X(t),t) + X'(t)\,u_x(X(t),t)
= \bigl(X'(t)-c\bigr)\,u_x(X(t),t).
\]

We want the value of $u$ to satisfy an ordinary differential equation that does not involve $u_x$, and the simplest possibility is to arrange for $dU/dt$ to vanish. This happens if we choose $X(t)$ to satisfy
\[
X'(t) = c.
\]
Thus the characteristic curves for this equation are precisely those along which the spatial coordinate moves with constant speed $c$.

Solving the ordinary differential equation $X'(t)=c$ with initial condition $X(0)=\xi$ gives
\[
X(t) = \xi + c t,
\]
where the parameter $\xi\in\mathbb{R}$ labels different curves. Eliminating $t$ and $\xi$, we see that the characteristic curves in the $(x,t)$–plane are the straight lines
\[
x - c t = \xi = \text{constant.}
\]

\medskip

\textbf{Step 2: Evolution of $u$ along characteristics.}
Along a characteristic $x=X(t)$ with $X'(t)=c$, we just computed
\[
\frac{d}{dt}u(X(t),t)
= \bigl(X'(t)-c\bigr)\,u_x(X(t),t) = 0.
\]
Thus $U(t) = u(X(t),t)$ is constant along each characteristic. In other words, if $(x,t)$ and $(\xi,0)$ lie on the same characteristic line $x-ct=\xi$, then
\[
u(x,t) = u(\xi,0).
\]

We now use the initial condition to identify this constant. At time $t=0$, the solution satisfies
\[
u(\xi,0) = f(\xi).
\]
Therefore, along the characteristic line $x-ct=\xi$, the solution is given by
\[
u(x,t) = f(\xi)
\quad\text{whenever}\quad x-ct = \xi.
\]

\medskip

\textbf{Step 3: Expressing the solution in terms of $(x,t)$.}
For a fixed point $(x,t)$ with $t>0$, there is a unique characteristic passing through it. This characteristic can be written as
\[
x-ct = \xi,
\]
so the parameter $\xi$ is determined by $(x,t)$ via
\[
\xi = x - c t.
\]
Substituting this into the expression for $u(x,t)$ along characteristics, we obtain
\[
u(x,t) = f(\xi) = f(x-ct).
\]
This formula holds for all $x\in\mathbb{R}$ and all $t\ge 0$. It is straightforward to verify directly that $u(x,t)=f(x-ct)$ satisfies both the transport equation and the initial condition:
\begin{align*}
u_t(x,t) &= f'(x-ct)\cdot(-c),\\
u_x(x,t) &= f'(x-ct),
\end{align*}
so
\[
u_t + c\,u_x = -c f'(x-ct) + c f'(x-ct) = 0,
\]
and at $t=0$ we have $u(x,0)=f(x)$ as required.

This completes part (a): the solution to the initial value problem is
\[
\boxed{\,u(x,t) = f(x-ct)\,}.
\]

\medskip

\textbf{Step 4: Propagation of the initial profile.}
The explicit solution $u(x,t)=f(x-ct)$ has a simple geometric interpretation. For each fixed time $t$, the function $x\mapsto u(x,t)$ is obtained by evaluating the initial profile at the shifted argument $x-ct$. This means that the graph of $u(\cdot,t)$ is the graph of $f$ translated rigidly in the $x$–direction.

To see the direction and speed of propagation, consider the effect of time increasing from $0$ to some $t>0$:

\begin{itemize}
  \item If $c>0$, then for each fixed $x$ we evaluate the initial data at the point $x-ct$, which lies to the \emph{left} of $x$ by a distance $ct$. Equivalently, the shape $f$ moves to the \emph{right} with speed $c$. In other words, the initial profile is convected in the positive $x$–direction without distortion.

  \item If $c<0$, we may write $c=-|c|$ and the solution becomes
  \[
  u(x,t) = f(x - (-|c|)t) = f(x + |c|t).
  \]
  In this case, the initial profile is translated to the \emph{left} with speed $|c|$, that is, in the negative $x$–direction.
\end{itemize}

Thus the answer to part (b) is: for $c>0$ the wave profile $f$ propagates rigidly to the right with speed $|c|=c$, and for $c<0$ it propagates rigidly to the left with speed $|c|=-c$.

\medskip

\textbf{Connection to hyperbolic PDE and wave propagation.}
This example illustrates several central features of hyperbolic partial differential equations in a homogeneous medium. The equation $u_t + c\,u_x = 0$ is a prototypical first-order hyperbolic equation. Its characteristics are straight lines in the $(x,t)$–plane, along which information (here, the values of $u$) is transported. The solution shows \emph{finite-speed propagation}: data prescribed at a point $(x_0,0)$ influences the solution at exactly those points $(x,t)$ that lie on the characteristic through $(x_0,0)$, that is, on the line $x=x_0+ct$. There is no spreading or smoothing; the initial shape is simply carried along by the flow. These ideas—characteristics, propagation along curves, and finite signal speed—reappear in more complex form for the second-order wave equations studied later in this chapter.
\end{solution}

% ===== Example 4: Vibrating String with Damping and Forcing (inquiry-based) =====
\begin{problem}[Vibrating String with Damping and Forcing]
Consider a taut string of length $L$ fixed at both ends, vibrating in a vertical plane. In the idealized undamped, unforced case, its transverse displacement $u(x,t)$ satisfies the one-dimensional wave equation, and its motion can be described in terms of normal modes. Now suppose the string experiences viscous damping, proportional to its velocity, and is driven by an external periodic forcing that acts along the entire length of the string with a prescribed spatial profile. Physically, one expects that the damping removes energy while the forcing injects it, and that for large times a steady oscillatory regime may emerge in which input and dissipation balance.

We will model this situation, reduce the partial differential equation to a family of ordinary differential equations for the modes, and analyze the long-time behavior.

\smallskip

Let $u(x,t)$ denote the transverse displacement of the string at position $x\in(0,L)$ and time $t>0$. Suppose that $u$ satisfies
\[
u_{tt}(x,t) + 2\beta\,u_t(x,t) \;=\; c^2\,u_{xx}(x,t) \;+\; f(x)\cos(\omega t),
\qquad 0<x<L,\ t>0,
\]
where $c>0$ is the wave speed, $\beta>0$ is a damping coefficient, and $f(x)$ is a given smooth function describing how the external forcing is distributed along the string. Assume fixed ends,
\[
u(0,t)=u(L,t)=0,\qquad t>0,
\]
and zero initial displacement and velocity,
\[
u(x,0)=0,\quad u_t(x,0)=0,\qquad 0<x<L.
\]

\begin{enumerate}[(a)]
  \item \textbf{Modeling and qualitative behavior.}
  \begin{enumerate}[(i)]
    \item Explain in words what each term in the equation
    \[
    u_{tt} + 2\beta u_t = c^2 u_{xx} + f(x)\cos(\omega t)
    \]
    represents physically. In particular, which term corresponds to inertia, which to restoring forces, which to damping, and which to the external driving? Why is it reasonable to take the damping term proportional to $u_t$?
    \item If $\beta>0$ is fixed and $f\equiv 0$, what do you expect to happen to any initial vibration of the string as $t\to\infty$? How does your answer change when $f\not\equiv 0$ but is time-periodic as above?
  \end{enumerate}
  % Hint: Think about energy: which terms can store energy, which can remove it, and which can add it to the system?

  \item \textbf{Spatial modes and eigenfunction expansion.}
  \begin{enumerate}[(i)]
    \item Recall that in the undamped, unforced case,
    \[
    u_{tt} = c^2 u_{xx},\quad u(0,t)=u(L,t)=0,
    \]
    one seeks separated solutions $u(x,t)=X(x)T(t)$ and arrives at an eigenvalue problem for $X$. Write down this eigenvalue problem for $X$ and determine all eigenfunctions and corresponding eigenvalues.
    % Hint: You should obtain a Sturm--Liouville problem for $-X''=\lambda X$ with homogeneous Dirichlet boundary conditions.

    \item Show that the eigenfunctions you found in part (i) form an orthogonal basis of $L^2(0,L)$ (you may quote this as a standard theorem once you have identified the eigenfunctions). Write expansions of the form
    \[
    u(x,t) = \sum_{n=1}^\infty y_n(t)\,\sin\!\Bigl(\frac{n\pi x}{L}\Bigr),\qquad
    f(x) = \sum_{n=1}^\infty f_n\,\sin\!\Bigl(\frac{n\pi x}{L}\Bigr),
    \]
    where you should give a formula for the coefficients $f_n$ in terms of $f$.
    % Hint: Use orthogonality of the sine functions on $(0,L)$.
  \end{enumerate}

  \item \textbf{Reduction to a family of forced, damped oscillators.}
  Substitute the series expansion for $u(x,t)$ into the damped, forced wave equation and use the orthogonality of the eigenfunctions to derive an ordinary differential equation for each modal coefficient $y_n(t)$.
  \begin{enumerate}[(i)]
    \item Show that each $y_n(t)$ satisfies an equation of the form
    \[
    y_n''(t) + 2\beta\,y_n'(t) + \Omega_n^2\,y_n(t) \;=\; f_n\cos(\omega t),
    \]
    where you should identify $\Omega_n$ explicitly in terms of $c$, $L$, and $n$.
    % Hint: Carefully compute $u_{tt}$ and $u_{xx}$ termwise, then compare coefficients of $\sin(n\pi x/L)$.

    \item As a warm-up, ignore the index $n$ and consider the scalar equation
    \[
    y''(t) + 2\beta\,y'(t) + \Omega^2\,y(t) = F\cos(\omega t),\qquad y(0)=0,\ y'(0)=0,
    \]
    where $\beta>0$, $\Omega>0$, and $F$ are constants. Solve this ODE and identify the part of the solution that persists as $t\to\infty$.
    % Hint: You may use the method of undetermined coefficients with an ansatz $y_p(t)=A\cos(\omega t)+B\sin(\omega t)$ for a particular solution, and then discuss the homogeneous solution separately.
  \end{enumerate}

  \item \textbf{Assembling the steady-state solution and resonance.}
  \begin{enumerate}[(i)]
    \item Returning to the full string problem, use your result from part (c) to write down the \emph{steady-state} (long-time periodic) solution in the form
    \[
    u_{\text{ss}}(x,t) = \sum_{n=1}^\infty \Bigl( A_n\cos(\omega t) + B_n\sin(\omega t) \Bigr)\,\sin\!\Bigl(\frac{n\pi x}{L}\Bigr),
    \]
    and express $A_n$ and $B_n$ in terms of $f_n$, $\beta$, $c$, $L$, $n$, and $\omega$.

    \item Show that for each fixed $n$, the oscillation in the $n$th mode has amplitude
    \[
    R_n(\omega) = \frac{|f_n|}{\sqrt{\bigl(c^2(n\pi/L)^2 - \omega^2\bigr)^2 + 4\beta^2\omega^2}}.
    \]
    For a given mode $n$, for which driving frequencies $\omega$ is this amplitude largest? Describe qualitatively how increasing the damping coefficient $\beta$ affects the maximum amplitude and how sharply it peaks near the ``natural frequency'' $\omega\approx c\,n\pi/L$.
    % Hint: View $R_n(\omega)$ as a function of $\omega$ and relate its graph to the familiar resonance curve of a forced, damped harmonic oscillator.
  \end{enumerate}

  \item \textbf{Extensions and variations.}
  \begin{enumerate}[(i)]
    \item What changes in your analysis if the damping is turned off, that is, if $\beta=0$? In particular, what happens to the steady-state amplitude $R_n(\omega)$ near the natural frequencies? Relate your answer to the physical phenomenon of resonance and to the role of damping in realistic systems.

    \item Suppose instead that the forcing acts only in a single mode, say
    \[
    f(x) = F_1\sin\!\Bigl(\frac{\pi x}{L}\Bigr),
    \]
    so that $f_1=F_1$ and $f_n=0$ for $n\ge2$. How does your series simplify, and what can you say in this case about the motion of the string and its energy distribution among modes?
    % Hint: Think about whether higher modes can be excited if they are not directly forced and the equation is linear with constant coefficients.
  \end{enumerate}
\end{enumerate}
\end{problem}

% ===== Example 4: Vibrating String with Damping and Forcing (full solution) =====
\begin{problem}[Vibrating String with Damping and Forcing]
Let $u(x,t)$ denote the transverse displacement of a taut string of length $L$, fixed at $x=0$ and $x=L$. The string is subject to viscous damping with coefficient $\beta>0$ and an external periodic forcing of the form $f(x)\cos(\omega t)$, where $f$ is a given smooth function on $(0,L)$. The motion is modeled by
\[
u_{tt}(x,t) + 2\beta\,u_t(x,t) = c^2\,u_{xx}(x,t) + f(x)\cos(\omega t),\qquad 0<x<L,\ t>0,
\]
with boundary conditions
\[
u(0,t)=u(L,t)=0,\qquad t>0,
\]
and initial conditions
\[
u(x,0)=0,\qquad u_t(x,0)=0,\qquad 0<x<L.
\]

\begin{enumerate}[(a)]
  \item Using the eigenfunctions of the one-dimensional Dirichlet Laplacian on $(0,L)$, expand $u$ and $f$ in a sine series and reduce the partial differential equation to an infinite family of ordinary differential equations for the modal coefficients $y_n(t)$ of $u$.

  \item Solve each of these ordinary differential equations and determine the steady-state (long-time periodic) solution $u_{\mathrm{ss}}(x,t)$. Express $u_{\mathrm{ss}}$ as a Fourier sine series
  \[
  u_{\mathrm{ss}}(x,t) = \sum_{n=1}^\infty \bigl(A_n\cos(\omega t)+B_n\sin(\omega t)\bigr)\,\sin\!\Bigl(\frac{n\pi x}{L}\Bigr),
  \]
  with explicit formulae for $A_n$ and $B_n$ in terms of $f$, $\beta$, $c$, $L$, $n$, and $\omega$.

  \item For each fixed mode $n$, show that the amplitude of oscillation in that mode is
  \[
  R_n(\omega)=\frac{|f_n|}{\sqrt{\bigl(c^2(n\pi/L)^2 - \omega^2\bigr)^2 + 4\beta^2\omega^2}},
  \]
  where $f_n$ are the sine coefficients of $f$. Discuss how $R_n(\omega)$ depends on the driving frequency $\omega$ and the damping coefficient $\beta$, and explain how this illustrates resonance and its damping in the context of the damped, forced wave equation.
\end{enumerate}
\end{problem}

\begin{solution}
We analyze the problem by expanding the solution in the spatial eigenfunctions of the Dirichlet Laplacian. This reduces the partial differential equation to a countable family of ordinary differential equations, one for each mode, which are precisely the equations of forced, damped harmonic oscillators. The long-time behavior is then read off from the known solutions of those ordinary differential equations.

\medskip

\noindent\textbf{(a) Eigenfunction expansion and reduction to ODEs.}
We begin with the standard eigenvalue problem associated with the one-dimensional Laplacian with homogeneous Dirichlet boundary conditions:
\[
-X''(x) = \lambda X(x),\qquad 0<x<L,\qquad X(0)=X(L)=0.
\]
Nontrivial solutions exist exactly when
\[
X_n(x) = \sin\!\Bigl(\frac{n\pi x}{L}\Bigr),\qquad \lambda_n = \biggl(\frac{n\pi}{L}\biggr)^2,\qquad n=1,2,3,\dots.
\]
The family $\{X_n\}_{n=1}^\infty$ is an orthogonal basis of $L^2(0,L)$, and we normalize it in the usual way if desired. In particular, any function in $L^2(0,L)$ can be expanded in this basis, with convergence in $L^2$ and, for smooth functions, pointwise.

We therefore expand the forcing and the solution as
\[
f(x) = \sum_{n=1}^\infty f_n\,\sin\!\Bigl(\frac{n\pi x}{L}\Bigr),\qquad
u(x,t) = \sum_{n=1}^\infty y_n(t)\,\sin\!\Bigl(\frac{n\pi x}{L}\Bigr),
\]
where the coefficients of $f$ are given by the usual Fourier sine series formula,
\[
f_n = \frac{2}{L}\int_0^L f(x)\,\sin\!\Bigl(\frac{n\pi x}{L}\Bigr)\,dx.
\]

We now substitute the series for $u$ into the damped, forced wave equation
\[
u_{tt} + 2\beta u_t = c^2 u_{xx} + f(x)\cos(\omega t).
\]
Termwise differentiation (justified by standard convergence results for Fourier series of smooth functions) gives
\[
u_t(x,t) = \sum_{n=1}^\infty y_n'(t)\,\sin\!\Bigl(\frac{n\pi x}{L}\Bigr),\qquad
u_{tt}(x,t) = \sum_{n=1}^\infty y_n''(t)\,\sin\!\Bigl(\frac{n\pi x}{L}\Bigr),
\]
and, using $X_n''(x) = -\lambda_n X_n(x)$ with $\lambda_n=(n\pi/L)^2$,
\[
u_{xx}(x,t) = \sum_{n=1}^\infty y_n(t)\,X_n''(x)
             = -\sum_{n=1}^\infty \lambda_n\,y_n(t)\,\sin\!\Bigl(\frac{n\pi x}{L}\Bigr).
\]

Substituting these into the PDE, we obtain
\[
\sum_{n=1}^\infty y_n''(t)\,\sin\!\Bigl(\frac{n\pi x}{L}\Bigr)
+ 2\beta \sum_{n=1}^\infty y_n'(t)\,\sin\!\Bigl(\frac{n\pi x}{L}\Bigr)
= -c^2 \sum_{n=1}^\infty \lambda_n y_n(t)\,\sin\!\Bigl(\frac{n\pi x}{L}\Bigr)
  + \cos(\omega t)\sum_{n=1}^\infty f_n\,\sin\!\Bigl(\frac{n\pi x}{L}\Bigr).
\]

Rearranging terms and comparing coefficients of the orthogonal family $\{\sin(n\pi x/L)\}$, we see that for each $n\ge1$,
\[
y_n''(t) + 2\beta\,y_n'(t) + c^2\lambda_n y_n(t) = f_n\cos(\omega t),
\]
where
\[
\lambda_n = \biggl(\frac{n\pi}{L}\biggr)^2.
\]
Thus each modal coefficient $y_n$ satisfies a forced, damped harmonic oscillator equation
\[
y_n'' + 2\beta y_n' + \Omega_n^2\,y_n = f_n\cos(\omega t),
\]
with ``natural frequency'' $\Omega_n = c\sqrt{\lambda_n} = c n\pi/L$. The initial conditions $u(x,0)=0$ and $u_t(x,0)=0$ imply
\[
y_n(0) = 0,\qquad y_n'(0)=0,\qquad n=1,2,\dots,
\]
since the sine basis is complete and orthogonal.

This completes the reduction to an infinite family of ordinary differential equations.

\medskip

\noindent\textbf{(b) Solution of the modal ODEs and the steady state.}
We now solve, for fixed $n$,
\[
y_n''(t) + 2\beta\,y_n'(t) + \Omega_n^2\,y_n(t) = f_n\cos(\omega t),
\qquad y_n(0)=0,\ y_n'(0)=0,
\]
where $\Omega_n = c n\pi/L$ and $\beta>0$. Since the structure is identical for each $n$, we first treat the general scalar equation
\[
y''(t) + 2\beta\,y'(t) + \Omega^2 y(t) = F\cos(\omega t),
\]
and then specialize.

The corresponding homogeneous equation
\[
y'' + 2\beta y' + \Omega^2 y = 0
\]
has characteristic polynomial $r^2+2\beta r+\Omega^2=0$ with roots
\[
r = -\beta \pm \sqrt{\beta^2-\Omega^2}.
\]
In the underdamped case $\beta<\Omega$, which is the typical physical situation, these roots are complex conjugates
\[
r = -\beta \pm i\,\sqrt{\Omega^2-\beta^2},
\]
and the homogeneous solution can be written as
\[
y_{\mathrm{hom}}(t) = e^{-\beta t}\Bigl(C_1\cos(\omega_d t)+C_2\sin(\omega_d t)\Bigr),
\quad \omega_d := \sqrt{\Omega^2-\beta^2}.
\]
In any case, $y_{\mathrm{hom}}(t)$ decays exponentially to zero as $t\to\infty$ because $\beta>0$.

To find a particular solution of the inhomogeneous equation, we use the method of undetermined coefficients. We seek $y_p(t)$ of the form
\[
y_p(t) = A\cos(\omega t) + B\sin(\omega t),
\]
where $A$ and $B$ are constants to be determined. Differentiating,
\[
y_p'(t) = -A\omega\sin(\omega t) + B\omega\cos(\omega t),
\]
\[
y_p''(t) = -A\omega^2\cos(\omega t) - B\omega^2\sin(\omega t).
\]
Substituting into the differential equation yields
\[
\begin{aligned}
&\bigl(-A\omega^2\cos\omega t - B\omega^2\sin\omega t\bigr)
+ 2\beta\bigl(-A\omega\sin\omega t + B\omega\cos\omega t\bigr)\\
&\qquad + \Omega^2\bigl(A\cos\omega t + B\sin\omega t\bigr)
= F\cos(\omega t).
\end{aligned}
\]
Grouping terms with $\cos(\omega t)$ and $\sin(\omega t)$ separately, we have
\[
\Bigl[(-\omega^2+\Omega^2)A + 2\beta\omega B\Bigr]\cos(\omega t)
+ \Bigl[(-\omega^2+\Omega^2)B - 2\beta\omega A\Bigr]\sin(\omega t)
= F\cos(\omega t).
\]
For this identity to hold for all $t$, the sine and cosine coefficients must match:
\[
(-\omega^2+\Omega^2)A + 2\beta\omega B = F,\qquad
(-\omega^2+\Omega^2)B - 2\beta\omega A = 0.
\]
Solving this $2\times2$ linear system for $A$ and $B$ gives
\[
A = \frac{F(\Omega^2-\omega^2)}{(\Omega^2-\omega^2)^2 + 4\beta^2\omega^2},\qquad
B = \frac{2\beta\omega F}{(\Omega^2-\omega^2)^2 + 4\beta^2\omega^2}.
\]
Thus a particular solution is
\[
y_p(t) = A\cos(\omega t) + B\sin(\omega t)
\]
with $A$ and $B$ as above. The general solution of the inhomogeneous equation is
\[
y(t) = y_{\mathrm{hom}}(t) + y_p(t).
\]

Returning now to the modal equation for $y_n$, we set $F=f_n$ and $\Omega=\Omega_n=c n\pi/L$. With zero initial data $y_n(0)=0$, $y_n'(0)=0$, the homogeneous constants $C_1$ and $C_2$ are chosen so that the total solution satisfies these initial conditions. However, regardless of their precise values, the homogeneous contribution $y_{\mathrm{hom},n}(t)$ decays exponentially because $\beta>0$. Therefore, the \emph{steady-state} or long-time periodic solution is given simply by the particular solution
\[
y_{n,\mathrm{ss}}(t) = A_n\cos(\omega t) + B_n\sin(\omega t),
\]
where
\[
A_n = \frac{f_n(\Omega_n^2-\omega^2)}{(\Omega_n^2-\omega^2)^2 + 4\beta^2\omega^2},
\qquad
B_n = \frac{2\beta\omega f_n}{(\Omega_n^2-\omega^2)^2 + 4\beta^2\omega^2},
\]
and
\[
\Omega_n^2 = c^2\lambda_n = c^2\biggl(\frac{n\pi}{L}\biggr)^2.
\]

Substituting back into the series for $u$, the steady-state solution of the original partial differential equation is
\[
u_{\mathrm{ss}}(x,t)
= \sum_{n=1}^\infty \Bigl(A_n\cos(\omega t) + B_n\sin(\omega t)\Bigr)\,\sin\!\Bigl(\frac{n\pi x}{L}\Bigr).
\]
This is a time-periodic solution with the same driving frequency $\omega$ as the forcing, and with spatial profile given by a Fourier sine series whose coefficients depend on $\omega$, $\beta$, and the forcing profile $f$ through the numbers $f_n$.

\medskip

\noindent\textbf{(c) Amplitudes, frequency dependence, and resonance.}
For each fixed mode $n$, the steady-state motion is sinusoidal in time with frequency $\omega$ and some phase shift relative to the forcing. It is convenient to compute the amplitude of oscillation in the $n$th mode.

The modal contribution is
\[
y_{n,\mathrm{ss}}(t) = A_n\cos(\omega t) + B_n\sin(\omega t).
\]
This can be rewritten in the single-sine form
\[
y_{n,\mathrm{ss}}(t) = R_n(\omega)\,\cos\bigl(\omega t - \phi_n\bigr),
\]
where
\[
R_n(\omega) = \sqrt{A_n^2 + B_n^2}
\]
is the amplitude and $\phi_n$ is a phase shift determined by $\tan\phi_n = B_n/A_n$ (with appropriate quadrant conventions).

Using the previous expressions for $A_n$ and $B_n$, we obtain
\[
\begin{aligned}
A_n^2 + B_n^2
&= \frac{f_n^2(\Omega_n^2-\omega^2)^2}{\bigl[(\Omega_n^2-\omega^2)^2 + 4\beta^2\omega^2\bigr]^2}
 + \frac{4\beta^2\omega^2 f_n^2}{\bigl[(\Omega_n^2-\omega^2)^2 + 4\beta^2\omega^2\bigr]^2}\\
&= \frac{f_n^2\bigl[(\Omega_n^2-\omega^2)^2 + 4\beta^2\omega^2\bigr]}
         {\bigl[(\Omega_n^2-\omega^2)^2 + 4\beta^2\omega^2\bigr]^2}\\
&= \frac{f_n^2}{(\Omega_n^2-\omega^2)^2 + 4\beta^2\omega^2}.
\end{aligned}
\]
Therefore the amplitude is
\[
R_n(\omega)
= \frac{|f_n|}{\sqrt{(\Omega_n^2-\omega^2)^2 + 4\beta^2\omega^2}}
= \frac{|f_n|}
       {\sqrt{\bigl(c^2(n\pi/L)^2-\omega^2\bigr)^2 + 4\beta^2\omega^2}}.
\]
This is precisely the formula requested in the problem statement.

Now we discuss the dependence on $\omega$ and $\beta$. As a function of $\omega$, for fixed $n$ and fixed parameters $c$, $L$, $\beta$, and $f_n$, the amplitude $R_n(\omega)$ behaves like the familiar resonance curve of a forced, damped harmonic oscillator:

\begin{itemize}
  \item The quantity $\Omega_n = c n\pi/L$ is the natural angular frequency of the $n$th mode of the undamped, unforced string. The denominator of $R_n(\omega)$ is smallest, and hence $R_n(\omega)$ is largest, for driving frequencies $\omega$ near $\Omega_n$, up to a small shift due to damping. This is the resonance phenomenon: the system responds most strongly when driven near its natural frequency.

  \item The presence of $\beta>0$ prevents the denominator from vanishing. In the undamped case $\beta=0$, the denominator would be $|\Omega_n^2-\omega^2|$, which becomes arbitrarily small as $\omega\to\Omega_n$, leading to unbounded growth of the amplitude in time (true resonance). With damping, however, the term $4\beta^2\omega^2$ ensures that the denominator remains positive, and the amplitude remains finite for all $\omega$.

  \item Increasing $\beta$ broadens and lowers the resonance peak: the maximum of $R_n(\omega)$ decreases, and the frequency interval over which the amplitude is relatively large widens. Physically, stronger damping dissipates energy more quickly, so even when the system is driven near resonance it cannot build up very large oscillations.

  \item For driving frequencies far from resonance ($|\omega|\gg\Omega_n$ or $|\omega|\ll\Omega_n$), the denominator is large and $R_n(\omega)$ is small, meaning that the forcing is inefficient at exciting that mode.
\end{itemize}

Putting all the modes together, the steady-state displacement is a superposition of sinusoidal oscillations at the driving frequency $\omega$, each weighted by the amplitude $R_n(\omega)$ and by the spatial mode shape $\sin(n\pi x/L)$. The forcing profile $f(x)$ enters only through its Fourier coefficients $
$f_n$; modes for which $|f_n|$ is small are weakly excited even if the driving frequency is close to their natural frequency, whereas modes with large $|f_n|$ dominate the steady-state response.

\end{solution}

% ===== Example 5: Two-Dimensional Membrane Vibrations (inquiry-based) =====
\begin{problem}[Two-Dimensional Membrane Vibrations]
A thin, elastic membrane is stretched over a rectangular frame, like a drumhead, and clamped along the entire boundary so that its edge cannot move. Let $u(x,y,t)$ denote the vertical displacement of the membrane above the $(x,y)$-plane at time $t$. Under standard small-amplitude and homogeneous-tension assumptions, $u$ satisfies the two-dimensional wave equation on the interior of the rectangle. In this problem you will rediscover how to separate variables in two spatial dimensions, identify the normal modes of vibration, and express the general motion as a superposition of these modes. Along the way, notice how the two one-dimensional problems in $x$ and $y$ combine to give a richer structure of eigenvalues and mode shapes.

Consider a rectangular membrane occupying the region
\[
0 < x < a,\qquad 0 < y < b,
\]
with wave speed $c>0$, whose boundary is fixed (clamped) along the entire rectangle. The vertical displacement $u(x,y,t)$ satisfies the two-dimensional wave equation
\[
u_{tt} = c^2 \bigl(u_{xx}+u_{yy}\bigr)
\]
for $0<x<a$, $0<y<b$, $t>0$.

\medskip

(a) Write down the boundary conditions that correspond to the membrane being fixed along all four sides of the rectangle. Then state a general form of initial conditions describing an initial displacement and initial velocity of the membrane. That is, give $u(x,y,0)$ and $u_t(x,y,0)$ in terms of arbitrary functions $f(x,y)$ and $g(x,y)$.

\medskip

(b) We aim to solve the initial–boundary value problem using separation of variables in the form
\[
u(x,y,t) = X(x)\,Y(y)\,T(t).
\]
Substitute this ansatz into the wave equation and separate the variables.

\begin{itemize}
\item[(i)] After substituting $u = X Y T$, divide the resulting equation so that each term depends on only one variable (or combination of variables). Show that you can rewrite the PDE in the form
\[
\frac{T''(t)}{c^2 T(t)} = \frac{X''(x)}{X(x)} + \frac{Y''(y)}{Y(y)}.
\]
Why must both sides of this equation be equal to a \emph{constant}? Introduce a separation constant and explain your choice of its sign.

\item[(ii)] Perform a second separation to split the $x$- and $y$-dependent parts. Introduce appropriate separation constants so that you arrive at three ordinary differential equations: one for $T(t)$, one for $X(x)$, and one for $Y(y)$.
\end{itemize}

Hint: Follow the pattern from the one-dimensional wave equation: try to arrange things so that the time equation is of the form $T'' + \omega^2 T = 0$ with $\omega^2>0$, giving oscillatory behavior in time.

\medskip

(c) The boundary conditions at $x=0$ and $x=a$ imply homogeneous boundary conditions for $X(x)$, and similarly the conditions at $y=0$ and $y=b$ imply conditions for $Y(y)$.

\begin{itemize}
\item[(i)] Using your separated ODEs, write the boundary value problems satisfied by $X(x)$ and $Y(y)$. Show that you obtain two Sturm–Liouville eigenvalue problems of the form
\[
X''(x) + \lambda_x X(x)=0,\quad X(0)=X(a)=0,
\]
\[
Y''(y) + \lambda_y Y(y)=0,\quad Y(0)=Y(b)=0,
\]
for suitable eigenvalues $\lambda_x$ and $\lambda_y$.

\item[(ii)] Solve each of these one-dimensional eigenvalue problems. Determine the eigenvalues and corresponding eigenfunctions explicitly. You should find discrete families
\[
\lambda_x = \left(\frac{m\pi}{a}\right)^2,\qquad X_m(x) = \sin\!\left(\frac{m\pi x}{a}\right),\quad m=1,2,3,\dots,
\]
and an analogous description for $Y_n(y)$.

\item[(iii)] Using your separation constants, express the corresponding time equation for the $(m,n)$-mode as an ODE for $T_{m,n}(t)$ and solve it.
\end{itemize}

Hint: For part (ii), recall the one-dimensional string fixed at both ends, and reuse that analysis carefully.

\medskip

(d) Combine your results to construct the normal modes of vibration and the general solution.

\begin{itemize}
\item[(i)] Show that for each pair of positive integers $(m,n)$ you obtain a separated solution of the form
\[
u_{m,n}(x,y,t) = \sin\!\left(\frac{m\pi x}{a}\right)\sin\!\left(\frac{n\pi y}{b}\right)
\Bigl(A_{m,n}\cos(\omega_{m,n} t) + B_{m,n}\sin(\omega_{m,n} t)\Bigr),
\]
and identify the angular frequency $\omega_{m,n}$ in terms of $c$, $a$, $b$, $m$, and $n$.

\item[(ii)] Argue (using the superposition principle) that the general solution satisfying the homogeneous boundary conditions can be written as a double series
\[
u(x,y,t) = \sum_{m=1}^{\infty}\sum_{n=1}^{\infty}
\sin\!\left(\frac{m\pi x}{a}\right)\sin\!\left(\frac{n\pi y}{b}\right)
\Bigl(A_{m,n}\cos(\omega_{m,n} t) + B_{m,n}\sin(\omega_{m,n} t)\Bigr).
\]

\item[(iii)] Use the initial conditions to derive formulas for $A_{m,n}$ and $B_{m,n}$ as double Fourier sine coefficients of $f(x,y)$ and $g(x,y)$. Write these formulas explicitly as integrals over the rectangle $(0,a)\times(0,b)$.

\end{itemize}

Hint: Use the orthogonality of the sine functions in $x$ and in $y$ separately, and then combine them. For example,
\[
\int_0^a \sin\!\left(\frac{m\pi x}{a}\right)\sin\!\left(\frac{m'\pi x}{a}\right)\,dx
= \begin{cases}
0, & m\neq m',\\[4pt]
\frac{a}{2}, & m=m'.
\end{cases}
\]

\medskip

(e) Extensions and variations.

\begin{itemize}
\item[(i)] \emph{Neumann boundary conditions:} Suppose the membrane is attached to a frictionless frame so that the slope normal to the boundary is zero, but the edge is free to move up and down. This is modeled by homogeneous Neumann boundary conditions $u_x(0,y,t)=u_x(a,y,t)=0$ and $u_y(x,0,t)=u_y(x,b,t)=0$. How does the spatial eigenvalue problem change? What do the $X_m(x)$ and $Y_n(y)$ eigenfunctions look like in this case?

\item[(ii)] \emph{Circular membrane:} Suppose instead that the membrane is a circular drum of radius $R$ and you use polar coordinates $(r,\theta)$. Without carrying out all computations, outline how the separation-of-variables procedure would proceed starting from the wave equation
\[
u_{tt} = c^2\left(u_{rr} + \frac{1}{r}u_r + \frac{1}{r^2}u_{\theta\theta}\right),
\]
with $u(R,\theta,t)=0$. Which new special functions arise in the radial eigenvalue problem, and why is this problem more intricate than the rectangular case?
\end{itemize}

\end{problem}

% ===== Example 5: Two-Dimensional Membrane Vibrations (full solution) =====
\begin{problem}[Two-Dimensional Membrane Vibrations]
Consider a rectangular membrane occupying $0<x<a$, $0<y<b$, with constant wave speed $c>0$ and clamped edges. The vertical displacement $u(x,y,t)$ satisfies
\[
u_{tt} = c^2(u_{xx}+u_{yy}),\qquad 0<x<a,\ 0<y<b,\ t>0,
\]
with boundary conditions
\[
u(0,y,t)=u(a,y,t)=u(x,0,t)=u(x,b,t)=0,
\]
and initial data
\[
u(x,y,0) = f(x,y),\qquad u_t(x,y,0) = g(x,y).
\]
Using separation of variables, find the normal modes of vibration and their frequencies, and show that the solution can be written as
\[
u(x,y,t) = \sum_{m=1}^{\infty}\sum_{n=1}^{\infty}
\sin\!\left(\frac{m\pi x}{a}\right)\sin\!\left(\frac{n\pi y}{b}\right)
\Bigl(A_{m,n}\cos(\omega_{m,n} t) + B_{m,n}\sin(\omega_{m,n} t)\Bigr),
\]
for suitable $\omega_{m,n}$, $A_{m,n}$, and $B_{m,n}$. Express $A_{m,n}$ and $B_{m,n}$ explicitly as double Fourier sine coefficients of $f$ and $g$.
\end{problem}

\begin{solution}
We solve the two-dimensional wave equation on a rectangle with homogeneous Dirichlet boundary conditions by separation of variables. This example illustrates the main ideas for hyperbolic partial differential equations in a homogeneous medium: separation into simpler eigenvalue problems, identification of normal modes, and expansion of general solutions as orthogonal series.

\medskip

\noindent\textbf{1. Separation of variables.}
We seek separated solutions of the form
\[
u(x,y,t) = X(x)\,Y(y)\,T(t).
\]
Substituting into the PDE $u_{tt}=c^2(u_{xx}+u_{yy})$ gives
\[
X(x)Y(y)T''(t) = c^2\bigl(X''(x)Y(y)T(t) + X(x)Y''(y)T(t)\bigr).
\]
Assuming $X$, $Y$, and $T$ are nonzero, we divide both sides by $c^2 X(x)Y(y)T(t)$ to obtain
\[
\frac{T''(t)}{c^2 T(t)} = \frac{X''(x)}{X(x)} + \frac{Y''(y)}{Y(y)}.
\]
The left-hand side depends on $t$ only, while the right-hand side depends on $x$ and $y$. Therefore both sides must be equal to a constant; denote this constant by $-\lambda$:
\[
\frac{T''(t)}{c^2 T(t)} = \frac{X''(x)}{X(x)} + \frac{Y''(y)}{Y(y)} = -\lambda.
\]
We will see that $\lambda>0$, leading to oscillatory time dependence.

Rewriting,
\[
\frac{X''(x)}{X(x)} + \frac{Y''(y)}{Y(y)} = -\lambda.
\]
Rearrange to group $x$- and $y$-dependence separately:
\[
\frac{X''(x)}{X(x)} + \lambda = -\,\frac{Y''(y)}{Y(y)}.
\]
The left side depends on $x$ only, the right side on $y$ only, so both must equal another constant, say $\mu$:
\[
\frac{X''(x)}{X(x)} + \lambda = \mu, \qquad -\frac{Y''(y)}{Y(y)} = \mu.
\]
Thus we obtain
\[
X''(x) = (\mu - \lambda) X(x), \qquad Y''(y) = -\mu\,Y(y),
\]
and the time equation
\[
T''(t) + c^2\lambda\,T(t) = 0.
\]

It is more convenient to parameterize by nonnegative constants that will become spatial eigenvalues. Write
\[
X''(x) + \lambda_x X(x) = 0,\qquad Y''(y) + \lambda_y Y(y) = 0,
\]
with
\[
\lambda_x>0,\quad \lambda_y>0,\quad \lambda_x + \lambda_y = \lambda,
\]
so that the time equation becomes
\[
T''(t) + c^2(\lambda_x+\lambda_y) T(t) = 0.
\]

\medskip

\noindent\textbf{2. Boundary conditions and spatial eigenvalue problems.}
The boundary conditions on $u$ induce boundary conditions on $X$ and $Y$.

From $u(0,y,t)=0$ for all $y$ and $t$, and $u(x,y,t)=X(x)Y(y)T(t)$, we have
\[
X(0)Y(y)T(t) = 0 \quad \text{for all } y,t.
\]
Nontrivial solutions require $Y$ and $T$ not identically zero, so we must have
\[
X(0) = 0.
\]
Similarly, from $u(a,y,t)=0$ for all $y,t$ we get $X(a)=0$. The conditions $u(x,0,t)=0$ and $u(x,b,t)=0$ give
\[
Y(0) = 0,\qquad Y(b)=0.
\]
Hence $X$ and $Y$ satisfy the Sturm–Liouville problems
\[
X''(x) + \lambda_x X(x)=0,\quad X(0)=X(a)=0,
\]
\[
Y''(y) + \lambda_y Y(y)=0,\quad Y(0)=Y(b)=0.
\]

These are the same eigenvalue problems that arise for a one-dimensional string fixed at both ends.

\medskip

\noindent\textbf{3. Solving the one-dimensional eigenvalue problems.}
Consider first $X'' + \lambda_x X = 0$ with $X(0)=X(a)=0$. As in the one-dimensional case, nontrivial solutions occur only for
\[
\lambda_x = \left(\frac{m\pi}{a}\right)^2,\quad m=1,2,3,\dots,
\]
with eigenfunctions
\[
X_m(x) = \sin\!\left(\frac{m\pi x}{a}\right).
\]
Similarly, for $Y$ we find
\[
\lambda_y = \left(\frac{n\pi}{b}\right)^2,\quad n=1,2,3,\dots,
\]
with eigenfunctions
\[
Y_n(y) = \sin\!\left(\frac{n\pi y}{b}\right).
\]
Each pair of positive integers $(m,n)$ yields a spatial eigenfunction
\[
\Phi_{m,n}(x,y) = X_m(x)Y_n(y)
= \sin\!\left(\frac{m\pi x}{a}\right)\sin\!\left(\frac{n\pi y}{b}\right).
\]

For this $(m,n)$-mode, the total spatial eigenvalue in the time equation is
\[
\lambda = \lambda_x + \lambda_y
= \left(\frac{m\pi}{a}\right)^2 + \left(\frac{n\pi}{b}\right)^2.
\]
Thus the time equation becomes
\[
T_{m,n}''(t) + c^2\biggl[\left(\frac{m\pi}{a}\right)^2 + \left(\frac{n\pi}{b}\right)^2\biggr]\,T_{m,n}(t) = 0.
\]
Set
\[
\omega_{m,n} = c\,\pi \sqrt{\left(\frac{m}{a}\right)^2 + \left(\frac{n}{b}\right)^2},
\]
so that the solution of the time ODE is
\[
T_{m,n}(t) = A_{m,n}\cos(\omega_{m,n} t) + B_{m,n}\sin(\omega_{m,n} t),
\]
for arbitrary constants $A_{m,n}$ and $B_{m,n}$.

\medskip

\noindent\textbf{4. Normal modes and general solution.}
Combining the spatial and temporal factors, the separated solution corresponding to the pair $(m,n)$ is
\[
u_{m,n}(x,y,t) =
\sin\!\left(\frac{m\pi x}{a}\right)\sin\!\left(\frac{n\pi y}{b}\right)
\Bigl(A_{m,n}\cos(\omega_{m,n} t) + B_{m,n}\sin(\omega_{m,n} t)\Bigr).
\]
Each $u_{m,n}$ is a \emph{normal mode} of vibration, with fixed spatial pattern
\[
\sin\!\left(\frac{m\pi x}{a}\right)\sin\!\left(\frac{n\pi y}{b}\right)
\]
and time-oscillation frequency $\omega_{m,n}$. 

The wave equation is linear with homogeneous boundary conditions, so we may superpose all such modes. The most general solution satisfying the boundary conditions can therefore be expressed as the double series
\[
u(x,y,t) = \sum_{m=1}^{\infty}\sum_{n=1}^{\infty}
\sin\!\left(\frac{m\pi x}{a}\right)\sin\!\left(\frac{n\pi y}{b}\right)
\Bigl(A_{m,n}\cos(\omega_{m,n} t) + B_{m,n}\sin(\omega_{m,n} t)\Bigr).
\]

\medskip

\noindent\textbf{5. Imposing the initial conditions.}
We now determine $A_{m,n}$ and $B_{m,n}$ from the given initial data
\[
u(x,y,0) = f(x,y),\qquad u_t(x,y,0) = g(x,y).
\]

At $t=0$ we have
\[
u(x,y,0) = \sum_{m=1}^{\infty}\sum_{n=1}^{\infty}
\sin\!\left(\frac{m\pi x}{a}\right)\sin\!\left(\frac{n\pi y}{b}\right)
\Bigl(A_{m,n}\cos 0 + B_{m,n}\sin 0\Bigr)
= \sum_{m,n} A_{m,n}
\sin\!\left(\frac{m\pi x}{a}\right)\sin\!\left(\frac{n\pi y}{b}\right).
\]
Thus $f$ has a double sine series expansion
\[
f(x,y) = \sum_{m=1}^{\infty}\sum_{n=1}^{\infty}
A_{m,n}\,\sin\!\left(\frac{m\pi x}{a}\right)\sin\!\left(\frac{n\pi y}{b}\right).
\]

Similarly, differentiate $u$ with respect to $t$:
\[
u_t(x,y,t) = \sum_{m=1}^{\infty}\sum_{n=1}^{\infty}
\sin\!\left(\frac{m\pi x}{a}\right)\sin\!\left(\frac{n\pi y}{b}\right)
\Bigl(-A_{m,n}\omega_{m,n}\sin(\omega_{m,n} t) + B_{m,n}\omega_{m,n}\cos(\omega_{m,n} t)\Bigr).
\]
At $t=0$,
\[
u_t(x,y,0) = \sum_{m=1}^{\infty}\sum_{n=1}^{\infty}
\sin\!\left(\frac{m\pi x}{a}\right)\sin\!\left(\frac{n\pi y}{b}\right)
\bigl(B_{m,n}\omega_{m,n}\bigr).
\]
Thus $g$ has the double sine series expansion
\[
g(x,y) = \sum_{m=1}^{\infty}\sum_{n=1}^{\infty}
B_{m,n}\omega_{m,n}\,\sin\!\left(\frac{m\pi x}{a}\right)\sin\!\left(\frac{n\pi y}{b}\right).
\]

We use orthogonality of sines on $(0,a)$ and $(0,b)$ to compute the coefficients. Recall
\[
\int_0^a \sin\!\left(\frac{m\pi x}{a}\right)\sin\!\left(\frac{m'\pi x}{a}\right)\,dx
= \begin{cases}
0, & m\neq m',\\[4pt]
\frac{a}{2}, & m=m',
\end{cases}
\]
and analogously
\[
\int_0^b \sin\!\left(\frac{n\pi y}{b}\right)\sin\!\left(\frac{n'\pi y}{b}\right)\,dy
= \begin{cases}
0, & n\neq n',\\[4pt]
\frac{b}{2}, & n=n'.
\end{cases}
\]

To obtain $A_{m,n}$, multiply the expansion for $f(x,y)$ by
\[
\sin\!\left(\frac{m\pi x}{a}\right)\sin\!\left(\frac{n\pi y}{b}\right)
\]
and integrate over the rectangle:
\[
\int_0^a\int_0^b f(x,y)
\sin\!\left(\frac{m\pi x}{a}\right)\sin\!\left(\frac{n\pi y}{b}\right)\,dy\,dx
= \sum_{m',n'} A_{m',n'}
\int_0^a \sin\!\left(\frac{m\pi x}{a}\right)\sin\!\left(\frac{m'\pi x}{a}\right)dx
\int_0^b \sin\!\left(\frac{n\pi y}{b}\right)\sin\!\left(\frac{n'\pi y}{b}\right)dy.
\]
By orthogonality, all terms vanish except when $m'=m$ and $n'=n$. Hence
\[
\int_0^a\int_0^b f(x,y)
\sin\!\left(\frac{m\pi x}{a}\right)\sin\!\left(\frac{n\pi y}{b}\right)\,dy\,dx
= A_{m,n}\,\frac{a}{2}\,\frac{b}{2} = A_{m,n}\,\frac{ab}{4}.
\]
Therefore
\[
A_{m,n} = \frac{4}{ab}\int_0^a\int_0^b f(x,y)
\sin\!\left(\frac{m\pi x}{a}\right)\sin\!\left(\frac{n\pi y}{b}\right)\,dy\,dx.
\]

The same procedure applied to $g(x,y)$ yields
\[
\int_0^a\int_0^b g(x,y)
\sin\!\left(\frac{m\pi x}{a}\right)\sin\!\left(\frac{n\pi y}{b}\right)\,dy\,dx
= B_{m,n}\omega_{m,n}\,\frac{ab}{4},
\]
so
\[
B_{m,n} = \frac{4}{ab\,\omega_{m,n}}
\int_0^a\int_0^b g(x,y)
\sin\!\left(\frac{m\pi x}{a}\right)\sin\!\left(\frac{n\pi y}{b}\right)\,dy\,dx.
\]

\medskip

\noindent\textbf{6. Final form and interpretation.}
We have shown that the displacement of the rectangular membrane can be written as
\[
u(x,y,t) = \sum_{m=1}^{\infty}\sum_{n=1}^{\infty}
\sin\!\left(\frac{m\pi x}{a}\right)\sin\!\left(\frac{n\pi y}{b}\right)
\Bigl(A_{m,n}\cos(\omega_{m,n} t) + B_{m,n}\sin(\omega_{m,n} t)\Bigr),
\]
with
\[
\omega_{m,n} = c\,\pi \sqrt{\left(\frac{m}{a}\right)^2 + \left(\frac{n}{b}\right)^2},
\]
and
\[
A_{m,n} = \frac{4}{ab}\int_0^a\int_0^b f(x,y)
\sin\!\left(\frac{m\pi x}{a}\right)\sin\!\left(\frac{n\pi y}{b}\right)\,dy\,dx,
\]
\[
B_{m,n} = \frac{4}{ab\,\omega_{m,n}}
\int_0^a\int_0^b g(x,y)
\sin\!\left(\frac{m\pi x}{a}\right)\sin\!\left(\frac{n\pi y}{b}\right)\,dy\,dx.
\]

Each normal mode labeled by $(m,n)$ is an eigenfunction of the Laplacian $-\Delta$ with Dirichlet boundary conditions, and its oscillation frequency is determined by the corresponding eigenvalue. The solution is a superposition of these modes with coefficients chosen to match the initial displacement and velocity. This is the two-dimensional analogue of the vibrating string, and it encapsulates the central ideas of waves in a homogeneous medium and hyperbolic partial differential equations: decomposition into eigenmodes, orthogonality, and the role of the spectrum of the spatial operator in determining the temporal behavior.
\end{solution}

\section{Diffusion Equation}
% --- Narrative plan (auto-generated) ---
% This section introduces the diffusion, or heat, equation as a central model for how quantities such as temperature, chemicals, or probability densities spread out in space over time. We begin with concrete physical and probabilistic pictures, such as heat flowing along a metal rod or dye dispersing in water, and progressively uncover the mathematical structure behind these phenomena. Along the way, we learn how local balancing laws, like conservation of energy or mass, naturally lead to partial differential equations that govern the evolution of entire systems.
%
% The diffusion equation sits at an important crossroads in applied mathematics. It is simple enough to be solved explicitly in many settings, yet rich enough to exhibit phenomena such as smoothing, long-time equilibration, and sensitivity to geometry and boundary conditions. Its solutions can often be expressed in terms of Fourier series or Fourier transforms, so studying diffusion gives us an opportunity to practice separation of variables and spectral methods, and to see how tools from ordinary differential equations reappear in the analysis of partial differential equations.
%
% Conceptually, this section connects to several major themes. The Gaussian function arises naturally as the fundamental solution of the diffusion equation, linking our study to probability theory and to complex analysis, where the same Gaussian integrals appear in contour methods. The long-time behavior of solutions can be understood using eigenvalues and eigenfunctions of associated spatial operators, tying diffusion to dynamical systems and linear algebra. By the end of the section, the reader will see how diffusion provides a model laboratory where techniques from Fourier analysis, linear ODE theory, and PDE theory all come together.

% ===== Example 1: Heat Flow in a Finite Rod with Fixed End Temperatures (inquiry-based) =====
\begin{problem}[Heat Flow in a Finite Rod with Fixed End Temperatures]
Consider a thin, homogeneous metal rod of length $L$, lying along the $x$-axis from $x=0$ to $x=L$. The rod is perfectly insulated along its lateral surface, so heat can flow only along the $x$-direction. The left end is held at temperature $0$, and the right end is held at a fixed temperature $T_0>0$. At time $t=0$ the temperature distribution along the rod is given by a prescribed function $f(x)$. We want to derive and solve the mathematical model that describes the temperature $u(x,t)$ in the rod for $t>0$.

(a) Use conservation of energy and Fourier's law of heat conduction to derive the one-dimensional diffusion (heat) equation for the temperature $u(x,t)$ in the rod. Clearly state any physical assumptions you make, and indicate the role of the thermal diffusivity $\kappa>0$.
\medskip

(b) Translate the physical description into mathematical initial and boundary conditions for $u(x,t)$. Write down the initial value problem you obtain in the form
\[
\begin{cases}
u_t = \kappa u_{xx}, & 0<x<L,\ t>0, \\
\text{(boundary conditions)}, & t>0, \\
\text{(initial condition)}, & 0<x<L.
\end{cases}
\]
What is the type of boundary condition at each end (Dirichlet, Neumann, or Robin), and are they homogeneous or inhomogeneous?
\medskip

(c) A standard strategy for dealing with inhomogeneous Dirichlet boundary conditions is to subtract off a steady-state solution. 

\begin{itemize}
\item[(i)] Find a time-independent temperature profile $v(x)$ that solves the \emph{steady-state} boundary value problem
\[
\kappa v''(x) = 0,\quad 0<x<L,\qquad
v(0) = 0,\quad v(L) = T_0.
\]
\item[(ii)] Define a new unknown $w(x,t) = u(x,t) - v(x)$. Show that $w$ satisfies a diffusion equation with \emph{homogeneous} boundary conditions. Write down explicitly the PDE, the boundary conditions, and the initial condition for $w$.
\end{itemize}
Hint: Carefully differentiate $w$ with respect to $t$ and $x$, and use the equation for $u$ and the fact that $v$ is independent of $t$.
\medskip

(d) Now solve the initial-boundary value problem for $w$ by separation of variables and Fourier series.

\begin{itemize}
\item[(i)] Look for separated solutions of the form $w(x,t)=X(x)T(t)$ and derive the two ordinary differential equations for $X$ and $T$, together with the boundary conditions for $X$.
\item[(ii)] Solve the spatial eigenvalue problem for $X(x)$ with homogeneous Dirichlet boundary conditions at $x=0$ and $x=L$. Identify the eigenvalues and the corresponding eigenfunctions.
\item[(iii)] Use these eigenfunctions to write down the general solution for $w(x,t)$ as an infinite series. How are the time-dependent factors determined?
\item[(iv)] Impose the initial condition for $w$ to obtain formulas for the Fourier coefficients in the series. Express the final answer for $u(x,t)$ in terms of $f(x)$, $T_0$, and $\kappa$.
\end{itemize}
Hint: You should obtain a sine series expansion for the function $f(x)-v(x)$ on the interval $(0,L)$.
\medskip

(e) Explore one or two variations on this model.

\begin{itemize}
\item[(i)] Suppose instead that \emph{both} ends of the rod are held at zero temperature, that is, $u(0,t)=u(L,t)=0$, but the initial condition $f(x)$ is nonzero. How does the method of separation of variables simplify in this purely homogeneous Dirichlet case? Which parts of your work in parts (c) and (d) are no longer needed?
\item[(ii)] Suppose the right end of the rod is perfectly insulated rather than held at fixed temperature; that is, the heat flux at $x=L$ is zero. How would you express this as a boundary condition on $u(x,t)$, and how would that change the eigenvalue problem in part (d)? (You do not need to solve the new problem completely, but describe in words and formulas what changes.)
\end{itemize}
\end{problem}

% ===== Example 1: Heat Flow in a Finite Rod with Fixed End Temperatures (full solution) =====
\begin{problem}[Heat Flow in a Finite Rod with Fixed End Temperatures]
A homogeneous rod of length $L$ lies along $0<x<L$ and is insulated along its sides. Its left end is held at temperature $0$ and its right end at temperature $T_0>0$, for all $t>0$. Let $u(x,t)$ denote the temperature at position $x$ and time $t$, and suppose the initial temperature profile is $u(x,0)=f(x)$.

\begin{enumerate}
\item[(i)] Derive the one-dimensional diffusion equation for $u(x,t)$ using conservation of energy and Fourier's law of heat conduction.
\item[(ii)] Solve the resulting initial–boundary value problem
\[
\begin{cases}
u_t = \kappa u_{xx}, & 0<x<L,\ t>0,\\[4pt]
u(0,t)=0,\quad u(L,t)=T_0, & t>0,\\[4pt]
u(x,0)=f(x), & 0<x<L,
\end{cases}
\]
by reducing to a problem with homogeneous boundary conditions and using separation of variables.
\end{enumerate}
Express your final answer for $u(x,t)$ in terms of $f(x)$, $T_0$, and $\kappa$.
\end{problem}

\begin{solution}
We first derive the governing equation and then solve the initial–boundary value problem. This example illustrates the central ideas for the diffusion equation: modeling via conservation and Fourier's law, reduction to homogeneous boundary conditions, and solution by eigenfunction expansions.

\medskip\noindent
\textbf{(i) Derivation of the diffusion equation.}
Let $u(x,t)$ denote the temperature in the rod at position $x$ and time $t$. Consider a small segment of the rod between $x$ and $x+\Delta x$. The amount of heat energy in this piece at time $t$ is
\[
\text{(heat in segment)} = c \rho A\, u(x,t)\,\Delta x,
\]
where $c$ is the specific heat, $\rho$ the density, and $A$ the cross-sectional area. These are assumed constant along the rod because it is homogeneous.

The rate of change of heat in the segment equals the net heat flux into the segment. Denote by $q(x,t)$ the heat flux (heat per unit area per unit time) along the $x$-direction, taken positive in the positive $x$-direction. Fourier's law of heat conduction states that
\[
q(x,t) = -k\,u_x(x,t),
\]
where $k>0$ is the thermal conductivity. Thus heat flows from hotter regions to colder regions.

The total heat flowing into the segment through its left face at $x$ per unit time is $-A q(x,t)$ (because flux in the positive $x$-direction corresponds to heat leaving through the right face). Similarly, the total heat flowing out through the right face at $x+\Delta x$ per unit time is $A q(x+\Delta x,t)$. Therefore the net rate of heat flow \emph{into} the segment is
\[
-A q(x,t) - A q(x+\Delta x,t)
= -A q(x,t) + A\bigl(-q(x+\Delta x,t)\bigr)
= -A\bigl(q(x+\Delta x,t) - q(x,t)\bigr).
\]
By conservation of energy,
\[
\frac{\partial}{\partial t}\bigl(c\rho A\,u(x,t)\,\Delta x\bigr)
= -A\bigl(q(x+\Delta x,t) - q(x,t)\bigr).
\]
Divide by $A\,\Delta x$ and rewrite the right-hand side as a difference quotient:
\[
c\rho\,u_t(x,t)
= -\,\frac{q(x+\Delta x,t) - q(x,t)}{\Delta x}.
\]
Letting $\Delta x\to 0$ and using the definition of the spatial derivative, we obtain
\[
c\rho\,u_t(x,t) = -q_x(x,t).
\]
Substituting Fourier's law $q=-k u_x$ yields
\[
c\rho\,u_t(x,t) = -\bigl(-k u_x(x,t)\bigr)_x = k u_{xx}(x,t).
\]
We define the \emph{thermal diffusivity}
\[
\kappa = \frac{k}{c\rho} > 0,
\]
which has units of length$^2$ per unit time. Then the governing equation becomes
\[
u_t = \kappa\,u_{xx},\qquad 0<x<L,\ t>0,
\]
which is the one-dimensional diffusion (or heat) equation.

\medskip\noindent
\textbf{(ii) Initial and boundary conditions.}
The ends are held at fixed temperatures:
\[
u(0,t) = 0,\qquad u(L,t) = T_0,\qquad t>0.
\]
These are Dirichlet boundary conditions, and they are inhomogeneous because the prescribed values are not both zero. The initial condition is
\[
u(x,0) = f(x),\qquad 0<x<L.
\]
Thus the initial–boundary value problem is exactly as stated in the problem.

\medskip\noindent
\textbf{Reduction to homogeneous boundary conditions.}
The inhomogeneous Dirichlet boundary conditions are not directly compatible with the usual sine-series separation of variables. A standard remedy is to subtract a steady-state solution that matches the boundary data.

We look for a time-independent function $v(x)$ satisfying the steady-state version of the PDE and the boundary conditions:
\[
\kappa v''(x)=0,\quad 0<x<L;\qquad v(0)=0,\quad v(L)=T_0.
\]
Integrating $v''(x)=0$ twice, we find
\[
v(x) = a x + b,
\]
for constants $a$ and $b$. The boundary conditions give
\[
v(0)=b=0,\qquad v(L) = aL + b = aL = T_0,
\]
so $a = T_0/L$. Thus
\[
v(x) = \frac{T_0}{L}x.
\]
This profile is the unique steady-state linear temperature distribution connecting the two end temperatures.

Now define
\[
w(x,t) = u(x,t) - v(x).
\]
We compute its derivatives:
\[
w_t = u_t - v_t = u_t,
\]
since $v$ is independent of time, and
\[
w_{xx} = u_{xx} - v_{xx}.
\]
Because $v''(x)=0$, we have $v_{xx}=0$, and hence $w_{xx}=u_{xx}$. Since $u$ satisfies $u_t = \kappa u_{xx}$, it follows that
\[
w_t = u_t = \kappa u_{xx} = \kappa w_{xx}.
\]
Thus $w$ satisfies the same diffusion equation:
\[
w_t = \kappa w_{xx},\qquad 0<x<L,\ t>0.
\]

Next we check the boundary conditions. At $x=0$ we have
\[
w(0,t) = u(0,t) - v(0) = 0 - 0 = 0,
\]
and at $x=L$,
\[
w(L,t) = u(L,t) - v(L) = T_0 - T_0 = 0.
\]
Hence $w$ satisfies \emph{homogeneous} Dirichlet boundary conditions:
\[
w(0,t)=0,\qquad w(L,t)=0,\qquad t>0.
\]
Finally, the initial condition becomes
\[
w(x,0) = u(x,0) - v(x) = f(x) - \frac{T_0}{L}x,\qquad 0<x<L.
\]

We have reduced the original problem to the following one for $w$:
\[
\begin{cases}
w_t = \kappa w_{xx}, & 0<x<L,\ t>0,\\[4pt]
w(0,t)=0,\quad w(L,t)=0, & t>0,\\[4pt]
w(x,0)=f(x) - \dfrac{T_0}{L}x, & 0<x<L.
\end{cases}
\]

\medskip\noindent
\textbf{Solution by separation of variables and Fourier series.}
We now solve the homogeneous Dirichlet problem for $w$. We look for separated solutions of the form
\[
w(x,t) = X(x)T(t).
\]
Substituting into the PDE gives
\[
X(x)T'(t) = \kappa X''(x)T(t).
\]
Assuming $X$ and $T$ are not identically zero, we divide by $\kappa X(x)T(t)$ to obtain
\[
\frac{T'(t)}{\kappa T(t)} = \frac{X''(x)}{X(x)} = -\lambda,
\]
where $-\lambda$ is a separation constant, independent of $x$ and $t$. This leads to the pair of ordinary differential equations
\[
\begin{cases}
X''(x) + \lambda X(x) = 0,\\[4pt]
T'(t) + \kappa\lambda T(t) = 0.
\end{cases}
\]
The boundary conditions on $w$ give conditions on $X$:
\[
w(0,t)=0 \implies X(0)T(t)=0 \implies X(0)=0, \quad
w(L,t)=0 \implies X(L)=0.
\]
Thus $X$ must satisfy
\[
X''(x) + \lambda X(x) = 0,\quad 0<x<L;\qquad X(0)=0,\ X(L)=0.
\]

This is a classical Sturm–Liouville eigenvalue problem. Nontrivial solutions exist only for certain values of $\lambda$, the eigenvalues, with corresponding eigenfunctions $X(x)$.

To find them, we analyze cases for $\lambda$:

\emph{Case 1:} $\lambda=0$. Then $X''(x)=0$, so $X(x)=ax+b$. The boundary conditions yield $X(0)=b=0$, $X(L)=aL+b=aL=0$, so $a=0$ as well. Thus $X\equiv 0$, which is trivial and discarded.

\emph{Case 2:} $\lambda<0$. Write $\lambda=-\mu^2$ with $\mu>0$. Then
\[
X''(x) - \mu^2 X(x) = 0,
\]
with general solution $X(x)=A e^{\mu x}+B e^{-\mu x}$. The boundary condition $X(0)=0$ gives $A+B=0$, so $B=-A$ and $X(x)=A(e^{\mu x}-e^{-\mu x})=2A\sinh(\mu x)$. The second boundary condition $X(L)=0$ implies $\sinh(\mu L)=0$, which forces $\mu=0$, contradicting $\mu>0$. Hence there are no nontrivial solutions for $\lambda<0$.

\emph{Case 3:} $\lambda>0$. Write $\lambda=\mu^2$ with $\mu>0$. Then
\[
X''(x) + \mu^2 X(x) = 0,
\]
with general solution $X(x)=A\cos(\mu x)+B\sin(\mu x)$. The condition $X(0)=0$ gives $A=0$, so $X(x)=B\sin(\mu x)$. The condition $X(L)=0$ then requires
\[
B\sin(\mu L)=0.
\]
For a nontrivial solution we need $B\neq 0$, so $\sin(\mu L)=0$, which implies
\[
\mu L = n\pi,\qquad n=1,2,3,\dots.
\]
Thus $\mu_n = \dfrac{n\pi}{L}$, and the eigenvalues are
\[
\lambda_n = \mu_n^2 = \left(\frac{n\pi}{L}\right)^2,\qquad n=1,2,3,\dots,
\]
with corresponding eigenfunctions
\[
X_n(x) = \sin\left(\frac{n\pi x}{L}\right).
\]

For each $\lambda_n$, the temporal equation $T'(t)+\kappa\lambda_n T(t)=0$ has solution
\[
T_n(t) = C_n \exp\bigl(-\kappa\lambda_n t\bigr)
= C_n \exp\left(-\kappa \left(\frac{n\pi}{L}\right)^2 t\right),
\]
where $C_n$ is a constant.

Multiplying $X_n$ and $T_n$, we obtain separated solutions
\[
w_n(x,t)=\sin\left(\frac{n\pi x}{L}\right)\exp\left(-\kappa\left(\frac{n\pi}{L}\right)^2 t\right).
\]
By linearity of the PDE and boundary conditions, any linear combination of these is again a solution. Hence the general solution satisfying homogeneous Dirichlet boundary conditions is
\[
w(x,t) = \sum_{n=1}^{\infty} b_n \sin\left(\frac{n\pi x}{L}\right)\exp\left(-\kappa\left(\frac{n\pi}{L}\right)^2 t\right),
\]
where the coefficients $b_n$ are determined from the initial condition.

We impose $w(x,0)=f(x)-\dfrac{T_0}{L}x$. At $t=0$ the exponential factors equal $1$, so
\[
w(x,0) = \sum_{n=1}^{\infty} b_n \sin\left(\frac{n\pi x}{L}\right)
= f(x) - \frac{T_0}{L}x,\qquad 0<x<L.
\]
This expresses the function $f(x)-\dfrac{T_0}{L}x$ as a Fourier sine series on $(0,L)$. The standard formula for the sine coefficients gives
\[
b_n = \frac{2}{L}\int_0^L\left(f(\xi) - \frac{T_0}{L}\xi\right)\sin\left(\frac{n\pi \xi}{L}\right)\,d\xi,
\qquad n=1,2,3,\dots.
\]

\medskip\noindent
\textbf{Final expression for the temperature.}
Recall that $u(x,t) = v(x) + w(x,t)$, with $v(x) = \dfrac{T_0}{L}x$. Therefore
\[
u(x,t) = \frac{T_0}{L}x
+ \sum_{n=1}^{\infty} b_n \sin\left(\frac{n\pi x}{L}\right)
\exp\left(-\kappa\left(\frac{n\pi}{L}\right)^2 t\right),
\]
where
\[
b_n = \frac{2}{L}\int_0^L\left(f(\xi) - \frac{T_0}{L}\xi\right)\sin\left(\frac{n\pi \xi}{L}\right)\,d\xi.
\]

This solution consists of a steady-state linear temperature profile plus a transient part expressed as a sine series with exponentially decaying modes. Each mode $\sin\left(\dfrac{n\pi x}{L}\right)$ decays at a rate proportional to $\exp\left(-\kappa\left(\dfrac{n\pi}{L}\right)^2 t\right)$, so higher spatial frequencies (larger $n$) decay faster. This behavior—smoothing in space and decay of high-frequency components in time—is characteristic of solutions of the diffusion equation and is a central qualitative feature of parabolic partial differential equations.
\end{solution}

% ===== Example 2: Infinite Line and the Gaussian Fundamental Solution (inquiry-based) =====
\begin{problem}[Infinite Line and the Gaussian Fundamental Solution]
Consider heat diffusion along an infinitely long, homogeneous rod. We assume the material properties are constant, and there are no internal sources or sinks of heat. At time $t=0$ we place a very concentrated amount of heat near $x=0$, so that the initial temperature is sharply peaked there and essentially zero far away. Symmetry and translation invariance suggest that the resulting temperature profile should spread out in a way that depends only on the distance from the origin and on time, and that the ``shape'' of this profile might be self-similar as time evolves.

We model the temperature $u(x,t)$ for $x\in\mathbb{R}$ and $t>0$ by the \emph{diffusion equation}
\[
u_t = k\,u_{xx}, \qquad -\infty < x < \infty,\ t>0,
\]
where $k>0$ is the thermal diffusivity.

\smallskip

(a) Physically, for an infinite rod with no sources or sinks, the \emph{total heat} should be conserved in time. Argue that the total heat at time $t$ is given by
\[
H(t) = \int_{-\infty}^{\infty} u(x,t)\,dx,
\]
and use the diffusion equation to show that $H'(t)=0$ provided $u$ and its derivatives decay sufficiently fast as $|x|\to\infty$. In other words, show that $H(t)$ is constant in time.  
Hint: Differentiate $H(t)$ under the integral sign and integrate by parts.

\smallskip

(b) We idealize a ``point heat source'' at the origin by requiring that the initial condition has unit total heat concentrated at $x=0$, that is,
\[
H(0)=\int_{-\infty}^{\infty} u(x,0)\,dx = 1,
\]
and that $u(x,0)$ is extremely localized near $x=0$. On the infinite line, the diffusion equation has several symmetries: translations in $x$, reflections $x\mapsto -x$, and a scaling $(x,t)\mapsto(\lambda x,\lambda^2 t)$ for any $\lambda>0$.

Explain why it is reasonable, based on these symmetries and on conservation of total heat, to look for a solution of the form
\[
u(x,t) = t^{-\alpha}\,\phi\!\left(\eta\right), \qquad \eta = \frac{x}{\sqrt{t}},
\]
for some exponent $\alpha$ and some profile function $\phi\colon\mathbb{R}\to\mathbb{R}$.  
(i) Use conservation of total heat to determine $\alpha$.  
(ii) Briefly describe the physical meaning of the scaling variable $\eta=x/\sqrt{t}$.

\smallskip

(c) Let $\alpha$ be the value you found in part (b), and write
\[
u(x,t) = t^{-\alpha}\,\phi(\eta),\qquad \eta = \frac{x}{\sqrt{t}}.
\]
Compute $u_t$ and $u_{xx}$ in terms of $\phi$ and its derivatives, and substitute into the diffusion equation $u_t = k u_{xx}$ to obtain an ordinary differential equation (ODE) for $\phi$. Write this ODE explicitly.  
Hint: Use the chain rule carefully:
\[
\eta = x t^{-1/2},\quad \frac{\partial\eta}{\partial t} = -\frac{\eta}{2t},\quad
\frac{\partial\eta}{\partial x} = t^{-1/2}.
\]

\smallskip

(d) Solve the ODE you obtained in part (c). Show that, under natural conditions of symmetry (evenness in $x$) and boundedness as $|\eta|\to\infty$, the profile must have a Gaussian form
\[
\phi(\eta) = C\,e^{-a\eta^2}
\]
for some constants $C>0$ and $a>0$, and determine $a$ in terms of the diffusivity $k$. Then use the condition that the total heat is $1$ to determine $C$, and write the explicit formula
\[
u(x,t) = G(x,t) = \frac{1}{\sqrt{4\pi k t}}\,
\exp\!\left(-\frac{x^2}{4kt}\right).
\]
Check directly (by differentiating) that $G$ satisfies the diffusion equation, and that $\displaystyle\int_{-\infty}^{\infty} G(x,t)\,dx = 1$ for all $t>0$.  
Hint: To evaluate the integral, use the change of variables $y = x/\sqrt{4kt}$ and the standard Gaussian integral $\displaystyle\int_{-\infty}^{\infty} e^{-y^2}\,dy = \sqrt{\pi}$.

\smallskip

(e) (Extensions and ``what if'' questions.)

(i) The function $G(x,t)$ is called the \emph{fundamental solution} of the diffusion equation on the line. Suppose now that the initial temperature distribution is some integrable function $f(x)$, not just a point source. Based on the linearity and translation invariance of the diffusion equation, formulate a conjecture for $u(x,t)$ in terms of $G$ and $f$. (You do not need to prove it rigorously yet.)

(ii) How do you expect the fundamental solution to change if the diffusion coefficient is a different positive constant $\kappa\neq k$? What would change in the formula, and what would stay the same?

(iii) Finally, speculate about what the fundamental solution might look like in higher spatial dimensions (for instance, on $\mathbb{R}^2$ or $\mathbb{R}^3$). Which parts of your derivation above might carry over, and which parts might need to be modified?
\end{problem}

% ===== Example 2: Infinite Line and the Gaussian Fundamental Solution (full solution) =====
\begin{problem}[Infinite Line and the Gaussian Fundamental Solution]
Consider the diffusion equation on the whole real line
\[
u_t = k\,u_{xx}, \qquad -\infty < x < \infty,\ t>0,
\]
with a point source of unit total heat at the origin at time $t=0$, modeled by the initial condition $u(x,0)=\delta(x)$ in the sense of distributions.  

(a) Using conservation of total heat and the scaling invariance of the equation, seek a self-similar solution of the form
\[
u(x,t) = t^{-\alpha}\,\phi\!\left(\frac{x}{\sqrt{t}}\right),
\]
and determine the correct exponent $\alpha$.  

(b) Derive the corresponding ordinary differential equation for the profile $\phi$ and solve it under natural conditions to show that $\phi$ must be a Gaussian.  

(c) Determine the normalization constant using $\displaystyle\int_{-\infty}^{\infty} u(x,t)\,dx = 1$, and conclude that the fundamental solution is
\[
G(x,t) = \frac{1}{\sqrt{4\pi k t}}\,
\exp\!\left(-\frac{x^2}{4kt}\right).
\]

(d) Verify directly that $G$ satisfies $G_t = k G_{xx}$ for $t>0$, that $\displaystyle\int_{-\infty}^{\infty} G(x,t)\,dx = 1$ for all $t>0$, and that $G(\cdot,t)$ tends to $\delta$ as $t\to 0^+$ in the sense of distributions.
\end{problem}

\begin{solution}
We consider the diffusion equation
\[
u_t = k\,u_{xx}, \qquad x\in\mathbb{R},\ t>0,
\]
with a unit point source at the origin at time $t=0$. The constant $k>0$ is the thermal diffusivity.

\medskip

\noindent\textbf{1. Conservation of total heat and the similarity ansatz.}

The total heat at time $t$ is
\[
H(t) = \int_{-\infty}^{\infty} u(x,t)\,dx.
\]
Assuming $u$ and $u_x$ decay sufficiently rapidly as $|x|\to\infty$, we differentiate under the integral sign and use the PDE:
\[
\frac{dH}{dt}
= \int_{-\infty}^{\infty} u_t(x,t)\,dx
= \int_{-\infty}^{\infty} k\,u_{xx}(x,t)\,dx.
\]
Integrating by parts gives
\[
\int_{-\infty}^{\infty} u_{xx}(x,t)\,dx
= u_x(x,t)\big|_{x=-\infty}^{x=\infty} = 0,
\]
so $H'(t)=0$. Thus the total heat is conserved:
\[
H(t) \equiv H(0).
\]
For a point source of unit strength at $t=0$ we impose $H(0)=1$, so we expect
\[
\int_{-\infty}^{\infty} u(x,t)\,dx = 1 \quad\text{for all } t>0.
\]

On the whole line, the diffusion equation is invariant under space translations and under the scaling
\[
x\mapsto \lambda x,\qquad t\mapsto \lambda^2 t,
\]
for any $\lambda>0$. The initial condition is localized at $x=0$, and there is no distinguished length scale in the problem. This suggests that the profile of $u(x,t)$ at different times should be related by such a scaling, so we seek a self-similar solution of the form
\[
u(x,t) = t^{-\alpha}\,\phi(\eta),\qquad \eta = \frac{x}{\sqrt{t}},
\]
for some exponent $\alpha$ and some profile function $\phi$.

We determine $\alpha$ using conservation of total heat. Compute
\[
\int_{-\infty}^{\infty} u(x,t)\,dx
= \int_{-\infty}^{\infty} t^{-\alpha}\,\phi\!\left(\frac{x}{\sqrt{t}}\right)\,dx.
\]
Make the change of variables $\eta = x/\sqrt{t}$, so that $x = \eta\sqrt{t}$ and $dx = \sqrt{t}\,d\eta$. Then
\[
\int_{-\infty}^{\infty} u(x,t)\,dx
= t^{-\alpha} \int_{-\infty}^{\infty} \phi(\eta)\,\sqrt{t}\,d\eta
= t^{-\alpha + 1/2} \int_{-\infty}^{\infty} \phi(\eta)\,d\eta.
\]
For this to be independent of $t$, we must have $-\alpha + 1/2 = 0$, so
\[
\alpha = \frac{1}{2}.
\]
Hence we consider
\[
u(x,t) = t^{-1/2}\,\phi(\eta), \qquad \eta = \frac{x}{\sqrt{t}}.
\]
The variable $\eta$ is a dimensionless similarity variable: as time grows, the characteristic spatial scale of diffusion grows like $\sqrt{t}$, so the solution at different times can be compared by rescaling $x$ by $\sqrt{t}$.

\medskip

\noindent\textbf{2. Deriving the ODE for the similarity profile.}

With
\[
u(x,t) = t^{-1/2}\,\phi(\eta),\qquad \eta = x t^{-1/2},
\]
we compute $u_t$ and $u_{xx}$ via the chain rule. First,
\[
u_t = \frac{\partial}{\partial t}\Bigl(t^{-1/2}\Bigr)\,\phi(\eta)
      + t^{-1/2}\,\phi'(\eta)\,\eta_t,
\]
where the prime denotes differentiation with respect to $\eta$. We have
\[
\frac{\partial}{\partial t}\Bigl(t^{-1/2}\Bigr)
= -\frac{1}{2}t^{-3/2}
\]
and
\[
\eta_t = \frac{\partial}{\partial t}(x t^{-1/2})
       = x\left(-\frac{1}{2}t^{-3/2}\right)
       = -\frac{1}{2} t^{-1}\,\eta.
\]
Therefore
\[
u_t = -\frac{1}{2} t^{-3/2}\,\phi(\eta)
      + t^{-1/2}\,\phi'(\eta)\left(-\frac{1}{2}t^{-1}\eta\right)
    = t^{-3/2}\Bigl(-\tfrac{1}{2}\phi(\eta)
                     - \tfrac{1}{2}\eta\phi'(\eta)\Bigr).
\]

Next, we compute $u_x$ and $u_{xx}$. We have
\[
u_x = t^{-1/2}\,\phi'(\eta)\,\eta_x.
\]
Since $\eta = x t^{-1/2}$, we have $\eta_x = t^{-1/2}$, so
\[
u_x = t^{-1/2}\,\phi'(\eta)\,t^{-1/2}
    = t^{-1}\,\phi'(\eta).
\]
Differentiating again,
\[
u_{xx} = \frac{\partial}{\partial x}\Bigl(t^{-1}\,\phi'(\eta)\Bigr)
       = t^{-1}\,\phi''(\eta)\,\eta_x
       = t^{-1}\,\phi''(\eta)\,t^{-1/2}
       = t^{-3/2}\,\phi''(\eta).
\]

Substituting into the PDE $u_t = k u_{xx}$ gives
\[
t^{-3/2}\Bigl(-\tfrac{1}{2}\phi - \tfrac{1}{2}\eta\phi'\Bigr)
= k\,t^{-3/2}\,\phi''.
\]
Cancelling the factor $t^{-3/2}$, we obtain the ordinary differential equation
\[
-\frac{1}{2}\phi(\eta) - \frac{1}{2}\eta\phi'(\eta)
= k\,\phi''(\eta).
\]
Multiplying by $2$ and rearranging, this can be written as
\[
2k\,\phi''(\eta) + \eta\,\phi'(\eta) + \phi(\eta) = 0.
\]

\medskip

\noindent\textbf{3. Solving the ODE: emergence of the Gaussian.}

We now solve
\[
2k\,\phi''(\eta) + \eta\,\phi'(\eta) + \phi(\eta) = 0.
\]
We are looking for a profile that is even in $\eta$ (because the problem is symmetric under $x\mapsto -x$) and that decays at infinity. A natural ansatz is a Gaussian:
\[
\phi(\eta) = C\,e^{-a\eta^2},
\]
where $C>0$ and $a>0$ are constants to be determined. Differentiating,
\[
\phi'(\eta) = -2a\eta\,\phi(\eta),
\]
\[
\phi''(\eta) = (-2a + 4a^2\eta^2)\,\phi(\eta).
\]
Substitute these into the ODE:
\[
2k(-2a + 4a^2\eta^2)\,\phi(\eta)
+ \eta(-2a\eta\,\phi(\eta))
+ \phi(\eta) = 0.
\]
Factor out $\phi(\eta)$:
\[
\bigl[2k(-2a + 4a^2\eta^2) - 2a\eta^2 + 1\bigr]\phi(\eta) = 0.
\]
Since $\phi(\eta)\neq 0$ for $\eta$ near $0$, the bracket must vanish for all $\eta$:
\[
2k(-2a + 4a^2\eta^2) - 2a\eta^2 + 1 = 0.
\]
Separate constant and $\eta^2$ terms. The constant term is
\[
2k(-2a) + 1 = -4ka + 1,
\]
and the coefficient of $\eta^2$ is
\[
2k(4a^2) - 2a = 8ka^2 - 2a.
\]
Thus we require
\[
-4ka + 1 = 0,
\qquad
8ka^2 - 2a = 0.
\]
From the first equation we obtain $a = 1/(4k)$. Substituting into the second gives
\[
8k\left(\frac{1}{4k}\right)^2 - 2\left(\frac{1}{4k}\right)
= \frac{8k}{16k^2} - \frac{1}{2k}
= \frac{1}{2k} - \frac{1}{2k} = 0,
\]
so the two conditions are consistent. Therefore the Gaussian ansatz indeed solves the ODE, with
\[
\phi(\eta) = C\,\exp\!\left(-\frac{\eta^2}{4k}\right).
\]

Thus the similarity solution has the form
\[
u(x,t) = t^{-1/2}\,C\,\exp\!\left(-\frac{1}{4k}\,\frac{x^2}{t}\right)
       = C\,t^{-1/2}\,\exp\!\left(-\frac{x^2}{4kt}\right).
\]

\medskip

\noindent\textbf{4. Determining the normalization constant.}

We now choose $C$ so that the total heat is $1$. Compute
\[
\int_{-\infty}^{\infty} u(x,t)\,dx
= \int_{-\infty}^{\infty} C\,t^{-1/2}\,
   \exp\!\left(-\frac{x^2}{4kt}\right)\,dx.
\]
Let $y = x/\sqrt{4kt}$, so that $x = y\sqrt{4kt}$ and $dx = \sqrt{4kt}\,dy$. Then
\[
\int_{-\infty}^{\infty} u(x,t)\,dx
= C\,t^{-1/2} \int_{-\infty}^{\infty}
  \exp(-y^2)\,\sqrt{4kt}\,dy
= C\,t^{-1/2}\,\sqrt{4kt} \int_{-\infty}^{\infty} e^{-y^2}\,dy.
\]
We use the standard Gaussian integral
\[
\int_{-\infty}^{\infty} e^{-y^2}\,dy = \sqrt{\pi},
\]
to obtain
\[
\int_{-\infty}^{\infty} u(x,t)\,dx
= C\,t^{-1/2}\,\sqrt{4kt}\,\sqrt{\pi}
= C\,\sqrt{4k\pi}.
\]
This expression is independent of $t$, as expected, and must equal $1$. Therefore
\[
C\,\sqrt{4k\pi} = 1
\quad\Longrightarrow\quad
C = \frac{1}{\sqrt{4\pi k}}.
\]
Thus the self-similar solution is
\[
G(x,t) = \frac{1}{\sqrt{4\pi k t}}\,
\exp\!\left(-\frac{x^2}{4kt}\right).
\]
This is the fundamental solution of the diffusion equation on the real line.

\medskip

\noindent\textbf{5. Verification: PDE, mass conservation, and initial data.}

\emph{(i) The PDE $G_t = k G_{xx}$.}
We verify directly that $G$ satisfies the diffusion equation for $t>0$. Write
\[
G(x,t) = A(t)\,\exp\!\left(-\frac{x^2}{4kt}\right),
\qquad
A(t) = \frac{1}{\sqrt{4\pi k t}}.
\]
First compute $G_t$. We have
\[
A'(t) = -\frac{1}{2}\frac{1}{\sqrt{4\pi k}}\,t^{-3/2}
      = -\frac{1}{2t}\,A(t),
\]
and
\[
\frac{\partial}{\partial t}\left(-\frac{x^2}{4kt}\right)
= \frac{x^2}{4kt^2}.
\]
Hence
\[
G_t = A'(t)\,e^{-x^2/(4kt)}
     + A(t)\,e^{-x^2/(4kt)}\,\frac{x^2}{4kt^2}
   = G\left(-\frac{1}{2t} + \frac{x^2}{4kt^2}\right).
\]

Next compute $G_x$ and $G_{xx}$. Since $A(t)$ does not depend on $x$,
\[
G_x = A(t)\,e^{-x^2/(4kt)}\left(-\frac{2x}{4kt}\right)
    = -\frac{x}{2kt}\,G.
\]
Differentiating once more,
\[
G_{xx}
= -\frac{1}{2kt}\,G -\frac{x}{2kt}\,G_x
= -\frac{1}{2kt}\,G -\frac{x}{2kt}\left(-\frac{x}{2kt}\,G\right)
= -\frac{1}{2kt}\,G + \frac{x^2}{4k^2t^2}\,G.
\]
Therefore
\[
k G_{xx}
= G\left(-\frac{1}{2t} + \frac{x^2}{4kt^2}\right)
= G_t,
\]
as required.

\medskip

\emph{(ii) Conservation of total heat.}
We have already computed
\[
\int_{-\infty}^{\infty} G(x,t)\,dx = 1
\]
for all $t>0$ when choosing $C = 1/\sqrt{4\pi k}$. This is the mathematical expression of conservation of total heat for the fundamental solution.

\medskip

\emph{(iii) Initial data in the sense of distributions.}
Finally, we check that $G(\cdot,t)$ tends to $\delta$ as $t\to 0^+$ in the sense of distributions. Let $\varphi$ be a smooth test function with compact support. We need to show
\[
\lim_{t\to 0^+} \int_{-\infty}^{\infty} G(x,t)\,\varphi(x)\,dx
= \varphi(0).
\]
Perform the same change of variables as before: $y = x/\sqrt{4kt}$, so $x = y\sqrt{4kt}$ and $dx = \sqrt{4kt}\,dy$. Then
\[
\int_{-\infty}^{\infty} G(x,t)\,\varphi(x)\,dx
= \frac{1}{\sqrt{4\pi k t}}
  \int_{-\infty}^{\infty}
  \exp(-y^2)\,\varphi\!\bigl(y\sqrt{4kt}\bigr)\,\sqrt{4kt}\,dy
= \frac{1}{\sqrt{\pi}}
  \int_{-\infty}^{\infty}
  e^{-y^2}\,\varphi\!\bigl(y\sqrt{4kt}\bigr)\,dy.
\]
For each fixed $y$, we have $y\sqrt{4kt}\to 0$ as $t\to 0^+$, so $\varphi(y\sqrt{4kt})\to \varphi(0)$. Moreover, $|\varphi(y\sqrt{4kt})|\le \|\varphi\|_\infty$ and $e^{-y^2}$ is integrable on $\mathbb{R}$. By the dominated convergence theorem,
\[
\lim_{t\to 0^+} \int_{-\infty}^{\infty} G(x,t)\,\varphi(x)\,dx
= \frac{1}{\sqrt{\pi}}
  \int_{-\infty}^{\infty} e^{-y^2}\,\varphi(0)\,dy
= \varphi(0)
  \frac{1}{\sqrt{\pi}}\int_{-\infty}^{\infty} e^{-y^2}\,dy
= \varphi(0).
\]
Thus $G(\cdot,t)\to \delta$ as $t\to 0^+$ in the sense of distributions.

\medskip

\noindent\textbf{6. Interpretation and connection to diffusion.}

The function
\[
G(x,t) = \frac{1}{\sqrt{4\pi k t}}\,
\exp\!\left(-\frac{x^2}{4kt}\right)
\]
is the fundamental solution of the diffusion equation on $\mathbb{R}$. It is Gaussian in space, with variance proportional to $t$: as time increases, the Gaussian spreads out (its standard deviation grows like $\sqrt{t}$) and its peak height decreases like $t^{-1/2}$, while the total area under the curve remains equal to $1$. This behavior is characteristic of diffusion: localized disturbances spread and smooth out over time, with a characteristic length scale $\sqrt{kt}$.

From the viewpoint of the theory of partial differential equations, this example illustrates several central ideas for the diffusion equation: conservation laws (total heat), scaling and self-similarity (the $\sqrt{t}$ spatial scaling), the smoothing effect of parabolic equations, and the role of fundamental solutions. For more general initial data $f$, the solution can be represented as a convolution
\[
u(x,t) = (G(\cdot,t)*f)(x)
= \int_{-\infty}^{\infty} G(x-y,t)\,f(y)\,dy,
\]
showing that the diffusion equation generates a linear smoothing semigroup whose kernel is precisely this Gaussian fundamental solution.
\end{solution}

% ===== Example 3: Random Walks and the Diffusion Equation (inquiry-based) =====
\begin{problem}[Random Walks and the Diffusion Equation]
Many diffusion phenomena in nature, such as heat spreading in a metal rod or dye dispersing in water, can be modeled by the diffusion equation. Surprisingly, this smooth partial differential equation can emerge from a very simple random, discrete model: a particle that takes random steps on a lattice. In this problem you will build, step by step, a bridge from a one-dimensional random walk to the diffusion equation, and see how the continuum description arises as an approximation to the discrete model.

Consider a particle moving on the one-dimensional lattice $\Delta x \,\mathbb{Z} = \{\dotsc,-2\Delta x,-\Delta x,0,\Delta x,2\Delta x,\dotsc\}$. Time is discrete with step size $\Delta t>0$, so $t_n = n\Delta t$ for $n=0,1,2,\dotsc$. At each time step, the particle moves one step to the right or one step to the left, each with probability $\tfrac12$.

Let $P_n(k)$ denote the probability that the particle is at the lattice site $x_k = k\Delta x$ at time $t_n = n\Delta t$.

\smallskip

(a) \textbf{Discrete evolution law (``master equation'').}  
Explain why the probabilities $\{P_n(k)\}$ satisfy a recursive relation of the form
\[
P_{n+1}(k) = \frac12\,P_n(k-1) + \frac12\,P_n(k+1).
\]
Write this equation carefully in words, and then derive it from the definition of the random walk.

Hint: Think about how the particle can arrive at position $x_k$ at time $t_{n+1}$, given where it could have been at time $t_n$.

\smallskip

(b) \textbf{Introducing a continuum description.}  
We now seek a smooth function $u(x,t)$ that approximates the discrete probabilities in the limit of small lattice spacing $\Delta x$ and small time step $\Delta t$. A common correspondence is
\[
u(x_k,t_n) \approx \frac{1}{\Delta x}\,P_n(k),
\]
so that $u(x,t)$ is a probability \emph{density} with respect to $x$.

\begin{enumerate}
\item[(i)] Briefly justify why dividing by $\Delta x$ is natural if we want $u(x,t)$ to represent a probability density in the continuum limit.
\item[(ii)] Rewrite the discrete evolution law from part (a) in terms of $u(x_k,t_n)$, $\Delta x$, and $\Delta t$, so that $P_n(k)$ is eliminated in favor of $u$.
\end{enumerate}

Hint: Start from the relation $P_n(k) \approx u(x_k,t_n)\,\Delta x$ and substitute into the recursion for $P_{n+1}(k)$.

\smallskip

(c) \textbf{Connecting the difference equation to derivatives.}  
We next interpret the discrete evolution law as a finite-difference approximation to a partial differential equation. For this, we will use Taylor expansions of $u(x,t)$ around the point $(x_k,t_n)$.

\begin{enumerate}
\item[(i)] Write the second-order Taylor expansion of $u(x_k\pm\Delta x,t_n)$ in powers of $\Delta x$ around $(x_k,t_n)$, and of $u(x_k,t_n+\Delta t)$ in powers of $\Delta t$ around $(x_k,t_n)$. Keep terms up to order $(\Delta x)^2$ and $\Delta t$, respectively.
\item[(ii)] Use these expansions to express the right-hand side and left-hand side of your equation from part (b)(ii) in terms of $u$ and its derivatives evaluated at $(x_k,t_n)$, plus higher-order error terms.
\end{enumerate}

Hint: You should obtain an expression where $u_t$ (a time derivative) is approximately proportional to $u_{xx}$ (a second spatial derivative). Be explicit about which terms you discard and why it is reasonable to neglect them in the limit of small $\Delta t$ and $\Delta x$.

\smallskip

(d) \textbf{Taking the diffusion scaling limit.}  
The continuum limit requires letting $\Delta x\to 0$ and $\Delta t\to 0$ in a coordinated way. Suppose we impose the \emph{diffusive scaling}
\[
\frac{(\Delta x)^2}{2\Delta t} \to D \quad\text{as}\quad \Delta x,\Delta t\to 0,
\]
where $D>0$ is a fixed constant called the diffusion coefficient.

\begin{enumerate}
\item[(i)] Starting from your expression in part (c)(ii), divide both sides by $\Delta t$ and then pass formally to the limit $\Delta x,\Delta t \to 0$ under the diffusive scaling above. Derive the continuous equation satisfied by $u(x,t)$.
\item[(ii)] State clearly the resulting partial differential equation, and explain in a sentence or two why it is called the \emph{diffusion equation}.
\end{enumerate}

Hint: After dividing by $\Delta t$, the factor $(\Delta x)^2/(2\Delta t)$ should appear multiplying $u_{xx}$. Replace this factor by $D$ in the limit and discard terms that vanish as $\Delta t,\Delta x\to 0$.

\smallskip

(e) \textbf{Extensions and variations.}  
Now explore how changes in the random walk change the limiting equation.

\begin{enumerate}
\item[(i)] Suppose that at each step the particle moves right with probability $p$ and left with probability $q=1-p$, with $p\neq q$. Write the new discrete evolution equation for $P_{n+1}(k)$ in terms of $P_n(k-1)$ and $P_n(k+1)$, and repeat (very briefly) the Taylor expansion reasoning of parts (b)--(d) to guess the form of the continuum limit. What new term appears in the partial differential equation?
\item[(ii)] How would you expect the diffusion coefficient $D$ and any new parameters in the continuum equation to depend on the microscopic step size $\Delta x$, time step $\Delta t$, and probabilities $p$ and $q$? Provide heuristic formulas and a short explanation.
\end{enumerate}

Hint: In the biased case, you should discover a first derivative term $u_x$ in addition to the $u_{xx}$ term. This corresponds to a \emph{drift} or \emph{advection} term in the limiting equation.
\end{problem}

% ===== Example 3: Random Walks and the Diffusion Equation (full solution) =====
\begin{problem}[Random Walks and the Diffusion Equation]
Consider a particle on the one-dimensional lattice $\Delta x\,\mathbb{Z}$ that moves in discrete time steps of length $\Delta t>0$. At each time step it moves one site to the right or left with equal probability $1/2$. Let $P_n(k)$ be the probability that the particle is at $x_k = k\Delta x$ at time $t_n = n\Delta t$, and define a probability density approximation
\[
u(x_k,t_n) \approx \frac{1}{\Delta x}\,P_n(k).
\]
\begin{enumerate}
\item[(a)] Derive the discrete evolution equation (master equation)
\[
P_{n+1}(k) = \frac12\,P_n(k-1) + \frac12\,P_n(k+1),
\]
and rewrite it in terms of $u(x_k,t_n)$, $\Delta x$, and $\Delta t$.
\item[(b)] Use Taylor expansions of $u(x_k\pm\Delta x,t_n)$ and $u(x_k,t_n+\Delta t)$ about $(x_k,t_n)$, up to second order in $\Delta x$ and first order in $\Delta t$, to interpret this discrete equation as a finite-difference approximation to a partial differential equation.
\item[(c)] Under the diffusive scaling
\[
\frac{(\Delta x)^2}{2\Delta t} \to D>0 \quad\text{as}\quad \Delta x,\Delta t\to 0,
\]
pass formally to the limit and derive the diffusion equation satisfied by $u(x,t)$.
\item[(d)] Briefly indicate how the result changes if the walk is biased: at each step the particle moves right with probability $p$ and left with probability $q=1-p\neq\tfrac12$. Identify the additional term that appears in the limiting equation and express its coefficient in terms of $\Delta x$, $\Delta t$, $p$, and $q$.
\end{enumerate}
\end{problem}

\begin{solution}
We begin from the discrete random walk and systematically pass to a continuum description. This example illustrates how the diffusion equation can arise as the large-scale, long-time limit of a microscopic random motion.

\medskip

\emph{(a) Discrete evolution and rewriting in terms of $u$.}  
At time $t_n$, the particle is at some lattice site $x_j = j\Delta x$. In one time step, it moves either to $x_{j+1}$ or $x_{j-1}$, each with probability $1/2$, independently of the past. To be at $x_k$ at time $t_{n+1}$, the particle must have been at $x_{k-1}$ at time $t_n$ and then step right, or at $x_{k+1}$ at time $t_n$ and then step left. Therefore
\[
P_{n+1}(k)
= \frac12\,P_n(k-1) + \frac12\,P_n(k+1),
\]
which is the master equation.

We want to express this in terms of a continuous density $u(x,t)$. By definition
\[
u(x_k,t_n) \approx \frac{1}{\Delta x}\,P_n(k),
\]
or equivalently
\[
P_n(k) \approx u(x_k,t_n)\,\Delta x.
\]
Substituting into the master equation gives
\[
u(x_k,t_{n+1})\,\Delta x
\approx \frac12\,u(x_{k-1},t_n)\,\Delta x
      + \frac12\,u(x_{k+1},t_n)\,\Delta x.
\]
We may cancel the common factor $\Delta x$ to obtain
\[
u(x_k,t_{n+1})
\approx \frac12\,u(x_{k-1},t_n)
      + \frac12\,u(x_{k+1},t_n).
\]
It will be convenient to write this in a form reminiscent of a finite-difference scheme:
\begin{equation} \label{eq:discrete-u}
u(x_k,t_{n+1}) - u(x_k,t_n)
\approx \frac12\bigl(u(x_{k-1},t_n) - 2u(x_k,t_n) + u(x_{k+1},t_n)\bigr).
\end{equation}
We have simply subtracted $u(x_k,t_n)$ from both sides and regrouped terms.

\medskip

\emph{(b) Taylor expansions and interpretation as finite differences.}  
We now express the right- and left-hand sides of \eqref{eq:discrete-u} using Taylor expansions of $u$ around the point $(x_k,t_n)$.

First, expand in space at fixed time $t_n$:
\[
u(x_k \pm \Delta x, t_n)
= u(x_k,t_n)
  \pm \Delta x\,u_x(x_k,t_n)
  + \frac{(\Delta x)^2}{2}\,u_{xx}(x_k,t_n)
  + O\bigl((\Delta x)^3\bigr).
\]
Adding these two expansions, the first derivatives cancel and we obtain
\[
u(x_{k-1},t_n) + u(x_{k+1},t_n)
= 2u(x_k,t_n) + (\Delta x)^2 u_{xx}(x_k,t_n)
  + O\bigl((\Delta x)^4\bigr),
\]
since the odd powers of $\Delta x$ cancel and the first nonzero correction beyond $(\Delta x)^2$ is of order $(\Delta x)^4$.

Therefore
\[
u(x_{k-1},t_n) - 2u(x_k,t_n) + u(x_{k+1},t_n)
= (\Delta x)^2 u_{xx}(x_k,t_n) + O\bigl((\Delta x)^4\bigr).
\]
Substituting this into the right-hand side of \eqref{eq:discrete-u} yields
\[
\text{RHS of \eqref{eq:discrete-u}}
\approx \frac12 (\Delta x)^2 u_{xx}(x_k,t_n) + O\bigl((\Delta x)^4\bigr).
\]

Next, expand in time at fixed $x_k$:
\[
u(x_k,t_{n+1})
= u(x_k,t_n+\Delta t)
= u(x_k,t_n) + \Delta t\,u_t(x_k,t_n)
  + O\bigl((\Delta t)^2\bigr).
\]
Thus the left-hand side of \eqref{eq:discrete-u} becomes
\[
u(x_k,t_{n+1}) - u(x_k,t_n)
= \Delta t\,u_t(x_k,t_n) + O\bigl((\Delta t)^2\bigr).
\]

Substituting both approximations into \eqref{eq:discrete-u}, we obtain
\[
\Delta t\,u_t(x_k,t_n) + O\bigl((\Delta t)^2\bigr)
\approx \frac12 (\Delta x)^2 u_{xx}(x_k,t_n)
        + O\bigl((\Delta x)^4\bigr).
\]
Neglecting the higher-order terms $O((\Delta t)^2)$ and $O((\Delta x)^4)$ in a formal way, this relation suggests
\begin{equation} \label{eq:prelimit}
\Delta t\,u_t(x_k,t_n) \approx \frac{(\Delta x)^2}{2}\,u_{xx}(x_k,t_n).
\end{equation}
This is already in the form of a discrete approximation to a partial differential equation connecting $u_t$ and $u_{xx}$.

\medskip

\emph{(c) Diffusive scaling and continuum limit.}  
We now implement the diffusive scaling
\[
\frac{(\Delta x)^2}{2\Delta t} \to D > 0
\quad\text{as}\quad \Delta x,\Delta t\to 0.
\]
Divide both sides of \eqref{eq:prelimit} by $\Delta t$ to obtain
\[
u_t(x_k,t_n) \approx \frac{(\Delta x)^2}{2\Delta t}\,u_{xx}(x_k,t_n).
\]
Under the assumed scaling, the factor $(\Delta x)^2/(2\Delta t)$ tends to $D$ as $\Delta x,\Delta t\to 0$. If we now regard $x_k$ and $t_n$ as generic space and time points $(x,t)$ in the continuum, and let $\Delta x,\Delta t \to 0$, the above relation formally converges to
\[
u_t(x,t) = D\,u_{xx}(x,t).
\]
This is the one-dimensional diffusion equation (also called the heat equation) with diffusion coefficient $D$.

The name ``diffusion equation'' reflects the fact that this equation describes the temporal evolution of a quantity that spreads out over time, such as heat, mass, or probability density, in a manner consistent with Fick's law: flux is proportional to the gradient, and conservation of the quantity leads to a second derivative in space.

\medskip

\emph{(d) Biased random walk and advection--diffusion.}  
Now suppose that at each step the particle moves right with probability $p$ and left with probability $q = 1-p$, where $p\neq q$. The master equation becomes
\[
P_{n+1}(k) = p\,P_n(k-1) + q\,P_n(k+1).
\]
As before, substituting $P_n(k) \approx u(x_k,t_n)\,\Delta x$ and cancelling $\Delta x$ yields
\begin{equation} \label{eq:biased-discrete}
u(x_k,t_{n+1})
\approx p\,u(x_{k-1},t_n) + q\,u(x_{k+1},t_n).
\end{equation}
Subtract $u(x_k,t_n)$ from both sides:
\[
u(x_k,t_{n+1}) - u(x_k,t_n)
\approx p\bigl[u(x_{k-1},t_n) - u(x_k,t_n)\bigr]
      + q\bigl[u(x_{k+1},t_n) - u(x_k,t_n)\bigr].
\]

We again use Taylor expansions
\[
u(x_k\pm\Delta x,t_n)
= u(x_k,t_n) \pm \Delta x\,u_x(x_k,t_n)
  + \frac{(\Delta x)^2}{2}\,u_{xx}(x_k,t_n)
  + O\bigl((\Delta x)^3\bigr).
\]
Then
\[
u(x_{k-1},t_n) - u(x_k,t_n)
= -\Delta x\,u_x + \frac{(\Delta x)^2}{2}\,u_{xx} + O\bigl((\Delta x)^3\bigr),
\]
\[
u(x_{k+1},t_n) - u(x_k,t_n)
= \Delta x\,u_x + \frac{(\Delta x)^2}{2}\,u_{xx} + O\bigl((\Delta x)^3\bigr),
\]
where all derivatives are evaluated at $(x_k,t_n)$.

Substituting into the right-hand side gives
\begin{align*}
\text{RHS}
&\approx p\left(-\Delta x\,u_x + \frac{(\Delta x)^2}{2}\,u_{xx}\right)
      + q\left(\Delta x\,u_x + \frac{(\Delta x)^2}{2}\,u_{xx}\right)
      + O\bigl((\Delta x)^3\bigr) \\
&= (q-p)\,\Delta x\,u_x
   + \frac{p+q}{2}(\Delta x)^2 u_{xx}
   + O\bigl((\Delta x)^3\bigr).
\end{align*}
Since $p+q=1$, this simplifies to
\[
\text{RHS}
\approx (q-p)\,\Delta x\,u_x
   + \frac{(\Delta x)^2}{2} u_{xx}
   + O\bigl((\Delta x)^3\bigr).
\]

As before, the left-hand side satisfies
\[
u(x_k,t_{n+1}) - u(x_k,t_n)
= \Delta t\,u_t(x_k,t_n) + O\bigl((\Delta t)^2\bigr).
\]
Thus we obtain
\[
\Delta t\,u_t(x_k,t_n)
\approx (q-p)\,\Delta x\,u_x(x_k,t_n)
   + \frac{(\Delta x)^2}{2} u_{xx}(x_k,t_n)
   + \text{higher-order terms}.
\]
Dividing by $\Delta t$ leads to
\[
u_t(x_k,t_n)
\approx \frac{(q-p)\Delta x}{\Delta t}\,u_x(x_k,t_n)
   + \frac{(\Delta x)^2}{2\Delta t}\,u_{xx}(x_k,t_n).
\]

In the continuum limit, we again impose the diffusive scaling
\[
\frac{(\Delta x)^2}{2\Delta t} \to D>0.
\]
In addition, we assume that the ratio
\[
v := \frac{(p-q)\Delta x}{\Delta t}
\]
converges to a finite limit. Notice the sign: since $q-p = -(p-q)$, the coefficient of $u_x$ is $-(p-q)\Delta x/\Delta t$, so the limiting equation becomes
\[
u_t + v\,u_x = D\,u_{xx},
\quad\text{with}\quad v = \lim_{\Delta x,\Delta t\to 0} \frac{(p-q)\Delta x}{\Delta t}.
\]
This is the \emph{advection--diffusion equation}: the term $v u_x$ represents deterministic drift (advection) with velocity $v$ superimposed on diffusive spreading with coefficient $D$.

In summary, for the unbiased walk we obtain the diffusion equation
\[
u_t = D\,u_{xx},
\]
while for a biased walk the continuum limit is
\[
u_t + v\,u_x = D\,u_{xx}.
\]
These results show how a deterministic partial differential equation describing diffusion (and drift) can emerge from an underlying random walk model, which is precisely one of the central themes of the diffusion-equation section: macroscopic smoothing behavior encoded in $u_t = D u_{xx}$ arises from microscopically random motion with suitable scaling.
\end{solution}

% ===== Example 4: Steady States and the Connection to Laplace’s Equation (inquiry-based) =====
\begin{problem}[Steady States and the Connection to Laplace’s Equation]
In this problem, we explore what happens to a diffusing temperature distribution after a very long time, when the boundary temperatures are held fixed. Physically, one expects that eventually nothing changes in time: the system reaches a steady state. Mathematically, this steady state should solve a purely spatial equation. You will discover that this spatial equation is precisely Laplace's equation in one dimension, and see how this connects the time-dependent diffusion equation to time-independent boundary value problems.

Consider a thin, homogeneous rod of length $L$, lying along the $x$-axis from $x=0$ to $x=L$. Let $u(x,t)$ denote the temperature at position $x$ and time $t$. The temperature evolves according to the one-dimensional diffusion (heat) equation
\[
u_t = k\,u_{xx}, \qquad 0 < x < L,\ t>0,
\]
where $k>0$ is the thermal diffusivity. The ends of the rod are kept at fixed (time-independent) temperatures:
\[
u(0,t) = T_0, \qquad u(L,t) = T_L \quad \text{for all } t>0,
\]
where $T_0$ and $T_L$ are given constants. At time $t=0$ the initial temperature profile is some given function $f(x)$:
\[
u(x,0) = f(x), \qquad 0 \le x \le L.
\]

(a) A \emph{steady state} is a temperature distribution $v(x)$ that does not change in time, that is, $u(x,t) \equiv v(x)$ for all $t$. If $u(x,t) = v(x)$ is independent of $t$, what does the diffusion equation $u_t = k u_{xx}$ reduce to? Write down the resulting ordinary differential equation for $v(x)$ and the associated boundary conditions at $x=0$ and $x=L$.  
Hint: If $u(x,t) = v(x)$, then $u_t(x,t) = 0$ for all $x$ and $t$.

(b) Solve the boundary value problem you found in part (a). Show that there is a unique steady-state temperature profile $v(x)$, and express it explicitly in terms of $T_0$, $T_L$, and $L$.  
Hint: The equation $v''(x)=0$ has general solution $v(x) = ax + b$. Use the boundary conditions to determine $a$ and $b$.

(c) Now consider the \emph{difference} between the actual temperature and the steady state:
\[
w(x,t) := u(x,t) - v(x).
\]
Write down the partial differential equation and boundary conditions satisfied by $w(x,t)$. In particular, show that $w$ satisfies a diffusion equation with \emph{homogeneous} (zero) boundary conditions. What is the initial condition for $w(x,t)$ in terms of $f$ and $v$?  
Hint: Compute $w_t$ and $w_{xx}$ in terms of $u$ and $v$, and use the equations that $u$ and $v$ satisfy.

(d) Suppose (as in the usual separation-of-variables treatment of the heat equation) that $w(x,t)$ can be represented as a sine series
\[
w(x,t) = \sum_{n=1}^{\infty} b_n e^{-k\lambda_n t} \sin\!\left(\frac{n\pi x}{L}\right),
\]
for suitable constants $b_n$ and positive numbers $\lambda_n$.  

(i) Using the diffusion equation for $w$, determine the values of $\lambda_n$.  

(ii) Using this representation, explain why $\displaystyle \lim_{t\to\infty} w(x,t) = 0$ for each fixed $x \in (0,L)$.  

(iii) Conclude that
\[
\lim_{t\to\infty} u(x,t) = v(x),
\]
and interpret this statement physically.  
% Hint for (i): Plug the series into $w_t = k w_{xx}$ and match coefficients of $\sin\left(\frac{n\pi x}{L}\right)$.
% Hint for (ii): Recall that $e^{-k\lambda_n t} \to 0$ as $t\to\infty$ when $k\lambda_n>0$.

(e) Extensions and “what if” questions.

(i) In this problem we considered a one-dimensional rod, so the steady state $v$ satisfies $v''(x) = 0$, which is the one-dimensional form of Laplace’s equation. Suppose instead that we have a thin rectangular plate occupying a region $D \subset \mathbb{R}^2$, and the boundary of the plate is held at fixed temperatures that do not depend on time. Write down the diffusion equation for $u(x,y,t)$, and then write down the \emph{steady-state} equation that $v(x,y)$ should satisfy if $u(x,y,t) \equiv v(x,y)$ is independent of $t$.  

(ii) How do your answers to (a)–(d) suggest a general principle for diffusion problems with time-independent boundary conditions in higher dimensions? State this principle in words.  
Hint: Think about the role played by $v$ and $w = u - v$ in the one-dimensional case, and how the Laplacian $\Delta$ appears in higher dimensions.
\end{problem}

% ===== Example 4: Steady States and the Connection to Laplace’s Equation (full solution) =====
\begin{problem}[Steady States and the Connection to Laplace’s Equation]
Consider the one-dimensional diffusion (heat) equation
\[
u_t = k\,u_{xx}, \qquad 0 < x < L,\ t>0,
\]
with constant Dirichlet boundary conditions
\[
u(0,t) = T_0, \qquad u(L,t) = T_L \quad (t>0),
\]
and initial condition
\[
u(x,0) = f(x), \qquad 0 \le x \le L,
\]
where $k>0$ and $T_0, T_L$ are given constants.

(a) Find the steady-state temperature profile $v(x)$ that is independent of $t$, satisfies the same boundary conditions, and solves the appropriate ordinary differential equation.  

(b) Define $w(x,t) = u(x,t) - v(x)$. Derive the partial differential equation, boundary conditions, and initial condition satisfied by $w$.  

(c) By solving $w$ using separation of variables and a sine series expansion, show that $w(x,t) \to 0$ as $t\to\infty$ for each fixed $x\in(0,L)$. Conclude that
\[
\lim_{t\to\infty} u(x,t) = v(x),
\]
and explain how this illustrates the connection between the diffusion equation and Laplace’s equation.
\end{problem}

\begin{solution}
We are given a one-dimensional diffusion problem on a finite interval with constant boundary temperatures. The main goal is to identify the long-time limit of the temperature profile and to understand how this limit is governed by a time-independent equation, namely Laplace’s equation in one dimension.

\medskip

\noindent\textbf{(a) The steady state and Laplace’s equation.}

A steady-state solution is a temperature profile $v(x)$ that does not depend on time. If $u(x,t) = v(x)$ for all $t$, then $u_t(x,t) = 0$ for all $x$ and $t$. Substituting this into the diffusion equation
\[
u_t = k\,u_{xx}
\]
gives
\[
0 = k\,v''(x),
\]
so $v$ satisfies the ordinary differential equation
\[
v''(x) = 0, \qquad 0 < x < L.
\]
This is the one-dimensional form of Laplace’s equation. The steady state must also satisfy the same boundary conditions as $u$, namely
\[
v(0) = T_0, \qquad v(L) = T_L.
\]

The general solution of $v''(x)=0$ is
\[
v(x) = ax + b,
\]
for some constants $a$ and $b$. Imposing the boundary conditions gives
\[
v(0) = b = T_0,\qquad
v(L) = aL + b = T_L.
\]
From the second equation we obtain
\[
a = \frac{T_L - b}{L} = \frac{T_L - T_0}{L}.
\]
Therefore the unique steady-state solution is
\[
v(x) = T_0 + \frac{T_L - T_0}{L}\,x.
\]
This is a linear temperature profile interpolating between $T_0$ and $T_L$.

\medskip

\noindent\textbf{(b) The equation for the deviation $w = u - v$.}

Define the deviation from steady state by
\[
w(x,t) := u(x,t) - v(x).
\]
We now derive the equation and conditions satisfied by $w$.

First, compute the time and spatial derivatives:
\[
w_t = u_t - 0 = u_t,\qquad
w_{xx} = u_{xx} - v''.
\]
Using the original diffusion equation for $u$ and the steady-state equation for $v$, we have
\[
u_t = k\,u_{xx},\qquad v'' = 0.
\]
Therefore
\[
w_t = u_t = k\,u_{xx} = k\,(w_{xx} + v'') = k\,(w_{xx} + 0) = k\,w_{xx}.
\]
So $w$ satisfies the same diffusion equation:
\[
w_t = k\,w_{xx}, \qquad 0<x<L,\ t>0.
\]

Next, we determine the boundary conditions. At $x=0$,
\[
w(0,t) = u(0,t) - v(0) = T_0 - T_0 = 0.
\]
Similarly, at $x=L$,
\[
w(L,t) = u(L,t) - v(L) = T_L - T_L = 0.
\]
So $w$ satisfies homogeneous Dirichlet boundary conditions:
\[
w(0,t) = 0,\qquad w(L,t) = 0 \quad \text{for } t>0.
\]

Finally, the initial condition for $w$ is obtained from
\[
w(x,0) = u(x,0) - v(x) = f(x) - v(x).
\]
Putting this together, $w$ solves
\[
\begin{cases}
w_t = k\,w_{xx}, & 0 < x < L,\ t>0,\\[4pt]
w(0,t) = 0,\ w(L,t) = 0, & t>0,\\[4pt]
w(x,0) = f(x) - v(x), & 0 \le x \le L.
\end{cases}
\]

\medskip

\noindent\textbf{(c) Separation of variables, decay, and convergence to the steady state.}

The problem for $w$ is the standard heat equation on $(0,L)$ with homogeneous Dirichlet (zero) boundary conditions. The central idea for solving such problems is separation of variables together with expansion in eigenfunctions of the spatial operator. The relevant eigenfunctions here are sines, which are orthogonal on $(0,L)$ and satisfy the homogeneous Dirichlet boundary conditions.

We look for separated solutions of the form $X(x)T(t)$ for $w$, leading to
\[
X(x)T'(t) = k\,X''(x)T(t).
\]
Dividing by $k\,X(x)T(t)$ (assuming neither factor is identically zero), we obtain
\[
\frac{T'(t)}{k\,T(t)} = \frac{X''(x)}{X(x)} = -\lambda,
\]
where $-\lambda$ is a separation constant. This gives a spatial eigenvalue problem
\[
X''(x) + \lambda X(x) = 0,\qquad X(0) = 0,\ X(L)=0.
\]
The nontrivial solutions of this boundary value problem occur when $\lambda = \lambda_n := \left(\frac{n\pi}{L}\right)^2$ for integers $n\ge1$, with corresponding eigenfunctions
\[
X_n(x) = \sin\!\left(\frac{n\pi x}{L}\right).
\]
For each $n$, the temporal factor $T_n(t)$ satisfies
\[
T_n'(t) = -k\lambda_n T_n(t),
\]
so
\[
T_n(t) = e^{-k\lambda_n t} C_n = C_n e^{-k\left(\frac{n\pi}{L}\right)^2 t},
\]
for some constant $C_n$.

Superposing these separated solutions, we may represent $w$ as
\[
w(x,t) = \sum_{n=1}^{\infty} b_n e^{-k\left(\frac{n\pi}{L}\right)^2 t}\,\sin\!\left(\frac{n\pi x}{L}\right),
\]
where the coefficients $b_n$ are determined by the initial condition
\[
w(x,0) = f(x) - v(x).
\]
Specifically, we expand $f(x) - v(x)$ in a sine series:
\[
f(x) - v(x) = \sum_{n=1}^{\infty} b_n \sin\!\left(\frac{n\pi x}{L}\right),
\]
with
\[
b_n = \frac{2}{L} \int_0^L \bigl(f(\xi) - v(\xi)\bigr) \sin\!\left(\frac{n\pi \xi}{L}\right)\,d\xi.
\]

Now we analyze the long-time behavior. For each fixed $n\ge1$ and fixed $x\in(0,L)$,
\[
e^{-k\left(\frac{n\pi}{L}\right)^2 t} \to 0 \quad \text{as } t\to\infty,
\]
because $k>0$ and $\left(\frac{n\pi}{L}\right)^2>0$. Therefore, term by term, we have
\[
b_n e^{-k\left(\frac{n\pi}{L}\right)^2 t}\,\sin\!\left(\frac{n\pi x}{L}\right) \to 0 \quad \text{as } t\to\infty.
\]
Under mild regularity hypotheses on $f$ (so that the sine series converges appropriately), we can pass the limit through the sum and conclude that, for each fixed $x\in(0,L)$,
\[
\lim_{t\to\infty} w(x,t) = 0.
\]

Since $u = v + w$, this implies
\[
\lim_{t\to\infty} u(x,t) = \lim_{t\to\infty} \bigl(v(x) + w(x,t)\bigr) = v(x) + 0 = v(x)
\]
for each fixed $x\in(0,L)$.

\medskip

\noindent\textbf{Interpretation and connection to Laplace’s equation.}

We have shown that the solution of the time-dependent diffusion problem converges, as $t\to\infty$, to the unique steady-state profile $v(x)$ that satisfies
\[
v''(x) = 0,\quad 0<x<L,\qquad v(0) = T_0,\ v(L)=T_L.
\]
This is exactly Laplace’s equation in one dimension with Dirichlet boundary conditions. Thus, for diffusion processes with time-independent Dirichlet boundary conditions, the long-time limit is governed by a \emph{static} boundary value problem for the Laplacian.

In higher dimensions, the same reasoning applies with the Laplacian $\Delta$ in place of the second derivative $d^2/dx^2$. If $u(\mathbf{x},t)$ satisfies the diffusion equation
\[
u_t = k\,\Delta u \quad \text{in a bounded region } D\subset\mathbb{R}^n,
\]
with time-independent boundary data, then any steady state $v(\mathbf{x})$ must satisfy
\[
\Delta v = 0 \quad \text{in } D
\]
with the same boundary values. This means the steady state is a harmonic function. Subtracting $v$ from $u$ again yields a function $w$ that satisfies the diffusion equation with homogeneous boundary conditions and decays to zero in time. 

This example illustrates a central idea in the study of diffusion equations: the time evolution smooths out initial data and drives the solution toward a harmonic equilibrium profile determined solely by the boundary conditions. The diffusion equation thus connects dynamical behavior (in time) to the static boundary value problems for Laplace’s equation.
\end{solution}

% ===== Example 5: Diffusion with Reaction or Decay (inquiry-based) =====
\begin{problem}[Diffusion with Reaction or Decay]
A chemical species diffuses along a thin one-dimensional rod of length $L$. In addition to diffusion, each particle of the species has a constant probability per unit time of disappearing, for example by radioactive decay or a first-order chemical reaction. We model the concentration $u(x,t)$ (mass per unit length) of the species at position $x\in(0,L)$ and time $t>0$. The rod ends are kept at zero concentration, representing perfect sinks. The governing equation is taken to be
\[
u_t = D\,u_{xx} - k\,u,\qquad 0<x<L,\ t>0,
\]
where $D>0$ is the diffusion coefficient and $k>0$ is the decay (reaction) rate, with boundary and initial conditions
\[
u(0,t)=u(L,t)=0,\qquad t>0,\qquad
u(x,0)=f(x),\qquad 0<x<L.
\]

\smallskip

(a) Explain in words why it is reasonable to add a term of the form $-k\,u$ to the diffusion equation to model decay or reaction. In particular:
\begin{itemize}
    \item Why should the loss rate be proportional to $u$?
    \item Why is the sign negative?
    \item What are the physical units of $k$?
\end{itemize}
How would you check, using the PDE, that the total mass $\displaystyle M(t)=\int_0^L u(x,t)\,dx$ is nonincreasing in time?
% Hint: Differentiate $M(t)$ with respect to $t$ and use the PDE and the boundary conditions.

\smallskip

(b) The reaction term $-k\,u$ makes the equation slightly more complicated than the pure diffusion equation. An important idea is to try a change of unknown that ``absorbs'' this term. 

Look for a new function $v(x,t)$ of the form
\[
v(x,t)=e^{\alpha t}\,u(x,t)
\]
for some constant $\alpha$ to be determined. Compute $v_t$ and $v_{xx}$ in terms of $u$, $u_t$, and $u_{xx}$, and use the PDE for $u$ to derive an equation for $v$. For which value of $\alpha$ does $v$ satisfy the \emph{pure} diffusion equation
\[
v_t = D\,v_{xx}\,?
\]
State also the boundary and initial conditions satisfied by $v$.
% Hint: First write $v_t = \alpha e^{\alpha t} u + e^{\alpha t} u_t$ and substitute $u_t$ from the original PDE.

\smallskip

(c) Now forget about $u$ for a moment and focus on the simpler problem you found for $v$ in part (b):
\[
v_t = D\,v_{xx},\quad 0<x<L,\ t>0,\qquad
v(0,t)=v(L,t)=0,\qquad
v(x,0)=f(x).
\]
Solve this initial–boundary value problem by separation of variables.

\begin{enumerate}
    \item[(i)] Assume $v(x,t)=X(x)T(t)$ and separate variables to obtain an eigenvalue problem for $X(x)$. What ordinary differential equation and boundary conditions does $X$ satisfy?
    \item[(ii)] Show that the nontrivial solutions occur for eigenvalues
    \[
    \lambda_n = \left(\frac{n\pi}{L}\right)^2,\qquad n=1,2,3,\dots,
    \]
    with corresponding eigenfunctions $X_n(x)=\sin\!\left(\frac{n\pi x}{L}\right)$.
    \item[(iii)] For each $n$, solve the corresponding time equation for $T_n(t)$.
    \item[(iv)] Write down the general solution for $v(x,t)$ as a series and express the coefficients in terms of the initial condition $f(x)$.
\end{enumerate}
Hint: You may recall from the heat equation that
\[
f(x) \sim \sum_{n=1}^\infty b_n \sin\!\left(\frac{n\pi x}{L}\right),\qquad
b_n = \frac{2}{L}\int_0^L f(x)\sin\!\left(\frac{n\pi x}{L}\right)\,dx.
\]

\smallskip

(d) Return now to the original concentration $u(x,t)$. Using your relation between $u$ and $v$ from part (b), express the solution $u(x,t)$ in terms of $f(x)$ and the parameters $D$, $k$, and $L$. 

Then analyze the long-time behavior:
\begin{itemize}
    \item What is $\displaystyle \lim_{t\to\infty} u(x,t)$?
    \item How does the presence of the reaction rate $k>0$ modify the rate at which the solution decays in time, compared with the pure diffusion case $k=0$?
\end{itemize}
% Hint: Compare the exponents in the time-dependent factors.

\smallskip

(e) ``What if'' questions and extensions.
\begin{enumerate}
    \item[(i)] Suppose instead that $k<0$, so that the reaction term is $-k\,u$ with $k<0$, corresponding to linear \emph{growth} rather than decay. How would your formula for $u(x,t)$ change, and what would you expect for the long-time behavior?
    \item[(ii)] Suppose the equation had an additional constant source term,
    \[
    u_t = D\,u_{xx} - k\,u + S,\qquad S>0\ \text{constant}.
    \]
    Outline (without full details) how you might adapt your method to handle this case. In particular, what kind of steady state solution would you look for, and how could you reduce the problem to one you already know how to solve?
    % Hint: Try to find a time-independent particular solution, then subtract it to obtain a homogeneous problem.
\end{enumerate}

\end{problem}

% ===== Example 5: Diffusion with Reaction or Decay (full solution) =====
\begin{problem}[Diffusion with Reaction or Decay]
Consider the one-dimensional reaction–diffusion equation on $0<x<L$,
\[
u_t = D\,u_{xx} - k\,u,\qquad D>0,\ k>0,
\]
with boundary conditions $u(0,t)=u(L,t)=0$ for $t>0$ and initial condition $u(x,0)=f(x)$ for $0<x<L$, where $f$ is sufficiently regular.

\begin{enumerate}
    \item[(i)] By introducing a transformed unknown $v(x,t)=e^{k t}u(x,t)$, show that $v$ satisfies the pure diffusion equation with homogeneous Dirichlet boundary conditions and initial data $v(x,0)=f(x)$.
    \item[(ii)] Solve the resulting initial–boundary value problem for $v$ by separation of variables, and hence obtain an explicit series representation for $u(x,t)$.
    \item[(iii)] Describe briefly how the reaction term $-k\,u$ affects the temporal decay rates and the long-time behavior, compared with the case $k=0$.
\end{enumerate}
\end{problem}

\begin{solution}
We are asked to solve a linear reaction–diffusion equation on a finite interval with homogeneous Dirichlet boundary conditions. The central ideas are to simplify the equation via an integrating-factor-type substitution, then to apply separation of variables and eigenfunction expansions, exactly as for the standard diffusion (heat) equation.

\medskip

\noindent\textbf{(i) Transforming to the pure diffusion equation.}
Define
\[
v(x,t)=e^{k t}u(x,t).
\]
We compute its derivatives. First,
\[
v_t(x,t) = k e^{kt} u(x,t) + e^{kt} u_t(x,t),
\]
and, since the exponential factor depends only on $t$,
\[
v_x(x,t) = e^{kt} u_x(x,t),\qquad v_{xx}(x,t) = e^{kt} u_{xx}(x,t).
\]
We substitute the given PDE for $u$,
\[
u_t = D u_{xx} - k u,
\]
into the expression for $v_t$:
\[
v_t = k e^{kt} u + e^{kt} (D u_{xx} - k u)
    = k e^{kt} u + D e^{kt} u_{xx} - k e^{kt} u
    = D e^{kt} u_{xx}.
\]
Using $v_{xx} = e^{kt} u_{xx}$, we obtain the simpler PDE
\[
v_t = D v_{xx},\qquad 0<x<L,\ t>0,
\]
which is the standard diffusion (heat) equation.

The boundary conditions for $v$ follow from those of $u$:
\[
v(0,t) = e^{kt} u(0,t) = 0,\qquad
v(L,t) = e^{kt} u(L,t) = 0,\qquad t>0.
\]
At $t=0$, we have
\[
v(x,0) = e^{k\cdot 0} u(x,0) = u(x,0) = f(x),\qquad 0<x<L.
\]
Thus $v$ solves the classical heat equation on $(0,L)$ with homogeneous Dirichlet boundary conditions and initial data $f$.

\medskip

\noindent\textbf{(ii) Separation of variables and series solution.}
We now solve
\[
\begin{cases}
v_t = D v_{xx}, & 0<x<L,\ t>0,\\[0.2em]
v(0,t)=v(L,t)=0, & t>0,\\[0.2em]
v(x,0)=f(x), & 0<x<L.
\end{cases}
\]
We use separation of variables. Seek solutions of the form $v(x,t) = X(x)T(t)$ with $X$ not identically zero and $T$ not identically zero. Substituting into the PDE gives
\[
X(x) T'(t) = D X''(x) T(t).
\]
For $X(x)T(t)\neq 0$, we divide both sides by $D X(x) T(t)$ to obtain
\[
\frac{T'(t)}{D T(t)} = \frac{X''(x)}{X(x)} = -\lambda
\]
for some separation constant $\lambda$ (independent of $x$ and $t$). We thus obtain the system
\[
\begin{cases}
X''(x) + \lambda X(x) = 0,\\
T'(t) + D \lambda T(t) = 0.
\end{cases}
\]

The boundary conditions for $v$ translate to
\[
X(0) T(t) = 0,\quad X(L) T(t) = 0\quad \text{for all } t>0.
\]
For nontrivial solutions with $T(t)\neq 0$, this requires
\[
X(0)=X(L)=0.
\]
Therefore $X$ satisfies the Sturm–Liouville problem
\[
X'' + \lambda X = 0,\quad X(0)=X(L)=0.
\]

It is standard (and easy to check) that nontrivial solutions arise precisely for
\[
\lambda_n = \left(\frac{n\pi}{L}\right)^2,\qquad n=1,2,3,\dots,
\]
with corresponding eigenfunctions
\[
X_n(x) = \sin\!\left(\frac{n\pi x}{L}\right).
\]
For each such $\lambda_n$, the time-dependent factor satisfies
\[
T_n'(t) + D \lambda_n T_n(t) = 0,
\]
whose solution is
\[
T_n(t) = A_n e^{-D \lambda_n t}
      = A_n \exp\!\left(-D\left(\frac{n\pi}{L}\right)^2 t\right),
\]
for some constant $A_n$.

By linearity and superposition, the general solution $v$ satisfying the boundary conditions may be expanded as
\[
v(x,t) = \sum_{n=1}^\infty b_n e^{-D (n\pi/L)^2 t}
          \sin\!\left(\frac{n\pi x}{L}\right),
\]
where the coefficients $\{b_n\}$ are determined by the initial condition $v(x,0)=f(x)$:
\[
f(x) = v(x,0) = \sum_{n=1}^\infty b_n \sin\!\left(\frac{n\pi x}{L}\right).
\]
This is precisely the sine series expansion of $f$ on $(0,L)$, so the coefficients are
\[
b_n = \frac{2}{L} \int_0^L f(x)\,\sin\!\left(\frac{n\pi x}{L}\right)\,dx,
\qquad n=1,2,3,\dots.
\]
Therefore
\[
v(x,t) = \sum_{n=1}^\infty 
\left[\frac{2}{L} \int_0^L f(\xi)\,\sin\!\left(\frac{n\pi \xi}{L}\right)\,d\xi\right]
\exp\!\left(-D\left(\frac{n\pi}{L}\right)^2 t\right)
\sin\!\left(\frac{n\pi x}{L}\right).
\]

Returning to $u$, we recall $u(x,t)=e^{-kt} v(x,t)$, so
\[
u(x,t) = e^{-kt} \sum_{n=1}^\infty b_n e^{-D (n\pi/L)^2 t}
          \sin\!\left(\frac{n\pi x}{L}\right)
       = \sum_{n=1}^\infty b_n
          \exp\!\left(-\bigl[D(n\pi/L)^2 + k\bigr] t\right)
          \sin\!\left(\frac{n\pi x}{L}\right),
\]
with
\[
b_n = \frac{2}{L} \int_0^L f(\xi)\,\sin\!\left(\frac{n\pi \xi}{L}\right)\,d\xi.
\]
This is the desired explicit series solution for $u$.

\medskip

\noindent\textbf{(iii) Effect of the reaction term on decay rates and long-time behavior.}
Each spatial mode $\sin\!\left(\frac{n\pi x}{L}\right)$ decays in time with a factor
\[
\exp\!\left(-\bigl[D(n\pi/L)^2 + k\bigr] t\right).
\]
For the pure diffusion equation ($k=0$), the decay rate of the $n$-th mode is $D (n\pi/L)^2$. The reaction term $-k u$ adds the constant $k$ to every decay rate, so each mode decays faster in time by a factor $\mathrm{e}^{-k t}$ relative to the $k=0$ case.

In particular, for any nontrivial initial data $f$ with at least one nonzero Fourier coefficient $b_n$, we have
\[
\lim_{t\to\infty} u(x,t) = 0
\]
for each fixed $x\in(0,L)$, since all the exponential factors tend to zero. The reaction term does not change the limiting state (which is already zero for the homogeneous Dirichlet heat equation), but it accelerates the approach to zero by imposing an additional uniform exponential decay.

\medskip

\noindent\textbf{Connection to the diffusion equation theme.}
This example illustrates two central ideas of the diffusion equation chapter:
\begin{itemize}
    \item A linear reaction term $-k u$ can often be removed by a simple time-dependent rescaling $v = e^{k t} u$, reducing the problem to the pure diffusion equation.
    \item Once reduced, the standard separation-of-variables method applies, leading to an eigenvalue problem for the spatial operator (here, the Laplacian with Dirichlet boundary conditions), and the full solution is expressed as a series of eigenfunctions with exponentially decaying time factors.
\end{itemize}
The reaction term therefore appears as a uniform shift in the eigenvalues governing temporal decay, leaving the spatial eigenfunctions unchanged.
\end{solution}

\section{Boundary Value Problems: Fourier Method}
% --- Narrative plan (auto-generated) ---
% In this section we study boundary value problems for partial differential equations using the Fourier method, which is built on separation of variables and Fourier series. The central idea is to convert a partial differential equation with spatial boundary conditions into a family of ordinary differential equations in time (or another variable) whose solutions can be written as infinite sums of spatial eigenfunctions. These eigenfunctions, typically sines and cosines, arise from solving a related Sturm–Liouville problem and form an orthogonal basis that allows us to expand the initial or boundary data.
%
% This approach is fundamental in applied mathematics because many physical models, such as heat flow, wave propagation, and diffusion in confined regions, naturally lead to boundary value problems. The Fourier method provides explicit formulas for solutions, reveals how different spatial modes evolve, and connects directly to notions of stability and long-time behavior in dynamical systems. Mathematically, it illustrates how tools from ordinary differential equations, linear algebra, and real and complex Fourier analysis come together to solve partial differential equations and prepares the ground for more advanced topics such as transform methods, spectral theory, and numerical approximation of PDEs.

% ===== Example 1: Heat Equation on a Finite Rod with Fixed End Temperatures (inquiry-based) =====
\begin{problem}[Heat Equation on a Finite Rod with Fixed End Temperatures]
Consider a thin, homogeneous rod of length $L$, lying along the $x$-axis from $x=0$ to $x=L$. The sides of the rod are perfectly insulated, so heat can flow only along the length of the rod. The two ends are kept in contact with ideal heat baths which maintain them at fixed temperatures (in this first model, at temperature zero). We are given an initial temperature distribution along the rod and we wish to determine the temperature at later times. In this problem you will discover how the Fourier method and separation of variables lead naturally to a solution in terms of a Fourier sine series.

(a) \textbf{Modeling the problem.}  
Assume that the temperature $u(x,t)$ in the rod satisfies the one-dimensional heat equation with constant thermal diffusivity $\kappa>0$. Write down the partial differential equation (PDE) for $u(x,t)$, the boundary conditions at $x=0$ and $x=L$ if both ends are held at temperature zero, and the initial condition if the initial temperature profile is a given function $f(x)$.

\medskip

(b) \textbf{Trying separation of variables.}  
We now look for solutions of the PDE that also satisfy the homogeneous boundary conditions, but not necessarily the initial condition yet. Suppose that
\[
u(x,t) = X(x)\,T(t)
\]
for some functions $X$ and $T$.

\begin{enumerate}
  \item[(i)] Substitute $u(x,t) = X(x)T(t)$ into the heat equation and simplify to obtain an equation where all dependence on $x$ and $t$ is separated.
  \item[(ii)] Explain why the separated equation must be equal to a constant, say $-\lambda$, and write down the resulting ordinary differential equations (ODEs) for $X$ and $T$.
  \item[(iii)] Translate the boundary conditions on $u$ into boundary conditions on $X$.
\end{enumerate}

Hint: After substitution and division by $\kappa X T$, you should obtain that a function of $x$ alone equals a function of $t$ alone. Conclude that both are constant.

\medskip

(c) \textbf{Solving the spatial eigenvalue problem.}  
You now have a boundary value problem for $X$ of the form
\[
X''(x) + \lambda X(x) = 0, \qquad X(0)=0, \quad X(L)=0,
\]
for some real parameter $\lambda$.

\begin{enumerate}
  \item[(i)] Consider separately the cases $\lambda<0$, $\lambda=0$, and $\lambda>0$. For each case, solve the ODE for $X$ and impose the boundary conditions to determine whether nontrivial solutions exist.
  \item[(ii)] For the case that admits nontrivial solutions, find the corresponding values of $\lambda$ and describe the corresponding eigenfunctions $X_n(x)$, $n=1,2,\dots$.
\end{enumerate}

Hint: You should find that $\lambda$ cannot be nonpositive if we want nontrivial solutions satisfying both boundary conditions. For $\lambda>0$, it is convenient to write $\lambda = \mu^2$ with $\mu>0$ and look for trigonometric solutions.

\medskip

(d) \textbf{Assembling the separated solutions and matching the initial condition.}  
For each admissible $\lambda_n$ you found, there is a corresponding time-dependent factor $T_n(t)$ solving the ODE from part (b). 

\begin{enumerate}
  \item[(i)] Solve the ODE for $T_n(t)$ and write down the product solution
  \[
  u_n(x,t) = X_n(x)\,T_n(t).
  \]
  \item[(ii)] Argue why a general solution $u(x,t)$ satisfying the PDE and homogeneous boundary conditions can be written as an (infinite) linear combination of these separated solutions:
  \[
  u(x,t) = \sum_{n=1}^\infty b_n\,X_n(x)\,T_n(t).
  \]
  \item[(iii)] Use the initial condition $u(x,0) = f(x)$ to derive a formula for the coefficients $b_n$ in terms of $f$. Express $b_n$ as an integral involving $f$ and $X_n$.
\end{enumerate}

Hint: At $t=0$ you obtain a Fourier series expansion of $f(x)$ in terms of the eigenfunctions $X_n(x)$. Use orthogonality of these eigenfunctions on the interval $[0,L]$.

\medskip

(e) \textbf{Extensions and variations.}  

\begin{enumerate}
  \item[(i)] Suppose now that the ends are held at the nonzero constant temperature $T_0$ instead of zero, that is, $u(0,t)=u(L,t)=T_0$. Outline how you would modify the above procedure to solve this problem. In particular, indicate how to reduce the boundary conditions to a homogeneous form.
  
  Hint: Look for a steady-state (time-independent) solution $v(x)$ that satisfies the boundary conditions, and then consider $w(x,t)=u(x,t)-v(x)$.
  
  \item[(ii)] In the original problem with zero end temperatures, what happens to $u(x,t)$ as $t\to\infty$? Give a qualitative explanation based on your series solution.
\end{enumerate}

\end{problem}

% ===== Example 1: Heat Equation on a Finite Rod with Fixed End Temperatures (full solution) =====
\begin{problem}[Heat Equation on a Finite Rod with Fixed End Temperatures]
Let $u(x,t)$ denote the temperature in a thin homogeneous rod of length $L$, for $0<x<L$ and $t>0$. The rod satisfies the one-dimensional heat equation
\[
u_t = \kappa u_{xx}, \qquad 0<x<L,\ t>0,
\]
with fixed end temperatures
\[
u(0,t)=0,\qquad u(L,t)=0,\qquad t>0,
\]
and initial temperature profile
\[
u(x,0)=f(x),\qquad 0<x<L,
\]
where $f$ is a given sufficiently nice function. Use the Fourier method (separation of variables and eigenfunction expansions) to find an explicit series representation of the solution $u(x,t)$, including formulas for the coefficients in terms of $f$. Describe also the qualitative behavior of $u(x,t)$ as $t\to\infty$.
\end{problem}

\begin{solution}
We seek to solve the initial–boundary value problem for the one-dimensional heat equation on the finite interval $(0,L)$ with homogeneous Dirichlet boundary conditions. This is a standard setting in which the Fourier method applies: we use separation of variables to obtain an eigenvalue problem in $x$, solve that boundary value problem, and then expand the initial data in the resulting eigenfunctions.

\medskip

\textbf{1. Separation of variables and the eigenvalue problem.}  
We first look for solutions of the PDE that satisfy the homogeneous boundary conditions, but not yet the initial condition, in the separated form
\[
u(x,t) = X(x)\,T(t).
\]
Substituting this into the PDE $u_t = \kappa u_{xx}$ gives
\[
X(x)\,T'(t) = \kappa X''(x)\,T(t).
\]
Assuming $X$ and $T$ are not identically zero, we divide by $\kappa X(x)T(t)$ to obtain
\[
\frac{T'(t)}{\kappa T(t)} = \frac{X''(x)}{X(x)}.
\]
The left-hand side depends only on $t$, while the right-hand side depends only on $x$. Therefore both sides must be equal to a constant, which we denote by $-\lambda$:
\[
\frac{T'(t)}{\kappa T(t)} = \frac{X''(x)}{X(x)} = -\lambda.
\]
This yields the system of ordinary differential equations
\[
\begin{cases}
X''(x) + \lambda X(x) = 0,\\[4pt]
T'(t) + \kappa \lambda T(t) = 0.
\end{cases}
\]
The boundary conditions on $u$ translate into conditions on $X$:
\[
u(0,t)=0 \ \Rightarrow\ X(0)T(t)=0 \ \Rightarrow\ X(0)=0,\qquad
u(L,t)=0 \ \Rightarrow\ X(L)T(t)=0 \ \Rightarrow\ X(L)=0,
\]
since we are not interested in the trivial solution $T\equiv 0$. Thus $X$ must solve the boundary value problem
\[
X''(x) + \lambda X(x) = 0,\qquad X(0)=0,\quad X(L)=0.
\]
This is a Sturm–Liouville eigenvalue problem, and its solutions will furnish the spatial modes for our Fourier expansion.

\medskip

\textbf{2. Solving the spatial boundary value problem.}  
We now determine the eigenvalues $\lambda$ for which there exist nontrivial solutions $X$ satisfying
\[
X'' + \lambda X = 0,\quad X(0)=0,\ X(L)=0.
\]
We consider three cases.

\medskip

\emph{Case 1: $\lambda=0$.} Then the ODE reduces to $X''(x)=0$, whose general solution is
\[
X(x) = A + Bx.
\]
The condition $X(0)=0$ implies $A=0$, so $X(x)=Bx$. The condition $X(L)=0$ then gives $B L = 0$, hence $B=0$. Thus $X\equiv 0$ is the only solution; there is no nontrivial eigenfunction when $\lambda=0$.

\medskip

\emph{Case 2: $\lambda<0$.} Write $\lambda=-\mu^2$ with $\mu>0$. The ODE becomes
\[
X''(x) - \mu^2 X(x)=0,
\]
whose general solution is
\[
X(x) = A e^{\mu x} + B e^{-\mu x}.
\]
Imposing $X(0)=0$ yields $A + B = 0$, so $B=-A$ and
\[
X(x) = A\left(e^{\mu x} - e^{-\mu x}\right) = 2A\sinh(\mu x).
\]
Then $X(L)=0$ implies $\sinh(\mu L)=0$, but $\sinh(y)=0$ only when $y=0$, so we must have $\mu L=0$, that is $\mu=0$, which contradicts $\mu>0$. Hence again only the trivial solution exists. There are no eigenvalues with $\lambda<0$.

\medskip

\emph{Case 3: $\lambda>0$.} Write $\lambda=\mu^2$ with $\mu>0$. The ODE becomes
\[
X''(x)+\mu^2 X(x)=0,
\]
whose general solution is
\[
X(x) = A\cos(\mu x) + B\sin(\mu x).
\]
The condition $X(0)=0$ gives $A\cos(0) + B\sin(0) = A=0$, so $X(x)=B\sin(\mu x)$. Then $X(L)=0$ implies $B\sin(\mu L)=0$. For a nontrivial solution we require $B\neq 0$, so we must have
\[
\sin(\mu L)=0 \quad\Longrightarrow\quad \mu L = n\pi,\quad n=1,2,3,\dots
\]
Thus
\[
\mu_n = \frac{n\pi}{L}, \qquad \lambda_n = \mu_n^2 = \left(\frac{n\pi}{L}\right)^2,
\]
and the corresponding eigenfunctions (up to scaling) are
\[
X_n(x) = \sin\left(\frac{n\pi x}{L}\right),\qquad n=1,2,3,\dots
\]
These eigenfunctions satisfy the boundary conditions $X_n(0)=X_n(L)=0$.

Therefore, the spatial part of the Fourier method provides an infinite sequence of eigenvalues $\lambda_n = (n\pi/L)^2$ and corresponding eigenfunctions $X_n(x)$ as above.

\medskip

\textbf{3. Time-dependent factors and separated solutions.}  
For each eigenvalue $\lambda_n$, the time-dependent function $T_n(t)$ solves
\[
T_n'(t) + \kappa \lambda_n T_n(t)=0.
\]
This is a first-order linear ODE whose general solution is
\[
T_n(t) = C_n e^{-\kappa \lambda_n t} = C_n \exp\!\left(-\kappa \left(\frac{n\pi}{L}\right)^2 t\right),
\]
where $C_n$ is an arbitrary constant. Absorbing $C_n$ into the overall multiplicative constant, one separated solution corresponding to mode $n$ is
\[
u_n(x,t) = \sin\!\left(\frac{n\pi x}{L}\right) e^{-\kappa \left(\frac{n\pi}{L}\right)^2 t}.
\]
Each $u_n$ satisfies the heat equation, the homogeneous boundary conditions, and is nontrivial.

The heat equation is linear and homogeneous, so any finite linear combination of the $u_n$ is also a solution. Under suitable assumptions on $f$ (for example, $f$ square-integrable or piecewise smooth), we may consider infinite series as well. Thus we look for the general solution in the form
\[
u(x,t) = \sum_{n=1}^\infty b_n \sin\!\left(\frac{n\pi x}{L}\right) e^{-\kappa \left(\frac{n\pi}{L}\right)^2 t},
\]
where the coefficients $b_n$ are to be determined from the initial condition.

\medskip

\textbf{4. Imposing the initial condition and Fourier sine coefficients.}  
At time $t=0$ the solution must satisfy
\[
u(x,0) = f(x).
\]
On the other hand, from the series representation we obtain
\[
u(x,0) = \sum_{n=1}^\infty b_n \sin\!\left(\frac{n\pi x}{L}\right) e^{0}
= \sum_{n=1}^\infty b_n \sin\!\left(\frac{n\pi x}{L}\right).
\]
Thus we require that $f$ admit an expansion in a Fourier sine series on $(0,L)$:
\[
f(x) \sim \sum_{n=1}^\infty b_n \sin\!\left(\frac{n\pi x}{L}\right).
\]
The sine functions $\{\sin(n\pi x/L)\}_{n=1}^\infty$ form an orthogonal set in $L^2(0,L)$ with respect to the standard inner product
\[
\langle g,h\rangle = \int_0^L g(x)h(x)\,dx.
\]
In fact, for $m,n\ge 1$,
\[
\int_0^L \sin\!\left(\frac{m\pi x}{L}\right)\sin\!\left(\frac{n\pi x}{L}\right)\,dx
= 
\begin{cases}
0, & m\neq n,\\[4pt]
\displaystyle \frac{L}{2}, & m=n.
\end{cases}
\]
To derive the coefficients $b_n$, multiply the series for $f$ by $\sin\left(\frac{m\pi x}{L}\right)$ and integrate from $0$ to $L$:
\[
\int_0^L f(x)\sin\!\left(\frac{m\pi x}{L}\right)\,dx
= \int_0^L \left(\sum_{n=1}^\infty b_n \sin\!\left(\frac{n\pi x}{L}\right)\right)\sin\!\left(\frac{m\pi x}{L}\right)\,dx.
\]
Under mild hypotheses on $f$ we may interchange sum and integral:
\[
\int_0^L f(x)\sin\!\left(\frac{m\pi x}{L}\right)\,dx
= \sum_{n=1}^\infty b_n \int_0^L \sin\!\left(\frac{n\pi x}{L}\right)\sin\!\left(\frac{m\pi x}{L}\right)\,dx
= b_m \frac{L}{2}.
\]
Therefore,
\[
b_m = \frac{2}{L}\int_0^L f(x)\sin\!\left(\frac{m\pi x}{L}\right)\,dx,\qquad m=1,2,3,\dots.
\]
Renaming the index, we have the explicit formula
\[
b_n = \frac{2}{L}\int_0^L f(x)\sin\!\left(\frac{n\pi x}{L}\right)\,dx,\qquad n\ge 1.
\]

\medskip

\textbf{5. Final series solution and long-time behavior.}  
Substituting these coefficients into the separated series, we arrive at the Fourier–series representation of the solution:
\[
u(x,t)
= \sum_{n=1}^\infty \left[ \frac{2}{L}\int_0^L f(\xi)\sin\!\left(\frac{n\pi \xi}{L}\right)\,d\xi \right]
\sin\!\left(\frac{n\pi x}{L}\right)
\exp\!\left(-\kappa \left(\frac{n\pi}{L}\right)^2 t\right).
\]
Here $\xi$ is a dummy integration variable. This function $u$ satisfies the heat equation, the homogeneous Dirichlet boundary conditions, and the initial condition in the sense of convergence of the Fourier series to $f$ (for example, pointwise at continuity points of $f$ and in $L^2$ if $f\in L^2(0,L)$).

To understand the behavior as $t\to\infty$, note that each term in the series has the factor
\[
\exp\!\left(-\kappa \left(\frac{n\pi}{L}\right)^2 t\right),
\]
which tends to zero exponentially fast as $t\to\infty$ for every $n\ge 1$. Therefore, $u(x,t)\to 0$ as $t\to\infty$, at least pointwise and in appropriate norms. Physically, this corresponds to the rod cooling down so that its temperature eventually becomes identically zero, consistent with the fact that the ends are held at zero temperature and the system has no internal heat sources.

\medskip

\textbf{6. Connection to the Fourier method for boundary value problems.}  
This example illustrates the main ideas of the section on boundary value problems and the Fourier method:

\begin{itemize}
  \item First, separation of variables reduces the PDE, together with homogeneous boundary conditions, to an eigenvalue problem for an ordinary differential operator in $x$.
  \item Second, the eigenfunctions of that boundary value problem (here, the sine functions) form an orthogonal basis appropriate to the geometry and boundary conditions of the problem.
  \item Third, the initial condition is enforced by expanding the initial data in this eigenfunction basis, using orthogonality to compute Fourier coefficients.
  \item Finally, the time evolution is encoded in simple exponential factors depending on the eigenvalues, leading to a series solution that makes qualitative features such as decay and smoothing of the temperature field transparent.
\end{itemize}

Thus the heat equation on a finite interval with fixed end temperatures provides a canonical and instructive model for the Fourier method applied to linear PDEs with boundary conditions.

\end{solution}

% ===== Example 2: Heat Equation with Insulated Ends (Neumann Boundary Conditions) (inquiry-based) =====
\begin{problem}[Heat Equation with Insulated Ends (Neumann Boundary Conditions)]
Consider a thin, homogeneous rod of length $L$ lying along the $x$-axis from $x=0$ to $x=L$. We again assume that heat flows only along the length of the rod and that the temperature is governed by the one-dimensional heat equation. This time, however, both ends of the rod are perfectly insulated, so no heat can flow in or out through $x=0$ or $x=L$. Physically, this means that whatever heat is initially in the rod will remain there forever; only its distribution can change. We will see that this physical picture is naturally reflected in the mathematics through Neumann boundary conditions and a special ``constant mode'' in the solution.

Throughout, let $u(x,t)$ denote the temperature at position $x$ and time $t$, and let $k>0$ denote the thermal diffusivity constant.

\medskip

(a) \textbf{Modeling the insulation.}  
Starting from the physical statement that “no heat flows through the ends of the rod,” explain why the appropriate boundary conditions at $x=0$ and $x=L$ are
\[
u_x(0,t) = 0, \qquad u_x(L,t) = 0, \qquad t>0.
\]
You may recall that the heat flux is proportional to $-u_x$. Why does the vanishing of $u_x$ correspond to perfect insulation?

\medskip

(b) \textbf{Setting up separation of variables.}  
Assume that $u$ satisfies the heat equation
\[
u_t = k\,u_{xx}, \qquad 0<x<L,\ t>0,
\]
with the Neumann boundary conditions from part (a) and an initial condition
\[
u(x,0) = f(x), \qquad 0\le x\le L,
\]
where $f$ is a given (reasonable) function.

Assume a separated solution of the form $u(x,t) = X(x)T(t)$ and substitute into the PDE and boundary conditions.

\begin{itemize}
  \item[(i)] Show that $X$ and $T$ must satisfy
  \[
  \frac{T'(t)}{k\,T(t)} = \frac{X''(x)}{X(x)} = -\lambda
  \]
  for some separation constant $\lambda \in \mathbb{R}$, and hence derive the ordinary differential equations
  \[
  X''(x) + \lambda X(x) = 0, \quad 0<x<L, \qquad T'(t) + k\lambda T(t) = 0.
  \]
  \item[(ii)] Translate the Neumann boundary conditions into conditions on $X$.
\end{itemize}

Hint: Focus on $X'(0)$ and $X'(L)$, using that $u_x(x,t) = X'(x)T(t)$.

\medskip

(c) \textbf{The spatial eigenvalue problem for insulated ends.}  
We now must solve the eigenvalue problem
\[
X''(x) + \lambda X(x) = 0,\quad 0<x<L,\qquad X'(0)=0,\quad X'(L)=0.
\]
Analyze this problem by considering three cases for $\lambda$:

\begin{itemize}
  \item[(i)] $\lambda < 0$. Show that any nontrivial solution violates at least one of the Neumann conditions, so there are no negative eigenvalues.
  \item[(ii)] $\lambda = 0$. Solve $X''(x) = 0$ and impose the Neumann conditions. What eigenfunctions arise in this case?
  \item[(iii)] $\lambda > 0$. Write $\lambda = \mu^2$ with $\mu>0$ and solve $X'' + \mu^2 X = 0$. Impose $X'(0)=0$ and $X'(L)=0$ to determine the allowable values of $\mu$ and the corresponding eigenfunctions.
\end{itemize}

Summarize your findings by listing the eigenvalues $\lambda_n$ and eigenfunctions $X_n(x)$, including the case $n=0$.

Hint: For $\lambda>0$, the general solution is $X(x)=A\cos(\mu x)+B\sin(\mu x)$.

\medskip

(d) \textbf{Assembling the general solution and matching the initial data.}  
Using the eigenfunctions from part (c), write the general separated solution as a linear combination
\[
u(x,t) = \sum_{n=0}^\infty c_n\,X_n(x)\,T_n(t),
\]
with each $T_n(t)$ solving the corresponding time ODE.

\begin{itemize}
  \item[(i)] Determine $T_n(t)$ for each eigenvalue $\lambda_n$. Pay special attention to the $n=0$ case.
  \item[(ii)] Using the orthogonality of the eigenfunctions $\{X_n\}$ on $[0,L]$, explain how to choose the constants $c_n$ so that $u(x,0)=f(x)$. Write down explicit formulas for the coefficients in terms of $f$ and integrals over $[0,L]$.
  \item[(iii)] What does the term corresponding to $n=0$ represent physically? Based on your formula for $c_0$, interpret this in terms of the \emph{average} temperature of the rod.
\end{itemize}

Hint: You should find that the spatial eigenfunctions are cosine functions, and that $c_0$ is proportional to the integral of $f$ over $[0,L]$.

\medskip

(e) \textbf{Extensions and qualitative behavior.}

\begin{itemize}
  \item[(i)] Show, directly from the PDE and Neumann boundary conditions (without using the explicit solution), that the total heat in the rod,
  \[
  H(t) = \int_0^L u(x,t)\,dx,
  \]
  is constant in time. How does this fact relate to your identification of the $n=0$ term in part (d)?
  
  Hint: Differentiate $H(t)$ with respect to $t$ and use the PDE and integration by parts.
  
  \item[(ii)] Suppose instead that the left end is held at zero temperature while the right end is insulated:
  \[
  u(0,t)=0,\quad u_x(L,t)=0.
  \]
  How would you expect the set of eigenfunctions to change compared to the purely insulated case? Would you still expect constant temperature modes?
  
  Hint: Think about how many derivative conditions versus value conditions you have, and what combinations of sines and cosines can satisfy them.
\end{itemize}

\end{problem}

% ===== Example 2: Heat Equation with Insulated Ends (Neumann Boundary Conditions) (full solution) =====
\begin{problem}[Heat Equation with Insulated Ends (Neumann Boundary Conditions)]
Consider the heat equation on a rod of length $L$ with insulated ends:
\[
u_t = k\,u_{xx}, \qquad 0<x<L,\ t>0,
\]
\[
u_x(0,t) = 0,\quad u_x(L,t) = 0,\qquad t>0,
\]
\[
u(x,0) = f(x), \qquad 0\le x\le L,
\]
where $k>0$ is constant and $f$ is a given function.

(a) Use separation of variables and the Fourier method to find the solution $u(x,t)$ in terms of a cosine series involving $f$.

(b) Identify the long-time limit $\displaystyle \lim_{t\to\infty} u(x,t)$ (if it exists), and express it in terms of $f$.

(c) Briefly explain how this example illustrates the role of Neumann boundary conditions and orthogonal eigenfunctions in the Fourier method for boundary value problems.
\end{problem}

\begin{solution}
We solve the initial–boundary value problem by the method of separation of variables and Fourier expansions adapted to Neumann boundary conditions.

\medskip

\emph{Step 1: Separation of variables and the eigenvalue problem.}  
We seek solutions of the form $u(x,t)=X(x)T(t)$. Substituting into the heat equation gives
\[
X(x)T'(t) = k X''(x)T(t).
\]
Assuming $X$ and $T$ are nonzero, we divide by $kX(x)T(t)$ to obtain
\[
\frac{T'(t)}{kT(t)} = \frac{X''(x)}{X(x)} = -\lambda,
\]
for some separation constant $\lambda\in\mathbb{R}$. Hence,
\[
X''(x)+\lambda X(x)=0,\qquad 0<x<L,
\]
\[
T'(t)+k\lambda T(t)=0,\qquad t>0.
\]
The boundary conditions become
\[
u_x(0,t)=X'(0)T(t)=0,\quad u_x(L,t)=X'(L)T(t)=0.
\]
Since we are interested in nontrivial solutions with $T(t)\neq 0$, these reduce to
\[
X'(0)=0,\qquad X'(L)=0.
\]
Thus, we must solve the eigenvalue problem
\[
X''(x)+\lambda X(x)=0,\quad 0<x<L,\qquad X'(0)=0,\ X'(L)=0.
\]

\medskip

\emph{Step 2: Solving the Neumann eigenvalue problem.}  
We analyze $X''+\lambda X=0$ with $X'(0)=X'(L)=0$.

\smallskip

\emph{Case 1: $\lambda<0$.}  
Let $\lambda=-\mu^2$ with $\mu>0$. Then
\[
X''(x)-\mu^2 X(x)=0,
\]
whose general solution is $X(x)=A e^{\mu x}+B e^{-\mu x}$. Differentiating gives
\[
X'(x)=A\mu e^{\mu x}-B\mu e^{-\mu x}.
\]
The condition $X'(0)=0$ implies $A\mu-B\mu=0$, so $A=B$. Then
\[
X'(L)=A\mu e^{\mu L}-A\mu e^{-\mu L}=A\mu\bigl(e^{\mu L}-e^{-\mu L}\bigr).
\]
For this to vanish, we must have either $A=0$ or $e^{\mu L}=e^{-\mu L}$, which is impossible for $\mu>0$. Hence $A=0$, and then $X\equiv 0$. Therefore, there is no nontrivial solution for $\lambda<0$.

\smallskip

\emph{Case 2: $\lambda=0$.}  
We solve $X''(x)=0$. The general solution is
\[
X(x)=A+Bx.
\]
Differentiating, $X'(x)=B$. The Neumann conditions give
\[
X'(0)=B=0,\quad X'(L)=B=0.
\]
Thus $B=0$ and $X(x)=A$ is constant. Any nonzero constant is an eigenfunction corresponding to $\lambda_0=0$. We usually take $X_0(x)\equiv 1$ as a normalized eigenfunction (up to a constant factor).

\smallskip

\emph{Case 3: $\lambda>0$.}  
Let $\lambda=\mu^2$ with $\mu>0$. Then
\[
X''(x)+\mu^2 X(x)=0,
\]
with general solution
\[
X(x)=A\cos(\mu x)+B\sin(\mu x).
\]
Differentiating,
\[
X'(x)=-A\mu\sin(\mu x)+B\mu\cos(\mu x).
\]
The boundary condition $X'(0)=0$ yields
\[
X'(0)=B\mu=0 \quad\Rightarrow\quad B=0.
\]
Therefore $X(x)=A\cos(\mu x)$ and $X'(x)=-A\mu\sin(\mu x)$. The second boundary condition $X'(L)=0$ gives
\[
-A\mu\sin(\mu L)=0.
\]
For a nontrivial eigenfunction, $A\neq 0$ and $\mu\neq 0$, so we require
\[
\sin(\mu L)=0 \quad\Rightarrow\quad \mu L=n\pi,\quad n=1,2,3,\dots.
\]
Thus,
\[
\mu_n = \frac{n\pi}{L},\qquad \lambda_n = \mu_n^2 = \left(\frac{n\pi}{L}\right)^2,\quad n=1,2,3,\dots,
\]
with corresponding eigenfunctions
\[
X_n(x)=\cos\left(\frac{n\pi x}{L}\right),\quad n\ge 1.
\]

\smallskip

Collecting all cases, the eigenvalues and eigenfunctions are
\[
\lambda_0=0,\quad X_0(x)\equiv 1;
\]
\[
\lambda_n=\left(\frac{n\pi}{L}\right)^2,\quad X_n(x)=\cos\left(\frac{n\pi x}{L}\right),\quad n=1,2,3,\dots.
\]
The family $\{X_n\}_{n=0}^\infty$ forms an orthogonal set in $L^2(0,L)$ with respect to the standard inner product.

\medskip

\emph{Step 3: Time dependence and general series solution.}  
For each eigenvalue $\lambda_n$, the time equation is
\[
T_n'(t)+k\lambda_n T_n(t)=0,
\]
whose solutions are exponentials:
\[
T_n(t)=C_n e^{-k\lambda_n t}.
\]

For $n\ge 1$, we have $\lambda_n=(n\pi/L)^2>0$, so
\[
T_n(t)=C_n\,e^{-k(n\pi/L)^2 t}.
\]
For $n=0$, we have $\lambda_0=0$, so the time equation is $T_0'(t)=0$, and therefore
\[
T_0(t)=C_0,
\]
a constant in time.

A general solution is an infinite linear combination of separated solutions,
\[
u(x,t) = A_0 X_0(x)T_0(t) + \sum_{n=1}^\infty A_n X_n(x)T_n(t).
\]
Absorbing constants into single coefficients, and using $X_0(x)\equiv 1$, $X_n(x)=\cos(n\pi x/L)$, we write
\[
u(x,t) = a_0 + \sum_{n=1}^\infty a_n e^{-k(n\pi/L)^2 t}\cos\left(\frac{n\pi x}{L}\right),
\]
for some constants $a_0,a_1,a_2,\dots$ to be determined from the initial condition.

\medskip

\emph{Step 4: Matching the initial condition and cosine coefficients.}  
At $t=0$, we must have
\[
u(x,0)=f(x)=a_0 + \sum_{n=1}^\infty a_n \cos\left(\frac{n\pi x}{L}\right).
\]
Thus $f$ is expanded as a cosine series on $[0,L]$. The orthogonality relations for the cosine functions are
\[
\int_0^L \cos\left(\frac{m\pi x}{L}\right)\cos\left(\frac{n\pi x}{L}\right)\,dx
=
\begin{cases}
L, & m=n=0,\\[4pt]
\frac{L}{2}, & m=n\ge 1,\\[4pt]
0, & m\neq n.
\end{cases}
\]

Multiplying the expansion of $f$ by $1$ and integrating from $0$ to $L$ gives the $n=0$ coefficient:
\[
\int_0^L f(x)\,dx
= \int_0^L a_0\,dx + \sum_{n=1}^\infty a_n \int_0^L \cos\left(\frac{n\pi x}{L}\right)\,dx
= a_0 L,
\]
since the integrals of the cosine terms vanish for $n\ge 1$. Thus
\[
a_0 = \frac{1}{L}\int_0^L f(x)\,dx.
\]

For $n\ge 1$, multiply the expansion by $\cos(n\pi x/L)$ and integrate:
\[
\int_0^L f(x)\cos\left(\frac{n\pi x}{L}\right)\,dx
=
a_0 \int_0^L \cos\left(\frac{n\pi x}{L}\right)\,dx
+ \sum_{m=1}^\infty a_m \int_0^L \cos\left(\frac{m\pi x}{L}\right)\cos\left(\frac{n\pi x}{L}\right)\,dx.
\]
The first integral on the right is zero, and the orthogonality relation yields
\[
\int_0^L f(x)\cos\left(\frac{n\pi x}{L}\right)\,dx
= a_n \cdot \frac{L}{2}.
\]
Hence
\[
a_n = \frac{2}{L}\int_0^L f(x)\cos\left(\frac{n\pi x}{L}\right)\,dx,\qquad n=1,2,3,\dots.
\]

Therefore the solution is
\[
u(x,t) = a_0 + \sum_{n=1}^\infty a_n e^{-k(n\pi/L)^2 t}\cos\left(\frac{n\pi x}{L}\right),
\]
where
\[
a_0 = \frac{1}{L}\int_0^L f(x)\,dx,\qquad
a_n = \frac{2}{L}\int_0^L f(x)\cos\left(\frac{n\pi x}{L}\right)\,dx,\quad n\ge 1.
\]

This answers part (a).

\medskip

\emph{Step 5: Long-time behavior and the steady state.}  
To analyze $\displaystyle \lim_{t\to\infty} u(x,t)$, we note that for each $n\ge 1$,
\[
e^{-k(n\pi/L)^2 t}\to 0\quad\text{as }t\to\infty.
\]
Hence all nonconstant modes decay to zero, and we are left with the constant mode:
\[
\lim_{t\to\infty} u(x,t) = a_0 = \frac{1}{L}\int_0^L f(x)\,dx.
\]
Thus the solution approaches a steady state that is spatially constant and equal to the initial average temperature. This answers part (b).

This behavior is consistent with the physical interpretation: the rod is thermally insulated at both ends, so no heat is gained or lost. The system redistributes the initial heat until it reaches a uniform temperature equal to the conserved average.

\medskip

\emph{Step 6: Conceptual remarks (role of Neumann conditions and Fourier method).}  
This example illustrates several key ideas of the Fourier method for boundary value problems:

\begin{itemize}
  \item The boundary conditions determine the appropriate family of spatial eigenfunctions. With Dirichlet (fixed temperature) conditions we obtain sine series, while with Neumann (insulated) conditions we obtain cosine series, including a constant mode.
  \item The spatial part of the PDE leads to a Sturm–Liouville eigenvalue problem. Here the problem $X''+\lambda X=0$ with $X'(0)=X'(L)=0$ yields a discrete set of eigenvalues and an orthogonal basis of eigenfunctions $\{1,\cos(n\pi x/L)\}$.
  \item The time dependence separates into exponentially decaying modes $e^{-k\lambda_n t}$, which encode diffusion. The zero eigenvalue $\lambda_0=0$ corresponds to a nondecaying steady-state mode, reflecting the conservation of total heat under Neumann boundary conditions.
  \item The initial condition is expanded in the eigenfunction basis, and the orthogonality of the eigenfunctions provides explicit formulas for the Fourier coefficients in terms of $f$.
\end{itemize}

Together, these features exemplify how the Fourier method turns a PDE with boundary conditions into an infinite system of decoupled ordinary differential equations, which can be solved explicitly and recombined into a series solution tailored to the geometry and boundary behavior of the problem.
\end{solution}

% ===== Example 3: Vibrating String with Fixed Ends (Wave Equation) (inquiry-based) =====
\begin{problem}[Vibrating String with Fixed Ends (Wave Equation)]
A thin, taut string of length $L$ is stretched between two rigid supports at $x=0$ and $x=L$. Let $u(x,t)$ denote the small transverse displacement of the string from its equilibrium position at point $x$ and time $t$. Under suitable physical assumptions (small vibrations, uniform density and tension), $u$ satisfies the one-dimensional wave equation. In this problem you will rediscover, step by step, how to solve the wave equation with fixed ends using separation of variables and Fourier series.

Assume that $u$ satisfies the wave equation
\[
u_{tt}(x,t) \;=\; c^2\,u_{xx}(x,t), \qquad 0<x<L,\; t>0,
\]
where $c>0$ is the wave speed, together with fixed-end boundary conditions
\[
u(0,t)=0,\qquad u(L,t)=0,\qquad t>0,
\]
and initial conditions
\[
u(x,0)=f(x),\qquad u_t(x,0)=g(x),\qquad 0<x<L,
\]
for some given functions $f$ and $g$.

\smallskip

(a) \textbf{Understanding the model.}  
Explain in words what each of the conditions above means physically:
\begin{itemize}
  \item the partial differential equation $u_{tt}=c^2 u_{xx}$,
  \item the boundary conditions $u(0,t)=u(L,t)=0$,
  \item the initial conditions $u(x,0)=f(x)$ and $u_t(x,0)=g(x)$.
\end{itemize}
What are some typical shapes for $f$ and $g$ that might arise from plucking or striking the string?

\medskip

(b) \textbf{Trying separation of variables.}  
We look for special solutions of the form
\[
u(x,t)=X(x)\,T(t),
\]
where $X$ depends only on $x$ and $T$ depends only on $t$.

\begin{enumerate}
  \item Substitute $u(x,t)=X(x)T(t)$ into the wave equation and show that
  \[
  \frac{T''(t)}{c^2\,T(t)} \;=\; \frac{X''(x)}{X(x)}.
  \]
  Explain why the left-hand side depends only on $t$ and the right-hand side only on $x$, and conclude that both sides must be equal to a constant, which we denote by $-\lambda$.
  
  \item Show that the boundary conditions imply
  \[
  X(0)=0,\qquad X(L)=0.
  \]
  (Why do the boundary conditions not involve $T$ at all?)
\end{enumerate}

What differential equations do $X$ and $T$ satisfy in terms of the separation constant $\lambda$?

\textit{Hint:} Carefully separate the variables and keep track of signs. Write $X''/X=-\lambda$ and $T''/(c^2 T)=-\lambda$.

\medskip

(c) \textbf{The spatial boundary value problem and eigenvalues.}  
The function $X$ must solve
\[
X''(x)+\lambda\,X(x)=0,\qquad X(0)=0,\quad X(L)=0.
\]
This is an eigenvalue problem. Its solutions depend qualitatively on the sign of $\lambda$.

\begin{enumerate}
  \item Analyze separately the three cases $\lambda<0$, $\lambda=0$, and $\lambda>0$. In each case, solve the ordinary differential equation for $X$ and impose the boundary conditions to determine when a nontrivial solution exists.
  
  \item Show that the only nontrivial solutions occur when
  \[
  \lambda_n = \left(\frac{n\pi}{L}\right)^2,\qquad n=1,2,3,\dots
  \]
  and that, up to a constant factor, the corresponding eigenfunctions are
  \[
  X_n(x) = \sin\!\left(\frac{n\pi x}{L}\right).
  \]
\end{enumerate}

\textit{Hint:} For $\lambda>0$, write $\lambda=\mu^2$ and solve $X''+\mu^2 X=0$. For $\lambda<0$, write $\lambda=-\mu^2$ and solve $X''-\mu^2 X=0$. Check which constants must vanish to satisfy $X(0)=0$ and $X(L)=0$.

\medskip

(d) \textbf{Putting space and time together; using Fourier series.}  

For each $n\ge 1$, you now have a spatial eigenfunction $X_n(x)$ and a temporal function $T_n(t)$ solving
\[
T_n''(t) + c^2 \lambda_n\, T_n(t) = 0, \qquad \lambda_n = \left(\frac{n\pi}{L}\right)^2.
\]

\begin{enumerate}
  \item Solve this ordinary differential equation for $T_n(t)$ and show that
  \[
  T_n(t) = A_n \cos\!\left(\frac{n\pi c t}{L}\right) + B_n \sin\!\left(\frac{n\pi c t}{L}\right),
  \]
  for some constants $A_n$ and $B_n$ depending on $n$.
  
  \item Argue that the general solution of the boundary value problem (with fixed ends) can be written as a superposition of these separated solutions:
  \[
  u(x,t) = \sum_{n=1}^{\infty} \left[ A_n \cos\!\left(\frac{n\pi c t}{L}\right) + B_n \sin\!\left(\frac{n\pi c t}{L}\right) \right]
  \sin\!\left(\frac{n\pi x}{L}\right).
  \]
  
  \item Use the initial displacement condition $u(x,0)=f(x)$ to show that
  \[
  f(x) = \sum_{n=1}^{\infty} A_n \sin\!\left(\frac{n\pi x}{L}\right),
  \]
  and use the initial velocity condition $u_t(x,0)=g(x)$ to show that
  \[
  g(x) = \sum_{n=1}^{\infty} \frac{n\pi c}{L} B_n \sin\!\left(\frac{n\pi x}{L}\right).
  \]
  Express $A_n$ and $B_n$ in terms of $f$ and $g$ using suitable integrals.
\end{enumerate}

\textit{Hint:} You will need the orthogonality of the sine functions
\[
\int_0^L \sin\!\left(\frac{m\pi x}{L}\right)\sin\!\left(\frac{n\pi x}{L}\right)\,dx
=
\begin{cases}
0, & m\ne n,\\[4pt]
\displaystyle \frac{L}{2}, & m=n.
\end{cases}
\]
To derive formulas for $A_n$ and $B_n$, multiply both sides of the series by $\sin\!\left(\frac{k\pi x}{L}\right)$ and integrate from $0$ to $L$.

\medskip

(e) \textbf{What if / extensions.}

\begin{enumerate}
  \item Suppose the initial velocity is zero, $g(x)\equiv 0$, and the initial displacement is $f(x)=h\,x(L-x)$ for some constant $h$. Write down the corresponding series for $u(x,t)$ as explicitly as you can. Where do the coefficients come from?
  
  \item How would the analysis change if the right end of the string were free instead of fixed, so that $u_x(L,t)=0$ instead of $u(L,t)=0$? What sort of spatial eigenfunctions would you expect in that case (sines, cosines, or something else), and why?
  
  \item (Conceptual) The functions $\sin(n\pi x/L)$ are called the normal modes or standing wave shapes of the string. How does the superposition of many such modes explain the way a plucked string vibrates and produces sound?
\end{enumerate}

\end{problem}

% ===== Example 3: Vibrating String with Fixed Ends (Wave Equation) (full solution) =====
\begin{problem}[Vibrating String with Fixed Ends (Wave Equation)]
Consider the initial–boundary value problem for the one-dimensional wave equation
\[
u_{tt}(x,t) = c^2 u_{xx}(x,t),\qquad 0<x<L,\ t>0,
\]
with fixed-end boundary conditions
\[
u(0,t)=0,\qquad u(L,t)=0,\qquad t>0,
\]
and initial conditions
\[
u(x,0)=f(x),\qquad u_t(x,0)=g(x),\qquad 0<x<L,
\]
where $c>0$ is constant and $f,g$ are given functions. Using separation of variables and Fourier series, find a series representation for the solution $u(x,t)$, and express its coefficients explicitly in terms of $f$ and $g$.
\end{problem}

\begin{solution}
We seek to solve the wave equation with homogeneous Dirichlet boundary conditions by the Fourier method. The central ideas are separation of variables, which leads to a spatial eigenvalue problem, and expansion of the initial data in terms of the corresponding eigenfunctions (a sine Fourier series).

\medskip

\textbf{1. Separation of variables and the eigenvalue problem.}

We look for special solutions of the form
\[
u(x,t) = X(x)\,T(t),
\]
where $X$ depends only on $x$ and $T$ on $t$. Substituting this product into the wave equation yields
\[
X(x) T''(t) = c^2 X''(x) T(t).
\]
Assuming $X$ and $T$ are nonzero, we divide both sides by $c^2 X(x) T(t)$ to obtain
\[
\frac{T''(t)}{c^2 T(t)} = \frac{X''(x)}{X(x)}.
\]
The left-hand side depends only on $t$, and the right-hand side only on $x$, so both must be equal to a constant, which we denote by $-\lambda$:
\[
\frac{X''(x)}{X(x)} = \frac{T''(t)}{c^2 T(t)} = -\lambda.
\]

Thus we obtain two ordinary differential equations,
\begin{align*}
X''(x) + \lambda X(x) &= 0,\\
T''(t) + c^2 \lambda T(t) &= 0.
\end{align*}
The boundary conditions $u(0,t)=0$ and $u(L,t)=0$ impose conditions on $X$:
\[
u(0,t)=X(0)T(t)=0,\quad u(L,t)=X(L)T(t)=0\quad\text{for all }t.
\]
We are interested in nontrivial separated solutions for which $T$ is not identically zero, so we must have
\[
X(0)=0,\qquad X(L)=0.
\]
Hence $X$ satisfies the boundary value problem
\[
X''(x) + \lambda X(x) = 0,\qquad X(0)=0,\quad X(L)=0.
\]

This is a standard Sturm–Liouville problem. Its nontrivial solutions exist only for discrete values of $\lambda$, the eigenvalues, and the corresponding functions $X$ are eigenfunctions.

\medskip

\textbf{2. Solving the spatial problem; eigenvalues and eigenfunctions.}

We analyze the spatial problem
\[
X'' + \lambda X = 0, \quad X(0)=0,\ X(L)=0
\]
by considering the sign of $\lambda$.

\emph{Case 1: $\lambda = 0$.}  
The equation becomes $X''=0$, so $X(x)=ax+b$. The boundary condition $X(0)=0$ implies $b=0$, and $X(L)=0$ then implies $aL=0$, hence $a=0$. Thus $X\equiv 0$ is the only solution, and there is no nontrivial eigenfunction for $\lambda=0$.

\emph{Case 2: $\lambda < 0$.}  
Let $\lambda = -\mu^2$ with $\mu>0$. Then the equation is $X'' - \mu^2 X = 0$, whose general solution is
\[
X(x) = A e^{\mu x} + B e^{-\mu x}.
\]
The condition $X(0)=0$ gives $A+B=0$, so $B=-A$ and
\[
X(x) = A\bigl(e^{\mu x}-e^{-\mu x}\bigr) = 2A \sinh(\mu x).
\]
Then $X(L)=0$ gives $\sinh(\mu L)=0$. Since $\sinh(\mu L)\neq 0$ for $\mu>0$, we must have $A=0$, and again $X\equiv 0$ is the only solution. Thus there are no negative eigenvalues.

\emph{Case 3: $\lambda > 0$.}  
Let $\lambda = \mu^2$ with $\mu>0$. Then the equation is $X'' + \mu^2 X = 0$, whose general solution is
\[
X(x) = A\cos(\mu x) + B\sin(\mu x).
\]
The condition $X(0)=0$ implies $A=0$, so
\[
X(x) = B \sin(\mu x).
\]
The second boundary condition $X(L)=0$ demands $\sin(\mu L)=0$, so $\mu L = n\pi$ for some integer $n$. The case $n=0$ yields the trivial solution; for nontrivial solutions we require $n\neq 0$. Since replacing $n$ by $-n$ does not change $\sin(n\pi x/L)$ up to sign, we may restrict to positive integers $n=1,2,3,\dots$.

Thus
\[
\mu_n = \frac{n\pi}{L},\qquad \lambda_n = \mu_n^2 = \left(\frac{n\pi}{L}\right)^2,
\]
and, up to multiplicative constants, the eigenfunctions are
\[
X_n(x) = \sin\!\left(\frac{n\pi x}{L}\right),\qquad n=1,2,3,\dots.
\]

These eigenfunctions are orthogonal in $L^2(0,L)$:
\[
\int_0^L \sin\!\left(\frac{m\pi x}{L}\right)\sin\!\left(\frac{n\pi x}{L}\right)\,dx
=
\begin{cases}
0, & m\neq n,\\[4pt]
\displaystyle \frac{L}{2}, & m=n.
\end{cases}
\]

\medskip

\textbf{3. Temporal factors and separated solutions.}

For each eigenvalue $\lambda_n$, the temporal equation
\[
T_n''(t) + c^2 \lambda_n T_n(t) = 0
\]
becomes
\[
T_n''(t) + c^2 \left(\frac{n\pi}{L}\right)^2 T_n(t) = 0.
\]
This is a harmonic oscillator equation with frequency $\omega_n = \dfrac{n\pi c}{L}$. Its general solution is
\[
T_n(t) = A_n \cos\!\left(\frac{n\pi c t}{L}\right)
        + B_n \sin\!\left(\frac{n\pi c t}{L}\right),
\]
where $A_n$ and $B_n$ are constants to be determined from the initial data.

For each $n$, the product
\[
u_n(x,t) = T_n(t)\,X_n(x)
\]
yields a separated solution that satisfies both the wave equation and the boundary conditions.

\medskip

\textbf{4. Superposition and the general series solution.}

Because the wave equation is linear and homogeneous, any (possibly infinite) linear combination of solutions is again a solution. Therefore, we may form
\[
u(x,t) = \sum_{n=1}^\infty \left[
A_n \cos\!\left(\frac{n\pi c t}{L}\right)
+ B_n \sin\!\left(\frac{n\pi c t}{L}\right)
\right] \sin\!\left(\frac{n\pi x}{L}\right).
\]
Each term in the sum satisfies the boundary conditions, so the whole sum does as well (formally, and under suitable convergence assumptions).

The remaining task is to choose the coefficients $A_n$ and $B_n$ so that the initial conditions are satisfied.

\medskip

\textbf{5. Imposing the initial conditions via Fourier sine series.}

First, impose the initial displacement:
\[
u(x,0) = f(x).
\]
At $t=0$, the sine terms in time vanish and the cosine terms equal $1$, so
\[
u(x,0) = \sum_{n=1}^\infty A_n \sin\!\left(\frac{n\pi x}{L}\right) = f(x).
\]
Thus $f$ is represented by its sine Fourier series on $(0,L)$ with coefficients $A_n$.

Next, impose the initial velocity:
\[
u_t(x,0) = g(x).
\]
We differentiate the series term by term with respect to $t$:
\[
u_t(x,t) = \sum_{n=1}^\infty \left[
-A_n \frac{n\pi c}{L}\sin\!\left(\frac{n\pi c t}{L}\right)
+ B_n  \frac{n\pi c}{L}\cos\!\left(\frac{n\pi c t}{L}\right)
\right] \sin\!\left(\frac{n\pi x}{L}\right).
\]
Evaluating at $t=0$, the sine terms vanish and the cosines become $1$, giving
\[
u_t(x,0) = \sum_{n=1}^\infty \frac{n\pi c}{L} B_n \sin\!\left(\frac{n\pi x}{L}\right)
= g(x).
\]
Hence $g$ also has a sine series representation, with coefficients proportional to $B_n$.

To find $A_n$ and $B_n$ explicitly, we use the orthogonality of the sine functions. For a series
\[
f(x) \sim \sum_{n=1}^\infty a_n \sin\!\left(\frac{n\pi x}{L}\right),
\]
the standard formula for the Fourier sine coefficients on $(0,L)$ is
\[
a_n = \frac{2}{L} \int_0^L f(x)\,\sin\!\left(\frac{n\pi x}{L}\right)\,dx.
\]
Comparing with
\[
f(x) = \sum_{n=1}^\infty A_n \sin\!\left(\frac{n\pi x}{L}\right),
\]
we obtain
\[
A_n = \frac{2}{L} \int_0^L f(x)\,\sin\!\left(\frac{n\pi x}{L}\right)\,dx,\qquad n=1,2,3,\dots.
\]

Similarly, for
\[
g(x) = \sum_{n=1}^\infty \frac{n\pi c}{L} B_n \sin\!\left(\frac{n\pi x}{L}\right),
\]
we may view $\dfrac{n\pi c}{L} B_n$ as the sine coefficient of $g$. Thus
\[
\frac{n\pi c}{L} B_n = \frac{2}{L} \int_0^L g(x)\,\sin\!\left(\frac{n\pi x}{L}\right)\,dx,
\]
which yields
\[
B_n = \frac{2}{n\pi c} \int_0^L g(x)\,\sin\!\left(\frac{n\pi x}{L}\right)\,dx,\qquad n=1,2,3,\dots.
\]

\medskip

\textbf{6. Final series solution and interpretation.}

Collecting all the pieces, the solution of the initial–boundary value problem is given formally by
\[
u(x,t) =
\sum_{n=1}^\infty
\left[
\left(\frac{2}{L} \int_0^L f(\xi)\,\sin\!\left(\frac{n\pi \xi}{L}\right)\,d\xi\right)
\cos\!\left(\frac{n\pi c t}{L}\right)
+
\left(\frac{2}{n\pi c} \int_0^L g(\xi)\,\sin\!\left(\frac{n\pi \xi}{L}\right)\,d\xi\right)
\sin\!\left(\frac{n\pi c t}{L}\right)
\right]
\sin\!\left(\frac{n\pi x}{L}\right).
\]

Each term
\[
\sin\!\left(\frac{n\pi x}{L}\right)\cos\!\left(\frac{n\pi c t}{L}\right)
\quad\text{and}\quad
\sin\!\left(\frac{n\pi x}{L}\right)\sin\!\left(\frac{n\pi c t}{L}\right)
\]
represents a standing wave mode of the string with spatial shape $\sin(n\pi x/L)$ and temporal oscillation at frequency $\omega_n = n\pi c / L$. The coefficients, determined by the Fourier sine series of the initial displacement $f$ and initial velocity $g$, specify how much of each normal mode is excited initially.

\medskip

\textbf{Relation to the Fourier method.}

This example illustrates the main ideas of the Fourier method for boundary value problems: separation of variables reduces the partial differential equation to a spatial eigenvalue problem and a temporal ODE; the boundary conditions lead to a discrete set of eigenfunctions (here, sine functions) that form an orthogonal basis; and the initial data are expanded in this basis using Fourier series. The resulting solution is a superposition of eigenmodes that evolves in time with frequencies determined by the eigenvalues.
\end{solution}

% ===== Example 4: Steady-State Temperature in a Rectangular Plate (Laplace’s Equation) (inquiry-based) =====
\begin{problem}[Steady-State Temperature in a Rectangular Plate (Laplace’s Equation)]
Consider a thin rectangular metal plate occupying the region
\[
0 < x < L, \qquad 0 < y < a,
\]
with negligible thickness in the $z$-direction. The plate is thermally insulated on its faces, so heat flows only in the $x$–$y$ plane, and we wait long enough for the temperature to reach a steady state. Along three edges of the plate we hold the temperature at zero, while along the top edge $y = a$ we prescribe a nonzero temperature profile. In this regime, the temperature $u(x,y)$ no longer depends on time and satisfies Laplace’s equation.

Suppose that
\[
u(x,0) = 0, \quad u(0,y) = 0, \quad u(L,y) = 0, \qquad 0< x < L,\; 0<y<a,
\]
while on the top edge we prescribe
\[
u(x,a) = f(x), \qquad 0<x<L,
\]
for some given function $f$.

\medskip

(a) Write down the partial differential equation and boundary conditions that describe the steady-state temperature $u(x,y)$ in the interior and on the boundary of the plate. Briefly explain (in words) why Laplace's equation is the correct steady-state model here.

\medskip

(b) A common way to approach such boundary value problems on rectangles is to use separation of variables. Assume a separated form
\[
u(x,y) = X(x)\,Y(y).
\]
Insert this ansatz into Laplace’s equation and derive the resulting pair of ordinary differential equations for $X$ and $Y$. Show that you obtain an equation of the form
\[
\frac{X''(x)}{X(x)} = -\frac{Y''(y)}{Y(y)} = -\lambda
\]
for some constant $\lambda$, called the separation constant.

\emph{Hint:} Carefully divide by $X(x)Y(y)$ at a point where both are nonzero, and argue why the resulting expression must be independent of both $x$ and $y$.

\medskip

(c) Now incorporate the boundary conditions at $x = 0$ and $x = L$ into the equation for $X(x)$.

\begin{itemize}
    \item[(i)] Use the fact that $u(0,y)=u(L,y)=0$ for all $0<y<a$ to deduce boundary conditions for $X(x)$. 
    \item[(ii)] Solve the eigenvalue problem
    \[
    X''(x) + \lambda X(x) = 0, \qquad X(0) = 0,\; X(L) = 0,
    \]
    and determine all admissible values of $\lambda$ and the corresponding eigenfunctions $X(x)$.
\end{itemize}

\emph{Hint:} Consider three cases: $\lambda = 0$, $\lambda < 0$, and $\lambda > 0$, and see which of them allow nontrivial (nonzero) solutions that satisfy both boundary conditions.

\medskip

(d) For each admissible $\lambda$, solve the corresponding ordinary differential equation for $Y(y)$:
\[
Y''(y) - \lambda Y(y) = 0.
\]
Use the boundary condition $u(x,0)=0$ to determine which constants in the general solution for $Y$ must vanish.

\begin{itemize}
    \item[(i)] Show that for the eigenvalues $\lambda_n$ you found in part (c), you obtain a family of separated solutions $u_n(x,y)$.
    \item[(ii)] Use the principle of superposition (linearity of Laplace's equation) to argue that a general solution satisfying the three homogeneous boundary conditions $u(0,y)=u(L,y)=u(x,0)=0$ can be written as a series
    \[
    u(x,y) = \sum_{n=1}^\infty A_n \,\phi_n(x)\,\psi_n(y),
    \]
    for suitable functions $\phi_n$ and $\psi_n$ that you should identify explicitly.
\end{itemize}

\emph{Hint:} Your $\phi_n$ should be the eigenfunctions in $x$, and your $\psi_n$ should be the corresponding solutions in $y$.

\medskip

(e) To incorporate the nonzero boundary condition $u(x,a)=f(x)$, evaluate your series from part (d) at $y=a$:
\[
u(x,a) = \sum_{n=1}^\infty A_n \,\phi_n(x)\,\psi_n(a).
\]
\begin{itemize}
    \item[(i)] Explain how this leads to representing $f(x)$ as a Fourier series in the eigenfunctions $\phi_n(x)$.
    \item[(ii)] Use orthogonality of the $\phi_n$ to derive a formula for the coefficients $A_n$ in terms of $f$.
    \item[(iii)] Write down the final series expression for $u(x,y)$ in terms of $f(x)$ and the eigenfunctions.
\end{itemize}

\emph{Hint:} You should end up with $f(x)$ expressed as a Fourier sine series on $(0,L)$, whose coefficients determine the $A_n$.

\medskip

(f) \textbf{Extensions and “what if” questions.}
\begin{itemize}
    \item[(i)] What changes in your solution method if the bottom edge has a nonzero temperature profile, say $u(x,0)=g(x)$ instead of zero? How might you reduce this to a problem with homogeneous (zero) boundary conditions?
    \item[(ii)] Suppose instead that the vertical edges at $x=0$ and $x=L$ are insulated, so that there is no heat flux through them, mathematically $u_x(0,y)=u_x(L,y)=0$. How would this modify the eigenvalue problem for $X(x)$, and what functions would you expect to appear instead of sines?
\end{itemize}

\end{problem}

% ===== Example 4: Steady-State Temperature in a Rectangular Plate (Laplace’s Equation) (full solution) =====
\begin{problem}[Steady-State Temperature in a Rectangular Plate (Laplace’s Equation)]
Let $u(x,y)$ denote the steady-state temperature in a thin rectangular plate
\[
0 < x < L,\qquad 0 < y < a.
\]
Assume $u$ satisfies Laplace’s equation
\[
u_{xx} + u_{yy} = 0
\]
in the interior of the rectangle and the boundary conditions
\[
u(x,0) = 0,\quad u(0,y) = 0,\quad u(L,y) = 0,\qquad 0 < x < L,\; 0 < y < a,
\]
and
\[
u(x,a) = f(x),\qquad 0 < x < L,
\]
for a given function $f$.

Use the Fourier method (separation of variables and Fourier series) to find a series expression for the steady-state solution $u(x,y)$ in terms of $f(x)$.
\end{problem}

\begin{solution}
In the steady state there is no time dependence, so conservation of heat combined with Fourier’s law yields Laplace’s equation
\[
u_{xx} + u_{yy} = 0
\]
for the temperature $u(x,y)$ inside the plate. The boundary conditions fix the temperature along all four sides, with three sides held at zero and one side prescribed by $f(x)$. This is a prototypical Dirichlet boundary value problem on a rectangle, well suited for the Fourier method.

\medskip

\textbf{1. Separation of variables and the eigenvalue problem.}

We seek separated solutions of the form
\[
u(x,y) = X(x)\,Y(y),
\]
where $X$ depends only on $x$ and $Y$ only on $y$. Substituting into Laplace’s equation gives
\[
X''(x)Y(y) + X(x)Y''(y) = 0.
\]
Assuming $X$ and $Y$ are not identically zero, we can divide by $X(x)Y(y)$ at points where both are nonzero:
\[
\frac{X''(x)}{X(x)} + \frac{Y''(y)}{Y(y)} = 0.
\]
The first term depends only on $x$, the second only on $y$, yet their sum is identically zero for all $(x,y)$ in the rectangle. Therefore, each term must be constant, say
\[
\frac{X''(x)}{X(x)} = -\lambda, \qquad \frac{Y''(y)}{Y(y)} = \lambda,
\]
for some constant $\lambda$ called the separation constant. Thus $X$ and $Y$ satisfy the ordinary differential equations
\[
X''(x) + \lambda X(x) = 0, \qquad Y''(y) - \lambda Y(y) = 0.
\]

We next impose the boundary conditions. From $u(0,y)=0$ and $u(L,y)=0$ for $0<y<a$, we obtain
\[
X(0)Y(y) = 0,\qquad X(L)Y(y) = 0 \quad \text{for all } 0<y<a.
\]
We are interested in nontrivial separated solutions, so $Y$ is not identically zero; therefore we must have
\[
X(0) = 0,\qquad X(L) = 0.
\]
The condition $u(x,0)=0$ yields
\[
X(x)Y(0) = 0 \quad \text{for } 0<x<L,
\]
and since $X$ is not identically zero, we deduce
\[
Y(0) = 0.
\]

Thus we must solve the eigenvalue problem
\[
X''(x) + \lambda X(x) = 0,\qquad X(0)=0,\; X(L)=0,
\]
and then, for each admissible $\lambda$, solve
\[
Y''(y) - \lambda Y(y) = 0,\qquad Y(0)=0.
\]

\medskip

\textbf{2. Solving the $X$-equation and determining eigenvalues.}

We consider the three standard cases for $\lambda$.

\emph{Case 1: $\lambda = 0$.} Then
\[
X''(x) = 0 \;\Rightarrow\; X(x) = A + Bx.
\]
Applying the boundary conditions $X(0)=0$ and $X(L)=0$ yields $A=0$ and $B L = 0$, so $B=0$ as well. Therefore $X$ is identically zero, and this case gives only the trivial solution, which we discard.

\emph{Case 2: $\lambda < 0$.} Write $\lambda = -\mu^2$ with $\mu>0$, then
\[
X''(x) - \mu^2 X(x) = 0,
\]
whose general solution is
\[
X(x) = A e^{\mu x} + B e^{-\mu x}.
\]
Imposing $X(0)=0$ gives $A + B = 0$, so $B = -A$ and
\[
X(x) = A(e^{\mu x} - e^{-\mu x}) = 2A \sinh(\mu x).
\]
Then $X(L)=0$ implies $2A \sinh(\mu L) = 0$. Since $\sinh(\mu L) \neq 0$ for $\mu>0$, we must have $A=0$. Thus again only the trivial solution arises, so $\lambda<0$ is not admissible.

\emph{Case 3: $\lambda > 0$.} Write $\lambda = \left(\frac{n\pi}{L}\right)^2$ for some $n>0$ (this form is anticipated; any positive $\lambda$ can be written as $\kappa^2$, and the boundary conditions will quantize $\kappa$). Then the equation
\[
X''(x) + \lambda X(x) = 0
\]
has general solution
\[
X(x) = A \cos\!\Bigl(\tfrac{n\pi}{L} x\Bigr) + B \sin\!\Bigl(\tfrac{n\pi}{L} x\Bigr).
\]
Applying $X(0) = 0$ gives
\[
X(0) = A = 0,
\]
so
\[
X(x) = B \sin\!\Bigl(\tfrac{n\pi}{L} x\Bigr).
\]
Then $X(L)=0$ becomes
\[
X(L) = B \sin(n\pi) = 0.
\]
This holds for any $B$ provided $\sin(n\pi) = 0$, that is, for every positive integer $n \in \mathbb{N}$. Hence the admissible eigenvalues and eigenfunctions are
\[
\lambda_n = \left(\frac{n\pi}{L}\right)^2, \qquad X_n(x) = \sin\!\Bigl(\tfrac{n\pi}{L} x\Bigr), \quad n=1,2,3,\dots,
\]
where we absorb any multiplicative constant into later coefficients.

Thus the spatial dependence in the $x$-direction is a Fourier sine basis, reflecting the homogeneous Dirichlet conditions at $x=0$ and $x=L$.

\medskip

\textbf{3. Solving the $Y$-equation for each eigenvalue.}

For each $n$, we now solve
\[
Y_n''(y) - \lambda_n Y_n(y) = 0, \qquad Y_n(0) = 0,
\]
with $\lambda_n = (n\pi/L)^2$. The general solution of this linear ODE is
\[
Y_n(y) = C_n e^{\sqrt{\lambda_n}\,y} + D_n e^{-\sqrt{\lambda_n}\,y}
       = C_n e^{\frac{n\pi}{L} y} + D_n e^{-\frac{n\pi}{L} y}.
\]
The condition $Y_n(0) = 0$ gives
\[
Y_n(0) = C_n + D_n = 0 \;\Rightarrow\; D_n = -C_n.
\]
Hence
\[
Y_n(y) = C_n\left(e^{\frac{n\pi}{L} y} - e^{-\frac{n\pi}{L} y}\right)
       = 2C_n \sinh\!\Bigl(\tfrac{n\pi}{L} y\Bigr).
\]
We can incorporate the constant $2C_n$ into the overall coefficient later, so we may take
\[
Y_n(y) = \sinh\!\Bigl(\tfrac{n\pi}{L} y\Bigr).
\]

Thus a separated solution for each $n$ is
\[
u_n(x,y) = X_n(x)Y_n(y) = \sin\!\Bigl(\tfrac{n\pi}{L} x\Bigr)\,\sinh\!\Bigl(\tfrac{n\pi}{L} y\Bigr).
\]

\medskip

\textbf{4. Superposition and satisfying the homogeneous boundary conditions.}

Laplace’s equation is linear and homogeneous, so any finite linear combination, and hence any (sufficiently convergent) infinite sum of solutions, is again a solution. Therefore, the general solution that satisfies the three homogeneous boundary conditions
\[
u(0,y) = u(L,y) = u(x,0) = 0
\]
is given by the series
\[
u(x,y) = \sum_{n=1}^{\infty} A_n\,\sin\!\Bigl(\tfrac{n\pi}{L} x\Bigr)\,\sinh\!\Bigl(\tfrac{n\pi}{L} y\Bigr),
\]
for some coefficients $A_n$ yet to be determined.

Indeed, this form ensures:
\begin{itemize}
    \item $u(0,y)=0$ and $u(L,y)=0$ because $\sin(\tfrac{n\pi}{L} x)$ vanishes at $x=0$ and $x=L$;
    \item $u(x,0)=0$ because $\sinh(\tfrac{n\pi}{L} y)$ vanishes at $y=0$.
\end{itemize}

\medskip

\textbf{5. Imposing the nonhomogeneous boundary condition at $y=a$.}

We now use the remaining boundary condition on the top edge:
\[
u(x,a) = f(x),\qquad 0<x<L.
\]
Evaluating the series at $y=a$ gives
\[
u(x,a) = \sum_{n=1}^{\infty} A_n\,\sin\!\Bigl(\tfrac{n\pi}{L} x\Bigr)\,\sinh\!\Bigl(\tfrac{n\pi}{L} a\Bigr).
\]
Thus
\[
f(x) = \sum_{n=1}^{\infty} \Bigl(A_n \sinh\!\Bigl(\tfrac{n\pi}{L} a\Bigr)\Bigr)\, \sin\!\Bigl(\tfrac{n\pi}{L} x\Bigr).
\]
This expresses $f(x)$ as a Fourier sine series on the interval $(0,L)$, with sine basis functions
\[
\sin\!\Bigl(\tfrac{n\pi}{L} x\Bigr),\quad n=1,2,\dots.
\]

The standard Fourier sine expansion of a function $f$ on $(0,L)$ (assuming $f$ is sufficiently regular) is
\[
f(x) = \sum_{n=1}^\infty b_n \sin\!\Bigl(\tfrac{n\pi}{L} x\Bigr),
\]
where the coefficients are given by the orthogonality of sines:
\[
b_n = \frac{2}{L}\int_0^L f(x)\,\sin\!\Bigl(\tfrac{n\pi}{L} x\Bigr)\,dx.
\]
Comparing with our expression, we identify
\[
b_n = A_n \sinh\!\Bigl(\tfrac{n\pi}{L} a\Bigr).
\]
Therefore
\[
A_n = \frac{b_n}{\sinh\!\bigl(\tfrac{n\pi}{L} a\bigr)}
    = \frac{2}{L\,\sinh\!\bigl(\tfrac{n\pi}{L} a\bigr)} \int_0^L f(x)\,\sin\!\Bigl(\tfrac{n\pi}{L} x\Bigr)\,dx.
\]

\medskip

\textbf{6. Final solution formula.}

Substituting these coefficients back into our series for $u(x,y)$, we obtain the steady-state temperature distribution:
\[
u(x,y)
= \sum_{n=1}^{\infty} 
\left[
\frac{2}{L\,\sinh\!\bigl(\tfrac{n\pi}{L} a\bigr)} 
\int_0^L f(\xi)\,\sin\!\Bigl(\tfrac{n\pi}{L} \xi\Bigr)\,d\xi
\right]
\sin\!\Bigl(\tfrac{n\pi}{L} x\Bigr)\,\sinh\!\Bigl(\tfrac{n\pi}{L} y\Bigr).
\]
It is often convenient to write this as
\[
u(x,y)
= \sum_{n=1}^{\infty} 
\left(\frac{2}{L} \int_0^L f(\xi)\,\sin\!\Bigl(\tfrac{n\pi}{L} \xi\Bigr)\,d\xi\right)
\frac{\sinh\!\bigl(\tfrac{n\pi}{L} y\bigr)}{\sinh\!\bigl(\tfrac{n\pi}{L} a\bigr)}
\,\sin\!\Bigl(\tfrac{n\pi}{L} x\Bigr).
\]

This series solves Laplace’s equation in the rectangle, satisfies $u(x,0)=u(0,y)=u(L,y)=0$, and matches the prescribed boundary data $u(x,a)=f(x)$.

\medskip

\textbf{7. Conceptual remarks and relation to the Fourier method.}

This example illustrates the key ideas of the Fourier method for boundary value problems:
\begin{itemize}
    \item \emph{Separation of variables} reduces a partial differential equation with two independent variables to a pair of ordinary differential equations, linked by a separation constant $\lambda$.
    \item The boundary conditions in one variable (here, in $x$) lead to an \emph{eigenvalue problem} for the corresponding ordinary differential equation. Only discrete values of $\lambda$ (the eigenvalues) admit nontrivial solutions satisfying the boundary conditions, and the corresponding eigenfunctions (here, sine functions) form an orthogonal basis.
    \item The solution in the other variable (here, $y$) is then determined for each eigenvalue, giving a family of separated solutions that can be superposed.
    \item The remaining nonhomogeneous boundary condition is enforced by expanding the prescribed boundary function (here, $f(x)$) into a \emph{Fourier series} in the eigenfunctions. Orthogonality of the eigenfunctions allows us to compute the coefficients explicitly.
\end{itemize}
In this way, the Fourier method turns the PDE boundary value problem into the problem of finding a Fourier series representation of the boundary data, yielding a solution as a convergent series of separated modes.

\end{solution}

% ===== Example 5: Heat Equation with Mixed (Robin) Boundary Conditions (inquiry-based) =====
\begin{problem}[Heat Equation with Mixed (Robin) Boundary Conditions]
A thin, homogeneous rod of length $L$ lies along the $x$-axis for $0<x<L$. Its lateral surface is perfectly insulated, but each end is exposed to air at a fixed ambient temperature $T_{\mathrm{env}}$. According to Newton’s law of cooling, the heat flux at each end is proportional to the temperature difference between the rod and the surrounding air. Let $u(x,t)$ denote the temperature in the rod at position $x$ and time $t$. We prescribe an initial temperature distribution $u(x,0)=f(x)$ and wish to understand how the temperature evolves.

The governing equation is the one-dimensional heat equation
\[
u_t = \kappa\, u_{xx},\qquad 0<x<L,\ t>0,
\]
where $\kappa>0$ is the thermal diffusivity. At the ends we impose Newton cooling:
\begin{align*}
- \kappa\, u_x(0,t) &= h\bigl(u(0,t) - T_{\mathrm{env}}\bigr),\\
\kappa\, u_x(L,t) &= h\bigl(u(L,t) - T_{\mathrm{env}}\bigr),
\end{align*}
where $h>0$ is the heat transfer coefficient. The initial condition is
\[
u(x,0) = f(x),\qquad 0<x<L.
\]

\medskip

(a) The boundary conditions are inhomogeneous because of the presence of $T_{\mathrm{env}}$. A standard trick is to measure temperature relative to the environment. Define
\[
v(x,t) = u(x,t) - T_{\mathrm{env}}.
\]
Write down the partial differential equation, boundary conditions, and initial condition satisfied by $v(x,t)$. Are the new boundary conditions homogeneous (that is, equal to zero)? Why is this useful for the method of separation of variables?

\medskip

(b) From now on, work with $v(x,t)$ and \emph{assume} that $v$ satisfies
\[
v_t = \kappa v_{xx},\qquad 0<x<L,\ t>0,
\]
with homogeneous Robin boundary conditions
\[
v_x(0,t) = -\alpha\, v(0,t),\qquad v_x(L,t) = \alpha\, v(L,t),
\]
where $\alpha = h/\kappa>0$, and initial condition $v(x,0)=g(x)$, where $g(x)=f(x)-T_{\mathrm{env}}$.

Use separation of variables $v(x,t)=X(x)T(t)$ to derive the ordinary differential equations for $X$ and $T$, together with the boundary conditions for $X$. Introduce a separation constant $-\lambda$ in the usual way. What eigenvalue problem does $X$ satisfy?

\emph{Hint:} After dividing by $\kappa X T$, you should find $T'/(\kappa T) = X''/X = -\lambda$, with an appropriate sign convention for $\lambda$ so that decay in time is possible.

\medskip

(c) In this part, analyze the spatial eigenvalue problem
\[
X''(x) + \lambda X(x) = 0,\qquad 0<x<L,
\]
with mixed (Robin) boundary conditions
\[
X'(0) = -\alpha X(0),\qquad X'(L) = \alpha X(L),
\]
where $\alpha>0$ is fixed.

\begin{enumerate}
\item[(i)] Show that $\lambda\le 0$ cannot give any nontrivial eigenfunctions. In particular, explain why $\lambda=0$ and $\lambda<0$ lead only to the trivial solution $X\equiv 0$.
\item[(ii)] Conclude that we must have $\lambda>0$, and write $\lambda=\mu^2$ with $\mu>0$. Show that any solution of $X''+\mu^2 X=0$ has the form
\[
X(x) = A\cos(\mu x) + B\sin(\mu x).
\]
\item[(iii)] Impose the two Robin boundary conditions to obtain a homogeneous linear system for $A$ and $B$. By eliminating $A$ and $B$, derive a transcendental equation for $\mu$ of the form
\[
F(\mu) = 0,
\]
where you should give $F(\mu)$ explicitly in terms of $\mu$, $\alpha$, and $L$.

\emph{Hint:} First use the condition at $x=0$ to express $B$ in terms of $A$; then substitute into the condition at $x=L$.
\item[(iv)] Argue (without fully solving $F(\mu)=0$) that there is an infinite sequence of positive roots
\[
0<\mu_1<\mu_2<\cdots<\mu_n<\cdots,\qquad \mu_n\to\infty,
\]
and that the corresponding eigenfunctions $X_n(x)$ form an orthogonal family in $L^2(0,L)$.
\end{enumerate}

\emph{Hint:} For orthogonality, recall the general Sturm–Liouville theory, or verify directly that eigenfunctions corresponding to distinct eigenvalues are orthogonal with respect to the usual inner product $\int_0^L X_m X_n\,dx$.

\medskip

(d) For each eigenvalue $\lambda_n=\mu_n^2$ from part (c), solve the time equation for $T_n(t)$ and write down the corresponding separated solution $v_n(x,t)$. Then form the general solution as a series
\[
v(x,t) = \sum_{n=1}^\infty c_n\, v_n(x,t).
\]

\begin{enumerate}
\item[(i)] What is the explicit formula for $T_n(t)$?
\item[(ii)] How do you determine the coefficients $c_n$ from the initial condition $v(x,0)=g(x)$?
\item[(iii)] Express $u(x,t)$ in terms of this series and the ambient temperature $T_{\mathrm{env}}$.
\end{enumerate}

\emph{Hint:} You should obtain a generalized Fourier series expansion of $g$ in terms of the eigenfunctions $\{X_n\}$.

\medskip

(e) Explorations and extensions.

\begin{enumerate}
\item[(i)] Consider the limiting cases $\alpha\to 0$ and $\alpha\to\infty$. What boundary conditions do you expect to recover at the ends of the rod? How might the eigenfunctions and eigenvalues simplify in these limits?
\item[(ii)] How would the analysis change if the ambient temperatures at the two ends were different, say $T_{\mathrm{env},0}$ at $x=0$ and $T_{\mathrm{env},L}$ at $x=L$? Sketch (in words or formulas) how you might first find a steady-state solution and then study the transient behavior.
\end{enumerate}

\end{problem}

% ===== Example 5: Heat Equation with Mixed (Robin) Boundary Conditions (full solution) =====
\begin{problem}[Heat Equation with Mixed (Robin) Boundary Conditions]
Let $u(x,t)$ denote the temperature in a homogeneous rod of length $L$ satisfying the heat equation
\[
u_t = \kappa u_{xx},\qquad 0<x<L,\ t>0,
\]
with Newton (Robin) boundary conditions at the ends:
\begin{align*}
- \kappa u_x(0,t) &= h\bigl(u(0,t)-T_{\mathrm{env}}\bigr),\\
\kappa u_x(L,t) &= h\bigl(u(L,t)-T_{\mathrm{env}}\bigr),
\end{align*}
where $\kappa>0$, $h>0$, and $T_{\mathrm{env}}$ are constants. The initial condition is $u(x,0)=f(x)$ for $0<x<L$.

Use the Fourier (separation of variables) method to obtain a series representation of $u(x,t)$ for $t>0$, expressing your answer in terms of the eigenvalues $\{\mu_n\}_{n=1}^{\infty}$ determined by an appropriate transcendental equation and the corresponding eigenfunctions. Clearly indicate the transformed problem, the eigenvalue problem, and how the coefficients of the series are determined from $f$.
\end{problem}

\begin{solution}
We begin by homogenizing the boundary conditions. The ambient temperature $T_{\mathrm{env}}$ is constant in space and time, so it is natural to measure temperature relative to this reference. Define
\[
v(x,t) = u(x,t) - T_{\mathrm{env}}.
\]
Since $T_{\mathrm{env}}$ is constant, we have $v_t = u_t$ and $v_{xx}=u_{xx}$, so $v$ satisfies the same heat equation:
\[
v_t = \kappa v_{xx},\qquad 0<x<L,\ t>0.
\]
The boundary conditions become
\[
- \kappa v_x(0,t) = h v(0,t),\qquad \kappa v_x(L,t) = h v(L,t).
\]
Dividing by $\kappa$ and defining $\alpha = h/\kappa>0$ gives the homogeneous Robin conditions
\[
v_x(0,t) = -\alpha\, v(0,t),\qquad v_x(L,t) = \alpha\, v(L,t).
\]
The initial condition becomes
\[
v(x,0) = g(x) := f(x) - T_{\mathrm{env}}.
\]
Thus our task is to solve
\begin{align*}
v_t &= \kappa v_{xx}, && 0<x<L,\ t>0,\\
v_x(0,t) &= -\alpha v(0,t), && t>0,\\
v_x(L,t) &= \alpha v(L,t), && t>0,\\
v(x,0) &= g(x), && 0<x<L.
\end{align*}
Once $v$ is known, we recover $u$ by $u(x,t)=v(x,t)+T_{\mathrm{env}}$.

\medskip

\textbf{Separation of variables and the eigenvalue problem.}
We apply the Fourier method by seeking separated solutions $v(x,t) = X(x)T(t)$ with $X\not\equiv 0$, $T\not\equiv 0$. Substituting into the PDE,
\[
X(x)T'(t) = \kappa X''(x)T(t).
\]
Assuming $X$ and $T$ are nonzero, we divide both sides by $\kappa X T$:
\[
\frac{T'(t)}{\kappa T(t)} = \frac{X''(x)}{X(x)} = -\lambda,
\]
where the common value must be a constant, which we denote by $-\lambda$ following the usual convention that leads to decaying-in-time solutions for $\lambda>0$.

This yields the system
\[
\frac{T'}{T} = -\kappa \lambda,\qquad X'' + \lambda X = 0.
\]
The boundary conditions on $v$ translate to $X$ (since $T(t)\neq 0$ for nontrivial separated solutions):
\[
X'(0) = -\alpha X(0),\qquad X'(L) = \alpha X(L).
\]
Hence $X$ must satisfy the eigenvalue problem
\[
X''(x) + \lambda X(x) = 0,\quad 0<x<L;\qquad
X'(0) = -\alpha X(0),\quad X'(L) = \alpha X(L).
\]

\medskip

\textbf{Analysis of the spatial eigenvalue problem.}
We must determine the admissible values of $\lambda$ (the eigenvalues) and the corresponding nontrivial functions $X$ (the eigenfunctions).

\emph{1. Excluding $\lambda\le 0$.} First consider $\lambda=0$. Then $X''=0$, so $X(x)=ax+b$. The boundary condition at $x=0$ gives
\[
X'(0) = a = -\alpha X(0) = -\alpha b.
\]
Thus $a=-\alpha b$. At $x=L$,
\[
X'(L) = a = \alpha X(L) = \alpha(aL + b).
\]
Using $a=-\alpha b$ in the second equation:
\[
-\alpha b = \alpha\bigl(-\alpha b\, L + b\bigr)
= \alpha b(1-\alpha L).
\]
If $b\neq 0$, dividing by $\alpha b$ yields $-1 = 1 - \alpha L$, or $\alpha L = 2$. But then $a=-\alpha b\neq 0$, so $X$ is nonconstant; however, substituting back shows this contradicts the previous boundary condition unless $\alpha L$ is tuned exactly, and even then one obtains at most a very special case. In the generic situation (and in Sturm–Liouville theory more generally), $\lambda=0$ does not produce a robust family of eigenfunctions. More systematically, one can check that for $\lambda=0$ the two boundary conditions force $a=b=0$, so $X\equiv 0$.

Now suppose $\lambda<0$. Write $\lambda=-\nu^2$ with $\nu>0$. Then $X''-\nu^2 X=0$, so
\[
X(x) = A e^{\nu x} + B e^{-\nu x}.
\]
Applying the boundary condition at $x=0$:
\[
X'(0) = \nu A - \nu B = -\alpha (A+B).
\]
At $x=L$:
\[
X'(L) = \nu A e^{\nu L} - \nu B e^{-\nu L}
= \alpha\bigl(A e^{\nu L} + B e^{-\nu L}\bigr).
\]
These two linear equations in $A$ and $B$ lead, after elimination, to a condition that cannot be satisfied for $\nu>0$ and $\alpha>0$ except trivially; equivalently, it is a standard fact that the Robin–Robin Sturm–Liouville problem with positive $\alpha$ has no negative eigenvalues. Thus $\lambda<0$ yields only the trivial solution $X\equiv 0$.

Therefore, the only relevant eigenvalues satisfy
\[
\lambda>0.
\]
We write $\lambda=\mu^2$ with $\mu>0$.

\emph{2. The case $\lambda=\mu^2>0$.} The equation
\[
X'' + \mu^2 X = 0
\]
has the general solution
\[
X(x) = A\cos(\mu x) + B\sin(\mu x).
\]
Its derivative is
\[
X'(x) = -A\mu\sin(\mu x) + B\mu\cos(\mu x).
\]

Applying $X'(0)=-\alpha X(0)$:
\[
X'(0) = B\mu = -\alpha X(0) = -\alpha A.
\]
Thus
\[
B = -\frac{\alpha}{\mu}A.
\]

Next apply $X'(L)=\alpha X(L)$:
\begin{align*}
X'(L) &= -A\mu\sin(\mu L) + B\mu\cos(\mu L)\\
&= -A\mu\sin(\mu L) - \alpha A\cos(\mu L),
\end{align*}
since $B\mu = -\alpha A$.
On the other hand,
\begin{align*}
X(L) &= A\cos(\mu L) + B\sin(\mu L)\\
&= A\cos(\mu L) - \frac{\alpha}{\mu}A\sin(\mu L).
\end{align*}
Thus the boundary condition $X'(L)=\alpha X(L)$ becomes
\[
- A\mu\sin(\mu L) - \alpha A\cos(\mu L)
= \alpha A\cos(\mu L) - \alpha^2\frac{A}{\mu}\sin(\mu L).
\]
Dividing by $A$ (which must be nonzero for a nontrivial eigenfunction) and multiplying by $\mu$ gives
\[
- \mu^2\sin(\mu L) - \alpha\mu\cos(\mu L)
= \alpha\mu\cos(\mu L) - \alpha^2\sin(\mu L).
\]
Bringing all terms to the left-hand side:
\[
(-\mu^2 + \alpha^2)\sin(\mu L) - 2\alpha\mu\cos(\mu L) = 0.
\]
Thus the eigenvalues $\lambda_n=\mu_n^2$ are determined by the transcendental equation
\[
F(\mu) := (\alpha^2-\mu^2)\sin(\mu L) - 2\alpha\mu\cos(\mu L) = 0.
\]
This equation has an infinite sequence of positive roots
\[
0<\mu_1<\mu_2<\cdots<\mu_n<\cdots,\qquad \mu_n\to\infty,
\]
as can be seen from general Sturm–Liouville theory or by inspecting $F(\mu)$ on successive intervals of length $\pi/L$.

For each root $\mu_n$, the corresponding eigenfunction can be taken as
\[
X_n(x) = \cos(\mu_n x) - \frac{\alpha}{\mu_n}\sin(\mu_n x),
\]
where we have chosen $A=1$ and $B=-\frac{\alpha}{\mu_n}$ for convenience. Each $X_n$ satisfies
\[
X_n'' + \mu_n^2 X_n = 0,\quad X_n'(0)=-\alpha X_n(0),\quad X_n'(L)=\alpha X_n(L).
\]
Standard Sturm–Liouville theory ensures that the family $\{X_n\}_{n=1}^\infty$ is orthogonal in $L^2(0,L)$ with respect to the usual inner product:
\[
\int_0^L X_m(x)X_n(x)\,dx = 0\quad\text{for }m\neq n.
\]
Moreover, under mild regularity assumptions on $g$, the eigenfunctions form a complete set, so $g$ can be expanded as a generalized Fourier series in $\{X_n\}$.

\medskip

\textbf{Time dependence and separated solutions.}
For each eigenvalue $\lambda_n=\mu_n^2$, the time equation
\[
T_n' = -\kappa\lambda_n T_n = -\kappa\mu_n^2 T_n
\]
has solution
\[
T_n(t) = e^{-\kappa\mu_n^2 t},
\]
up to a multiplicative constant that we absorb into the spatial coefficient.

Thus each separated solution has the form
\[
v_n(x,t) = X_n(x)\, e^{-\kappa\mu_n^2 t}.
\]
By linearity, the general solution that satisfies the Robin boundary conditions is a (possibly infinite) linear combination of these separated solutions:
\[
v(x,t) = \sum_{n=1}^\infty c_n X_n(x) e^{-\kappa\mu_n^2 t},
\]
where the coefficients $c_n$ are determined by the initial condition.

\medskip

\textbf{Determining the coefficients from the initial data.}
At $t=0$ we have
\[
v(x,0) = g(x) = f(x) - T_{\mathrm{env}} = \sum_{n=1}^\infty c_n X_n(x).
\]
This is a generalized Fourier expansion of $g$ in the orthogonal basis $\{X_n\}$.

Let us denote
\[
\|X_n\|^2 := \int_0^L X_n(x)^2\,dx.
\]
Orthogonality implies that
\[
c_n = \frac{\displaystyle \int_0^L g(x)\, X_n(x)\, dx}{\displaystyle \int_0^L X_n(x)^2\, dx}
= \frac{\displaystyle \int_0^L \bigl(f(x)-T_{\mathrm{env}}\bigr)\, X_n(x)\, dx}{\displaystyle \int_0^L X_n(x)^2\, dx}.
\]
In practice, the denominator can be computed explicitly (though the resulting expression is somewhat cumbersome), or left in integral form.

Therefore the solution $v$ is
\[
v(x,t) = \sum_{n=1}^\infty
\left(
\frac{\displaystyle \int_0^L \bigl(f(\xi)-T_{\mathrm{env}}\bigr)\, X_n(\xi)\, d\xi}{\displaystyle \int_0^L X_n(\xi)^2\, d\xi}
\right)
X_n(x)\, e^{-\kappa\mu_n^2 t},
\]
with
\[
X_n(x) = \cos(\mu_n x) - \frac{\alpha}{\mu_n}\sin(\mu_n x),
\]
and $\mu_n$ the positive roots of
\[
(\alpha^2-\mu^2)\sin(\mu L) - 2\alpha\mu\cos(\mu L) = 0.
\]

Finally, recalling that $u(x,t)=v(x,t)+T_{\mathrm{env}}$, we obtain
\[
u(x,t) = T_{\mathrm{env}}
+ \sum_{n=1}^\infty
\left(
\frac{\displaystyle \int_0^L \bigl(f(\xi)-T_{\mathrm{env}}\bigr)\, X_n(\xi)\, d\xi}{\displaystyle \int_0^L X_n(\xi)^2\, d\xi}
\right)
X_n(x)\, e^{-\kappa\mu_n^2 t}.
\]

\medskip

\textbf{Conceptual remarks.}
This example illustrates several central ideas of the Fourier method for boundary value problems:

\begin{itemize}
\item Homogenization of boundary conditions by subtracting an appropriate steady or reference state (here, the constant ambient temperature).
\item Reduction of the PDE with boundary conditions to a Sturm–Liouville eigenvalue problem for the spatial part, here with Robin (mixed) conditions involving both the function and its derivative.
\item The appearance of a transcendental eigenvalue equation, whose roots define the spectrum, and the use of the corresponding eigenfunctions as a generalized Fourier basis.
\item Expansion of the initial data in this basis and exponential decay of each mode according to its eigenvalue.
\end{itemize}

Even though the eigenfunctions are no longer the simple sine or cosine functions arising from Dirichlet or Neumann boundary conditions, the overall structure of the Fourier method is unchanged: one still builds the solution as a superposition of separated modes adapted to the given boundary conditions.
\end{solution}

% ===== Example 6: Nonhomogeneous Forcing in a Heat Equation Boundary Value Problem (inquiry-based) =====
\begin{problem}[Nonhomogeneous Forcing in a Heat Equation Boundary Value Problem]
Consider a thin, perfectly insulated rod of length $\pi$. Its ends at $x=0$ and $x=\pi$ are kept at zero temperature by contact with large ice baths. Inside the rod there is an internal heat source whose intensity varies in space and in time. We denote the temperature by $u(x,t)$ and suppose that the internal heat source is described by a function $S(x,t)$ added to the usual heat equation.

We take the simplest nontrivial case
\[
u_t = u_{xx} + S(x,t), \qquad 0<x<\pi,\ t>0,
\]
with Dirichlet boundary conditions
\[
u(0,t)=0,\qquad u(\pi,t)=0,\qquad t>0,
\]
and initial condition
\[
u(x,0)=0,\qquad 0<x<\pi.
\]
In this problem you will discover how to adapt the Fourier method to handle the nonhomogeneous forcing $S(x,t)$, and then work out a concrete example.

\medskip

(a) First recall the spatial eigenvalue problem that appears when using separation of variables for the \emph{homogeneous} heat equation (that is, with $S\equiv 0$). Consider
\[
X''(x) + \lambda X(x) = 0,\qquad X(0)=0,\quad X(\pi)=0.
\]
Find all eigenvalues $\lambda_n$ and corresponding eigenfunctions $X_n(x)$.

\emph{Hint:} You should recover a familiar sine series basis on the interval $(0,\pi)$.

\medskip

(b) Now return to the \emph{nonhomogeneous} problem with an internal source $S(x,t)$. Motivated by part (a), suppose that $u(x,t)$ can be expanded in terms of the eigenfunctions you found:
\[
u(x,t) = \sum_{n=1}^\infty T_n(t)\,X_n(x).
\]
Explain why it is natural also to expand the source term $S(x,t)$ in the same eigenfunctions,
\[
S(x,t) = \sum_{n=1}^\infty b_n(t)\,X_n(x),
\]
for some time-dependent coefficients $b_n(t)$. How can you compute the functions $b_n(t)$ from $S(x,t)$ using orthogonality?

\emph{Hint:} Recall the orthogonality relation for the eigenfunctions on $(0,\pi)$ with respect to the standard inner product
\[
\langle f,g\rangle = \int_0^\pi f(x)g(x)\,dx.
\]

\medskip

(c) Substitute the expansions
\[
u(x,t)=\sum_{n=1}^\infty T_n(t)X_n(x), \qquad 
S(x,t)=\sum_{n=1}^\infty b_n(t)X_n(x)
\]
into the PDE
\[
u_t = u_{xx} + S(x,t),
\]
and use the fact that $X_n''(x) = -\lambda_n X_n(x)$. Then use orthogonality of the $X_n$ to derive a system of uncoupled first-order ordinary differential equations for the time-dependent coefficients $T_n(t)$.

Write down the ODE satisfied by $T_n(t)$ in terms of $\lambda_n$ and $b_n(t)$, and write the initial condition for $T_n(0)$ in terms of the initial data $u(x,0)$.

\emph{Hint:} Multiply both sides of the PDE (after substituting the series) by $X_m(x)$, integrate from $0$ to $\pi$, and use orthogonality to isolate a single index $m$.

\medskip

(d) We now examine a specific internal heat source
\[
S(x,t) = e^{-t}\sin x.
\]
Notice that this is already written as a single eigenfunction of the spatial operator. Use your work in part (c) to solve the full boundary value problem
\[
\begin{cases}
u_t = u_{xx} + e^{-t}\sin x, & 0<x<\pi,\ t>0,\\[4pt]
u(0,t)=0,\quad u(\pi,t)=0, & t>0,\\[4pt]
u(x,0)=0, & 0<x<\pi.
\end{cases}
\]

\begin{enumerate}
\item[(i)] Determine the coefficients $b_n(t)$ for this particular forcing $S(x,t)$.
\item[(ii)] Show that the system of ODEs for $T_n(t)$ simplifies dramatically, and identify which modes $n$ are actually forced.
\item[(iii)] Solve the resulting first-order ODE(s) for $T_n(t)$ with the given initial data and write the final expression for $u(x,t)$.
\end{enumerate}

\emph{Hint:} For the nontrivial mode, you will encounter an ODE of the form
\[
T'(t) + \lambda T(t) = e^{-t}, \qquad T(0)=0,
\]
which you can solve by an integrating factor.

\medskip

(e) Explore two extensions of this example.

\begin{enumerate}
\item[(i)] Suppose the source is \emph{time-independent}: $S(x,t)=S(x)$ only. Use your general formula from part (c) to describe qualitatively what you expect for the long-time behavior of $u(x,t)$ as $t\to\infty$. Do you expect a steady-state solution? Why?

\item[(ii)] Suppose instead that the boundary conditions are changed to \emph{Neumann} conditions
\[
u_x(0,t)=0,\qquad u_x(\pi,t)=0,
\]
while keeping the same PDE and source $S(x,t)=e^{-t}\sin x$. How would the eigenvalue problem and eigenfunctions change? In outline, how would the Fourier method for this nonhomogeneous Neumann problem differ from what you did above?
\end{enumerate}

\end{problem}

% ===== Example 6: Nonhomogeneous Forcing in a Heat Equation Boundary Value Problem (full solution) =====
\begin{problem}[Nonhomogeneous Forcing in a Heat Equation Boundary Value Problem]
Consider the initial–boundary value problem
\[
\begin{cases}
u_t = u_{xx} + e^{-t}\sin x, & 0<x<\pi,\ t>0,\\[4pt]
u(0,t)=0,\quad u(\pi,t)=0, & t>0,\\[4pt]
u(x,0)=0, & 0<x<\pi.
\end{cases}
\]
Use the Fourier (eigenfunction) method to find the temperature $u(x,t)$.
\end{problem}

\begin{solution}
We solve this nonhomogeneous heat equation by expanding in the eigenfunctions of the associated homogeneous spatial operator. This illustrates the general principle of the Fourier method for boundary value problems: we diagonalize the spatial operator using its eigenfunctions, expand both the solution and the forcing in that basis, and obtain decoupled ordinary differential equations in time for the Fourier coefficients.

\medskip

\noindent\textbf{1. Spatial eigenvalue problem and eigenfunctions.}

We first recall the eigenfunctions of the one-dimensional Laplacian with homogeneous Dirichlet boundary conditions on $(0,\pi)$. The associated eigenvalue problem is
\[
X''(x) + \lambda X(x)=0,\qquad X(0)=0,\quad X(\pi)=0.
\]
For $\lambda>0$, the general solution is $X(x)=A\cos(\sqrt{\lambda}\,x)+B\sin(\sqrt{\lambda}\,x)$. The condition $X(0)=0$ forces $A=0$, so $X(x)=B\sin(\sqrt{\lambda}\,x)$. The condition $X(\pi)=0$ then yields
\[
B\sin(\sqrt{\lambda}\,\pi)=0.
\]
To obtain nontrivial eigenfunctions ($B\neq 0$) we must have $\sin(\sqrt{\lambda}\,\pi)=0$, hence
\[
\sqrt{\lambda} = n,\qquad n=1,2,3,\dots,
\]
so that
\[
\lambda_n = n^2,\qquad X_n(x)=\sin(nx).
\]
These eigenfunctions form an orthogonal basis of $L^2(0,\pi)$ with respect to the usual inner product
\[
\langle f,g\rangle = \int_0^\pi f(x)g(x)\,dx.
\]

\medskip

\noindent\textbf{2. Eigenfunction expansion of $u$ and of the forcing.}

Because the spatial operator $u\mapsto u_{xx}$ with Dirichlet boundary conditions has eigenfunctions $\sin(nx)$, it is natural to look for a solution $u(x,t)$ of the form
\[
u(x,t) = \sum_{n=1}^\infty T_n(t)\,\sin(nx),
\]
where the time-dependent coefficients $T_n(t)$ are to be determined.

We also expand the forcing term in the same basis. Our source is
\[
S(x,t) = e^{-t}\sin x.
\]
But $\sin x$ is exactly the first eigenfunction $X_1(x)$, so we immediately see that
\[
S(x,t) = e^{-t}\sin x = b_1(t)\sin x + \sum_{n=2}^\infty b_n(t)\sin(nx),
\]
with
\[
b_1(t)=e^{-t}, \qquad b_n(t)=0 \ \text{for }n\ge 2.
\]
Equivalently, the only nonzero Fourier coefficient of $S(x,t)$ is its first sine coefficient.

\medskip

\noindent\textbf{3. Deriving ODEs for the Fourier coefficients.}

We now substitute the eigenfunction expansion for $u$ into the PDE. Compute
\[
u_t(x,t) = \sum_{n=1}^\infty T_n'(t)\,\sin(nx),
\]
and
\[
u_{xx}(x,t) = \sum_{n=1}^\infty T_n(t)\,\frac{d^2}{dx^2}\sin(nx)
= \sum_{n=1}^\infty T_n(t)\,(-n^2)\sin(nx)
= -\sum_{n=1}^\infty n^2 T_n(t)\,\sin(nx).
\]
The PDE
\[
u_t = u_{xx} + e^{-t}\sin x
\]
becomes
\[
\sum_{n=1}^\infty T_n'(t)\,\sin(nx)
=
-\sum_{n=1}^\infty n^2 T_n(t)\,\sin(nx)
+ e^{-t}\sin x.
\]
We regard $e^{-t}\sin x$ as $e^{-t}\sin x + 0\cdot \sin(2x)+0\cdot \sin(3x)+\cdots$, that is,
\[
e^{-t}\sin x = \sum_{n=1}^\infty b_n(t)\,\sin(nx),
\]
with $b_1(t)=e^{-t}$ and $b_n(t)=0$ for $n\ge 2$.

Thus the PDE can be written as
\[
\sum_{n=1}^\infty T_n'(t)\,\sin(nx)
=
-\sum_{n=1}^\infty n^2 T_n(t)\,\sin(nx)
+\sum_{n=1}^\infty b_n(t)\,\sin(nx).
\]
Because the sine functions form an orthogonal basis, equality of these series for all $x$ implies equality of the coefficients of each $\sin(nx)$:
\[
T_n'(t) = -n^2 T_n(t) + b_n(t), \qquad n=1,2,\dots.
\]
In our case,
\[
b_1(t)=e^{-t},\qquad b_n(t)=0\ \text{for }n\ge 2,
\]
so
\[
\begin{cases}
T_1'(t) = -1^2 T_1(t) + e^{-t},\\[2pt]
T_n'(t) = -n^2 T_n(t), & n\ge 2.
\end{cases}
\]

The initial condition $u(x,0)=0$ gives
\[
u(x,0) = \sum_{n=1}^\infty T_n(0)\,\sin(nx) = 0.
\]
Since $\{\sin(nx)\}$ is a basis, all coefficients must vanish:
\[
T_n(0)=0,\qquad n=1,2,\dots.
\]

\medskip

\noindent\textbf{4. Solving the ODEs for $T_n(t)$.}

For $n\ge 2$ the ODEs are homogeneous:
\[
T_n'(t) = -n^2 T_n(t),\qquad T_n(0)=0.
\]
The unique solution is
\[
T_n(t)\equiv 0,\qquad n\ge 2,
\]
since the general solution is $T_n(t)=C_n e^{-n^2 t}$ and the initial condition forces $C_n=0$.

For $n=1$ we have the forced first-order linear ODE
\[
T_1'(t) + T_1(t) = e^{-t},\qquad T_1(0)=0.
\]
We solve this by using an integrating factor. The integrating factor is $e^{\int 1\,dt}=e^{t}$, and multiplying the equation by $e^{t}$ gives
\[
e^{t}T_1'(t) + e^{t}T_1(t) = e^{t}\,e^{-t} = 1.
\]
The left-hand side is the derivative of $e^{t}T_1(t)$:
\[
\frac{d}{dt}\bigl(e^t T_1(t)\bigr) = 1.
\]
Integrating from $0$ to $t$ yields
\[
e^{t}T_1(t) - e^{0}T_1(0) = \int_0^t 1\,ds = t.
\]
Since $T_1(0)=0$, we obtain
\[
e^{t}T_1(t) = t,\qquad\text{so}\qquad
T_1(t) = t e^{-t}.
\]

In summary,
\[
T_1(t) = t e^{-t},\qquad T_n(t) = 0\ \text{for }n\ge 2.
\]

\medskip

\noindent\textbf{5. Assembling the solution.}

Returning to the eigenfunction expansion
\[
u(x,t) = \sum_{n=1}^\infty T_n(t)\,\sin(nx),
\]
we see that only the first term survives:
\[
u(x,t) = T_1(t)\sin x = t e^{-t}\sin x.
\]

We may check quickly that this formula satisfies all conditions.

First, the PDE:
\[
u_t = \frac{\partial}{\partial t}\bigl(t e^{-t}\sin x\bigr)
= (e^{-t} - t e^{-t})\sin x
= (1-t)e^{-t}\sin x,
\]
and
\[
u_{xx} = \frac{\partial^2}{\partial x^2}\bigl(t e^{-t}\sin x\bigr)
= t e^{-t}(-\sin x) = -t e^{-t}\sin x.
\]
Therefore
\[
u_{xx} + e^{-t}\sin x
= -t e^{-t}\sin x + e^{-t}\sin x
= (1-t)e^{-t}\sin x
= u_t,
\]
so the PDE is satisfied.

Next, the boundary conditions:
\[
u(0,t) = t e^{-t}\sin 0 = 0,\qquad
u(\pi,t) = t e^{-t}\sin \pi = 0,
\]
so the Dirichlet conditions hold. Finally, the initial condition:
\[
u(x,0) = 0\cdot e^{0}\sin x = 0,
\]
which matches $u(x,0)=0$.

\medskip

\noindent\textbf{6. Conceptual remarks.}

This example illustrates the central ideas of the Fourier method for boundary value problems with nonhomogeneous forcing:

\begin{itemize}
\item We first solve the spatial eigenvalue problem associated with the homogeneous operator and boundary conditions, obtaining an orthogonal basis of eigenfunctions (here, $\sin(nx)$).
\item We expand both the unknown solution $u(x,t)$ and the given forcing term $S(x,t)$ in this eigenbasis.
\item Orthogonality reduces the PDE to a family of decoupled first-order linear ODEs in time for the Fourier coefficients $T_n(t)$, with nonhomogeneous terms given by the Fourier coefficients of $S$.
\item In the present problem, only a single spatial mode is forced, so only one ODE is nontrivial, and the final solution is a single time-dependent Fourier mode.
\end{itemize}

In more complicated problems, the same procedure leads to a system of ODEs that can be solved by integrating factors or other standard ODE methods. The principle remains the same: the eigenfunctions encode the boundary conditions and diagonalize the spatial operator, allowing us to treat time evolution and forcing mode by mode.

\end{solution}

\section{Case Study: Burgers' Equation (*)}
% --- Narrative plan (auto-generated) ---
% This section uses Burgers’ equation as a unifying case study to explore several core ideas in partial differential equations, including nonlinear advection, diffusion, shock formation, and the interplay between analytic and numerical methods. Burgers’ equation is simple enough to be written on a single line, yet rich enough to display many of the qualitative behaviors encountered in fluid dynamics, traffic flow, and conservation laws, such as steepening waves, shock fronts, and smoothing by viscosity. By examining both the inviscid and viscous forms, we see how adding a small diffusive term changes the nature of solutions and resolves singularities.
%
% The techniques we develop here connect to many other parts of applied mathematics. The inviscid equation highlights the method of characteristics and connects directly to ordinary differential equations and dynamical systems along characteristic curves. The viscous equation can be transformed into the heat equation via the Hopf–Cole transform, which leads naturally to Fourier analysis, convolution with the heat kernel, and, in more advanced treatments, connections with complex analysis through integral representations. Throughout this section we emphasize how these tools, first encountered in simpler settings, come together in a coherent way around a single nonlinear PDE model.
%
% Our goal is to let you rediscover standard techniques through guided problems rather than only reading formal solutions. By starting from linear transport and diffusion equations and gradually adding nonlinearity and viscosity, you will see how methods are adapted and extended. This case study prepares you for more sophisticated conservation laws and nonlinear PDEs in fluid mechanics, as well as for numerical methods that must cope with sharp gradients and discontinuities.

% ===== Example 1: Warm-Up: Linear Transport and the Method of Characteristics (inquiry-based) =====
\begin{problem}[Warm-Up: Linear Transport and the Method of Characteristics]
A basic model for advection of a scalar quantity along a one-dimensional medium is the \emph{linear transport equation}
\[
u_t + c\,u_x = 0,
\]
where $u(t,x)$ is the concentration (or temperature, density, etc.), and $c$ is a constant velocity. In words, the model says that $u$ is simply carried along with the flow, without diffusion, reaction, or sources. Before we confront nonlinear transport such as Burgers' equation, we will use this example to introduce and practice the method of characteristics. The goal is to see how the partial differential equation encodes the translation of the initial profile and how that motion is represented geometrically by families of curves in the $(t,x)$-plane.

Consider the Cauchy problem on the whole line:
\[
\begin{cases}
u_t + c\,u_x = 0, & t>0,\; x\in\mathbb{R},\\[0.3em]
u(0,x) = u_0(x), & x\in\mathbb{R},
\end{cases}
\]
where $c$ is a fixed real constant and $u_0:\mathbb{R}\to\mathbb{R}$ is a given initial profile.

\smallskip

(a) The method of characteristics seeks curves in the $(t,x)$-plane along which the partial differential equation reduces to an ordinary differential equation. Let a characteristic curve be parameterized by $s\mapsto (t(s),x(s))$, and let $u(s):=u(t(s),x(s))$ be the value of the solution along this curve.

\quad(i) Use the chain rule to express $\dfrac{du}{ds}$ in terms of $u_t$, $u_x$, $t'(s)$, and $x'(s)$.

\quad(ii) We would like to choose $t'(s)$ and $x'(s)$ so that the PDE implies $\dfrac{du}{ds}=0$, that is, $u$ is constant along each characteristic. Show that if you pick $t'(s)=1$ and $x'(s)=c$, then the condition $u_t + c\,u_x =0$ indeed gives $\dfrac{du}{ds}=0$.

\quad(iii) Solve the resulting system of ordinary differential equations for the characteristic curves:
\[
\frac{dt}{ds}=1, \qquad \frac{dx}{ds}=c.
\]
Describe the family of characteristic curves in the $(t,x)$-plane (for example, are they straight lines, what is their slope, and what parameter labels which curve we are on?).

\emph{Hint:} One convenient way is to eliminate the parameter $s$ and find a direct relationship between $t$ and $x$ along a characteristic.

\smallskip

(b) Use your description from part (a) to write an explicit formula for all characteristic curves passing through the initial line $t=0$. More concretely:

\quad(i) Let $\xi\in\mathbb{R}$ denote the $x$--coordinate where a characteristic intersects the initial line $t=0$. Express $x$ at a later time $t$ along this characteristic in terms of $\xi$, $c$, and $t$.

\quad(ii) Argue that we can label each characteristic by its intersection point $\xi$ at $t=0$, and that along this characteristic, the solution satisfies
\[
u(t,x(t)) = u(0,\xi)=u_0(\xi).
\]

\quad(iii) Use the relation between $x$, $t$, and $\xi$ to eliminate $\xi$ and obtain a formula for $u(t,x)$ directly in terms of $u_0$ and the variables $t$ and $x$.

\emph{Hint:} Solve your formula from part (b)(i) for $\xi$ as a function of $(t,x)$, then substitute into $u_0(\xi)$.

\smallskip

(c) Let us now explore how different initial profiles move under this transport equation.

\quad(i) Take
\[
u_0(x) = H(x) := \begin{cases}
0, & x<0,\\
1, & x\ge 0,
\end{cases}
\]
the \emph{Heaviside step} profile. Using your formula from part (b), write an explicit expression for $u(t,x)$ in this case. Carefully interpret how the discontinuity moves in the $(t,x)$-plane.

\quad(ii) Take instead a smooth ``bump'' profile
\[
u_0(x) = e^{-x^2}.
\]
Again, write down $u(t,x)$ and describe in words what happens to the shape and position of the bump over time.

\quad(iii) Finally, consider an oscillatory profile
\[
u_0(x) = \cos(kx), \quad k>0.
\]
Compute $u(t,x)$ and simplify it as much as possible. Is the frequency of oscillations changing with time, or only their phase? How can you see this in your formula?

\emph{Hint:} For (iii), use the cosine addition formula.

\smallskip

(d) In this part you will consolidate the geometric and analytic points of view.

\quad(i) Sketch a few characteristic lines $x-ct=\text{constant}$ in the $(t,x)$-plane. On the line $t=0$, sketch your favorite initial profile $u_0(x)$ (for example, the bump $e^{-x^2}$). Use arrows along characteristics to indicate how the values of $u$ are transported.

\quad(ii) Explain in words why the solution you found in part (b) can be described as follows: ``the entire initial profile is translated rigidly at constant speed $c$ without any change in shape.''

\quad(iii) Show that if $u_0$ is continuous and bounded, then for each fixed $t>0$ the solution $u(t,\cdot)$ is also continuous and bounded, and
\[
\|u(t,\cdot)\|_{L^\infty(\mathbb{R})} = \|u_0\|_{L^\infty(\mathbb{R})}.
\]
Why does this make sense physically for a pure transport equation?

\emph{Hint:} Use your explicit formula for $u(t,x)$ and basic properties of translations of functions.

\smallskip

(e) This simple linear model is a stepping stone toward understanding the nonlinear Burgers' equation
\[
u_t + u\,u_x = 0,
\]
which we will study later in this chapter.

\quad(i) In the linear model, the characteristic curves are straight lines $x-ct=\text{constant}$ with a fixed slope $1/c$ (or vertical if $c=0$). For Burgers' equation, the characteristic speed is not constant but depends on $u$ itself. Based on your current understanding, describe informally what you expect to change in the geometric picture of characteristics when going from $u_t + c u_x = 0$ to $u_t + u u_x =0$.

\quad(ii) Suppose in the linear case we take $c<0$ instead of $c>0$. How does this change the direction of motion of the initial profile? How does this show up in your explicit formula and in the picture of characteristics?

\emph{Hint:} Think about whether characteristics tilt to the left or to the right as $t$ increases, and whether the sign of $c$ affects the \emph{magnitude} or only the \emph{direction} of the translation.
\end{problem}

% ===== Example 1: Warm-Up: Linear Transport and the Method of Characteristics (full solution) =====
\begin{problem}[Warm-Up: Linear Transport and the Method of Characteristics]
Consider the linear transport equation with constant velocity $c\in\mathbb{R}$:
\[
\begin{cases}
u_t + c\,u_x = 0, & t>0,\; x\in\mathbb{R},\\[0.3em]
u(0,x)=u_0(x), & x\in\mathbb{R},
\end{cases}
\]
where $u_0:\mathbb{R}\to\mathbb{R}$ is a given function.

\begin{enumerate}
\item Use the method of characteristics to derive an explicit formula for the solution $u(t,x)$ in terms of $u_0$.
\item Apply your formula to the following initial data:
  \begin{enumerate}
  \item The Heaviside step $u_0(x)=H(x)$.
  \item The Gaussian bump $u_0(x)=e^{-x^2}$.
  \item The oscillatory profile $u_0(x)=\cos(kx)$ with $k>0$.
  \end{enumerate}
  In each case, describe briefly how the profile evolves in time.
\item Show that if $u_0$ is bounded and continuous, then for each $t>0$ the function $u(t,\cdot)$ is also bounded and continuous, and
\[
\|u(t,\cdot)\|_{L^\infty(\mathbb{R})} = \|u_0\|_{L^\infty(\mathbb{R})}.
\]
\end{enumerate}
Briefly indicate how this example illustrates the method of characteristics that will be used later for Burgers' equation $u_t + u u_x=0$.
\end{problem}

\begin{solution}
We solve the transport equation
\[
u_t + c\,u_x = 0
\]
by the method of characteristics. The central idea of this method is to reduce the partial differential equation to a family of ordinary differential equations along special curves in the $(t,x)$-plane, called characteristic curves.

\medskip

\textbf{1. Derivation of the characteristic equations and solution formula.}

Let a characteristic curve be parameterized by a variable $s\mapsto (t(s),x(s))$, and consider the function $u$ along this curve:
\[
U(s) := u(t(s),x(s)).
\]
By the chain rule,
\[
\frac{dU}{ds} = u_t(t(s),x(s))\,t'(s) + u_x(t(s),x(s))\,x'(s).
\]
We wish to arrange that $\dfrac{dU}{ds}=0$, so that $u$ is constant along each characteristic. Comparing with the PDE $u_t + c\,u_x = 0$, it is natural to choose
\[
t'(s) = 1, \qquad x'(s) = c.
\]
With this choice, we obtain
\[
\frac{dU}{ds}
= u_t(t(s),x(s))\cdot 1 + u_x(t(s),x(s))\cdot c
= u_t + c\,u_x = 0.
\]
Thus $U(s)$ is constant along any solution of this system. The characteristic curves therefore satisfy the system of ordinary differential equations
\[
\frac{dt}{ds} = 1, \qquad \frac{dx}{ds} = c.
\]

Integrating these, we find
\[
t(s) = s + C_1, \qquad x(s) = cs + C_2,
\]
where $C_1$ and $C_2$ are constants. Eliminating $s$ between these two equations, we obtain
\[
x - c t = C_2 - c C_1 = \text{constant}.
\]
Thus the characteristics in the $(t,x)$-plane are straight lines of slope $c$ in the $(s,x)$ representation, or more usefully, lines of the form
\[
x - c t = \text{constant},
\]
when viewed in the $(t,x)$-plane. Each such line is a characteristic along which $u$ remains constant.

To connect this with the initial condition at time $t=0$, we label a characteristic by its intersection with the initial line. Let $\xi\in\mathbb{R}$ denote the point where the characteristic meets the line $t=0$:
\[
t = 0,\quad x = \xi.
\]
Along the characteristic starting from $(0,\xi)$, the relation $x - c t = \text{constant}$ yields
\[
x - c t = \xi.
\]
Equivalently,
\[
x(t) = \xi + c t.
\]
The value of $u$ remains constant along this characteristic, so for all $t\ge 0$,
\[
u(t,x(t)) = u(0,\xi) = u_0(\xi).
\]
Now we fix $(t,x)$ and identify which characteristic passes through this point. The line $x - ct = \text{constant}$ passing through $(t,x)$ has constant equal to $x-ct$, so its intersection with $t=0$ occurs at the point $(0,\xi)$ where
\[
\xi = x - c t.
\]
Therefore the point $(t,x)$ lies on the characteristic that originated from $(0,x-ct)$, and along that characteristic the value of $u$ is $u_0(x-ct)$. Hence
\[
u(t,x) = u_0(x - c t).
\]
This is the explicit solution formula obtained by the method of characteristics.

We can summarize the result:

\emph{For the Cauchy problem}
\[
u_t + c\,u_x = 0, \quad u(0,x) = u_0(x),
\]
\emph{the solution is given by}
\[
u(t,x) = u_0(x - c t), \quad t\ge 0,\; x\in\mathbb{R}.
\]

Analytically, the PDE has been reduced to a simple translation of the initial data. Geometrically, each characteristic line $x-ct=\text{constant}$ carries the value of $u$ from the initial line $t=0$ forward in time without change.

\medskip

\textbf{2. Examples of evolving profiles.}

We now apply the formula $u(t,x)=u_0(x-ct)$ to the three specified initial profiles.

\smallskip

\emph{(a) Heaviside step $u_0(x)=H(x)$.}

Recall that
\[
H(x) = \begin{cases}
0, & x<0,\\[0.2em]
1, & x\ge 0.
\end{cases}
\]
Then the solution is
\[
u(t,x) = u_0(x - c t) = H(x - c t).
\]
By the definition of $H$, this equals
\[
u(t,x) = 
\begin{cases}
0, & x - c t < 0 \;\;(\text{i.e. } x < c t),\\[0.2em]
1, & x - c t \ge 0 \;\;(\text{i.e. } x \ge c t).
\end{cases}
\]
Thus the discontinuity (the jump from $0$ to $1$) occurs at $x = c t$ and moves with constant speed $c$. If $c>0$, the step moves to the right; if $c<0$, it moves to the left. The height of the jump and the values on either side are unchanged.

\smallskip

\emph{(b) Gaussian bump $u_0(x)=e^{-x^2}$.}

In this case the solution is
\[
u(t,x) = u_0(x-ct) = e^{-(x-ct)^2}.
\]
The shape of the function in $x$ is exactly the same Gaussian bump, but its center has moved from $x=0$ to $x=ct$. There is no spreading or deformation of the profile; it is simply translated to the right if $c>0$ or to the left if $c<0$.

\smallskip

\emph{(c) Oscillatory profile $u_0(x)=\cos(kx)$.}

We obtain
\[
u(t,x) = u_0(x-ct) = \cos\big(k(x - c t)\big).
\]
Using the trigonometric identity
\[
\cos(\alpha - \beta) = \cos\alpha\cos\beta + \sin\alpha\sin\beta,
\]
we can write
\[
u(t,x) = \cos(kx - kct)
= \cos(kx)\cos(kct) + \sin(kx)\,\sin(kct).
\]
For each fixed time $t$, this is a cosine wave in $x$ with the same spatial frequency $k$ and the same amplitude $1$. The effect of time is only to change the phase by $kct$, that is, to shift the pattern in space. Equivalently, it is a rigid translation of the wave to the right (if $c>0$) with constant speed $c$. The wavelength and amplitude remain unchanged.

\medskip

\textbf{3. Preservation of boundedness and the $L^\infty$ norm.}

Assume that $u_0$ is bounded and continuous on $\mathbb{R}$. Then the solution is
\[
u(t,x) = u_0(x-ct).
\]

First, continuity is preserved. For any fixed $t>0$, the map $x\mapsto x-ct$ is a continuous bijection from $\mathbb{R}$ to $\mathbb{R}$. The composition of a continuous function $u_0$ with a continuous map is continuous, hence $x\mapsto u_0(x-ct)$ is continuous. Therefore $u(t,\cdot)$ is continuous for each $t>0$.

Second, boundedness and the supremum norm are preserved. Since $u_0$ is bounded, there exists $M\ge 0$ such that $|u_0(x)|\le M$ for all $x\in\mathbb{R}$, so
\[
|u(t,x)| = |u_0(x-ct)| \le M
\]
for all $x\in\mathbb{R}$. Thus $u(t,\cdot)$ is bounded, and
\[
\|u(t,\cdot)\|_{L^\infty(\mathbb{R})}
= \sup_{x\in\mathbb{R}} |u(t,x)|
= \sup_{x\in\mathbb{R}} |u_0(x-ct)|.
\]
But as $x$ ranges over $\mathbb{R}$, the argument $x-ct$ also ranges over $\mathbb{R}$. Therefore
\[
\sup_{x\in\mathbb{R}} |u_0(x-ct)|
= \sup_{y\in\mathbb{R}} |u_0(y)|
= \|u_0\|_{L^\infty(\mathbb{R})}.
\]
Hence the $L^\infty$ norm is exactly conserved:
\[
\|u(t,\cdot)\|_{L^\infty(\mathbb{R})} = \|u_0\|_{L^\infty(\mathbb{R})}, \quad t\ge 0.
\]

From a physical standpoint, this expresses the idea that the transport equation models pure advection without sources, sinks, or diffusion. The magnitude of the quantity being transported does not grow or decay; it is merely rearranged in space by a rigid translation.

\medskip

\textbf{4. Connection with Burgers' equation.}

This example illustrates the core mechanism of the method of characteristics. We identified characteristic curves in the $(t,x)$-plane along which the solution is constant, reduced the PDE to an ODE system, and then used the initial condition to parametrize the characteristics and obtain an explicit solution formula.

In the linear transport equation, the characteristic speed $c$ is constant, so the characteristics are straight parallel lines $x-ct=\text{constant}$ and never intersect. In the nonlinear Burgers' equation
\[
u_t + u\,u_x = 0,
\]
the characteristic speed is the solution itself, so the characteristic equations become
\[
\frac{dt}{ds} = 1, \qquad \frac{dx}{ds} = u,\qquad \frac{du}{ds} = 0.
\]
Along each characteristic the value of $u$ still remains constant, but now the slope of each line in the $(t,x)$-plane depends on that constant value. As a result, characteristics can converge and intersect, leading to the formation of shocks and more intricate behavior. The linear case worked out here is therefore an essential warm-up: it shows how characteristics encode the geometry of transport in the simplest setting, preparing us to analyze these more complex phenomena in Burgers' equation.
\end{solution}

% ===== Example 2: Inviscid Burgers’ Equation and Shock Formation (inquiry-based) =====
\begin{problem}[Inviscid Burgers’ Equation and Shock Formation]
Burgers’ equation is a simple nonlinear model that already captures many phenomena of hyperbolic conservation laws, such as wave steepening and shock formation. In this problem we study the \emph{inviscid} Burgers’ equation, where the advecting velocity equals the solution itself. Starting from smooth initial data, we will follow characteristics, see how they can intersect in finite time, and interpret this as the breakdown of a classical solution and the onset of a shock. The goal is to understand both the mathematics (characteristics, loss of invertibility) and the physical picture (wave steepening and breaking).

Consider the inviscid Burgers’ equation on the real line
\[
u_t + u\,u_x = 0, \qquad x \in \mathbb{R},\ t > 0,
\]
with smooth initial data
\[
u(x,0) = u_0(x), \qquad x \in \mathbb{R}.
\]

\smallskip
(a) \textbf{Characteristics and constancy along them.}  

Think of the equation as an advection equation with “velocity” $u(x,t)$:
\[
u_t + a(x,t)\,u_x = 0 \quad \text{with} \quad a(x,t) = u(x,t).
\]
We seek characteristic curves $t \mapsto x(t)$ along which $u$ is constant.

\begin{enumerate}
\item[(i)] Write down the system of ordinary differential equations for $x(t)$ and $u(t)$ that encodes the idea “move with the flow so that $u$ does not change along the path.”
\item[(ii)] Show from this system that along each characteristic,
\[
\frac{d}{dt}u(x(t),t) = 0,
\]
that is, $u$ is constant on each characteristic curve.
\end{enumerate}
Hint: For (ii), apply the chain rule to $u(x(t),t)$ and then use the PDE to simplify.

\smallskip
(b) \textbf{Using initial data to parametrize characteristics.}  

We now label each characteristic by the point $\xi$ where it starts from the initial line $t=0$. That is, we consider
\[
x(0) = \xi, \qquad u(x(0),0) = u_0(\xi).
\]

\begin{enumerate}
\item[(i)] Using your ODE system from part (a), solve for $u(t)$ along a characteristic that starts at $(\xi,0)$. Express $u(t)$ in terms of $u_0(\xi)$.
\item[(ii)] Solve for the characteristic curve $x(t;\xi)$ that starts at $x(0)=\xi$, using the fact that $u$ is constant along it. Show that
\[
x(t;\xi) = \xi + t\,u_0(\xi).
\]
\item[(iii)] Explain why, as long as we can solve for $\xi$ in terms of $x$ and $t$ (that is, as long as the mapping $\xi \mapsto x(t;\xi)$ is invertible), we can write the solution in the form
\[
u(x,t) = u_0(\xi(x,t)).
\]
\end{enumerate}
% Hint: You may find it useful to think of $\xi$ as a Lagrangian (particle) label, and $(x,t)$ as Eulerian coordinates.

\smallskip
(c) \textbf{When do characteristics intersect? Condition for shock formation.}  

The characteristics are the curves $x(t;\xi)$ in the $(x,t)$-plane. If two distinct values $\xi_1 \neq \xi_2$ lead to the same point $(x,t)$, the solution formula $u(x,t) = u_0(\xi)$ becomes multi-valued, indicating a breakdown of the classical solution.

\begin{enumerate}
\item[(i)] Show that the map $\xi \mapsto x(t;\xi)$ is strictly increasing if and only if
\[
\frac{\partial x}{\partial \xi}(t;\xi) > 0 \quad \text{for all }\xi.
\]
Compute $\dfrac{\partial x}{\partial \xi}(t;\xi)$ for
\[
x(t;\xi) = \xi + t\,u_0(\xi).
\]
\item[(ii)] Using your expression, derive the condition for the first time $t>0$ when the map $\xi \mapsto x(t;\xi)$ ceases to be one-to-one. Show that this happens when, for some $\xi$,
\[
\frac{\partial x}{\partial \xi}(t;\xi) = 0.
\]
\item[(iii)] Assume $u_0$ is smooth and that its derivative attains a negative minimum:
\[
m := \min_{\xi \in \mathbb{R}} u_0'(\xi) < 0.
\]
Using the condition from (ii), argue that the earliest possible such time is
\[
t_s = -\frac{1}{m} = -\frac{1}{\displaystyle\min_{\xi} u_0'(\xi)}.
\]
Interpret $t_s$ as the \emph{shock formation time}.
\end{enumerate}
Hint: You should get $\partial x/\partial \xi = 1 + t\,u_0'(\xi)$.

\smallskip
(d) \textbf{A concrete example: computing the shock time and location.}  

Consider the specific initial condition
\[
u_0(x) = -x.
\]

\begin{enumerate}
\item[(i)] Compute $u_0'(x)$ and the minimum value of $u_0'(x)$ over $\mathbb{R}$.
\item[(ii)] Use your formula for $t_s$ from part (c) to find the shock time for this initial condition.
\item[(iii)] Write down the explicit formula for the characteristics $x(t;\xi)$ for this $u_0$, and sketch a family of characteristic curves in the $(x,t)$-plane. From your sketch, identify the point $(x_s,t_s)$ where characteristics first intersect.
\item[(iv)] Using your expressions, describe qualitatively what happens to the shape of the graph $x \mapsto u(x,t)$ as $t$ approaches $t_s$ from below. How does this relate to the idea of “wave steepening”?
\end{enumerate}
% Hint: For this example, you should find $x(t;\xi) = \xi(1-t)$.

\smallskip
(e) \textbf{Extensions and “what if” questions.}

\begin{enumerate}
\item[(i)] Suppose instead that $u_0'(x) \ge 0$ for all $x$ (for example, $u_0(x) = x$). Use the formula $\partial x/\partial \xi = 1 + t u_0'(\xi)$ to argue that characteristics never intersect for $t>0$. What does this say about the long-time behavior of the solution, in contrast to the compressive case where $u_0'$ is negative somewhere?
\item[(ii)] (Conceptual) The inviscid Burgers’ equation is often “regularized” by adding viscosity:
\[
u_t + u\,u_x = \nu\,u_{xx}, \qquad \nu > 0.
\]
Based on your understanding of shock formation, briefly explain why adding the term $\nu u_{xx}$ prevents the appearance of a multi-valued solution, and instead leads to a sharp but smooth transition layer (a “viscous shock”) as $\nu \to 0^+$. You do not need to compute any explicit solutions; just describe the expected behavior qualitatively.
\end{enumerate}

\end{problem}

% ===== Example 2: Inviscid Burgers’ Equation and Shock Formation (full solution) =====
\begin{problem}[Inviscid Burgers’ Equation and Shock Formation]
Consider the inviscid Burgers’ equation
\[
u_t + u\,u_x = 0, \qquad x \in \mathbb{R},\ t>0,
\]
with smooth initial data $u(x,0) = u_0(x)$.

\begin{enumerate}
\item[(a)] Use the method of characteristics to show that along a characteristic curve starting from $(\xi,0)$, one has
\[
u(x(t;\xi),t) = u_0(\xi), \qquad x(t;\xi) = \xi + t\,u_0(\xi).
\]
\item[(b)] Compute $\dfrac{\partial x}{\partial \xi}(t;\xi)$ and show that the classical solution remains single‐valued as long as $\dfrac{\partial x}{\partial \xi}(t;\xi)>0$ for all $\xi$. Assuming
\[
m := \min_{\xi \in \mathbb{R}} u_0'(\xi) < 0,
\]
prove that the earliest time $t_s$ at which characteristics intersect is
\[
t_s = -\frac{1}{m} = -\frac{1}{\displaystyle\min_{\xi} u_0'(\xi)}.
\]
\item[(c)] For the specific initial data $u_0(x) = -x$:
\begin{enumerate}
\item[(i)] Compute $u_0'(x)$ and the shock time $t_s$.
\item[(ii)] Find $x(t;\xi)$ explicitly and determine the point $(x_s,t_s)$ where characteristics first intersect.
\item[(iii)] Describe qualitatively how the graph $x \mapsto u(x,t)$ behaves as $t \uparrow t_s$, and interpret this as wave steepening leading to shock formation.
\end{enumerate}
\end{enumerate}
\end{problem}

\begin{solution}
We analyze the inviscid Burgers’ equation
\[
u_t + u\,u_x = 0
\]
with smooth initial data $u(x,0) = u_0(x)$ using the method of characteristics. This equation is a nonlinear conservation law; the central idea is to move along the flow determined by $u$ itself so that the solution becomes constant along suitable curves in the $(x,t)$-plane.

\medskip
\noindent\textbf{(a) Characteristics and explicit form.}

We seek curves $t \mapsto x(t)$ for which the value of $u$ stays constant along the path. Consider a parametrization of a characteristic as $t \mapsto x(t;\xi)$, where $\xi$ labels the point where the curve meets the initial line $t=0$. Along such a curve, define
\[
U(t) := u(x(t;\xi),t).
\]
By the chain rule,
\[
\frac{dU}{dt} = u_t(x(t;\xi),t) + u_x(x(t;\xi),t)\,\frac{dx}{dt}.
\]
We now choose the characteristic velocity to be the advecting velocity from the PDE:
\[
\frac{dx}{dt} = u(x(t;\xi),t) = U(t).
\]
Substituting this into the expression for $dU/dt$ and using the PDE $u_t + u u_x = 0$, we obtain
\[
\frac{dU}{dt} 
= u_t + u_x\,\frac{dx}{dt}
= u_t + u_x\,u
= u_t + u\,u_x
= 0.
\]
Therefore, along each characteristic,
\[
\frac{d}{dt}u(x(t;\xi),t) = 0,
\]
so $u$ is constant along the characteristic. The constant value is determined by the initial condition at $t=0$.

On the initial line we have
\[
x(0;\xi) = \xi, \qquad u(x(0;\xi),0) = u_0(\xi).
\]
Since $dU/dt=0$ and $U(0) = u_0(\xi)$, we conclude
\[
u(x(t;\xi),t) = U(t) = U(0) = u_0(\xi)
\]
for all $t$ for which the characteristic is defined and $u$ remains smooth.

To find the characteristic paths $x(t;\xi)$, we use the equation
\[
\frac{dx}{dt} = u(x(t;\xi),t) = u_0(\xi),
\]
because $u$ is constant along each characteristic. This is now a simple linear ordinary differential equation in $t$:
\[
\frac{dx}{dt} = u_0(\xi), \qquad x(0;\xi) = \xi.
\]
Integrating, we find
\[
x(t;\xi) = \xi + t\,u_0(\xi).
\]
Thus the characteristic curves are straight lines in the $(x,t)$-plane with slope $u_0(\xi)$, and the solution along each such line is given by
\[
u(x(t;\xi),t) = u_0(\xi).
\]

As long as the mapping $\xi \mapsto x(t;\xi)$ can be inverted to solve for $\xi$ as a function of $(x,t)$, we can write the solution implicitly as
\[
u(x,t) = u_0(\xi(x,t)),
\]
where $\xi(x,t)$ is determined from the relation $x = \xi + t\,u_0(\xi)$.

\medskip
\noindent\textbf{(b) Intersection of characteristics and shock time.}

The characteristic map at fixed time $t$ is
\[
X_t : \xi \mapsto x(t;\xi) = \xi + t\,u_0(\xi).
\]
A classical (single-valued) solution exists as long as $X_t$ is one-to-one, so that each point $(x,t)$ comes from a unique initial label $\xi$. For a smooth mapping from $\mathbb{R}$ to $\mathbb{R}$, a sufficient condition for injectivity is that the derivative never vanishes and does not change sign. We compute
\[
\frac{\partial x}{\partial \xi}(t;\xi)
= \frac{\partial}{\partial \xi}\bigl(\xi + t\,u_0(\xi)\bigr)
= 1 + t\,u_0'(\xi).
\]

If $\partial x/\partial \xi>0$ for all $\xi$, then the map is strictly increasing and thus injective. In that case, the implicit formula
\[
x = \xi + t\,u_0(\xi), \qquad u(x,t) = u_0(\xi)
\]
defines a smooth classical solution.

Loss of injectivity first occurs when there exists some $\xi$ and $t>0$ such that
\[
\frac{\partial x}{\partial \xi}(t;\xi) = 0,
\]
because then two nearby characteristics become tangent and the mapping starts to fold. Using our expression,
\[
\frac{\partial x}{\partial \xi}(t;\xi) = 1 + t\,u_0'(\xi),
\]
the condition $\partial x/\partial \xi = 0$ becomes
\[
1 + t\,u_0'(\xi) = 0
\quad\Longleftrightarrow\quad
t = -\frac{1}{u_0'(\xi)}
\]
for those $\xi$ where $u_0'(\xi) < 0$.

Assume that $u_0$ is smooth and that its derivative attains a negative minimum
\[
m := \min_{\xi \in \mathbb{R}} u_0'(\xi) < 0.
\]
For each fixed $\xi$ with $u_0'(\xi) < 0$, the time at which the corresponding characteristic derivative vanishes is
\[
t(\xi) = -\frac{1}{u_0'(\xi)} > 0.
\]
The earliest time at which \emph{any} characteristic family loses injectivity is then
\[
t_s = \inf_{\{\xi: u_0'(\xi)<0\}} t(\xi)
= \inf_{\{\xi: u_0'(\xi)<0\}} \left(-\frac{1}{u_0'(\xi)}\right).
\]
Since $u_0'(\xi) \ge m$ for all $\xi$ and $m<0$, the function $-1/u_0'(\xi)$ is minimized when $u_0'(\xi)$ is as small as possible, that is, when $u_0'(\xi) = m$. Hence
\[
t_s = -\frac{1}{m}
= -\frac{1}{\displaystyle\min_{\xi} u_0'(\xi)} > 0.
\]
We call $t_s$ the \emph{shock formation time}. At $t = t_s$, the characteristic curves first develop tangencies and then intersections, and the formally constructed solution $u(x,t) = u_0(\xi(x,t))$ becomes multi-valued. In the theory of conservation laws, this signals the breakdown of the classical solution and the need to pass to weak (and then entropy) solutions, which develop a discontinuity (a shock) at or shortly after $t_s$.

\medskip
\noindent\textbf{(c) Example: $u_0(x) = -x$. Wave steepening and shock.}

We now specialize to the initial condition
\[
u_0(x) = -x.
\]

\smallskip
\emph{(i) Derivative and shock time.}

We have
\[
u_0'(x) = -1
\]
for all $x \in \mathbb{R}$. Thus
\[
m = \min_{x\in\mathbb{R}} u_0'(x) = -1.
\]
Using the general formula for the shock time,
\[
t_s = -\frac{1}{m} = -\frac{1}{-1} = 1.
\]
Therefore, the solution remains a smooth classical solution for $0 \le t < 1$, and characteristics first intersect at time $t=1$.

\smallskip
\emph{(ii) Characteristics and intersection point.}

For this $u_0$, the characteristic equation becomes
\[
x(t;\xi) = \xi + t\,u_0(\xi) = \xi + t(-\xi) = \xi(1 - t).
\]
Along such a characteristic,
\[
u(x(t;\xi),t) = u_0(\xi) = -\xi.
\]

The family of characteristic curves is therefore
\[
x = \xi(1-t), \quad t \ge 0, \quad \xi \in \mathbb{R}.
\]
For $0 \le t < 1$, each line is straight with slope in the $x$–$t$ plane determined by $u_0(\xi)$, and the factor $(1-t)$ simply compresses the $x$-coordinate. As $t \uparrow 1$, we see that
\[
x(1;\xi) = \xi(1-1) = 0
\]
for \emph{every} $\xi$. Thus all characteristics meet at the single point
\[
(x_s, t_s) = (0, 1).
\]
This is the first time and place where the mapping $\xi \mapsto x(t;\xi)$ fails to be one-to-one: for $t<1$,
\[
\frac{\partial x}{\partial \xi}(t;\xi) = 1 + t\,u_0'(\xi) = 1 + t(-1) = 1 - t > 0,
\]
so the map is strictly increasing; at $t=1$, the derivative vanishes everywhere.

\smallskip
\emph{(iii) Shape of $u(x,t)$ and wave steepening.}

For $0 \le t < 1$, we can invert the relation $x = \xi(1-t)$ to obtain $\xi = x/(1-t)$. Then the solution can be written explicitly as
\[
u(x,t) = u_0\!\left(\frac{x}{1-t}\right)
= -\frac{x}{1-t}.
\]
Thus, for any fixed $t \in [0,1)$, the profile $x \mapsto u(x,t)$ is a straight line with slope
\[
\frac{\partial u}{\partial x}(x,t) = -\frac{1}{1-t},
\]
which is negative and whose magnitude grows without bound as $t \uparrow 1$:
\[
\left|\frac{\partial u}{\partial x}(x,t)\right|
= \frac{1}{1-t} \to \infty \quad \text{as } t \to 1^-.
\]
This means the wave profile becomes increasingly steep. Graphically, the line $u(x,t)$ pivots and steepens, and at $t=1$ the slope becomes infinite, corresponding to vertical tangents in the $(x,u)$-plane. In the characteristic picture, this is the same as the straight characteristic lines collapsing onto $(0,1)$.

This phenomenon is called \emph{wave steepening}: regions where $u_0'(x)$ is negative are “compressive,” causing characteristics to converge and the gradient $u_x$ to blow up in finite time. In the full theory of conservation laws, the physically relevant solution beyond $t_s$ includes a discontinuity (a shock) at $x=0$, $t\ge1$, rather than a multi-valued profile.

\medskip
\noindent\textbf{Connection to the Burgers’ equation case study.}

This example encapsulates the key ideas discussed in the case study on Burgers’ equation. The method of characteristics reduces the nonlinear PDE to a family of ODEs, revealing that $u$ is transported along curves whose speed depends on $u$ itself. Because different parts of the wave travel at different speeds, compressive regions (where $u_0'$ is negative) produce intersecting characteristics in finite time. The explicit formula
\[
t_s = -\frac{1}{\min u_0'}
\]
quantifies the shock formation time for smooth data. For the simple initial profile $u_0(x)=-x$, we can see explicitly how the solution steepens and how characteristics collapse, which makes this a canonical example for understanding nonlinear self-advection, finite-time gradient blow-up, and the onset of shocks in conservation laws.

\end{solution}

% ===== Example 3: Viscous Burgers’ Equation and the Hopf–Cole Transform (inquiry-based) =====
\begin{problem}[Viscous Burgers’ Equation and the Hopf–Cole Transform]
The scalar Burgers’ equation is a classical toy model for nonlinear waves, traffic flow, and turbulence. In its inviscid form it behaves like a nonlinear transport equation and can form shocks in finite time. If we add a small viscosity, the equation becomes parabolic: nonlinear steepening now competes with diffusive smoothing. In this problem you will discover the Hopf–Cole transform, a logarithmic change of variables that converts the nonlinear viscous Burgers’ equation into the linear heat equation, and then use it to build solutions.

Consider the \emph{viscous Burgers’ equation} on the real line
\[
u_t + u\,u_x \;=\; \nu\,u_{xx}, \qquad t>0,\ x\in\mathbb{R},\quad \nu>0.
\]

\smallskip
\noindent
(a) \textbf{From inviscid to viscous Burgers.}
Recall that the inviscid Burgers’ equation can be written in conservation form as
\[
u_t + \left(\frac{u^2}{2}\right)_x = 0.
\]
Explain briefly (you may appeal to the method of characteristics) why solutions of the inviscid Burgers’ equation with smooth initial data can develop shocks in finite time. Then show that the viscous Burgers’ equation can be written in the conservation–diffusion form
\[
u_t + \left(\frac{u^2}{2}\right)_x = \nu\,u_{xx}.
\]
Why does the added term $\nu u_{xx}$ have a smoothing, or regularizing, effect on solutions?

\medskip
\noindent
(b) \textbf{Guessing a transform: from $u$ to $\phi$.}
Our goal is to transform the nonlinear equation for $u$ into a linear equation for a new unknown. Consider the change of variables
\[
u(t,x) = -\,2\nu\,\frac{\phi_x(t,x)}{\phi(t,x)},
\]
where $\phi(t,x)$ is assumed to be positive and smooth.

\begin{itemize}
\item[(i)] Compute $u_x$ and $u_{xx}$ in terms of $\phi$ and its derivatives.
\item[(ii)] Compute $u_t$ in terms of $\phi$ and its derivatives.
\item[(iii)] Substitute these expressions into the viscous Burgers’ equation and simplify.
\end{itemize}
Show that, under this substitution, $u$ satisfies the viscous Burgers’ equation if and only if $\phi$ satisfies a \emph{linear} partial differential equation. Identify this equation.

\emph{Hint:} It is much easier to work with $w = \ln \phi$ first. Note that
\[
\frac{\phi_x}{\phi} = w_x,\qquad \frac{\phi_t}{\phi} = w_t,
\]
and express $u$, $u_x$, $u_{xx}$, and $u_t$ in terms of $w$ and its derivatives before simplifying.

\medskip
\noindent
(c) \textbf{Relating initial data for $u$ and for $\phi$.}
Suppose we are given initial data
\[
u(0,x) = u_0(x),
\]
where $u_0$ is a smooth function that decays sufficiently fast as $|x|\to\infty$.

\begin{itemize}
\item[(i)] Using the relation $u = -2\nu \phi_x/\phi$, derive an ordinary differential equation in $x$ for the initial profile $\phi_0(x) := \phi(0,x)$ in terms of $u_0(x)$.
\item[(ii)] Solve this ODE to express $\phi_0(x)$ in terms of $u_0(x)$, up to a multiplicative constant.
\item[(iii)] Argue why this multiplicative constant does not affect the corresponding solution $u(t,x)$.
\end{itemize}

\emph{Hint:} Your ODE should be first order and separable. You may find it helpful to integrate from a fixed reference point, such as $x=0$.

\medskip
\noindent
(d) \textbf{Solving the linear problem and transforming back.}
In part (b) you should have found that $\phi$ satisfies the heat equation
\[
\phi_t = \nu\,\phi_{xx}
\]
with initial data $\phi(0,x) = \phi_0(x)$ obtained in part (c).

\begin{itemize}
\item[(i)] Recall (or look up) the heat kernel on $\mathbb{R}$,
\[
G_\nu(t,x) = \frac{1}{\sqrt{4\pi \nu t}}\exp\!\left(-\frac{x^2}{4\nu t}\right), \qquad t>0.
\]
Write $\phi(t,x)$ as a convolution of $G_\nu$ with $\phi_0$.

\item[(ii)] Use $u = -2\nu\,\phi_x/\phi$ to express $u(t,x)$ in terms of $\phi_0$ and the heat kernel. You may leave your answer in the form
\[
u(t,x) = -2\nu\,\partial_x \ln \Phi(t,x),
\]
for a suitable integral expression $\Phi(t,x)$.

\item[(iii)] Now obtain an explicit example by \emph{choosing} a simple initial profile for $\phi$:
\[
\phi_0(x) = 1 + e^{-x^2}.
\]
First compute the corresponding initial velocity $u_0(x)$ using the relation
\[
u_0(x) = -2\nu\,\frac{\phi_0'(x)}{\phi_0(x)}.
\]
Then solve the heat equation with this initial data. (You may use without proof that the convolution of two Gaussian functions is again a Gaussian function with an explicitly computable variance.) Finally, compute $u(t,x)$ from the formula $u = -2\nu\,\phi_x/\phi$.
\end{itemize}

\emph{Hint:} For the convolution step, you will need to compute
\[
\bigl(G_\nu(t,\cdot)*e^{-(\cdot)^2}\bigr)(x) = \int_{-\infty}^{\infty}
\frac{1}{\sqrt{4\pi \nu t}}\exp\!\left(-\frac{(x-y)^2}{4\nu t}\right)e^{-y^2}\,dy.
\]
Rewrite the exponent as a quadratic polynomial in $y$ and complete the square.

\medskip
\noindent
(e) \textbf{Extensions and limits.}

\begin{itemize}
\item[(i)] Consider the \emph{inviscid} Burgers’ equation
\[
u_t + u\,u_x = 0
\]
with the same initial data $u(0,x) = u_0(x)$ obtained from $\phi_0(x) = 1 + e^{-x^2}$. Use the method of characteristics (you do not need to give full details) to explain qualitatively what you expect the solution to look like as $t$ increases. How does this compare to the viscous solution you found in part (d)(iii)?

\item[(ii)] Suppose instead that the spatial domain is the $2\pi$-periodic interval and that $u$ is $2\pi$-periodic in $x$. How would you adapt the Hopf–Cole transform and the solution method? In particular, what replaces the heat kernel representation of $\phi$?

\item[(iii)] (Conceptual question.) Based on your formulas, what do you expect to happen in the limit $\nu\to 0^+$ for fixed $t>0$? How might this limit be related to shock formation in the inviscid Burgers’ equation?
\end{itemize}
\end{problem}

% ===== Example 3: Viscous Burgers’ Equation and the Hopf–Cole Transform (full solution) =====
\begin{problem}[Viscous Burgers’ Equation and the Hopf–Cole Transform]
Consider the viscous Burgers’ equation on $\mathbb{R}$,
\[
u_t + u\,u_x = \nu\,u_{xx}, \qquad t>0,\ x\in\mathbb{R},\quad \nu>0.
\]
\begin{enumerate}
\item[(a)] Show that the change of variables
\[
u(t,x) = -\,2\nu\,\frac{\phi_x(t,x)}{\phi(t,x)}
\]
transforms the viscous Burgers’ equation into the linear heat equation
\[
\phi_t = \nu\,\phi_{xx}.
\]
\item[(b)] Given smooth initial data $u(0,x)=u_0(x)$ decaying sufficiently fast at infinity, express the corresponding initial data $\phi(0,x)=\phi_0(x)$ in terms of $u_0(x)$, up to a multiplicative constant. Then use the heat kernel
\[
G_\nu(t,x) = \frac{1}{\sqrt{4\pi \nu t}}\exp\!\left(-\frac{x^2}{4\nu t}\right)
\]
to obtain an integral representation of $u(t,x)$.

\item[(c)] As a concrete example, take
\[
\phi_0(x) = 1 + e^{-x^2}.
\]
Compute the induced initial velocity
\[
u_0(x) = -\,2\nu\,\frac{\phi_0'(x)}{\phi_0(x)},
\]
solve the heat equation for $\phi(t,x)$ with this initial data, and then compute $u(t,x)$. Briefly explain how this viscous solution illustrates the competition between nonlinear steepening and diffusive smoothing, in comparison with the inviscid Burgers’ equation.
\end{enumerate}
\end{problem}

\begin{solution}
We study the scalar viscous Burgers’ equation
\[
u_t + u\,u_x = \nu\,u_{xx},
\]
which is a nonlinear conservation law with an added diffusive term. The central idea is the Hopf–Cole transform, which converts this nonlinear parabolic equation into the linear heat equation.

\medskip
\noindent\textbf{(a) The Hopf–Cole transform and the heat equation.}
We introduce a new function $\phi(t,x)$, assumed positive and smooth, by
\[
u(t,x) = -\,2\nu\,\frac{\phi_x(t,x)}{\phi(t,x)}.
\]
It is convenient to write $\phi = e^{w}$, so that $w = \ln\phi$ and
\[
\frac{\phi_x}{\phi} = w_x,\qquad \frac{\phi_t}{\phi} = w_t.
\]
Then the transformation becomes
\[
u = -2\nu\,w_x.
\]
Differentiating with respect to $x$ and $t$ yields
\[
u_x = -2\nu\,w_{xx},\qquad
u_{xx} = -2\nu\,w_{xxx},\qquad
u_t = -2\nu\,w_{xt}.
\]
Substituting these expressions into the viscous Burgers’ equation gives
\[
-2\nu\,w_{xt} + \bigl(-2\nu\,w_x\bigr)\bigl(-2\nu\,w_{xx}\bigr)
= \nu\bigl(-2\nu\,w_{xxx}\bigr).
\]
We now simplify:
\[
-2\nu\,w_{xt} + 4\nu^2 w_x w_{xx} = -2\nu^2 w_{xxx}.
\]
Divide both sides by $-2\nu$:
\[
w_{xt} - 2\nu w_x w_{xx} = \nu w_{xxx}.
\]
On the other hand, if we compute the equation satisfied by $w$ under the assumption that $\phi$ solves the heat equation $\phi_t = \nu\phi_{xx}$, we obtain
\[
\frac{\phi_t}{\phi} = \frac{\nu\phi_{xx}}{\phi}
\quad\Longrightarrow\quad
w_t = \nu\,\frac{\phi_{xx}}{\phi}.
\]
Using $\phi = e^w$, we have
\[
\phi_x = \phi w_x,\qquad
\phi_{xx} = \phi(w_{xx} + (w_x)^2).
\]
Hence
\[
w_t = \nu\,(w_{xx} + (w_x)^2).
\]
Differentiating this relation with respect to $x$ yields
\[
w_{xt} = \nu(w_{xxx} + 2 w_x w_{xx}).
\]
Rewriting,
\[
w_{xt} - 2\nu w_x w_{xx} = \nu w_{xxx},
\]
which is exactly the relation obtained by substituting the transform into the viscous Burgers’ equation.

Thus:

\smallskip
\begin{itemize}
\item If $\phi$ satisfies the heat equation $\phi_t = \nu\phi_{xx}$, then $w=\ln\phi$ satisfies $w_t = \nu(w_{xx} + (w_x)^2)$, and consequently $u=-2\nu w_x$ satisfies $u_t + u u_x = \nu u_{xx}$.
\item Conversely, if $u$ solves viscous Burgers and can be written as $u = -2\nu\,\phi_x/\phi$ with $\phi>0$, then $w=\ln\phi$ satisfies the above quasi-linear equation, which in turn is equivalent to $\phi$ satisfying the heat equation.
\end{itemize}

Therefore the Hopf–Cole transform $u = -2\nu \phi_x/\phi$ linearizes the viscous Burgers’ equation into the heat equation
\[
\phi_t = \nu\,\phi_{xx}.
\]

\medskip
\noindent\textbf{(b) Initial data and integral representation.}
We are given $u(0,x)=u_0(x)$, a smooth, decaying function. At time $t=0$ the relation $u = -2\nu \phi_x/\phi$ becomes
\[
u_0(x) = -\,2\nu\,\frac{\phi_0'(x)}{\phi_0(x)},
\qquad \phi_0(x):=\phi(0,x).
\]
This is a first-order linear ordinary differential equation in $x$ for $\phi_0$:
\[
\frac{\phi_0'(x)}{\phi_0(x)} = -\frac{1}{2\nu}u_0(x).
\]
Separating variables and integrating from $0$ to $x$,
\[
\int_{0}^{x} \frac{\phi_0'(y)}{\phi_0(y)}\,dy
= -\frac{1}{2\nu}\int_{0}^{x} u_0(y)\,dy.
\]
The left-hand side is $\ln\
hand side is $\ln\phi_0(x)-\ln\phi_0(0)$, so we obtain
\[
\ln\frac{\phi_0(x)}{\phi_0(0)}
= -\frac{1}{2\nu}\int_{0}^{x} u_0(y)\,dy.
\]
Exponentiating gives
\[
\phi_0(x) = \phi_0(0)\,\exp\!\left(-\frac{1}{2\nu}\int_{0}^{x} u_0(y)\,dy\right)
= C \exp\!\left(-\frac{1}{2\nu}\int_{0}^{x} u_0(y)\,dy\right),
\]
where $C>0$ is an arbitrary constant.

This constant has no effect on $u$, because
\[
u(t,x) = -2\nu\,\frac{\phi_x(t,x)}{\phi(t,x)}
\]
is invariant under the scaling $\phi \mapsto C\phi$ (both numerator and denominator are multiplied by $C$).

Since $\phi$ satisfies the heat equation with initial data $\phi_0$, the solution is given by convolution with the heat kernel,
\[
\phi(t,x) = \bigl(G_\nu(t,\cdot)*\phi_0\bigr)(x)
= \int_{\mathbb{R}} G_\nu(t,x-y)\,\phi_0(y)\,dy,
\]
where
\[
G_\nu(t,x) = \frac{1}{\sqrt{4\pi\nu t}}\exp\!\left(-\frac{x^2}{4\nu t}\right).
\]
Thus
\[
u(t,x) = -2\nu\,\frac{\partial_x\phi(t,x)}{\phi(t,x)}
= -2\nu\,\partial_x \ln\Phi(t,x),
\]
where
\[
\Phi(t,x) := \int_{\mathbb{R}} G_\nu(t,x-y)\,\phi_0(y)\,dy.
\]
Equivalently,
\[
u(t,x)
= -2\nu\,\frac{\displaystyle\int_{\mathbb{R}} \partial_x G_\nu(t,x-y)\,\phi_0(y)\,dy}
{\displaystyle\int_{\mathbb{R}} G_\nu(t,x-y)\,\phi_0(y)\,dy}.
\]

\medskip
\noindent\textbf{(c) Explicit example with $\phi_0(x)=1+e^{-x^2}$.}

\smallskip
\noindent\emph{Initial velocity.}
Here
\[
\phi_0(x) = 1 + e^{-x^2},\qquad
\phi_0'(x) = -2x\,e^{-x^2},
\]
so
\[
u_0(x) = -2\nu\,\frac{\phi_0'(x)}{\phi_0(x)}
= -2\nu\,\frac{-2x\,e^{-x^2}}{1+e^{-x^2}}
= \frac{4\nu x\,e^{-x^2}}{1+e^{-x^2}}.
\]

\smallskip
\noindent\emph{Solving the heat equation for $\phi$.}
By linearity,
\[
\phi(t,x) = \bigl(G_\nu(t,\cdot)*\phi_0\bigr)(x)
= \bigl(G_\nu(t,\cdot)*1\bigr)(x)
 + \bigl(G_\nu(t,\cdot)*e^{-(\cdot)^2}\bigr)(x).
\]
The first term is simply $1$, because $G_\nu$ has total mass $1$. For the second term,
\[
\bigl(G_\nu(t,\cdot)*e^{-(\cdot)^2}\bigr)(x)
= \int_{-\infty}^{\infty}
\frac{1}{\sqrt{4\pi \nu t}}
\exp\!\left(-\frac{(x-y)^2}{4\nu t}\right)e^{-y^2}\,dy.
\]
Completing the square in the exponent and evaluating the Gaussian integral (or using the standard formula for the evolution of a Gaussian under the heat equation) yields
\[
\bigl(G_\nu(t,\cdot)*e^{-(\cdot)^2}\bigr)(x)
= \frac{1}{\sqrt{1+4\nu t}}
\exp\!\left(-\frac{x^2}{1+4\nu t}\right).
\]
Hence
\[
\phi(t,x)
= 1 + \frac{1}{\sqrt{1+4\nu t}}
\exp\!\left(-\frac{x^2}{1+4\nu t}\right).
\]

\smallskip
\noindent\emph{Computing $u(t,x)$.}
Differentiate $\phi$ with respect to $x$:
\[
\phi_x(t,x)
= \frac{1}{\sqrt{1+4\nu t}}\,
\frac{d}{dx}\exp\!\left(-\frac{x^2}{1+4\nu t}\right)
= -\,\frac{2x}{(1+4\nu t)^{3/2}}
\exp\!\left(-\frac{x^2}{1+4\nu t}\right).
\]
Therefore
\[
u(t,x)
= -2\nu\,\frac{\phi_x(t,x)}{\phi(t,x)}
= \frac{4\nu x}{(1+4\nu t)^{3/2}}\,
\frac{\exp\!\left(-\frac{x^2}{1+4\nu t}\right)}
{1 + \dfrac{1}{\sqrt{1+4\nu t}}
\exp\!\left(-\frac{x^2}{1+4\nu t}\right)}.
\]
This is a smooth, globally defined solution of the viscous Burgers’ equation for all $t>0$.

\smallskip
\noindent\emph{Qualitative behavior.}
The Gaussian component in $\phi(t,x)$ spreads and decreases in amplitude on the diffusive length scale $\sqrt{\nu t}$, and this spreading is inherited by $u(t,x)$ through the logarithmic derivative. The nonlinear term $u\,u_x$ tends to steepen gradients and transport the profile, while the viscous term $\nu u_{xx}$ spreads and smooths it. The explicit formula for $u(t,x)$ remains bounded with gradients that decay in time, illustrating how viscosity prevents shock formation and enforces smoothing, in contrast with the inviscid Burgers’ equation, where the same initial velocity profile would typically develop a shock in finite time.

\end{solution}

% ===== Example 4: Traveling Wave and Shock Profile Solutions (inquiry-based) =====
\begin{problem}[Traveling Wave and Shock Profile Solutions]
In this problem we study traveling wave solutions of the \emph{viscous Burgers' equation}
\[
u_t + \frac{1}{2}(u^2)_x = \nu\,u_{xx}, \qquad \nu>0,
\]
which is a simple model for a scalar conservation law with dissipation. We are interested in waves that move with constant speed and connect two different constant states at infinity. These solutions model a ``smoothed'' shock wave whose steepness depends on the viscosity $\nu$ and whose speed is determined by a conservation-law condition (the Rankine--Hugoniot jump condition).

We will assume a traveling wave ansatz and reduce the partial differential equation to an ordinary differential equation. We will then integrate this ODE explicitly and interpret the resulting profile.

\smallskip

Let $u(x,t)$ solve
\[
u_t + \frac{1}{2}(u^2)_x = \nu\,u_{xx}.
\]
We look for traveling wave solutions of the form
\[
u(x,t) = U(\xi), \qquad \xi = x - st,
\]
that satisfy the asymptotic conditions
\[
\lim_{\xi\to -\infty} U(\xi) = u_-, 
\qquad 
\lim_{\xi\to +\infty} U(\xi) = u_+,
\]
for given constants $u_-$ and $u_+$ with $u_- \neq u_+$, and some wave speed $s$ to be determined.

\begin{enumerate}[(a)]
\item (Setting up the traveling wave ODE)  
Compute $u_t$, $u_x$, and $u_{xx}$ in terms of $U$ and its derivatives with respect to $\xi$. Substitute the ansatz $u(x,t)=U(x-st)$ into Burgers' equation and simplify to obtain an ordinary differential equation for $U(\xi)$.

State your final ODE in the form
\[
\nu U''(\xi) + (s - U(\xi))\,U'(\xi) = 0.
\]

% Hint: Use the chain rule: $u_t = -s\,U'(\xi)$, $u_x = U'(\xi)$, $u_{xx} = U''(\xi)$.

\item (First integration and the role of the far-field states)  
The ODE from part (a) is autonomous (it does not depend explicitly on $\xi$), so it can be integrated once.

\begin{enumerate}
\item Regard $U'(\xi)$ as a function of $U$ and use the chain rule $U'' = \dfrac{dU'}{d\xi} = \dfrac{dU'}{dU}U'$ to rewrite the second-order ODE as a first-order ODE for $p(U) := U'(\xi)$ as a function of $U$.

\item Solve this first-order ODE to obtain an expression of the form
\[
U'(\xi) = \frac{1}{2\nu}\bigl((U(\xi)-s)^2 - C\bigr),
\]
for some constant $C$.

\end{enumerate}

\textit{Hint:} After substituting $U'' = p'(U)U'$, divide by $U'$ (assuming $U'$ is not identically zero) to reduce to a first-order linear ODE for $p(U)$.

\item (Determining the wave speed: Rankine--Hugoniot condition)  
Use the asymptotic conditions at $\xi\to\pm\infty$ to determine both the constant $C$ and the wave speed $s$ in terms of $u_-$ and $u_+$.

\begin{enumerate}
\item Explain why
\[
\lim_{\xi\to\pm\infty} U'(\xi) = 0.
\]

\item Evaluate the first-integral expression for $U'(\xi)$ as $\xi\to\pm\infty$ and show that $C = (u_\pm - s)^2$.

\item By equating the two expressions for $C$ corresponding to $u_-$ and $u_+$, derive a formula for $s$ in terms of $u_-$ and $u_+$. Show that
\[
s = \frac{u_- + u_+}{2}.
\]

\end{enumerate}

\textit{Hint:} At $\xi\to\pm\infty$, both $U(\xi)$ and $U'(\xi)$ should approach constant limits; use these to evaluate the constant of integration from part (b).

\item (Solving for the traveling wave profile)  
With the value of $s$ from part (c), rewrite the first-integral equation in a factorized form involving $(U-u_-)$ and $(U-u_+)$, and then solve explicitly for $U(\xi)$.

\begin{enumerate}
\item Show that, with $s = (u_- + u_+)/2$, the first-integral can be written as
\[
U'(\xi) = \frac{1}{2\nu}\,(U(\xi) - u_-)\,(U(\xi) - u_+).
\]

\item Rewrite this as a separable ODE and integrate to obtain an implicit formula for $U(\xi)$ of the form
\[
\frac{U(\xi)-u_-}{U(\xi)-u_+} = C\,e^{\alpha \xi},
\]
for suitable constants $C$ and $\alpha$ depending on $u_\pm$ and $\nu$.

\item Choose a convenient shift of the coordinate $\xi$ (for instance, by requiring $U(0) = s$) to determine $C$, and manipulate the exponential expression into a hyperbolic tangent form. Show that one can write the solution as
\[
U(\xi)
= \frac{u_- + u_+}{2}
  - \frac{u_- - u_+}{2}
    \tanh\!\left(
      \frac{(u_- - u_+)\,\xi}{4\nu}
    \right),
\]
for the case $u_- > u_+$.

\end{enumerate}

\textit{Hint:} For the last step, recall the identity
\[
\tanh\left(\frac{y}{2}\right) 
= \frac{1 - e^{-y}}{1 + e^{-y}},
\]
and try to rearrange your exponential expression into the form of a ratio similar to the right-hand side.

\item (Interpretation and extensions)  

\begin{enumerate}
\item Describe in words the qualitative shape of the profile $U(\xi)$ when $u_- > u_+$. What happens to the profile as $\nu \to 0^+$? How does the ``thickness'' of the transition layer scale with $\nu$?

\item What changes in your analysis, if any, when $u_- < u_+$ (that is, when the left state is smaller than the right state)? What does the traveling wave solution look like in that case?

\item (Optional extension) Burgers' equation without viscosity is
\[
u_t + \frac{1}{2}(u^2)_x = 0.
\]
Based on your computations above, argue informally that in the inviscid limit $\nu\to 0^+$, the viscous traveling wave solution converges to a discontinuous shock moving with speed $s = (u_- + u_+)/2$. How does this speed relate to the Rankine--Hugoniot condition for the conservation law?
\end{enumerate}

\end{enumerate}
\end{problem}

% ===== Example 4: Traveling Wave and Shock Profile Solutions (full solution) =====
\begin{problem}[Traveling Wave and Shock Profile Solutions]
Consider the viscous Burgers' equation
\[
u_t + \frac{1}{2}(u^2)_x = \nu\,u_{xx}, \qquad \nu>0.
\]
Seek traveling wave solutions of the form $u(x,t) = U(\xi)$ with $\xi = x - st$ that satisfy
\[
\lim_{\xi\to -\infty} U(\xi) = u_-, \qquad \lim_{\xi\to +\infty} U(\xi) = u_+,
\]
for given constants $u_\pm$ with $u_- \neq u_+$.  

\begin{enumerate}[(i)]
\item Derive the ordinary differential equation for $U(\xi)$ and integrate it once to obtain a first-integral.
\item Use the far-field conditions to determine the wave speed $s$ in terms of $u_-$ and $u_+$, and show that $s = (u_- + u_+)/2$.
\item Show that $U$ satisfies a separable first-order ODE and solve it explicitly to obtain the traveling wave profile. For $u_- > u_+$, show that
\[
U(\xi)
= \frac{u_- + u_+}{2}
  - \frac{u_- - u_+}{2}
    \tanh\!\left(
      \frac{(u_- - u_+)\,\xi}{4\nu}
    \right).
\]
\item Briefly describe what happens to this profile as $\nu \to 0^+$ and relate the wave speed to the Rankine--Hugoniot condition for the inviscid conservation law $u_t + \tfrac12 (u^2)_x = 0$.
\end{enumerate}
\end{problem}

\begin{solution}
We are studying a model conservation law with diffusion,
\[
u_t + \frac{1}{2}(u^2)_x = \nu\,u_{xx},
\]
and we look for traveling waves of the form $u(x,t) = U(\xi)$, where $\xi = x - st$ and $s$ is a constant speed to be determined. The profile $U$ is required to connect two distinct constant states $u_-$ and $u_+$ at spatial infinity.

\medskip

\noindent\textbf{(i) Deriving and integrating the ODE.}
Using the chain rule with $\xi = x - st$, we compute
\[
u_t = \frac{\partial U}{\partial \xi}\,\frac{\partial \xi}{\partial t}
= -s\,U'(\xi), 
\quad
u_x = U'(\xi),
\quad
u_{xx} = U''(\xi),
\]
where primes denote derivatives with respect to $\xi$. Also,
\[
(u^2)_x = (U(\xi)^2)_x = \frac{\mathrm{d}}{\mathrm{d}\xi}(U(\xi)^2)\,\frac{\partial \xi}{\partial x}
= 2U(\xi)U'(\xi).
\]
Substituting into Burgers' equation gives
\[
-s\,U'(\xi) + \frac{1}{2}\cdot 2U(\xi)U'(\xi) = \nu\,U''(\xi),
\]
that is,
\[
-s\,U' + U\,U' = \nu\,U''.
\]
Rearranging, we obtain the second-order ordinary differential equation
\[
\nu U''(\xi) + (s - U(\xi))\,U'(\xi) = 0.
\]

This ODE is autonomous (it does not depend explicitly on $\xi$), so we integrate it once. Introduce $p(U) := U'(\xi)$ and use the chain rule
\[
U''(\xi) = \frac{\mathrm{d}U'}{\mathrm{d}\xi} 
= \frac{\mathrm{d}p}{\mathrm{d}U}\,\frac{\mathrm{d}U}{\mathrm{d}\xi}
= p'(U)\,p(U).
\]
Substituting $U'' = p'(U)p(U)$ and $U' = p(U)$ into the ODE gives
\[
\nu\,p'(U)\,p(U) + (s - U)\,p(U) = 0.
\]
Assuming the profile is nontrivial so that $p(U)\not\equiv 0$, we may divide by $p(U)$ to obtain the first-order linear ODE
\[
\nu\,p'(U) + s - U = 0.
\]
Solving for $p'(U)$,
\[
p'(U) = \frac{U - s}{\nu}.
\]
Integrating with respect to $U$ yields
\[
p(U) = \int \frac{U - s}{\nu}\,\mathrm{d}U
= \frac{1}{2\nu}(U - s)^2 + C_1,
\]
where $C_1$ is a constant of integration. Returning to the original notation $p(U) = U'(\xi)$, we have the first-integral
\[
U'(\xi) = \frac{1}{2\nu}(U(\xi) - s)^2 + C_1.
\]
It will be convenient to denote $C := -2\nu C_1$, so we can rewrite this as
\[
U'(\xi) = \frac{1}{2\nu}\left[(U(\xi) - s)^2 - C\right],
\]
which is the desired first-integral form.

\medskip

\noindent\textbf{(ii) Determining the wave speed via the far-field states.}
We are given that
\[
\lim_{\xi\to -\infty}U(\xi) = u_-,
\qquad
\lim_{\xi\to +\infty}U(\xi) = u_+,
\]
with $u_- \neq u_+$. Because the profile is smooth and connects two constants, the derivative $U'(\xi)$ tends to zero at both infinities:
\[
\lim_{\xi\to\pm\infty} U'(\xi) = 0.
\]
Intuitively, far from the transition region the solution has settled down to a constant, so its spatial derivative vanishes.

We use this information in the first-integral. As $\xi\to -\infty$, we have $U(\xi)\to u_-$ and $U'(\xi)\to 0$, so
\[
0 = \lim_{\xi\to -\infty} U'(\xi)
= \lim_{\xi\to -\infty} \frac{1}{2\nu}\Bigl[(U(\xi) - s)^2 - C\Bigr]
= \frac{1}{2\nu}\Bigl[(u_- - s)^2 - C\Bigr].
\]
Thus
\[
C = (u_- - s)^2.
\]
Similarly, as $\xi\to +\infty$,
\[
0 = \lim_{\xi\to +\infty} U'(\xi)
= \frac{1}{2\nu}\Bigl[(u_+ - s)^2 - C\Bigr],
\]
so
\[
C = (u_+ - s)^2.
\]
Equating the two expressions for $C$ gives
\[
(u_- - s)^2 = (u_+ - s)^2.
\]
Taking square roots produces two possibilities:
\[
u_- - s = u_+ - s
\quad\text{or}\quad
u_- - s = -(u_+ - s).
\]
The first possibility $u_- - s = u_+ - s$ would imply $u_- = u_+$, which contradicts $u_- \neq u_+$. Therefore we must have
\[
u_- - s = -(u_+ - s),
\]
which simplifies to
\[
u_- - s = -u_+ + s
\quad\Longrightarrow\quad
2s = u_- + u_+.
\]
Thus the wave speed is
\[
s = \frac{u_- + u_+}{2}.
\]

This is precisely the Rankine--Hugoniot velocity for the conservation law $u_t + \tfrac12 (u^2)_x = 0$, because in general the shock speed $s$ satisfies
\[
s = \frac{f(u_+) - f(u_-)}{u_+ - u_-}
\quad\text{with}\quad f(u) = \frac{u^2}{2}.
\]
Evaluating,
\[
s = \frac{\tfrac12 u_+^2 - \tfrac12 u_-^2}{u_+ - u_-}
= \frac{(u_+ - u_-)(u_+ + u_-)}{2(u_+ - u_-)}
= \frac{u_- + u_+}{2},
\]
in agreement with our computation.

\medskip

\noindent\textbf{(iii) Separable first-order ODE and explicit profile.}
Now that $s$ is known, we can simplify the first-integral. From $C = (u_- - s)^2 = (u_+ - s)^2$ and $s = (u_- + u_+)/2$, it is straightforward to check that
\[
(u - s)^2 - C = (u - u_-)(u - u_+)
\]
for any real $u$. Indeed,
\[
(u - s)^2 - (u_- - s)^2 = (u - u_-)(u - u_+),
\]
which can be verified by direct algebra or by factoring the left-hand side as a difference of squares. Substituting this into our first-integral,
\[
U'(\xi) = \frac{1}{2\nu}\bigl((U(\xi) - s)^2 - C\bigr)
= \frac{1}{2\nu}\,(U(\xi) - u_-)(U(\xi) - u_+).
\]
Thus $U$ satisfies the first-order autonomous ODE
\[
U'(\xi) = \frac{1}{2\nu}\,(U(\xi) - u_-)(U(\xi) - u_+).
\]

This equation is separable. Assuming $U(\xi)$ stays between $u_-$ and $u_+$ (which will be the case for the monotone shock profile), we may write
\[
\frac{\mathrm{d}U}{(U - u_-)(U - u_+)} = \frac{\mathrm{d}\xi}{2\nu}.
\]
We integrate both sides. Using partial fractions,
\[
\frac{1}{(U - u_-)(U - u_+)}
= \frac{1}{u_- - u_+}\left(\frac{1}{U - u_-} - \frac{1}{U - u_+}\right).
\]
Integrating with respect to $U$ gives
\[
\int \frac{\mathrm{d}U}{(U - u_-)(U - u_+)}
= \frac{1}{u_- - u_+}\ln\left|\frac{U - u_-}{U - u_+}\right| + C_2,
\]
where $C_2$ is a constant of integration. On the right-hand side, integrating with respect to $\xi$ yields
\[
\int \frac{\mathrm{d}\xi}{2\nu} = \frac{\xi}{2\nu} + C_3,
\]
for another constant $C_3$. We may combine the constants into a single constant $C_4$. Dropping the absolute values (since we know the sign of the numerator and denominator from the ordering of $u_-$ and $u_+$ and the monotonicity of the profile), we obtain
\[
\frac{1}{u_- - u_+}\ln\left(\frac{U(\xi) - u_-}{U(\xi) - u_+}\right)
= \frac{\xi}{2\nu} + C_4.
\]
Multiplying by $(u_- - u_+)$ gives
\[
\ln\left(\frac{U(\xi) - u_-}{U(\xi) - u_+}\right)
= \frac{(u_- - u_+)\,\xi}{2\nu} + C_5,
\]
with $C_5 = (u_- - u_+)C_4$. Exponentiating both sides,
\[
\frac{U(\xi) - u_-}{U(\xi) - u_+}
= C\,\exp\!\left(\frac{(u_- - u_+)\,\xi}{2\nu}\right),
\]
where $C = e^{C_5}$ is a nonzero constant.

Next we choose a convenient origin for $\xi$ to fix $C$. It is natural to center the wave by requiring that the midpoint value $s = (u_- + u_+)/2$ occur at $\xi = 0$:
\[
U(0) = s.
\]
Substituting $\xi = 0$ and $U(0) = s$ into the implicit formula gives
\[
\frac{s - u_-}{s - u_+} = C.
\]
Since $s = (u_- + u_+)/2$, we have
\[
s - u_- = \frac{u_- + u_+}{2} - u_- = \frac{u_+ - u_-}{2},
\qquad
s - u_+ = \frac{u_- + u_+}{2} - u_+ = \frac{u_- - u_+}{2}.
\]
Hence
\[
C = \frac{s - u_-}{s - u_+}
= \frac{\tfrac{u_+ - u_-}{2}}{\tfrac{u_- - u_+}{2}} = -1.
\]
Therefore the implicit relation simplifies to
\[
\frac{U(\xi) - u_-}{U(\xi) - u_+}
= -\exp\!\left(\frac{(u_- - u_+)\,\xi}{2\nu}\right).
\]

To express $U$ in a more transparent form, let
\[
\Delta := u_- - u_+ > 0.
\]
Then
\[
\frac{U(\xi) - u_-}{U(\xi) - u_+}
= -\exp\!\left(\frac{\Delta\,\xi}{2\nu}\right).
\]
Solving for $U(\xi)$, write
\[
U(\xi) - u_- = -e^{\Delta\xi/(2\nu)}\bigl(U(\xi) - u_+\bigr),
\]
so
\[
U(\xi)\bigl(1 + e^{\Delta\xi/(2\nu)}\bigr)
= u_- + u_+\,e^{\Delta\xi/(2\nu)}.
\]
Thus
\[
U(\xi)
= \frac{u_- + u_+\,e^{\Delta\xi/(2\nu)}}{1 + e^{\Delta\xi/(2\nu)}}.
\]
This is already a legitimate explicit formula. We can simplify it further to
\[
U(\xi)
= \frac{u_- + u_+}{2}
  - \frac{u_- - u_+}{2}
    \tanh\!\left(\frac{(u_- - u_+)\,\xi}{4\nu}\right).
\]

To see this, write
\[
U(\xi)
= \frac{u_- + u_+\,e^{\Delta\xi/(2\nu)}}{1 + e^{\Delta\xi/(2\nu)}}
= u_+ + \frac{u_- - u_+}{1 + e^{\Delta\xi/(2\nu)}}.
\]
Set
\[
z := e^{-\Delta\xi/(2\nu)}.
\]
Then
\[
\frac{1}{1 + e^{\Delta\xi/(2\nu)}}
= \frac{z}{1 + z}.
\]
Recall the identity
\[
\tanh\!\left(\frac{y}{2}\right)
= \frac{1 - e^{-y}}{1 + e^{-y}},
\]
which, with $y = \Delta\xi/(2\nu)$, gives
\[
\tanh\!\left(\frac{\Delta\xi}{4\nu}\right)
= \frac{1 - e^{-\Delta\xi/(2\nu)}}{1 + e^{-\Delta\xi/(2\nu)}}
= \frac{1 - z}{1 + z}.
\]
Solving for $\dfrac{z}{1+z}$,
\[
\frac{z}{1+z}
= \frac{1 - \tanh\!\left(\dfrac{\Delta\xi}{4\nu}\right)}{2}.
\]
Hence
\[
U(\xi)
= u_+ + (u_- - u_+)\,\frac{1}{2}
   \Bigl[1 - \tanh\!\Bigl(\frac{\Delta\xi}{4\nu}\Bigr)\Bigr]
= \frac{u_- + u_+}{2}
  - \frac{\Delta}{2}\,\tanh\!\Bigl(\frac{\Delta\xi}{4\nu}\Bigr),
\]
and since $\Delta = u_- - u_+$, we obtain
\[
U(\xi)
= \frac{u_- + u_+}{2}
  - \frac{u_- - u_+}{2}
    \tanh\!\left(\frac{(u_- - u_+)\,\xi}{4\nu}\right),
\]
which is the claimed explicit traveling wave profile for $u_- > u_+$.

\medskip

\noindent\textbf{(iv) Inviscid limit and Rankine--Hugoniot speed.}
For $u_- > u_+$, the function $U(\xi)$ is a smooth, monotone decreasing transition from $u_-$ (as $\xi\to -\infty$) to $u_+$ (as $\xi\to +\infty$). The transition occurs over a characteristic length scale given by the argument of the hyperbolic tangent,
\[
\frac{(u_- - u_+)\,\xi}{4\nu} = O(1)
\quad\Longrightarrow\quad
|\xi| = O\!\left(\frac{\nu}{|u_- - u_+|}\right).
\]
Thus, as $\nu \to 0^+$, the thickness of the transition layer shrinks like $O(\nu)$, and $U(\xi)$ approaches a discontinuous step function:
\[
U(\xi) \to
\begin{cases}
u_-, & \xi < 0,\\[4pt]
u_+, & \xi > 0,
\end{cases}
\]
i.e., a shock located at $\xi = 0$ (or, in $(x,t)$ variables, at $x = st$).

The corresponding solution of the inviscid conservation law
\[
u_t + \frac{1}{2}(u^2)_x = 0
\]
is a propagating shock joining $u_-$ to $u_+$. The speed $s$ of this shock is given by the Rankine--Hugoniot condition
\[
s = \frac{f(u_+) - f(u_-)}{u_+ - u_-},
\quad f(u) = \frac{u^2}{2},
\]
which yields
\[
s = \frac{\tfrac12 u_+^2 - \tfrac12 u_-^2}{u_+ - u_-}
= \frac{u_- + u_+}{2},
\]
exactly the speed obtained from the traveling wave analysis. Thus, in the limit $\nu\to 0^+$, the viscous shock profile converges to the entropy-satisfying inviscid shock moving at the Rankine--Hugoniot speed.

\end{solution}

% ===== Example 5: Boundary-Value Problems and Numerical Exploration for Burgers’ Equation (inquiry-based) =====
\begin{problem}[Boundary-Value Problems and Numerical Exploration for Burgers’ Equation]
In this case study we look at viscous Burgers’ equation on a finite spatial interval with imposed inflow and outflow conditions. This is a simple model for one-dimensional flow in a narrow channel driven by a difference in boundary velocities. The equation combines nonlinear advection, which tends to steepen gradients and form shock-like transitions, with viscosity, which tends to smooth them out. Our goal is to (i) set up a boundary-value problem, (ii) derive a simple finite difference scheme, (iii) understand a basic stability constraint, and (iv) predict qualitatively what numerical solutions should look like when we vary the viscosity and the grid.

Consider viscous Burgers’ equation
\[
u_t + u\,u_x = \nu\,u_{xx}, \qquad 0<x<1,\quad t>0,
\]
with viscosity $\nu>0$. We impose Dirichlet boundary conditions
\[
u(0,t) = 1,\qquad u(1,t)=0,\qquad t>0,
\]
and an initial condition $u(x,0)=u_0(x)$ with $0\le u_0(x)\le 1$. You may think of $u$ as a scalar velocity, with fluid entering from the left at speed $1$ and exiting to the right at speed $0$.

\smallskip

(a) Rewrite Burgers’ equation in conservative form by introducing an appropriate flux function $f(u)$ so that the equation becomes
\[
u_t + \bigl(f(u)\bigr)_x = \nu\,u_{xx}.
\]
What is $f(u)$? Explain briefly why this conservative form is natural if we think of $u$ as a transported quantity subject to diffusion.

\medskip

(b) We now discretize the problem on a uniform grid $x_j = j\,\Delta x$ for $j=0,1,\dots,N$, where $\Delta x = 1/N$. Let $u_j^n$ be an approximation to $u(x_j,t^n)$ at time $t^n = n\,\Delta t$.

\begin{itemize}
    \item[(i)] For the diffusion term $u_{xx}$, propose a standard second-order centered finite difference at interior grid points. Write your formula explicitly.
    \item[(ii)] For the nonlinear advection term $u\,u_x$, notice that in our problem the solution is expected to stay between $0$ and $1$, so the characteristic velocity $u$ is nonnegative. Use this to motivate a first-order upwind approximation of the form
    \[
    (u\,u_x)(x_j,t^n) \approx u_j^n \,\frac{u_j^n - u_{j-1}^n}{\Delta x},
    \]
    and explain (in words) why using a backward difference in $x$ makes sense when $u\ge 0$.
\end{itemize}

Using these ideas, assemble a fully explicit finite difference scheme for the interior points $j=1,\dots,N-1$ of the form
\[
u_j^{n+1} = u_j^n + \text{(convection contribution)} + \text{(diffusion contribution)}.
\]
Write the scheme out completely in terms of $u_{j-1}^n, u_j^n, u_{j+1}^n$, and the parameters $\Delta t, \Delta x, \nu$.

\emph{Hint:} Start from $u_t = -u\,u_x + \nu u_{xx}$, approximate $u_t$ by a forward difference in time, then plug in your spatial approximations.

\medskip

(c) We incorporate the boundary conditions directly into the discrete system by setting
\[
u_0^n = 1,\qquad u_N^n = 0 \qquad \text{for all } n\ge 0.
\]
Explain how these fixed boundary values enter the update formula for $u_1^{n+1}$ and $u_{N-1}^{n+1}$. Write the explicit update for $u_1^{n+1}$, and indicate which terms come from $u_0^n$.

\emph{Hint:} Your interior formula for $u_j^{n+1}$ is valid for $j=1,\dots,N-1$. When $j=1$, the value $u_{j-1}^n$ is simply $u_0^n$, which is known from the boundary condition.

\medskip

(d) To understand stability, linearize the scheme around a constant state $u\equiv U$ with $0 < U \le 1$. The linearized partial differential equation is
\[
v_t + U\,v_x = \nu\,v_{xx},
\]
where $v$ is a small perturbation. The corresponding linearized finite difference scheme (using the same stencils as in part (b)) has the form
\[
v_j^{n+1} = v_j^n - C\,(v_j^n - v_{j-1}^n) + \mu\,(v_{j+1}^n - 2v_j^n + v_{j-1}^n),
\]
for suitable dimensionless parameters $C$ and $\mu$.

\begin{itemize}
    \item[(i)] Identify $C$ and $\mu$ in terms of $U$, $\nu$, $\Delta t$, and $\Delta x$.
    \item[(ii)] Rewrite this update rule in the form
    \[
    v_j^{n+1} = a\,v_{j-1}^n + b\,v_j^n + c\,v_{j+1}^n,
    \]
    and express $a,b,c$ in terms of $C$ and $\mu$.
    \item[(iii)] A very robust (though somewhat conservative) way to guarantee stability in the maximum norm is to demand $a,b,c\ge 0$ and $a+b+c=1$, so that the new value is a convex combination of the three neighboring values. Use these conditions to derive an inequality relating $C$ and $\mu$.
\end{itemize}

Finally, rewrite your inequality as a time-step restriction of the form
\[
\Delta t \;\le\; \frac{1}{\dfrac{U}{\Delta x} \;+\; \dfrac{2\nu}{\Delta x^2}}.
\]
Explain how this restriction changes when the viscosity $\nu$ is very small, and when the grid is very fine (that is, when $\Delta x$ is very small).

\emph{Hint:} Verify that $a+b+c=1$ automatically, and focus on the nonnegativity of each coefficient.

\medskip

(e) Conceptual exploration.

\begin{itemize}
    \item[(i)] Suppose you run your scheme with a moderate viscosity, say $\nu=10^{-2}$, and start from the initial condition $u_0(x)\equiv 0$. Based on the continuous equation and on your discrete scheme, describe qualitatively what you expect the solution $u(x,t)$ to look like for large $t$. In particular, what kind of spatial profile do you expect between $x=0$ and $x=1$? Where will the gradients be largest, and how does the thickness of the transition layer depend on $\nu$?
    \item[(ii)] Now imagine decreasing $\nu$ by a factor of $10$ while keeping the same grid and time step (still satisfying your stability condition), and then, alternatively, refining the grid $\Delta x$ while keeping $\nu$ fixed. In each case, explain qualitatively how the numerical solution changes. When might you start to see numerical artifacts such as undershoots, overshoots, or excessive smearing near the steep gradient region?
\end{itemize}

\emph{Hint:} Think about the competition between physical diffusion (controlled by $\nu$) and numerical diffusion or dispersion (introduced by the discretization and finite grid resolution). How does each one affect the sharpness of a shock-like transition?
\end{problem}

% ===== Example 5: Boundary-Value Problems and Numerical Exploration for Burgers’ Equation (full solution) =====
\begin{problem}[Boundary-Value Problems and Numerical Exploration for Burgers’ Equation]
Consider viscous Burgers’ equation
\[
u_t + u\,u_x = \nu\,u_{xx}, \qquad 0<x<1,\ t>0,
\]
with viscosity $\nu>0$, Dirichlet boundary conditions
\[
u(0,t) = 1,\quad u(1,t)=0,\quad t>0,
\]
and an initial condition $u(x,0)=u_0(x)$ with $0\le u_0(x)\le 1$.

\begin{enumerate}
    \item Rewrite the equation in conservative form $u_t + (f(u))_x = \nu u_{xx}$ and identify $f(u)$.
    \item Discretize the problem on a uniform grid $x_j=j\Delta x$ ($j=0,\dots,N$), $\Delta x=1/N$, with time step $\Delta t$, using:
    \begin{itemize}
        \item forward Euler in time for $u_t$,
        \item a second-order centered difference for $u_{xx}$ at interior points,
        \item a first-order upwind approximation $u\,u_x \approx u_j^n (u_j^n - u_{j-1}^n)/\Delta x$ for the advection term (assuming $0\le u\le 1$).
    \end{itemize}
    Derive the explicit finite difference scheme for interior nodes $j=1,\dots,N-1$, and indicate how the Dirichlet boundary conditions enter at $j=1$ and $j=N-1$.
    \item Linearize the scheme about a constant state $u\equiv U$ with $0<U\le 1$ to obtain a scheme for perturbations $v$ of the form
    \[
    v_j^{n+1} = v_j^n - C\,(v_j^n - v_{j-1}^n) + \mu\,(v_{j+1}^n - 2v_j^n + v_{j-1}^n).
    \]
    Express $C$ and $\mu$ in terms of $U,\nu,\Delta t,\Delta x$, rewrite the update as
    \[
    v_j^{n+1} = a\,v_{j-1}^n + b\,v_j^n + c\,v_{j+1}^n,
    \]
    and derive a sufficient stability condition by requiring $a,b,c\ge 0$ and $a+b+c=1$. Show that this yields the time-step restriction
    \[
    \Delta t \;\le\; \frac{1}{\dfrac{U}{\Delta x} \;+\; \dfrac{2\nu}{\Delta x^2}}.
    \]
    \item Briefly discuss the qualitative behavior of the numerical solution for large times, and how it depends on the viscosity $\nu$ and the grid spacing $\Delta x$. In particular, comment on the thickness of the shock-like transition between $x=0$ and $x=1$, and on how under-resolved steep gradients can lead to numerical artifacts.
\end{enumerate}
\end{problem}

\begin{solution}
We proceed step by step, highlighting the interaction between the partial differential equation, the boundary conditions, and the numerical discretization.

\medskip

\noindent\textbf{(1) Conservative form.}
Starting from
\[
u_t + u\,u_x = \nu\,u_{xx},
\]
we recognize that $u\,u_x$ is the $x$-derivative of $u^2/2$. Indeed,
\[
\frac{d}{dx}\left(\frac{u^2}{2}\right) = u\,u_x.
\]
Thus we may rewrite the equation in conservation form
\[
u_t + \bigl(f(u)\bigr)_x = \nu\,u_{xx},
\]
with flux
\[
f(u) = \frac{u^2}{2}.
\]

In this form, the left-hand side expresses a local conservation law for the quantity $u$, transported by the nonlinear flux $f(u)$. The right-hand side adds a diffusive term, representing viscous smoothing. When we impose boundary conditions at $x=0$ and $x=1$, we are effectively fixing the inflow and outflow values of $u$ (and hence controlling the boundary fluxes) while allowing the interior to adjust dynamically.

\medskip

\noindent\textbf{(2) Finite difference discretization.}
We introduce a uniform spatial grid
\[
x_j = j\,\Delta x,\qquad j=0,1,\dots,N,\qquad \Delta x = \frac{1}{N},
\]
and discrete time levels
\[
t^n = n\,\Delta t,\qquad n=0,1,2,\dots.
\]
We denote by $u_j^n$ an approximation to $u(x_j,t^n)$.

We first rewrite the partial differential equation as
\[
u_t = -u\,u_x + \nu\,u_{xx}.
\]
We approximate $u_t$ at time level $t^n$ by a forward difference,
\[
u_t(x_j,t^n) \approx \frac{u_j^{n+1} - u_j^n}{\Delta t}.
\]

For the diffusion term, at an interior point $x_j$ with $1\le j\le N-1$ we use the standard centered second difference,
\[
u_{xx}(x_j,t^n) \approx \frac{u_{j+1}^n - 2u_j^n + u_{j-1}^n}{\Delta x^2}.
\]

For the nonlinear advection term $u\,u_x$, we use that in our setting the solution is expected to satisfy $0\le u\le 1$ for all $x$ and $t$. The advection velocity $u$ is thus nonnegative, so characteristics move from left to right. For an upwind discretization of $u_x$ in this situation, we use a backward difference,
\[
u_x(x_j,t^n) \approx \frac{u_j^n - u_{j-1}^n}{\Delta x}.
\]
Multiplying by $u_j^n$ gives the approximation
\[
(u\,u_x)(x_j,t^n) \approx u_j^n\,\frac{u_j^n - u_{j-1}^n}{\Delta x}.
\]
Using a backward difference is natural: when information propagates from left to right, the value at $x_j$ is determined by what has happened at points $x_{j-1}$, $x_{j-2}$, and so on; the backward stencil looks ``into the wind'' and thereby adds numerical damping that stabilizes the scheme.

Putting these pieces together, the discrete version of $u_t = -u\,u_x + \nu u_{xx}$ at an interior point is
\[
\frac{u_j^{n+1} - u_j^n}{\Delta t}
= - u_j^n\,\frac{u_j^n - u_{j-1}^n}{\Delta x}
+ \nu\,\frac{u_{j+1}^n - 2u_j^n + u_{j-1}^n}{\Delta x^2},
\]
for $j=1,\dots,N-1$.

Solving for $u_j^{n+1}$ we obtain the explicit finite difference scheme
\begin{equation}\label{eq:explicit-scheme}
u_j^{n+1}
= u_j^n
- \frac{\Delta t}{\Delta x}\,u_j^n\,(u_j^n - u_{j-1}^n)
+ \nu\,\frac{\Delta t}{\Delta x^2}\,\bigl(u_{j+1}^n - 2u_j^n + u_{j-1}^n\bigr),
\qquad j=1,\dots,N-1.
\end{equation}
This is an explicit scheme combining a first-order upwind approximation of nonlinear advection with a second-order centered approximation of diffusion.

\medskip

\noindent\textbf{Boundary conditions in the scheme.}
The Dirichlet boundary conditions
\[
u(0,t) = 1,\qquad u(1,t) = 0
\]
are imposed discretely as
\[
u_0^n = 1,\qquad u_N^n = 0\qquad\text{for all }n.
\]
These values are held fixed at every time step and enter into the update of the neighboring interior points.

For example, for $j=1$, the scheme \eqref{eq:explicit-scheme} becomes
\[
u_1^{n+1}
= u_1^n
- \frac{\Delta t}{\Delta x}\,u_1^n\,(u_1^n - u_0^n)
+ \nu\,\frac{\Delta t}{\Delta x^2}\,\bigl(u_2^n - 2u_1^n + u_0^n\bigr),
\]
and here $u_0^n=1$ is substituted from the boundary condition. Thus the left boundary value explicitly influences the update at the first interior point. An analogous statement holds at $j=N-1$, where $u_N^n=0$ enters the scheme.

\medskip

\noindent\textbf{(3) Linearization and stability via convex combinations.}
To study stability, we linearize about a constant state $u\equiv U$ with $0<U\le 1$. We write
\[
u_j^n = U + v_j^n,
\]
where $v_j^n$ is a small perturbation. Substituting into \eqref{eq:explicit-scheme
} and discarding terms that are quadratic in $v$, we obtain
\[
U + v_j^{n+1}
= U + v_j^n
- \frac{\Delta t}{\Delta x}\,U\,(v_j^n - v_{j-1}^n)
+ \nu\,\frac{\Delta t}{\Delta x^2}\,\bigl(v_{j+1}^n - 2v_j^n + v_{j-1}^n\bigr).
\]
The constant $U$ cancels from both sides, leaving
\[
v_j^{n+1}
= v_j^n
- \frac{U\Delta t}{\Delta x}\,(v_j^n - v_{j-1}^n)
+ \nu\,\frac{\Delta t}{\Delta x^2}\,\bigl(v_{j+1}^n - 2v_j^n + v_{j-1}^n\bigr).
\]

Comparing with the given linearized form,
\[
v_j^{n+1} = v_j^n - C\,(v_j^n - v_{j-1}^n) + \mu\,(v_{j+1}^n - 2v_j^n + v_{j-1}^n),
\]
we can read off
\[
C = \frac{U\,\Delta t}{\Delta x}, \qquad
\mu = \frac{\nu\,\Delta t}{\Delta x^2}.
\]

\medskip

\noindent\emph{Rewriting as a three-point stencil.}
We expand
\begin{align*}
v_j^{n+1}
&= v_j^n - C\,(v_j^n - v_{j-1}^n)
   + \mu\,(v_{j+1}^n - 2v_j^n + v_{j-1}^n) \\
&= v_j^n - C v_j^n + C v_{j-1}^n
   + \mu v_{j+1}^n - 2\mu v_j^n + \mu v_{j-1}^n \\
&= (C+\mu)\,v_{j-1}^n
   + \bigl(1 - C - 2\mu\bigr)\,v_j^n
   + \mu\,v_{j+1}^n.
\end{align*}
Thus
\[
a = C+\mu,\qquad b = 1 - C - 2\mu,\qquad c = \mu.
\]
One readily checks
\[
a + b + c = (C+\mu) + (1 - C - 2\mu) + \mu = 1.
\]

\medskip

\noindent\emph{Convex-combination stability condition.}
A sufficient condition for stability in the maximum norm is that
\[
a\ge 0,\quad b\ge 0,\quad c\ge 0,\quad a+b+c=1,
\]
so that $v_j^{n+1}$ is a convex combination of neighboring values.

We already have $a+b+c=1$. The nonnegativity conditions are:
\begin{itemize}
    \item $c = \mu \ge 0$, which holds automatically since $\nu,\Delta t,\Delta x^2>0$;
    \item $a = C+\mu \ge 0$, which holds automatically since $U>0$;
    \item $b = 1 - C - 2\mu \ge 0$, which gives the only nontrivial constraint:
    \[
    C + 2\mu \le 1.
    \]
\end{itemize}
In terms of the original parameters,
\[
C + 2\mu
= \frac{U\,\Delta t}{\Delta x} + 2\,\frac{\nu\,\Delta t}{\Delta x^2}
= \Delta t\left(\frac{U}{\Delta x} + \frac{2\nu}{\Delta x^2}\right) \le 1.
\]
Thus a sufficient stability condition is
\[
\boxed{\;
\Delta t \;\le\; \frac{1}{\dfrac{U}{\Delta x} + \dfrac{2\nu}{\Delta x^2}}
\;}
\]
as claimed.

\medskip

\noindent\emph{Behavior for small $\nu$ and small $\Delta x$.}
\begin{itemize}
    \item If $\nu$ is very small, the diffusive term contributes little to the restriction and the condition is dominated by advection:
    \[
    \Delta t \lesssim \frac{\Delta x}{U}.
    \]
    The time step can then scale linearly with $\Delta x$ (a typical CFL condition for advection-dominated problems).

    \item If the grid is very fine (small $\Delta x$) with fixed $\nu$, the diffusive term
    \[
    \frac{2\nu}{\Delta x^2}
    \]
    becomes large and dominates the bound. Then
    \[
    \Delta t \lesssim \frac{\Delta x^2}{2\nu},
    \]
    which is the classic parabolic time-step restriction: as $\Delta x$ halves, $\Delta t$ must be reduced by roughly a factor of $4$ to maintain stability.
\end{itemize}

\medskip

\noindent\textbf{(4) Qualitative behavior and dependence on $\nu$ and $\Delta x$.}

\medskip

\noindent\emph{(i) Long-time behavior for moderate viscosity.}
Take $\nu = 10^{-2}$ and initial data $u_0(x)\equiv 0$. The left boundary enforces $u(0,t)=1$ and the right boundary enforces $u(1,t)=0$. Over time, the solution is driven from the initial state towards a steady profile $u_\infty(x)$ that solves the stationary boundary-value problem
\[
u\,u_x = \nu\,u_{xx}, \qquad 0<x<1,\qquad u_\infty(0)=1,\quad u_\infty(1)=0.
\]
Qualitatively, this steady solution is
\begin{itemize}
    \item monotone decreasing from $u\approx 1$ near $x=0$ to $u\approx 0$ near $x=1$,
    \item smooth, with a single transition layer (shock-like region) somewhere between $x=0$ and $x=1$,
    \item relatively broad for a moderate viscosity: the gradient is largest in the transition region, whose thickness is of order $\nu$ (up to factors depending on the precise profile).
\end{itemize}
In the discrete scheme, $u_j^n$ will relax towards a numerically steady state with a similar smooth, monotone profile. The largest changes from one grid point to the next occur in the same transition region; away from it the solution will be nearly flat (close to $1$ on the left and $0$ on the right).

\medskip

\noindent\emph{(ii) Effect of decreasing $\nu$ and of refining the grid.}

\smallskip
\noindent\underline{Decreasing $\nu$ (fixed grid, stable $\Delta t$).}
If we reduce $\nu$ by a factor of $10$ while keeping the grid and (stability-respecting) time step the same:
\begin{itemize}
    \item The physical diffusion becomes weaker, so the continuous steady profile develops a sharper transition layer: the region over which $u$ drops from $1$ to $0$ becomes thinner.
    \item On a fixed grid, once the true physical layer becomes thinner than a few mesh cells, the scheme cannot fully resolve it. The numerically computed layer will then be smeared over several grid points, with a thickness controlled more by numerical diffusion (from the upwind advection and discrete diffusion) and grid spacing than by the physical $\nu$.
    \item In this first-order upwind/explicit-diffusion scheme, the main artifact is \emph{excessive smearing}: the shock-like transition will appear too wide compared to the true solution. Strong oscillatory artifacts (undershoots/overshoots) are less likely here because the scheme is monotone under the convex-combination stability condition derived above.
\end{itemize}

\smallskip
\noindent\underline{Refining the grid $\Delta x$ (fixed $\nu$).}
Now keep $\nu$ fixed and decrease $\Delta x$:
\begin{itemize}
    \item The physical transition layer becomes better resolved: with more grid points across the layer, the discrete profile more closely matches the sharpness of the continuous solution.
    \item However, the stability restriction becomes more severe, since for small $\Delta x$ the term $2\nu/\Delta x^2$ dominates. To keep $C+2\mu\le 1$, $\Delta t$ must be reduced roughly like $\Delta x^2$.
    \item If $\Delta x$ is refined \emph{without} a corresponding reduction in $\Delta t$ (so that $C+2\mu>1$), the convex-combination property is lost. Then one may begin to see numerical instabilities, which can manifest as spurious oscillations, undershoots, or overshoots near the steep gradient, and in the worst case as blow-up of the numerical solution.
    \item When the stability condition is respected, refinement typically reduces numerical diffusion, leading to a \emph{sharper} and more accurate representation of the shock-like transition, with less artificial smearing.
\end{itemize}

In summary, the sharpness and accuracy of the numerical transition layer are governed by both the physical viscosity $\nu$ and the effective numerical resolution $\Delta x$ (together with a stable $\Delta t$). Lower viscosity creates steeper true gradients, which require finer grids (and smaller time steps) to be represented faithfully without excessive numerical smearing or instability.

\end{solution}



%========================================
% Back matter
%========================================
\backmatter

\chapter*{Summary of Topics}

Here you can keep a running list of topics, theorems, and page
references for exam review, plus a mapping between exam problems
and the thematic sections where they naturally belong.

\end{document}
